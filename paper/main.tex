\documentclass{article}
\usepackage[utf8]{inputenc}
\usepackage[margin=1in]{geometry} 
\usepackage{listings}

\usepackage{color}
 
\definecolor{codegreen}{rgb}{0,0.6,0}
\definecolor{codered}{rgb}{0.6,0,0}
\definecolor{codegray}{rgb}{0.5,0.5,0.5}
\definecolor{codepurple}{rgb}{0.58,0,0.82}
\definecolor{backcolour}{rgb}{0.95,0.95,0.92}

% Python style for highlighting
\lstdefinestyle{pystyle}{
    language=python,
    otherkeywords={self},
    keywordstyle=\color{codered},
    emphstyle=\color{codepurple},    % Custom highlighting style
    stringstyle=\color{codegreen},
    frame=tb,                         % Any extra options here
    showstringspaces=false            % 
}

\lstset{style=pystyle}

\usepackage{fontspec}
\usepackage{pgfplots}
\newlength\figurewidth
\newlength\figureheight

\pretolerance=100

\twocolumn
\begin{document}
\begin{sloppypar}

\begin{titlepage}
    \title{Optimizing Pypy's Garbage Collector for Copy On Write Performance given a forky Web-Server Workload}
    \author{Andrew Nelson\\CSC 550}

    \maketitle

    \begin{abstract}
        Most research into Garbage Collection is focused on analyzing a single process in isolation.  This paper focuses on the Garbage Collection of many processes with shared memory.  We examine a deployment of a Django webserver (written in Python) running with multiple forked workers and profile its memory usage with various Garbage Collection strategies.  We use PyPy, a Python runtime environment with a modular codebase, and modify the Garbage Collector to improve its use of copy-on-write memory. Specifically, we take the Garbage Collector state which was previously stored inside the objects and instead store that state in separate memory pages.  That way when a collection event occurs, those pages which were shared between Python processes can remain untouched during collection.  Our analysis found that the modification decreases private memory usage in worker processes by up to 20\% and preserves shared memory between workers.  
    \end{abstract}
\end{titlepage}

\newpage

\tableofcontents

\newpage
\pagenumbering{arabic}
\twocolumn


\section{Introduction}
Advances in interpreters for high level languages have enabled their use in performance critical applications.  Some of the most visited websites on the Internet as of 2018 are written in high level languages such as Python, Ruby, and PHP.  A common problem with these deployments is memory usage: Garbage Collected languages tend to use more memory than their manually-memory-managed counterparts. In order for these popular websites to continue to scale up, the tools on which they are developed need to get more efficient.  Many of these systems are memory bound and so in order to make them more efficient we must make them use less memory.  

Most research into Garbage Collection focuses on the process in isolation.  This paper, however, examines Garbage Collection in a group of processes which share memory in a Copy-On-Write fashion.  In a webserver environment where worker processes are often forked repeatedly during operation, this situation is very common.  Workers forked from the master process initially share some memory with the master and so the memory usage of the system as a whole is directly related to how much memory can be shared between processes.  This is the type of environment that many of today's largest websites like Instagram operate with\cite{dismissing_garbage}.

Oftentimes in environments like this, the Garbage Collector will disturb pages during collection by writing to them.  When doing a Mark-and-Sweep operation, the Garbage Collector traverses the entire object-space and marks the objects which are live.  This marking disturbs the page regardless of whether any objects within it are actually garbage.  While Garbage Collection has the intention of reducing the number of pages used by a process, by disturbing shared pages in forked workers, the Garbage Collector ends up increasing memory usage in the group as a whole \cite{dismissing_garbage}.  

This paper examines an optimization in PyPy's Generational Garbage Collector\cite{pypy-doc} which is designed to reduce the number of shared pages that get disturbed during a major collection.  To accomplish this, we take the Garbage Collector state which was previously stored within each object and we move that state into a dense index stored in its own page in memory.  This prevents the Garbage Collector from needing to write to shared memory during collection.  We then provide experimental evidence showing that this optimization does in fact reduce memory usage of a group of processes.  

Section \ref{sec:background} explains various topics relating to the optimization and builds up to a demonstration intended to create some intuition into how this optimization is designed to work.  Section \ref{sec:design} explains the details of the optimization and builds intuition as to which data structures need to be changed.  Section \ref{sec:implementation} summarizes the source changes made to PyPy's Garbage Collector in order to implement this.  Section \ref{sec:method} talks about how these changes were tested and section \ref{sec:analysis} examines the results of these tests and explains them.  Finally, section \ref{sec:relatedwork} talks about other optimizations that have been made to Garbage Collectors to optimize for Copy-On-write behavior and explains how they differ from this research.  

\section{Background}\label{sec:background}

Before presenting the modifications made to PyPy's Garbage Collector, we first motivate the need for Copy-on-Write-Sensitive Garbage Collection by explaining how shared memory gets wasted.  First we review the Linux memory model and Copy-on-Write semantics.  Next we review Reference Counting Garbage Collection as well as Mark-and-Sweep Garbage Collection and Compacting Generational Garbage Collection.  Finally we put this information together to demonstrate how these algorithms when used in conjunction can end up unnecessarily wasting memory in a shared-memory environment.  

\subsection{Copy-on-Write (CoW) semantics}\label{sec:CoW}

When a process in Linux forks, a new process is spawned with an identical address space.  In order to save memory, the Linux kernel does not copy all the physical pages in the parent process, but rather it copies the page table of the parent process and marks all physical pages as shared and read-only \cite{VMM}.  If the physical page was already marked shared, the reference count of the page is incremented.  When either the parent or child process tries to read from a location in memory, the virtual address referenced by the process gets translated to a physical address using the page table for that process.  This happens transparently from the perspective of the process \cite{VMM}.  Since both processes can share the same physical pages in memory, the combined memory consumption of these two processes is less than the sum of the individual memory consumptions.

When either the parent or the child process attempt to write to a shared page in memory, a minor page fault occurs.  Control is shifted to the kernel which determines how to proceed.  If the physical page is still being shared, the kernel allocates a new physical page, copies the contents of the shared page into the new physical page, decrements the reference count in the shared page, and inserts the new physical page into the faulted process's page table \cite{VMM}.  If the original physical page is no longer being shared (for example if all the other processes already made their own copies) then the page is converted from a read-only shared page to a read-write private page \cite{VMM}.  In this way, memory is allocated and the process's page-table is updated in an extremely lazy manor which has proven to be very efficient in practice.  

In the event that a process forks multiple times, the same mechanism still works.  A single address space can be shared among $N$ processes.  Since each page has an entire word dedicated to reference count, quite a few pages can refer to a single page and copying will happen as needed \cite{kernelsource,VMM}.  

\subsection{Reference-Counting Garbage Collection (RCGC)}\label{sec:RCGC}

Under a Reference-Counting Garbage Collection regime, every object has an internal reference count attribute.  When new references to this object are created, the reference count attribute is incremented.  When references to this object are updated or when objects which refer to this object are collected, the reference count attribute is decremented.  When the reference count of an attribute reaches zero, it is Garbage Collected  and the object is deleted in place.  When an object is collected, all the objects that it referred to are updated accordingly \cite{GC-continuum}.  

This Garbage Collection algorithm can be visualized as a directed graph where every node is an object in memory and where every edge is a reference from one object to another.  The reference count for each object is the in-degree of its associated node.  When there are no edges into a node, the algorithm knows that this node (and all its out-edges) can be collected.  This scheme has advantages in that many objects can be collected as soon as they become unreachable.  However, if a group of unreachable objects form a reference cycle, their reference counts will never reach zero and so they will never be collected by a strictly reference-counted system.  Additionally, because objects are collected in place, this collection strategy is subject to fragmentation over time \cite{GC-continuum}.  

This algorithm has been used extensively in CPython since Python2 and accounts for the majority of object collection in most running systems \cite{cpython3-doc}.  Whenever a reference is created, updated, or deleted, CPython visits the objects being referenced and updates their reference counts appropriately.  To account for cycles in the object graph, CPython does occasional collections using the Mark-And-Sweep algorithm.  

\subsection{Mark-And-Sweep Garbage Collection (MSGC)}\label{sec:MSGC}

Mark-and-Sweep Garbage Collection can be modeled as a traversal of the object graph starting with a set of ``accessible roots''.  These accessible roots are objects that are being directly referenced by the program (e.g. globals, variables on the stack, intermediate values of a computation, etc.).  Starting with these nodes, the algorithm recursively visits adjacent nodes marking them as ``accessible'' when they are visited.  At the end of this first sweep, the algorithm has divided all objects into two sets: those which are accessible to the program (nodes tagged as ``accessible'') and those which are not accessible to the program (nodes not tagged as ``accessible'').  To do the collection, the algorithm makes a pass over every object in the system.  Those not tagged as ``accessible'' are deleted in place \cite{GC-continuum}. 

Much like with RCGC, a common problem with this algorithm is that objects are deleted in-place and so the physical memory where objects reside tends to become fragmented over time.  This increases the number of physical pages in use and increases the cache miss rate.  Unlike the RCGC, because a complete Mark-and-Sweep collection needs to traverse the entire object-graph, there is a significant pause while collection occurs \cite{GC-continuum}.  

This algorithm has been used in conjunction with Reference Counting in CPython since Python2.  While the reference counter is always running in the background during CPython execution, CPython will initiate a MSGC (usually called a ``major collection event'') collection once memory usage exceeds a certain threshold.  In combination, these algorithms eliminate most garbage in a timely manor while also correctly collecting cycles that form on occasion \cite{cpython3-doc}.  

\subsection{Compacting Generational Garbage Collection (CGGC)}\label{sec:CGGC}

In Compacting Generational Garbage Collection, objects are divided into multiple ``generations'' which are each located on separate physical pages of memory.  New objects are inserted into generation 0 (often called the ``nursery'').  Once generation 0 reaches a certain size, the objects in generation 0 are traversed starting with the accessible roots.  Objects in generation 0 which are still accessible are copied to generation 1.  When all the accessible objects in generation 0 have been relocated, generation 0 is cleared \cite{GC-continuum}.  

In many cases there are 3 or 4 successively larger generations.  The same rules for moving objects from generation 0 to generation 1 apply.  Indeed when generation $N$ gets large enough, its contents are traversed, accessible objects are copied to generation $N+1$, and generation $N$ is cleared.

In order to correctly account for inter-generational references, each generation must keep track of references from other generations into itself.  When an object in generation $M$ is updated to reference an object in generation $N$, that object is added to generation $N$'s cross-reference list.  When generation $N$ is collected, the Garbage Collector adds references in the cross-reference list to the root accessible set and those references are used when tracing which objects are in use.  When an object in generation $M$ is copied to generation $M + 1$, all the updates to that object must be updated to refer to this new location in memory \cite{GC-continuum}.  

In most implementations there is a final generation (often called the ``mature'' generation) for objects that have survived all the prior collections.  There are several ways to handle this final generation - the most common being to simply never collect it.  This implementation is not technically correct because mature objects which become garbage will never be collected but it performs well in practice.  

The PyPy Python environment uses CGGC with two generations: a ``nursery generation'' for young objects and a ``mature generation'' for mature objets.  On occasion, the mature generation is collected using the MSGC.  Memory fragmentation in the mature generation can be filled when objects are copied from younger generations into mature \cite{pypy-doc}.  

\subsection{Tragedy of the commons: Dirtying Shared Pages During GC}\label{sec:CoWGC}

Each of these algorithms alone serve to minimize the memory used by an individual process as it runs.  CoW allows related processes to share physical memory, and all three Garbage Collection algorithms reduce the number of physical pages needed by each process.  However, when used in conjunction we often see that the memory performance of a collection of related processes suffers \cite{dismissing_garbage}.

As a thought experiment, consider a forking webserver written in Python running atop a standard CPython interpreter.  When a request comes in, the process forks to spawn a worker thread which begins processing the request.  The parent process has much in memory the child process can reference during its computation.  

As the child process continues, it will surely create or update a reference to some existing constant data structure.  When this happens, the reference count field in the referenced object must be updated.  This triggers a copy-on-write because even though none of of the objects stored on the page changed, the meta-data that the Garbage Collector stores on the page has changed.  In this way, RCGC disturbs shared pages which can increase memory used by the system.

When a CPython major collection event occurs, the MSGC algorithm will start a full heap traversal that marks objects as ``accessible'' or ``not-accessible''.  Because the Garbage Collector stores this information adjacent to the objects in memory, the shared pages containing these objects get written to which triggers a copy-on-write and the page sharing breaks.  In this way, MSGC disturbs shared pages which can increase memory used by the system.

This problem is somewhat alleviated when using CGGC such as is included in PyPy because we can copy objects out of a shared page without writing to it, however PyPy's GC is also subject to the dirtying of pages that happens when the mature generation is Mark-and-Swept.  

While these algorithms are successful at reducing the memory usage of an individual process, when examining a collection of related processes they do poorly at reducing total combined memory usage.  The next section is dedicated to modifying PyPy's CGGC to prevent the Mark-and-Sweep phase from disturbing shared pages.  

\section{Design}\label{sec:design}

In this paper we modify the Garbage Collector in the PyPy Python runtime environment.  We choose PyPy because it is written in a dialect of Python which makes the interpreter easy to modify for experimental purposes.  Also, because PyPy's Garbage Collector is a relatively simple CGGC, the modification and analysis is relatively straightforward.  PyPy's Garbage Collector is a CGGC with generations: a ``nursery generation'' for young objects and a ``mature generation'' for mature objets.  On occasion, the mature generation is collected using the MSGC.  Copying Generational Garbage Collection events are referred to as ``minor collections'' while Mark-and-Sweep Garbage Collection events are referred to as ``major collections'' \cite{pypy-doc}.  The modification described here is intended only to affect major collections.

To prevent the Garbage Collector from touching objects in place, we add a hashtable to the Garbage Collector that contains the set of all the elements that have been visited.  When a Mark-and-Sweep occurs in the mature generation, rather than writing to the objects in place, the Garbage Collector inserts the object's id into the hashtable.  When the Mark-and-Sweep finishes, that hashtable is cleared.  This hashtable will be stored in different physical pages from the objects which it describes.  By doing this we create two categories of pages: those that contain objects and those that contain object metadata.  The GC usually only needs to modify the pages that contain object metadata which means that the pages containing objects can remain undisturbed.  

\section{Implementation}\label{sec:implementation}

PyPy has several Garbage Collectors but the one we chose to patch for the purposes of this paper is called ``minimark''.  Minimark is a CGGC with two generations \cite{pypy-doc}.  New objects are created in the ``nursery'' and are moved to the ``mature'' generation on minor collection events.  When the mature generation exceeds a certain size, a major collection starts and the mature generation is Mark-and-Swept.  During this MSGC, each object in the mature generation is visited and a bit is set in-place which tags that object as ``safe'' during this collection.  

To implement this change, we added a member variable to the MiniMarkGC class of type $rpython.memory.support.AddressDict$.  PyPy's memory library does not provide an AddressSet type so this was chosen as an approximation.  Objects which are in the mapping and which map to themselves are considered visited and objects which either are not in the mapping or which map to $rpython.memory.Null$ are considered unvisited.  

The following code snippet was added to MiniMarkGC and demonstrates the new interface for determining an object's visited state.

\noindent
\begin{minipage}{\linewidth}
\begin{lstlisting}[language=python]
def obj_visited(self, obj):
  return (
    self.visited_objs.contains(obj) and
    self.visited_objs.get(obj) == obj)

def mark_obj_visited(self, obj):
  self.visited_objs.setitem(obj, obj)

def unmark_obj_visited(self, obj):
  self.visited_objs.setitem(obj,
        llmemory.NULL)
\end{lstlisting}
\end{minipage}

We treat $visited\_objs$ like a set.  At the start of collection, $visited\_objs$ is initialized to be an empty set.  During the Mark phase, objects are inserted into this set.  During the sweep phase, the entire object-space is scanned again.  When objects are traversed, they are either collected if they were not visited during the Mark phase or are unmarked if they were visited in the Mark phase.  Afterwards, if running in debug mode, PyPy traverses the entire object-space a third time to ensure consistency.  In a clasic MSGC, the third pass is unnecessary, but PyPy does it to simplify development.  

\section{Methodologies}\label{sec:method}

This paper focuses on a web-server workload and so to validate our patch, we chose to create the closest approximation of a production webserver possible.  Our model was Instagram's web-tier servers which are briefly described in their external blog.  Instagram is a massive Django instance running atop CPython2.7 using Ngnx as a front-end server and using uWSGI to manage Django worker processes \cite{dismissing_garbage}.  uWSGI is running in pre-fork mode meaning that a worker is started and then forked several times to create more workers.  When a worker dies, an existing worker is forked to replace it.  

With this test environment we create 2 experimental groups.  The groups are as follows:

\begin{enumerate}
    \item Minimark GC without patch
    \item Minimark GC with patch
\end{enumerate}

\subsection{Test Environment}

In order to replicate this environment, we created a docker image that had Ngnx, uWSGI, Django, and PyPy installed on it.  We ran uWSGI in pre-fork mode with 4 worker processes.  The docker image was capped at 8GB of memory and 3 cpu cores.  All benchmarks were ron on a ThinkPad T460 with 12GB ram and an i5-6300 2.4GHz quad-core processor.  Because the docker image was capped at well below the host's resource capacity, we believe benchmarks are reasonably isolated from resource contention with host processes.

Because Instagram is closed-source, proprietary software, we were unable to procure its source code for use in testing.  Instead we created a Django application which was created specifically to emulate a production web-server load.  Specifically, we created a Django application which allocates a large chunk of memory at startup and which allocates an additional chunk of memory for each request.  Below are the snippets of code used to create test-garbage:

\noindent
\begin{minipage}{\linewidth}
\begin{lstlisting}[language=python]
class LargeObject():
  # Use a bunch of memory
  def __init__(self):
    self.lists = []
    self.strs = []
    for i in range(1600):
      self.lists.append(["text"] * i)
      for j in range(40):
        self.strs.append("str" * 8)

def on_startup():
  global static_memory
  static_memory = [
    LargeObject()
    for x
    in range(5)]

def on_page_load():
  global glob_a, glob_b, glob_c
  glob_a, glob_b, glob_c = (
    LargeObject(), glob_a, glob_b)
\end{lstlisting}
\end{minipage}

We believe that this mix of memory allocation somewhat approximates what might be seen in a production web-server.  

\subsection{Stress-Testing the Deployment}

To generate a test load we used httperf which makes a series of http requests to a given http endpoint.  We chose to track two specific metrics for each experimental group: The time it took to handle $N$ requests and the memory used after handling $N$ requests.

To determine the time to handle $N$ requests, we use the httperf tool to make $N$ requests and record the wall-time it took for all requests to get a response.  We run this experiment with $N$ values of 5k, 10k, 15k, and 20k.  

To track memory usage as the number of requests increases, we make a burst of 100 requests and then check the memory usage of each uWSGI worker process.  We repeat this process until we've made 50 bursts of requests or 5000 requests in total.  At each step we check memory usage by scraping that process's $/proc/PID/smaps$ file.  The private memory used by that process is the sum of all lines starting with ``Private\_clean'' and ``Private\_dirty'' and the shared memory used by that process is the sum of all lines starting with ``Shared\_clean'' and ``Shared\_dirty''.  

\section{Analysis}\label{sec:analysis}

Experimental results show that this approach was successful in preserving shared memory in each worker process.  During our verification step we measured two types of performance:  memory usage and latency.  In the first section here we see that the Patched version of the minimark GC uses less memory overall than the Standard version by preserving shared memory.  In the second section here we see that the Standard version serves more requests per second up until a certain point after which it underperforms because of paging.  The latency in the Patched version can be attributed to the runtime overhead of looking up objects in the visited-objects set.  The observed latency implies that depending on the setup, the cost of the modification can be outweighed by the savings in memory.  

\subsection{Memory Usage}

Figures \ref{fig:patched_mem} and \ref{fig:standard_mem} clearly show that the Patched version of the Garbage Collector preserves more shared memory than the Standard version.  In the upper portion of each figure we see the shared memory for each case.  In the Standard GC group, we see that the amount of shared memory trails off as the number of requests grows because when each of the worker processes runs major Garbage Collection they taint many of the pages.  On the other hand, the shared memory chart for the Patched GC is almost entirely flat meaning that those shared pages are untouched and preserved.  In the lower chart we can see that after 5000 requests, the Patched worker processes take up somewhere in the realm of 800Mb of space per worker while the Standard worker processes take up 1000 to 1200Mb of space per worker.  This accounts for a reduction of around 20\%.

\begin{figure}
    \begin{center}\textbf{Memory Usage of Patched GC}\end{center}
    \resizebox{0.5\textwidth}{!}{%% Creator: Matplotlib, PGF backend
%%
%% To include the figure in your LaTeX document, write
%%   \input{<filename>.pgf}
%%
%% Make sure the required packages are loaded in your preamble
%%   \usepackage{pgf}
%%
%% Figures using additional raster images can only be included by \input if
%% they are in the same directory as the main LaTeX file. For loading figures
%% from other directories you can use the `import` package
%%   \usepackage{import}
%% and then include the figures with
%%   \import{<path to file>}{<filename>.pgf}
%%
%% Matplotlib used the following preamble
%%   \usepackage{fontspec}
%%   \setmainfont{Bitstream Vera Serif}
%%   \setsansfont{Bitstream Vera Sans}
%%   \setmonofont{Bitstream Vera Sans Mono}
%%
\begingroup%
\makeatletter%
\begin{pgfpicture}%
\pgfpathrectangle{\pgfpointorigin}{\pgfqpoint{8.000000in}{6.000000in}}%
\pgfusepath{use as bounding box, clip}%
\begin{pgfscope}%
\pgfsetbuttcap%
\pgfsetmiterjoin%
\definecolor{currentfill}{rgb}{1.000000,1.000000,1.000000}%
\pgfsetfillcolor{currentfill}%
\pgfsetlinewidth{0.000000pt}%
\definecolor{currentstroke}{rgb}{1.000000,1.000000,1.000000}%
\pgfsetstrokecolor{currentstroke}%
\pgfsetdash{}{0pt}%
\pgfpathmoveto{\pgfqpoint{0.000000in}{0.000000in}}%
\pgfpathlineto{\pgfqpoint{8.000000in}{0.000000in}}%
\pgfpathlineto{\pgfqpoint{8.000000in}{6.000000in}}%
\pgfpathlineto{\pgfqpoint{0.000000in}{6.000000in}}%
\pgfpathclose%
\pgfusepath{fill}%
\end{pgfscope}%
\begin{pgfscope}%
\pgfsetbuttcap%
\pgfsetmiterjoin%
\definecolor{currentfill}{rgb}{1.000000,1.000000,1.000000}%
\pgfsetfillcolor{currentfill}%
\pgfsetlinewidth{0.000000pt}%
\definecolor{currentstroke}{rgb}{0.000000,0.000000,0.000000}%
\pgfsetstrokecolor{currentstroke}%
\pgfsetstrokeopacity{0.000000}%
\pgfsetdash{}{0pt}%
\pgfpathmoveto{\pgfqpoint{0.894063in}{3.540000in}}%
\pgfpathlineto{\pgfqpoint{7.607500in}{3.540000in}}%
\pgfpathlineto{\pgfqpoint{7.607500in}{5.600556in}}%
\pgfpathlineto{\pgfqpoint{0.894063in}{5.600556in}}%
\pgfpathclose%
\pgfusepath{fill}%
\end{pgfscope}%
\begin{pgfscope}%
\pgfpathrectangle{\pgfqpoint{0.894063in}{3.540000in}}{\pgfqpoint{6.713438in}{2.060556in}} %
\pgfusepath{clip}%
\pgfsetbuttcap%
\pgfsetroundjoin%
\definecolor{currentfill}{rgb}{0.000000,0.500000,0.000000}%
\pgfsetfillcolor{currentfill}%
\pgfsetlinewidth{1.003750pt}%
\definecolor{currentstroke}{rgb}{0.000000,0.500000,0.000000}%
\pgfsetstrokecolor{currentstroke}%
\pgfsetdash{}{0pt}%
\pgfpathmoveto{\pgfqpoint{6.667619in}{5.257530in}}%
\pgfpathcurveto{\pgfqpoint{6.675855in}{5.257530in}}{\pgfqpoint{6.683755in}{5.260802in}}{\pgfqpoint{6.689579in}{5.266626in}}%
\pgfpathcurveto{\pgfqpoint{6.695403in}{5.272450in}}{\pgfqpoint{6.698675in}{5.280350in}}{\pgfqpoint{6.698675in}{5.288586in}}%
\pgfpathcurveto{\pgfqpoint{6.698675in}{5.296822in}}{\pgfqpoint{6.695403in}{5.304722in}}{\pgfqpoint{6.689579in}{5.310546in}}%
\pgfpathcurveto{\pgfqpoint{6.683755in}{5.316370in}}{\pgfqpoint{6.675855in}{5.319643in}}{\pgfqpoint{6.667619in}{5.319643in}}%
\pgfpathcurveto{\pgfqpoint{6.659382in}{5.319643in}}{\pgfqpoint{6.651482in}{5.316370in}}{\pgfqpoint{6.645658in}{5.310546in}}%
\pgfpathcurveto{\pgfqpoint{6.639835in}{5.304722in}}{\pgfqpoint{6.636562in}{5.296822in}}{\pgfqpoint{6.636562in}{5.288586in}}%
\pgfpathcurveto{\pgfqpoint{6.636562in}{5.280350in}}{\pgfqpoint{6.639835in}{5.272450in}}{\pgfqpoint{6.645658in}{5.266626in}}%
\pgfpathcurveto{\pgfqpoint{6.651482in}{5.260802in}}{\pgfqpoint{6.659382in}{5.257530in}}{\pgfqpoint{6.667619in}{5.257530in}}%
\pgfpathclose%
\pgfusepath{stroke,fill}%
\end{pgfscope}%
\begin{pgfscope}%
\pgfpathrectangle{\pgfqpoint{0.894063in}{3.540000in}}{\pgfqpoint{6.713438in}{2.060556in}} %
\pgfusepath{clip}%
\pgfsetbuttcap%
\pgfsetroundjoin%
\definecolor{currentfill}{rgb}{0.000000,0.500000,0.000000}%
\pgfsetfillcolor{currentfill}%
\pgfsetlinewidth{1.003750pt}%
\definecolor{currentstroke}{rgb}{0.000000,0.500000,0.000000}%
\pgfsetstrokecolor{currentstroke}%
\pgfsetdash{}{0pt}%
\pgfpathmoveto{\pgfqpoint{2.639556in}{5.257767in}}%
\pgfpathcurveto{\pgfqpoint{2.647793in}{5.257767in}}{\pgfqpoint{2.655693in}{5.261040in}}{\pgfqpoint{2.661517in}{5.266863in}}%
\pgfpathcurveto{\pgfqpoint{2.667340in}{5.272687in}}{\pgfqpoint{2.670613in}{5.280587in}}{\pgfqpoint{2.670613in}{5.288824in}}%
\pgfpathcurveto{\pgfqpoint{2.670613in}{5.297060in}}{\pgfqpoint{2.667340in}{5.304960in}}{\pgfqpoint{2.661517in}{5.310784in}}%
\pgfpathcurveto{\pgfqpoint{2.655693in}{5.316608in}}{\pgfqpoint{2.647793in}{5.319880in}}{\pgfqpoint{2.639556in}{5.319880in}}%
\pgfpathcurveto{\pgfqpoint{2.631320in}{5.319880in}}{\pgfqpoint{2.623420in}{5.316608in}}{\pgfqpoint{2.617596in}{5.310784in}}%
\pgfpathcurveto{\pgfqpoint{2.611772in}{5.304960in}}{\pgfqpoint{2.608500in}{5.297060in}}{\pgfqpoint{2.608500in}{5.288824in}}%
\pgfpathcurveto{\pgfqpoint{2.608500in}{5.280587in}}{\pgfqpoint{2.611772in}{5.272687in}}{\pgfqpoint{2.617596in}{5.266863in}}%
\pgfpathcurveto{\pgfqpoint{2.623420in}{5.261040in}}{\pgfqpoint{2.631320in}{5.257767in}}{\pgfqpoint{2.639556in}{5.257767in}}%
\pgfpathclose%
\pgfusepath{stroke,fill}%
\end{pgfscope}%
\begin{pgfscope}%
\pgfpathrectangle{\pgfqpoint{0.894063in}{3.540000in}}{\pgfqpoint{6.713438in}{2.060556in}} %
\pgfusepath{clip}%
\pgfsetbuttcap%
\pgfsetroundjoin%
\definecolor{currentfill}{rgb}{0.000000,0.500000,0.000000}%
\pgfsetfillcolor{currentfill}%
\pgfsetlinewidth{1.003750pt}%
\definecolor{currentstroke}{rgb}{0.000000,0.500000,0.000000}%
\pgfsetstrokecolor{currentstroke}%
\pgfsetdash{}{0pt}%
\pgfpathmoveto{\pgfqpoint{1.699675in}{5.258175in}}%
\pgfpathcurveto{\pgfqpoint{1.707911in}{5.258175in}}{\pgfqpoint{1.715811in}{5.261448in}}{\pgfqpoint{1.721635in}{5.267271in}}%
\pgfpathcurveto{\pgfqpoint{1.727459in}{5.273095in}}{\pgfqpoint{1.730731in}{5.280995in}}{\pgfqpoint{1.730731in}{5.289232in}}%
\pgfpathcurveto{\pgfqpoint{1.730731in}{5.297468in}}{\pgfqpoint{1.727459in}{5.305368in}}{\pgfqpoint{1.721635in}{5.311192in}}%
\pgfpathcurveto{\pgfqpoint{1.715811in}{5.317016in}}{\pgfqpoint{1.707911in}{5.320288in}}{\pgfqpoint{1.699675in}{5.320288in}}%
\pgfpathcurveto{\pgfqpoint{1.691439in}{5.320288in}}{\pgfqpoint{1.683539in}{5.317016in}}{\pgfqpoint{1.677715in}{5.311192in}}%
\pgfpathcurveto{\pgfqpoint{1.671891in}{5.305368in}}{\pgfqpoint{1.668619in}{5.297468in}}{\pgfqpoint{1.668619in}{5.289232in}}%
\pgfpathcurveto{\pgfqpoint{1.668619in}{5.280995in}}{\pgfqpoint{1.671891in}{5.273095in}}{\pgfqpoint{1.677715in}{5.267271in}}%
\pgfpathcurveto{\pgfqpoint{1.683539in}{5.261448in}}{\pgfqpoint{1.691439in}{5.258175in}}{\pgfqpoint{1.699675in}{5.258175in}}%
\pgfpathclose%
\pgfusepath{stroke,fill}%
\end{pgfscope}%
\begin{pgfscope}%
\pgfpathrectangle{\pgfqpoint{0.894063in}{3.540000in}}{\pgfqpoint{6.713438in}{2.060556in}} %
\pgfusepath{clip}%
\pgfsetbuttcap%
\pgfsetroundjoin%
\definecolor{currentfill}{rgb}{0.000000,0.500000,0.000000}%
\pgfsetfillcolor{currentfill}%
\pgfsetlinewidth{1.003750pt}%
\definecolor{currentstroke}{rgb}{0.000000,0.500000,0.000000}%
\pgfsetstrokecolor{currentstroke}%
\pgfsetdash{}{0pt}%
\pgfpathmoveto{\pgfqpoint{1.162600in}{5.258265in}}%
\pgfpathcurveto{\pgfqpoint{1.170836in}{5.258265in}}{\pgfqpoint{1.178736in}{5.261537in}}{\pgfqpoint{1.184560in}{5.267361in}}%
\pgfpathcurveto{\pgfqpoint{1.190384in}{5.273185in}}{\pgfqpoint{1.193656in}{5.281085in}}{\pgfqpoint{1.193656in}{5.289321in}}%
\pgfpathcurveto{\pgfqpoint{1.193656in}{5.297557in}}{\pgfqpoint{1.190384in}{5.305457in}}{\pgfqpoint{1.184560in}{5.311281in}}%
\pgfpathcurveto{\pgfqpoint{1.178736in}{5.317105in}}{\pgfqpoint{1.170836in}{5.320378in}}{\pgfqpoint{1.162600in}{5.320378in}}%
\pgfpathcurveto{\pgfqpoint{1.154364in}{5.320378in}}{\pgfqpoint{1.146464in}{5.317105in}}{\pgfqpoint{1.140640in}{5.311281in}}%
\pgfpathcurveto{\pgfqpoint{1.134816in}{5.305457in}}{\pgfqpoint{1.131544in}{5.297557in}}{\pgfqpoint{1.131544in}{5.289321in}}%
\pgfpathcurveto{\pgfqpoint{1.131544in}{5.281085in}}{\pgfqpoint{1.134816in}{5.273185in}}{\pgfqpoint{1.140640in}{5.267361in}}%
\pgfpathcurveto{\pgfqpoint{1.146464in}{5.261537in}}{\pgfqpoint{1.154364in}{5.258265in}}{\pgfqpoint{1.162600in}{5.258265in}}%
\pgfpathclose%
\pgfusepath{stroke,fill}%
\end{pgfscope}%
\begin{pgfscope}%
\pgfpathrectangle{\pgfqpoint{0.894063in}{3.540000in}}{\pgfqpoint{6.713438in}{2.060556in}} %
\pgfusepath{clip}%
\pgfsetbuttcap%
\pgfsetroundjoin%
\definecolor{currentfill}{rgb}{0.000000,0.500000,0.000000}%
\pgfsetfillcolor{currentfill}%
\pgfsetlinewidth{1.003750pt}%
\definecolor{currentstroke}{rgb}{0.000000,0.500000,0.000000}%
\pgfsetstrokecolor{currentstroke}%
\pgfsetdash{}{0pt}%
\pgfpathmoveto{\pgfqpoint{1.833944in}{5.258166in}}%
\pgfpathcurveto{\pgfqpoint{1.842180in}{5.258166in}}{\pgfqpoint{1.850080in}{5.261438in}}{\pgfqpoint{1.855904in}{5.267262in}}%
\pgfpathcurveto{\pgfqpoint{1.861728in}{5.273086in}}{\pgfqpoint{1.865000in}{5.280986in}}{\pgfqpoint{1.865000in}{5.289222in}}%
\pgfpathcurveto{\pgfqpoint{1.865000in}{5.297458in}}{\pgfqpoint{1.861728in}{5.305358in}}{\pgfqpoint{1.855904in}{5.311182in}}%
\pgfpathcurveto{\pgfqpoint{1.850080in}{5.317006in}}{\pgfqpoint{1.842180in}{5.320279in}}{\pgfqpoint{1.833944in}{5.320279in}}%
\pgfpathcurveto{\pgfqpoint{1.825707in}{5.320279in}}{\pgfqpoint{1.817807in}{5.317006in}}{\pgfqpoint{1.811983in}{5.311182in}}%
\pgfpathcurveto{\pgfqpoint{1.806160in}{5.305358in}}{\pgfqpoint{1.802887in}{5.297458in}}{\pgfqpoint{1.802887in}{5.289222in}}%
\pgfpathcurveto{\pgfqpoint{1.802887in}{5.280986in}}{\pgfqpoint{1.806160in}{5.273086in}}{\pgfqpoint{1.811983in}{5.267262in}}%
\pgfpathcurveto{\pgfqpoint{1.817807in}{5.261438in}}{\pgfqpoint{1.825707in}{5.258166in}}{\pgfqpoint{1.833944in}{5.258166in}}%
\pgfpathclose%
\pgfusepath{stroke,fill}%
\end{pgfscope}%
\begin{pgfscope}%
\pgfpathrectangle{\pgfqpoint{0.894063in}{3.540000in}}{\pgfqpoint{6.713438in}{2.060556in}} %
\pgfusepath{clip}%
\pgfsetbuttcap%
\pgfsetroundjoin%
\definecolor{currentfill}{rgb}{0.000000,0.500000,0.000000}%
\pgfsetfillcolor{currentfill}%
\pgfsetlinewidth{1.003750pt}%
\definecolor{currentstroke}{rgb}{0.000000,0.500000,0.000000}%
\pgfsetstrokecolor{currentstroke}%
\pgfsetdash{}{0pt}%
\pgfpathmoveto{\pgfqpoint{5.996275in}{5.257542in}}%
\pgfpathcurveto{\pgfqpoint{6.004511in}{5.257542in}}{\pgfqpoint{6.012411in}{5.260814in}}{\pgfqpoint{6.018235in}{5.266638in}}%
\pgfpathcurveto{\pgfqpoint{6.024059in}{5.272462in}}{\pgfqpoint{6.027331in}{5.280362in}}{\pgfqpoint{6.027331in}{5.288598in}}%
\pgfpathcurveto{\pgfqpoint{6.027331in}{5.296835in}}{\pgfqpoint{6.024059in}{5.304735in}}{\pgfqpoint{6.018235in}{5.310559in}}%
\pgfpathcurveto{\pgfqpoint{6.012411in}{5.316383in}}{\pgfqpoint{6.004511in}{5.319655in}}{\pgfqpoint{5.996275in}{5.319655in}}%
\pgfpathcurveto{\pgfqpoint{5.988039in}{5.319655in}}{\pgfqpoint{5.980139in}{5.316383in}}{\pgfqpoint{5.974315in}{5.310559in}}%
\pgfpathcurveto{\pgfqpoint{5.968491in}{5.304735in}}{\pgfqpoint{5.965219in}{5.296835in}}{\pgfqpoint{5.965219in}{5.288598in}}%
\pgfpathcurveto{\pgfqpoint{5.965219in}{5.280362in}}{\pgfqpoint{5.968491in}{5.272462in}}{\pgfqpoint{5.974315in}{5.266638in}}%
\pgfpathcurveto{\pgfqpoint{5.980139in}{5.260814in}}{\pgfqpoint{5.988039in}{5.257542in}}{\pgfqpoint{5.996275in}{5.257542in}}%
\pgfpathclose%
\pgfusepath{stroke,fill}%
\end{pgfscope}%
\begin{pgfscope}%
\pgfpathrectangle{\pgfqpoint{0.894063in}{3.540000in}}{\pgfqpoint{6.713438in}{2.060556in}} %
\pgfusepath{clip}%
\pgfsetbuttcap%
\pgfsetroundjoin%
\definecolor{currentfill}{rgb}{0.000000,0.500000,0.000000}%
\pgfsetfillcolor{currentfill}%
\pgfsetlinewidth{1.003750pt}%
\definecolor{currentstroke}{rgb}{0.000000,0.500000,0.000000}%
\pgfsetstrokecolor{currentstroke}%
\pgfsetdash{}{0pt}%
\pgfpathmoveto{\pgfqpoint{6.399081in}{5.257532in}}%
\pgfpathcurveto{\pgfqpoint{6.407318in}{5.257532in}}{\pgfqpoint{6.415218in}{5.260805in}}{\pgfqpoint{6.421042in}{5.266629in}}%
\pgfpathcurveto{\pgfqpoint{6.426865in}{5.272452in}}{\pgfqpoint{6.430138in}{5.280353in}}{\pgfqpoint{6.430138in}{5.288589in}}%
\pgfpathcurveto{\pgfqpoint{6.430138in}{5.296825in}}{\pgfqpoint{6.426865in}{5.304725in}}{\pgfqpoint{6.421042in}{5.310549in}}%
\pgfpathcurveto{\pgfqpoint{6.415218in}{5.316373in}}{\pgfqpoint{6.407318in}{5.319645in}}{\pgfqpoint{6.399081in}{5.319645in}}%
\pgfpathcurveto{\pgfqpoint{6.390845in}{5.319645in}}{\pgfqpoint{6.382945in}{5.316373in}}{\pgfqpoint{6.377121in}{5.310549in}}%
\pgfpathcurveto{\pgfqpoint{6.371297in}{5.304725in}}{\pgfqpoint{6.368025in}{5.296825in}}{\pgfqpoint{6.368025in}{5.288589in}}%
\pgfpathcurveto{\pgfqpoint{6.368025in}{5.280353in}}{\pgfqpoint{6.371297in}{5.272452in}}{\pgfqpoint{6.377121in}{5.266629in}}%
\pgfpathcurveto{\pgfqpoint{6.382945in}{5.260805in}}{\pgfqpoint{6.390845in}{5.257532in}}{\pgfqpoint{6.399081in}{5.257532in}}%
\pgfpathclose%
\pgfusepath{stroke,fill}%
\end{pgfscope}%
\begin{pgfscope}%
\pgfpathrectangle{\pgfqpoint{0.894063in}{3.540000in}}{\pgfqpoint{6.713438in}{2.060556in}} %
\pgfusepath{clip}%
\pgfsetbuttcap%
\pgfsetroundjoin%
\definecolor{currentfill}{rgb}{0.000000,0.500000,0.000000}%
\pgfsetfillcolor{currentfill}%
\pgfsetlinewidth{1.003750pt}%
\definecolor{currentstroke}{rgb}{0.000000,0.500000,0.000000}%
\pgfsetstrokecolor{currentstroke}%
\pgfsetdash{}{0pt}%
\pgfpathmoveto{\pgfqpoint{4.787856in}{5.257572in}}%
\pgfpathcurveto{\pgfqpoint{4.796093in}{5.257572in}}{\pgfqpoint{4.803993in}{5.260844in}}{\pgfqpoint{4.809817in}{5.266668in}}%
\pgfpathcurveto{\pgfqpoint{4.815640in}{5.272492in}}{\pgfqpoint{4.818913in}{5.280392in}}{\pgfqpoint{4.818913in}{5.288629in}}%
\pgfpathcurveto{\pgfqpoint{4.818913in}{5.296865in}}{\pgfqpoint{4.815640in}{5.304765in}}{\pgfqpoint{4.809817in}{5.310589in}}%
\pgfpathcurveto{\pgfqpoint{4.803993in}{5.316413in}}{\pgfqpoint{4.796093in}{5.319685in}}{\pgfqpoint{4.787856in}{5.319685in}}%
\pgfpathcurveto{\pgfqpoint{4.779620in}{5.319685in}}{\pgfqpoint{4.771720in}{5.316413in}}{\pgfqpoint{4.765896in}{5.310589in}}%
\pgfpathcurveto{\pgfqpoint{4.760072in}{5.304765in}}{\pgfqpoint{4.756800in}{5.296865in}}{\pgfqpoint{4.756800in}{5.288629in}}%
\pgfpathcurveto{\pgfqpoint{4.756800in}{5.280392in}}{\pgfqpoint{4.760072in}{5.272492in}}{\pgfqpoint{4.765896in}{5.266668in}}%
\pgfpathcurveto{\pgfqpoint{4.771720in}{5.260844in}}{\pgfqpoint{4.779620in}{5.257572in}}{\pgfqpoint{4.787856in}{5.257572in}}%
\pgfpathclose%
\pgfusepath{stroke,fill}%
\end{pgfscope}%
\begin{pgfscope}%
\pgfpathrectangle{\pgfqpoint{0.894063in}{3.540000in}}{\pgfqpoint{6.713438in}{2.060556in}} %
\pgfusepath{clip}%
\pgfsetbuttcap%
\pgfsetroundjoin%
\definecolor{currentfill}{rgb}{0.000000,0.500000,0.000000}%
\pgfsetfillcolor{currentfill}%
\pgfsetlinewidth{1.003750pt}%
\definecolor{currentstroke}{rgb}{0.000000,0.500000,0.000000}%
\pgfsetstrokecolor{currentstroke}%
\pgfsetdash{}{0pt}%
\pgfpathmoveto{\pgfqpoint{4.922125in}{5.257563in}}%
\pgfpathcurveto{\pgfqpoint{4.930361in}{5.257563in}}{\pgfqpoint{4.938261in}{5.260835in}}{\pgfqpoint{4.944085in}{5.266659in}}%
\pgfpathcurveto{\pgfqpoint{4.949909in}{5.272483in}}{\pgfqpoint{4.953181in}{5.280383in}}{\pgfqpoint{4.953181in}{5.288619in}}%
\pgfpathcurveto{\pgfqpoint{4.953181in}{5.296855in}}{\pgfqpoint{4.949909in}{5.304755in}}{\pgfqpoint{4.944085in}{5.310579in}}%
\pgfpathcurveto{\pgfqpoint{4.938261in}{5.316403in}}{\pgfqpoint{4.930361in}{5.319676in}}{\pgfqpoint{4.922125in}{5.319676in}}%
\pgfpathcurveto{\pgfqpoint{4.913889in}{5.319676in}}{\pgfqpoint{4.905989in}{5.316403in}}{\pgfqpoint{4.900165in}{5.310579in}}%
\pgfpathcurveto{\pgfqpoint{4.894341in}{5.304755in}}{\pgfqpoint{4.891069in}{5.296855in}}{\pgfqpoint{4.891069in}{5.288619in}}%
\pgfpathcurveto{\pgfqpoint{4.891069in}{5.280383in}}{\pgfqpoint{4.894341in}{5.272483in}}{\pgfqpoint{4.900165in}{5.266659in}}%
\pgfpathcurveto{\pgfqpoint{4.905989in}{5.260835in}}{\pgfqpoint{4.913889in}{5.257563in}}{\pgfqpoint{4.922125in}{5.257563in}}%
\pgfpathclose%
\pgfusepath{stroke,fill}%
\end{pgfscope}%
\begin{pgfscope}%
\pgfpathrectangle{\pgfqpoint{0.894063in}{3.540000in}}{\pgfqpoint{6.713438in}{2.060556in}} %
\pgfusepath{clip}%
\pgfsetbuttcap%
\pgfsetroundjoin%
\definecolor{currentfill}{rgb}{0.000000,0.500000,0.000000}%
\pgfsetfillcolor{currentfill}%
\pgfsetlinewidth{1.003750pt}%
\definecolor{currentstroke}{rgb}{0.000000,0.500000,0.000000}%
\pgfsetstrokecolor{currentstroke}%
\pgfsetdash{}{0pt}%
\pgfpathmoveto{\pgfqpoint{6.130544in}{5.257539in}}%
\pgfpathcurveto{\pgfqpoint{6.138780in}{5.257539in}}{\pgfqpoint{6.146680in}{5.260811in}}{\pgfqpoint{6.152504in}{5.266635in}}%
\pgfpathcurveto{\pgfqpoint{6.158328in}{5.272459in}}{\pgfqpoint{6.161600in}{5.280359in}}{\pgfqpoint{6.161600in}{5.288596in}}%
\pgfpathcurveto{\pgfqpoint{6.161600in}{5.296832in}}{\pgfqpoint{6.158328in}{5.304732in}}{\pgfqpoint{6.152504in}{5.310556in}}%
\pgfpathcurveto{\pgfqpoint{6.146680in}{5.316380in}}{\pgfqpoint{6.138780in}{5.319652in}}{\pgfqpoint{6.130544in}{5.319652in}}%
\pgfpathcurveto{\pgfqpoint{6.122307in}{5.319652in}}{\pgfqpoint{6.114407in}{5.316380in}}{\pgfqpoint{6.108583in}{5.310556in}}%
\pgfpathcurveto{\pgfqpoint{6.102760in}{5.304732in}}{\pgfqpoint{6.099487in}{5.296832in}}{\pgfqpoint{6.099487in}{5.288596in}}%
\pgfpathcurveto{\pgfqpoint{6.099487in}{5.280359in}}{\pgfqpoint{6.102760in}{5.272459in}}{\pgfqpoint{6.108583in}{5.266635in}}%
\pgfpathcurveto{\pgfqpoint{6.114407in}{5.260811in}}{\pgfqpoint{6.122307in}{5.257539in}}{\pgfqpoint{6.130544in}{5.257539in}}%
\pgfpathclose%
\pgfusepath{stroke,fill}%
\end{pgfscope}%
\begin{pgfscope}%
\pgfpathrectangle{\pgfqpoint{0.894063in}{3.540000in}}{\pgfqpoint{6.713438in}{2.060556in}} %
\pgfusepath{clip}%
\pgfsetbuttcap%
\pgfsetroundjoin%
\definecolor{currentfill}{rgb}{0.000000,0.500000,0.000000}%
\pgfsetfillcolor{currentfill}%
\pgfsetlinewidth{1.003750pt}%
\definecolor{currentstroke}{rgb}{0.000000,0.500000,0.000000}%
\pgfsetstrokecolor{currentstroke}%
\pgfsetdash{}{0pt}%
\pgfpathmoveto{\pgfqpoint{5.727738in}{5.257543in}}%
\pgfpathcurveto{\pgfqpoint{5.735974in}{5.257543in}}{\pgfqpoint{5.743874in}{5.260816in}}{\pgfqpoint{5.749698in}{5.266640in}}%
\pgfpathcurveto{\pgfqpoint{5.755522in}{5.272463in}}{\pgfqpoint{5.758794in}{5.280364in}}{\pgfqpoint{5.758794in}{5.288600in}}%
\pgfpathcurveto{\pgfqpoint{5.758794in}{5.296836in}}{\pgfqpoint{5.755522in}{5.304736in}}{\pgfqpoint{5.749698in}{5.310560in}}%
\pgfpathcurveto{\pgfqpoint{5.743874in}{5.316384in}}{\pgfqpoint{5.735974in}{5.319656in}}{\pgfqpoint{5.727738in}{5.319656in}}%
\pgfpathcurveto{\pgfqpoint{5.719501in}{5.319656in}}{\pgfqpoint{5.711601in}{5.316384in}}{\pgfqpoint{5.705777in}{5.310560in}}%
\pgfpathcurveto{\pgfqpoint{5.699953in}{5.304736in}}{\pgfqpoint{5.696681in}{5.296836in}}{\pgfqpoint{5.696681in}{5.288600in}}%
\pgfpathcurveto{\pgfqpoint{5.696681in}{5.280364in}}{\pgfqpoint{5.699953in}{5.272463in}}{\pgfqpoint{5.705777in}{5.266640in}}%
\pgfpathcurveto{\pgfqpoint{5.711601in}{5.260816in}}{\pgfqpoint{5.719501in}{5.257543in}}{\pgfqpoint{5.727738in}{5.257543in}}%
\pgfpathclose%
\pgfusepath{stroke,fill}%
\end{pgfscope}%
\begin{pgfscope}%
\pgfpathrectangle{\pgfqpoint{0.894063in}{3.540000in}}{\pgfqpoint{6.713438in}{2.060556in}} %
\pgfusepath{clip}%
\pgfsetbuttcap%
\pgfsetroundjoin%
\definecolor{currentfill}{rgb}{0.000000,0.500000,0.000000}%
\pgfsetfillcolor{currentfill}%
\pgfsetlinewidth{1.003750pt}%
\definecolor{currentstroke}{rgb}{0.000000,0.500000,0.000000}%
\pgfsetstrokecolor{currentstroke}%
\pgfsetdash{}{0pt}%
\pgfpathmoveto{\pgfqpoint{1.028331in}{5.258325in}}%
\pgfpathcurveto{\pgfqpoint{1.036568in}{5.258325in}}{\pgfqpoint{1.044468in}{5.261597in}}{\pgfqpoint{1.050292in}{5.267421in}}%
\pgfpathcurveto{\pgfqpoint{1.056115in}{5.273245in}}{\pgfqpoint{1.059388in}{5.281145in}}{\pgfqpoint{1.059388in}{5.289381in}}%
\pgfpathcurveto{\pgfqpoint{1.059388in}{5.297618in}}{\pgfqpoint{1.056115in}{5.305518in}}{\pgfqpoint{1.050292in}{5.311342in}}%
\pgfpathcurveto{\pgfqpoint{1.044468in}{5.317166in}}{\pgfqpoint{1.036568in}{5.320438in}}{\pgfqpoint{1.028331in}{5.320438in}}%
\pgfpathcurveto{\pgfqpoint{1.020095in}{5.320438in}}{\pgfqpoint{1.012195in}{5.317166in}}{\pgfqpoint{1.006371in}{5.311342in}}%
\pgfpathcurveto{\pgfqpoint{1.000547in}{5.305518in}}{\pgfqpoint{0.997275in}{5.297618in}}{\pgfqpoint{0.997275in}{5.289381in}}%
\pgfpathcurveto{\pgfqpoint{0.997275in}{5.281145in}}{\pgfqpoint{1.000547in}{5.273245in}}{\pgfqpoint{1.006371in}{5.267421in}}%
\pgfpathcurveto{\pgfqpoint{1.012195in}{5.261597in}}{\pgfqpoint{1.020095in}{5.258325in}}{\pgfqpoint{1.028331in}{5.258325in}}%
\pgfpathclose%
\pgfusepath{stroke,fill}%
\end{pgfscope}%
\begin{pgfscope}%
\pgfpathrectangle{\pgfqpoint{0.894063in}{3.540000in}}{\pgfqpoint{6.713438in}{2.060556in}} %
\pgfusepath{clip}%
\pgfsetbuttcap%
\pgfsetroundjoin%
\definecolor{currentfill}{rgb}{0.000000,0.500000,0.000000}%
\pgfsetfillcolor{currentfill}%
\pgfsetlinewidth{1.003750pt}%
\definecolor{currentstroke}{rgb}{0.000000,0.500000,0.000000}%
\pgfsetstrokecolor{currentstroke}%
\pgfsetdash{}{0pt}%
\pgfpathmoveto{\pgfqpoint{5.324931in}{5.257545in}}%
\pgfpathcurveto{\pgfqpoint{5.333168in}{5.257545in}}{\pgfqpoint{5.341068in}{5.260817in}}{\pgfqpoint{5.346892in}{5.266641in}}%
\pgfpathcurveto{\pgfqpoint{5.352715in}{5.272465in}}{\pgfqpoint{5.355988in}{5.280365in}}{\pgfqpoint{5.355988in}{5.288601in}}%
\pgfpathcurveto{\pgfqpoint{5.355988in}{5.296837in}}{\pgfqpoint{5.352715in}{5.304738in}}{\pgfqpoint{5.346892in}{5.310561in}}%
\pgfpathcurveto{\pgfqpoint{5.341068in}{5.316385in}}{\pgfqpoint{5.333168in}{5.319658in}}{\pgfqpoint{5.324931in}{5.319658in}}%
\pgfpathcurveto{\pgfqpoint{5.316695in}{5.319658in}}{\pgfqpoint{5.308795in}{5.316385in}}{\pgfqpoint{5.302971in}{5.310561in}}%
\pgfpathcurveto{\pgfqpoint{5.297147in}{5.304738in}}{\pgfqpoint{5.293875in}{5.296837in}}{\pgfqpoint{5.293875in}{5.288601in}}%
\pgfpathcurveto{\pgfqpoint{5.293875in}{5.280365in}}{\pgfqpoint{5.297147in}{5.272465in}}{\pgfqpoint{5.302971in}{5.266641in}}%
\pgfpathcurveto{\pgfqpoint{5.308795in}{5.260817in}}{\pgfqpoint{5.316695in}{5.257545in}}{\pgfqpoint{5.324931in}{5.257545in}}%
\pgfpathclose%
\pgfusepath{stroke,fill}%
\end{pgfscope}%
\begin{pgfscope}%
\pgfpathrectangle{\pgfqpoint{0.894063in}{3.540000in}}{\pgfqpoint{6.713438in}{2.060556in}} %
\pgfusepath{clip}%
\pgfsetbuttcap%
\pgfsetroundjoin%
\definecolor{currentfill}{rgb}{0.000000,0.500000,0.000000}%
\pgfsetfillcolor{currentfill}%
\pgfsetlinewidth{1.003750pt}%
\definecolor{currentstroke}{rgb}{0.000000,0.500000,0.000000}%
\pgfsetstrokecolor{currentstroke}%
\pgfsetdash{}{0pt}%
\pgfpathmoveto{\pgfqpoint{7.338963in}{5.256799in}}%
\pgfpathcurveto{\pgfqpoint{7.347199in}{5.256799in}}{\pgfqpoint{7.355099in}{5.260071in}}{\pgfqpoint{7.360923in}{5.265895in}}%
\pgfpathcurveto{\pgfqpoint{7.366747in}{5.271719in}}{\pgfqpoint{7.370019in}{5.279619in}}{\pgfqpoint{7.370019in}{5.287855in}}%
\pgfpathcurveto{\pgfqpoint{7.370019in}{5.296092in}}{\pgfqpoint{7.366747in}{5.303992in}}{\pgfqpoint{7.360923in}{5.309816in}}%
\pgfpathcurveto{\pgfqpoint{7.355099in}{5.315639in}}{\pgfqpoint{7.347199in}{5.318912in}}{\pgfqpoint{7.338963in}{5.318912in}}%
\pgfpathcurveto{\pgfqpoint{7.330726in}{5.318912in}}{\pgfqpoint{7.322826in}{5.315639in}}{\pgfqpoint{7.317002in}{5.309816in}}%
\pgfpathcurveto{\pgfqpoint{7.311178in}{5.303992in}}{\pgfqpoint{7.307906in}{5.296092in}}{\pgfqpoint{7.307906in}{5.287855in}}%
\pgfpathcurveto{\pgfqpoint{7.307906in}{5.279619in}}{\pgfqpoint{7.311178in}{5.271719in}}{\pgfqpoint{7.317002in}{5.265895in}}%
\pgfpathcurveto{\pgfqpoint{7.322826in}{5.260071in}}{\pgfqpoint{7.330726in}{5.256799in}}{\pgfqpoint{7.338963in}{5.256799in}}%
\pgfpathclose%
\pgfusepath{stroke,fill}%
\end{pgfscope}%
\begin{pgfscope}%
\pgfpathrectangle{\pgfqpoint{0.894063in}{3.540000in}}{\pgfqpoint{6.713438in}{2.060556in}} %
\pgfusepath{clip}%
\pgfsetbuttcap%
\pgfsetroundjoin%
\definecolor{currentfill}{rgb}{0.000000,0.500000,0.000000}%
\pgfsetfillcolor{currentfill}%
\pgfsetlinewidth{1.003750pt}%
\definecolor{currentstroke}{rgb}{0.000000,0.500000,0.000000}%
\pgfsetstrokecolor{currentstroke}%
\pgfsetdash{}{0pt}%
\pgfpathmoveto{\pgfqpoint{7.204694in}{5.256807in}}%
\pgfpathcurveto{\pgfqpoint{7.212930in}{5.256807in}}{\pgfqpoint{7.220830in}{5.260079in}}{\pgfqpoint{7.226654in}{5.265903in}}%
\pgfpathcurveto{\pgfqpoint{7.232478in}{5.271727in}}{\pgfqpoint{7.235750in}{5.279627in}}{\pgfqpoint{7.235750in}{5.287864in}}%
\pgfpathcurveto{\pgfqpoint{7.235750in}{5.296100in}}{\pgfqpoint{7.232478in}{5.304000in}}{\pgfqpoint{7.226654in}{5.309824in}}%
\pgfpathcurveto{\pgfqpoint{7.220830in}{5.315648in}}{\pgfqpoint{7.212930in}{5.318920in}}{\pgfqpoint{7.204694in}{5.318920in}}%
\pgfpathcurveto{\pgfqpoint{7.196457in}{5.318920in}}{\pgfqpoint{7.188557in}{5.315648in}}{\pgfqpoint{7.182733in}{5.309824in}}%
\pgfpathcurveto{\pgfqpoint{7.176910in}{5.304000in}}{\pgfqpoint{7.173637in}{5.296100in}}{\pgfqpoint{7.173637in}{5.287864in}}%
\pgfpathcurveto{\pgfqpoint{7.173637in}{5.279627in}}{\pgfqpoint{7.176910in}{5.271727in}}{\pgfqpoint{7.182733in}{5.265903in}}%
\pgfpathcurveto{\pgfqpoint{7.188557in}{5.260079in}}{\pgfqpoint{7.196457in}{5.256807in}}{\pgfqpoint{7.204694in}{5.256807in}}%
\pgfpathclose%
\pgfusepath{stroke,fill}%
\end{pgfscope}%
\begin{pgfscope}%
\pgfpathrectangle{\pgfqpoint{0.894063in}{3.540000in}}{\pgfqpoint{6.713438in}{2.060556in}} %
\pgfusepath{clip}%
\pgfsetbuttcap%
\pgfsetroundjoin%
\definecolor{currentfill}{rgb}{0.000000,0.500000,0.000000}%
\pgfsetfillcolor{currentfill}%
\pgfsetlinewidth{1.003750pt}%
\definecolor{currentstroke}{rgb}{0.000000,0.500000,0.000000}%
\pgfsetstrokecolor{currentstroke}%
\pgfsetdash{}{0pt}%
\pgfpathmoveto{\pgfqpoint{6.264813in}{5.257534in}}%
\pgfpathcurveto{\pgfqpoint{6.273049in}{5.257534in}}{\pgfqpoint{6.280949in}{5.260806in}}{\pgfqpoint{6.286773in}{5.266630in}}%
\pgfpathcurveto{\pgfqpoint{6.292597in}{5.272454in}}{\pgfqpoint{6.295869in}{5.280354in}}{\pgfqpoint{6.295869in}{5.288590in}}%
\pgfpathcurveto{\pgfqpoint{6.295869in}{5.296826in}}{\pgfqpoint{6.292597in}{5.304727in}}{\pgfqpoint{6.286773in}{5.310550in}}%
\pgfpathcurveto{\pgfqpoint{6.280949in}{5.316374in}}{\pgfqpoint{6.273049in}{5.319647in}}{\pgfqpoint{6.264813in}{5.319647in}}%
\pgfpathcurveto{\pgfqpoint{6.256576in}{5.319647in}}{\pgfqpoint{6.248676in}{5.316374in}}{\pgfqpoint{6.242852in}{5.310550in}}%
\pgfpathcurveto{\pgfqpoint{6.237028in}{5.304727in}}{\pgfqpoint{6.233756in}{5.296826in}}{\pgfqpoint{6.233756in}{5.288590in}}%
\pgfpathcurveto{\pgfqpoint{6.233756in}{5.280354in}}{\pgfqpoint{6.237028in}{5.272454in}}{\pgfqpoint{6.242852in}{5.266630in}}%
\pgfpathcurveto{\pgfqpoint{6.248676in}{5.260806in}}{\pgfqpoint{6.256576in}{5.257534in}}{\pgfqpoint{6.264813in}{5.257534in}}%
\pgfpathclose%
\pgfusepath{stroke,fill}%
\end{pgfscope}%
\begin{pgfscope}%
\pgfpathrectangle{\pgfqpoint{0.894063in}{3.540000in}}{\pgfqpoint{6.713438in}{2.060556in}} %
\pgfusepath{clip}%
\pgfsetbuttcap%
\pgfsetroundjoin%
\definecolor{currentfill}{rgb}{0.000000,0.500000,0.000000}%
\pgfsetfillcolor{currentfill}%
\pgfsetlinewidth{1.003750pt}%
\definecolor{currentstroke}{rgb}{0.000000,0.500000,0.000000}%
\pgfsetstrokecolor{currentstroke}%
\pgfsetdash{}{0pt}%
\pgfpathmoveto{\pgfqpoint{7.473231in}{5.256785in}}%
\pgfpathcurveto{\pgfqpoint{7.481468in}{5.256785in}}{\pgfqpoint{7.489368in}{5.260057in}}{\pgfqpoint{7.495192in}{5.265881in}}%
\pgfpathcurveto{\pgfqpoint{7.501015in}{5.271705in}}{\pgfqpoint{7.504288in}{5.279605in}}{\pgfqpoint{7.504288in}{5.287842in}}%
\pgfpathcurveto{\pgfqpoint{7.504288in}{5.296078in}}{\pgfqpoint{7.501015in}{5.303978in}}{\pgfqpoint{7.495192in}{5.309802in}}%
\pgfpathcurveto{\pgfqpoint{7.489368in}{5.315626in}}{\pgfqpoint{7.481468in}{5.318898in}}{\pgfqpoint{7.473231in}{5.318898in}}%
\pgfpathcurveto{\pgfqpoint{7.464995in}{5.318898in}}{\pgfqpoint{7.457095in}{5.315626in}}{\pgfqpoint{7.451271in}{5.309802in}}%
\pgfpathcurveto{\pgfqpoint{7.445447in}{5.303978in}}{\pgfqpoint{7.442175in}{5.296078in}}{\pgfqpoint{7.442175in}{5.287842in}}%
\pgfpathcurveto{\pgfqpoint{7.442175in}{5.279605in}}{\pgfqpoint{7.445447in}{5.271705in}}{\pgfqpoint{7.451271in}{5.265881in}}%
\pgfpathcurveto{\pgfqpoint{7.457095in}{5.260057in}}{\pgfqpoint{7.464995in}{5.256785in}}{\pgfqpoint{7.473231in}{5.256785in}}%
\pgfpathclose%
\pgfusepath{stroke,fill}%
\end{pgfscope}%
\begin{pgfscope}%
\pgfpathrectangle{\pgfqpoint{0.894063in}{3.540000in}}{\pgfqpoint{6.713438in}{2.060556in}} %
\pgfusepath{clip}%
\pgfsetbuttcap%
\pgfsetroundjoin%
\definecolor{currentfill}{rgb}{0.000000,0.500000,0.000000}%
\pgfsetfillcolor{currentfill}%
\pgfsetlinewidth{1.003750pt}%
\definecolor{currentstroke}{rgb}{0.000000,0.500000,0.000000}%
\pgfsetstrokecolor{currentstroke}%
\pgfsetdash{}{0pt}%
\pgfpathmoveto{\pgfqpoint{5.056394in}{5.257558in}}%
\pgfpathcurveto{\pgfqpoint{5.064630in}{5.257558in}}{\pgfqpoint{5.072530in}{5.260831in}}{\pgfqpoint{5.078354in}{5.266655in}}%
\pgfpathcurveto{\pgfqpoint{5.084178in}{5.272479in}}{\pgfqpoint{5.087450in}{5.280379in}}{\pgfqpoint{5.087450in}{5.288615in}}%
\pgfpathcurveto{\pgfqpoint{5.087450in}{5.296851in}}{\pgfqpoint{5.084178in}{5.304751in}}{\pgfqpoint{5.078354in}{5.310575in}}%
\pgfpathcurveto{\pgfqpoint{5.072530in}{5.316399in}}{\pgfqpoint{5.064630in}{5.319671in}}{\pgfqpoint{5.056394in}{5.319671in}}%
\pgfpathcurveto{\pgfqpoint{5.048157in}{5.319671in}}{\pgfqpoint{5.040257in}{5.316399in}}{\pgfqpoint{5.034433in}{5.310575in}}%
\pgfpathcurveto{\pgfqpoint{5.028610in}{5.304751in}}{\pgfqpoint{5.025337in}{5.296851in}}{\pgfqpoint{5.025337in}{5.288615in}}%
\pgfpathcurveto{\pgfqpoint{5.025337in}{5.280379in}}{\pgfqpoint{5.028610in}{5.272479in}}{\pgfqpoint{5.034433in}{5.266655in}}%
\pgfpathcurveto{\pgfqpoint{5.040257in}{5.260831in}}{\pgfqpoint{5.048157in}{5.257558in}}{\pgfqpoint{5.056394in}{5.257558in}}%
\pgfpathclose%
\pgfusepath{stroke,fill}%
\end{pgfscope}%
\begin{pgfscope}%
\pgfpathrectangle{\pgfqpoint{0.894063in}{3.540000in}}{\pgfqpoint{6.713438in}{2.060556in}} %
\pgfusepath{clip}%
\pgfsetbuttcap%
\pgfsetroundjoin%
\definecolor{currentfill}{rgb}{0.000000,0.500000,0.000000}%
\pgfsetfillcolor{currentfill}%
\pgfsetlinewidth{1.003750pt}%
\definecolor{currentstroke}{rgb}{0.000000,0.500000,0.000000}%
\pgfsetstrokecolor{currentstroke}%
\pgfsetdash{}{0pt}%
\pgfpathmoveto{\pgfqpoint{2.908094in}{5.257701in}}%
\pgfpathcurveto{\pgfqpoint{2.916330in}{5.257701in}}{\pgfqpoint{2.924230in}{5.260974in}}{\pgfqpoint{2.930054in}{5.266798in}}%
\pgfpathcurveto{\pgfqpoint{2.935878in}{5.272621in}}{\pgfqpoint{2.939150in}{5.280522in}}{\pgfqpoint{2.939150in}{5.288758in}}%
\pgfpathcurveto{\pgfqpoint{2.939150in}{5.296994in}}{\pgfqpoint{2.935878in}{5.304894in}}{\pgfqpoint{2.930054in}{5.310718in}}%
\pgfpathcurveto{\pgfqpoint{2.924230in}{5.316542in}}{\pgfqpoint{2.916330in}{5.319814in}}{\pgfqpoint{2.908094in}{5.319814in}}%
\pgfpathcurveto{\pgfqpoint{2.899857in}{5.319814in}}{\pgfqpoint{2.891957in}{5.316542in}}{\pgfqpoint{2.886133in}{5.310718in}}%
\pgfpathcurveto{\pgfqpoint{2.880310in}{5.304894in}}{\pgfqpoint{2.877037in}{5.296994in}}{\pgfqpoint{2.877037in}{5.288758in}}%
\pgfpathcurveto{\pgfqpoint{2.877037in}{5.280522in}}{\pgfqpoint{2.880310in}{5.272621in}}{\pgfqpoint{2.886133in}{5.266798in}}%
\pgfpathcurveto{\pgfqpoint{2.891957in}{5.260974in}}{\pgfqpoint{2.899857in}{5.257701in}}{\pgfqpoint{2.908094in}{5.257701in}}%
\pgfpathclose%
\pgfusepath{stroke,fill}%
\end{pgfscope}%
\begin{pgfscope}%
\pgfpathrectangle{\pgfqpoint{0.894063in}{3.540000in}}{\pgfqpoint{6.713438in}{2.060556in}} %
\pgfusepath{clip}%
\pgfsetbuttcap%
\pgfsetroundjoin%
\definecolor{currentfill}{rgb}{0.000000,0.500000,0.000000}%
\pgfsetfillcolor{currentfill}%
\pgfsetlinewidth{1.003750pt}%
\definecolor{currentstroke}{rgb}{0.000000,0.500000,0.000000}%
\pgfsetstrokecolor{currentstroke}%
\pgfsetdash{}{0pt}%
\pgfpathmoveto{\pgfqpoint{3.445169in}{5.257692in}}%
\pgfpathcurveto{\pgfqpoint{3.453405in}{5.257692in}}{\pgfqpoint{3.461305in}{5.260964in}}{\pgfqpoint{3.467129in}{5.266788in}}%
\pgfpathcurveto{\pgfqpoint{3.472953in}{5.272612in}}{\pgfqpoint{3.476225in}{5.280512in}}{\pgfqpoint{3.476225in}{5.288748in}}%
\pgfpathcurveto{\pgfqpoint{3.476225in}{5.296984in}}{\pgfqpoint{3.472953in}{5.304884in}}{\pgfqpoint{3.467129in}{5.310708in}}%
\pgfpathcurveto{\pgfqpoint{3.461305in}{5.316532in}}{\pgfqpoint{3.453405in}{5.319805in}}{\pgfqpoint{3.445169in}{5.319805in}}%
\pgfpathcurveto{\pgfqpoint{3.436932in}{5.319805in}}{\pgfqpoint{3.429032in}{5.316532in}}{\pgfqpoint{3.423208in}{5.310708in}}%
\pgfpathcurveto{\pgfqpoint{3.417385in}{5.304884in}}{\pgfqpoint{3.414112in}{5.296984in}}{\pgfqpoint{3.414112in}{5.288748in}}%
\pgfpathcurveto{\pgfqpoint{3.414112in}{5.280512in}}{\pgfqpoint{3.417385in}{5.272612in}}{\pgfqpoint{3.423208in}{5.266788in}}%
\pgfpathcurveto{\pgfqpoint{3.429032in}{5.260964in}}{\pgfqpoint{3.436932in}{5.257692in}}{\pgfqpoint{3.445169in}{5.257692in}}%
\pgfpathclose%
\pgfusepath{stroke,fill}%
\end{pgfscope}%
\begin{pgfscope}%
\pgfpathrectangle{\pgfqpoint{0.894063in}{3.540000in}}{\pgfqpoint{6.713438in}{2.060556in}} %
\pgfusepath{clip}%
\pgfsetbuttcap%
\pgfsetroundjoin%
\definecolor{currentfill}{rgb}{0.000000,0.500000,0.000000}%
\pgfsetfillcolor{currentfill}%
\pgfsetlinewidth{1.003750pt}%
\definecolor{currentstroke}{rgb}{0.000000,0.500000,0.000000}%
\pgfsetstrokecolor{currentstroke}%
\pgfsetdash{}{0pt}%
\pgfpathmoveto{\pgfqpoint{4.116513in}{5.257608in}}%
\pgfpathcurveto{\pgfqpoint{4.124749in}{5.257608in}}{\pgfqpoint{4.132649in}{5.260880in}}{\pgfqpoint{4.138473in}{5.266704in}}%
\pgfpathcurveto{\pgfqpoint{4.144297in}{5.272528in}}{\pgfqpoint{4.147569in}{5.280428in}}{\pgfqpoint{4.147569in}{5.288664in}}%
\pgfpathcurveto{\pgfqpoint{4.147569in}{5.296901in}}{\pgfqpoint{4.144297in}{5.304801in}}{\pgfqpoint{4.138473in}{5.310625in}}%
\pgfpathcurveto{\pgfqpoint{4.132649in}{5.316449in}}{\pgfqpoint{4.124749in}{5.319721in}}{\pgfqpoint{4.116513in}{5.319721in}}%
\pgfpathcurveto{\pgfqpoint{4.108276in}{5.319721in}}{\pgfqpoint{4.100376in}{5.316449in}}{\pgfqpoint{4.094552in}{5.310625in}}%
\pgfpathcurveto{\pgfqpoint{4.088728in}{5.304801in}}{\pgfqpoint{4.085456in}{5.296901in}}{\pgfqpoint{4.085456in}{5.288664in}}%
\pgfpathcurveto{\pgfqpoint{4.085456in}{5.280428in}}{\pgfqpoint{4.088728in}{5.272528in}}{\pgfqpoint{4.094552in}{5.266704in}}%
\pgfpathcurveto{\pgfqpoint{4.100376in}{5.260880in}}{\pgfqpoint{4.108276in}{5.257608in}}{\pgfqpoint{4.116513in}{5.257608in}}%
\pgfpathclose%
\pgfusepath{stroke,fill}%
\end{pgfscope}%
\begin{pgfscope}%
\pgfpathrectangle{\pgfqpoint{0.894063in}{3.540000in}}{\pgfqpoint{6.713438in}{2.060556in}} %
\pgfusepath{clip}%
\pgfsetbuttcap%
\pgfsetroundjoin%
\definecolor{currentfill}{rgb}{0.000000,0.500000,0.000000}%
\pgfsetfillcolor{currentfill}%
\pgfsetlinewidth{1.003750pt}%
\definecolor{currentstroke}{rgb}{0.000000,0.500000,0.000000}%
\pgfsetstrokecolor{currentstroke}%
\pgfsetdash{}{0pt}%
\pgfpathmoveto{\pgfqpoint{1.431138in}{5.258190in}}%
\pgfpathcurveto{\pgfqpoint{1.439374in}{5.258190in}}{\pgfqpoint{1.447274in}{5.261463in}}{\pgfqpoint{1.453098in}{5.267287in}}%
\pgfpathcurveto{\pgfqpoint{1.458922in}{5.273110in}}{\pgfqpoint{1.462194in}{5.281011in}}{\pgfqpoint{1.462194in}{5.289247in}}%
\pgfpathcurveto{\pgfqpoint{1.462194in}{5.297483in}}{\pgfqpoint{1.458922in}{5.305383in}}{\pgfqpoint{1.453098in}{5.311207in}}%
\pgfpathcurveto{\pgfqpoint{1.447274in}{5.317031in}}{\pgfqpoint{1.439374in}{5.320303in}}{\pgfqpoint{1.431138in}{5.320303in}}%
\pgfpathcurveto{\pgfqpoint{1.422901in}{5.320303in}}{\pgfqpoint{1.415001in}{5.317031in}}{\pgfqpoint{1.409177in}{5.311207in}}%
\pgfpathcurveto{\pgfqpoint{1.403353in}{5.305383in}}{\pgfqpoint{1.400081in}{5.297483in}}{\pgfqpoint{1.400081in}{5.289247in}}%
\pgfpathcurveto{\pgfqpoint{1.400081in}{5.281011in}}{\pgfqpoint{1.403353in}{5.273110in}}{\pgfqpoint{1.409177in}{5.267287in}}%
\pgfpathcurveto{\pgfqpoint{1.415001in}{5.261463in}}{\pgfqpoint{1.422901in}{5.258190in}}{\pgfqpoint{1.431138in}{5.258190in}}%
\pgfpathclose%
\pgfusepath{stroke,fill}%
\end{pgfscope}%
\begin{pgfscope}%
\pgfpathrectangle{\pgfqpoint{0.894063in}{3.540000in}}{\pgfqpoint{6.713438in}{2.060556in}} %
\pgfusepath{clip}%
\pgfsetbuttcap%
\pgfsetroundjoin%
\definecolor{currentfill}{rgb}{0.000000,0.500000,0.000000}%
\pgfsetfillcolor{currentfill}%
\pgfsetlinewidth{1.003750pt}%
\definecolor{currentstroke}{rgb}{0.000000,0.500000,0.000000}%
\pgfsetstrokecolor{currentstroke}%
\pgfsetdash{}{0pt}%
\pgfpathmoveto{\pgfqpoint{2.773825in}{5.257707in}}%
\pgfpathcurveto{\pgfqpoint{2.782061in}{5.257707in}}{\pgfqpoint{2.789961in}{5.260979in}}{\pgfqpoint{2.795785in}{5.266803in}}%
\pgfpathcurveto{\pgfqpoint{2.801609in}{5.272627in}}{\pgfqpoint{2.804881in}{5.280527in}}{\pgfqpoint{2.804881in}{5.288763in}}%
\pgfpathcurveto{\pgfqpoint{2.804881in}{5.297000in}}{\pgfqpoint{2.801609in}{5.304900in}}{\pgfqpoint{2.795785in}{5.310724in}}%
\pgfpathcurveto{\pgfqpoint{2.789961in}{5.316547in}}{\pgfqpoint{2.782061in}{5.319820in}}{\pgfqpoint{2.773825in}{5.319820in}}%
\pgfpathcurveto{\pgfqpoint{2.765589in}{5.319820in}}{\pgfqpoint{2.757689in}{5.316547in}}{\pgfqpoint{2.751865in}{5.310724in}}%
\pgfpathcurveto{\pgfqpoint{2.746041in}{5.304900in}}{\pgfqpoint{2.742769in}{5.297000in}}{\pgfqpoint{2.742769in}{5.288763in}}%
\pgfpathcurveto{\pgfqpoint{2.742769in}{5.280527in}}{\pgfqpoint{2.746041in}{5.272627in}}{\pgfqpoint{2.751865in}{5.266803in}}%
\pgfpathcurveto{\pgfqpoint{2.757689in}{5.260979in}}{\pgfqpoint{2.765589in}{5.257707in}}{\pgfqpoint{2.773825in}{5.257707in}}%
\pgfpathclose%
\pgfusepath{stroke,fill}%
\end{pgfscope}%
\begin{pgfscope}%
\pgfpathrectangle{\pgfqpoint{0.894063in}{3.540000in}}{\pgfqpoint{6.713438in}{2.060556in}} %
\pgfusepath{clip}%
\pgfsetbuttcap%
\pgfsetroundjoin%
\definecolor{currentfill}{rgb}{0.000000,0.500000,0.000000}%
\pgfsetfillcolor{currentfill}%
\pgfsetlinewidth{1.003750pt}%
\definecolor{currentstroke}{rgb}{0.000000,0.500000,0.000000}%
\pgfsetstrokecolor{currentstroke}%
\pgfsetdash{}{0pt}%
\pgfpathmoveto{\pgfqpoint{1.565406in}{5.258181in}}%
\pgfpathcurveto{\pgfqpoint{1.573643in}{5.258181in}}{\pgfqpoint{1.581543in}{5.261453in}}{\pgfqpoint{1.587367in}{5.267277in}}%
\pgfpathcurveto{\pgfqpoint{1.593190in}{5.273101in}}{\pgfqpoint{1.596463in}{5.281001in}}{\pgfqpoint{1.596463in}{5.289237in}}%
\pgfpathcurveto{\pgfqpoint{1.596463in}{5.297473in}}{\pgfqpoint{1.593190in}{5.305374in}}{\pgfqpoint{1.587367in}{5.311197in}}%
\pgfpathcurveto{\pgfqpoint{1.581543in}{5.317021in}}{\pgfqpoint{1.573643in}{5.320294in}}{\pgfqpoint{1.565406in}{5.320294in}}%
\pgfpathcurveto{\pgfqpoint{1.557170in}{5.320294in}}{\pgfqpoint{1.549270in}{5.317021in}}{\pgfqpoint{1.543446in}{5.311197in}}%
\pgfpathcurveto{\pgfqpoint{1.537622in}{5.305374in}}{\pgfqpoint{1.534350in}{5.297473in}}{\pgfqpoint{1.534350in}{5.289237in}}%
\pgfpathcurveto{\pgfqpoint{1.534350in}{5.281001in}}{\pgfqpoint{1.537622in}{5.273101in}}{\pgfqpoint{1.543446in}{5.267277in}}%
\pgfpathcurveto{\pgfqpoint{1.549270in}{5.261453in}}{\pgfqpoint{1.557170in}{5.258181in}}{\pgfqpoint{1.565406in}{5.258181in}}%
\pgfpathclose%
\pgfusepath{stroke,fill}%
\end{pgfscope}%
\begin{pgfscope}%
\pgfpathrectangle{\pgfqpoint{0.894063in}{3.540000in}}{\pgfqpoint{6.713438in}{2.060556in}} %
\pgfusepath{clip}%
\pgfsetbuttcap%
\pgfsetroundjoin%
\definecolor{currentfill}{rgb}{0.000000,0.500000,0.000000}%
\pgfsetfillcolor{currentfill}%
\pgfsetlinewidth{1.003750pt}%
\definecolor{currentstroke}{rgb}{0.000000,0.500000,0.000000}%
\pgfsetstrokecolor{currentstroke}%
\pgfsetdash{}{0pt}%
\pgfpathmoveto{\pgfqpoint{4.250781in}{5.257606in}}%
\pgfpathcurveto{\pgfqpoint{4.259018in}{5.257606in}}{\pgfqpoint{4.266918in}{5.260879in}}{\pgfqpoint{4.272742in}{5.266703in}}%
\pgfpathcurveto{\pgfqpoint{4.278565in}{5.272527in}}{\pgfqpoint{4.281838in}{5.280427in}}{\pgfqpoint{4.281838in}{5.288663in}}%
\pgfpathcurveto{\pgfqpoint{4.281838in}{5.296899in}}{\pgfqpoint{4.278565in}{5.304799in}}{\pgfqpoint{4.272742in}{5.310623in}}%
\pgfpathcurveto{\pgfqpoint{4.266918in}{5.316447in}}{\pgfqpoint{4.259018in}{5.319719in}}{\pgfqpoint{4.250781in}{5.319719in}}%
\pgfpathcurveto{\pgfqpoint{4.242545in}{5.319719in}}{\pgfqpoint{4.234645in}{5.316447in}}{\pgfqpoint{4.228821in}{5.310623in}}%
\pgfpathcurveto{\pgfqpoint{4.222997in}{5.304799in}}{\pgfqpoint{4.219725in}{5.296899in}}{\pgfqpoint{4.219725in}{5.288663in}}%
\pgfpathcurveto{\pgfqpoint{4.219725in}{5.280427in}}{\pgfqpoint{4.222997in}{5.272527in}}{\pgfqpoint{4.228821in}{5.266703in}}%
\pgfpathcurveto{\pgfqpoint{4.234645in}{5.260879in}}{\pgfqpoint{4.242545in}{5.257606in}}{\pgfqpoint{4.250781in}{5.257606in}}%
\pgfpathclose%
\pgfusepath{stroke,fill}%
\end{pgfscope}%
\begin{pgfscope}%
\pgfpathrectangle{\pgfqpoint{0.894063in}{3.540000in}}{\pgfqpoint{6.713438in}{2.060556in}} %
\pgfusepath{clip}%
\pgfsetbuttcap%
\pgfsetroundjoin%
\definecolor{currentfill}{rgb}{0.000000,0.500000,0.000000}%
\pgfsetfillcolor{currentfill}%
\pgfsetlinewidth{1.003750pt}%
\definecolor{currentstroke}{rgb}{0.000000,0.500000,0.000000}%
\pgfsetstrokecolor{currentstroke}%
\pgfsetdash{}{0pt}%
\pgfpathmoveto{\pgfqpoint{3.847975in}{5.257627in}}%
\pgfpathcurveto{\pgfqpoint{3.856211in}{5.257627in}}{\pgfqpoint{3.864111in}{5.260899in}}{\pgfqpoint{3.869935in}{5.266723in}}%
\pgfpathcurveto{\pgfqpoint{3.875759in}{5.272547in}}{\pgfqpoint{3.879031in}{5.280447in}}{\pgfqpoint{3.879031in}{5.288684in}}%
\pgfpathcurveto{\pgfqpoint{3.879031in}{5.296920in}}{\pgfqpoint{3.875759in}{5.304820in}}{\pgfqpoint{3.869935in}{5.310644in}}%
\pgfpathcurveto{\pgfqpoint{3.864111in}{5.316468in}}{\pgfqpoint{3.856211in}{5.319740in}}{\pgfqpoint{3.847975in}{5.319740in}}%
\pgfpathcurveto{\pgfqpoint{3.839739in}{5.319740in}}{\pgfqpoint{3.831839in}{5.316468in}}{\pgfqpoint{3.826015in}{5.310644in}}%
\pgfpathcurveto{\pgfqpoint{3.820191in}{5.304820in}}{\pgfqpoint{3.816919in}{5.296920in}}{\pgfqpoint{3.816919in}{5.288684in}}%
\pgfpathcurveto{\pgfqpoint{3.816919in}{5.280447in}}{\pgfqpoint{3.820191in}{5.272547in}}{\pgfqpoint{3.826015in}{5.266723in}}%
\pgfpathcurveto{\pgfqpoint{3.831839in}{5.260899in}}{\pgfqpoint{3.839739in}{5.257627in}}{\pgfqpoint{3.847975in}{5.257627in}}%
\pgfpathclose%
\pgfusepath{stroke,fill}%
\end{pgfscope}%
\begin{pgfscope}%
\pgfpathrectangle{\pgfqpoint{0.894063in}{3.540000in}}{\pgfqpoint{6.713438in}{2.060556in}} %
\pgfusepath{clip}%
\pgfsetbuttcap%
\pgfsetroundjoin%
\definecolor{currentfill}{rgb}{0.000000,0.500000,0.000000}%
\pgfsetfillcolor{currentfill}%
\pgfsetlinewidth{1.003750pt}%
\definecolor{currentstroke}{rgb}{0.000000,0.500000,0.000000}%
\pgfsetstrokecolor{currentstroke}%
\pgfsetdash{}{0pt}%
\pgfpathmoveto{\pgfqpoint{7.607500in}{5.256782in}}%
\pgfpathcurveto{\pgfqpoint{7.615736in}{5.256782in}}{\pgfqpoint{7.623636in}{5.260055in}}{\pgfqpoint{7.629460in}{5.265879in}}%
\pgfpathcurveto{\pgfqpoint{7.635284in}{5.271702in}}{\pgfqpoint{7.638556in}{5.279602in}}{\pgfqpoint{7.638556in}{5.287839in}}%
\pgfpathcurveto{\pgfqpoint{7.638556in}{5.296075in}}{\pgfqpoint{7.635284in}{5.303975in}}{\pgfqpoint{7.629460in}{5.309799in}}%
\pgfpathcurveto{\pgfqpoint{7.623636in}{5.315623in}}{\pgfqpoint{7.615736in}{5.318895in}}{\pgfqpoint{7.607500in}{5.318895in}}%
\pgfpathcurveto{\pgfqpoint{7.599264in}{5.318895in}}{\pgfqpoint{7.591364in}{5.315623in}}{\pgfqpoint{7.585540in}{5.309799in}}%
\pgfpathcurveto{\pgfqpoint{7.579716in}{5.303975in}}{\pgfqpoint{7.576444in}{5.296075in}}{\pgfqpoint{7.576444in}{5.287839in}}%
\pgfpathcurveto{\pgfqpoint{7.576444in}{5.279602in}}{\pgfqpoint{7.579716in}{5.271702in}}{\pgfqpoint{7.585540in}{5.265879in}}%
\pgfpathcurveto{\pgfqpoint{7.591364in}{5.260055in}}{\pgfqpoint{7.599264in}{5.256782in}}{\pgfqpoint{7.607500in}{5.256782in}}%
\pgfpathclose%
\pgfusepath{stroke,fill}%
\end{pgfscope}%
\begin{pgfscope}%
\pgfpathrectangle{\pgfqpoint{0.894063in}{3.540000in}}{\pgfqpoint{6.713438in}{2.060556in}} %
\pgfusepath{clip}%
\pgfsetbuttcap%
\pgfsetroundjoin%
\definecolor{currentfill}{rgb}{0.000000,0.500000,0.000000}%
\pgfsetfillcolor{currentfill}%
\pgfsetlinewidth{1.003750pt}%
\definecolor{currentstroke}{rgb}{0.000000,0.500000,0.000000}%
\pgfsetstrokecolor{currentstroke}%
\pgfsetdash{}{0pt}%
\pgfpathmoveto{\pgfqpoint{4.385050in}{5.257605in}}%
\pgfpathcurveto{\pgfqpoint{4.393286in}{5.257605in}}{\pgfqpoint{4.401186in}{5.260877in}}{\pgfqpoint{4.407010in}{5.266701in}}%
\pgfpathcurveto{\pgfqpoint{4.412834in}{5.272525in}}{\pgfqpoint{4.416106in}{5.280425in}}{\pgfqpoint{4.416106in}{5.288662in}}%
\pgfpathcurveto{\pgfqpoint{4.416106in}{5.296898in}}{\pgfqpoint{4.412834in}{5.304798in}}{\pgfqpoint{4.407010in}{5.310622in}}%
\pgfpathcurveto{\pgfqpoint{4.401186in}{5.316446in}}{\pgfqpoint{4.393286in}{5.319718in}}{\pgfqpoint{4.385050in}{5.319718in}}%
\pgfpathcurveto{\pgfqpoint{4.376814in}{5.319718in}}{\pgfqpoint{4.368914in}{5.316446in}}{\pgfqpoint{4.363090in}{5.310622in}}%
\pgfpathcurveto{\pgfqpoint{4.357266in}{5.304798in}}{\pgfqpoint{4.353994in}{5.296898in}}{\pgfqpoint{4.353994in}{5.288662in}}%
\pgfpathcurveto{\pgfqpoint{4.353994in}{5.280425in}}{\pgfqpoint{4.357266in}{5.272525in}}{\pgfqpoint{4.363090in}{5.266701in}}%
\pgfpathcurveto{\pgfqpoint{4.368914in}{5.260877in}}{\pgfqpoint{4.376814in}{5.257605in}}{\pgfqpoint{4.385050in}{5.257605in}}%
\pgfpathclose%
\pgfusepath{stroke,fill}%
\end{pgfscope}%
\begin{pgfscope}%
\pgfpathrectangle{\pgfqpoint{0.894063in}{3.540000in}}{\pgfqpoint{6.713438in}{2.060556in}} %
\pgfusepath{clip}%
\pgfsetbuttcap%
\pgfsetroundjoin%
\definecolor{currentfill}{rgb}{0.000000,0.500000,0.000000}%
\pgfsetfillcolor{currentfill}%
\pgfsetlinewidth{1.003750pt}%
\definecolor{currentstroke}{rgb}{0.000000,0.500000,0.000000}%
\pgfsetstrokecolor{currentstroke}%
\pgfsetdash{}{0pt}%
\pgfpathmoveto{\pgfqpoint{6.533350in}{5.257530in}}%
\pgfpathcurveto{\pgfqpoint{6.541586in}{5.257530in}}{\pgfqpoint{6.549486in}{5.260802in}}{\pgfqpoint{6.555310in}{5.266626in}}%
\pgfpathcurveto{\pgfqpoint{6.561134in}{5.272450in}}{\pgfqpoint{6.564406in}{5.280350in}}{\pgfqpoint{6.564406in}{5.288586in}}%
\pgfpathcurveto{\pgfqpoint{6.564406in}{5.296822in}}{\pgfqpoint{6.561134in}{5.304722in}}{\pgfqpoint{6.555310in}{5.310546in}}%
\pgfpathcurveto{\pgfqpoint{6.549486in}{5.316370in}}{\pgfqpoint{6.541586in}{5.319643in}}{\pgfqpoint{6.533350in}{5.319643in}}%
\pgfpathcurveto{\pgfqpoint{6.525114in}{5.319643in}}{\pgfqpoint{6.517214in}{5.316370in}}{\pgfqpoint{6.511390in}{5.310546in}}%
\pgfpathcurveto{\pgfqpoint{6.505566in}{5.304722in}}{\pgfqpoint{6.502294in}{5.296822in}}{\pgfqpoint{6.502294in}{5.288586in}}%
\pgfpathcurveto{\pgfqpoint{6.502294in}{5.280350in}}{\pgfqpoint{6.505566in}{5.272450in}}{\pgfqpoint{6.511390in}{5.266626in}}%
\pgfpathcurveto{\pgfqpoint{6.517214in}{5.260802in}}{\pgfqpoint{6.525114in}{5.257530in}}{\pgfqpoint{6.533350in}{5.257530in}}%
\pgfpathclose%
\pgfusepath{stroke,fill}%
\end{pgfscope}%
\begin{pgfscope}%
\pgfpathrectangle{\pgfqpoint{0.894063in}{3.540000in}}{\pgfqpoint{6.713438in}{2.060556in}} %
\pgfusepath{clip}%
\pgfsetbuttcap%
\pgfsetroundjoin%
\definecolor{currentfill}{rgb}{0.000000,0.500000,0.000000}%
\pgfsetfillcolor{currentfill}%
\pgfsetlinewidth{1.003750pt}%
\definecolor{currentstroke}{rgb}{0.000000,0.500000,0.000000}%
\pgfsetstrokecolor{currentstroke}%
\pgfsetdash{}{0pt}%
\pgfpathmoveto{\pgfqpoint{1.296869in}{5.258204in}}%
\pgfpathcurveto{\pgfqpoint{1.305105in}{5.258204in}}{\pgfqpoint{1.313005in}{5.261476in}}{\pgfqpoint{1.318829in}{5.267300in}}%
\pgfpathcurveto{\pgfqpoint{1.324653in}{5.273124in}}{\pgfqpoint{1.327925in}{5.281024in}}{\pgfqpoint{1.327925in}{5.289261in}}%
\pgfpathcurveto{\pgfqpoint{1.327925in}{5.297497in}}{\pgfqpoint{1.324653in}{5.305397in}}{\pgfqpoint{1.318829in}{5.311221in}}%
\pgfpathcurveto{\pgfqpoint{1.313005in}{5.317045in}}{\pgfqpoint{1.305105in}{5.320317in}}{\pgfqpoint{1.296869in}{5.320317in}}%
\pgfpathcurveto{\pgfqpoint{1.288632in}{5.320317in}}{\pgfqpoint{1.280732in}{5.317045in}}{\pgfqpoint{1.274908in}{5.311221in}}%
\pgfpathcurveto{\pgfqpoint{1.269085in}{5.305397in}}{\pgfqpoint{1.265812in}{5.297497in}}{\pgfqpoint{1.265812in}{5.289261in}}%
\pgfpathcurveto{\pgfqpoint{1.265812in}{5.281024in}}{\pgfqpoint{1.269085in}{5.273124in}}{\pgfqpoint{1.274908in}{5.267300in}}%
\pgfpathcurveto{\pgfqpoint{1.280732in}{5.261476in}}{\pgfqpoint{1.288632in}{5.258204in}}{\pgfqpoint{1.296869in}{5.258204in}}%
\pgfpathclose%
\pgfusepath{stroke,fill}%
\end{pgfscope}%
\begin{pgfscope}%
\pgfpathrectangle{\pgfqpoint{0.894063in}{3.540000in}}{\pgfqpoint{6.713438in}{2.060556in}} %
\pgfusepath{clip}%
\pgfsetbuttcap%
\pgfsetroundjoin%
\definecolor{currentfill}{rgb}{0.000000,0.500000,0.000000}%
\pgfsetfillcolor{currentfill}%
\pgfsetlinewidth{1.003750pt}%
\definecolor{currentstroke}{rgb}{0.000000,0.500000,0.000000}%
\pgfsetstrokecolor{currentstroke}%
\pgfsetdash{}{0pt}%
\pgfpathmoveto{\pgfqpoint{4.519319in}{5.257601in}}%
\pgfpathcurveto{\pgfqpoint{4.527555in}{5.257601in}}{\pgfqpoint{4.535455in}{5.260873in}}{\pgfqpoint{4.541279in}{5.266697in}}%
\pgfpathcurveto{\pgfqpoint{4.547103in}{5.272521in}}{\pgfqpoint{4.550375in}{5.280421in}}{\pgfqpoint{4.550375in}{5.288658in}}%
\pgfpathcurveto{\pgfqpoint{4.550375in}{5.296894in}}{\pgfqpoint{4.547103in}{5.304794in}}{\pgfqpoint{4.541279in}{5.310618in}}%
\pgfpathcurveto{\pgfqpoint{4.535455in}{5.316442in}}{\pgfqpoint{4.527555in}{5.319714in}}{\pgfqpoint{4.519319in}{5.319714in}}%
\pgfpathcurveto{\pgfqpoint{4.511082in}{5.319714in}}{\pgfqpoint{4.503182in}{5.316442in}}{\pgfqpoint{4.497358in}{5.310618in}}%
\pgfpathcurveto{\pgfqpoint{4.491535in}{5.304794in}}{\pgfqpoint{4.488262in}{5.296894in}}{\pgfqpoint{4.488262in}{5.288658in}}%
\pgfpathcurveto{\pgfqpoint{4.488262in}{5.280421in}}{\pgfqpoint{4.491535in}{5.272521in}}{\pgfqpoint{4.497358in}{5.266697in}}%
\pgfpathcurveto{\pgfqpoint{4.503182in}{5.260873in}}{\pgfqpoint{4.511082in}{5.257601in}}{\pgfqpoint{4.519319in}{5.257601in}}%
\pgfpathclose%
\pgfusepath{stroke,fill}%
\end{pgfscope}%
\begin{pgfscope}%
\pgfpathrectangle{\pgfqpoint{0.894063in}{3.540000in}}{\pgfqpoint{6.713438in}{2.060556in}} %
\pgfusepath{clip}%
\pgfsetbuttcap%
\pgfsetroundjoin%
\definecolor{currentfill}{rgb}{0.000000,0.500000,0.000000}%
\pgfsetfillcolor{currentfill}%
\pgfsetlinewidth{1.003750pt}%
\definecolor{currentstroke}{rgb}{0.000000,0.500000,0.000000}%
\pgfsetstrokecolor{currentstroke}%
\pgfsetdash{}{0pt}%
\pgfpathmoveto{\pgfqpoint{2.505288in}{5.257839in}}%
\pgfpathcurveto{\pgfqpoint{2.513524in}{5.257839in}}{\pgfqpoint{2.521424in}{5.261111in}}{\pgfqpoint{2.527248in}{5.266935in}}%
\pgfpathcurveto{\pgfqpoint{2.533072in}{5.272759in}}{\pgfqpoint{2.536344in}{5.280659in}}{\pgfqpoint{2.536344in}{5.288895in}}%
\pgfpathcurveto{\pgfqpoint{2.536344in}{5.297131in}}{\pgfqpoint{2.533072in}{5.305031in}}{\pgfqpoint{2.527248in}{5.310855in}}%
\pgfpathcurveto{\pgfqpoint{2.521424in}{5.316679in}}{\pgfqpoint{2.513524in}{5.319952in}}{\pgfqpoint{2.505288in}{5.319952in}}%
\pgfpathcurveto{\pgfqpoint{2.497051in}{5.319952in}}{\pgfqpoint{2.489151in}{5.316679in}}{\pgfqpoint{2.483327in}{5.310855in}}%
\pgfpathcurveto{\pgfqpoint{2.477503in}{5.305031in}}{\pgfqpoint{2.474231in}{5.297131in}}{\pgfqpoint{2.474231in}{5.288895in}}%
\pgfpathcurveto{\pgfqpoint{2.474231in}{5.280659in}}{\pgfqpoint{2.477503in}{5.272759in}}{\pgfqpoint{2.483327in}{5.266935in}}%
\pgfpathcurveto{\pgfqpoint{2.489151in}{5.261111in}}{\pgfqpoint{2.497051in}{5.257839in}}{\pgfqpoint{2.505288in}{5.257839in}}%
\pgfpathclose%
\pgfusepath{stroke,fill}%
\end{pgfscope}%
\begin{pgfscope}%
\pgfpathrectangle{\pgfqpoint{0.894063in}{3.540000in}}{\pgfqpoint{6.713438in}{2.060556in}} %
\pgfusepath{clip}%
\pgfsetbuttcap%
\pgfsetroundjoin%
\definecolor{currentfill}{rgb}{0.000000,0.500000,0.000000}%
\pgfsetfillcolor{currentfill}%
\pgfsetlinewidth{1.003750pt}%
\definecolor{currentstroke}{rgb}{0.000000,0.500000,0.000000}%
\pgfsetstrokecolor{currentstroke}%
\pgfsetdash{}{0pt}%
\pgfpathmoveto{\pgfqpoint{5.459200in}{5.257545in}}%
\pgfpathcurveto{\pgfqpoint{5.467436in}{5.257545in}}{\pgfqpoint{5.475336in}{5.260817in}}{\pgfqpoint{5.481160in}{5.266641in}}%
\pgfpathcurveto{\pgfqpoint{5.486984in}{5.272465in}}{\pgfqpoint{5.490256in}{5.280365in}}{\pgfqpoint{5.490256in}{5.288601in}}%
\pgfpathcurveto{\pgfqpoint{5.490256in}{5.296837in}}{\pgfqpoint{5.486984in}{5.304738in}}{\pgfqpoint{5.481160in}{5.310561in}}%
\pgfpathcurveto{\pgfqpoint{5.475336in}{5.316385in}}{\pgfqpoint{5.467436in}{5.319658in}}{\pgfqpoint{5.459200in}{5.319658in}}%
\pgfpathcurveto{\pgfqpoint{5.450964in}{5.319658in}}{\pgfqpoint{5.443064in}{5.316385in}}{\pgfqpoint{5.437240in}{5.310561in}}%
\pgfpathcurveto{\pgfqpoint{5.431416in}{5.304738in}}{\pgfqpoint{5.428144in}{5.296837in}}{\pgfqpoint{5.428144in}{5.288601in}}%
\pgfpathcurveto{\pgfqpoint{5.428144in}{5.280365in}}{\pgfqpoint{5.431416in}{5.272465in}}{\pgfqpoint{5.437240in}{5.266641in}}%
\pgfpathcurveto{\pgfqpoint{5.443064in}{5.260817in}}{\pgfqpoint{5.450964in}{5.257545in}}{\pgfqpoint{5.459200in}{5.257545in}}%
\pgfpathclose%
\pgfusepath{stroke,fill}%
\end{pgfscope}%
\begin{pgfscope}%
\pgfpathrectangle{\pgfqpoint{0.894063in}{3.540000in}}{\pgfqpoint{6.713438in}{2.060556in}} %
\pgfusepath{clip}%
\pgfsetbuttcap%
\pgfsetroundjoin%
\definecolor{currentfill}{rgb}{0.000000,0.500000,0.000000}%
\pgfsetfillcolor{currentfill}%
\pgfsetlinewidth{1.003750pt}%
\definecolor{currentstroke}{rgb}{0.000000,0.500000,0.000000}%
\pgfsetstrokecolor{currentstroke}%
\pgfsetdash{}{0pt}%
\pgfpathmoveto{\pgfqpoint{6.936156in}{5.256818in}}%
\pgfpathcurveto{\pgfqpoint{6.944393in}{5.256818in}}{\pgfqpoint{6.952293in}{5.260090in}}{\pgfqpoint{6.958117in}{5.265914in}}%
\pgfpathcurveto{\pgfqpoint{6.963940in}{5.271738in}}{\pgfqpoint{6.967213in}{5.279638in}}{\pgfqpoint{6.967213in}{5.287874in}}%
\pgfpathcurveto{\pgfqpoint{6.967213in}{5.296111in}}{\pgfqpoint{6.963940in}{5.304011in}}{\pgfqpoint{6.958117in}{5.309835in}}%
\pgfpathcurveto{\pgfqpoint{6.952293in}{5.315659in}}{\pgfqpoint{6.944393in}{5.318931in}}{\pgfqpoint{6.936156in}{5.318931in}}%
\pgfpathcurveto{\pgfqpoint{6.927920in}{5.318931in}}{\pgfqpoint{6.920020in}{5.315659in}}{\pgfqpoint{6.914196in}{5.309835in}}%
\pgfpathcurveto{\pgfqpoint{6.908372in}{5.304011in}}{\pgfqpoint{6.905100in}{5.296111in}}{\pgfqpoint{6.905100in}{5.287874in}}%
\pgfpathcurveto{\pgfqpoint{6.905100in}{5.279638in}}{\pgfqpoint{6.908372in}{5.271738in}}{\pgfqpoint{6.914196in}{5.265914in}}%
\pgfpathcurveto{\pgfqpoint{6.920020in}{5.260090in}}{\pgfqpoint{6.927920in}{5.256818in}}{\pgfqpoint{6.936156in}{5.256818in}}%
\pgfpathclose%
\pgfusepath{stroke,fill}%
\end{pgfscope}%
\begin{pgfscope}%
\pgfpathrectangle{\pgfqpoint{0.894063in}{3.540000in}}{\pgfqpoint{6.713438in}{2.060556in}} %
\pgfusepath{clip}%
\pgfsetbuttcap%
\pgfsetroundjoin%
\definecolor{currentfill}{rgb}{0.000000,0.500000,0.000000}%
\pgfsetfillcolor{currentfill}%
\pgfsetlinewidth{1.003750pt}%
\definecolor{currentstroke}{rgb}{0.000000,0.500000,0.000000}%
\pgfsetstrokecolor{currentstroke}%
\pgfsetdash{}{0pt}%
\pgfpathmoveto{\pgfqpoint{5.862006in}{5.257543in}}%
\pgfpathcurveto{\pgfqpoint{5.870243in}{5.257543in}}{\pgfqpoint{5.878143in}{5.260816in}}{\pgfqpoint{5.883967in}{5.266640in}}%
\pgfpathcurveto{\pgfqpoint{5.889790in}{5.272463in}}{\pgfqpoint{5.893063in}{5.280364in}}{\pgfqpoint{5.893063in}{5.288600in}}%
\pgfpathcurveto{\pgfqpoint{5.893063in}{5.296836in}}{\pgfqpoint{5.889790in}{5.304736in}}{\pgfqpoint{5.883967in}{5.310560in}}%
\pgfpathcurveto{\pgfqpoint{5.878143in}{5.316384in}}{\pgfqpoint{5.870243in}{5.319656in}}{\pgfqpoint{5.862006in}{5.319656in}}%
\pgfpathcurveto{\pgfqpoint{5.853770in}{5.319656in}}{\pgfqpoint{5.845870in}{5.316384in}}{\pgfqpoint{5.840046in}{5.310560in}}%
\pgfpathcurveto{\pgfqpoint{5.834222in}{5.304736in}}{\pgfqpoint{5.830950in}{5.296836in}}{\pgfqpoint{5.830950in}{5.288600in}}%
\pgfpathcurveto{\pgfqpoint{5.830950in}{5.280364in}}{\pgfqpoint{5.834222in}{5.272463in}}{\pgfqpoint{5.840046in}{5.266640in}}%
\pgfpathcurveto{\pgfqpoint{5.845870in}{5.260816in}}{\pgfqpoint{5.853770in}{5.257543in}}{\pgfqpoint{5.862006in}{5.257543in}}%
\pgfpathclose%
\pgfusepath{stroke,fill}%
\end{pgfscope}%
\begin{pgfscope}%
\pgfpathrectangle{\pgfqpoint{0.894063in}{3.540000in}}{\pgfqpoint{6.713438in}{2.060556in}} %
\pgfusepath{clip}%
\pgfsetbuttcap%
\pgfsetroundjoin%
\definecolor{currentfill}{rgb}{0.000000,0.500000,0.000000}%
\pgfsetfillcolor{currentfill}%
\pgfsetlinewidth{1.003750pt}%
\definecolor{currentstroke}{rgb}{0.000000,0.500000,0.000000}%
\pgfsetstrokecolor{currentstroke}%
\pgfsetdash{}{0pt}%
\pgfpathmoveto{\pgfqpoint{7.070425in}{5.256795in}}%
\pgfpathcurveto{\pgfqpoint{7.078661in}{5.256795in}}{\pgfqpoint{7.086561in}{5.260067in}}{\pgfqpoint{7.092385in}{5.265891in}}%
\pgfpathcurveto{\pgfqpoint{7.098209in}{5.271715in}}{\pgfqpoint{7.101481in}{5.279615in}}{\pgfqpoint{7.101481in}{5.287851in}}%
\pgfpathcurveto{\pgfqpoint{7.101481in}{5.296087in}}{\pgfqpoint{7.098209in}{5.303987in}}{\pgfqpoint{7.092385in}{5.309811in}}%
\pgfpathcurveto{\pgfqpoint{7.086561in}{5.315635in}}{\pgfqpoint{7.078661in}{5.318908in}}{\pgfqpoint{7.070425in}{5.318908in}}%
\pgfpathcurveto{\pgfqpoint{7.062189in}{5.318908in}}{\pgfqpoint{7.054289in}{5.315635in}}{\pgfqpoint{7.048465in}{5.309811in}}%
\pgfpathcurveto{\pgfqpoint{7.042641in}{5.303987in}}{\pgfqpoint{7.039369in}{5.296087in}}{\pgfqpoint{7.039369in}{5.287851in}}%
\pgfpathcurveto{\pgfqpoint{7.039369in}{5.279615in}}{\pgfqpoint{7.042641in}{5.271715in}}{\pgfqpoint{7.048465in}{5.265891in}}%
\pgfpathcurveto{\pgfqpoint{7.054289in}{5.260067in}}{\pgfqpoint{7.062189in}{5.256795in}}{\pgfqpoint{7.070425in}{5.256795in}}%
\pgfpathclose%
\pgfusepath{stroke,fill}%
\end{pgfscope}%
\begin{pgfscope}%
\pgfpathrectangle{\pgfqpoint{0.894063in}{3.540000in}}{\pgfqpoint{6.713438in}{2.060556in}} %
\pgfusepath{clip}%
\pgfsetbuttcap%
\pgfsetroundjoin%
\definecolor{currentfill}{rgb}{0.000000,0.500000,0.000000}%
\pgfsetfillcolor{currentfill}%
\pgfsetlinewidth{1.003750pt}%
\definecolor{currentstroke}{rgb}{0.000000,0.500000,0.000000}%
\pgfsetstrokecolor{currentstroke}%
\pgfsetdash{}{0pt}%
\pgfpathmoveto{\pgfqpoint{3.176631in}{5.257690in}}%
\pgfpathcurveto{\pgfqpoint{3.184868in}{5.257690in}}{\pgfqpoint{3.192768in}{5.260963in}}{\pgfqpoint{3.198592in}{5.266787in}}%
\pgfpathcurveto{\pgfqpoint{3.204415in}{5.272610in}}{\pgfqpoint{3.207688in}{5.280511in}}{\pgfqpoint{3.207688in}{5.288747in}}%
\pgfpathcurveto{\pgfqpoint{3.207688in}{5.296983in}}{\pgfqpoint{3.204415in}{5.304883in}}{\pgfqpoint{3.198592in}{5.310707in}}%
\pgfpathcurveto{\pgfqpoint{3.192768in}{5.316531in}}{\pgfqpoint{3.184868in}{5.319803in}}{\pgfqpoint{3.176631in}{5.319803in}}%
\pgfpathcurveto{\pgfqpoint{3.168395in}{5.319803in}}{\pgfqpoint{3.160495in}{5.316531in}}{\pgfqpoint{3.154671in}{5.310707in}}%
\pgfpathcurveto{\pgfqpoint{3.148847in}{5.304883in}}{\pgfqpoint{3.145575in}{5.296983in}}{\pgfqpoint{3.145575in}{5.288747in}}%
\pgfpathcurveto{\pgfqpoint{3.145575in}{5.280511in}}{\pgfqpoint{3.148847in}{5.272610in}}{\pgfqpoint{3.154671in}{5.266787in}}%
\pgfpathcurveto{\pgfqpoint{3.160495in}{5.260963in}}{\pgfqpoint{3.168395in}{5.257690in}}{\pgfqpoint{3.176631in}{5.257690in}}%
\pgfpathclose%
\pgfusepath{stroke,fill}%
\end{pgfscope}%
\begin{pgfscope}%
\pgfpathrectangle{\pgfqpoint{0.894063in}{3.540000in}}{\pgfqpoint{6.713438in}{2.060556in}} %
\pgfusepath{clip}%
\pgfsetbuttcap%
\pgfsetroundjoin%
\definecolor{currentfill}{rgb}{0.000000,0.500000,0.000000}%
\pgfsetfillcolor{currentfill}%
\pgfsetlinewidth{1.003750pt}%
\definecolor{currentstroke}{rgb}{0.000000,0.500000,0.000000}%
\pgfsetstrokecolor{currentstroke}%
\pgfsetdash{}{0pt}%
\pgfpathmoveto{\pgfqpoint{2.102481in}{5.258142in}}%
\pgfpathcurveto{\pgfqpoint{2.110718in}{5.258142in}}{\pgfqpoint{2.118618in}{5.261415in}}{\pgfqpoint{2.124442in}{5.267238in}}%
\pgfpathcurveto{\pgfqpoint{2.130265in}{5.273062in}}{\pgfqpoint{2.133538in}{5.280962in}}{\pgfqpoint{2.133538in}{5.289199in}}%
\pgfpathcurveto{\pgfqpoint{2.133538in}{5.297435in}}{\pgfqpoint{2.130265in}{5.305335in}}{\pgfqpoint{2.124442in}{5.311159in}}%
\pgfpathcurveto{\pgfqpoint{2.118618in}{5.316983in}}{\pgfqpoint{2.110718in}{5.320255in}}{\pgfqpoint{2.102481in}{5.320255in}}%
\pgfpathcurveto{\pgfqpoint{2.094245in}{5.320255in}}{\pgfqpoint{2.086345in}{5.316983in}}{\pgfqpoint{2.080521in}{5.311159in}}%
\pgfpathcurveto{\pgfqpoint{2.074697in}{5.305335in}}{\pgfqpoint{2.071425in}{5.297435in}}{\pgfqpoint{2.071425in}{5.289199in}}%
\pgfpathcurveto{\pgfqpoint{2.071425in}{5.280962in}}{\pgfqpoint{2.074697in}{5.273062in}}{\pgfqpoint{2.080521in}{5.267238in}}%
\pgfpathcurveto{\pgfqpoint{2.086345in}{5.261415in}}{\pgfqpoint{2.094245in}{5.258142in}}{\pgfqpoint{2.102481in}{5.258142in}}%
\pgfpathclose%
\pgfusepath{stroke,fill}%
\end{pgfscope}%
\begin{pgfscope}%
\pgfpathrectangle{\pgfqpoint{0.894063in}{3.540000in}}{\pgfqpoint{6.713438in}{2.060556in}} %
\pgfusepath{clip}%
\pgfsetbuttcap%
\pgfsetroundjoin%
\definecolor{currentfill}{rgb}{0.000000,0.500000,0.000000}%
\pgfsetfillcolor{currentfill}%
\pgfsetlinewidth{1.003750pt}%
\definecolor{currentstroke}{rgb}{0.000000,0.500000,0.000000}%
\pgfsetstrokecolor{currentstroke}%
\pgfsetdash{}{0pt}%
\pgfpathmoveto{\pgfqpoint{1.968213in}{5.258159in}}%
\pgfpathcurveto{\pgfqpoint{1.976449in}{5.258159in}}{\pgfqpoint{1.984349in}{5.261431in}}{\pgfqpoint{1.990173in}{5.267255in}}%
\pgfpathcurveto{\pgfqpoint{1.995997in}{5.273079in}}{\pgfqpoint{1.999269in}{5.280979in}}{\pgfqpoint{1.999269in}{5.289215in}}%
\pgfpathcurveto{\pgfqpoint{1.999269in}{5.297452in}}{\pgfqpoint{1.995997in}{5.305352in}}{\pgfqpoint{1.990173in}{5.311175in}}%
\pgfpathcurveto{\pgfqpoint{1.984349in}{5.316999in}}{\pgfqpoint{1.976449in}{5.320272in}}{\pgfqpoint{1.968213in}{5.320272in}}%
\pgfpathcurveto{\pgfqpoint{1.959976in}{5.320272in}}{\pgfqpoint{1.952076in}{5.316999in}}{\pgfqpoint{1.946252in}{5.311175in}}%
\pgfpathcurveto{\pgfqpoint{1.940428in}{5.305352in}}{\pgfqpoint{1.937156in}{5.297452in}}{\pgfqpoint{1.937156in}{5.289215in}}%
\pgfpathcurveto{\pgfqpoint{1.937156in}{5.280979in}}{\pgfqpoint{1.940428in}{5.273079in}}{\pgfqpoint{1.946252in}{5.267255in}}%
\pgfpathcurveto{\pgfqpoint{1.952076in}{5.261431in}}{\pgfqpoint{1.959976in}{5.258159in}}{\pgfqpoint{1.968213in}{5.258159in}}%
\pgfpathclose%
\pgfusepath{stroke,fill}%
\end{pgfscope}%
\begin{pgfscope}%
\pgfpathrectangle{\pgfqpoint{0.894063in}{3.540000in}}{\pgfqpoint{6.713438in}{2.060556in}} %
\pgfusepath{clip}%
\pgfsetbuttcap%
\pgfsetroundjoin%
\definecolor{currentfill}{rgb}{0.000000,0.500000,0.000000}%
\pgfsetfillcolor{currentfill}%
\pgfsetlinewidth{1.003750pt}%
\definecolor{currentstroke}{rgb}{0.000000,0.500000,0.000000}%
\pgfsetstrokecolor{currentstroke}%
\pgfsetdash{}{0pt}%
\pgfpathmoveto{\pgfqpoint{3.310900in}{5.257688in}}%
\pgfpathcurveto{\pgfqpoint{3.319136in}{5.257688in}}{\pgfqpoint{3.327036in}{5.260960in}}{\pgfqpoint{3.332860in}{5.266784in}}%
\pgfpathcurveto{\pgfqpoint{3.338684in}{5.272608in}}{\pgfqpoint{3.341956in}{5.280508in}}{\pgfqpoint{3.341956in}{5.288744in}}%
\pgfpathcurveto{\pgfqpoint{3.341956in}{5.296980in}}{\pgfqpoint{3.338684in}{5.304880in}}{\pgfqpoint{3.332860in}{5.310704in}}%
\pgfpathcurveto{\pgfqpoint{3.327036in}{5.316528in}}{\pgfqpoint{3.319136in}{5.319801in}}{\pgfqpoint{3.310900in}{5.319801in}}%
\pgfpathcurveto{\pgfqpoint{3.302664in}{5.319801in}}{\pgfqpoint{3.294764in}{5.316528in}}{\pgfqpoint{3.288940in}{5.310704in}}%
\pgfpathcurveto{\pgfqpoint{3.283116in}{5.304880in}}{\pgfqpoint{3.279844in}{5.296980in}}{\pgfqpoint{3.279844in}{5.288744in}}%
\pgfpathcurveto{\pgfqpoint{3.279844in}{5.280508in}}{\pgfqpoint{3.283116in}{5.272608in}}{\pgfqpoint{3.288940in}{5.266784in}}%
\pgfpathcurveto{\pgfqpoint{3.294764in}{5.260960in}}{\pgfqpoint{3.302664in}{5.257688in}}{\pgfqpoint{3.310900in}{5.257688in}}%
\pgfpathclose%
\pgfusepath{stroke,fill}%
\end{pgfscope}%
\begin{pgfscope}%
\pgfpathrectangle{\pgfqpoint{0.894063in}{3.540000in}}{\pgfqpoint{6.713438in}{2.060556in}} %
\pgfusepath{clip}%
\pgfsetbuttcap%
\pgfsetroundjoin%
\definecolor{currentfill}{rgb}{0.000000,0.500000,0.000000}%
\pgfsetfillcolor{currentfill}%
\pgfsetlinewidth{1.003750pt}%
\definecolor{currentstroke}{rgb}{0.000000,0.500000,0.000000}%
\pgfsetstrokecolor{currentstroke}%
\pgfsetdash{}{0pt}%
\pgfpathmoveto{\pgfqpoint{5.593469in}{5.257545in}}%
\pgfpathcurveto{\pgfqpoint{5.601705in}{5.257545in}}{\pgfqpoint{5.609605in}{5.260817in}}{\pgfqpoint{5.615429in}{5.266641in}}%
\pgfpathcurveto{\pgfqpoint{5.621253in}{5.272465in}}{\pgfqpoint{5.624525in}{5.280365in}}{\pgfqpoint{5.624525in}{5.288601in}}%
\pgfpathcurveto{\pgfqpoint{5.624525in}{5.296837in}}{\pgfqpoint{5.621253in}{5.304738in}}{\pgfqpoint{5.615429in}{5.310561in}}%
\pgfpathcurveto{\pgfqpoint{5.609605in}{5.316385in}}{\pgfqpoint{5.601705in}{5.319658in}}{\pgfqpoint{5.593469in}{5.319658in}}%
\pgfpathcurveto{\pgfqpoint{5.585232in}{5.319658in}}{\pgfqpoint{5.577332in}{5.316385in}}{\pgfqpoint{5.571508in}{5.310561in}}%
\pgfpathcurveto{\pgfqpoint{5.565685in}{5.304738in}}{\pgfqpoint{5.562412in}{5.296837in}}{\pgfqpoint{5.562412in}{5.288601in}}%
\pgfpathcurveto{\pgfqpoint{5.562412in}{5.280365in}}{\pgfqpoint{5.565685in}{5.272465in}}{\pgfqpoint{5.571508in}{5.266641in}}%
\pgfpathcurveto{\pgfqpoint{5.577332in}{5.260817in}}{\pgfqpoint{5.585232in}{5.257545in}}{\pgfqpoint{5.593469in}{5.257545in}}%
\pgfpathclose%
\pgfusepath{stroke,fill}%
\end{pgfscope}%
\begin{pgfscope}%
\pgfpathrectangle{\pgfqpoint{0.894063in}{3.540000in}}{\pgfqpoint{6.713438in}{2.060556in}} %
\pgfusepath{clip}%
\pgfsetbuttcap%
\pgfsetroundjoin%
\definecolor{currentfill}{rgb}{0.000000,0.500000,0.000000}%
\pgfsetfillcolor{currentfill}%
\pgfsetlinewidth{1.003750pt}%
\definecolor{currentstroke}{rgb}{0.000000,0.500000,0.000000}%
\pgfsetstrokecolor{currentstroke}%
\pgfsetdash{}{0pt}%
\pgfpathmoveto{\pgfqpoint{3.042363in}{5.257693in}}%
\pgfpathcurveto{\pgfqpoint{3.050599in}{5.257693in}}{\pgfqpoint{3.058499in}{5.260965in}}{\pgfqpoint{3.064323in}{5.266789in}}%
\pgfpathcurveto{\pgfqpoint{3.070147in}{5.272613in}}{\pgfqpoint{3.073419in}{5.280513in}}{\pgfqpoint{3.073419in}{5.288750in}}%
\pgfpathcurveto{\pgfqpoint{3.073419in}{5.296986in}}{\pgfqpoint{3.070147in}{5.304886in}}{\pgfqpoint{3.064323in}{5.310710in}}%
\pgfpathcurveto{\pgfqpoint{3.058499in}{5.316534in}}{\pgfqpoint{3.050599in}{5.319806in}}{\pgfqpoint{3.042363in}{5.319806in}}%
\pgfpathcurveto{\pgfqpoint{3.034126in}{5.319806in}}{\pgfqpoint{3.026226in}{5.316534in}}{\pgfqpoint{3.020402in}{5.310710in}}%
\pgfpathcurveto{\pgfqpoint{3.014578in}{5.304886in}}{\pgfqpoint{3.011306in}{5.296986in}}{\pgfqpoint{3.011306in}{5.288750in}}%
\pgfpathcurveto{\pgfqpoint{3.011306in}{5.280513in}}{\pgfqpoint{3.014578in}{5.272613in}}{\pgfqpoint{3.020402in}{5.266789in}}%
\pgfpathcurveto{\pgfqpoint{3.026226in}{5.260965in}}{\pgfqpoint{3.034126in}{5.257693in}}{\pgfqpoint{3.042363in}{5.257693in}}%
\pgfpathclose%
\pgfusepath{stroke,fill}%
\end{pgfscope}%
\begin{pgfscope}%
\pgfpathrectangle{\pgfqpoint{0.894063in}{3.540000in}}{\pgfqpoint{6.713438in}{2.060556in}} %
\pgfusepath{clip}%
\pgfsetbuttcap%
\pgfsetroundjoin%
\definecolor{currentfill}{rgb}{0.000000,0.500000,0.000000}%
\pgfsetfillcolor{currentfill}%
\pgfsetlinewidth{1.003750pt}%
\definecolor{currentstroke}{rgb}{0.000000,0.500000,0.000000}%
\pgfsetstrokecolor{currentstroke}%
\pgfsetdash{}{0pt}%
\pgfpathmoveto{\pgfqpoint{5.190663in}{5.257556in}}%
\pgfpathcurveto{\pgfqpoint{5.198899in}{5.257556in}}{\pgfqpoint{5.206799in}{5.260828in}}{\pgfqpoint{5.212623in}{5.266652in}}%
\pgfpathcurveto{\pgfqpoint{5.218447in}{5.272476in}}{\pgfqpoint{5.221719in}{5.280376in}}{\pgfqpoint{5.221719in}{5.288612in}}%
\pgfpathcurveto{\pgfqpoint{5.221719in}{5.296848in}}{\pgfqpoint{5.218447in}{5.304749in}}{\pgfqpoint{5.212623in}{5.310572in}}%
\pgfpathcurveto{\pgfqpoint{5.206799in}{5.316396in}}{\pgfqpoint{5.198899in}{5.319669in}}{\pgfqpoint{5.190663in}{5.319669in}}%
\pgfpathcurveto{\pgfqpoint{5.182426in}{5.319669in}}{\pgfqpoint{5.174526in}{5.316396in}}{\pgfqpoint{5.168702in}{5.310572in}}%
\pgfpathcurveto{\pgfqpoint{5.162878in}{5.304749in}}{\pgfqpoint{5.159606in}{5.296848in}}{\pgfqpoint{5.159606in}{5.288612in}}%
\pgfpathcurveto{\pgfqpoint{5.159606in}{5.280376in}}{\pgfqpoint{5.162878in}{5.272476in}}{\pgfqpoint{5.168702in}{5.266652in}}%
\pgfpathcurveto{\pgfqpoint{5.174526in}{5.260828in}}{\pgfqpoint{5.182426in}{5.257556in}}{\pgfqpoint{5.190663in}{5.257556in}}%
\pgfpathclose%
\pgfusepath{stroke,fill}%
\end{pgfscope}%
\begin{pgfscope}%
\pgfpathrectangle{\pgfqpoint{0.894063in}{3.540000in}}{\pgfqpoint{6.713438in}{2.060556in}} %
\pgfusepath{clip}%
\pgfsetbuttcap%
\pgfsetroundjoin%
\definecolor{currentfill}{rgb}{0.000000,0.500000,0.000000}%
\pgfsetfillcolor{currentfill}%
\pgfsetlinewidth{1.003750pt}%
\definecolor{currentstroke}{rgb}{0.000000,0.500000,0.000000}%
\pgfsetstrokecolor{currentstroke}%
\pgfsetdash{}{0pt}%
\pgfpathmoveto{\pgfqpoint{6.801888in}{5.257525in}}%
\pgfpathcurveto{\pgfqpoint{6.810124in}{5.257525in}}{\pgfqpoint{6.818024in}{5.260798in}}{\pgfqpoint{6.823848in}{5.266622in}}%
\pgfpathcurveto{\pgfqpoint{6.829672in}{5.272446in}}{\pgfqpoint{6.832944in}{5.280346in}}{\pgfqpoint{6.832944in}{5.288582in}}%
\pgfpathcurveto{\pgfqpoint{6.832944in}{5.296818in}}{\pgfqpoint{6.829672in}{5.304718in}}{\pgfqpoint{6.823848in}{5.310542in}}%
\pgfpathcurveto{\pgfqpoint{6.818024in}{5.316366in}}{\pgfqpoint{6.810124in}{5.319638in}}{\pgfqpoint{6.801888in}{5.319638in}}%
\pgfpathcurveto{\pgfqpoint{6.793651in}{5.319638in}}{\pgfqpoint{6.785751in}{5.316366in}}{\pgfqpoint{6.779927in}{5.310542in}}%
\pgfpathcurveto{\pgfqpoint{6.774103in}{5.304718in}}{\pgfqpoint{6.770831in}{5.296818in}}{\pgfqpoint{6.770831in}{5.288582in}}%
\pgfpathcurveto{\pgfqpoint{6.770831in}{5.280346in}}{\pgfqpoint{6.774103in}{5.272446in}}{\pgfqpoint{6.779927in}{5.266622in}}%
\pgfpathcurveto{\pgfqpoint{6.785751in}{5.260798in}}{\pgfqpoint{6.793651in}{5.257525in}}{\pgfqpoint{6.801888in}{5.257525in}}%
\pgfpathclose%
\pgfusepath{stroke,fill}%
\end{pgfscope}%
\begin{pgfscope}%
\pgfpathrectangle{\pgfqpoint{0.894063in}{3.540000in}}{\pgfqpoint{6.713438in}{2.060556in}} %
\pgfusepath{clip}%
\pgfsetbuttcap%
\pgfsetroundjoin%
\definecolor{currentfill}{rgb}{0.000000,0.500000,0.000000}%
\pgfsetfillcolor{currentfill}%
\pgfsetlinewidth{1.003750pt}%
\definecolor{currentstroke}{rgb}{0.000000,0.500000,0.000000}%
\pgfsetstrokecolor{currentstroke}%
\pgfsetdash{}{0pt}%
\pgfpathmoveto{\pgfqpoint{3.579438in}{5.257671in}}%
\pgfpathcurveto{\pgfqpoint{3.587674in}{5.257671in}}{\pgfqpoint{3.595574in}{5.260943in}}{\pgfqpoint{3.601398in}{5.266767in}}%
\pgfpathcurveto{\pgfqpoint{3.607222in}{5.272591in}}{\pgfqpoint{3.610494in}{5.280491in}}{\pgfqpoint{3.610494in}{5.288728in}}%
\pgfpathcurveto{\pgfqpoint{3.610494in}{5.296964in}}{\pgfqpoint{3.607222in}{5.304864in}}{\pgfqpoint{3.601398in}{5.310688in}}%
\pgfpathcurveto{\pgfqpoint{3.595574in}{5.316512in}}{\pgfqpoint{3.587674in}{5.319784in}}{\pgfqpoint{3.579438in}{5.319784in}}%
\pgfpathcurveto{\pgfqpoint{3.571201in}{5.319784in}}{\pgfqpoint{3.563301in}{5.316512in}}{\pgfqpoint{3.557477in}{5.310688in}}%
\pgfpathcurveto{\pgfqpoint{3.551653in}{5.304864in}}{\pgfqpoint{3.548381in}{5.296964in}}{\pgfqpoint{3.548381in}{5.288728in}}%
\pgfpathcurveto{\pgfqpoint{3.548381in}{5.280491in}}{\pgfqpoint{3.551653in}{5.272591in}}{\pgfqpoint{3.557477in}{5.266767in}}%
\pgfpathcurveto{\pgfqpoint{3.563301in}{5.260943in}}{\pgfqpoint{3.571201in}{5.257671in}}{\pgfqpoint{3.579438in}{5.257671in}}%
\pgfpathclose%
\pgfusepath{stroke,fill}%
\end{pgfscope}%
\begin{pgfscope}%
\pgfpathrectangle{\pgfqpoint{0.894063in}{3.540000in}}{\pgfqpoint{6.713438in}{2.060556in}} %
\pgfusepath{clip}%
\pgfsetbuttcap%
\pgfsetroundjoin%
\definecolor{currentfill}{rgb}{0.000000,0.500000,0.000000}%
\pgfsetfillcolor{currentfill}%
\pgfsetlinewidth{1.003750pt}%
\definecolor{currentstroke}{rgb}{0.000000,0.500000,0.000000}%
\pgfsetstrokecolor{currentstroke}%
\pgfsetdash{}{0pt}%
\pgfpathmoveto{\pgfqpoint{2.371019in}{5.257841in}}%
\pgfpathcurveto{\pgfqpoint{2.379255in}{5.257841in}}{\pgfqpoint{2.387155in}{5.261114in}}{\pgfqpoint{2.392979in}{5.266938in}}%
\pgfpathcurveto{\pgfqpoint{2.398803in}{5.272762in}}{\pgfqpoint{2.402075in}{5.280662in}}{\pgfqpoint{2.402075in}{5.288898in}}%
\pgfpathcurveto{\pgfqpoint{2.402075in}{5.297134in}}{\pgfqpoint{2.398803in}{5.305034in}}{\pgfqpoint{2.392979in}{5.310858in}}%
\pgfpathcurveto{\pgfqpoint{2.387155in}{5.316682in}}{\pgfqpoint{2.379255in}{5.319954in}}{\pgfqpoint{2.371019in}{5.319954in}}%
\pgfpathcurveto{\pgfqpoint{2.362782in}{5.319954in}}{\pgfqpoint{2.354882in}{5.316682in}}{\pgfqpoint{2.349058in}{5.310858in}}%
\pgfpathcurveto{\pgfqpoint{2.343235in}{5.305034in}}{\pgfqpoint{2.339962in}{5.297134in}}{\pgfqpoint{2.339962in}{5.288898in}}%
\pgfpathcurveto{\pgfqpoint{2.339962in}{5.280662in}}{\pgfqpoint{2.343235in}{5.272762in}}{\pgfqpoint{2.349058in}{5.266938in}}%
\pgfpathcurveto{\pgfqpoint{2.354882in}{5.261114in}}{\pgfqpoint{2.362782in}{5.257841in}}{\pgfqpoint{2.371019in}{5.257841in}}%
\pgfpathclose%
\pgfusepath{stroke,fill}%
\end{pgfscope}%
\begin{pgfscope}%
\pgfpathrectangle{\pgfqpoint{0.894063in}{3.540000in}}{\pgfqpoint{6.713438in}{2.060556in}} %
\pgfusepath{clip}%
\pgfsetbuttcap%
\pgfsetroundjoin%
\definecolor{currentfill}{rgb}{0.000000,0.500000,0.000000}%
\pgfsetfillcolor{currentfill}%
\pgfsetlinewidth{1.003750pt}%
\definecolor{currentstroke}{rgb}{0.000000,0.500000,0.000000}%
\pgfsetstrokecolor{currentstroke}%
\pgfsetdash{}{0pt}%
\pgfpathmoveto{\pgfqpoint{3.982244in}{5.257623in}}%
\pgfpathcurveto{\pgfqpoint{3.990480in}{5.257623in}}{\pgfqpoint{3.998380in}{5.260895in}}{\pgfqpoint{4.004204in}{5.266719in}}%
\pgfpathcurveto{\pgfqpoint{4.010028in}{5.272543in}}{\pgfqpoint{4.013300in}{5.280443in}}{\pgfqpoint{4.013300in}{5.288679in}}%
\pgfpathcurveto{\pgfqpoint{4.013300in}{5.296916in}}{\pgfqpoint{4.010028in}{5.304816in}}{\pgfqpoint{4.004204in}{5.310640in}}%
\pgfpathcurveto{\pgfqpoint{3.998380in}{5.316464in}}{\pgfqpoint{3.990480in}{5.319736in}}{\pgfqpoint{3.982244in}{5.319736in}}%
\pgfpathcurveto{\pgfqpoint{3.974007in}{5.319736in}}{\pgfqpoint{3.966107in}{5.316464in}}{\pgfqpoint{3.960283in}{5.310640in}}%
\pgfpathcurveto{\pgfqpoint{3.954460in}{5.304816in}}{\pgfqpoint{3.951187in}{5.296916in}}{\pgfqpoint{3.951187in}{5.288679in}}%
\pgfpathcurveto{\pgfqpoint{3.951187in}{5.280443in}}{\pgfqpoint{3.954460in}{5.272543in}}{\pgfqpoint{3.960283in}{5.266719in}}%
\pgfpathcurveto{\pgfqpoint{3.966107in}{5.260895in}}{\pgfqpoint{3.974007in}{5.257623in}}{\pgfqpoint{3.982244in}{5.257623in}}%
\pgfpathclose%
\pgfusepath{stroke,fill}%
\end{pgfscope}%
\begin{pgfscope}%
\pgfpathrectangle{\pgfqpoint{0.894063in}{3.540000in}}{\pgfqpoint{6.713438in}{2.060556in}} %
\pgfusepath{clip}%
\pgfsetbuttcap%
\pgfsetroundjoin%
\definecolor{currentfill}{rgb}{0.000000,0.500000,0.000000}%
\pgfsetfillcolor{currentfill}%
\pgfsetlinewidth{1.003750pt}%
\definecolor{currentstroke}{rgb}{0.000000,0.500000,0.000000}%
\pgfsetstrokecolor{currentstroke}%
\pgfsetdash{}{0pt}%
\pgfpathmoveto{\pgfqpoint{4.653588in}{5.257576in}}%
\pgfpathcurveto{\pgfqpoint{4.661824in}{5.257576in}}{\pgfqpoint{4.669724in}{5.260849in}}{\pgfqpoint{4.675548in}{5.266673in}}%
\pgfpathcurveto{\pgfqpoint{4.681372in}{5.272496in}}{\pgfqpoint{4.684644in}{5.280396in}}{\pgfqpoint{4.684644in}{5.288633in}}%
\pgfpathcurveto{\pgfqpoint{4.684644in}{5.296869in}}{\pgfqpoint{4.681372in}{5.304769in}}{\pgfqpoint{4.675548in}{5.310593in}}%
\pgfpathcurveto{\pgfqpoint{4.669724in}{5.316417in}}{\pgfqpoint{4.661824in}{5.319689in}}{\pgfqpoint{4.653588in}{5.319689in}}%
\pgfpathcurveto{\pgfqpoint{4.645351in}{5.319689in}}{\pgfqpoint{4.637451in}{5.316417in}}{\pgfqpoint{4.631627in}{5.310593in}}%
\pgfpathcurveto{\pgfqpoint{4.625803in}{5.304769in}}{\pgfqpoint{4.622531in}{5.296869in}}{\pgfqpoint{4.622531in}{5.288633in}}%
\pgfpathcurveto{\pgfqpoint{4.622531in}{5.280396in}}{\pgfqpoint{4.625803in}{5.272496in}}{\pgfqpoint{4.631627in}{5.266673in}}%
\pgfpathcurveto{\pgfqpoint{4.637451in}{5.260849in}}{\pgfqpoint{4.645351in}{5.257576in}}{\pgfqpoint{4.653588in}{5.257576in}}%
\pgfpathclose%
\pgfusepath{stroke,fill}%
\end{pgfscope}%
\begin{pgfscope}%
\pgfpathrectangle{\pgfqpoint{0.894063in}{3.540000in}}{\pgfqpoint{6.713438in}{2.060556in}} %
\pgfusepath{clip}%
\pgfsetbuttcap%
\pgfsetroundjoin%
\definecolor{currentfill}{rgb}{0.000000,0.500000,0.000000}%
\pgfsetfillcolor{currentfill}%
\pgfsetlinewidth{1.003750pt}%
\definecolor{currentstroke}{rgb}{0.000000,0.500000,0.000000}%
\pgfsetstrokecolor{currentstroke}%
\pgfsetdash{}{0pt}%
\pgfpathmoveto{\pgfqpoint{3.713706in}{5.257666in}}%
\pgfpathcurveto{\pgfqpoint{3.721943in}{5.257666in}}{\pgfqpoint{3.729843in}{5.260938in}}{\pgfqpoint{3.735667in}{5.266762in}}%
\pgfpathcurveto{\pgfqpoint{3.741490in}{5.272586in}}{\pgfqpoint{3.744763in}{5.280486in}}{\pgfqpoint{3.744763in}{5.288722in}}%
\pgfpathcurveto{\pgfqpoint{3.744763in}{5.296958in}}{\pgfqpoint{3.741490in}{5.304858in}}{\pgfqpoint{3.735667in}{5.310682in}}%
\pgfpathcurveto{\pgfqpoint{3.729843in}{5.316506in}}{\pgfqpoint{3.721943in}{5.319779in}}{\pgfqpoint{3.713706in}{5.319779in}}%
\pgfpathcurveto{\pgfqpoint{3.705470in}{5.319779in}}{\pgfqpoint{3.697570in}{5.316506in}}{\pgfqpoint{3.691746in}{5.310682in}}%
\pgfpathcurveto{\pgfqpoint{3.685922in}{5.304858in}}{\pgfqpoint{3.682650in}{5.296958in}}{\pgfqpoint{3.682650in}{5.288722in}}%
\pgfpathcurveto{\pgfqpoint{3.682650in}{5.280486in}}{\pgfqpoint{3.685922in}{5.272586in}}{\pgfqpoint{3.691746in}{5.266762in}}%
\pgfpathcurveto{\pgfqpoint{3.697570in}{5.260938in}}{\pgfqpoint{3.705470in}{5.257666in}}{\pgfqpoint{3.713706in}{5.257666in}}%
\pgfpathclose%
\pgfusepath{stroke,fill}%
\end{pgfscope}%
\begin{pgfscope}%
\pgfpathrectangle{\pgfqpoint{0.894063in}{3.540000in}}{\pgfqpoint{6.713438in}{2.060556in}} %
\pgfusepath{clip}%
\pgfsetbuttcap%
\pgfsetroundjoin%
\definecolor{currentfill}{rgb}{0.000000,0.500000,0.000000}%
\pgfsetfillcolor{currentfill}%
\pgfsetlinewidth{1.003750pt}%
\definecolor{currentstroke}{rgb}{0.000000,0.500000,0.000000}%
\pgfsetstrokecolor{currentstroke}%
\pgfsetdash{}{0pt}%
\pgfpathmoveto{\pgfqpoint{2.236750in}{5.257847in}}%
\pgfpathcurveto{\pgfqpoint{2.244986in}{5.257847in}}{\pgfqpoint{2.252886in}{5.261119in}}{\pgfqpoint{2.258710in}{5.266943in}}%
\pgfpathcurveto{\pgfqpoint{2.264534in}{5.272767in}}{\pgfqpoint{2.267806in}{5.280667in}}{\pgfqpoint{2.267806in}{5.288903in}}%
\pgfpathcurveto{\pgfqpoint{2.267806in}{5.297140in}}{\pgfqpoint{2.264534in}{5.305040in}}{\pgfqpoint{2.258710in}{5.310864in}}%
\pgfpathcurveto{\pgfqpoint{2.252886in}{5.316688in}}{\pgfqpoint{2.244986in}{5.319960in}}{\pgfqpoint{2.236750in}{5.319960in}}%
\pgfpathcurveto{\pgfqpoint{2.228514in}{5.319960in}}{\pgfqpoint{2.220614in}{5.316688in}}{\pgfqpoint{2.214790in}{5.310864in}}%
\pgfpathcurveto{\pgfqpoint{2.208966in}{5.305040in}}{\pgfqpoint{2.205694in}{5.297140in}}{\pgfqpoint{2.205694in}{5.288903in}}%
\pgfpathcurveto{\pgfqpoint{2.205694in}{5.280667in}}{\pgfqpoint{2.208966in}{5.272767in}}{\pgfqpoint{2.214790in}{5.266943in}}%
\pgfpathcurveto{\pgfqpoint{2.220614in}{5.261119in}}{\pgfqpoint{2.228514in}{5.257847in}}{\pgfqpoint{2.236750in}{5.257847in}}%
\pgfpathclose%
\pgfusepath{stroke,fill}%
\end{pgfscope}%
\begin{pgfscope}%
\pgfpathrectangle{\pgfqpoint{0.894063in}{3.540000in}}{\pgfqpoint{6.713438in}{2.060556in}} %
\pgfusepath{clip}%
\pgfsetbuttcap%
\pgfsetroundjoin%
\definecolor{currentfill}{rgb}{0.000000,0.000000,0.000000}%
\pgfsetfillcolor{currentfill}%
\pgfsetlinewidth{1.003750pt}%
\definecolor{currentstroke}{rgb}{0.000000,0.000000,0.000000}%
\pgfsetstrokecolor{currentstroke}%
\pgfsetdash{}{0pt}%
\pgfpathmoveto{\pgfqpoint{6.667619in}{5.257530in}}%
\pgfpathcurveto{\pgfqpoint{6.675855in}{5.257530in}}{\pgfqpoint{6.683755in}{5.260802in}}{\pgfqpoint{6.689579in}{5.266626in}}%
\pgfpathcurveto{\pgfqpoint{6.695403in}{5.272450in}}{\pgfqpoint{6.698675in}{5.280350in}}{\pgfqpoint{6.698675in}{5.288586in}}%
\pgfpathcurveto{\pgfqpoint{6.698675in}{5.296822in}}{\pgfqpoint{6.695403in}{5.304722in}}{\pgfqpoint{6.689579in}{5.310546in}}%
\pgfpathcurveto{\pgfqpoint{6.683755in}{5.316370in}}{\pgfqpoint{6.675855in}{5.319643in}}{\pgfqpoint{6.667619in}{5.319643in}}%
\pgfpathcurveto{\pgfqpoint{6.659382in}{5.319643in}}{\pgfqpoint{6.651482in}{5.316370in}}{\pgfqpoint{6.645658in}{5.310546in}}%
\pgfpathcurveto{\pgfqpoint{6.639835in}{5.304722in}}{\pgfqpoint{6.636562in}{5.296822in}}{\pgfqpoint{6.636562in}{5.288586in}}%
\pgfpathcurveto{\pgfqpoint{6.636562in}{5.280350in}}{\pgfqpoint{6.639835in}{5.272450in}}{\pgfqpoint{6.645658in}{5.266626in}}%
\pgfpathcurveto{\pgfqpoint{6.651482in}{5.260802in}}{\pgfqpoint{6.659382in}{5.257530in}}{\pgfqpoint{6.667619in}{5.257530in}}%
\pgfpathclose%
\pgfusepath{stroke,fill}%
\end{pgfscope}%
\begin{pgfscope}%
\pgfpathrectangle{\pgfqpoint{0.894063in}{3.540000in}}{\pgfqpoint{6.713438in}{2.060556in}} %
\pgfusepath{clip}%
\pgfsetbuttcap%
\pgfsetroundjoin%
\definecolor{currentfill}{rgb}{0.000000,0.000000,0.000000}%
\pgfsetfillcolor{currentfill}%
\pgfsetlinewidth{1.003750pt}%
\definecolor{currentstroke}{rgb}{0.000000,0.000000,0.000000}%
\pgfsetstrokecolor{currentstroke}%
\pgfsetdash{}{0pt}%
\pgfpathmoveto{\pgfqpoint{2.639556in}{5.257710in}}%
\pgfpathcurveto{\pgfqpoint{2.647793in}{5.257710in}}{\pgfqpoint{2.655693in}{5.260982in}}{\pgfqpoint{2.661517in}{5.266806in}}%
\pgfpathcurveto{\pgfqpoint{2.667340in}{5.272630in}}{\pgfqpoint{2.670613in}{5.280530in}}{\pgfqpoint{2.670613in}{5.288766in}}%
\pgfpathcurveto{\pgfqpoint{2.670613in}{5.297002in}}{\pgfqpoint{2.667340in}{5.304902in}}{\pgfqpoint{2.661517in}{5.310726in}}%
\pgfpathcurveto{\pgfqpoint{2.655693in}{5.316550in}}{\pgfqpoint{2.647793in}{5.319823in}}{\pgfqpoint{2.639556in}{5.319823in}}%
\pgfpathcurveto{\pgfqpoint{2.631320in}{5.319823in}}{\pgfqpoint{2.623420in}{5.316550in}}{\pgfqpoint{2.617596in}{5.310726in}}%
\pgfpathcurveto{\pgfqpoint{2.611772in}{5.304902in}}{\pgfqpoint{2.608500in}{5.297002in}}{\pgfqpoint{2.608500in}{5.288766in}}%
\pgfpathcurveto{\pgfqpoint{2.608500in}{5.280530in}}{\pgfqpoint{2.611772in}{5.272630in}}{\pgfqpoint{2.617596in}{5.266806in}}%
\pgfpathcurveto{\pgfqpoint{2.623420in}{5.260982in}}{\pgfqpoint{2.631320in}{5.257710in}}{\pgfqpoint{2.639556in}{5.257710in}}%
\pgfpathclose%
\pgfusepath{stroke,fill}%
\end{pgfscope}%
\begin{pgfscope}%
\pgfpathrectangle{\pgfqpoint{0.894063in}{3.540000in}}{\pgfqpoint{6.713438in}{2.060556in}} %
\pgfusepath{clip}%
\pgfsetbuttcap%
\pgfsetroundjoin%
\definecolor{currentfill}{rgb}{0.000000,0.000000,0.000000}%
\pgfsetfillcolor{currentfill}%
\pgfsetlinewidth{1.003750pt}%
\definecolor{currentstroke}{rgb}{0.000000,0.000000,0.000000}%
\pgfsetstrokecolor{currentstroke}%
\pgfsetdash{}{0pt}%
\pgfpathmoveto{\pgfqpoint{1.699675in}{5.258175in}}%
\pgfpathcurveto{\pgfqpoint{1.707911in}{5.258175in}}{\pgfqpoint{1.715811in}{5.261448in}}{\pgfqpoint{1.721635in}{5.267271in}}%
\pgfpathcurveto{\pgfqpoint{1.727459in}{5.273095in}}{\pgfqpoint{1.730731in}{5.280995in}}{\pgfqpoint{1.730731in}{5.289232in}}%
\pgfpathcurveto{\pgfqpoint{1.730731in}{5.297468in}}{\pgfqpoint{1.727459in}{5.305368in}}{\pgfqpoint{1.721635in}{5.311192in}}%
\pgfpathcurveto{\pgfqpoint{1.715811in}{5.317016in}}{\pgfqpoint{1.707911in}{5.320288in}}{\pgfqpoint{1.699675in}{5.320288in}}%
\pgfpathcurveto{\pgfqpoint{1.691439in}{5.320288in}}{\pgfqpoint{1.683539in}{5.317016in}}{\pgfqpoint{1.677715in}{5.311192in}}%
\pgfpathcurveto{\pgfqpoint{1.671891in}{5.305368in}}{\pgfqpoint{1.668619in}{5.297468in}}{\pgfqpoint{1.668619in}{5.289232in}}%
\pgfpathcurveto{\pgfqpoint{1.668619in}{5.280995in}}{\pgfqpoint{1.671891in}{5.273095in}}{\pgfqpoint{1.677715in}{5.267271in}}%
\pgfpathcurveto{\pgfqpoint{1.683539in}{5.261448in}}{\pgfqpoint{1.691439in}{5.258175in}}{\pgfqpoint{1.699675in}{5.258175in}}%
\pgfpathclose%
\pgfusepath{stroke,fill}%
\end{pgfscope}%
\begin{pgfscope}%
\pgfpathrectangle{\pgfqpoint{0.894063in}{3.540000in}}{\pgfqpoint{6.713438in}{2.060556in}} %
\pgfusepath{clip}%
\pgfsetbuttcap%
\pgfsetroundjoin%
\definecolor{currentfill}{rgb}{0.000000,0.000000,0.000000}%
\pgfsetfillcolor{currentfill}%
\pgfsetlinewidth{1.003750pt}%
\definecolor{currentstroke}{rgb}{0.000000,0.000000,0.000000}%
\pgfsetstrokecolor{currentstroke}%
\pgfsetdash{}{0pt}%
\pgfpathmoveto{\pgfqpoint{1.162600in}{5.258322in}}%
\pgfpathcurveto{\pgfqpoint{1.170836in}{5.258322in}}{\pgfqpoint{1.178736in}{5.261595in}}{\pgfqpoint{1.184560in}{5.267418in}}%
\pgfpathcurveto{\pgfqpoint{1.190384in}{5.273242in}}{\pgfqpoint{1.193656in}{5.281142in}}{\pgfqpoint{1.193656in}{5.289379in}}%
\pgfpathcurveto{\pgfqpoint{1.193656in}{5.297615in}}{\pgfqpoint{1.190384in}{5.305515in}}{\pgfqpoint{1.184560in}{5.311339in}}%
\pgfpathcurveto{\pgfqpoint{1.178736in}{5.317163in}}{\pgfqpoint{1.170836in}{5.320435in}}{\pgfqpoint{1.162600in}{5.320435in}}%
\pgfpathcurveto{\pgfqpoint{1.154364in}{5.320435in}}{\pgfqpoint{1.146464in}{5.317163in}}{\pgfqpoint{1.140640in}{5.311339in}}%
\pgfpathcurveto{\pgfqpoint{1.134816in}{5.305515in}}{\pgfqpoint{1.131544in}{5.297615in}}{\pgfqpoint{1.131544in}{5.289379in}}%
\pgfpathcurveto{\pgfqpoint{1.131544in}{5.281142in}}{\pgfqpoint{1.134816in}{5.273242in}}{\pgfqpoint{1.140640in}{5.267418in}}%
\pgfpathcurveto{\pgfqpoint{1.146464in}{5.261595in}}{\pgfqpoint{1.154364in}{5.258322in}}{\pgfqpoint{1.162600in}{5.258322in}}%
\pgfpathclose%
\pgfusepath{stroke,fill}%
\end{pgfscope}%
\begin{pgfscope}%
\pgfpathrectangle{\pgfqpoint{0.894063in}{3.540000in}}{\pgfqpoint{6.713438in}{2.060556in}} %
\pgfusepath{clip}%
\pgfsetbuttcap%
\pgfsetroundjoin%
\definecolor{currentfill}{rgb}{0.000000,0.000000,0.000000}%
\pgfsetfillcolor{currentfill}%
\pgfsetlinewidth{1.003750pt}%
\definecolor{currentstroke}{rgb}{0.000000,0.000000,0.000000}%
\pgfsetstrokecolor{currentstroke}%
\pgfsetdash{}{0pt}%
\pgfpathmoveto{\pgfqpoint{1.833944in}{5.258166in}}%
\pgfpathcurveto{\pgfqpoint{1.842180in}{5.258166in}}{\pgfqpoint{1.850080in}{5.261438in}}{\pgfqpoint{1.855904in}{5.267262in}}%
\pgfpathcurveto{\pgfqpoint{1.861728in}{5.273086in}}{\pgfqpoint{1.865000in}{5.280986in}}{\pgfqpoint{1.865000in}{5.289222in}}%
\pgfpathcurveto{\pgfqpoint{1.865000in}{5.297458in}}{\pgfqpoint{1.861728in}{5.305358in}}{\pgfqpoint{1.855904in}{5.311182in}}%
\pgfpathcurveto{\pgfqpoint{1.850080in}{5.317006in}}{\pgfqpoint{1.842180in}{5.320279in}}{\pgfqpoint{1.833944in}{5.320279in}}%
\pgfpathcurveto{\pgfqpoint{1.825707in}{5.320279in}}{\pgfqpoint{1.817807in}{5.317006in}}{\pgfqpoint{1.811983in}{5.311182in}}%
\pgfpathcurveto{\pgfqpoint{1.806160in}{5.305358in}}{\pgfqpoint{1.802887in}{5.297458in}}{\pgfqpoint{1.802887in}{5.289222in}}%
\pgfpathcurveto{\pgfqpoint{1.802887in}{5.280986in}}{\pgfqpoint{1.806160in}{5.273086in}}{\pgfqpoint{1.811983in}{5.267262in}}%
\pgfpathcurveto{\pgfqpoint{1.817807in}{5.261438in}}{\pgfqpoint{1.825707in}{5.258166in}}{\pgfqpoint{1.833944in}{5.258166in}}%
\pgfpathclose%
\pgfusepath{stroke,fill}%
\end{pgfscope}%
\begin{pgfscope}%
\pgfpathrectangle{\pgfqpoint{0.894063in}{3.540000in}}{\pgfqpoint{6.713438in}{2.060556in}} %
\pgfusepath{clip}%
\pgfsetbuttcap%
\pgfsetroundjoin%
\definecolor{currentfill}{rgb}{0.000000,0.000000,0.000000}%
\pgfsetfillcolor{currentfill}%
\pgfsetlinewidth{1.003750pt}%
\definecolor{currentstroke}{rgb}{0.000000,0.000000,0.000000}%
\pgfsetstrokecolor{currentstroke}%
\pgfsetdash{}{0pt}%
\pgfpathmoveto{\pgfqpoint{5.996275in}{5.257541in}}%
\pgfpathcurveto{\pgfqpoint{6.004511in}{5.257541in}}{\pgfqpoint{6.012411in}{5.260813in}}{\pgfqpoint{6.018235in}{5.266637in}}%
\pgfpathcurveto{\pgfqpoint{6.024059in}{5.272461in}}{\pgfqpoint{6.027331in}{5.280361in}}{\pgfqpoint{6.027331in}{5.288597in}}%
\pgfpathcurveto{\pgfqpoint{6.027331in}{5.296833in}}{\pgfqpoint{6.024059in}{5.304733in}}{\pgfqpoint{6.018235in}{5.310557in}}%
\pgfpathcurveto{\pgfqpoint{6.012411in}{5.316381in}}{\pgfqpoint{6.004511in}{5.319654in}}{\pgfqpoint{5.996275in}{5.319654in}}%
\pgfpathcurveto{\pgfqpoint{5.988039in}{5.319654in}}{\pgfqpoint{5.980139in}{5.316381in}}{\pgfqpoint{5.974315in}{5.310557in}}%
\pgfpathcurveto{\pgfqpoint{5.968491in}{5.304733in}}{\pgfqpoint{5.965219in}{5.296833in}}{\pgfqpoint{5.965219in}{5.288597in}}%
\pgfpathcurveto{\pgfqpoint{5.965219in}{5.280361in}}{\pgfqpoint{5.968491in}{5.272461in}}{\pgfqpoint{5.974315in}{5.266637in}}%
\pgfpathcurveto{\pgfqpoint{5.980139in}{5.260813in}}{\pgfqpoint{5.988039in}{5.257541in}}{\pgfqpoint{5.996275in}{5.257541in}}%
\pgfpathclose%
\pgfusepath{stroke,fill}%
\end{pgfscope}%
\begin{pgfscope}%
\pgfpathrectangle{\pgfqpoint{0.894063in}{3.540000in}}{\pgfqpoint{6.713438in}{2.060556in}} %
\pgfusepath{clip}%
\pgfsetbuttcap%
\pgfsetroundjoin%
\definecolor{currentfill}{rgb}{0.000000,0.000000,0.000000}%
\pgfsetfillcolor{currentfill}%
\pgfsetlinewidth{1.003750pt}%
\definecolor{currentstroke}{rgb}{0.000000,0.000000,0.000000}%
\pgfsetstrokecolor{currentstroke}%
\pgfsetdash{}{0pt}%
\pgfpathmoveto{\pgfqpoint{6.399081in}{5.257532in}}%
\pgfpathcurveto{\pgfqpoint{6.407318in}{5.257532in}}{\pgfqpoint{6.415218in}{5.260805in}}{\pgfqpoint{6.421042in}{5.266629in}}%
\pgfpathcurveto{\pgfqpoint{6.426865in}{5.272452in}}{\pgfqpoint{6.430138in}{5.280353in}}{\pgfqpoint{6.430138in}{5.288589in}}%
\pgfpathcurveto{\pgfqpoint{6.430138in}{5.296825in}}{\pgfqpoint{6.426865in}{5.304725in}}{\pgfqpoint{6.421042in}{5.310549in}}%
\pgfpathcurveto{\pgfqpoint{6.415218in}{5.316373in}}{\pgfqpoint{6.407318in}{5.319645in}}{\pgfqpoint{6.399081in}{5.319645in}}%
\pgfpathcurveto{\pgfqpoint{6.390845in}{5.319645in}}{\pgfqpoint{6.382945in}{5.316373in}}{\pgfqpoint{6.377121in}{5.310549in}}%
\pgfpathcurveto{\pgfqpoint{6.371297in}{5.304725in}}{\pgfqpoint{6.368025in}{5.296825in}}{\pgfqpoint{6.368025in}{5.288589in}}%
\pgfpathcurveto{\pgfqpoint{6.368025in}{5.280353in}}{\pgfqpoint{6.371297in}{5.272452in}}{\pgfqpoint{6.377121in}{5.266629in}}%
\pgfpathcurveto{\pgfqpoint{6.382945in}{5.260805in}}{\pgfqpoint{6.390845in}{5.257532in}}{\pgfqpoint{6.399081in}{5.257532in}}%
\pgfpathclose%
\pgfusepath{stroke,fill}%
\end{pgfscope}%
\begin{pgfscope}%
\pgfpathrectangle{\pgfqpoint{0.894063in}{3.540000in}}{\pgfqpoint{6.713438in}{2.060556in}} %
\pgfusepath{clip}%
\pgfsetbuttcap%
\pgfsetroundjoin%
\definecolor{currentfill}{rgb}{0.000000,0.000000,0.000000}%
\pgfsetfillcolor{currentfill}%
\pgfsetlinewidth{1.003750pt}%
\definecolor{currentstroke}{rgb}{0.000000,0.000000,0.000000}%
\pgfsetstrokecolor{currentstroke}%
\pgfsetdash{}{0pt}%
\pgfpathmoveto{\pgfqpoint{4.787856in}{5.257576in}}%
\pgfpathcurveto{\pgfqpoint{4.796093in}{5.257576in}}{\pgfqpoint{4.803993in}{5.260849in}}{\pgfqpoint{4.809817in}{5.266673in}}%
\pgfpathcurveto{\pgfqpoint{4.815640in}{5.272496in}}{\pgfqpoint{4.818913in}{5.280396in}}{\pgfqpoint{4.818913in}{5.288633in}}%
\pgfpathcurveto{\pgfqpoint{4.818913in}{5.296869in}}{\pgfqpoint{4.815640in}{5.304769in}}{\pgfqpoint{4.809817in}{5.310593in}}%
\pgfpathcurveto{\pgfqpoint{4.803993in}{5.316417in}}{\pgfqpoint{4.796093in}{5.319689in}}{\pgfqpoint{4.787856in}{5.319689in}}%
\pgfpathcurveto{\pgfqpoint{4.779620in}{5.319689in}}{\pgfqpoint{4.771720in}{5.316417in}}{\pgfqpoint{4.765896in}{5.310593in}}%
\pgfpathcurveto{\pgfqpoint{4.760072in}{5.304769in}}{\pgfqpoint{4.756800in}{5.296869in}}{\pgfqpoint{4.756800in}{5.288633in}}%
\pgfpathcurveto{\pgfqpoint{4.756800in}{5.280396in}}{\pgfqpoint{4.760072in}{5.272496in}}{\pgfqpoint{4.765896in}{5.266673in}}%
\pgfpathcurveto{\pgfqpoint{4.771720in}{5.260849in}}{\pgfqpoint{4.779620in}{5.257576in}}{\pgfqpoint{4.787856in}{5.257576in}}%
\pgfpathclose%
\pgfusepath{stroke,fill}%
\end{pgfscope}%
\begin{pgfscope}%
\pgfpathrectangle{\pgfqpoint{0.894063in}{3.540000in}}{\pgfqpoint{6.713438in}{2.060556in}} %
\pgfusepath{clip}%
\pgfsetbuttcap%
\pgfsetroundjoin%
\definecolor{currentfill}{rgb}{0.000000,0.000000,0.000000}%
\pgfsetfillcolor{currentfill}%
\pgfsetlinewidth{1.003750pt}%
\definecolor{currentstroke}{rgb}{0.000000,0.000000,0.000000}%
\pgfsetstrokecolor{currentstroke}%
\pgfsetdash{}{0pt}%
\pgfpathmoveto{\pgfqpoint{4.922125in}{5.257572in}}%
\pgfpathcurveto{\pgfqpoint{4.930361in}{5.257572in}}{\pgfqpoint{4.938261in}{5.260844in}}{\pgfqpoint{4.944085in}{5.266668in}}%
\pgfpathcurveto{\pgfqpoint{4.949909in}{5.272492in}}{\pgfqpoint{4.953181in}{5.280392in}}{\pgfqpoint{4.953181in}{5.288629in}}%
\pgfpathcurveto{\pgfqpoint{4.953181in}{5.296865in}}{\pgfqpoint{4.949909in}{5.304765in}}{\pgfqpoint{4.944085in}{5.310589in}}%
\pgfpathcurveto{\pgfqpoint{4.938261in}{5.316413in}}{\pgfqpoint{4.930361in}{5.319685in}}{\pgfqpoint{4.922125in}{5.319685in}}%
\pgfpathcurveto{\pgfqpoint{4.913889in}{5.319685in}}{\pgfqpoint{4.905989in}{5.316413in}}{\pgfqpoint{4.900165in}{5.310589in}}%
\pgfpathcurveto{\pgfqpoint{4.894341in}{5.304765in}}{\pgfqpoint{4.891069in}{5.296865in}}{\pgfqpoint{4.891069in}{5.288629in}}%
\pgfpathcurveto{\pgfqpoint{4.891069in}{5.280392in}}{\pgfqpoint{4.894341in}{5.272492in}}{\pgfqpoint{4.900165in}{5.266668in}}%
\pgfpathcurveto{\pgfqpoint{4.905989in}{5.260844in}}{\pgfqpoint{4.913889in}{5.257572in}}{\pgfqpoint{4.922125in}{5.257572in}}%
\pgfpathclose%
\pgfusepath{stroke,fill}%
\end{pgfscope}%
\begin{pgfscope}%
\pgfpathrectangle{\pgfqpoint{0.894063in}{3.540000in}}{\pgfqpoint{6.713438in}{2.060556in}} %
\pgfusepath{clip}%
\pgfsetbuttcap%
\pgfsetroundjoin%
\definecolor{currentfill}{rgb}{0.000000,0.000000,0.000000}%
\pgfsetfillcolor{currentfill}%
\pgfsetlinewidth{1.003750pt}%
\definecolor{currentstroke}{rgb}{0.000000,0.000000,0.000000}%
\pgfsetstrokecolor{currentstroke}%
\pgfsetdash{}{0pt}%
\pgfpathmoveto{\pgfqpoint{6.130544in}{5.257534in}}%
\pgfpathcurveto{\pgfqpoint{6.138780in}{5.257534in}}{\pgfqpoint{6.146680in}{5.260806in}}{\pgfqpoint{6.152504in}{5.266630in}}%
\pgfpathcurveto{\pgfqpoint{6.158328in}{5.272454in}}{\pgfqpoint{6.161600in}{5.280354in}}{\pgfqpoint{6.161600in}{5.288590in}}%
\pgfpathcurveto{\pgfqpoint{6.161600in}{5.296826in}}{\pgfqpoint{6.158328in}{5.304727in}}{\pgfqpoint{6.152504in}{5.310550in}}%
\pgfpathcurveto{\pgfqpoint{6.146680in}{5.316374in}}{\pgfqpoint{6.138780in}{5.319647in}}{\pgfqpoint{6.130544in}{5.319647in}}%
\pgfpathcurveto{\pgfqpoint{6.122307in}{5.319647in}}{\pgfqpoint{6.114407in}{5.316374in}}{\pgfqpoint{6.108583in}{5.310550in}}%
\pgfpathcurveto{\pgfqpoint{6.102760in}{5.304727in}}{\pgfqpoint{6.099487in}{5.296826in}}{\pgfqpoint{6.099487in}{5.288590in}}%
\pgfpathcurveto{\pgfqpoint{6.099487in}{5.280354in}}{\pgfqpoint{6.102760in}{5.272454in}}{\pgfqpoint{6.108583in}{5.266630in}}%
\pgfpathcurveto{\pgfqpoint{6.114407in}{5.260806in}}{\pgfqpoint{6.122307in}{5.257534in}}{\pgfqpoint{6.130544in}{5.257534in}}%
\pgfpathclose%
\pgfusepath{stroke,fill}%
\end{pgfscope}%
\begin{pgfscope}%
\pgfpathrectangle{\pgfqpoint{0.894063in}{3.540000in}}{\pgfqpoint{6.713438in}{2.060556in}} %
\pgfusepath{clip}%
\pgfsetbuttcap%
\pgfsetroundjoin%
\definecolor{currentfill}{rgb}{0.000000,0.000000,0.000000}%
\pgfsetfillcolor{currentfill}%
\pgfsetlinewidth{1.003750pt}%
\definecolor{currentstroke}{rgb}{0.000000,0.000000,0.000000}%
\pgfsetstrokecolor{currentstroke}%
\pgfsetdash{}{0pt}%
\pgfpathmoveto{\pgfqpoint{5.727738in}{5.257543in}}%
\pgfpathcurveto{\pgfqpoint{5.735974in}{5.257543in}}{\pgfqpoint{5.743874in}{5.260816in}}{\pgfqpoint{5.749698in}{5.266640in}}%
\pgfpathcurveto{\pgfqpoint{5.755522in}{5.272463in}}{\pgfqpoint{5.758794in}{5.280364in}}{\pgfqpoint{5.758794in}{5.288600in}}%
\pgfpathcurveto{\pgfqpoint{5.758794in}{5.296836in}}{\pgfqpoint{5.755522in}{5.304736in}}{\pgfqpoint{5.749698in}{5.310560in}}%
\pgfpathcurveto{\pgfqpoint{5.743874in}{5.316384in}}{\pgfqpoint{5.735974in}{5.319656in}}{\pgfqpoint{5.727738in}{5.319656in}}%
\pgfpathcurveto{\pgfqpoint{5.719501in}{5.319656in}}{\pgfqpoint{5.711601in}{5.316384in}}{\pgfqpoint{5.705777in}{5.310560in}}%
\pgfpathcurveto{\pgfqpoint{5.699953in}{5.304736in}}{\pgfqpoint{5.696681in}{5.296836in}}{\pgfqpoint{5.696681in}{5.288600in}}%
\pgfpathcurveto{\pgfqpoint{5.696681in}{5.280364in}}{\pgfqpoint{5.699953in}{5.272463in}}{\pgfqpoint{5.705777in}{5.266640in}}%
\pgfpathcurveto{\pgfqpoint{5.711601in}{5.260816in}}{\pgfqpoint{5.719501in}{5.257543in}}{\pgfqpoint{5.727738in}{5.257543in}}%
\pgfpathclose%
\pgfusepath{stroke,fill}%
\end{pgfscope}%
\begin{pgfscope}%
\pgfpathrectangle{\pgfqpoint{0.894063in}{3.540000in}}{\pgfqpoint{6.713438in}{2.060556in}} %
\pgfusepath{clip}%
\pgfsetbuttcap%
\pgfsetroundjoin%
\definecolor{currentfill}{rgb}{0.000000,0.000000,0.000000}%
\pgfsetfillcolor{currentfill}%
\pgfsetlinewidth{1.003750pt}%
\definecolor{currentstroke}{rgb}{0.000000,0.000000,0.000000}%
\pgfsetstrokecolor{currentstroke}%
\pgfsetdash{}{0pt}%
\pgfpathmoveto{\pgfqpoint{1.028331in}{5.258357in}}%
\pgfpathcurveto{\pgfqpoint{1.036568in}{5.258357in}}{\pgfqpoint{1.044468in}{5.261629in}}{\pgfqpoint{1.050292in}{5.267453in}}%
\pgfpathcurveto{\pgfqpoint{1.056115in}{5.273277in}}{\pgfqpoint{1.059388in}{5.281177in}}{\pgfqpoint{1.059388in}{5.289413in}}%
\pgfpathcurveto{\pgfqpoint{1.059388in}{5.297649in}}{\pgfqpoint{1.056115in}{5.305549in}}{\pgfqpoint{1.050292in}{5.311373in}}%
\pgfpathcurveto{\pgfqpoint{1.044468in}{5.317197in}}{\pgfqpoint{1.036568in}{5.320470in}}{\pgfqpoint{1.028331in}{5.320470in}}%
\pgfpathcurveto{\pgfqpoint{1.020095in}{5.320470in}}{\pgfqpoint{1.012195in}{5.317197in}}{\pgfqpoint{1.006371in}{5.311373in}}%
\pgfpathcurveto{\pgfqpoint{1.000547in}{5.305549in}}{\pgfqpoint{0.997275in}{5.297649in}}{\pgfqpoint{0.997275in}{5.289413in}}%
\pgfpathcurveto{\pgfqpoint{0.997275in}{5.281177in}}{\pgfqpoint{1.000547in}{5.273277in}}{\pgfqpoint{1.006371in}{5.267453in}}%
\pgfpathcurveto{\pgfqpoint{1.012195in}{5.261629in}}{\pgfqpoint{1.020095in}{5.258357in}}{\pgfqpoint{1.028331in}{5.258357in}}%
\pgfpathclose%
\pgfusepath{stroke,fill}%
\end{pgfscope}%
\begin{pgfscope}%
\pgfpathrectangle{\pgfqpoint{0.894063in}{3.540000in}}{\pgfqpoint{6.713438in}{2.060556in}} %
\pgfusepath{clip}%
\pgfsetbuttcap%
\pgfsetroundjoin%
\definecolor{currentfill}{rgb}{0.000000,0.000000,0.000000}%
\pgfsetfillcolor{currentfill}%
\pgfsetlinewidth{1.003750pt}%
\definecolor{currentstroke}{rgb}{0.000000,0.000000,0.000000}%
\pgfsetstrokecolor{currentstroke}%
\pgfsetdash{}{0pt}%
\pgfpathmoveto{\pgfqpoint{5.324931in}{5.257546in}}%
\pgfpathcurveto{\pgfqpoint{5.333168in}{5.257546in}}{\pgfqpoint{5.341068in}{5.260818in}}{\pgfqpoint{5.346892in}{5.266642in}}%
\pgfpathcurveto{\pgfqpoint{5.352715in}{5.272466in}}{\pgfqpoint{5.355988in}{5.280366in}}{\pgfqpoint{5.355988in}{5.288603in}}%
\pgfpathcurveto{\pgfqpoint{5.355988in}{5.296839in}}{\pgfqpoint{5.352715in}{5.304739in}}{\pgfqpoint{5.346892in}{5.310563in}}%
\pgfpathcurveto{\pgfqpoint{5.341068in}{5.316387in}}{\pgfqpoint{5.333168in}{5.319659in}}{\pgfqpoint{5.324931in}{5.319659in}}%
\pgfpathcurveto{\pgfqpoint{5.316695in}{5.319659in}}{\pgfqpoint{5.308795in}{5.316387in}}{\pgfqpoint{5.302971in}{5.310563in}}%
\pgfpathcurveto{\pgfqpoint{5.297147in}{5.304739in}}{\pgfqpoint{5.293875in}{5.296839in}}{\pgfqpoint{5.293875in}{5.288603in}}%
\pgfpathcurveto{\pgfqpoint{5.293875in}{5.280366in}}{\pgfqpoint{5.297147in}{5.272466in}}{\pgfqpoint{5.302971in}{5.266642in}}%
\pgfpathcurveto{\pgfqpoint{5.308795in}{5.260818in}}{\pgfqpoint{5.316695in}{5.257546in}}{\pgfqpoint{5.324931in}{5.257546in}}%
\pgfpathclose%
\pgfusepath{stroke,fill}%
\end{pgfscope}%
\begin{pgfscope}%
\pgfpathrectangle{\pgfqpoint{0.894063in}{3.540000in}}{\pgfqpoint{6.713438in}{2.060556in}} %
\pgfusepath{clip}%
\pgfsetbuttcap%
\pgfsetroundjoin%
\definecolor{currentfill}{rgb}{0.000000,0.000000,0.000000}%
\pgfsetfillcolor{currentfill}%
\pgfsetlinewidth{1.003750pt}%
\definecolor{currentstroke}{rgb}{0.000000,0.000000,0.000000}%
\pgfsetstrokecolor{currentstroke}%
\pgfsetdash{}{0pt}%
\pgfpathmoveto{\pgfqpoint{7.338963in}{5.256799in}}%
\pgfpathcurveto{\pgfqpoint{7.347199in}{5.256799in}}{\pgfqpoint{7.355099in}{5.260071in}}{\pgfqpoint{7.360923in}{5.265895in}}%
\pgfpathcurveto{\pgfqpoint{7.366747in}{5.271719in}}{\pgfqpoint{7.370019in}{5.279619in}}{\pgfqpoint{7.370019in}{5.287855in}}%
\pgfpathcurveto{\pgfqpoint{7.370019in}{5.296092in}}{\pgfqpoint{7.366747in}{5.303992in}}{\pgfqpoint{7.360923in}{5.309816in}}%
\pgfpathcurveto{\pgfqpoint{7.355099in}{5.315639in}}{\pgfqpoint{7.347199in}{5.318912in}}{\pgfqpoint{7.338963in}{5.318912in}}%
\pgfpathcurveto{\pgfqpoint{7.330726in}{5.318912in}}{\pgfqpoint{7.322826in}{5.315639in}}{\pgfqpoint{7.317002in}{5.309816in}}%
\pgfpathcurveto{\pgfqpoint{7.311178in}{5.303992in}}{\pgfqpoint{7.307906in}{5.296092in}}{\pgfqpoint{7.307906in}{5.287855in}}%
\pgfpathcurveto{\pgfqpoint{7.307906in}{5.279619in}}{\pgfqpoint{7.311178in}{5.271719in}}{\pgfqpoint{7.317002in}{5.265895in}}%
\pgfpathcurveto{\pgfqpoint{7.322826in}{5.260071in}}{\pgfqpoint{7.330726in}{5.256799in}}{\pgfqpoint{7.338963in}{5.256799in}}%
\pgfpathclose%
\pgfusepath{stroke,fill}%
\end{pgfscope}%
\begin{pgfscope}%
\pgfpathrectangle{\pgfqpoint{0.894063in}{3.540000in}}{\pgfqpoint{6.713438in}{2.060556in}} %
\pgfusepath{clip}%
\pgfsetbuttcap%
\pgfsetroundjoin%
\definecolor{currentfill}{rgb}{0.000000,0.000000,0.000000}%
\pgfsetfillcolor{currentfill}%
\pgfsetlinewidth{1.003750pt}%
\definecolor{currentstroke}{rgb}{0.000000,0.000000,0.000000}%
\pgfsetstrokecolor{currentstroke}%
\pgfsetdash{}{0pt}%
\pgfpathmoveto{\pgfqpoint{7.204694in}{5.256807in}}%
\pgfpathcurveto{\pgfqpoint{7.212930in}{5.256807in}}{\pgfqpoint{7.220830in}{5.260079in}}{\pgfqpoint{7.226654in}{5.265903in}}%
\pgfpathcurveto{\pgfqpoint{7.232478in}{5.271727in}}{\pgfqpoint{7.235750in}{5.279627in}}{\pgfqpoint{7.235750in}{5.287864in}}%
\pgfpathcurveto{\pgfqpoint{7.235750in}{5.296100in}}{\pgfqpoint{7.232478in}{5.304000in}}{\pgfqpoint{7.226654in}{5.309824in}}%
\pgfpathcurveto{\pgfqpoint{7.220830in}{5.315648in}}{\pgfqpoint{7.212930in}{5.318920in}}{\pgfqpoint{7.204694in}{5.318920in}}%
\pgfpathcurveto{\pgfqpoint{7.196457in}{5.318920in}}{\pgfqpoint{7.188557in}{5.315648in}}{\pgfqpoint{7.182733in}{5.309824in}}%
\pgfpathcurveto{\pgfqpoint{7.176910in}{5.304000in}}{\pgfqpoint{7.173637in}{5.296100in}}{\pgfqpoint{7.173637in}{5.287864in}}%
\pgfpathcurveto{\pgfqpoint{7.173637in}{5.279627in}}{\pgfqpoint{7.176910in}{5.271727in}}{\pgfqpoint{7.182733in}{5.265903in}}%
\pgfpathcurveto{\pgfqpoint{7.188557in}{5.260079in}}{\pgfqpoint{7.196457in}{5.256807in}}{\pgfqpoint{7.204694in}{5.256807in}}%
\pgfpathclose%
\pgfusepath{stroke,fill}%
\end{pgfscope}%
\begin{pgfscope}%
\pgfpathrectangle{\pgfqpoint{0.894063in}{3.540000in}}{\pgfqpoint{6.713438in}{2.060556in}} %
\pgfusepath{clip}%
\pgfsetbuttcap%
\pgfsetroundjoin%
\definecolor{currentfill}{rgb}{0.000000,0.000000,0.000000}%
\pgfsetfillcolor{currentfill}%
\pgfsetlinewidth{1.003750pt}%
\definecolor{currentstroke}{rgb}{0.000000,0.000000,0.000000}%
\pgfsetstrokecolor{currentstroke}%
\pgfsetdash{}{0pt}%
\pgfpathmoveto{\pgfqpoint{6.264813in}{5.257534in}}%
\pgfpathcurveto{\pgfqpoint{6.273049in}{5.257534in}}{\pgfqpoint{6.280949in}{5.260806in}}{\pgfqpoint{6.286773in}{5.266630in}}%
\pgfpathcurveto{\pgfqpoint{6.292597in}{5.272454in}}{\pgfqpoint{6.295869in}{5.280354in}}{\pgfqpoint{6.295869in}{5.288590in}}%
\pgfpathcurveto{\pgfqpoint{6.295869in}{5.296826in}}{\pgfqpoint{6.292597in}{5.304727in}}{\pgfqpoint{6.286773in}{5.310550in}}%
\pgfpathcurveto{\pgfqpoint{6.280949in}{5.316374in}}{\pgfqpoint{6.273049in}{5.319647in}}{\pgfqpoint{6.264813in}{5.319647in}}%
\pgfpathcurveto{\pgfqpoint{6.256576in}{5.319647in}}{\pgfqpoint{6.248676in}{5.316374in}}{\pgfqpoint{6.242852in}{5.310550in}}%
\pgfpathcurveto{\pgfqpoint{6.237028in}{5.304727in}}{\pgfqpoint{6.233756in}{5.296826in}}{\pgfqpoint{6.233756in}{5.288590in}}%
\pgfpathcurveto{\pgfqpoint{6.233756in}{5.280354in}}{\pgfqpoint{6.237028in}{5.272454in}}{\pgfqpoint{6.242852in}{5.266630in}}%
\pgfpathcurveto{\pgfqpoint{6.248676in}{5.260806in}}{\pgfqpoint{6.256576in}{5.257534in}}{\pgfqpoint{6.264813in}{5.257534in}}%
\pgfpathclose%
\pgfusepath{stroke,fill}%
\end{pgfscope}%
\begin{pgfscope}%
\pgfpathrectangle{\pgfqpoint{0.894063in}{3.540000in}}{\pgfqpoint{6.713438in}{2.060556in}} %
\pgfusepath{clip}%
\pgfsetbuttcap%
\pgfsetroundjoin%
\definecolor{currentfill}{rgb}{0.000000,0.000000,0.000000}%
\pgfsetfillcolor{currentfill}%
\pgfsetlinewidth{1.003750pt}%
\definecolor{currentstroke}{rgb}{0.000000,0.000000,0.000000}%
\pgfsetstrokecolor{currentstroke}%
\pgfsetdash{}{0pt}%
\pgfpathmoveto{\pgfqpoint{7.473231in}{5.256785in}}%
\pgfpathcurveto{\pgfqpoint{7.481468in}{5.256785in}}{\pgfqpoint{7.489368in}{5.260057in}}{\pgfqpoint{7.495192in}{5.265881in}}%
\pgfpathcurveto{\pgfqpoint{7.501015in}{5.271705in}}{\pgfqpoint{7.504288in}{5.279605in}}{\pgfqpoint{7.504288in}{5.287842in}}%
\pgfpathcurveto{\pgfqpoint{7.504288in}{5.296078in}}{\pgfqpoint{7.501015in}{5.303978in}}{\pgfqpoint{7.495192in}{5.309802in}}%
\pgfpathcurveto{\pgfqpoint{7.489368in}{5.315626in}}{\pgfqpoint{7.481468in}{5.318898in}}{\pgfqpoint{7.473231in}{5.318898in}}%
\pgfpathcurveto{\pgfqpoint{7.464995in}{5.318898in}}{\pgfqpoint{7.457095in}{5.315626in}}{\pgfqpoint{7.451271in}{5.309802in}}%
\pgfpathcurveto{\pgfqpoint{7.445447in}{5.303978in}}{\pgfqpoint{7.442175in}{5.296078in}}{\pgfqpoint{7.442175in}{5.287842in}}%
\pgfpathcurveto{\pgfqpoint{7.442175in}{5.279605in}}{\pgfqpoint{7.445447in}{5.271705in}}{\pgfqpoint{7.451271in}{5.265881in}}%
\pgfpathcurveto{\pgfqpoint{7.457095in}{5.260057in}}{\pgfqpoint{7.464995in}{5.256785in}}{\pgfqpoint{7.473231in}{5.256785in}}%
\pgfpathclose%
\pgfusepath{stroke,fill}%
\end{pgfscope}%
\begin{pgfscope}%
\pgfpathrectangle{\pgfqpoint{0.894063in}{3.540000in}}{\pgfqpoint{6.713438in}{2.060556in}} %
\pgfusepath{clip}%
\pgfsetbuttcap%
\pgfsetroundjoin%
\definecolor{currentfill}{rgb}{0.000000,0.000000,0.000000}%
\pgfsetfillcolor{currentfill}%
\pgfsetlinewidth{1.003750pt}%
\definecolor{currentstroke}{rgb}{0.000000,0.000000,0.000000}%
\pgfsetstrokecolor{currentstroke}%
\pgfsetdash{}{0pt}%
\pgfpathmoveto{\pgfqpoint{5.056394in}{5.257565in}}%
\pgfpathcurveto{\pgfqpoint{5.064630in}{5.257565in}}{\pgfqpoint{5.072530in}{5.260838in}}{\pgfqpoint{5.078354in}{5.266662in}}%
\pgfpathcurveto{\pgfqpoint{5.084178in}{5.272485in}}{\pgfqpoint{5.087450in}{5.280386in}}{\pgfqpoint{5.087450in}{5.288622in}}%
\pgfpathcurveto{\pgfqpoint{5.087450in}{5.296858in}}{\pgfqpoint{5.084178in}{5.304758in}}{\pgfqpoint{5.078354in}{5.310582in}}%
\pgfpathcurveto{\pgfqpoint{5.072530in}{5.316406in}}{\pgfqpoint{5.064630in}{5.319678in}}{\pgfqpoint{5.056394in}{5.319678in}}%
\pgfpathcurveto{\pgfqpoint{5.048157in}{5.319678in}}{\pgfqpoint{5.040257in}{5.316406in}}{\pgfqpoint{5.034433in}{5.310582in}}%
\pgfpathcurveto{\pgfqpoint{5.028610in}{5.304758in}}{\pgfqpoint{5.025337in}{5.296858in}}{\pgfqpoint{5.025337in}{5.288622in}}%
\pgfpathcurveto{\pgfqpoint{5.025337in}{5.280386in}}{\pgfqpoint{5.028610in}{5.272485in}}{\pgfqpoint{5.034433in}{5.266662in}}%
\pgfpathcurveto{\pgfqpoint{5.040257in}{5.260838in}}{\pgfqpoint{5.048157in}{5.257565in}}{\pgfqpoint{5.056394in}{5.257565in}}%
\pgfpathclose%
\pgfusepath{stroke,fill}%
\end{pgfscope}%
\begin{pgfscope}%
\pgfpathrectangle{\pgfqpoint{0.894063in}{3.540000in}}{\pgfqpoint{6.713438in}{2.060556in}} %
\pgfusepath{clip}%
\pgfsetbuttcap%
\pgfsetroundjoin%
\definecolor{currentfill}{rgb}{0.000000,0.000000,0.000000}%
\pgfsetfillcolor{currentfill}%
\pgfsetlinewidth{1.003750pt}%
\definecolor{currentstroke}{rgb}{0.000000,0.000000,0.000000}%
\pgfsetstrokecolor{currentstroke}%
\pgfsetdash{}{0pt}%
\pgfpathmoveto{\pgfqpoint{2.908094in}{5.257703in}}%
\pgfpathcurveto{\pgfqpoint{2.916330in}{5.257703in}}{\pgfqpoint{2.924230in}{5.260975in}}{\pgfqpoint{2.930054in}{5.266799in}}%
\pgfpathcurveto{\pgfqpoint{2.935878in}{5.272623in}}{\pgfqpoint{2.939150in}{5.280523in}}{\pgfqpoint{2.939150in}{5.288759in}}%
\pgfpathcurveto{\pgfqpoint{2.939150in}{5.296995in}}{\pgfqpoint{2.935878in}{5.304895in}}{\pgfqpoint{2.930054in}{5.310719in}}%
\pgfpathcurveto{\pgfqpoint{2.924230in}{5.316543in}}{\pgfqpoint{2.916330in}{5.319816in}}{\pgfqpoint{2.908094in}{5.319816in}}%
\pgfpathcurveto{\pgfqpoint{2.899857in}{5.319816in}}{\pgfqpoint{2.891957in}{5.316543in}}{\pgfqpoint{2.886133in}{5.310719in}}%
\pgfpathcurveto{\pgfqpoint{2.880310in}{5.304895in}}{\pgfqpoint{2.877037in}{5.296995in}}{\pgfqpoint{2.877037in}{5.288759in}}%
\pgfpathcurveto{\pgfqpoint{2.877037in}{5.280523in}}{\pgfqpoint{2.880310in}{5.272623in}}{\pgfqpoint{2.886133in}{5.266799in}}%
\pgfpathcurveto{\pgfqpoint{2.891957in}{5.260975in}}{\pgfqpoint{2.899857in}{5.257703in}}{\pgfqpoint{2.908094in}{5.257703in}}%
\pgfpathclose%
\pgfusepath{stroke,fill}%
\end{pgfscope}%
\begin{pgfscope}%
\pgfpathrectangle{\pgfqpoint{0.894063in}{3.540000in}}{\pgfqpoint{6.713438in}{2.060556in}} %
\pgfusepath{clip}%
\pgfsetbuttcap%
\pgfsetroundjoin%
\definecolor{currentfill}{rgb}{0.000000,0.000000,0.000000}%
\pgfsetfillcolor{currentfill}%
\pgfsetlinewidth{1.003750pt}%
\definecolor{currentstroke}{rgb}{0.000000,0.000000,0.000000}%
\pgfsetstrokecolor{currentstroke}%
\pgfsetdash{}{0pt}%
\pgfpathmoveto{\pgfqpoint{3.445169in}{5.257690in}}%
\pgfpathcurveto{\pgfqpoint{3.453405in}{5.257690in}}{\pgfqpoint{3.461305in}{5.260963in}}{\pgfqpoint{3.467129in}{5.266787in}}%
\pgfpathcurveto{\pgfqpoint{3.472953in}{5.272610in}}{\pgfqpoint{3.476225in}{5.280511in}}{\pgfqpoint{3.476225in}{5.288747in}}%
\pgfpathcurveto{\pgfqpoint{3.476225in}{5.296983in}}{\pgfqpoint{3.472953in}{5.304883in}}{\pgfqpoint{3.467129in}{5.310707in}}%
\pgfpathcurveto{\pgfqpoint{3.461305in}{5.316531in}}{\pgfqpoint{3.453405in}{5.319803in}}{\pgfqpoint{3.445169in}{5.319803in}}%
\pgfpathcurveto{\pgfqpoint{3.436932in}{5.319803in}}{\pgfqpoint{3.429032in}{5.316531in}}{\pgfqpoint{3.423208in}{5.310707in}}%
\pgfpathcurveto{\pgfqpoint{3.417385in}{5.304883in}}{\pgfqpoint{3.414112in}{5.296983in}}{\pgfqpoint{3.414112in}{5.288747in}}%
\pgfpathcurveto{\pgfqpoint{3.414112in}{5.280511in}}{\pgfqpoint{3.417385in}{5.272610in}}{\pgfqpoint{3.423208in}{5.266787in}}%
\pgfpathcurveto{\pgfqpoint{3.429032in}{5.260963in}}{\pgfqpoint{3.436932in}{5.257690in}}{\pgfqpoint{3.445169in}{5.257690in}}%
\pgfpathclose%
\pgfusepath{stroke,fill}%
\end{pgfscope}%
\begin{pgfscope}%
\pgfpathrectangle{\pgfqpoint{0.894063in}{3.540000in}}{\pgfqpoint{6.713438in}{2.060556in}} %
\pgfusepath{clip}%
\pgfsetbuttcap%
\pgfsetroundjoin%
\definecolor{currentfill}{rgb}{0.000000,0.000000,0.000000}%
\pgfsetfillcolor{currentfill}%
\pgfsetlinewidth{1.003750pt}%
\definecolor{currentstroke}{rgb}{0.000000,0.000000,0.000000}%
\pgfsetstrokecolor{currentstroke}%
\pgfsetdash{}{0pt}%
\pgfpathmoveto{\pgfqpoint{4.116513in}{5.257608in}}%
\pgfpathcurveto{\pgfqpoint{4.124749in}{5.257608in}}{\pgfqpoint{4.132649in}{5.260880in}}{\pgfqpoint{4.138473in}{5.266704in}}%
\pgfpathcurveto{\pgfqpoint{4.144297in}{5.272528in}}{\pgfqpoint{4.147569in}{5.280428in}}{\pgfqpoint{4.147569in}{5.288664in}}%
\pgfpathcurveto{\pgfqpoint{4.147569in}{5.296901in}}{\pgfqpoint{4.144297in}{5.304801in}}{\pgfqpoint{4.138473in}{5.310625in}}%
\pgfpathcurveto{\pgfqpoint{4.132649in}{5.316449in}}{\pgfqpoint{4.124749in}{5.319721in}}{\pgfqpoint{4.116513in}{5.319721in}}%
\pgfpathcurveto{\pgfqpoint{4.108276in}{5.319721in}}{\pgfqpoint{4.100376in}{5.316449in}}{\pgfqpoint{4.094552in}{5.310625in}}%
\pgfpathcurveto{\pgfqpoint{4.088728in}{5.304801in}}{\pgfqpoint{4.085456in}{5.296901in}}{\pgfqpoint{4.085456in}{5.288664in}}%
\pgfpathcurveto{\pgfqpoint{4.085456in}{5.280428in}}{\pgfqpoint{4.088728in}{5.272528in}}{\pgfqpoint{4.094552in}{5.266704in}}%
\pgfpathcurveto{\pgfqpoint{4.100376in}{5.260880in}}{\pgfqpoint{4.108276in}{5.257608in}}{\pgfqpoint{4.116513in}{5.257608in}}%
\pgfpathclose%
\pgfusepath{stroke,fill}%
\end{pgfscope}%
\begin{pgfscope}%
\pgfpathrectangle{\pgfqpoint{0.894063in}{3.540000in}}{\pgfqpoint{6.713438in}{2.060556in}} %
\pgfusepath{clip}%
\pgfsetbuttcap%
\pgfsetroundjoin%
\definecolor{currentfill}{rgb}{0.000000,0.000000,0.000000}%
\pgfsetfillcolor{currentfill}%
\pgfsetlinewidth{1.003750pt}%
\definecolor{currentstroke}{rgb}{0.000000,0.000000,0.000000}%
\pgfsetstrokecolor{currentstroke}%
\pgfsetdash{}{0pt}%
\pgfpathmoveto{\pgfqpoint{1.431138in}{5.258194in}}%
\pgfpathcurveto{\pgfqpoint{1.439374in}{5.258194in}}{\pgfqpoint{1.447274in}{5.261467in}}{\pgfqpoint{1.453098in}{5.267291in}}%
\pgfpathcurveto{\pgfqpoint{1.458922in}{5.273115in}}{\pgfqpoint{1.462194in}{5.281015in}}{\pgfqpoint{1.462194in}{5.289251in}}%
\pgfpathcurveto{\pgfqpoint{1.462194in}{5.297487in}}{\pgfqpoint{1.458922in}{5.305387in}}{\pgfqpoint{1.453098in}{5.311211in}}%
\pgfpathcurveto{\pgfqpoint{1.447274in}{5.317035in}}{\pgfqpoint{1.439374in}{5.320307in}}{\pgfqpoint{1.431138in}{5.320307in}}%
\pgfpathcurveto{\pgfqpoint{1.422901in}{5.320307in}}{\pgfqpoint{1.415001in}{5.317035in}}{\pgfqpoint{1.409177in}{5.311211in}}%
\pgfpathcurveto{\pgfqpoint{1.403353in}{5.305387in}}{\pgfqpoint{1.400081in}{5.297487in}}{\pgfqpoint{1.400081in}{5.289251in}}%
\pgfpathcurveto{\pgfqpoint{1.400081in}{5.281015in}}{\pgfqpoint{1.403353in}{5.273115in}}{\pgfqpoint{1.409177in}{5.267291in}}%
\pgfpathcurveto{\pgfqpoint{1.415001in}{5.261467in}}{\pgfqpoint{1.422901in}{5.258194in}}{\pgfqpoint{1.431138in}{5.258194in}}%
\pgfpathclose%
\pgfusepath{stroke,fill}%
\end{pgfscope}%
\begin{pgfscope}%
\pgfpathrectangle{\pgfqpoint{0.894063in}{3.540000in}}{\pgfqpoint{6.713438in}{2.060556in}} %
\pgfusepath{clip}%
\pgfsetbuttcap%
\pgfsetroundjoin%
\definecolor{currentfill}{rgb}{0.000000,0.000000,0.000000}%
\pgfsetfillcolor{currentfill}%
\pgfsetlinewidth{1.003750pt}%
\definecolor{currentstroke}{rgb}{0.000000,0.000000,0.000000}%
\pgfsetstrokecolor{currentstroke}%
\pgfsetdash{}{0pt}%
\pgfpathmoveto{\pgfqpoint{2.773825in}{5.257705in}}%
\pgfpathcurveto{\pgfqpoint{2.782061in}{5.257705in}}{\pgfqpoint{2.789961in}{5.260978in}}{\pgfqpoint{2.795785in}{5.266802in}}%
\pgfpathcurveto{\pgfqpoint{2.801609in}{5.272626in}}{\pgfqpoint{2.804881in}{5.280526in}}{\pgfqpoint{2.804881in}{5.288762in}}%
\pgfpathcurveto{\pgfqpoint{2.804881in}{5.296998in}}{\pgfqpoint{2.801609in}{5.304898in}}{\pgfqpoint{2.795785in}{5.310722in}}%
\pgfpathcurveto{\pgfqpoint{2.789961in}{5.316546in}}{\pgfqpoint{2.782061in}{5.319818in}}{\pgfqpoint{2.773825in}{5.319818in}}%
\pgfpathcurveto{\pgfqpoint{2.765589in}{5.319818in}}{\pgfqpoint{2.757689in}{5.316546in}}{\pgfqpoint{2.751865in}{5.310722in}}%
\pgfpathcurveto{\pgfqpoint{2.746041in}{5.304898in}}{\pgfqpoint{2.742769in}{5.296998in}}{\pgfqpoint{2.742769in}{5.288762in}}%
\pgfpathcurveto{\pgfqpoint{2.742769in}{5.280526in}}{\pgfqpoint{2.746041in}{5.272626in}}{\pgfqpoint{2.751865in}{5.266802in}}%
\pgfpathcurveto{\pgfqpoint{2.757689in}{5.260978in}}{\pgfqpoint{2.765589in}{5.257705in}}{\pgfqpoint{2.773825in}{5.257705in}}%
\pgfpathclose%
\pgfusepath{stroke,fill}%
\end{pgfscope}%
\begin{pgfscope}%
\pgfpathrectangle{\pgfqpoint{0.894063in}{3.540000in}}{\pgfqpoint{6.713438in}{2.060556in}} %
\pgfusepath{clip}%
\pgfsetbuttcap%
\pgfsetroundjoin%
\definecolor{currentfill}{rgb}{0.000000,0.000000,0.000000}%
\pgfsetfillcolor{currentfill}%
\pgfsetlinewidth{1.003750pt}%
\definecolor{currentstroke}{rgb}{0.000000,0.000000,0.000000}%
\pgfsetstrokecolor{currentstroke}%
\pgfsetdash{}{0pt}%
\pgfpathmoveto{\pgfqpoint{1.565406in}{5.258183in}}%
\pgfpathcurveto{\pgfqpoint{1.573643in}{5.258183in}}{\pgfqpoint{1.581543in}{5.261456in}}{\pgfqpoint{1.587367in}{5.267280in}}%
\pgfpathcurveto{\pgfqpoint{1.593190in}{5.273104in}}{\pgfqpoint{1.596463in}{5.281004in}}{\pgfqpoint{1.596463in}{5.289240in}}%
\pgfpathcurveto{\pgfqpoint{1.596463in}{5.297476in}}{\pgfqpoint{1.593190in}{5.305376in}}{\pgfqpoint{1.587367in}{5.311200in}}%
\pgfpathcurveto{\pgfqpoint{1.581543in}{5.317024in}}{\pgfqpoint{1.573643in}{5.320296in}}{\pgfqpoint{1.565406in}{5.320296in}}%
\pgfpathcurveto{\pgfqpoint{1.557170in}{5.320296in}}{\pgfqpoint{1.549270in}{5.317024in}}{\pgfqpoint{1.543446in}{5.311200in}}%
\pgfpathcurveto{\pgfqpoint{1.537622in}{5.305376in}}{\pgfqpoint{1.534350in}{5.297476in}}{\pgfqpoint{1.534350in}{5.289240in}}%
\pgfpathcurveto{\pgfqpoint{1.534350in}{5.281004in}}{\pgfqpoint{1.537622in}{5.273104in}}{\pgfqpoint{1.543446in}{5.267280in}}%
\pgfpathcurveto{\pgfqpoint{1.549270in}{5.261456in}}{\pgfqpoint{1.557170in}{5.258183in}}{\pgfqpoint{1.565406in}{5.258183in}}%
\pgfpathclose%
\pgfusepath{stroke,fill}%
\end{pgfscope}%
\begin{pgfscope}%
\pgfpathrectangle{\pgfqpoint{0.894063in}{3.540000in}}{\pgfqpoint{6.713438in}{2.060556in}} %
\pgfusepath{clip}%
\pgfsetbuttcap%
\pgfsetroundjoin%
\definecolor{currentfill}{rgb}{0.000000,0.000000,0.000000}%
\pgfsetfillcolor{currentfill}%
\pgfsetlinewidth{1.003750pt}%
\definecolor{currentstroke}{rgb}{0.000000,0.000000,0.000000}%
\pgfsetstrokecolor{currentstroke}%
\pgfsetdash{}{0pt}%
\pgfpathmoveto{\pgfqpoint{4.250781in}{5.257608in}}%
\pgfpathcurveto{\pgfqpoint{4.259018in}{5.257608in}}{\pgfqpoint{4.266918in}{5.260880in}}{\pgfqpoint{4.272742in}{5.266704in}}%
\pgfpathcurveto{\pgfqpoint{4.278565in}{5.272528in}}{\pgfqpoint{4.281838in}{5.280428in}}{\pgfqpoint{4.281838in}{5.288664in}}%
\pgfpathcurveto{\pgfqpoint{4.281838in}{5.296901in}}{\pgfqpoint{4.278565in}{5.304801in}}{\pgfqpoint{4.272742in}{5.310625in}}%
\pgfpathcurveto{\pgfqpoint{4.266918in}{5.316449in}}{\pgfqpoint{4.259018in}{5.319721in}}{\pgfqpoint{4.250781in}{5.319721in}}%
\pgfpathcurveto{\pgfqpoint{4.242545in}{5.319721in}}{\pgfqpoint{4.234645in}{5.316449in}}{\pgfqpoint{4.228821in}{5.310625in}}%
\pgfpathcurveto{\pgfqpoint{4.222997in}{5.304801in}}{\pgfqpoint{4.219725in}{5.296901in}}{\pgfqpoint{4.219725in}{5.288664in}}%
\pgfpathcurveto{\pgfqpoint{4.219725in}{5.280428in}}{\pgfqpoint{4.222997in}{5.272528in}}{\pgfqpoint{4.228821in}{5.266704in}}%
\pgfpathcurveto{\pgfqpoint{4.234645in}{5.260880in}}{\pgfqpoint{4.242545in}{5.257608in}}{\pgfqpoint{4.250781in}{5.257608in}}%
\pgfpathclose%
\pgfusepath{stroke,fill}%
\end{pgfscope}%
\begin{pgfscope}%
\pgfpathrectangle{\pgfqpoint{0.894063in}{3.540000in}}{\pgfqpoint{6.713438in}{2.060556in}} %
\pgfusepath{clip}%
\pgfsetbuttcap%
\pgfsetroundjoin%
\definecolor{currentfill}{rgb}{0.000000,0.000000,0.000000}%
\pgfsetfillcolor{currentfill}%
\pgfsetlinewidth{1.003750pt}%
\definecolor{currentstroke}{rgb}{0.000000,0.000000,0.000000}%
\pgfsetstrokecolor{currentstroke}%
\pgfsetdash{}{0pt}%
\pgfpathmoveto{\pgfqpoint{3.847975in}{5.257653in}}%
\pgfpathcurveto{\pgfqpoint{3.856211in}{5.257653in}}{\pgfqpoint{3.864111in}{5.260926in}}{\pgfqpoint{3.869935in}{5.266749in}}%
\pgfpathcurveto{\pgfqpoint{3.875759in}{5.272573in}}{\pgfqpoint{3.879031in}{5.280473in}}{\pgfqpoint{3.879031in}{5.288710in}}%
\pgfpathcurveto{\pgfqpoint{3.879031in}{5.296946in}}{\pgfqpoint{3.875759in}{5.304846in}}{\pgfqpoint{3.869935in}{5.310670in}}%
\pgfpathcurveto{\pgfqpoint{3.864111in}{5.316494in}}{\pgfqpoint{3.856211in}{5.319766in}}{\pgfqpoint{3.847975in}{5.319766in}}%
\pgfpathcurveto{\pgfqpoint{3.839739in}{5.319766in}}{\pgfqpoint{3.831839in}{5.316494in}}{\pgfqpoint{3.826015in}{5.310670in}}%
\pgfpathcurveto{\pgfqpoint{3.820191in}{5.304846in}}{\pgfqpoint{3.816919in}{5.296946in}}{\pgfqpoint{3.816919in}{5.288710in}}%
\pgfpathcurveto{\pgfqpoint{3.816919in}{5.280473in}}{\pgfqpoint{3.820191in}{5.272573in}}{\pgfqpoint{3.826015in}{5.266749in}}%
\pgfpathcurveto{\pgfqpoint{3.831839in}{5.260926in}}{\pgfqpoint{3.839739in}{5.257653in}}{\pgfqpoint{3.847975in}{5.257653in}}%
\pgfpathclose%
\pgfusepath{stroke,fill}%
\end{pgfscope}%
\begin{pgfscope}%
\pgfpathrectangle{\pgfqpoint{0.894063in}{3.540000in}}{\pgfqpoint{6.713438in}{2.060556in}} %
\pgfusepath{clip}%
\pgfsetbuttcap%
\pgfsetroundjoin%
\definecolor{currentfill}{rgb}{0.000000,0.000000,0.000000}%
\pgfsetfillcolor{currentfill}%
\pgfsetlinewidth{1.003750pt}%
\definecolor{currentstroke}{rgb}{0.000000,0.000000,0.000000}%
\pgfsetstrokecolor{currentstroke}%
\pgfsetdash{}{0pt}%
\pgfpathmoveto{\pgfqpoint{7.607500in}{5.256781in}}%
\pgfpathcurveto{\pgfqpoint{7.615736in}{5.256781in}}{\pgfqpoint{7.623636in}{5.260053in}}{\pgfqpoint{7.629460in}{5.265877in}}%
\pgfpathcurveto{\pgfqpoint{7.635284in}{5.271701in}}{\pgfqpoint{7.638556in}{5.279601in}}{\pgfqpoint{7.638556in}{5.287837in}}%
\pgfpathcurveto{\pgfqpoint{7.638556in}{5.296074in}}{\pgfqpoint{7.635284in}{5.303974in}}{\pgfqpoint{7.629460in}{5.309798in}}%
\pgfpathcurveto{\pgfqpoint{7.623636in}{5.315622in}}{\pgfqpoint{7.615736in}{5.318894in}}{\pgfqpoint{7.607500in}{5.318894in}}%
\pgfpathcurveto{\pgfqpoint{7.599264in}{5.318894in}}{\pgfqpoint{7.591364in}{5.315622in}}{\pgfqpoint{7.585540in}{5.309798in}}%
\pgfpathcurveto{\pgfqpoint{7.579716in}{5.303974in}}{\pgfqpoint{7.576444in}{5.296074in}}{\pgfqpoint{7.576444in}{5.287837in}}%
\pgfpathcurveto{\pgfqpoint{7.576444in}{5.279601in}}{\pgfqpoint{7.579716in}{5.271701in}}{\pgfqpoint{7.585540in}{5.265877in}}%
\pgfpathcurveto{\pgfqpoint{7.591364in}{5.260053in}}{\pgfqpoint{7.599264in}{5.256781in}}{\pgfqpoint{7.607500in}{5.256781in}}%
\pgfpathclose%
\pgfusepath{stroke,fill}%
\end{pgfscope}%
\begin{pgfscope}%
\pgfpathrectangle{\pgfqpoint{0.894063in}{3.540000in}}{\pgfqpoint{6.713438in}{2.060556in}} %
\pgfusepath{clip}%
\pgfsetbuttcap%
\pgfsetroundjoin%
\definecolor{currentfill}{rgb}{0.000000,0.000000,0.000000}%
\pgfsetfillcolor{currentfill}%
\pgfsetlinewidth{1.003750pt}%
\definecolor{currentstroke}{rgb}{0.000000,0.000000,0.000000}%
\pgfsetstrokecolor{currentstroke}%
\pgfsetdash{}{0pt}%
\pgfpathmoveto{\pgfqpoint{4.385050in}{5.257605in}}%
\pgfpathcurveto{\pgfqpoint{4.393286in}{5.257605in}}{\pgfqpoint{4.401186in}{5.260877in}}{\pgfqpoint{4.407010in}{5.266701in}}%
\pgfpathcurveto{\pgfqpoint{4.412834in}{5.272525in}}{\pgfqpoint{4.416106in}{5.280425in}}{\pgfqpoint{4.416106in}{5.288662in}}%
\pgfpathcurveto{\pgfqpoint{4.416106in}{5.296898in}}{\pgfqpoint{4.412834in}{5.304798in}}{\pgfqpoint{4.407010in}{5.310622in}}%
\pgfpathcurveto{\pgfqpoint{4.401186in}{5.316446in}}{\pgfqpoint{4.393286in}{5.319718in}}{\pgfqpoint{4.385050in}{5.319718in}}%
\pgfpathcurveto{\pgfqpoint{4.376814in}{5.319718in}}{\pgfqpoint{4.368914in}{5.316446in}}{\pgfqpoint{4.363090in}{5.310622in}}%
\pgfpathcurveto{\pgfqpoint{4.357266in}{5.304798in}}{\pgfqpoint{4.353994in}{5.296898in}}{\pgfqpoint{4.353994in}{5.288662in}}%
\pgfpathcurveto{\pgfqpoint{4.353994in}{5.280425in}}{\pgfqpoint{4.357266in}{5.272525in}}{\pgfqpoint{4.363090in}{5.266701in}}%
\pgfpathcurveto{\pgfqpoint{4.368914in}{5.260877in}}{\pgfqpoint{4.376814in}{5.257605in}}{\pgfqpoint{4.385050in}{5.257605in}}%
\pgfpathclose%
\pgfusepath{stroke,fill}%
\end{pgfscope}%
\begin{pgfscope}%
\pgfpathrectangle{\pgfqpoint{0.894063in}{3.540000in}}{\pgfqpoint{6.713438in}{2.060556in}} %
\pgfusepath{clip}%
\pgfsetbuttcap%
\pgfsetroundjoin%
\definecolor{currentfill}{rgb}{0.000000,0.000000,0.000000}%
\pgfsetfillcolor{currentfill}%
\pgfsetlinewidth{1.003750pt}%
\definecolor{currentstroke}{rgb}{0.000000,0.000000,0.000000}%
\pgfsetstrokecolor{currentstroke}%
\pgfsetdash{}{0pt}%
\pgfpathmoveto{\pgfqpoint{6.533350in}{5.257530in}}%
\pgfpathcurveto{\pgfqpoint{6.541586in}{5.257530in}}{\pgfqpoint{6.549486in}{5.260802in}}{\pgfqpoint{6.555310in}{5.266626in}}%
\pgfpathcurveto{\pgfqpoint{6.561134in}{5.272450in}}{\pgfqpoint{6.564406in}{5.280350in}}{\pgfqpoint{6.564406in}{5.288586in}}%
\pgfpathcurveto{\pgfqpoint{6.564406in}{5.296822in}}{\pgfqpoint{6.561134in}{5.304722in}}{\pgfqpoint{6.555310in}{5.310546in}}%
\pgfpathcurveto{\pgfqpoint{6.549486in}{5.316370in}}{\pgfqpoint{6.541586in}{5.319643in}}{\pgfqpoint{6.533350in}{5.319643in}}%
\pgfpathcurveto{\pgfqpoint{6.525114in}{5.319643in}}{\pgfqpoint{6.517214in}{5.316370in}}{\pgfqpoint{6.511390in}{5.310546in}}%
\pgfpathcurveto{\pgfqpoint{6.505566in}{5.304722in}}{\pgfqpoint{6.502294in}{5.296822in}}{\pgfqpoint{6.502294in}{5.288586in}}%
\pgfpathcurveto{\pgfqpoint{6.502294in}{5.280350in}}{\pgfqpoint{6.505566in}{5.272450in}}{\pgfqpoint{6.511390in}{5.266626in}}%
\pgfpathcurveto{\pgfqpoint{6.517214in}{5.260802in}}{\pgfqpoint{6.525114in}{5.257530in}}{\pgfqpoint{6.533350in}{5.257530in}}%
\pgfpathclose%
\pgfusepath{stroke,fill}%
\end{pgfscope}%
\begin{pgfscope}%
\pgfpathrectangle{\pgfqpoint{0.894063in}{3.540000in}}{\pgfqpoint{6.713438in}{2.060556in}} %
\pgfusepath{clip}%
\pgfsetbuttcap%
\pgfsetroundjoin%
\definecolor{currentfill}{rgb}{0.000000,0.000000,0.000000}%
\pgfsetfillcolor{currentfill}%
\pgfsetlinewidth{1.003750pt}%
\definecolor{currentstroke}{rgb}{0.000000,0.000000,0.000000}%
\pgfsetstrokecolor{currentstroke}%
\pgfsetdash{}{0pt}%
\pgfpathmoveto{\pgfqpoint{1.296869in}{5.258240in}}%
\pgfpathcurveto{\pgfqpoint{1.305105in}{5.258240in}}{\pgfqpoint{1.313005in}{5.261512in}}{\pgfqpoint{1.318829in}{5.267336in}}%
\pgfpathcurveto{\pgfqpoint{1.324653in}{5.273160in}}{\pgfqpoint{1.327925in}{5.281060in}}{\pgfqpoint{1.327925in}{5.289296in}}%
\pgfpathcurveto{\pgfqpoint{1.327925in}{5.297533in}}{\pgfqpoint{1.324653in}{5.305433in}}{\pgfqpoint{1.318829in}{5.311257in}}%
\pgfpathcurveto{\pgfqpoint{1.313005in}{5.317080in}}{\pgfqpoint{1.305105in}{5.320353in}}{\pgfqpoint{1.296869in}{5.320353in}}%
\pgfpathcurveto{\pgfqpoint{1.288632in}{5.320353in}}{\pgfqpoint{1.280732in}{5.317080in}}{\pgfqpoint{1.274908in}{5.311257in}}%
\pgfpathcurveto{\pgfqpoint{1.269085in}{5.305433in}}{\pgfqpoint{1.265812in}{5.297533in}}{\pgfqpoint{1.265812in}{5.289296in}}%
\pgfpathcurveto{\pgfqpoint{1.265812in}{5.281060in}}{\pgfqpoint{1.269085in}{5.273160in}}{\pgfqpoint{1.274908in}{5.267336in}}%
\pgfpathcurveto{\pgfqpoint{1.280732in}{5.261512in}}{\pgfqpoint{1.288632in}{5.258240in}}{\pgfqpoint{1.296869in}{5.258240in}}%
\pgfpathclose%
\pgfusepath{stroke,fill}%
\end{pgfscope}%
\begin{pgfscope}%
\pgfpathrectangle{\pgfqpoint{0.894063in}{3.540000in}}{\pgfqpoint{6.713438in}{2.060556in}} %
\pgfusepath{clip}%
\pgfsetbuttcap%
\pgfsetroundjoin%
\definecolor{currentfill}{rgb}{0.000000,0.000000,0.000000}%
\pgfsetfillcolor{currentfill}%
\pgfsetlinewidth{1.003750pt}%
\definecolor{currentstroke}{rgb}{0.000000,0.000000,0.000000}%
\pgfsetstrokecolor{currentstroke}%
\pgfsetdash{}{0pt}%
\pgfpathmoveto{\pgfqpoint{4.519319in}{5.257601in}}%
\pgfpathcurveto{\pgfqpoint{4.527555in}{5.257601in}}{\pgfqpoint{4.535455in}{5.260873in}}{\pgfqpoint{4.541279in}{5.266697in}}%
\pgfpathcurveto{\pgfqpoint{4.547103in}{5.272521in}}{\pgfqpoint{4.550375in}{5.280421in}}{\pgfqpoint{4.550375in}{5.288658in}}%
\pgfpathcurveto{\pgfqpoint{4.550375in}{5.296894in}}{\pgfqpoint{4.547103in}{5.304794in}}{\pgfqpoint{4.541279in}{5.310618in}}%
\pgfpathcurveto{\pgfqpoint{4.535455in}{5.316442in}}{\pgfqpoint{4.527555in}{5.319714in}}{\pgfqpoint{4.519319in}{5.319714in}}%
\pgfpathcurveto{\pgfqpoint{4.511082in}{5.319714in}}{\pgfqpoint{4.503182in}{5.316442in}}{\pgfqpoint{4.497358in}{5.310618in}}%
\pgfpathcurveto{\pgfqpoint{4.491535in}{5.304794in}}{\pgfqpoint{4.488262in}{5.296894in}}{\pgfqpoint{4.488262in}{5.288658in}}%
\pgfpathcurveto{\pgfqpoint{4.488262in}{5.280421in}}{\pgfqpoint{4.491535in}{5.272521in}}{\pgfqpoint{4.497358in}{5.266697in}}%
\pgfpathcurveto{\pgfqpoint{4.503182in}{5.260873in}}{\pgfqpoint{4.511082in}{5.257601in}}{\pgfqpoint{4.519319in}{5.257601in}}%
\pgfpathclose%
\pgfusepath{stroke,fill}%
\end{pgfscope}%
\begin{pgfscope}%
\pgfpathrectangle{\pgfqpoint{0.894063in}{3.540000in}}{\pgfqpoint{6.713438in}{2.060556in}} %
\pgfusepath{clip}%
\pgfsetbuttcap%
\pgfsetroundjoin%
\definecolor{currentfill}{rgb}{0.000000,0.000000,0.000000}%
\pgfsetfillcolor{currentfill}%
\pgfsetlinewidth{1.003750pt}%
\definecolor{currentstroke}{rgb}{0.000000,0.000000,0.000000}%
\pgfsetstrokecolor{currentstroke}%
\pgfsetdash{}{0pt}%
\pgfpathmoveto{\pgfqpoint{2.505288in}{5.257837in}}%
\pgfpathcurveto{\pgfqpoint{2.513524in}{5.257837in}}{\pgfqpoint{2.521424in}{5.261110in}}{\pgfqpoint{2.527248in}{5.266934in}}%
\pgfpathcurveto{\pgfqpoint{2.533072in}{5.272757in}}{\pgfqpoint{2.536344in}{5.280658in}}{\pgfqpoint{2.536344in}{5.288894in}}%
\pgfpathcurveto{\pgfqpoint{2.536344in}{5.297130in}}{\pgfqpoint{2.533072in}{5.305030in}}{\pgfqpoint{2.527248in}{5.310854in}}%
\pgfpathcurveto{\pgfqpoint{2.521424in}{5.316678in}}{\pgfqpoint{2.513524in}{5.319950in}}{\pgfqpoint{2.505288in}{5.319950in}}%
\pgfpathcurveto{\pgfqpoint{2.497051in}{5.319950in}}{\pgfqpoint{2.489151in}{5.316678in}}{\pgfqpoint{2.483327in}{5.310854in}}%
\pgfpathcurveto{\pgfqpoint{2.477503in}{5.305030in}}{\pgfqpoint{2.474231in}{5.297130in}}{\pgfqpoint{2.474231in}{5.288894in}}%
\pgfpathcurveto{\pgfqpoint{2.474231in}{5.280658in}}{\pgfqpoint{2.477503in}{5.272757in}}{\pgfqpoint{2.483327in}{5.266934in}}%
\pgfpathcurveto{\pgfqpoint{2.489151in}{5.261110in}}{\pgfqpoint{2.497051in}{5.257837in}}{\pgfqpoint{2.505288in}{5.257837in}}%
\pgfpathclose%
\pgfusepath{stroke,fill}%
\end{pgfscope}%
\begin{pgfscope}%
\pgfpathrectangle{\pgfqpoint{0.894063in}{3.540000in}}{\pgfqpoint{6.713438in}{2.060556in}} %
\pgfusepath{clip}%
\pgfsetbuttcap%
\pgfsetroundjoin%
\definecolor{currentfill}{rgb}{0.000000,0.000000,0.000000}%
\pgfsetfillcolor{currentfill}%
\pgfsetlinewidth{1.003750pt}%
\definecolor{currentstroke}{rgb}{0.000000,0.000000,0.000000}%
\pgfsetstrokecolor{currentstroke}%
\pgfsetdash{}{0pt}%
\pgfpathmoveto{\pgfqpoint{5.459200in}{5.257545in}}%
\pgfpathcurveto{\pgfqpoint{5.467436in}{5.257545in}}{\pgfqpoint{5.475336in}{5.260817in}}{\pgfqpoint{5.481160in}{5.266641in}}%
\pgfpathcurveto{\pgfqpoint{5.486984in}{5.272465in}}{\pgfqpoint{5.490256in}{5.280365in}}{\pgfqpoint{5.490256in}{5.288601in}}%
\pgfpathcurveto{\pgfqpoint{5.490256in}{5.296837in}}{\pgfqpoint{5.486984in}{5.304738in}}{\pgfqpoint{5.481160in}{5.310561in}}%
\pgfpathcurveto{\pgfqpoint{5.475336in}{5.316385in}}{\pgfqpoint{5.467436in}{5.319658in}}{\pgfqpoint{5.459200in}{5.319658in}}%
\pgfpathcurveto{\pgfqpoint{5.450964in}{5.319658in}}{\pgfqpoint{5.443064in}{5.316385in}}{\pgfqpoint{5.437240in}{5.310561in}}%
\pgfpathcurveto{\pgfqpoint{5.431416in}{5.304738in}}{\pgfqpoint{5.428144in}{5.296837in}}{\pgfqpoint{5.428144in}{5.288601in}}%
\pgfpathcurveto{\pgfqpoint{5.428144in}{5.280365in}}{\pgfqpoint{5.431416in}{5.272465in}}{\pgfqpoint{5.437240in}{5.266641in}}%
\pgfpathcurveto{\pgfqpoint{5.443064in}{5.260817in}}{\pgfqpoint{5.450964in}{5.257545in}}{\pgfqpoint{5.459200in}{5.257545in}}%
\pgfpathclose%
\pgfusepath{stroke,fill}%
\end{pgfscope}%
\begin{pgfscope}%
\pgfpathrectangle{\pgfqpoint{0.894063in}{3.540000in}}{\pgfqpoint{6.713438in}{2.060556in}} %
\pgfusepath{clip}%
\pgfsetbuttcap%
\pgfsetroundjoin%
\definecolor{currentfill}{rgb}{0.000000,0.000000,0.000000}%
\pgfsetfillcolor{currentfill}%
\pgfsetlinewidth{1.003750pt}%
\definecolor{currentstroke}{rgb}{0.000000,0.000000,0.000000}%
\pgfsetstrokecolor{currentstroke}%
\pgfsetdash{}{0pt}%
\pgfpathmoveto{\pgfqpoint{6.936156in}{5.256818in}}%
\pgfpathcurveto{\pgfqpoint{6.944393in}{5.256818in}}{\pgfqpoint{6.952293in}{5.260090in}}{\pgfqpoint{6.958117in}{5.265914in}}%
\pgfpathcurveto{\pgfqpoint{6.963940in}{5.271738in}}{\pgfqpoint{6.967213in}{5.279638in}}{\pgfqpoint{6.967213in}{5.287874in}}%
\pgfpathcurveto{\pgfqpoint{6.967213in}{5.296111in}}{\pgfqpoint{6.963940in}{5.304011in}}{\pgfqpoint{6.958117in}{5.309835in}}%
\pgfpathcurveto{\pgfqpoint{6.952293in}{5.315659in}}{\pgfqpoint{6.944393in}{5.318931in}}{\pgfqpoint{6.936156in}{5.318931in}}%
\pgfpathcurveto{\pgfqpoint{6.927920in}{5.318931in}}{\pgfqpoint{6.920020in}{5.315659in}}{\pgfqpoint{6.914196in}{5.309835in}}%
\pgfpathcurveto{\pgfqpoint{6.908372in}{5.304011in}}{\pgfqpoint{6.905100in}{5.296111in}}{\pgfqpoint{6.905100in}{5.287874in}}%
\pgfpathcurveto{\pgfqpoint{6.905100in}{5.279638in}}{\pgfqpoint{6.908372in}{5.271738in}}{\pgfqpoint{6.914196in}{5.265914in}}%
\pgfpathcurveto{\pgfqpoint{6.920020in}{5.260090in}}{\pgfqpoint{6.927920in}{5.256818in}}{\pgfqpoint{6.936156in}{5.256818in}}%
\pgfpathclose%
\pgfusepath{stroke,fill}%
\end{pgfscope}%
\begin{pgfscope}%
\pgfpathrectangle{\pgfqpoint{0.894063in}{3.540000in}}{\pgfqpoint{6.713438in}{2.060556in}} %
\pgfusepath{clip}%
\pgfsetbuttcap%
\pgfsetroundjoin%
\definecolor{currentfill}{rgb}{0.000000,0.000000,0.000000}%
\pgfsetfillcolor{currentfill}%
\pgfsetlinewidth{1.003750pt}%
\definecolor{currentstroke}{rgb}{0.000000,0.000000,0.000000}%
\pgfsetstrokecolor{currentstroke}%
\pgfsetdash{}{0pt}%
\pgfpathmoveto{\pgfqpoint{5.862006in}{5.257542in}}%
\pgfpathcurveto{\pgfqpoint{5.870243in}{5.257542in}}{\pgfqpoint{5.878143in}{5.260814in}}{\pgfqpoint{5.883967in}{5.266638in}}%
\pgfpathcurveto{\pgfqpoint{5.889790in}{5.272462in}}{\pgfqpoint{5.893063in}{5.280362in}}{\pgfqpoint{5.893063in}{5.288598in}}%
\pgfpathcurveto{\pgfqpoint{5.893063in}{5.296835in}}{\pgfqpoint{5.889790in}{5.304735in}}{\pgfqpoint{5.883967in}{5.310559in}}%
\pgfpathcurveto{\pgfqpoint{5.878143in}{5.316383in}}{\pgfqpoint{5.870243in}{5.319655in}}{\pgfqpoint{5.862006in}{5.319655in}}%
\pgfpathcurveto{\pgfqpoint{5.853770in}{5.319655in}}{\pgfqpoint{5.845870in}{5.316383in}}{\pgfqpoint{5.840046in}{5.310559in}}%
\pgfpathcurveto{\pgfqpoint{5.834222in}{5.304735in}}{\pgfqpoint{5.830950in}{5.296835in}}{\pgfqpoint{5.830950in}{5.288598in}}%
\pgfpathcurveto{\pgfqpoint{5.830950in}{5.280362in}}{\pgfqpoint{5.834222in}{5.272462in}}{\pgfqpoint{5.840046in}{5.266638in}}%
\pgfpathcurveto{\pgfqpoint{5.845870in}{5.260814in}}{\pgfqpoint{5.853770in}{5.257542in}}{\pgfqpoint{5.862006in}{5.257542in}}%
\pgfpathclose%
\pgfusepath{stroke,fill}%
\end{pgfscope}%
\begin{pgfscope}%
\pgfpathrectangle{\pgfqpoint{0.894063in}{3.540000in}}{\pgfqpoint{6.713438in}{2.060556in}} %
\pgfusepath{clip}%
\pgfsetbuttcap%
\pgfsetroundjoin%
\definecolor{currentfill}{rgb}{0.000000,0.000000,0.000000}%
\pgfsetfillcolor{currentfill}%
\pgfsetlinewidth{1.003750pt}%
\definecolor{currentstroke}{rgb}{0.000000,0.000000,0.000000}%
\pgfsetstrokecolor{currentstroke}%
\pgfsetdash{}{0pt}%
\pgfpathmoveto{\pgfqpoint{7.070425in}{5.256795in}}%
\pgfpathcurveto{\pgfqpoint{7.078661in}{5.256795in}}{\pgfqpoint{7.086561in}{5.260067in}}{\pgfqpoint{7.092385in}{5.265891in}}%
\pgfpathcurveto{\pgfqpoint{7.098209in}{5.271715in}}{\pgfqpoint{7.101481in}{5.279615in}}{\pgfqpoint{7.101481in}{5.287851in}}%
\pgfpathcurveto{\pgfqpoint{7.101481in}{5.296087in}}{\pgfqpoint{7.098209in}{5.303987in}}{\pgfqpoint{7.092385in}{5.309811in}}%
\pgfpathcurveto{\pgfqpoint{7.086561in}{5.315635in}}{\pgfqpoint{7.078661in}{5.318908in}}{\pgfqpoint{7.070425in}{5.318908in}}%
\pgfpathcurveto{\pgfqpoint{7.062189in}{5.318908in}}{\pgfqpoint{7.054289in}{5.315635in}}{\pgfqpoint{7.048465in}{5.309811in}}%
\pgfpathcurveto{\pgfqpoint{7.042641in}{5.303987in}}{\pgfqpoint{7.039369in}{5.296087in}}{\pgfqpoint{7.039369in}{5.287851in}}%
\pgfpathcurveto{\pgfqpoint{7.039369in}{5.279615in}}{\pgfqpoint{7.042641in}{5.271715in}}{\pgfqpoint{7.048465in}{5.265891in}}%
\pgfpathcurveto{\pgfqpoint{7.054289in}{5.260067in}}{\pgfqpoint{7.062189in}{5.256795in}}{\pgfqpoint{7.070425in}{5.256795in}}%
\pgfpathclose%
\pgfusepath{stroke,fill}%
\end{pgfscope}%
\begin{pgfscope}%
\pgfpathrectangle{\pgfqpoint{0.894063in}{3.540000in}}{\pgfqpoint{6.713438in}{2.060556in}} %
\pgfusepath{clip}%
\pgfsetbuttcap%
\pgfsetroundjoin%
\definecolor{currentfill}{rgb}{0.000000,0.000000,0.000000}%
\pgfsetfillcolor{currentfill}%
\pgfsetlinewidth{1.003750pt}%
\definecolor{currentstroke}{rgb}{0.000000,0.000000,0.000000}%
\pgfsetstrokecolor{currentstroke}%
\pgfsetdash{}{0pt}%
\pgfpathmoveto{\pgfqpoint{3.176631in}{5.257689in}}%
\pgfpathcurveto{\pgfqpoint{3.184868in}{5.257689in}}{\pgfqpoint{3.192768in}{5.260961in}}{\pgfqpoint{3.198592in}{5.266785in}}%
\pgfpathcurveto{\pgfqpoint{3.204415in}{5.272609in}}{\pgfqpoint{3.207688in}{5.280509in}}{\pgfqpoint{3.207688in}{5.288745in}}%
\pgfpathcurveto{\pgfqpoint{3.207688in}{5.296982in}}{\pgfqpoint{3.204415in}{5.304882in}}{\pgfqpoint{3.198592in}{5.310706in}}%
\pgfpathcurveto{\pgfqpoint{3.192768in}{5.316530in}}{\pgfqpoint{3.184868in}{5.319802in}}{\pgfqpoint{3.176631in}{5.319802in}}%
\pgfpathcurveto{\pgfqpoint{3.168395in}{5.319802in}}{\pgfqpoint{3.160495in}{5.316530in}}{\pgfqpoint{3.154671in}{5.310706in}}%
\pgfpathcurveto{\pgfqpoint{3.148847in}{5.304882in}}{\pgfqpoint{3.145575in}{5.296982in}}{\pgfqpoint{3.145575in}{5.288745in}}%
\pgfpathcurveto{\pgfqpoint{3.145575in}{5.280509in}}{\pgfqpoint{3.148847in}{5.272609in}}{\pgfqpoint{3.154671in}{5.266785in}}%
\pgfpathcurveto{\pgfqpoint{3.160495in}{5.260961in}}{\pgfqpoint{3.168395in}{5.257689in}}{\pgfqpoint{3.176631in}{5.257689in}}%
\pgfpathclose%
\pgfusepath{stroke,fill}%
\end{pgfscope}%
\begin{pgfscope}%
\pgfpathrectangle{\pgfqpoint{0.894063in}{3.540000in}}{\pgfqpoint{6.713438in}{2.060556in}} %
\pgfusepath{clip}%
\pgfsetbuttcap%
\pgfsetroundjoin%
\definecolor{currentfill}{rgb}{0.000000,0.000000,0.000000}%
\pgfsetfillcolor{currentfill}%
\pgfsetlinewidth{1.003750pt}%
\definecolor{currentstroke}{rgb}{0.000000,0.000000,0.000000}%
\pgfsetstrokecolor{currentstroke}%
\pgfsetdash{}{0pt}%
\pgfpathmoveto{\pgfqpoint{2.102481in}{5.258111in}}%
\pgfpathcurveto{\pgfqpoint{2.110718in}{5.258111in}}{\pgfqpoint{2.118618in}{5.261383in}}{\pgfqpoint{2.124442in}{5.267207in}}%
\pgfpathcurveto{\pgfqpoint{2.130265in}{5.273031in}}{\pgfqpoint{2.133538in}{5.280931in}}{\pgfqpoint{2.133538in}{5.289167in}}%
\pgfpathcurveto{\pgfqpoint{2.133538in}{5.297403in}}{\pgfqpoint{2.130265in}{5.305303in}}{\pgfqpoint{2.124442in}{5.311127in}}%
\pgfpathcurveto{\pgfqpoint{2.118618in}{5.316951in}}{\pgfqpoint{2.110718in}{5.320224in}}{\pgfqpoint{2.102481in}{5.320224in}}%
\pgfpathcurveto{\pgfqpoint{2.094245in}{5.320224in}}{\pgfqpoint{2.086345in}{5.316951in}}{\pgfqpoint{2.080521in}{5.311127in}}%
\pgfpathcurveto{\pgfqpoint{2.074697in}{5.305303in}}{\pgfqpoint{2.071425in}{5.297403in}}{\pgfqpoint{2.071425in}{5.289167in}}%
\pgfpathcurveto{\pgfqpoint{2.071425in}{5.280931in}}{\pgfqpoint{2.074697in}{5.273031in}}{\pgfqpoint{2.080521in}{5.267207in}}%
\pgfpathcurveto{\pgfqpoint{2.086345in}{5.261383in}}{\pgfqpoint{2.094245in}{5.258111in}}{\pgfqpoint{2.102481in}{5.258111in}}%
\pgfpathclose%
\pgfusepath{stroke,fill}%
\end{pgfscope}%
\begin{pgfscope}%
\pgfpathrectangle{\pgfqpoint{0.894063in}{3.540000in}}{\pgfqpoint{6.713438in}{2.060556in}} %
\pgfusepath{clip}%
\pgfsetbuttcap%
\pgfsetroundjoin%
\definecolor{currentfill}{rgb}{0.000000,0.000000,0.000000}%
\pgfsetfillcolor{currentfill}%
\pgfsetlinewidth{1.003750pt}%
\definecolor{currentstroke}{rgb}{0.000000,0.000000,0.000000}%
\pgfsetstrokecolor{currentstroke}%
\pgfsetdash{}{0pt}%
\pgfpathmoveto{\pgfqpoint{1.968213in}{5.258159in}}%
\pgfpathcurveto{\pgfqpoint{1.976449in}{5.258159in}}{\pgfqpoint{1.984349in}{5.261431in}}{\pgfqpoint{1.990173in}{5.267255in}}%
\pgfpathcurveto{\pgfqpoint{1.995997in}{5.273079in}}{\pgfqpoint{1.999269in}{5.280979in}}{\pgfqpoint{1.999269in}{5.289215in}}%
\pgfpathcurveto{\pgfqpoint{1.999269in}{5.297452in}}{\pgfqpoint{1.995997in}{5.305352in}}{\pgfqpoint{1.990173in}{5.311175in}}%
\pgfpathcurveto{\pgfqpoint{1.984349in}{5.316999in}}{\pgfqpoint{1.976449in}{5.320272in}}{\pgfqpoint{1.968213in}{5.320272in}}%
\pgfpathcurveto{\pgfqpoint{1.959976in}{5.320272in}}{\pgfqpoint{1.952076in}{5.316999in}}{\pgfqpoint{1.946252in}{5.311175in}}%
\pgfpathcurveto{\pgfqpoint{1.940428in}{5.305352in}}{\pgfqpoint{1.937156in}{5.297452in}}{\pgfqpoint{1.937156in}{5.289215in}}%
\pgfpathcurveto{\pgfqpoint{1.937156in}{5.280979in}}{\pgfqpoint{1.940428in}{5.273079in}}{\pgfqpoint{1.946252in}{5.267255in}}%
\pgfpathcurveto{\pgfqpoint{1.952076in}{5.261431in}}{\pgfqpoint{1.959976in}{5.258159in}}{\pgfqpoint{1.968213in}{5.258159in}}%
\pgfpathclose%
\pgfusepath{stroke,fill}%
\end{pgfscope}%
\begin{pgfscope}%
\pgfpathrectangle{\pgfqpoint{0.894063in}{3.540000in}}{\pgfqpoint{6.713438in}{2.060556in}} %
\pgfusepath{clip}%
\pgfsetbuttcap%
\pgfsetroundjoin%
\definecolor{currentfill}{rgb}{0.000000,0.000000,0.000000}%
\pgfsetfillcolor{currentfill}%
\pgfsetlinewidth{1.003750pt}%
\definecolor{currentstroke}{rgb}{0.000000,0.000000,0.000000}%
\pgfsetstrokecolor{currentstroke}%
\pgfsetdash{}{0pt}%
\pgfpathmoveto{\pgfqpoint{3.310900in}{5.257686in}}%
\pgfpathcurveto{\pgfqpoint{3.319136in}{5.257686in}}{\pgfqpoint{3.327036in}{5.260958in}}{\pgfqpoint{3.332860in}{5.266782in}}%
\pgfpathcurveto{\pgfqpoint{3.338684in}{5.272606in}}{\pgfqpoint{3.341956in}{5.280506in}}{\pgfqpoint{3.341956in}{5.288743in}}%
\pgfpathcurveto{\pgfqpoint{3.341956in}{5.296979in}}{\pgfqpoint{3.338684in}{5.304879in}}{\pgfqpoint{3.332860in}{5.310703in}}%
\pgfpathcurveto{\pgfqpoint{3.327036in}{5.316527in}}{\pgfqpoint{3.319136in}{5.319799in}}{\pgfqpoint{3.310900in}{5.319799in}}%
\pgfpathcurveto{\pgfqpoint{3.302664in}{5.319799in}}{\pgfqpoint{3.294764in}{5.316527in}}{\pgfqpoint{3.288940in}{5.310703in}}%
\pgfpathcurveto{\pgfqpoint{3.283116in}{5.304879in}}{\pgfqpoint{3.279844in}{5.296979in}}{\pgfqpoint{3.279844in}{5.288743in}}%
\pgfpathcurveto{\pgfqpoint{3.279844in}{5.280506in}}{\pgfqpoint{3.283116in}{5.272606in}}{\pgfqpoint{3.288940in}{5.266782in}}%
\pgfpathcurveto{\pgfqpoint{3.294764in}{5.260958in}}{\pgfqpoint{3.302664in}{5.257686in}}{\pgfqpoint{3.310900in}{5.257686in}}%
\pgfpathclose%
\pgfusepath{stroke,fill}%
\end{pgfscope}%
\begin{pgfscope}%
\pgfpathrectangle{\pgfqpoint{0.894063in}{3.540000in}}{\pgfqpoint{6.713438in}{2.060556in}} %
\pgfusepath{clip}%
\pgfsetbuttcap%
\pgfsetroundjoin%
\definecolor{currentfill}{rgb}{0.000000,0.000000,0.000000}%
\pgfsetfillcolor{currentfill}%
\pgfsetlinewidth{1.003750pt}%
\definecolor{currentstroke}{rgb}{0.000000,0.000000,0.000000}%
\pgfsetstrokecolor{currentstroke}%
\pgfsetdash{}{0pt}%
\pgfpathmoveto{\pgfqpoint{5.593469in}{5.257545in}}%
\pgfpathcurveto{\pgfqpoint{5.601705in}{5.257545in}}{\pgfqpoint{5.609605in}{5.260817in}}{\pgfqpoint{5.615429in}{5.266641in}}%
\pgfpathcurveto{\pgfqpoint{5.621253in}{5.272465in}}{\pgfqpoint{5.624525in}{5.280365in}}{\pgfqpoint{5.624525in}{5.288601in}}%
\pgfpathcurveto{\pgfqpoint{5.624525in}{5.296837in}}{\pgfqpoint{5.621253in}{5.304738in}}{\pgfqpoint{5.615429in}{5.310561in}}%
\pgfpathcurveto{\pgfqpoint{5.609605in}{5.316385in}}{\pgfqpoint{5.601705in}{5.319658in}}{\pgfqpoint{5.593469in}{5.319658in}}%
\pgfpathcurveto{\pgfqpoint{5.585232in}{5.319658in}}{\pgfqpoint{5.577332in}{5.316385in}}{\pgfqpoint{5.571508in}{5.310561in}}%
\pgfpathcurveto{\pgfqpoint{5.565685in}{5.304738in}}{\pgfqpoint{5.562412in}{5.296837in}}{\pgfqpoint{5.562412in}{5.288601in}}%
\pgfpathcurveto{\pgfqpoint{5.562412in}{5.280365in}}{\pgfqpoint{5.565685in}{5.272465in}}{\pgfqpoint{5.571508in}{5.266641in}}%
\pgfpathcurveto{\pgfqpoint{5.577332in}{5.260817in}}{\pgfqpoint{5.585232in}{5.257545in}}{\pgfqpoint{5.593469in}{5.257545in}}%
\pgfpathclose%
\pgfusepath{stroke,fill}%
\end{pgfscope}%
\begin{pgfscope}%
\pgfpathrectangle{\pgfqpoint{0.894063in}{3.540000in}}{\pgfqpoint{6.713438in}{2.060556in}} %
\pgfusepath{clip}%
\pgfsetbuttcap%
\pgfsetroundjoin%
\definecolor{currentfill}{rgb}{0.000000,0.000000,0.000000}%
\pgfsetfillcolor{currentfill}%
\pgfsetlinewidth{1.003750pt}%
\definecolor{currentstroke}{rgb}{0.000000,0.000000,0.000000}%
\pgfsetstrokecolor{currentstroke}%
\pgfsetdash{}{0pt}%
\pgfpathmoveto{\pgfqpoint{3.042363in}{5.257699in}}%
\pgfpathcurveto{\pgfqpoint{3.050599in}{5.257699in}}{\pgfqpoint{3.058499in}{5.260971in}}{\pgfqpoint{3.064323in}{5.266795in}}%
\pgfpathcurveto{\pgfqpoint{3.070147in}{5.272619in}}{\pgfqpoint{3.073419in}{5.280519in}}{\pgfqpoint{3.073419in}{5.288755in}}%
\pgfpathcurveto{\pgfqpoint{3.073419in}{5.296991in}}{\pgfqpoint{3.070147in}{5.304891in}}{\pgfqpoint{3.064323in}{5.310715in}}%
\pgfpathcurveto{\pgfqpoint{3.058499in}{5.316539in}}{\pgfqpoint{3.050599in}{5.319812in}}{\pgfqpoint{3.042363in}{5.319812in}}%
\pgfpathcurveto{\pgfqpoint{3.034126in}{5.319812in}}{\pgfqpoint{3.026226in}{5.316539in}}{\pgfqpoint{3.020402in}{5.310715in}}%
\pgfpathcurveto{\pgfqpoint{3.014578in}{5.304891in}}{\pgfqpoint{3.011306in}{5.296991in}}{\pgfqpoint{3.011306in}{5.288755in}}%
\pgfpathcurveto{\pgfqpoint{3.011306in}{5.280519in}}{\pgfqpoint{3.014578in}{5.272619in}}{\pgfqpoint{3.020402in}{5.266795in}}%
\pgfpathcurveto{\pgfqpoint{3.026226in}{5.260971in}}{\pgfqpoint{3.034126in}{5.257699in}}{\pgfqpoint{3.042363in}{5.257699in}}%
\pgfpathclose%
\pgfusepath{stroke,fill}%
\end{pgfscope}%
\begin{pgfscope}%
\pgfpathrectangle{\pgfqpoint{0.894063in}{3.540000in}}{\pgfqpoint{6.713438in}{2.060556in}} %
\pgfusepath{clip}%
\pgfsetbuttcap%
\pgfsetroundjoin%
\definecolor{currentfill}{rgb}{0.000000,0.000000,0.000000}%
\pgfsetfillcolor{currentfill}%
\pgfsetlinewidth{1.003750pt}%
\definecolor{currentstroke}{rgb}{0.000000,0.000000,0.000000}%
\pgfsetstrokecolor{currentstroke}%
\pgfsetdash{}{0pt}%
\pgfpathmoveto{\pgfqpoint{5.190663in}{5.257560in}}%
\pgfpathcurveto{\pgfqpoint{5.198899in}{5.257560in}}{\pgfqpoint{5.206799in}{5.260832in}}{\pgfqpoint{5.212623in}{5.266656in}}%
\pgfpathcurveto{\pgfqpoint{5.218447in}{5.272480in}}{\pgfqpoint{5.221719in}{5.280380in}}{\pgfqpoint{5.221719in}{5.288616in}}%
\pgfpathcurveto{\pgfqpoint{5.221719in}{5.296853in}}{\pgfqpoint{5.218447in}{5.304753in}}{\pgfqpoint{5.212623in}{5.310577in}}%
\pgfpathcurveto{\pgfqpoint{5.206799in}{5.316400in}}{\pgfqpoint{5.198899in}{5.319673in}}{\pgfqpoint{5.190663in}{5.319673in}}%
\pgfpathcurveto{\pgfqpoint{5.182426in}{5.319673in}}{\pgfqpoint{5.174526in}{5.316400in}}{\pgfqpoint{5.168702in}{5.310577in}}%
\pgfpathcurveto{\pgfqpoint{5.162878in}{5.304753in}}{\pgfqpoint{5.159606in}{5.296853in}}{\pgfqpoint{5.159606in}{5.288616in}}%
\pgfpathcurveto{\pgfqpoint{5.159606in}{5.280380in}}{\pgfqpoint{5.162878in}{5.272480in}}{\pgfqpoint{5.168702in}{5.266656in}}%
\pgfpathcurveto{\pgfqpoint{5.174526in}{5.260832in}}{\pgfqpoint{5.182426in}{5.257560in}}{\pgfqpoint{5.190663in}{5.257560in}}%
\pgfpathclose%
\pgfusepath{stroke,fill}%
\end{pgfscope}%
\begin{pgfscope}%
\pgfpathrectangle{\pgfqpoint{0.894063in}{3.540000in}}{\pgfqpoint{6.713438in}{2.060556in}} %
\pgfusepath{clip}%
\pgfsetbuttcap%
\pgfsetroundjoin%
\definecolor{currentfill}{rgb}{0.000000,0.000000,0.000000}%
\pgfsetfillcolor{currentfill}%
\pgfsetlinewidth{1.003750pt}%
\definecolor{currentstroke}{rgb}{0.000000,0.000000,0.000000}%
\pgfsetstrokecolor{currentstroke}%
\pgfsetdash{}{0pt}%
\pgfpathmoveto{\pgfqpoint{6.801888in}{5.257525in}}%
\pgfpathcurveto{\pgfqpoint{6.810124in}{5.257525in}}{\pgfqpoint{6.818024in}{5.260798in}}{\pgfqpoint{6.823848in}{5.266622in}}%
\pgfpathcurveto{\pgfqpoint{6.829672in}{5.272446in}}{\pgfqpoint{6.832944in}{5.280346in}}{\pgfqpoint{6.832944in}{5.288582in}}%
\pgfpathcurveto{\pgfqpoint{6.832944in}{5.296818in}}{\pgfqpoint{6.829672in}{5.304718in}}{\pgfqpoint{6.823848in}{5.310542in}}%
\pgfpathcurveto{\pgfqpoint{6.818024in}{5.316366in}}{\pgfqpoint{6.810124in}{5.319638in}}{\pgfqpoint{6.801888in}{5.319638in}}%
\pgfpathcurveto{\pgfqpoint{6.793651in}{5.319638in}}{\pgfqpoint{6.785751in}{5.316366in}}{\pgfqpoint{6.779927in}{5.310542in}}%
\pgfpathcurveto{\pgfqpoint{6.774103in}{5.304718in}}{\pgfqpoint{6.770831in}{5.296818in}}{\pgfqpoint{6.770831in}{5.288582in}}%
\pgfpathcurveto{\pgfqpoint{6.770831in}{5.280346in}}{\pgfqpoint{6.774103in}{5.272446in}}{\pgfqpoint{6.779927in}{5.266622in}}%
\pgfpathcurveto{\pgfqpoint{6.785751in}{5.260798in}}{\pgfqpoint{6.793651in}{5.257525in}}{\pgfqpoint{6.801888in}{5.257525in}}%
\pgfpathclose%
\pgfusepath{stroke,fill}%
\end{pgfscope}%
\begin{pgfscope}%
\pgfpathrectangle{\pgfqpoint{0.894063in}{3.540000in}}{\pgfqpoint{6.713438in}{2.060556in}} %
\pgfusepath{clip}%
\pgfsetbuttcap%
\pgfsetroundjoin%
\definecolor{currentfill}{rgb}{0.000000,0.000000,0.000000}%
\pgfsetfillcolor{currentfill}%
\pgfsetlinewidth{1.003750pt}%
\definecolor{currentstroke}{rgb}{0.000000,0.000000,0.000000}%
\pgfsetstrokecolor{currentstroke}%
\pgfsetdash{}{0pt}%
\pgfpathmoveto{\pgfqpoint{3.579438in}{5.257671in}}%
\pgfpathcurveto{\pgfqpoint{3.587674in}{5.257671in}}{\pgfqpoint{3.595574in}{5.260943in}}{\pgfqpoint{3.601398in}{5.266767in}}%
\pgfpathcurveto{\pgfqpoint{3.607222in}{5.272591in}}{\pgfqpoint{3.610494in}{5.280491in}}{\pgfqpoint{3.610494in}{5.288728in}}%
\pgfpathcurveto{\pgfqpoint{3.610494in}{5.296964in}}{\pgfqpoint{3.607222in}{5.304864in}}{\pgfqpoint{3.601398in}{5.310688in}}%
\pgfpathcurveto{\pgfqpoint{3.595574in}{5.316512in}}{\pgfqpoint{3.587674in}{5.319784in}}{\pgfqpoint{3.579438in}{5.319784in}}%
\pgfpathcurveto{\pgfqpoint{3.571201in}{5.319784in}}{\pgfqpoint{3.563301in}{5.316512in}}{\pgfqpoint{3.557477in}{5.310688in}}%
\pgfpathcurveto{\pgfqpoint{3.551653in}{5.304864in}}{\pgfqpoint{3.548381in}{5.296964in}}{\pgfqpoint{3.548381in}{5.288728in}}%
\pgfpathcurveto{\pgfqpoint{3.548381in}{5.280491in}}{\pgfqpoint{3.551653in}{5.272591in}}{\pgfqpoint{3.557477in}{5.266767in}}%
\pgfpathcurveto{\pgfqpoint{3.563301in}{5.260943in}}{\pgfqpoint{3.571201in}{5.257671in}}{\pgfqpoint{3.579438in}{5.257671in}}%
\pgfpathclose%
\pgfusepath{stroke,fill}%
\end{pgfscope}%
\begin{pgfscope}%
\pgfpathrectangle{\pgfqpoint{0.894063in}{3.540000in}}{\pgfqpoint{6.713438in}{2.060556in}} %
\pgfusepath{clip}%
\pgfsetbuttcap%
\pgfsetroundjoin%
\definecolor{currentfill}{rgb}{0.000000,0.000000,0.000000}%
\pgfsetfillcolor{currentfill}%
\pgfsetlinewidth{1.003750pt}%
\definecolor{currentstroke}{rgb}{0.000000,0.000000,0.000000}%
\pgfsetstrokecolor{currentstroke}%
\pgfsetdash{}{0pt}%
\pgfpathmoveto{\pgfqpoint{2.371019in}{5.257840in}}%
\pgfpathcurveto{\pgfqpoint{2.379255in}{5.257840in}}{\pgfqpoint{2.387155in}{5.261112in}}{\pgfqpoint{2.392979in}{5.266936in}}%
\pgfpathcurveto{\pgfqpoint{2.398803in}{5.272760in}}{\pgfqpoint{2.402075in}{5.280660in}}{\pgfqpoint{2.402075in}{5.288897in}}%
\pgfpathcurveto{\pgfqpoint{2.402075in}{5.297133in}}{\pgfqpoint{2.398803in}{5.305033in}}{\pgfqpoint{2.392979in}{5.310857in}}%
\pgfpathcurveto{\pgfqpoint{2.387155in}{5.316681in}}{\pgfqpoint{2.379255in}{5.319953in}}{\pgfqpoint{2.371019in}{5.319953in}}%
\pgfpathcurveto{\pgfqpoint{2.362782in}{5.319953in}}{\pgfqpoint{2.354882in}{5.316681in}}{\pgfqpoint{2.349058in}{5.310857in}}%
\pgfpathcurveto{\pgfqpoint{2.343235in}{5.305033in}}{\pgfqpoint{2.339962in}{5.297133in}}{\pgfqpoint{2.339962in}{5.288897in}}%
\pgfpathcurveto{\pgfqpoint{2.339962in}{5.280660in}}{\pgfqpoint{2.343235in}{5.272760in}}{\pgfqpoint{2.349058in}{5.266936in}}%
\pgfpathcurveto{\pgfqpoint{2.354882in}{5.261112in}}{\pgfqpoint{2.362782in}{5.257840in}}{\pgfqpoint{2.371019in}{5.257840in}}%
\pgfpathclose%
\pgfusepath{stroke,fill}%
\end{pgfscope}%
\begin{pgfscope}%
\pgfpathrectangle{\pgfqpoint{0.894063in}{3.540000in}}{\pgfqpoint{6.713438in}{2.060556in}} %
\pgfusepath{clip}%
\pgfsetbuttcap%
\pgfsetroundjoin%
\definecolor{currentfill}{rgb}{0.000000,0.000000,0.000000}%
\pgfsetfillcolor{currentfill}%
\pgfsetlinewidth{1.003750pt}%
\definecolor{currentstroke}{rgb}{0.000000,0.000000,0.000000}%
\pgfsetstrokecolor{currentstroke}%
\pgfsetdash{}{0pt}%
\pgfpathmoveto{\pgfqpoint{3.982244in}{5.257623in}}%
\pgfpathcurveto{\pgfqpoint{3.990480in}{5.257623in}}{\pgfqpoint{3.998380in}{5.260895in}}{\pgfqpoint{4.004204in}{5.266719in}}%
\pgfpathcurveto{\pgfqpoint{4.010028in}{5.272543in}}{\pgfqpoint{4.013300in}{5.280443in}}{\pgfqpoint{4.013300in}{5.288679in}}%
\pgfpathcurveto{\pgfqpoint{4.013300in}{5.296916in}}{\pgfqpoint{4.010028in}{5.304816in}}{\pgfqpoint{4.004204in}{5.310640in}}%
\pgfpathcurveto{\pgfqpoint{3.998380in}{5.316464in}}{\pgfqpoint{3.990480in}{5.319736in}}{\pgfqpoint{3.982244in}{5.319736in}}%
\pgfpathcurveto{\pgfqpoint{3.974007in}{5.319736in}}{\pgfqpoint{3.966107in}{5.316464in}}{\pgfqpoint{3.960283in}{5.310640in}}%
\pgfpathcurveto{\pgfqpoint{3.954460in}{5.304816in}}{\pgfqpoint{3.951187in}{5.296916in}}{\pgfqpoint{3.951187in}{5.288679in}}%
\pgfpathcurveto{\pgfqpoint{3.951187in}{5.280443in}}{\pgfqpoint{3.954460in}{5.272543in}}{\pgfqpoint{3.960283in}{5.266719in}}%
\pgfpathcurveto{\pgfqpoint{3.966107in}{5.260895in}}{\pgfqpoint{3.974007in}{5.257623in}}{\pgfqpoint{3.982244in}{5.257623in}}%
\pgfpathclose%
\pgfusepath{stroke,fill}%
\end{pgfscope}%
\begin{pgfscope}%
\pgfpathrectangle{\pgfqpoint{0.894063in}{3.540000in}}{\pgfqpoint{6.713438in}{2.060556in}} %
\pgfusepath{clip}%
\pgfsetbuttcap%
\pgfsetroundjoin%
\definecolor{currentfill}{rgb}{0.000000,0.000000,0.000000}%
\pgfsetfillcolor{currentfill}%
\pgfsetlinewidth{1.003750pt}%
\definecolor{currentstroke}{rgb}{0.000000,0.000000,0.000000}%
\pgfsetstrokecolor{currentstroke}%
\pgfsetdash{}{0pt}%
\pgfpathmoveto{\pgfqpoint{4.653588in}{5.257589in}}%
\pgfpathcurveto{\pgfqpoint{4.661824in}{5.257589in}}{\pgfqpoint{4.669724in}{5.260861in}}{\pgfqpoint{4.675548in}{5.266685in}}%
\pgfpathcurveto{\pgfqpoint{4.681372in}{5.272509in}}{\pgfqpoint{4.684644in}{5.280409in}}{\pgfqpoint{4.684644in}{5.288645in}}%
\pgfpathcurveto{\pgfqpoint{4.684644in}{5.296881in}}{\pgfqpoint{4.681372in}{5.304781in}}{\pgfqpoint{4.675548in}{5.310605in}}%
\pgfpathcurveto{\pgfqpoint{4.669724in}{5.316429in}}{\pgfqpoint{4.661824in}{5.319702in}}{\pgfqpoint{4.653588in}{5.319702in}}%
\pgfpathcurveto{\pgfqpoint{4.645351in}{5.319702in}}{\pgfqpoint{4.637451in}{5.316429in}}{\pgfqpoint{4.631627in}{5.310605in}}%
\pgfpathcurveto{\pgfqpoint{4.625803in}{5.304781in}}{\pgfqpoint{4.622531in}{5.296881in}}{\pgfqpoint{4.622531in}{5.288645in}}%
\pgfpathcurveto{\pgfqpoint{4.622531in}{5.280409in}}{\pgfqpoint{4.625803in}{5.272509in}}{\pgfqpoint{4.631627in}{5.266685in}}%
\pgfpathcurveto{\pgfqpoint{4.637451in}{5.260861in}}{\pgfqpoint{4.645351in}{5.257589in}}{\pgfqpoint{4.653588in}{5.257589in}}%
\pgfpathclose%
\pgfusepath{stroke,fill}%
\end{pgfscope}%
\begin{pgfscope}%
\pgfpathrectangle{\pgfqpoint{0.894063in}{3.540000in}}{\pgfqpoint{6.713438in}{2.060556in}} %
\pgfusepath{clip}%
\pgfsetbuttcap%
\pgfsetroundjoin%
\definecolor{currentfill}{rgb}{0.000000,0.000000,0.000000}%
\pgfsetfillcolor{currentfill}%
\pgfsetlinewidth{1.003750pt}%
\definecolor{currentstroke}{rgb}{0.000000,0.000000,0.000000}%
\pgfsetstrokecolor{currentstroke}%
\pgfsetdash{}{0pt}%
\pgfpathmoveto{\pgfqpoint{3.713706in}{5.257666in}}%
\pgfpathcurveto{\pgfqpoint{3.721943in}{5.257666in}}{\pgfqpoint{3.729843in}{5.260938in}}{\pgfqpoint{3.735667in}{5.266762in}}%
\pgfpathcurveto{\pgfqpoint{3.741490in}{5.272586in}}{\pgfqpoint{3.744763in}{5.280486in}}{\pgfqpoint{3.744763in}{5.288722in}}%
\pgfpathcurveto{\pgfqpoint{3.744763in}{5.296958in}}{\pgfqpoint{3.741490in}{5.304858in}}{\pgfqpoint{3.735667in}{5.310682in}}%
\pgfpathcurveto{\pgfqpoint{3.729843in}{5.316506in}}{\pgfqpoint{3.721943in}{5.319779in}}{\pgfqpoint{3.713706in}{5.319779in}}%
\pgfpathcurveto{\pgfqpoint{3.705470in}{5.319779in}}{\pgfqpoint{3.697570in}{5.316506in}}{\pgfqpoint{3.691746in}{5.310682in}}%
\pgfpathcurveto{\pgfqpoint{3.685922in}{5.304858in}}{\pgfqpoint{3.682650in}{5.296958in}}{\pgfqpoint{3.682650in}{5.288722in}}%
\pgfpathcurveto{\pgfqpoint{3.682650in}{5.280486in}}{\pgfqpoint{3.685922in}{5.272586in}}{\pgfqpoint{3.691746in}{5.266762in}}%
\pgfpathcurveto{\pgfqpoint{3.697570in}{5.260938in}}{\pgfqpoint{3.705470in}{5.257666in}}{\pgfqpoint{3.713706in}{5.257666in}}%
\pgfpathclose%
\pgfusepath{stroke,fill}%
\end{pgfscope}%
\begin{pgfscope}%
\pgfpathrectangle{\pgfqpoint{0.894063in}{3.540000in}}{\pgfqpoint{6.713438in}{2.060556in}} %
\pgfusepath{clip}%
\pgfsetbuttcap%
\pgfsetroundjoin%
\definecolor{currentfill}{rgb}{0.000000,0.000000,0.000000}%
\pgfsetfillcolor{currentfill}%
\pgfsetlinewidth{1.003750pt}%
\definecolor{currentstroke}{rgb}{0.000000,0.000000,0.000000}%
\pgfsetstrokecolor{currentstroke}%
\pgfsetdash{}{0pt}%
\pgfpathmoveto{\pgfqpoint{2.236750in}{5.257844in}}%
\pgfpathcurveto{\pgfqpoint{2.244986in}{5.257844in}}{\pgfqpoint{2.252886in}{5.261116in}}{\pgfqpoint{2.258710in}{5.266940in}}%
\pgfpathcurveto{\pgfqpoint{2.264534in}{5.272764in}}{\pgfqpoint{2.267806in}{5.280664in}}{\pgfqpoint{2.267806in}{5.288901in}}%
\pgfpathcurveto{\pgfqpoint{2.267806in}{5.297137in}}{\pgfqpoint{2.264534in}{5.305037in}}{\pgfqpoint{2.258710in}{5.310861in}}%
\pgfpathcurveto{\pgfqpoint{2.252886in}{5.316685in}}{\pgfqpoint{2.244986in}{5.319957in}}{\pgfqpoint{2.236750in}{5.319957in}}%
\pgfpathcurveto{\pgfqpoint{2.228514in}{5.319957in}}{\pgfqpoint{2.220614in}{5.316685in}}{\pgfqpoint{2.214790in}{5.310861in}}%
\pgfpathcurveto{\pgfqpoint{2.208966in}{5.305037in}}{\pgfqpoint{2.205694in}{5.297137in}}{\pgfqpoint{2.205694in}{5.288901in}}%
\pgfpathcurveto{\pgfqpoint{2.205694in}{5.280664in}}{\pgfqpoint{2.208966in}{5.272764in}}{\pgfqpoint{2.214790in}{5.266940in}}%
\pgfpathcurveto{\pgfqpoint{2.220614in}{5.261116in}}{\pgfqpoint{2.228514in}{5.257844in}}{\pgfqpoint{2.236750in}{5.257844in}}%
\pgfpathclose%
\pgfusepath{stroke,fill}%
\end{pgfscope}%
\begin{pgfscope}%
\pgfpathrectangle{\pgfqpoint{0.894063in}{3.540000in}}{\pgfqpoint{6.713438in}{2.060556in}} %
\pgfusepath{clip}%
\pgfsetbuttcap%
\pgfsetroundjoin%
\definecolor{currentfill}{rgb}{0.000000,0.750000,0.750000}%
\pgfsetfillcolor{currentfill}%
\pgfsetlinewidth{1.003750pt}%
\definecolor{currentstroke}{rgb}{0.000000,0.750000,0.750000}%
\pgfsetstrokecolor{currentstroke}%
\pgfsetdash{}{0pt}%
\pgfpathmoveto{\pgfqpoint{6.667619in}{5.256815in}}%
\pgfpathcurveto{\pgfqpoint{6.675855in}{5.256815in}}{\pgfqpoint{6.683755in}{5.260088in}}{\pgfqpoint{6.689579in}{5.265911in}}%
\pgfpathcurveto{\pgfqpoint{6.695403in}{5.271735in}}{\pgfqpoint{6.698675in}{5.279635in}}{\pgfqpoint{6.698675in}{5.287872in}}%
\pgfpathcurveto{\pgfqpoint{6.698675in}{5.296108in}}{\pgfqpoint{6.695403in}{5.304008in}}{\pgfqpoint{6.689579in}{5.309832in}}%
\pgfpathcurveto{\pgfqpoint{6.683755in}{5.315656in}}{\pgfqpoint{6.675855in}{5.318928in}}{\pgfqpoint{6.667619in}{5.318928in}}%
\pgfpathcurveto{\pgfqpoint{6.659382in}{5.318928in}}{\pgfqpoint{6.651482in}{5.315656in}}{\pgfqpoint{6.645658in}{5.309832in}}%
\pgfpathcurveto{\pgfqpoint{6.639835in}{5.304008in}}{\pgfqpoint{6.636562in}{5.296108in}}{\pgfqpoint{6.636562in}{5.287872in}}%
\pgfpathcurveto{\pgfqpoint{6.636562in}{5.279635in}}{\pgfqpoint{6.639835in}{5.271735in}}{\pgfqpoint{6.645658in}{5.265911in}}%
\pgfpathcurveto{\pgfqpoint{6.651482in}{5.260088in}}{\pgfqpoint{6.659382in}{5.256815in}}{\pgfqpoint{6.667619in}{5.256815in}}%
\pgfpathclose%
\pgfusepath{stroke,fill}%
\end{pgfscope}%
\begin{pgfscope}%
\pgfpathrectangle{\pgfqpoint{0.894063in}{3.540000in}}{\pgfqpoint{6.713438in}{2.060556in}} %
\pgfusepath{clip}%
\pgfsetbuttcap%
\pgfsetroundjoin%
\definecolor{currentfill}{rgb}{0.000000,0.750000,0.750000}%
\pgfsetfillcolor{currentfill}%
\pgfsetlinewidth{1.003750pt}%
\definecolor{currentstroke}{rgb}{0.000000,0.750000,0.750000}%
\pgfsetstrokecolor{currentstroke}%
\pgfsetdash{}{0pt}%
\pgfpathmoveto{\pgfqpoint{2.639556in}{5.257705in}}%
\pgfpathcurveto{\pgfqpoint{2.647793in}{5.257705in}}{\pgfqpoint{2.655693in}{5.260978in}}{\pgfqpoint{2.661517in}{5.266802in}}%
\pgfpathcurveto{\pgfqpoint{2.667340in}{5.272626in}}{\pgfqpoint{2.670613in}{5.280526in}}{\pgfqpoint{2.670613in}{5.288762in}}%
\pgfpathcurveto{\pgfqpoint{2.670613in}{5.296998in}}{\pgfqpoint{2.667340in}{5.304898in}}{\pgfqpoint{2.661517in}{5.310722in}}%
\pgfpathcurveto{\pgfqpoint{2.655693in}{5.316546in}}{\pgfqpoint{2.647793in}{5.319818in}}{\pgfqpoint{2.639556in}{5.319818in}}%
\pgfpathcurveto{\pgfqpoint{2.631320in}{5.319818in}}{\pgfqpoint{2.623420in}{5.316546in}}{\pgfqpoint{2.617596in}{5.310722in}}%
\pgfpathcurveto{\pgfqpoint{2.611772in}{5.304898in}}{\pgfqpoint{2.608500in}{5.296998in}}{\pgfqpoint{2.608500in}{5.288762in}}%
\pgfpathcurveto{\pgfqpoint{2.608500in}{5.280526in}}{\pgfqpoint{2.611772in}{5.272626in}}{\pgfqpoint{2.617596in}{5.266802in}}%
\pgfpathcurveto{\pgfqpoint{2.623420in}{5.260978in}}{\pgfqpoint{2.631320in}{5.257705in}}{\pgfqpoint{2.639556in}{5.257705in}}%
\pgfpathclose%
\pgfusepath{stroke,fill}%
\end{pgfscope}%
\begin{pgfscope}%
\pgfpathrectangle{\pgfqpoint{0.894063in}{3.540000in}}{\pgfqpoint{6.713438in}{2.060556in}} %
\pgfusepath{clip}%
\pgfsetbuttcap%
\pgfsetroundjoin%
\definecolor{currentfill}{rgb}{0.000000,0.750000,0.750000}%
\pgfsetfillcolor{currentfill}%
\pgfsetlinewidth{1.003750pt}%
\definecolor{currentstroke}{rgb}{0.000000,0.750000,0.750000}%
\pgfsetstrokecolor{currentstroke}%
\pgfsetdash{}{0pt}%
\pgfpathmoveto{\pgfqpoint{1.699675in}{5.258183in}}%
\pgfpathcurveto{\pgfqpoint{1.707911in}{5.258183in}}{\pgfqpoint{1.715811in}{5.261456in}}{\pgfqpoint{1.721635in}{5.267280in}}%
\pgfpathcurveto{\pgfqpoint{1.727459in}{5.273104in}}{\pgfqpoint{1.730731in}{5.281004in}}{\pgfqpoint{1.730731in}{5.289240in}}%
\pgfpathcurveto{\pgfqpoint{1.730731in}{5.297476in}}{\pgfqpoint{1.727459in}{5.305376in}}{\pgfqpoint{1.721635in}{5.311200in}}%
\pgfpathcurveto{\pgfqpoint{1.715811in}{5.317024in}}{\pgfqpoint{1.707911in}{5.320296in}}{\pgfqpoint{1.699675in}{5.320296in}}%
\pgfpathcurveto{\pgfqpoint{1.691439in}{5.320296in}}{\pgfqpoint{1.683539in}{5.317024in}}{\pgfqpoint{1.677715in}{5.311200in}}%
\pgfpathcurveto{\pgfqpoint{1.671891in}{5.305376in}}{\pgfqpoint{1.668619in}{5.297476in}}{\pgfqpoint{1.668619in}{5.289240in}}%
\pgfpathcurveto{\pgfqpoint{1.668619in}{5.281004in}}{\pgfqpoint{1.671891in}{5.273104in}}{\pgfqpoint{1.677715in}{5.267280in}}%
\pgfpathcurveto{\pgfqpoint{1.683539in}{5.261456in}}{\pgfqpoint{1.691439in}{5.258183in}}{\pgfqpoint{1.699675in}{5.258183in}}%
\pgfpathclose%
\pgfusepath{stroke,fill}%
\end{pgfscope}%
\begin{pgfscope}%
\pgfpathrectangle{\pgfqpoint{0.894063in}{3.540000in}}{\pgfqpoint{6.713438in}{2.060556in}} %
\pgfusepath{clip}%
\pgfsetbuttcap%
\pgfsetroundjoin%
\definecolor{currentfill}{rgb}{0.000000,0.750000,0.750000}%
\pgfsetfillcolor{currentfill}%
\pgfsetlinewidth{1.003750pt}%
\definecolor{currentstroke}{rgb}{0.000000,0.750000,0.750000}%
\pgfsetstrokecolor{currentstroke}%
\pgfsetdash{}{0pt}%
\pgfpathmoveto{\pgfqpoint{1.162600in}{5.258328in}}%
\pgfpathcurveto{\pgfqpoint{1.170836in}{5.258328in}}{\pgfqpoint{1.178736in}{5.261600in}}{\pgfqpoint{1.184560in}{5.267424in}}%
\pgfpathcurveto{\pgfqpoint{1.190384in}{5.273248in}}{\pgfqpoint{1.193656in}{5.281148in}}{\pgfqpoint{1.193656in}{5.289384in}}%
\pgfpathcurveto{\pgfqpoint{1.193656in}{5.297620in}}{\pgfqpoint{1.190384in}{5.305521in}}{\pgfqpoint{1.184560in}{5.311344in}}%
\pgfpathcurveto{\pgfqpoint{1.178736in}{5.317168in}}{\pgfqpoint{1.170836in}{5.320441in}}{\pgfqpoint{1.162600in}{5.320441in}}%
\pgfpathcurveto{\pgfqpoint{1.154364in}{5.320441in}}{\pgfqpoint{1.146464in}{5.317168in}}{\pgfqpoint{1.140640in}{5.311344in}}%
\pgfpathcurveto{\pgfqpoint{1.134816in}{5.305521in}}{\pgfqpoint{1.131544in}{5.297620in}}{\pgfqpoint{1.131544in}{5.289384in}}%
\pgfpathcurveto{\pgfqpoint{1.131544in}{5.281148in}}{\pgfqpoint{1.134816in}{5.273248in}}{\pgfqpoint{1.140640in}{5.267424in}}%
\pgfpathcurveto{\pgfqpoint{1.146464in}{5.261600in}}{\pgfqpoint{1.154364in}{5.258328in}}{\pgfqpoint{1.162600in}{5.258328in}}%
\pgfpathclose%
\pgfusepath{stroke,fill}%
\end{pgfscope}%
\begin{pgfscope}%
\pgfpathrectangle{\pgfqpoint{0.894063in}{3.540000in}}{\pgfqpoint{6.713438in}{2.060556in}} %
\pgfusepath{clip}%
\pgfsetbuttcap%
\pgfsetroundjoin%
\definecolor{currentfill}{rgb}{0.000000,0.750000,0.750000}%
\pgfsetfillcolor{currentfill}%
\pgfsetlinewidth{1.003750pt}%
\definecolor{currentstroke}{rgb}{0.000000,0.750000,0.750000}%
\pgfsetstrokecolor{currentstroke}%
\pgfsetdash{}{0pt}%
\pgfpathmoveto{\pgfqpoint{1.833944in}{5.258168in}}%
\pgfpathcurveto{\pgfqpoint{1.842180in}{5.258168in}}{\pgfqpoint{1.850080in}{5.261441in}}{\pgfqpoint{1.855904in}{5.267265in}}%
\pgfpathcurveto{\pgfqpoint{1.861728in}{5.273089in}}{\pgfqpoint{1.865000in}{5.280989in}}{\pgfqpoint{1.865000in}{5.289225in}}%
\pgfpathcurveto{\pgfqpoint{1.865000in}{5.297461in}}{\pgfqpoint{1.861728in}{5.305361in}}{\pgfqpoint{1.855904in}{5.311185in}}%
\pgfpathcurveto{\pgfqpoint{1.850080in}{5.317009in}}{\pgfqpoint{1.842180in}{5.320281in}}{\pgfqpoint{1.833944in}{5.320281in}}%
\pgfpathcurveto{\pgfqpoint{1.825707in}{5.320281in}}{\pgfqpoint{1.817807in}{5.317009in}}{\pgfqpoint{1.811983in}{5.311185in}}%
\pgfpathcurveto{\pgfqpoint{1.806160in}{5.305361in}}{\pgfqpoint{1.802887in}{5.297461in}}{\pgfqpoint{1.802887in}{5.289225in}}%
\pgfpathcurveto{\pgfqpoint{1.802887in}{5.280989in}}{\pgfqpoint{1.806160in}{5.273089in}}{\pgfqpoint{1.811983in}{5.267265in}}%
\pgfpathcurveto{\pgfqpoint{1.817807in}{5.261441in}}{\pgfqpoint{1.825707in}{5.258168in}}{\pgfqpoint{1.833944in}{5.258168in}}%
\pgfpathclose%
\pgfusepath{stroke,fill}%
\end{pgfscope}%
\begin{pgfscope}%
\pgfpathrectangle{\pgfqpoint{0.894063in}{3.540000in}}{\pgfqpoint{6.713438in}{2.060556in}} %
\pgfusepath{clip}%
\pgfsetbuttcap%
\pgfsetroundjoin%
\definecolor{currentfill}{rgb}{0.000000,0.750000,0.750000}%
\pgfsetfillcolor{currentfill}%
\pgfsetlinewidth{1.003750pt}%
\definecolor{currentstroke}{rgb}{0.000000,0.750000,0.750000}%
\pgfsetstrokecolor{currentstroke}%
\pgfsetdash{}{0pt}%
\pgfpathmoveto{\pgfqpoint{5.996275in}{5.257532in}}%
\pgfpathcurveto{\pgfqpoint{6.004511in}{5.257532in}}{\pgfqpoint{6.012411in}{5.260805in}}{\pgfqpoint{6.018235in}{5.266629in}}%
\pgfpathcurveto{\pgfqpoint{6.024059in}{5.272452in}}{\pgfqpoint{6.027331in}{5.280353in}}{\pgfqpoint{6.027331in}{5.288589in}}%
\pgfpathcurveto{\pgfqpoint{6.027331in}{5.296825in}}{\pgfqpoint{6.024059in}{5.304725in}}{\pgfqpoint{6.018235in}{5.310549in}}%
\pgfpathcurveto{\pgfqpoint{6.012411in}{5.316373in}}{\pgfqpoint{6.004511in}{5.319645in}}{\pgfqpoint{5.996275in}{5.319645in}}%
\pgfpathcurveto{\pgfqpoint{5.988039in}{5.319645in}}{\pgfqpoint{5.980139in}{5.316373in}}{\pgfqpoint{5.974315in}{5.310549in}}%
\pgfpathcurveto{\pgfqpoint{5.968491in}{5.304725in}}{\pgfqpoint{5.965219in}{5.296825in}}{\pgfqpoint{5.965219in}{5.288589in}}%
\pgfpathcurveto{\pgfqpoint{5.965219in}{5.280353in}}{\pgfqpoint{5.968491in}{5.272452in}}{\pgfqpoint{5.974315in}{5.266629in}}%
\pgfpathcurveto{\pgfqpoint{5.980139in}{5.260805in}}{\pgfqpoint{5.988039in}{5.257532in}}{\pgfqpoint{5.996275in}{5.257532in}}%
\pgfpathclose%
\pgfusepath{stroke,fill}%
\end{pgfscope}%
\begin{pgfscope}%
\pgfpathrectangle{\pgfqpoint{0.894063in}{3.540000in}}{\pgfqpoint{6.713438in}{2.060556in}} %
\pgfusepath{clip}%
\pgfsetbuttcap%
\pgfsetroundjoin%
\definecolor{currentfill}{rgb}{0.000000,0.750000,0.750000}%
\pgfsetfillcolor{currentfill}%
\pgfsetlinewidth{1.003750pt}%
\definecolor{currentstroke}{rgb}{0.000000,0.750000,0.750000}%
\pgfsetstrokecolor{currentstroke}%
\pgfsetdash{}{0pt}%
\pgfpathmoveto{\pgfqpoint{6.399081in}{5.257525in}}%
\pgfpathcurveto{\pgfqpoint{6.407318in}{5.257525in}}{\pgfqpoint{6.415218in}{5.260798in}}{\pgfqpoint{6.421042in}{5.266622in}}%
\pgfpathcurveto{\pgfqpoint{6.426865in}{5.272446in}}{\pgfqpoint{6.430138in}{5.280346in}}{\pgfqpoint{6.430138in}{5.288582in}}%
\pgfpathcurveto{\pgfqpoint{6.430138in}{5.296818in}}{\pgfqpoint{6.426865in}{5.304718in}}{\pgfqpoint{6.421042in}{5.310542in}}%
\pgfpathcurveto{\pgfqpoint{6.415218in}{5.316366in}}{\pgfqpoint{6.407318in}{5.319638in}}{\pgfqpoint{6.399081in}{5.319638in}}%
\pgfpathcurveto{\pgfqpoint{6.390845in}{5.319638in}}{\pgfqpoint{6.382945in}{5.316366in}}{\pgfqpoint{6.377121in}{5.310542in}}%
\pgfpathcurveto{\pgfqpoint{6.371297in}{5.304718in}}{\pgfqpoint{6.368025in}{5.296818in}}{\pgfqpoint{6.368025in}{5.288582in}}%
\pgfpathcurveto{\pgfqpoint{6.368025in}{5.280346in}}{\pgfqpoint{6.371297in}{5.272446in}}{\pgfqpoint{6.377121in}{5.266622in}}%
\pgfpathcurveto{\pgfqpoint{6.382945in}{5.260798in}}{\pgfqpoint{6.390845in}{5.257525in}}{\pgfqpoint{6.399081in}{5.257525in}}%
\pgfpathclose%
\pgfusepath{stroke,fill}%
\end{pgfscope}%
\begin{pgfscope}%
\pgfpathrectangle{\pgfqpoint{0.894063in}{3.540000in}}{\pgfqpoint{6.713438in}{2.060556in}} %
\pgfusepath{clip}%
\pgfsetbuttcap%
\pgfsetroundjoin%
\definecolor{currentfill}{rgb}{0.000000,0.750000,0.750000}%
\pgfsetfillcolor{currentfill}%
\pgfsetlinewidth{1.003750pt}%
\definecolor{currentstroke}{rgb}{0.000000,0.750000,0.750000}%
\pgfsetstrokecolor{currentstroke}%
\pgfsetdash{}{0pt}%
\pgfpathmoveto{\pgfqpoint{4.787856in}{5.257572in}}%
\pgfpathcurveto{\pgfqpoint{4.796093in}{5.257572in}}{\pgfqpoint{4.803993in}{5.260844in}}{\pgfqpoint{4.809817in}{5.266668in}}%
\pgfpathcurveto{\pgfqpoint{4.815640in}{5.272492in}}{\pgfqpoint{4.818913in}{5.280392in}}{\pgfqpoint{4.818913in}{5.288629in}}%
\pgfpathcurveto{\pgfqpoint{4.818913in}{5.296865in}}{\pgfqpoint{4.815640in}{5.304765in}}{\pgfqpoint{4.809817in}{5.310589in}}%
\pgfpathcurveto{\pgfqpoint{4.803993in}{5.316413in}}{\pgfqpoint{4.796093in}{5.319685in}}{\pgfqpoint{4.787856in}{5.319685in}}%
\pgfpathcurveto{\pgfqpoint{4.779620in}{5.319685in}}{\pgfqpoint{4.771720in}{5.316413in}}{\pgfqpoint{4.765896in}{5.310589in}}%
\pgfpathcurveto{\pgfqpoint{4.760072in}{5.304765in}}{\pgfqpoint{4.756800in}{5.296865in}}{\pgfqpoint{4.756800in}{5.288629in}}%
\pgfpathcurveto{\pgfqpoint{4.756800in}{5.280392in}}{\pgfqpoint{4.760072in}{5.272492in}}{\pgfqpoint{4.765896in}{5.266668in}}%
\pgfpathcurveto{\pgfqpoint{4.771720in}{5.260844in}}{\pgfqpoint{4.779620in}{5.257572in}}{\pgfqpoint{4.787856in}{5.257572in}}%
\pgfpathclose%
\pgfusepath{stroke,fill}%
\end{pgfscope}%
\begin{pgfscope}%
\pgfpathrectangle{\pgfqpoint{0.894063in}{3.540000in}}{\pgfqpoint{6.713438in}{2.060556in}} %
\pgfusepath{clip}%
\pgfsetbuttcap%
\pgfsetroundjoin%
\definecolor{currentfill}{rgb}{0.000000,0.750000,0.750000}%
\pgfsetfillcolor{currentfill}%
\pgfsetlinewidth{1.003750pt}%
\definecolor{currentstroke}{rgb}{0.000000,0.750000,0.750000}%
\pgfsetstrokecolor{currentstroke}%
\pgfsetdash{}{0pt}%
\pgfpathmoveto{\pgfqpoint{4.922125in}{5.257561in}}%
\pgfpathcurveto{\pgfqpoint{4.930361in}{5.257561in}}{\pgfqpoint{4.938261in}{5.260833in}}{\pgfqpoint{4.944085in}{5.266657in}}%
\pgfpathcurveto{\pgfqpoint{4.949909in}{5.272481in}}{\pgfqpoint{4.953181in}{5.280381in}}{\pgfqpoint{4.953181in}{5.288618in}}%
\pgfpathcurveto{\pgfqpoint{4.953181in}{5.296854in}}{\pgfqpoint{4.949909in}{5.304754in}}{\pgfqpoint{4.944085in}{5.310578in}}%
\pgfpathcurveto{\pgfqpoint{4.938261in}{5.316402in}}{\pgfqpoint{4.930361in}{5.319674in}}{\pgfqpoint{4.922125in}{5.319674in}}%
\pgfpathcurveto{\pgfqpoint{4.913889in}{5.319674in}}{\pgfqpoint{4.905989in}{5.316402in}}{\pgfqpoint{4.900165in}{5.310578in}}%
\pgfpathcurveto{\pgfqpoint{4.894341in}{5.304754in}}{\pgfqpoint{4.891069in}{5.296854in}}{\pgfqpoint{4.891069in}{5.288618in}}%
\pgfpathcurveto{\pgfqpoint{4.891069in}{5.280381in}}{\pgfqpoint{4.894341in}{5.272481in}}{\pgfqpoint{4.900165in}{5.266657in}}%
\pgfpathcurveto{\pgfqpoint{4.905989in}{5.260833in}}{\pgfqpoint{4.913889in}{5.257561in}}{\pgfqpoint{4.922125in}{5.257561in}}%
\pgfpathclose%
\pgfusepath{stroke,fill}%
\end{pgfscope}%
\begin{pgfscope}%
\pgfpathrectangle{\pgfqpoint{0.894063in}{3.540000in}}{\pgfqpoint{6.713438in}{2.060556in}} %
\pgfusepath{clip}%
\pgfsetbuttcap%
\pgfsetroundjoin%
\definecolor{currentfill}{rgb}{0.000000,0.750000,0.750000}%
\pgfsetfillcolor{currentfill}%
\pgfsetlinewidth{1.003750pt}%
\definecolor{currentstroke}{rgb}{0.000000,0.750000,0.750000}%
\pgfsetstrokecolor{currentstroke}%
\pgfsetdash{}{0pt}%
\pgfpathmoveto{\pgfqpoint{6.130544in}{5.257531in}}%
\pgfpathcurveto{\pgfqpoint{6.138780in}{5.257531in}}{\pgfqpoint{6.146680in}{5.260803in}}{\pgfqpoint{6.152504in}{5.266627in}}%
\pgfpathcurveto{\pgfqpoint{6.158328in}{5.272451in}}{\pgfqpoint{6.161600in}{5.280351in}}{\pgfqpoint{6.161600in}{5.288587in}}%
\pgfpathcurveto{\pgfqpoint{6.161600in}{5.296824in}}{\pgfqpoint{6.158328in}{5.304724in}}{\pgfqpoint{6.152504in}{5.310548in}}%
\pgfpathcurveto{\pgfqpoint{6.146680in}{5.316372in}}{\pgfqpoint{6.138780in}{5.319644in}}{\pgfqpoint{6.130544in}{5.319644in}}%
\pgfpathcurveto{\pgfqpoint{6.122307in}{5.319644in}}{\pgfqpoint{6.114407in}{5.316372in}}{\pgfqpoint{6.108583in}{5.310548in}}%
\pgfpathcurveto{\pgfqpoint{6.102760in}{5.304724in}}{\pgfqpoint{6.099487in}{5.296824in}}{\pgfqpoint{6.099487in}{5.288587in}}%
\pgfpathcurveto{\pgfqpoint{6.099487in}{5.280351in}}{\pgfqpoint{6.102760in}{5.272451in}}{\pgfqpoint{6.108583in}{5.266627in}}%
\pgfpathcurveto{\pgfqpoint{6.114407in}{5.260803in}}{\pgfqpoint{6.122307in}{5.257531in}}{\pgfqpoint{6.130544in}{5.257531in}}%
\pgfpathclose%
\pgfusepath{stroke,fill}%
\end{pgfscope}%
\begin{pgfscope}%
\pgfpathrectangle{\pgfqpoint{0.894063in}{3.540000in}}{\pgfqpoint{6.713438in}{2.060556in}} %
\pgfusepath{clip}%
\pgfsetbuttcap%
\pgfsetroundjoin%
\definecolor{currentfill}{rgb}{0.000000,0.750000,0.750000}%
\pgfsetfillcolor{currentfill}%
\pgfsetlinewidth{1.003750pt}%
\definecolor{currentstroke}{rgb}{0.000000,0.750000,0.750000}%
\pgfsetstrokecolor{currentstroke}%
\pgfsetdash{}{0pt}%
\pgfpathmoveto{\pgfqpoint{5.727738in}{5.257542in}}%
\pgfpathcurveto{\pgfqpoint{5.735974in}{5.257542in}}{\pgfqpoint{5.743874in}{5.260814in}}{\pgfqpoint{5.749698in}{5.266638in}}%
\pgfpathcurveto{\pgfqpoint{5.755522in}{5.272462in}}{\pgfqpoint{5.758794in}{5.280362in}}{\pgfqpoint{5.758794in}{5.288598in}}%
\pgfpathcurveto{\pgfqpoint{5.758794in}{5.296835in}}{\pgfqpoint{5.755522in}{5.304735in}}{\pgfqpoint{5.749698in}{5.310559in}}%
\pgfpathcurveto{\pgfqpoint{5.743874in}{5.316383in}}{\pgfqpoint{5.735974in}{5.319655in}}{\pgfqpoint{5.727738in}{5.319655in}}%
\pgfpathcurveto{\pgfqpoint{5.719501in}{5.319655in}}{\pgfqpoint{5.711601in}{5.316383in}}{\pgfqpoint{5.705777in}{5.310559in}}%
\pgfpathcurveto{\pgfqpoint{5.699953in}{5.304735in}}{\pgfqpoint{5.696681in}{5.296835in}}{\pgfqpoint{5.696681in}{5.288598in}}%
\pgfpathcurveto{\pgfqpoint{5.696681in}{5.280362in}}{\pgfqpoint{5.699953in}{5.272462in}}{\pgfqpoint{5.705777in}{5.266638in}}%
\pgfpathcurveto{\pgfqpoint{5.711601in}{5.260814in}}{\pgfqpoint{5.719501in}{5.257542in}}{\pgfqpoint{5.727738in}{5.257542in}}%
\pgfpathclose%
\pgfusepath{stroke,fill}%
\end{pgfscope}%
\begin{pgfscope}%
\pgfpathrectangle{\pgfqpoint{0.894063in}{3.540000in}}{\pgfqpoint{6.713438in}{2.060556in}} %
\pgfusepath{clip}%
\pgfsetbuttcap%
\pgfsetroundjoin%
\definecolor{currentfill}{rgb}{0.000000,0.750000,0.750000}%
\pgfsetfillcolor{currentfill}%
\pgfsetlinewidth{1.003750pt}%
\definecolor{currentstroke}{rgb}{0.000000,0.750000,0.750000}%
\pgfsetstrokecolor{currentstroke}%
\pgfsetdash{}{0pt}%
\pgfpathmoveto{\pgfqpoint{1.028331in}{5.258428in}}%
\pgfpathcurveto{\pgfqpoint{1.036568in}{5.258428in}}{\pgfqpoint{1.044468in}{5.261700in}}{\pgfqpoint{1.050292in}{5.267524in}}%
\pgfpathcurveto{\pgfqpoint{1.056115in}{5.273348in}}{\pgfqpoint{1.059388in}{5.281248in}}{\pgfqpoint{1.059388in}{5.289484in}}%
\pgfpathcurveto{\pgfqpoint{1.059388in}{5.297721in}}{\pgfqpoint{1.056115in}{5.305621in}}{\pgfqpoint{1.050292in}{5.311445in}}%
\pgfpathcurveto{\pgfqpoint{1.044468in}{5.317269in}}{\pgfqpoint{1.036568in}{5.320541in}}{\pgfqpoint{1.028331in}{5.320541in}}%
\pgfpathcurveto{\pgfqpoint{1.020095in}{5.320541in}}{\pgfqpoint{1.012195in}{5.317269in}}{\pgfqpoint{1.006371in}{5.311445in}}%
\pgfpathcurveto{\pgfqpoint{1.000547in}{5.305621in}}{\pgfqpoint{0.997275in}{5.297721in}}{\pgfqpoint{0.997275in}{5.289484in}}%
\pgfpathcurveto{\pgfqpoint{0.997275in}{5.281248in}}{\pgfqpoint{1.000547in}{5.273348in}}{\pgfqpoint{1.006371in}{5.267524in}}%
\pgfpathcurveto{\pgfqpoint{1.012195in}{5.261700in}}{\pgfqpoint{1.020095in}{5.258428in}}{\pgfqpoint{1.028331in}{5.258428in}}%
\pgfpathclose%
\pgfusepath{stroke,fill}%
\end{pgfscope}%
\begin{pgfscope}%
\pgfpathrectangle{\pgfqpoint{0.894063in}{3.540000in}}{\pgfqpoint{6.713438in}{2.060556in}} %
\pgfusepath{clip}%
\pgfsetbuttcap%
\pgfsetroundjoin%
\definecolor{currentfill}{rgb}{0.000000,0.750000,0.750000}%
\pgfsetfillcolor{currentfill}%
\pgfsetlinewidth{1.003750pt}%
\definecolor{currentstroke}{rgb}{0.000000,0.750000,0.750000}%
\pgfsetstrokecolor{currentstroke}%
\pgfsetdash{}{0pt}%
\pgfpathmoveto{\pgfqpoint{5.324931in}{5.257545in}}%
\pgfpathcurveto{\pgfqpoint{5.333168in}{5.257545in}}{\pgfqpoint{5.341068in}{5.260817in}}{\pgfqpoint{5.346892in}{5.266641in}}%
\pgfpathcurveto{\pgfqpoint{5.352715in}{5.272465in}}{\pgfqpoint{5.355988in}{5.280365in}}{\pgfqpoint{5.355988in}{5.288601in}}%
\pgfpathcurveto{\pgfqpoint{5.355988in}{5.296837in}}{\pgfqpoint{5.352715in}{5.304738in}}{\pgfqpoint{5.346892in}{5.310561in}}%
\pgfpathcurveto{\pgfqpoint{5.341068in}{5.316385in}}{\pgfqpoint{5.333168in}{5.319658in}}{\pgfqpoint{5.324931in}{5.319658in}}%
\pgfpathcurveto{\pgfqpoint{5.316695in}{5.319658in}}{\pgfqpoint{5.308795in}{5.316385in}}{\pgfqpoint{5.302971in}{5.310561in}}%
\pgfpathcurveto{\pgfqpoint{5.297147in}{5.304738in}}{\pgfqpoint{5.293875in}{5.296837in}}{\pgfqpoint{5.293875in}{5.288601in}}%
\pgfpathcurveto{\pgfqpoint{5.293875in}{5.280365in}}{\pgfqpoint{5.297147in}{5.272465in}}{\pgfqpoint{5.302971in}{5.266641in}}%
\pgfpathcurveto{\pgfqpoint{5.308795in}{5.260817in}}{\pgfqpoint{5.316695in}{5.257545in}}{\pgfqpoint{5.324931in}{5.257545in}}%
\pgfpathclose%
\pgfusepath{stroke,fill}%
\end{pgfscope}%
\begin{pgfscope}%
\pgfpathrectangle{\pgfqpoint{0.894063in}{3.540000in}}{\pgfqpoint{6.713438in}{2.060556in}} %
\pgfusepath{clip}%
\pgfsetbuttcap%
\pgfsetroundjoin%
\definecolor{currentfill}{rgb}{0.000000,0.750000,0.750000}%
\pgfsetfillcolor{currentfill}%
\pgfsetlinewidth{1.003750pt}%
\definecolor{currentstroke}{rgb}{0.000000,0.750000,0.750000}%
\pgfsetstrokecolor{currentstroke}%
\pgfsetdash{}{0pt}%
\pgfpathmoveto{\pgfqpoint{7.338963in}{5.256078in}}%
\pgfpathcurveto{\pgfqpoint{7.347199in}{5.256078in}}{\pgfqpoint{7.355099in}{5.259350in}}{\pgfqpoint{7.360923in}{5.265174in}}%
\pgfpathcurveto{\pgfqpoint{7.366747in}{5.270998in}}{\pgfqpoint{7.370019in}{5.278898in}}{\pgfqpoint{7.370019in}{5.287134in}}%
\pgfpathcurveto{\pgfqpoint{7.370019in}{5.295370in}}{\pgfqpoint{7.366747in}{5.303270in}}{\pgfqpoint{7.360923in}{5.309094in}}%
\pgfpathcurveto{\pgfqpoint{7.355099in}{5.314918in}}{\pgfqpoint{7.347199in}{5.318191in}}{\pgfqpoint{7.338963in}{5.318191in}}%
\pgfpathcurveto{\pgfqpoint{7.330726in}{5.318191in}}{\pgfqpoint{7.322826in}{5.314918in}}{\pgfqpoint{7.317002in}{5.309094in}}%
\pgfpathcurveto{\pgfqpoint{7.311178in}{5.303270in}}{\pgfqpoint{7.307906in}{5.295370in}}{\pgfqpoint{7.307906in}{5.287134in}}%
\pgfpathcurveto{\pgfqpoint{7.307906in}{5.278898in}}{\pgfqpoint{7.311178in}{5.270998in}}{\pgfqpoint{7.317002in}{5.265174in}}%
\pgfpathcurveto{\pgfqpoint{7.322826in}{5.259350in}}{\pgfqpoint{7.330726in}{5.256078in}}{\pgfqpoint{7.338963in}{5.256078in}}%
\pgfpathclose%
\pgfusepath{stroke,fill}%
\end{pgfscope}%
\begin{pgfscope}%
\pgfpathrectangle{\pgfqpoint{0.894063in}{3.540000in}}{\pgfqpoint{6.713438in}{2.060556in}} %
\pgfusepath{clip}%
\pgfsetbuttcap%
\pgfsetroundjoin%
\definecolor{currentfill}{rgb}{0.000000,0.750000,0.750000}%
\pgfsetfillcolor{currentfill}%
\pgfsetlinewidth{1.003750pt}%
\definecolor{currentstroke}{rgb}{0.000000,0.750000,0.750000}%
\pgfsetstrokecolor{currentstroke}%
\pgfsetdash{}{0pt}%
\pgfpathmoveto{\pgfqpoint{7.204694in}{5.256782in}}%
\pgfpathcurveto{\pgfqpoint{7.212930in}{5.256782in}}{\pgfqpoint{7.220830in}{5.260055in}}{\pgfqpoint{7.226654in}{5.265879in}}%
\pgfpathcurveto{\pgfqpoint{7.232478in}{5.271702in}}{\pgfqpoint{7.235750in}{5.279602in}}{\pgfqpoint{7.235750in}{5.287839in}}%
\pgfpathcurveto{\pgfqpoint{7.235750in}{5.296075in}}{\pgfqpoint{7.232478in}{5.303975in}}{\pgfqpoint{7.226654in}{5.309799in}}%
\pgfpathcurveto{\pgfqpoint{7.220830in}{5.315623in}}{\pgfqpoint{7.212930in}{5.318895in}}{\pgfqpoint{7.204694in}{5.318895in}}%
\pgfpathcurveto{\pgfqpoint{7.196457in}{5.318895in}}{\pgfqpoint{7.188557in}{5.315623in}}{\pgfqpoint{7.182733in}{5.309799in}}%
\pgfpathcurveto{\pgfqpoint{7.176910in}{5.303975in}}{\pgfqpoint{7.173637in}{5.296075in}}{\pgfqpoint{7.173637in}{5.287839in}}%
\pgfpathcurveto{\pgfqpoint{7.173637in}{5.279602in}}{\pgfqpoint{7.176910in}{5.271702in}}{\pgfqpoint{7.182733in}{5.265879in}}%
\pgfpathcurveto{\pgfqpoint{7.188557in}{5.260055in}}{\pgfqpoint{7.196457in}{5.256782in}}{\pgfqpoint{7.204694in}{5.256782in}}%
\pgfpathclose%
\pgfusepath{stroke,fill}%
\end{pgfscope}%
\begin{pgfscope}%
\pgfpathrectangle{\pgfqpoint{0.894063in}{3.540000in}}{\pgfqpoint{6.713438in}{2.060556in}} %
\pgfusepath{clip}%
\pgfsetbuttcap%
\pgfsetroundjoin%
\definecolor{currentfill}{rgb}{0.000000,0.750000,0.750000}%
\pgfsetfillcolor{currentfill}%
\pgfsetlinewidth{1.003750pt}%
\definecolor{currentstroke}{rgb}{0.000000,0.750000,0.750000}%
\pgfsetstrokecolor{currentstroke}%
\pgfsetdash{}{0pt}%
\pgfpathmoveto{\pgfqpoint{6.264813in}{5.257530in}}%
\pgfpathcurveto{\pgfqpoint{6.273049in}{5.257530in}}{\pgfqpoint{6.280949in}{5.260802in}}{\pgfqpoint{6.286773in}{5.266626in}}%
\pgfpathcurveto{\pgfqpoint{6.292597in}{5.272450in}}{\pgfqpoint{6.295869in}{5.280350in}}{\pgfqpoint{6.295869in}{5.288586in}}%
\pgfpathcurveto{\pgfqpoint{6.295869in}{5.296822in}}{\pgfqpoint{6.292597in}{5.304722in}}{\pgfqpoint{6.286773in}{5.310546in}}%
\pgfpathcurveto{\pgfqpoint{6.280949in}{5.316370in}}{\pgfqpoint{6.273049in}{5.319643in}}{\pgfqpoint{6.264813in}{5.319643in}}%
\pgfpathcurveto{\pgfqpoint{6.256576in}{5.319643in}}{\pgfqpoint{6.248676in}{5.316370in}}{\pgfqpoint{6.242852in}{5.310546in}}%
\pgfpathcurveto{\pgfqpoint{6.237028in}{5.304722in}}{\pgfqpoint{6.233756in}{5.296822in}}{\pgfqpoint{6.233756in}{5.288586in}}%
\pgfpathcurveto{\pgfqpoint{6.233756in}{5.280350in}}{\pgfqpoint{6.237028in}{5.272450in}}{\pgfqpoint{6.242852in}{5.266626in}}%
\pgfpathcurveto{\pgfqpoint{6.248676in}{5.260802in}}{\pgfqpoint{6.256576in}{5.257530in}}{\pgfqpoint{6.264813in}{5.257530in}}%
\pgfpathclose%
\pgfusepath{stroke,fill}%
\end{pgfscope}%
\begin{pgfscope}%
\pgfpathrectangle{\pgfqpoint{0.894063in}{3.540000in}}{\pgfqpoint{6.713438in}{2.060556in}} %
\pgfusepath{clip}%
\pgfsetbuttcap%
\pgfsetroundjoin%
\definecolor{currentfill}{rgb}{0.000000,0.750000,0.750000}%
\pgfsetfillcolor{currentfill}%
\pgfsetlinewidth{1.003750pt}%
\definecolor{currentstroke}{rgb}{0.000000,0.750000,0.750000}%
\pgfsetstrokecolor{currentstroke}%
\pgfsetdash{}{0pt}%
\pgfpathmoveto{\pgfqpoint{7.473231in}{5.256076in}}%
\pgfpathcurveto{\pgfqpoint{7.481468in}{5.256076in}}{\pgfqpoint{7.489368in}{5.259349in}}{\pgfqpoint{7.495192in}{5.265172in}}%
\pgfpathcurveto{\pgfqpoint{7.501015in}{5.270996in}}{\pgfqpoint{7.504288in}{5.278896in}}{\pgfqpoint{7.504288in}{5.287133in}}%
\pgfpathcurveto{\pgfqpoint{7.504288in}{5.295369in}}{\pgfqpoint{7.501015in}{5.303269in}}{\pgfqpoint{7.495192in}{5.309093in}}%
\pgfpathcurveto{\pgfqpoint{7.489368in}{5.314917in}}{\pgfqpoint{7.481468in}{5.318189in}}{\pgfqpoint{7.473231in}{5.318189in}}%
\pgfpathcurveto{\pgfqpoint{7.464995in}{5.318189in}}{\pgfqpoint{7.457095in}{5.314917in}}{\pgfqpoint{7.451271in}{5.309093in}}%
\pgfpathcurveto{\pgfqpoint{7.445447in}{5.303269in}}{\pgfqpoint{7.442175in}{5.295369in}}{\pgfqpoint{7.442175in}{5.287133in}}%
\pgfpathcurveto{\pgfqpoint{7.442175in}{5.278896in}}{\pgfqpoint{7.445447in}{5.270996in}}{\pgfqpoint{7.451271in}{5.265172in}}%
\pgfpathcurveto{\pgfqpoint{7.457095in}{5.259349in}}{\pgfqpoint{7.464995in}{5.256076in}}{\pgfqpoint{7.473231in}{5.256076in}}%
\pgfpathclose%
\pgfusepath{stroke,fill}%
\end{pgfscope}%
\begin{pgfscope}%
\pgfpathrectangle{\pgfqpoint{0.894063in}{3.540000in}}{\pgfqpoint{6.713438in}{2.060556in}} %
\pgfusepath{clip}%
\pgfsetbuttcap%
\pgfsetroundjoin%
\definecolor{currentfill}{rgb}{0.000000,0.750000,0.750000}%
\pgfsetfillcolor{currentfill}%
\pgfsetlinewidth{1.003750pt}%
\definecolor{currentstroke}{rgb}{0.000000,0.750000,0.750000}%
\pgfsetstrokecolor{currentstroke}%
\pgfsetdash{}{0pt}%
\pgfpathmoveto{\pgfqpoint{5.056394in}{5.257558in}}%
\pgfpathcurveto{\pgfqpoint{5.064630in}{5.257558in}}{\pgfqpoint{5.072530in}{5.260831in}}{\pgfqpoint{5.078354in}{5.266655in}}%
\pgfpathcurveto{\pgfqpoint{5.084178in}{5.272479in}}{\pgfqpoint{5.087450in}{5.280379in}}{\pgfqpoint{5.087450in}{5.288615in}}%
\pgfpathcurveto{\pgfqpoint{5.087450in}{5.296851in}}{\pgfqpoint{5.084178in}{5.304751in}}{\pgfqpoint{5.078354in}{5.310575in}}%
\pgfpathcurveto{\pgfqpoint{5.072530in}{5.316399in}}{\pgfqpoint{5.064630in}{5.319671in}}{\pgfqpoint{5.056394in}{5.319671in}}%
\pgfpathcurveto{\pgfqpoint{5.048157in}{5.319671in}}{\pgfqpoint{5.040257in}{5.316399in}}{\pgfqpoint{5.034433in}{5.310575in}}%
\pgfpathcurveto{\pgfqpoint{5.028610in}{5.304751in}}{\pgfqpoint{5.025337in}{5.296851in}}{\pgfqpoint{5.025337in}{5.288615in}}%
\pgfpathcurveto{\pgfqpoint{5.025337in}{5.280379in}}{\pgfqpoint{5.028610in}{5.272479in}}{\pgfqpoint{5.034433in}{5.266655in}}%
\pgfpathcurveto{\pgfqpoint{5.040257in}{5.260831in}}{\pgfqpoint{5.048157in}{5.257558in}}{\pgfqpoint{5.056394in}{5.257558in}}%
\pgfpathclose%
\pgfusepath{stroke,fill}%
\end{pgfscope}%
\begin{pgfscope}%
\pgfpathrectangle{\pgfqpoint{0.894063in}{3.540000in}}{\pgfqpoint{6.713438in}{2.060556in}} %
\pgfusepath{clip}%
\pgfsetbuttcap%
\pgfsetroundjoin%
\definecolor{currentfill}{rgb}{0.000000,0.750000,0.750000}%
\pgfsetfillcolor{currentfill}%
\pgfsetlinewidth{1.003750pt}%
\definecolor{currentstroke}{rgb}{0.000000,0.750000,0.750000}%
\pgfsetstrokecolor{currentstroke}%
\pgfsetdash{}{0pt}%
\pgfpathmoveto{\pgfqpoint{2.908094in}{5.257693in}}%
\pgfpathcurveto{\pgfqpoint{2.916330in}{5.257693in}}{\pgfqpoint{2.924230in}{5.260965in}}{\pgfqpoint{2.930054in}{5.266789in}}%
\pgfpathcurveto{\pgfqpoint{2.935878in}{5.272613in}}{\pgfqpoint{2.939150in}{5.280513in}}{\pgfqpoint{2.939150in}{5.288750in}}%
\pgfpathcurveto{\pgfqpoint{2.939150in}{5.296986in}}{\pgfqpoint{2.935878in}{5.304886in}}{\pgfqpoint{2.930054in}{5.310710in}}%
\pgfpathcurveto{\pgfqpoint{2.924230in}{5.316534in}}{\pgfqpoint{2.916330in}{5.319806in}}{\pgfqpoint{2.908094in}{5.319806in}}%
\pgfpathcurveto{\pgfqpoint{2.899857in}{5.319806in}}{\pgfqpoint{2.891957in}{5.316534in}}{\pgfqpoint{2.886133in}{5.310710in}}%
\pgfpathcurveto{\pgfqpoint{2.880310in}{5.304886in}}{\pgfqpoint{2.877037in}{5.296986in}}{\pgfqpoint{2.877037in}{5.288750in}}%
\pgfpathcurveto{\pgfqpoint{2.877037in}{5.280513in}}{\pgfqpoint{2.880310in}{5.272613in}}{\pgfqpoint{2.886133in}{5.266789in}}%
\pgfpathcurveto{\pgfqpoint{2.891957in}{5.260965in}}{\pgfqpoint{2.899857in}{5.257693in}}{\pgfqpoint{2.908094in}{5.257693in}}%
\pgfpathclose%
\pgfusepath{stroke,fill}%
\end{pgfscope}%
\begin{pgfscope}%
\pgfpathrectangle{\pgfqpoint{0.894063in}{3.540000in}}{\pgfqpoint{6.713438in}{2.060556in}} %
\pgfusepath{clip}%
\pgfsetbuttcap%
\pgfsetroundjoin%
\definecolor{currentfill}{rgb}{0.000000,0.750000,0.750000}%
\pgfsetfillcolor{currentfill}%
\pgfsetlinewidth{1.003750pt}%
\definecolor{currentstroke}{rgb}{0.000000,0.750000,0.750000}%
\pgfsetstrokecolor{currentstroke}%
\pgfsetdash{}{0pt}%
\pgfpathmoveto{\pgfqpoint{3.445169in}{5.257671in}}%
\pgfpathcurveto{\pgfqpoint{3.453405in}{5.257671in}}{\pgfqpoint{3.461305in}{5.260943in}}{\pgfqpoint{3.467129in}{5.266767in}}%
\pgfpathcurveto{\pgfqpoint{3.472953in}{5.272591in}}{\pgfqpoint{3.476225in}{5.280491in}}{\pgfqpoint{3.476225in}{5.288728in}}%
\pgfpathcurveto{\pgfqpoint{3.476225in}{5.296964in}}{\pgfqpoint{3.472953in}{5.304864in}}{\pgfqpoint{3.467129in}{5.310688in}}%
\pgfpathcurveto{\pgfqpoint{3.461305in}{5.316512in}}{\pgfqpoint{3.453405in}{5.319784in}}{\pgfqpoint{3.445169in}{5.319784in}}%
\pgfpathcurveto{\pgfqpoint{3.436932in}{5.319784in}}{\pgfqpoint{3.429032in}{5.316512in}}{\pgfqpoint{3.423208in}{5.310688in}}%
\pgfpathcurveto{\pgfqpoint{3.417385in}{5.304864in}}{\pgfqpoint{3.414112in}{5.296964in}}{\pgfqpoint{3.414112in}{5.288728in}}%
\pgfpathcurveto{\pgfqpoint{3.414112in}{5.280491in}}{\pgfqpoint{3.417385in}{5.272591in}}{\pgfqpoint{3.423208in}{5.266767in}}%
\pgfpathcurveto{\pgfqpoint{3.429032in}{5.260943in}}{\pgfqpoint{3.436932in}{5.257671in}}{\pgfqpoint{3.445169in}{5.257671in}}%
\pgfpathclose%
\pgfusepath{stroke,fill}%
\end{pgfscope}%
\begin{pgfscope}%
\pgfpathrectangle{\pgfqpoint{0.894063in}{3.540000in}}{\pgfqpoint{6.713438in}{2.060556in}} %
\pgfusepath{clip}%
\pgfsetbuttcap%
\pgfsetroundjoin%
\definecolor{currentfill}{rgb}{0.000000,0.750000,0.750000}%
\pgfsetfillcolor{currentfill}%
\pgfsetlinewidth{1.003750pt}%
\definecolor{currentstroke}{rgb}{0.000000,0.750000,0.750000}%
\pgfsetstrokecolor{currentstroke}%
\pgfsetdash{}{0pt}%
\pgfpathmoveto{\pgfqpoint{4.116513in}{5.257608in}}%
\pgfpathcurveto{\pgfqpoint{4.124749in}{5.257608in}}{\pgfqpoint{4.132649in}{5.260880in}}{\pgfqpoint{4.138473in}{5.266704in}}%
\pgfpathcurveto{\pgfqpoint{4.144297in}{5.272528in}}{\pgfqpoint{4.147569in}{5.280428in}}{\pgfqpoint{4.147569in}{5.288664in}}%
\pgfpathcurveto{\pgfqpoint{4.147569in}{5.296901in}}{\pgfqpoint{4.144297in}{5.304801in}}{\pgfqpoint{4.138473in}{5.310625in}}%
\pgfpathcurveto{\pgfqpoint{4.132649in}{5.316449in}}{\pgfqpoint{4.124749in}{5.319721in}}{\pgfqpoint{4.116513in}{5.319721in}}%
\pgfpathcurveto{\pgfqpoint{4.108276in}{5.319721in}}{\pgfqpoint{4.100376in}{5.316449in}}{\pgfqpoint{4.094552in}{5.310625in}}%
\pgfpathcurveto{\pgfqpoint{4.088728in}{5.304801in}}{\pgfqpoint{4.085456in}{5.296901in}}{\pgfqpoint{4.085456in}{5.288664in}}%
\pgfpathcurveto{\pgfqpoint{4.085456in}{5.280428in}}{\pgfqpoint{4.088728in}{5.272528in}}{\pgfqpoint{4.094552in}{5.266704in}}%
\pgfpathcurveto{\pgfqpoint{4.100376in}{5.260880in}}{\pgfqpoint{4.108276in}{5.257608in}}{\pgfqpoint{4.116513in}{5.257608in}}%
\pgfpathclose%
\pgfusepath{stroke,fill}%
\end{pgfscope}%
\begin{pgfscope}%
\pgfpathrectangle{\pgfqpoint{0.894063in}{3.540000in}}{\pgfqpoint{6.713438in}{2.060556in}} %
\pgfusepath{clip}%
\pgfsetbuttcap%
\pgfsetroundjoin%
\definecolor{currentfill}{rgb}{0.000000,0.750000,0.750000}%
\pgfsetfillcolor{currentfill}%
\pgfsetlinewidth{1.003750pt}%
\definecolor{currentstroke}{rgb}{0.000000,0.750000,0.750000}%
\pgfsetstrokecolor{currentstroke}%
\pgfsetdash{}{0pt}%
\pgfpathmoveto{\pgfqpoint{1.431138in}{5.258208in}}%
\pgfpathcurveto{\pgfqpoint{1.439374in}{5.258208in}}{\pgfqpoint{1.447274in}{5.261480in}}{\pgfqpoint{1.453098in}{5.267304in}}%
\pgfpathcurveto{\pgfqpoint{1.458922in}{5.273128in}}{\pgfqpoint{1.462194in}{5.281028in}}{\pgfqpoint{1.462194in}{5.289265in}}%
\pgfpathcurveto{\pgfqpoint{1.462194in}{5.297501in}}{\pgfqpoint{1.458922in}{5.305401in}}{\pgfqpoint{1.453098in}{5.311225in}}%
\pgfpathcurveto{\pgfqpoint{1.447274in}{5.317049in}}{\pgfqpoint{1.439374in}{5.320321in}}{\pgfqpoint{1.431138in}{5.320321in}}%
\pgfpathcurveto{\pgfqpoint{1.422901in}{5.320321in}}{\pgfqpoint{1.415001in}{5.317049in}}{\pgfqpoint{1.409177in}{5.311225in}}%
\pgfpathcurveto{\pgfqpoint{1.403353in}{5.305401in}}{\pgfqpoint{1.400081in}{5.297501in}}{\pgfqpoint{1.400081in}{5.289265in}}%
\pgfpathcurveto{\pgfqpoint{1.400081in}{5.281028in}}{\pgfqpoint{1.403353in}{5.273128in}}{\pgfqpoint{1.409177in}{5.267304in}}%
\pgfpathcurveto{\pgfqpoint{1.415001in}{5.261480in}}{\pgfqpoint{1.422901in}{5.258208in}}{\pgfqpoint{1.431138in}{5.258208in}}%
\pgfpathclose%
\pgfusepath{stroke,fill}%
\end{pgfscope}%
\begin{pgfscope}%
\pgfpathrectangle{\pgfqpoint{0.894063in}{3.540000in}}{\pgfqpoint{6.713438in}{2.060556in}} %
\pgfusepath{clip}%
\pgfsetbuttcap%
\pgfsetroundjoin%
\definecolor{currentfill}{rgb}{0.000000,0.750000,0.750000}%
\pgfsetfillcolor{currentfill}%
\pgfsetlinewidth{1.003750pt}%
\definecolor{currentstroke}{rgb}{0.000000,0.750000,0.750000}%
\pgfsetstrokecolor{currentstroke}%
\pgfsetdash{}{0pt}%
\pgfpathmoveto{\pgfqpoint{2.773825in}{5.257701in}}%
\pgfpathcurveto{\pgfqpoint{2.782061in}{5.257701in}}{\pgfqpoint{2.789961in}{5.260974in}}{\pgfqpoint{2.795785in}{5.266798in}}%
\pgfpathcurveto{\pgfqpoint{2.801609in}{5.272621in}}{\pgfqpoint{2.804881in}{5.280522in}}{\pgfqpoint{2.804881in}{5.288758in}}%
\pgfpathcurveto{\pgfqpoint{2.804881in}{5.296994in}}{\pgfqpoint{2.801609in}{5.304894in}}{\pgfqpoint{2.795785in}{5.310718in}}%
\pgfpathcurveto{\pgfqpoint{2.789961in}{5.316542in}}{\pgfqpoint{2.782061in}{5.319814in}}{\pgfqpoint{2.773825in}{5.319814in}}%
\pgfpathcurveto{\pgfqpoint{2.765589in}{5.319814in}}{\pgfqpoint{2.757689in}{5.316542in}}{\pgfqpoint{2.751865in}{5.310718in}}%
\pgfpathcurveto{\pgfqpoint{2.746041in}{5.304894in}}{\pgfqpoint{2.742769in}{5.296994in}}{\pgfqpoint{2.742769in}{5.288758in}}%
\pgfpathcurveto{\pgfqpoint{2.742769in}{5.280522in}}{\pgfqpoint{2.746041in}{5.272621in}}{\pgfqpoint{2.751865in}{5.266798in}}%
\pgfpathcurveto{\pgfqpoint{2.757689in}{5.260974in}}{\pgfqpoint{2.765589in}{5.257701in}}{\pgfqpoint{2.773825in}{5.257701in}}%
\pgfpathclose%
\pgfusepath{stroke,fill}%
\end{pgfscope}%
\begin{pgfscope}%
\pgfpathrectangle{\pgfqpoint{0.894063in}{3.540000in}}{\pgfqpoint{6.713438in}{2.060556in}} %
\pgfusepath{clip}%
\pgfsetbuttcap%
\pgfsetroundjoin%
\definecolor{currentfill}{rgb}{0.000000,0.750000,0.750000}%
\pgfsetfillcolor{currentfill}%
\pgfsetlinewidth{1.003750pt}%
\definecolor{currentstroke}{rgb}{0.000000,0.750000,0.750000}%
\pgfsetstrokecolor{currentstroke}%
\pgfsetdash{}{0pt}%
\pgfpathmoveto{\pgfqpoint{1.565406in}{5.258194in}}%
\pgfpathcurveto{\pgfqpoint{1.573643in}{5.258194in}}{\pgfqpoint{1.581543in}{5.261467in}}{\pgfqpoint{1.587367in}{5.267291in}}%
\pgfpathcurveto{\pgfqpoint{1.593190in}{5.273115in}}{\pgfqpoint{1.596463in}{5.281015in}}{\pgfqpoint{1.596463in}{5.289251in}}%
\pgfpathcurveto{\pgfqpoint{1.596463in}{5.297487in}}{\pgfqpoint{1.593190in}{5.305387in}}{\pgfqpoint{1.587367in}{5.311211in}}%
\pgfpathcurveto{\pgfqpoint{1.581543in}{5.317035in}}{\pgfqpoint{1.573643in}{5.320307in}}{\pgfqpoint{1.565406in}{5.320307in}}%
\pgfpathcurveto{\pgfqpoint{1.557170in}{5.320307in}}{\pgfqpoint{1.549270in}{5.317035in}}{\pgfqpoint{1.543446in}{5.311211in}}%
\pgfpathcurveto{\pgfqpoint{1.537622in}{5.305387in}}{\pgfqpoint{1.534350in}{5.297487in}}{\pgfqpoint{1.534350in}{5.289251in}}%
\pgfpathcurveto{\pgfqpoint{1.534350in}{5.281015in}}{\pgfqpoint{1.537622in}{5.273115in}}{\pgfqpoint{1.543446in}{5.267291in}}%
\pgfpathcurveto{\pgfqpoint{1.549270in}{5.261467in}}{\pgfqpoint{1.557170in}{5.258194in}}{\pgfqpoint{1.565406in}{5.258194in}}%
\pgfpathclose%
\pgfusepath{stroke,fill}%
\end{pgfscope}%
\begin{pgfscope}%
\pgfpathrectangle{\pgfqpoint{0.894063in}{3.540000in}}{\pgfqpoint{6.713438in}{2.060556in}} %
\pgfusepath{clip}%
\pgfsetbuttcap%
\pgfsetroundjoin%
\definecolor{currentfill}{rgb}{0.000000,0.750000,0.750000}%
\pgfsetfillcolor{currentfill}%
\pgfsetlinewidth{1.003750pt}%
\definecolor{currentstroke}{rgb}{0.000000,0.750000,0.750000}%
\pgfsetstrokecolor{currentstroke}%
\pgfsetdash{}{0pt}%
\pgfpathmoveto{\pgfqpoint{4.250781in}{5.257605in}}%
\pgfpathcurveto{\pgfqpoint{4.259018in}{5.257605in}}{\pgfqpoint{4.266918in}{5.260877in}}{\pgfqpoint{4.272742in}{5.266701in}}%
\pgfpathcurveto{\pgfqpoint{4.278565in}{5.272525in}}{\pgfqpoint{4.281838in}{5.280425in}}{\pgfqpoint{4.281838in}{5.288662in}}%
\pgfpathcurveto{\pgfqpoint{4.281838in}{5.296898in}}{\pgfqpoint{4.278565in}{5.304798in}}{\pgfqpoint{4.272742in}{5.310622in}}%
\pgfpathcurveto{\pgfqpoint{4.266918in}{5.316446in}}{\pgfqpoint{4.259018in}{5.319718in}}{\pgfqpoint{4.250781in}{5.319718in}}%
\pgfpathcurveto{\pgfqpoint{4.242545in}{5.319718in}}{\pgfqpoint{4.234645in}{5.316446in}}{\pgfqpoint{4.228821in}{5.310622in}}%
\pgfpathcurveto{\pgfqpoint{4.222997in}{5.304798in}}{\pgfqpoint{4.219725in}{5.296898in}}{\pgfqpoint{4.219725in}{5.288662in}}%
\pgfpathcurveto{\pgfqpoint{4.219725in}{5.280425in}}{\pgfqpoint{4.222997in}{5.272525in}}{\pgfqpoint{4.228821in}{5.266701in}}%
\pgfpathcurveto{\pgfqpoint{4.234645in}{5.260877in}}{\pgfqpoint{4.242545in}{5.257605in}}{\pgfqpoint{4.250781in}{5.257605in}}%
\pgfpathclose%
\pgfusepath{stroke,fill}%
\end{pgfscope}%
\begin{pgfscope}%
\pgfpathrectangle{\pgfqpoint{0.894063in}{3.540000in}}{\pgfqpoint{6.713438in}{2.060556in}} %
\pgfusepath{clip}%
\pgfsetbuttcap%
\pgfsetroundjoin%
\definecolor{currentfill}{rgb}{0.000000,0.750000,0.750000}%
\pgfsetfillcolor{currentfill}%
\pgfsetlinewidth{1.003750pt}%
\definecolor{currentstroke}{rgb}{0.000000,0.750000,0.750000}%
\pgfsetstrokecolor{currentstroke}%
\pgfsetdash{}{0pt}%
\pgfpathmoveto{\pgfqpoint{3.847975in}{5.257624in}}%
\pgfpathcurveto{\pgfqpoint{3.856211in}{5.257624in}}{\pgfqpoint{3.864111in}{5.260897in}}{\pgfqpoint{3.869935in}{5.266721in}}%
\pgfpathcurveto{\pgfqpoint{3.875759in}{5.272545in}}{\pgfqpoint{3.879031in}{5.280445in}}{\pgfqpoint{3.879031in}{5.288681in}}%
\pgfpathcurveto{\pgfqpoint{3.879031in}{5.296917in}}{\pgfqpoint{3.875759in}{5.304817in}}{\pgfqpoint{3.869935in}{5.310641in}}%
\pgfpathcurveto{\pgfqpoint{3.864111in}{5.316465in}}{\pgfqpoint{3.856211in}{5.319737in}}{\pgfqpoint{3.847975in}{5.319737in}}%
\pgfpathcurveto{\pgfqpoint{3.839739in}{5.319737in}}{\pgfqpoint{3.831839in}{5.316465in}}{\pgfqpoint{3.826015in}{5.310641in}}%
\pgfpathcurveto{\pgfqpoint{3.820191in}{5.304817in}}{\pgfqpoint{3.816919in}{5.296917in}}{\pgfqpoint{3.816919in}{5.288681in}}%
\pgfpathcurveto{\pgfqpoint{3.816919in}{5.280445in}}{\pgfqpoint{3.820191in}{5.272545in}}{\pgfqpoint{3.826015in}{5.266721in}}%
\pgfpathcurveto{\pgfqpoint{3.831839in}{5.260897in}}{\pgfqpoint{3.839739in}{5.257624in}}{\pgfqpoint{3.847975in}{5.257624in}}%
\pgfpathclose%
\pgfusepath{stroke,fill}%
\end{pgfscope}%
\begin{pgfscope}%
\pgfpathrectangle{\pgfqpoint{0.894063in}{3.540000in}}{\pgfqpoint{6.713438in}{2.060556in}} %
\pgfusepath{clip}%
\pgfsetbuttcap%
\pgfsetroundjoin%
\definecolor{currentfill}{rgb}{0.000000,0.750000,0.750000}%
\pgfsetfillcolor{currentfill}%
\pgfsetlinewidth{1.003750pt}%
\definecolor{currentstroke}{rgb}{0.000000,0.750000,0.750000}%
\pgfsetstrokecolor{currentstroke}%
\pgfsetdash{}{0pt}%
\pgfpathmoveto{\pgfqpoint{7.607500in}{5.256073in}}%
\pgfpathcurveto{\pgfqpoint{7.615736in}{5.256073in}}{\pgfqpoint{7.623636in}{5.259346in}}{\pgfqpoint{7.629460in}{5.265170in}}%
\pgfpathcurveto{\pgfqpoint{7.635284in}{5.270994in}}{\pgfqpoint{7.638556in}{5.278894in}}{\pgfqpoint{7.638556in}{5.287130in}}%
\pgfpathcurveto{\pgfqpoint{7.638556in}{5.295366in}}{\pgfqpoint{7.635284in}{5.303266in}}{\pgfqpoint{7.629460in}{5.309090in}}%
\pgfpathcurveto{\pgfqpoint{7.623636in}{5.314914in}}{\pgfqpoint{7.615736in}{5.318186in}}{\pgfqpoint{7.607500in}{5.318186in}}%
\pgfpathcurveto{\pgfqpoint{7.599264in}{5.318186in}}{\pgfqpoint{7.591364in}{5.314914in}}{\pgfqpoint{7.585540in}{5.309090in}}%
\pgfpathcurveto{\pgfqpoint{7.579716in}{5.303266in}}{\pgfqpoint{7.576444in}{5.295366in}}{\pgfqpoint{7.576444in}{5.287130in}}%
\pgfpathcurveto{\pgfqpoint{7.576444in}{5.278894in}}{\pgfqpoint{7.579716in}{5.270994in}}{\pgfqpoint{7.585540in}{5.265170in}}%
\pgfpathcurveto{\pgfqpoint{7.591364in}{5.259346in}}{\pgfqpoint{7.599264in}{5.256073in}}{\pgfqpoint{7.607500in}{5.256073in}}%
\pgfpathclose%
\pgfusepath{stroke,fill}%
\end{pgfscope}%
\begin{pgfscope}%
\pgfpathrectangle{\pgfqpoint{0.894063in}{3.540000in}}{\pgfqpoint{6.713438in}{2.060556in}} %
\pgfusepath{clip}%
\pgfsetbuttcap%
\pgfsetroundjoin%
\definecolor{currentfill}{rgb}{0.000000,0.750000,0.750000}%
\pgfsetfillcolor{currentfill}%
\pgfsetlinewidth{1.003750pt}%
\definecolor{currentstroke}{rgb}{0.000000,0.750000,0.750000}%
\pgfsetstrokecolor{currentstroke}%
\pgfsetdash{}{0pt}%
\pgfpathmoveto{\pgfqpoint{4.385050in}{5.257604in}}%
\pgfpathcurveto{\pgfqpoint{4.393286in}{5.257604in}}{\pgfqpoint{4.401186in}{5.260876in}}{\pgfqpoint{4.407010in}{5.266700in}}%
\pgfpathcurveto{\pgfqpoint{4.412834in}{5.272524in}}{\pgfqpoint{4.416106in}{5.280424in}}{\pgfqpoint{4.416106in}{5.288660in}}%
\pgfpathcurveto{\pgfqpoint{4.416106in}{5.296897in}}{\pgfqpoint{4.412834in}{5.304797in}}{\pgfqpoint{4.407010in}{5.310621in}}%
\pgfpathcurveto{\pgfqpoint{4.401186in}{5.316444in}}{\pgfqpoint{4.393286in}{5.319717in}}{\pgfqpoint{4.385050in}{5.319717in}}%
\pgfpathcurveto{\pgfqpoint{4.376814in}{5.319717in}}{\pgfqpoint{4.368914in}{5.316444in}}{\pgfqpoint{4.363090in}{5.310621in}}%
\pgfpathcurveto{\pgfqpoint{4.357266in}{5.304797in}}{\pgfqpoint{4.353994in}{5.296897in}}{\pgfqpoint{4.353994in}{5.288660in}}%
\pgfpathcurveto{\pgfqpoint{4.353994in}{5.280424in}}{\pgfqpoint{4.357266in}{5.272524in}}{\pgfqpoint{4.363090in}{5.266700in}}%
\pgfpathcurveto{\pgfqpoint{4.368914in}{5.260876in}}{\pgfqpoint{4.376814in}{5.257604in}}{\pgfqpoint{4.385050in}{5.257604in}}%
\pgfpathclose%
\pgfusepath{stroke,fill}%
\end{pgfscope}%
\begin{pgfscope}%
\pgfpathrectangle{\pgfqpoint{0.894063in}{3.540000in}}{\pgfqpoint{6.713438in}{2.060556in}} %
\pgfusepath{clip}%
\pgfsetbuttcap%
\pgfsetroundjoin%
\definecolor{currentfill}{rgb}{0.000000,0.750000,0.750000}%
\pgfsetfillcolor{currentfill}%
\pgfsetlinewidth{1.003750pt}%
\definecolor{currentstroke}{rgb}{0.000000,0.750000,0.750000}%
\pgfsetstrokecolor{currentstroke}%
\pgfsetdash{}{0pt}%
\pgfpathmoveto{\pgfqpoint{6.533350in}{5.256819in}}%
\pgfpathcurveto{\pgfqpoint{6.541586in}{5.256819in}}{\pgfqpoint{6.549486in}{5.260092in}}{\pgfqpoint{6.555310in}{5.265916in}}%
\pgfpathcurveto{\pgfqpoint{6.561134in}{5.271740in}}{\pgfqpoint{6.564406in}{5.279640in}}{\pgfqpoint{6.564406in}{5.287876in}}%
\pgfpathcurveto{\pgfqpoint{6.564406in}{5.296112in}}{\pgfqpoint{6.561134in}{5.304012in}}{\pgfqpoint{6.555310in}{5.309836in}}%
\pgfpathcurveto{\pgfqpoint{6.549486in}{5.315660in}}{\pgfqpoint{6.541586in}{5.318932in}}{\pgfqpoint{6.533350in}{5.318932in}}%
\pgfpathcurveto{\pgfqpoint{6.525114in}{5.318932in}}{\pgfqpoint{6.517214in}{5.315660in}}{\pgfqpoint{6.511390in}{5.309836in}}%
\pgfpathcurveto{\pgfqpoint{6.505566in}{5.304012in}}{\pgfqpoint{6.502294in}{5.296112in}}{\pgfqpoint{6.502294in}{5.287876in}}%
\pgfpathcurveto{\pgfqpoint{6.502294in}{5.279640in}}{\pgfqpoint{6.505566in}{5.271740in}}{\pgfqpoint{6.511390in}{5.265916in}}%
\pgfpathcurveto{\pgfqpoint{6.517214in}{5.260092in}}{\pgfqpoint{6.525114in}{5.256819in}}{\pgfqpoint{6.533350in}{5.256819in}}%
\pgfpathclose%
\pgfusepath{stroke,fill}%
\end{pgfscope}%
\begin{pgfscope}%
\pgfpathrectangle{\pgfqpoint{0.894063in}{3.540000in}}{\pgfqpoint{6.713438in}{2.060556in}} %
\pgfusepath{clip}%
\pgfsetbuttcap%
\pgfsetroundjoin%
\definecolor{currentfill}{rgb}{0.000000,0.750000,0.750000}%
\pgfsetfillcolor{currentfill}%
\pgfsetlinewidth{1.003750pt}%
\definecolor{currentstroke}{rgb}{0.000000,0.750000,0.750000}%
\pgfsetstrokecolor{currentstroke}%
\pgfsetdash{}{0pt}%
\pgfpathmoveto{\pgfqpoint{1.296869in}{5.258269in}}%
\pgfpathcurveto{\pgfqpoint{1.305105in}{5.258269in}}{\pgfqpoint{1.313005in}{5.261541in}}{\pgfqpoint{1.318829in}{5.267365in}}%
\pgfpathcurveto{\pgfqpoint{1.324653in}{5.273189in}}{\pgfqpoint{1.327925in}{5.281089in}}{\pgfqpoint{1.327925in}{5.289325in}}%
\pgfpathcurveto{\pgfqpoint{1.327925in}{5.297561in}}{\pgfqpoint{1.324653in}{5.305461in}}{\pgfqpoint{1.318829in}{5.311285in}}%
\pgfpathcurveto{\pgfqpoint{1.313005in}{5.317109in}}{\pgfqpoint{1.305105in}{5.320382in}}{\pgfqpoint{1.296869in}{5.320382in}}%
\pgfpathcurveto{\pgfqpoint{1.288632in}{5.320382in}}{\pgfqpoint{1.280732in}{5.317109in}}{\pgfqpoint{1.274908in}{5.311285in}}%
\pgfpathcurveto{\pgfqpoint{1.269085in}{5.305461in}}{\pgfqpoint{1.265812in}{5.297561in}}{\pgfqpoint{1.265812in}{5.289325in}}%
\pgfpathcurveto{\pgfqpoint{1.265812in}{5.281089in}}{\pgfqpoint{1.269085in}{5.273189in}}{\pgfqpoint{1.274908in}{5.267365in}}%
\pgfpathcurveto{\pgfqpoint{1.280732in}{5.261541in}}{\pgfqpoint{1.288632in}{5.258269in}}{\pgfqpoint{1.296869in}{5.258269in}}%
\pgfpathclose%
\pgfusepath{stroke,fill}%
\end{pgfscope}%
\begin{pgfscope}%
\pgfpathrectangle{\pgfqpoint{0.894063in}{3.540000in}}{\pgfqpoint{6.713438in}{2.060556in}} %
\pgfusepath{clip}%
\pgfsetbuttcap%
\pgfsetroundjoin%
\definecolor{currentfill}{rgb}{0.000000,0.750000,0.750000}%
\pgfsetfillcolor{currentfill}%
\pgfsetlinewidth{1.003750pt}%
\definecolor{currentstroke}{rgb}{0.000000,0.750000,0.750000}%
\pgfsetstrokecolor{currentstroke}%
\pgfsetdash{}{0pt}%
\pgfpathmoveto{\pgfqpoint{4.519319in}{5.257601in}}%
\pgfpathcurveto{\pgfqpoint{4.527555in}{5.257601in}}{\pgfqpoint{4.535455in}{5.260873in}}{\pgfqpoint{4.541279in}{5.266697in}}%
\pgfpathcurveto{\pgfqpoint{4.547103in}{5.272521in}}{\pgfqpoint{4.550375in}{5.280421in}}{\pgfqpoint{4.550375in}{5.288658in}}%
\pgfpathcurveto{\pgfqpoint{4.550375in}{5.296894in}}{\pgfqpoint{4.547103in}{5.304794in}}{\pgfqpoint{4.541279in}{5.310618in}}%
\pgfpathcurveto{\pgfqpoint{4.535455in}{5.316442in}}{\pgfqpoint{4.527555in}{5.319714in}}{\pgfqpoint{4.519319in}{5.319714in}}%
\pgfpathcurveto{\pgfqpoint{4.511082in}{5.319714in}}{\pgfqpoint{4.503182in}{5.316442in}}{\pgfqpoint{4.497358in}{5.310618in}}%
\pgfpathcurveto{\pgfqpoint{4.491535in}{5.304794in}}{\pgfqpoint{4.488262in}{5.296894in}}{\pgfqpoint{4.488262in}{5.288658in}}%
\pgfpathcurveto{\pgfqpoint{4.488262in}{5.280421in}}{\pgfqpoint{4.491535in}{5.272521in}}{\pgfqpoint{4.497358in}{5.266697in}}%
\pgfpathcurveto{\pgfqpoint{4.503182in}{5.260873in}}{\pgfqpoint{4.511082in}{5.257601in}}{\pgfqpoint{4.519319in}{5.257601in}}%
\pgfpathclose%
\pgfusepath{stroke,fill}%
\end{pgfscope}%
\begin{pgfscope}%
\pgfpathrectangle{\pgfqpoint{0.894063in}{3.540000in}}{\pgfqpoint{6.713438in}{2.060556in}} %
\pgfusepath{clip}%
\pgfsetbuttcap%
\pgfsetroundjoin%
\definecolor{currentfill}{rgb}{0.000000,0.750000,0.750000}%
\pgfsetfillcolor{currentfill}%
\pgfsetlinewidth{1.003750pt}%
\definecolor{currentstroke}{rgb}{0.000000,0.750000,0.750000}%
\pgfsetstrokecolor{currentstroke}%
\pgfsetdash{}{0pt}%
\pgfpathmoveto{\pgfqpoint{2.505288in}{5.257837in}}%
\pgfpathcurveto{\pgfqpoint{2.513524in}{5.257837in}}{\pgfqpoint{2.521424in}{5.261110in}}{\pgfqpoint{2.527248in}{5.266934in}}%
\pgfpathcurveto{\pgfqpoint{2.533072in}{5.272757in}}{\pgfqpoint{2.536344in}{5.280658in}}{\pgfqpoint{2.536344in}{5.288894in}}%
\pgfpathcurveto{\pgfqpoint{2.536344in}{5.297130in}}{\pgfqpoint{2.533072in}{5.305030in}}{\pgfqpoint{2.527248in}{5.310854in}}%
\pgfpathcurveto{\pgfqpoint{2.521424in}{5.316678in}}{\pgfqpoint{2.513524in}{5.319950in}}{\pgfqpoint{2.505288in}{5.319950in}}%
\pgfpathcurveto{\pgfqpoint{2.497051in}{5.319950in}}{\pgfqpoint{2.489151in}{5.316678in}}{\pgfqpoint{2.483327in}{5.310854in}}%
\pgfpathcurveto{\pgfqpoint{2.477503in}{5.305030in}}{\pgfqpoint{2.474231in}{5.297130in}}{\pgfqpoint{2.474231in}{5.288894in}}%
\pgfpathcurveto{\pgfqpoint{2.474231in}{5.280658in}}{\pgfqpoint{2.477503in}{5.272757in}}{\pgfqpoint{2.483327in}{5.266934in}}%
\pgfpathcurveto{\pgfqpoint{2.489151in}{5.261110in}}{\pgfqpoint{2.497051in}{5.257837in}}{\pgfqpoint{2.505288in}{5.257837in}}%
\pgfpathclose%
\pgfusepath{stroke,fill}%
\end{pgfscope}%
\begin{pgfscope}%
\pgfpathrectangle{\pgfqpoint{0.894063in}{3.540000in}}{\pgfqpoint{6.713438in}{2.060556in}} %
\pgfusepath{clip}%
\pgfsetbuttcap%
\pgfsetroundjoin%
\definecolor{currentfill}{rgb}{0.000000,0.750000,0.750000}%
\pgfsetfillcolor{currentfill}%
\pgfsetlinewidth{1.003750pt}%
\definecolor{currentstroke}{rgb}{0.000000,0.750000,0.750000}%
\pgfsetstrokecolor{currentstroke}%
\pgfsetdash{}{0pt}%
\pgfpathmoveto{\pgfqpoint{5.459200in}{5.257543in}}%
\pgfpathcurveto{\pgfqpoint{5.467436in}{5.257543in}}{\pgfqpoint{5.475336in}{5.260816in}}{\pgfqpoint{5.481160in}{5.266640in}}%
\pgfpathcurveto{\pgfqpoint{5.486984in}{5.272463in}}{\pgfqpoint{5.490256in}{5.280364in}}{\pgfqpoint{5.490256in}{5.288600in}}%
\pgfpathcurveto{\pgfqpoint{5.490256in}{5.296836in}}{\pgfqpoint{5.486984in}{5.304736in}}{\pgfqpoint{5.481160in}{5.310560in}}%
\pgfpathcurveto{\pgfqpoint{5.475336in}{5.316384in}}{\pgfqpoint{5.467436in}{5.319656in}}{\pgfqpoint{5.459200in}{5.319656in}}%
\pgfpathcurveto{\pgfqpoint{5.450964in}{5.319656in}}{\pgfqpoint{5.443064in}{5.316384in}}{\pgfqpoint{5.437240in}{5.310560in}}%
\pgfpathcurveto{\pgfqpoint{5.431416in}{5.304736in}}{\pgfqpoint{5.428144in}{5.296836in}}{\pgfqpoint{5.428144in}{5.288600in}}%
\pgfpathcurveto{\pgfqpoint{5.428144in}{5.280364in}}{\pgfqpoint{5.431416in}{5.272463in}}{\pgfqpoint{5.437240in}{5.266640in}}%
\pgfpathcurveto{\pgfqpoint{5.443064in}{5.260816in}}{\pgfqpoint{5.450964in}{5.257543in}}{\pgfqpoint{5.459200in}{5.257543in}}%
\pgfpathclose%
\pgfusepath{stroke,fill}%
\end{pgfscope}%
\begin{pgfscope}%
\pgfpathrectangle{\pgfqpoint{0.894063in}{3.540000in}}{\pgfqpoint{6.713438in}{2.060556in}} %
\pgfusepath{clip}%
\pgfsetbuttcap%
\pgfsetroundjoin%
\definecolor{currentfill}{rgb}{0.000000,0.750000,0.750000}%
\pgfsetfillcolor{currentfill}%
\pgfsetlinewidth{1.003750pt}%
\definecolor{currentstroke}{rgb}{0.000000,0.750000,0.750000}%
\pgfsetstrokecolor{currentstroke}%
\pgfsetdash{}{0pt}%
\pgfpathmoveto{\pgfqpoint{6.936156in}{5.256826in}}%
\pgfpathcurveto{\pgfqpoint{6.944393in}{5.256826in}}{\pgfqpoint{6.952293in}{5.260099in}}{\pgfqpoint{6.958117in}{5.265922in}}%
\pgfpathcurveto{\pgfqpoint{6.963940in}{5.271746in}}{\pgfqpoint{6.967213in}{5.279646in}}{\pgfqpoint{6.967213in}{5.287883in}}%
\pgfpathcurveto{\pgfqpoint{6.967213in}{5.296119in}}{\pgfqpoint{6.963940in}{5.304019in}}{\pgfqpoint{6.958117in}{5.309843in}}%
\pgfpathcurveto{\pgfqpoint{6.952293in}{5.315667in}}{\pgfqpoint{6.944393in}{5.318939in}}{\pgfqpoint{6.936156in}{5.318939in}}%
\pgfpathcurveto{\pgfqpoint{6.927920in}{5.318939in}}{\pgfqpoint{6.920020in}{5.315667in}}{\pgfqpoint{6.914196in}{5.309843in}}%
\pgfpathcurveto{\pgfqpoint{6.908372in}{5.304019in}}{\pgfqpoint{6.905100in}{5.296119in}}{\pgfqpoint{6.905100in}{5.287883in}}%
\pgfpathcurveto{\pgfqpoint{6.905100in}{5.279646in}}{\pgfqpoint{6.908372in}{5.271746in}}{\pgfqpoint{6.914196in}{5.265922in}}%
\pgfpathcurveto{\pgfqpoint{6.920020in}{5.260099in}}{\pgfqpoint{6.927920in}{5.256826in}}{\pgfqpoint{6.936156in}{5.256826in}}%
\pgfpathclose%
\pgfusepath{stroke,fill}%
\end{pgfscope}%
\begin{pgfscope}%
\pgfpathrectangle{\pgfqpoint{0.894063in}{3.540000in}}{\pgfqpoint{6.713438in}{2.060556in}} %
\pgfusepath{clip}%
\pgfsetbuttcap%
\pgfsetroundjoin%
\definecolor{currentfill}{rgb}{0.000000,0.750000,0.750000}%
\pgfsetfillcolor{currentfill}%
\pgfsetlinewidth{1.003750pt}%
\definecolor{currentstroke}{rgb}{0.000000,0.750000,0.750000}%
\pgfsetstrokecolor{currentstroke}%
\pgfsetdash{}{0pt}%
\pgfpathmoveto{\pgfqpoint{5.862006in}{5.257534in}}%
\pgfpathcurveto{\pgfqpoint{5.870243in}{5.257534in}}{\pgfqpoint{5.878143in}{5.260806in}}{\pgfqpoint{5.883967in}{5.266630in}}%
\pgfpathcurveto{\pgfqpoint{5.889790in}{5.272454in}}{\pgfqpoint{5.893063in}{5.280354in}}{\pgfqpoint{5.893063in}{5.288590in}}%
\pgfpathcurveto{\pgfqpoint{5.893063in}{5.296826in}}{\pgfqpoint{5.889790in}{5.304727in}}{\pgfqpoint{5.883967in}{5.310550in}}%
\pgfpathcurveto{\pgfqpoint{5.878143in}{5.316374in}}{\pgfqpoint{5.870243in}{5.319647in}}{\pgfqpoint{5.862006in}{5.319647in}}%
\pgfpathcurveto{\pgfqpoint{5.853770in}{5.319647in}}{\pgfqpoint{5.845870in}{5.316374in}}{\pgfqpoint{5.840046in}{5.310550in}}%
\pgfpathcurveto{\pgfqpoint{5.834222in}{5.304727in}}{\pgfqpoint{5.830950in}{5.296826in}}{\pgfqpoint{5.830950in}{5.288590in}}%
\pgfpathcurveto{\pgfqpoint{5.830950in}{5.280354in}}{\pgfqpoint{5.834222in}{5.272454in}}{\pgfqpoint{5.840046in}{5.266630in}}%
\pgfpathcurveto{\pgfqpoint{5.845870in}{5.260806in}}{\pgfqpoint{5.853770in}{5.257534in}}{\pgfqpoint{5.862006in}{5.257534in}}%
\pgfpathclose%
\pgfusepath{stroke,fill}%
\end{pgfscope}%
\begin{pgfscope}%
\pgfpathrectangle{\pgfqpoint{0.894063in}{3.540000in}}{\pgfqpoint{6.713438in}{2.060556in}} %
\pgfusepath{clip}%
\pgfsetbuttcap%
\pgfsetroundjoin%
\definecolor{currentfill}{rgb}{0.000000,0.750000,0.750000}%
\pgfsetfillcolor{currentfill}%
\pgfsetlinewidth{1.003750pt}%
\definecolor{currentstroke}{rgb}{0.000000,0.750000,0.750000}%
\pgfsetstrokecolor{currentstroke}%
\pgfsetdash{}{0pt}%
\pgfpathmoveto{\pgfqpoint{7.070425in}{5.256789in}}%
\pgfpathcurveto{\pgfqpoint{7.078661in}{5.256789in}}{\pgfqpoint{7.086561in}{5.260061in}}{\pgfqpoint{7.092385in}{5.265885in}}%
\pgfpathcurveto{\pgfqpoint{7.098209in}{5.271709in}}{\pgfqpoint{7.101481in}{5.279609in}}{\pgfqpoint{7.101481in}{5.287846in}}%
\pgfpathcurveto{\pgfqpoint{7.101481in}{5.296082in}}{\pgfqpoint{7.098209in}{5.303982in}}{\pgfqpoint{7.092385in}{5.309806in}}%
\pgfpathcurveto{\pgfqpoint{7.086561in}{5.315630in}}{\pgfqpoint{7.078661in}{5.318902in}}{\pgfqpoint{7.070425in}{5.318902in}}%
\pgfpathcurveto{\pgfqpoint{7.062189in}{5.318902in}}{\pgfqpoint{7.054289in}{5.315630in}}{\pgfqpoint{7.048465in}{5.309806in}}%
\pgfpathcurveto{\pgfqpoint{7.042641in}{5.303982in}}{\pgfqpoint{7.039369in}{5.296082in}}{\pgfqpoint{7.039369in}{5.287846in}}%
\pgfpathcurveto{\pgfqpoint{7.039369in}{5.279609in}}{\pgfqpoint{7.042641in}{5.271709in}}{\pgfqpoint{7.048465in}{5.265885in}}%
\pgfpathcurveto{\pgfqpoint{7.054289in}{5.260061in}}{\pgfqpoint{7.062189in}{5.256789in}}{\pgfqpoint{7.070425in}{5.256789in}}%
\pgfpathclose%
\pgfusepath{stroke,fill}%
\end{pgfscope}%
\begin{pgfscope}%
\pgfpathrectangle{\pgfqpoint{0.894063in}{3.540000in}}{\pgfqpoint{6.713438in}{2.060556in}} %
\pgfusepath{clip}%
\pgfsetbuttcap%
\pgfsetroundjoin%
\definecolor{currentfill}{rgb}{0.000000,0.750000,0.750000}%
\pgfsetfillcolor{currentfill}%
\pgfsetlinewidth{1.003750pt}%
\definecolor{currentstroke}{rgb}{0.000000,0.750000,0.750000}%
\pgfsetstrokecolor{currentstroke}%
\pgfsetdash{}{0pt}%
\pgfpathmoveto{\pgfqpoint{3.176631in}{5.257688in}}%
\pgfpathcurveto{\pgfqpoint{3.184868in}{5.257688in}}{\pgfqpoint{3.192768in}{5.260960in}}{\pgfqpoint{3.198592in}{5.266784in}}%
\pgfpathcurveto{\pgfqpoint{3.204415in}{5.272608in}}{\pgfqpoint{3.207688in}{5.280508in}}{\pgfqpoint{3.207688in}{5.288744in}}%
\pgfpathcurveto{\pgfqpoint{3.207688in}{5.296980in}}{\pgfqpoint{3.204415in}{5.304880in}}{\pgfqpoint{3.198592in}{5.310704in}}%
\pgfpathcurveto{\pgfqpoint{3.192768in}{5.316528in}}{\pgfqpoint{3.184868in}{5.319801in}}{\pgfqpoint{3.176631in}{5.319801in}}%
\pgfpathcurveto{\pgfqpoint{3.168395in}{5.319801in}}{\pgfqpoint{3.160495in}{5.316528in}}{\pgfqpoint{3.154671in}{5.310704in}}%
\pgfpathcurveto{\pgfqpoint{3.148847in}{5.304880in}}{\pgfqpoint{3.145575in}{5.296980in}}{\pgfqpoint{3.145575in}{5.288744in}}%
\pgfpathcurveto{\pgfqpoint{3.145575in}{5.280508in}}{\pgfqpoint{3.148847in}{5.272608in}}{\pgfqpoint{3.154671in}{5.266784in}}%
\pgfpathcurveto{\pgfqpoint{3.160495in}{5.260960in}}{\pgfqpoint{3.168395in}{5.257688in}}{\pgfqpoint{3.176631in}{5.257688in}}%
\pgfpathclose%
\pgfusepath{stroke,fill}%
\end{pgfscope}%
\begin{pgfscope}%
\pgfpathrectangle{\pgfqpoint{0.894063in}{3.540000in}}{\pgfqpoint{6.713438in}{2.060556in}} %
\pgfusepath{clip}%
\pgfsetbuttcap%
\pgfsetroundjoin%
\definecolor{currentfill}{rgb}{0.000000,0.750000,0.750000}%
\pgfsetfillcolor{currentfill}%
\pgfsetlinewidth{1.003750pt}%
\definecolor{currentstroke}{rgb}{0.000000,0.750000,0.750000}%
\pgfsetstrokecolor{currentstroke}%
\pgfsetdash{}{0pt}%
\pgfpathmoveto{\pgfqpoint{2.102481in}{5.258142in}}%
\pgfpathcurveto{\pgfqpoint{2.110718in}{5.258142in}}{\pgfqpoint{2.118618in}{5.261415in}}{\pgfqpoint{2.124442in}{5.267238in}}%
\pgfpathcurveto{\pgfqpoint{2.130265in}{5.273062in}}{\pgfqpoint{2.133538in}{5.280962in}}{\pgfqpoint{2.133538in}{5.289199in}}%
\pgfpathcurveto{\pgfqpoint{2.133538in}{5.297435in}}{\pgfqpoint{2.130265in}{5.305335in}}{\pgfqpoint{2.124442in}{5.311159in}}%
\pgfpathcurveto{\pgfqpoint{2.118618in}{5.316983in}}{\pgfqpoint{2.110718in}{5.320255in}}{\pgfqpoint{2.102481in}{5.320255in}}%
\pgfpathcurveto{\pgfqpoint{2.094245in}{5.320255in}}{\pgfqpoint{2.086345in}{5.316983in}}{\pgfqpoint{2.080521in}{5.311159in}}%
\pgfpathcurveto{\pgfqpoint{2.074697in}{5.305335in}}{\pgfqpoint{2.071425in}{5.297435in}}{\pgfqpoint{2.071425in}{5.289199in}}%
\pgfpathcurveto{\pgfqpoint{2.071425in}{5.280962in}}{\pgfqpoint{2.074697in}{5.273062in}}{\pgfqpoint{2.080521in}{5.267238in}}%
\pgfpathcurveto{\pgfqpoint{2.086345in}{5.261415in}}{\pgfqpoint{2.094245in}{5.258142in}}{\pgfqpoint{2.102481in}{5.258142in}}%
\pgfpathclose%
\pgfusepath{stroke,fill}%
\end{pgfscope}%
\begin{pgfscope}%
\pgfpathrectangle{\pgfqpoint{0.894063in}{3.540000in}}{\pgfqpoint{6.713438in}{2.060556in}} %
\pgfusepath{clip}%
\pgfsetbuttcap%
\pgfsetroundjoin%
\definecolor{currentfill}{rgb}{0.000000,0.750000,0.750000}%
\pgfsetfillcolor{currentfill}%
\pgfsetlinewidth{1.003750pt}%
\definecolor{currentstroke}{rgb}{0.000000,0.750000,0.750000}%
\pgfsetstrokecolor{currentstroke}%
\pgfsetdash{}{0pt}%
\pgfpathmoveto{\pgfqpoint{1.968213in}{5.258160in}}%
\pgfpathcurveto{\pgfqpoint{1.976449in}{5.258160in}}{\pgfqpoint{1.984349in}{5.261432in}}{\pgfqpoint{1.990173in}{5.267256in}}%
\pgfpathcurveto{\pgfqpoint{1.995997in}{5.273080in}}{\pgfqpoint{1.999269in}{5.280980in}}{\pgfqpoint{1.999269in}{5.289217in}}%
\pgfpathcurveto{\pgfqpoint{1.999269in}{5.297453in}}{\pgfqpoint{1.995997in}{5.305353in}}{\pgfqpoint{1.990173in}{5.311177in}}%
\pgfpathcurveto{\pgfqpoint{1.984349in}{5.317001in}}{\pgfqpoint{1.976449in}{5.320273in}}{\pgfqpoint{1.968213in}{5.320273in}}%
\pgfpathcurveto{\pgfqpoint{1.959976in}{5.320273in}}{\pgfqpoint{1.952076in}{5.317001in}}{\pgfqpoint{1.946252in}{5.311177in}}%
\pgfpathcurveto{\pgfqpoint{1.940428in}{5.305353in}}{\pgfqpoint{1.937156in}{5.297453in}}{\pgfqpoint{1.937156in}{5.289217in}}%
\pgfpathcurveto{\pgfqpoint{1.937156in}{5.280980in}}{\pgfqpoint{1.940428in}{5.273080in}}{\pgfqpoint{1.946252in}{5.267256in}}%
\pgfpathcurveto{\pgfqpoint{1.952076in}{5.261432in}}{\pgfqpoint{1.959976in}{5.258160in}}{\pgfqpoint{1.968213in}{5.258160in}}%
\pgfpathclose%
\pgfusepath{stroke,fill}%
\end{pgfscope}%
\begin{pgfscope}%
\pgfpathrectangle{\pgfqpoint{0.894063in}{3.540000in}}{\pgfqpoint{6.713438in}{2.060556in}} %
\pgfusepath{clip}%
\pgfsetbuttcap%
\pgfsetroundjoin%
\definecolor{currentfill}{rgb}{0.000000,0.750000,0.750000}%
\pgfsetfillcolor{currentfill}%
\pgfsetlinewidth{1.003750pt}%
\definecolor{currentstroke}{rgb}{0.000000,0.750000,0.750000}%
\pgfsetstrokecolor{currentstroke}%
\pgfsetdash{}{0pt}%
\pgfpathmoveto{\pgfqpoint{3.310900in}{5.257690in}}%
\pgfpathcurveto{\pgfqpoint{3.319136in}{5.257690in}}{\pgfqpoint{3.327036in}{5.260963in}}{\pgfqpoint{3.332860in}{5.266787in}}%
\pgfpathcurveto{\pgfqpoint{3.338684in}{5.272610in}}{\pgfqpoint{3.341956in}{5.280511in}}{\pgfqpoint{3.341956in}{5.288747in}}%
\pgfpathcurveto{\pgfqpoint{3.341956in}{5.296983in}}{\pgfqpoint{3.338684in}{5.304883in}}{\pgfqpoint{3.332860in}{5.310707in}}%
\pgfpathcurveto{\pgfqpoint{3.327036in}{5.316531in}}{\pgfqpoint{3.319136in}{5.319803in}}{\pgfqpoint{3.310900in}{5.319803in}}%
\pgfpathcurveto{\pgfqpoint{3.302664in}{5.319803in}}{\pgfqpoint{3.294764in}{5.316531in}}{\pgfqpoint{3.288940in}{5.310707in}}%
\pgfpathcurveto{\pgfqpoint{3.283116in}{5.304883in}}{\pgfqpoint{3.279844in}{5.296983in}}{\pgfqpoint{3.279844in}{5.288747in}}%
\pgfpathcurveto{\pgfqpoint{3.279844in}{5.280511in}}{\pgfqpoint{3.283116in}{5.272610in}}{\pgfqpoint{3.288940in}{5.266787in}}%
\pgfpathcurveto{\pgfqpoint{3.294764in}{5.260963in}}{\pgfqpoint{3.302664in}{5.257690in}}{\pgfqpoint{3.310900in}{5.257690in}}%
\pgfpathclose%
\pgfusepath{stroke,fill}%
\end{pgfscope}%
\begin{pgfscope}%
\pgfpathrectangle{\pgfqpoint{0.894063in}{3.540000in}}{\pgfqpoint{6.713438in}{2.060556in}} %
\pgfusepath{clip}%
\pgfsetbuttcap%
\pgfsetroundjoin%
\definecolor{currentfill}{rgb}{0.000000,0.750000,0.750000}%
\pgfsetfillcolor{currentfill}%
\pgfsetlinewidth{1.003750pt}%
\definecolor{currentstroke}{rgb}{0.000000,0.750000,0.750000}%
\pgfsetstrokecolor{currentstroke}%
\pgfsetdash{}{0pt}%
\pgfpathmoveto{\pgfqpoint{5.593469in}{5.257542in}}%
\pgfpathcurveto{\pgfqpoint{5.601705in}{5.257542in}}{\pgfqpoint{5.609605in}{5.260814in}}{\pgfqpoint{5.615429in}{5.266638in}}%
\pgfpathcurveto{\pgfqpoint{5.621253in}{5.272462in}}{\pgfqpoint{5.624525in}{5.280362in}}{\pgfqpoint{5.624525in}{5.288598in}}%
\pgfpathcurveto{\pgfqpoint{5.624525in}{5.296835in}}{\pgfqpoint{5.621253in}{5.304735in}}{\pgfqpoint{5.615429in}{5.310559in}}%
\pgfpathcurveto{\pgfqpoint{5.609605in}{5.316383in}}{\pgfqpoint{5.601705in}{5.319655in}}{\pgfqpoint{5.593469in}{5.319655in}}%
\pgfpathcurveto{\pgfqpoint{5.585232in}{5.319655in}}{\pgfqpoint{5.577332in}{5.316383in}}{\pgfqpoint{5.571508in}{5.310559in}}%
\pgfpathcurveto{\pgfqpoint{5.565685in}{5.304735in}}{\pgfqpoint{5.562412in}{5.296835in}}{\pgfqpoint{5.562412in}{5.288598in}}%
\pgfpathcurveto{\pgfqpoint{5.562412in}{5.280362in}}{\pgfqpoint{5.565685in}{5.272462in}}{\pgfqpoint{5.571508in}{5.266638in}}%
\pgfpathcurveto{\pgfqpoint{5.577332in}{5.260814in}}{\pgfqpoint{5.585232in}{5.257542in}}{\pgfqpoint{5.593469in}{5.257542in}}%
\pgfpathclose%
\pgfusepath{stroke,fill}%
\end{pgfscope}%
\begin{pgfscope}%
\pgfpathrectangle{\pgfqpoint{0.894063in}{3.540000in}}{\pgfqpoint{6.713438in}{2.060556in}} %
\pgfusepath{clip}%
\pgfsetbuttcap%
\pgfsetroundjoin%
\definecolor{currentfill}{rgb}{0.000000,0.750000,0.750000}%
\pgfsetfillcolor{currentfill}%
\pgfsetlinewidth{1.003750pt}%
\definecolor{currentstroke}{rgb}{0.000000,0.750000,0.750000}%
\pgfsetstrokecolor{currentstroke}%
\pgfsetdash{}{0pt}%
\pgfpathmoveto{\pgfqpoint{3.042363in}{5.257689in}}%
\pgfpathcurveto{\pgfqpoint{3.050599in}{5.257689in}}{\pgfqpoint{3.058499in}{5.260961in}}{\pgfqpoint{3.064323in}{5.266785in}}%
\pgfpathcurveto{\pgfqpoint{3.070147in}{5.272609in}}{\pgfqpoint{3.073419in}{5.280509in}}{\pgfqpoint{3.073419in}{5.288745in}}%
\pgfpathcurveto{\pgfqpoint{3.073419in}{5.296982in}}{\pgfqpoint{3.070147in}{5.304882in}}{\pgfqpoint{3.064323in}{5.310706in}}%
\pgfpathcurveto{\pgfqpoint{3.058499in}{5.316530in}}{\pgfqpoint{3.050599in}{5.319802in}}{\pgfqpoint{3.042363in}{5.319802in}}%
\pgfpathcurveto{\pgfqpoint{3.034126in}{5.319802in}}{\pgfqpoint{3.026226in}{5.316530in}}{\pgfqpoint{3.020402in}{5.310706in}}%
\pgfpathcurveto{\pgfqpoint{3.014578in}{5.304882in}}{\pgfqpoint{3.011306in}{5.296982in}}{\pgfqpoint{3.011306in}{5.288745in}}%
\pgfpathcurveto{\pgfqpoint{3.011306in}{5.280509in}}{\pgfqpoint{3.014578in}{5.272609in}}{\pgfqpoint{3.020402in}{5.266785in}}%
\pgfpathcurveto{\pgfqpoint{3.026226in}{5.260961in}}{\pgfqpoint{3.034126in}{5.257689in}}{\pgfqpoint{3.042363in}{5.257689in}}%
\pgfpathclose%
\pgfusepath{stroke,fill}%
\end{pgfscope}%
\begin{pgfscope}%
\pgfpathrectangle{\pgfqpoint{0.894063in}{3.540000in}}{\pgfqpoint{6.713438in}{2.060556in}} %
\pgfusepath{clip}%
\pgfsetbuttcap%
\pgfsetroundjoin%
\definecolor{currentfill}{rgb}{0.000000,0.750000,0.750000}%
\pgfsetfillcolor{currentfill}%
\pgfsetlinewidth{1.003750pt}%
\definecolor{currentstroke}{rgb}{0.000000,0.750000,0.750000}%
\pgfsetstrokecolor{currentstroke}%
\pgfsetdash{}{0pt}%
\pgfpathmoveto{\pgfqpoint{5.190663in}{5.257545in}}%
\pgfpathcurveto{\pgfqpoint{5.198899in}{5.257545in}}{\pgfqpoint{5.206799in}{5.260817in}}{\pgfqpoint{5.212623in}{5.266641in}}%
\pgfpathcurveto{\pgfqpoint{5.218447in}{5.272465in}}{\pgfqpoint{5.221719in}{5.280365in}}{\pgfqpoint{5.221719in}{5.288601in}}%
\pgfpathcurveto{\pgfqpoint{5.221719in}{5.296837in}}{\pgfqpoint{5.218447in}{5.304738in}}{\pgfqpoint{5.212623in}{5.310561in}}%
\pgfpathcurveto{\pgfqpoint{5.206799in}{5.316385in}}{\pgfqpoint{5.198899in}{5.319658in}}{\pgfqpoint{5.190663in}{5.319658in}}%
\pgfpathcurveto{\pgfqpoint{5.182426in}{5.319658in}}{\pgfqpoint{5.174526in}{5.316385in}}{\pgfqpoint{5.168702in}{5.310561in}}%
\pgfpathcurveto{\pgfqpoint{5.162878in}{5.304738in}}{\pgfqpoint{5.159606in}{5.296837in}}{\pgfqpoint{5.159606in}{5.288601in}}%
\pgfpathcurveto{\pgfqpoint{5.159606in}{5.280365in}}{\pgfqpoint{5.162878in}{5.272465in}}{\pgfqpoint{5.168702in}{5.266641in}}%
\pgfpathcurveto{\pgfqpoint{5.174526in}{5.260817in}}{\pgfqpoint{5.182426in}{5.257545in}}{\pgfqpoint{5.190663in}{5.257545in}}%
\pgfpathclose%
\pgfusepath{stroke,fill}%
\end{pgfscope}%
\begin{pgfscope}%
\pgfpathrectangle{\pgfqpoint{0.894063in}{3.540000in}}{\pgfqpoint{6.713438in}{2.060556in}} %
\pgfusepath{clip}%
\pgfsetbuttcap%
\pgfsetroundjoin%
\definecolor{currentfill}{rgb}{0.000000,0.750000,0.750000}%
\pgfsetfillcolor{currentfill}%
\pgfsetlinewidth{1.003750pt}%
\definecolor{currentstroke}{rgb}{0.000000,0.750000,0.750000}%
\pgfsetstrokecolor{currentstroke}%
\pgfsetdash{}{0pt}%
\pgfpathmoveto{\pgfqpoint{6.801888in}{5.256832in}}%
\pgfpathcurveto{\pgfqpoint{6.810124in}{5.256832in}}{\pgfqpoint{6.818024in}{5.260104in}}{\pgfqpoint{6.823848in}{5.265928in}}%
\pgfpathcurveto{\pgfqpoint{6.829672in}{5.271752in}}{\pgfqpoint{6.832944in}{5.279652in}}{\pgfqpoint{6.832944in}{5.287888in}}%
\pgfpathcurveto{\pgfqpoint{6.832944in}{5.296125in}}{\pgfqpoint{6.829672in}{5.304025in}}{\pgfqpoint{6.823848in}{5.309848in}}%
\pgfpathcurveto{\pgfqpoint{6.818024in}{5.315672in}}{\pgfqpoint{6.810124in}{5.318945in}}{\pgfqpoint{6.801888in}{5.318945in}}%
\pgfpathcurveto{\pgfqpoint{6.793651in}{5.318945in}}{\pgfqpoint{6.785751in}{5.315672in}}{\pgfqpoint{6.779927in}{5.309848in}}%
\pgfpathcurveto{\pgfqpoint{6.774103in}{5.304025in}}{\pgfqpoint{6.770831in}{5.296125in}}{\pgfqpoint{6.770831in}{5.287888in}}%
\pgfpathcurveto{\pgfqpoint{6.770831in}{5.279652in}}{\pgfqpoint{6.774103in}{5.271752in}}{\pgfqpoint{6.779927in}{5.265928in}}%
\pgfpathcurveto{\pgfqpoint{6.785751in}{5.260104in}}{\pgfqpoint{6.793651in}{5.256832in}}{\pgfqpoint{6.801888in}{5.256832in}}%
\pgfpathclose%
\pgfusepath{stroke,fill}%
\end{pgfscope}%
\begin{pgfscope}%
\pgfpathrectangle{\pgfqpoint{0.894063in}{3.540000in}}{\pgfqpoint{6.713438in}{2.060556in}} %
\pgfusepath{clip}%
\pgfsetbuttcap%
\pgfsetroundjoin%
\definecolor{currentfill}{rgb}{0.000000,0.750000,0.750000}%
\pgfsetfillcolor{currentfill}%
\pgfsetlinewidth{1.003750pt}%
\definecolor{currentstroke}{rgb}{0.000000,0.750000,0.750000}%
\pgfsetstrokecolor{currentstroke}%
\pgfsetdash{}{0pt}%
\pgfpathmoveto{\pgfqpoint{3.579438in}{5.257666in}}%
\pgfpathcurveto{\pgfqpoint{3.587674in}{5.257666in}}{\pgfqpoint{3.595574in}{5.260938in}}{\pgfqpoint{3.601398in}{5.266762in}}%
\pgfpathcurveto{\pgfqpoint{3.607222in}{5.272586in}}{\pgfqpoint{3.610494in}{5.280486in}}{\pgfqpoint{3.610494in}{5.288722in}}%
\pgfpathcurveto{\pgfqpoint{3.610494in}{5.296958in}}{\pgfqpoint{3.607222in}{5.304858in}}{\pgfqpoint{3.601398in}{5.310682in}}%
\pgfpathcurveto{\pgfqpoint{3.595574in}{5.316506in}}{\pgfqpoint{3.587674in}{5.319779in}}{\pgfqpoint{3.579438in}{5.319779in}}%
\pgfpathcurveto{\pgfqpoint{3.571201in}{5.319779in}}{\pgfqpoint{3.563301in}{5.316506in}}{\pgfqpoint{3.557477in}{5.310682in}}%
\pgfpathcurveto{\pgfqpoint{3.551653in}{5.304858in}}{\pgfqpoint{3.548381in}{5.296958in}}{\pgfqpoint{3.548381in}{5.288722in}}%
\pgfpathcurveto{\pgfqpoint{3.548381in}{5.280486in}}{\pgfqpoint{3.551653in}{5.272586in}}{\pgfqpoint{3.557477in}{5.266762in}}%
\pgfpathcurveto{\pgfqpoint{3.563301in}{5.260938in}}{\pgfqpoint{3.571201in}{5.257666in}}{\pgfqpoint{3.579438in}{5.257666in}}%
\pgfpathclose%
\pgfusepath{stroke,fill}%
\end{pgfscope}%
\begin{pgfscope}%
\pgfpathrectangle{\pgfqpoint{0.894063in}{3.540000in}}{\pgfqpoint{6.713438in}{2.060556in}} %
\pgfusepath{clip}%
\pgfsetbuttcap%
\pgfsetroundjoin%
\definecolor{currentfill}{rgb}{0.000000,0.750000,0.750000}%
\pgfsetfillcolor{currentfill}%
\pgfsetlinewidth{1.003750pt}%
\definecolor{currentstroke}{rgb}{0.000000,0.750000,0.750000}%
\pgfsetstrokecolor{currentstroke}%
\pgfsetdash{}{0pt}%
\pgfpathmoveto{\pgfqpoint{2.371019in}{5.257840in}}%
\pgfpathcurveto{\pgfqpoint{2.379255in}{5.257840in}}{\pgfqpoint{2.387155in}{5.261112in}}{\pgfqpoint{2.392979in}{5.266936in}}%
\pgfpathcurveto{\pgfqpoint{2.398803in}{5.272760in}}{\pgfqpoint{2.402075in}{5.280660in}}{\pgfqpoint{2.402075in}{5.288897in}}%
\pgfpathcurveto{\pgfqpoint{2.402075in}{5.297133in}}{\pgfqpoint{2.398803in}{5.305033in}}{\pgfqpoint{2.392979in}{5.310857in}}%
\pgfpathcurveto{\pgfqpoint{2.387155in}{5.316681in}}{\pgfqpoint{2.379255in}{5.319953in}}{\pgfqpoint{2.371019in}{5.319953in}}%
\pgfpathcurveto{\pgfqpoint{2.362782in}{5.319953in}}{\pgfqpoint{2.354882in}{5.316681in}}{\pgfqpoint{2.349058in}{5.310857in}}%
\pgfpathcurveto{\pgfqpoint{2.343235in}{5.305033in}}{\pgfqpoint{2.339962in}{5.297133in}}{\pgfqpoint{2.339962in}{5.288897in}}%
\pgfpathcurveto{\pgfqpoint{2.339962in}{5.280660in}}{\pgfqpoint{2.343235in}{5.272760in}}{\pgfqpoint{2.349058in}{5.266936in}}%
\pgfpathcurveto{\pgfqpoint{2.354882in}{5.261112in}}{\pgfqpoint{2.362782in}{5.257840in}}{\pgfqpoint{2.371019in}{5.257840in}}%
\pgfpathclose%
\pgfusepath{stroke,fill}%
\end{pgfscope}%
\begin{pgfscope}%
\pgfpathrectangle{\pgfqpoint{0.894063in}{3.540000in}}{\pgfqpoint{6.713438in}{2.060556in}} %
\pgfusepath{clip}%
\pgfsetbuttcap%
\pgfsetroundjoin%
\definecolor{currentfill}{rgb}{0.000000,0.750000,0.750000}%
\pgfsetfillcolor{currentfill}%
\pgfsetlinewidth{1.003750pt}%
\definecolor{currentstroke}{rgb}{0.000000,0.750000,0.750000}%
\pgfsetstrokecolor{currentstroke}%
\pgfsetdash{}{0pt}%
\pgfpathmoveto{\pgfqpoint{3.982244in}{5.257608in}}%
\pgfpathcurveto{\pgfqpoint{3.990480in}{5.257608in}}{\pgfqpoint{3.998380in}{5.260880in}}{\pgfqpoint{4.004204in}{5.266704in}}%
\pgfpathcurveto{\pgfqpoint{4.010028in}{5.272528in}}{\pgfqpoint{4.013300in}{5.280428in}}{\pgfqpoint{4.013300in}{5.288664in}}%
\pgfpathcurveto{\pgfqpoint{4.013300in}{5.296901in}}{\pgfqpoint{4.010028in}{5.304801in}}{\pgfqpoint{4.004204in}{5.310625in}}%
\pgfpathcurveto{\pgfqpoint{3.998380in}{5.316449in}}{\pgfqpoint{3.990480in}{5.319721in}}{\pgfqpoint{3.982244in}{5.319721in}}%
\pgfpathcurveto{\pgfqpoint{3.974007in}{5.319721in}}{\pgfqpoint{3.966107in}{5.316449in}}{\pgfqpoint{3.960283in}{5.310625in}}%
\pgfpathcurveto{\pgfqpoint{3.954460in}{5.304801in}}{\pgfqpoint{3.951187in}{5.296901in}}{\pgfqpoint{3.951187in}{5.288664in}}%
\pgfpathcurveto{\pgfqpoint{3.951187in}{5.280428in}}{\pgfqpoint{3.954460in}{5.272528in}}{\pgfqpoint{3.960283in}{5.266704in}}%
\pgfpathcurveto{\pgfqpoint{3.966107in}{5.260880in}}{\pgfqpoint{3.974007in}{5.257608in}}{\pgfqpoint{3.982244in}{5.257608in}}%
\pgfpathclose%
\pgfusepath{stroke,fill}%
\end{pgfscope}%
\begin{pgfscope}%
\pgfpathrectangle{\pgfqpoint{0.894063in}{3.540000in}}{\pgfqpoint{6.713438in}{2.060556in}} %
\pgfusepath{clip}%
\pgfsetbuttcap%
\pgfsetroundjoin%
\definecolor{currentfill}{rgb}{0.000000,0.750000,0.750000}%
\pgfsetfillcolor{currentfill}%
\pgfsetlinewidth{1.003750pt}%
\definecolor{currentstroke}{rgb}{0.000000,0.750000,0.750000}%
\pgfsetstrokecolor{currentstroke}%
\pgfsetdash{}{0pt}%
\pgfpathmoveto{\pgfqpoint{4.653588in}{5.257589in}}%
\pgfpathcurveto{\pgfqpoint{4.661824in}{5.257589in}}{\pgfqpoint{4.669724in}{5.260861in}}{\pgfqpoint{4.675548in}{5.266685in}}%
\pgfpathcurveto{\pgfqpoint{4.681372in}{5.272509in}}{\pgfqpoint{4.684644in}{5.280409in}}{\pgfqpoint{4.684644in}{5.288645in}}%
\pgfpathcurveto{\pgfqpoint{4.684644in}{5.296881in}}{\pgfqpoint{4.681372in}{5.304781in}}{\pgfqpoint{4.675548in}{5.310605in}}%
\pgfpathcurveto{\pgfqpoint{4.669724in}{5.316429in}}{\pgfqpoint{4.661824in}{5.319702in}}{\pgfqpoint{4.653588in}{5.319702in}}%
\pgfpathcurveto{\pgfqpoint{4.645351in}{5.319702in}}{\pgfqpoint{4.637451in}{5.316429in}}{\pgfqpoint{4.631627in}{5.310605in}}%
\pgfpathcurveto{\pgfqpoint{4.625803in}{5.304781in}}{\pgfqpoint{4.622531in}{5.296881in}}{\pgfqpoint{4.622531in}{5.288645in}}%
\pgfpathcurveto{\pgfqpoint{4.622531in}{5.280409in}}{\pgfqpoint{4.625803in}{5.272509in}}{\pgfqpoint{4.631627in}{5.266685in}}%
\pgfpathcurveto{\pgfqpoint{4.637451in}{5.260861in}}{\pgfqpoint{4.645351in}{5.257589in}}{\pgfqpoint{4.653588in}{5.257589in}}%
\pgfpathclose%
\pgfusepath{stroke,fill}%
\end{pgfscope}%
\begin{pgfscope}%
\pgfpathrectangle{\pgfqpoint{0.894063in}{3.540000in}}{\pgfqpoint{6.713438in}{2.060556in}} %
\pgfusepath{clip}%
\pgfsetbuttcap%
\pgfsetroundjoin%
\definecolor{currentfill}{rgb}{0.000000,0.750000,0.750000}%
\pgfsetfillcolor{currentfill}%
\pgfsetlinewidth{1.003750pt}%
\definecolor{currentstroke}{rgb}{0.000000,0.750000,0.750000}%
\pgfsetstrokecolor{currentstroke}%
\pgfsetdash{}{0pt}%
\pgfpathmoveto{\pgfqpoint{3.713706in}{5.257655in}}%
\pgfpathcurveto{\pgfqpoint{3.721943in}{5.257655in}}{\pgfqpoint{3.729843in}{5.260927in}}{\pgfqpoint{3.735667in}{5.266751in}}%
\pgfpathcurveto{\pgfqpoint{3.741490in}{5.272575in}}{\pgfqpoint{3.744763in}{5.280475in}}{\pgfqpoint{3.744763in}{5.288711in}}%
\pgfpathcurveto{\pgfqpoint{3.744763in}{5.296947in}}{\pgfqpoint{3.741490in}{5.304847in}}{\pgfqpoint{3.735667in}{5.310671in}}%
\pgfpathcurveto{\pgfqpoint{3.729843in}{5.316495in}}{\pgfqpoint{3.721943in}{5.319768in}}{\pgfqpoint{3.713706in}{5.319768in}}%
\pgfpathcurveto{\pgfqpoint{3.705470in}{5.319768in}}{\pgfqpoint{3.697570in}{5.316495in}}{\pgfqpoint{3.691746in}{5.310671in}}%
\pgfpathcurveto{\pgfqpoint{3.685922in}{5.304847in}}{\pgfqpoint{3.682650in}{5.296947in}}{\pgfqpoint{3.682650in}{5.288711in}}%
\pgfpathcurveto{\pgfqpoint{3.682650in}{5.280475in}}{\pgfqpoint{3.685922in}{5.272575in}}{\pgfqpoint{3.691746in}{5.266751in}}%
\pgfpathcurveto{\pgfqpoint{3.697570in}{5.260927in}}{\pgfqpoint{3.705470in}{5.257655in}}{\pgfqpoint{3.713706in}{5.257655in}}%
\pgfpathclose%
\pgfusepath{stroke,fill}%
\end{pgfscope}%
\begin{pgfscope}%
\pgfpathrectangle{\pgfqpoint{0.894063in}{3.540000in}}{\pgfqpoint{6.713438in}{2.060556in}} %
\pgfusepath{clip}%
\pgfsetbuttcap%
\pgfsetroundjoin%
\definecolor{currentfill}{rgb}{0.000000,0.750000,0.750000}%
\pgfsetfillcolor{currentfill}%
\pgfsetlinewidth{1.003750pt}%
\definecolor{currentstroke}{rgb}{0.000000,0.750000,0.750000}%
\pgfsetstrokecolor{currentstroke}%
\pgfsetdash{}{0pt}%
\pgfpathmoveto{\pgfqpoint{2.236750in}{5.257847in}}%
\pgfpathcurveto{\pgfqpoint{2.244986in}{5.257847in}}{\pgfqpoint{2.252886in}{5.261119in}}{\pgfqpoint{2.258710in}{5.266943in}}%
\pgfpathcurveto{\pgfqpoint{2.264534in}{5.272767in}}{\pgfqpoint{2.267806in}{5.280667in}}{\pgfqpoint{2.267806in}{5.288903in}}%
\pgfpathcurveto{\pgfqpoint{2.267806in}{5.297140in}}{\pgfqpoint{2.264534in}{5.305040in}}{\pgfqpoint{2.258710in}{5.310864in}}%
\pgfpathcurveto{\pgfqpoint{2.252886in}{5.316688in}}{\pgfqpoint{2.244986in}{5.319960in}}{\pgfqpoint{2.236750in}{5.319960in}}%
\pgfpathcurveto{\pgfqpoint{2.228514in}{5.319960in}}{\pgfqpoint{2.220614in}{5.316688in}}{\pgfqpoint{2.214790in}{5.310864in}}%
\pgfpathcurveto{\pgfqpoint{2.208966in}{5.305040in}}{\pgfqpoint{2.205694in}{5.297140in}}{\pgfqpoint{2.205694in}{5.288903in}}%
\pgfpathcurveto{\pgfqpoint{2.205694in}{5.280667in}}{\pgfqpoint{2.208966in}{5.272767in}}{\pgfqpoint{2.214790in}{5.266943in}}%
\pgfpathcurveto{\pgfqpoint{2.220614in}{5.261119in}}{\pgfqpoint{2.228514in}{5.257847in}}{\pgfqpoint{2.236750in}{5.257847in}}%
\pgfpathclose%
\pgfusepath{stroke,fill}%
\end{pgfscope}%
\begin{pgfscope}%
\pgfpathrectangle{\pgfqpoint{0.894063in}{3.540000in}}{\pgfqpoint{6.713438in}{2.060556in}} %
\pgfusepath{clip}%
\pgfsetbuttcap%
\pgfsetroundjoin%
\definecolor{currentfill}{rgb}{0.501961,0.000000,0.501961}%
\pgfsetfillcolor{currentfill}%
\pgfsetlinewidth{1.003750pt}%
\definecolor{currentstroke}{rgb}{0.501961,0.000000,0.501961}%
\pgfsetstrokecolor{currentstroke}%
\pgfsetdash{}{0pt}%
\pgfpathmoveto{\pgfqpoint{6.667619in}{5.256098in}}%
\pgfpathcurveto{\pgfqpoint{6.675855in}{5.256098in}}{\pgfqpoint{6.683755in}{5.259370in}}{\pgfqpoint{6.689579in}{5.265194in}}%
\pgfpathcurveto{\pgfqpoint{6.695403in}{5.271018in}}{\pgfqpoint{6.698675in}{5.278918in}}{\pgfqpoint{6.698675in}{5.287155in}}%
\pgfpathcurveto{\pgfqpoint{6.698675in}{5.295391in}}{\pgfqpoint{6.695403in}{5.303291in}}{\pgfqpoint{6.689579in}{5.309115in}}%
\pgfpathcurveto{\pgfqpoint{6.683755in}{5.314939in}}{\pgfqpoint{6.675855in}{5.318211in}}{\pgfqpoint{6.667619in}{5.318211in}}%
\pgfpathcurveto{\pgfqpoint{6.659382in}{5.318211in}}{\pgfqpoint{6.651482in}{5.314939in}}{\pgfqpoint{6.645658in}{5.309115in}}%
\pgfpathcurveto{\pgfqpoint{6.639835in}{5.303291in}}{\pgfqpoint{6.636562in}{5.295391in}}{\pgfqpoint{6.636562in}{5.287155in}}%
\pgfpathcurveto{\pgfqpoint{6.636562in}{5.278918in}}{\pgfqpoint{6.639835in}{5.271018in}}{\pgfqpoint{6.645658in}{5.265194in}}%
\pgfpathcurveto{\pgfqpoint{6.651482in}{5.259370in}}{\pgfqpoint{6.659382in}{5.256098in}}{\pgfqpoint{6.667619in}{5.256098in}}%
\pgfpathclose%
\pgfusepath{stroke,fill}%
\end{pgfscope}%
\begin{pgfscope}%
\pgfpathrectangle{\pgfqpoint{0.894063in}{3.540000in}}{\pgfqpoint{6.713438in}{2.060556in}} %
\pgfusepath{clip}%
\pgfsetbuttcap%
\pgfsetroundjoin%
\definecolor{currentfill}{rgb}{0.501961,0.000000,0.501961}%
\pgfsetfillcolor{currentfill}%
\pgfsetlinewidth{1.003750pt}%
\definecolor{currentstroke}{rgb}{0.501961,0.000000,0.501961}%
\pgfsetstrokecolor{currentstroke}%
\pgfsetdash{}{0pt}%
\pgfpathmoveto{\pgfqpoint{2.639556in}{5.257692in}}%
\pgfpathcurveto{\pgfqpoint{2.647793in}{5.257692in}}{\pgfqpoint{2.655693in}{5.260964in}}{\pgfqpoint{2.661517in}{5.266788in}}%
\pgfpathcurveto{\pgfqpoint{2.667340in}{5.272612in}}{\pgfqpoint{2.670613in}{5.280512in}}{\pgfqpoint{2.670613in}{5.288748in}}%
\pgfpathcurveto{\pgfqpoint{2.670613in}{5.296984in}}{\pgfqpoint{2.667340in}{5.304884in}}{\pgfqpoint{2.661517in}{5.310708in}}%
\pgfpathcurveto{\pgfqpoint{2.655693in}{5.316532in}}{\pgfqpoint{2.647793in}{5.319805in}}{\pgfqpoint{2.639556in}{5.319805in}}%
\pgfpathcurveto{\pgfqpoint{2.631320in}{5.319805in}}{\pgfqpoint{2.623420in}{5.316532in}}{\pgfqpoint{2.617596in}{5.310708in}}%
\pgfpathcurveto{\pgfqpoint{2.611772in}{5.304884in}}{\pgfqpoint{2.608500in}{5.296984in}}{\pgfqpoint{2.608500in}{5.288748in}}%
\pgfpathcurveto{\pgfqpoint{2.608500in}{5.280512in}}{\pgfqpoint{2.611772in}{5.272612in}}{\pgfqpoint{2.617596in}{5.266788in}}%
\pgfpathcurveto{\pgfqpoint{2.623420in}{5.260964in}}{\pgfqpoint{2.631320in}{5.257692in}}{\pgfqpoint{2.639556in}{5.257692in}}%
\pgfpathclose%
\pgfusepath{stroke,fill}%
\end{pgfscope}%
\begin{pgfscope}%
\pgfpathrectangle{\pgfqpoint{0.894063in}{3.540000in}}{\pgfqpoint{6.713438in}{2.060556in}} %
\pgfusepath{clip}%
\pgfsetbuttcap%
\pgfsetroundjoin%
\definecolor{currentfill}{rgb}{0.501961,0.000000,0.501961}%
\pgfsetfillcolor{currentfill}%
\pgfsetlinewidth{1.003750pt}%
\definecolor{currentstroke}{rgb}{0.501961,0.000000,0.501961}%
\pgfsetstrokecolor{currentstroke}%
\pgfsetdash{}{0pt}%
\pgfpathmoveto{\pgfqpoint{1.699675in}{5.258167in}}%
\pgfpathcurveto{\pgfqpoint{1.707911in}{5.258167in}}{\pgfqpoint{1.715811in}{5.261439in}}{\pgfqpoint{1.721635in}{5.267263in}}%
\pgfpathcurveto{\pgfqpoint{1.727459in}{5.273087in}}{\pgfqpoint{1.730731in}{5.280987in}}{\pgfqpoint{1.730731in}{5.289223in}}%
\pgfpathcurveto{\pgfqpoint{1.730731in}{5.297460in}}{\pgfqpoint{1.727459in}{5.305360in}}{\pgfqpoint{1.721635in}{5.311184in}}%
\pgfpathcurveto{\pgfqpoint{1.715811in}{5.317008in}}{\pgfqpoint{1.707911in}{5.320280in}}{\pgfqpoint{1.699675in}{5.320280in}}%
\pgfpathcurveto{\pgfqpoint{1.691439in}{5.320280in}}{\pgfqpoint{1.683539in}{5.317008in}}{\pgfqpoint{1.677715in}{5.311184in}}%
\pgfpathcurveto{\pgfqpoint{1.671891in}{5.305360in}}{\pgfqpoint{1.668619in}{5.297460in}}{\pgfqpoint{1.668619in}{5.289223in}}%
\pgfpathcurveto{\pgfqpoint{1.668619in}{5.280987in}}{\pgfqpoint{1.671891in}{5.273087in}}{\pgfqpoint{1.677715in}{5.267263in}}%
\pgfpathcurveto{\pgfqpoint{1.683539in}{5.261439in}}{\pgfqpoint{1.691439in}{5.258167in}}{\pgfqpoint{1.699675in}{5.258167in}}%
\pgfpathclose%
\pgfusepath{stroke,fill}%
\end{pgfscope}%
\begin{pgfscope}%
\pgfpathrectangle{\pgfqpoint{0.894063in}{3.540000in}}{\pgfqpoint{6.713438in}{2.060556in}} %
\pgfusepath{clip}%
\pgfsetbuttcap%
\pgfsetroundjoin%
\definecolor{currentfill}{rgb}{0.501961,0.000000,0.501961}%
\pgfsetfillcolor{currentfill}%
\pgfsetlinewidth{1.003750pt}%
\definecolor{currentstroke}{rgb}{0.501961,0.000000,0.501961}%
\pgfsetstrokecolor{currentstroke}%
\pgfsetdash{}{0pt}%
\pgfpathmoveto{\pgfqpoint{1.162600in}{5.258306in}}%
\pgfpathcurveto{\pgfqpoint{1.170836in}{5.258306in}}{\pgfqpoint{1.178736in}{5.261578in}}{\pgfqpoint{1.184560in}{5.267402in}}%
\pgfpathcurveto{\pgfqpoint{1.190384in}{5.273226in}}{\pgfqpoint{1.193656in}{5.281126in}}{\pgfqpoint{1.193656in}{5.289362in}}%
\pgfpathcurveto{\pgfqpoint{1.193656in}{5.297598in}}{\pgfqpoint{1.190384in}{5.305499in}}{\pgfqpoint{1.184560in}{5.311322in}}%
\pgfpathcurveto{\pgfqpoint{1.178736in}{5.317146in}}{\pgfqpoint{1.170836in}{5.320419in}}{\pgfqpoint{1.162600in}{5.320419in}}%
\pgfpathcurveto{\pgfqpoint{1.154364in}{5.320419in}}{\pgfqpoint{1.146464in}{5.317146in}}{\pgfqpoint{1.140640in}{5.311322in}}%
\pgfpathcurveto{\pgfqpoint{1.134816in}{5.305499in}}{\pgfqpoint{1.131544in}{5.297598in}}{\pgfqpoint{1.131544in}{5.289362in}}%
\pgfpathcurveto{\pgfqpoint{1.131544in}{5.281126in}}{\pgfqpoint{1.134816in}{5.273226in}}{\pgfqpoint{1.140640in}{5.267402in}}%
\pgfpathcurveto{\pgfqpoint{1.146464in}{5.261578in}}{\pgfqpoint{1.154364in}{5.258306in}}{\pgfqpoint{1.162600in}{5.258306in}}%
\pgfpathclose%
\pgfusepath{stroke,fill}%
\end{pgfscope}%
\begin{pgfscope}%
\pgfpathrectangle{\pgfqpoint{0.894063in}{3.540000in}}{\pgfqpoint{6.713438in}{2.060556in}} %
\pgfusepath{clip}%
\pgfsetbuttcap%
\pgfsetroundjoin%
\definecolor{currentfill}{rgb}{0.501961,0.000000,0.501961}%
\pgfsetfillcolor{currentfill}%
\pgfsetlinewidth{1.003750pt}%
\definecolor{currentstroke}{rgb}{0.501961,0.000000,0.501961}%
\pgfsetstrokecolor{currentstroke}%
\pgfsetdash{}{0pt}%
\pgfpathmoveto{\pgfqpoint{1.833944in}{5.258159in}}%
\pgfpathcurveto{\pgfqpoint{1.842180in}{5.258159in}}{\pgfqpoint{1.850080in}{5.261431in}}{\pgfqpoint{1.855904in}{5.267255in}}%
\pgfpathcurveto{\pgfqpoint{1.861728in}{5.273079in}}{\pgfqpoint{1.865000in}{5.280979in}}{\pgfqpoint{1.865000in}{5.289215in}}%
\pgfpathcurveto{\pgfqpoint{1.865000in}{5.297452in}}{\pgfqpoint{1.861728in}{5.305352in}}{\pgfqpoint{1.855904in}{5.311175in}}%
\pgfpathcurveto{\pgfqpoint{1.850080in}{5.316999in}}{\pgfqpoint{1.842180in}{5.320272in}}{\pgfqpoint{1.833944in}{5.320272in}}%
\pgfpathcurveto{\pgfqpoint{1.825707in}{5.320272in}}{\pgfqpoint{1.817807in}{5.316999in}}{\pgfqpoint{1.811983in}{5.311175in}}%
\pgfpathcurveto{\pgfqpoint{1.806160in}{5.305352in}}{\pgfqpoint{1.802887in}{5.297452in}}{\pgfqpoint{1.802887in}{5.289215in}}%
\pgfpathcurveto{\pgfqpoint{1.802887in}{5.280979in}}{\pgfqpoint{1.806160in}{5.273079in}}{\pgfqpoint{1.811983in}{5.267255in}}%
\pgfpathcurveto{\pgfqpoint{1.817807in}{5.261431in}}{\pgfqpoint{1.825707in}{5.258159in}}{\pgfqpoint{1.833944in}{5.258159in}}%
\pgfpathclose%
\pgfusepath{stroke,fill}%
\end{pgfscope}%
\begin{pgfscope}%
\pgfpathrectangle{\pgfqpoint{0.894063in}{3.540000in}}{\pgfqpoint{6.713438in}{2.060556in}} %
\pgfusepath{clip}%
\pgfsetbuttcap%
\pgfsetroundjoin%
\definecolor{currentfill}{rgb}{0.501961,0.000000,0.501961}%
\pgfsetfillcolor{currentfill}%
\pgfsetlinewidth{1.003750pt}%
\definecolor{currentstroke}{rgb}{0.501961,0.000000,0.501961}%
\pgfsetstrokecolor{currentstroke}%
\pgfsetdash{}{0pt}%
\pgfpathmoveto{\pgfqpoint{5.996275in}{5.256815in}}%
\pgfpathcurveto{\pgfqpoint{6.004511in}{5.256815in}}{\pgfqpoint{6.012411in}{5.260088in}}{\pgfqpoint{6.018235in}{5.265911in}}%
\pgfpathcurveto{\pgfqpoint{6.024059in}{5.271735in}}{\pgfqpoint{6.027331in}{5.279635in}}{\pgfqpoint{6.027331in}{5.287872in}}%
\pgfpathcurveto{\pgfqpoint{6.027331in}{5.296108in}}{\pgfqpoint{6.024059in}{5.304008in}}{\pgfqpoint{6.018235in}{5.309832in}}%
\pgfpathcurveto{\pgfqpoint{6.012411in}{5.315656in}}{\pgfqpoint{6.004511in}{5.318928in}}{\pgfqpoint{5.996275in}{5.318928in}}%
\pgfpathcurveto{\pgfqpoint{5.988039in}{5.318928in}}{\pgfqpoint{5.980139in}{5.315656in}}{\pgfqpoint{5.974315in}{5.309832in}}%
\pgfpathcurveto{\pgfqpoint{5.968491in}{5.304008in}}{\pgfqpoint{5.965219in}{5.296108in}}{\pgfqpoint{5.965219in}{5.287872in}}%
\pgfpathcurveto{\pgfqpoint{5.965219in}{5.279635in}}{\pgfqpoint{5.968491in}{5.271735in}}{\pgfqpoint{5.974315in}{5.265911in}}%
\pgfpathcurveto{\pgfqpoint{5.980139in}{5.260088in}}{\pgfqpoint{5.988039in}{5.256815in}}{\pgfqpoint{5.996275in}{5.256815in}}%
\pgfpathclose%
\pgfusepath{stroke,fill}%
\end{pgfscope}%
\begin{pgfscope}%
\pgfpathrectangle{\pgfqpoint{0.894063in}{3.540000in}}{\pgfqpoint{6.713438in}{2.060556in}} %
\pgfusepath{clip}%
\pgfsetbuttcap%
\pgfsetroundjoin%
\definecolor{currentfill}{rgb}{0.501961,0.000000,0.501961}%
\pgfsetfillcolor{currentfill}%
\pgfsetlinewidth{1.003750pt}%
\definecolor{currentstroke}{rgb}{0.501961,0.000000,0.501961}%
\pgfsetstrokecolor{currentstroke}%
\pgfsetdash{}{0pt}%
\pgfpathmoveto{\pgfqpoint{6.399081in}{5.256807in}}%
\pgfpathcurveto{\pgfqpoint{6.407318in}{5.256807in}}{\pgfqpoint{6.415218in}{5.260079in}}{\pgfqpoint{6.421042in}{5.265903in}}%
\pgfpathcurveto{\pgfqpoint{6.426865in}{5.271727in}}{\pgfqpoint{6.430138in}{5.279627in}}{\pgfqpoint{6.430138in}{5.287864in}}%
\pgfpathcurveto{\pgfqpoint{6.430138in}{5.296100in}}{\pgfqpoint{6.426865in}{5.304000in}}{\pgfqpoint{6.421042in}{5.309824in}}%
\pgfpathcurveto{\pgfqpoint{6.415218in}{5.315648in}}{\pgfqpoint{6.407318in}{5.318920in}}{\pgfqpoint{6.399081in}{5.318920in}}%
\pgfpathcurveto{\pgfqpoint{6.390845in}{5.318920in}}{\pgfqpoint{6.382945in}{5.315648in}}{\pgfqpoint{6.377121in}{5.309824in}}%
\pgfpathcurveto{\pgfqpoint{6.371297in}{5.304000in}}{\pgfqpoint{6.368025in}{5.296100in}}{\pgfqpoint{6.368025in}{5.287864in}}%
\pgfpathcurveto{\pgfqpoint{6.368025in}{5.279627in}}{\pgfqpoint{6.371297in}{5.271727in}}{\pgfqpoint{6.377121in}{5.265903in}}%
\pgfpathcurveto{\pgfqpoint{6.382945in}{5.260079in}}{\pgfqpoint{6.390845in}{5.256807in}}{\pgfqpoint{6.399081in}{5.256807in}}%
\pgfpathclose%
\pgfusepath{stroke,fill}%
\end{pgfscope}%
\begin{pgfscope}%
\pgfpathrectangle{\pgfqpoint{0.894063in}{3.540000in}}{\pgfqpoint{6.713438in}{2.060556in}} %
\pgfusepath{clip}%
\pgfsetbuttcap%
\pgfsetroundjoin%
\definecolor{currentfill}{rgb}{0.501961,0.000000,0.501961}%
\pgfsetfillcolor{currentfill}%
\pgfsetlinewidth{1.003750pt}%
\definecolor{currentstroke}{rgb}{0.501961,0.000000,0.501961}%
\pgfsetstrokecolor{currentstroke}%
\pgfsetdash{}{0pt}%
\pgfpathmoveto{\pgfqpoint{4.787856in}{5.257545in}}%
\pgfpathcurveto{\pgfqpoint{4.796093in}{5.257545in}}{\pgfqpoint{4.803993in}{5.260817in}}{\pgfqpoint{4.809817in}{5.266641in}}%
\pgfpathcurveto{\pgfqpoint{4.815640in}{5.272465in}}{\pgfqpoint{4.818913in}{5.280365in}}{\pgfqpoint{4.818913in}{5.288601in}}%
\pgfpathcurveto{\pgfqpoint{4.818913in}{5.296837in}}{\pgfqpoint{4.815640in}{5.304738in}}{\pgfqpoint{4.809817in}{5.310561in}}%
\pgfpathcurveto{\pgfqpoint{4.803993in}{5.316385in}}{\pgfqpoint{4.796093in}{5.319658in}}{\pgfqpoint{4.787856in}{5.319658in}}%
\pgfpathcurveto{\pgfqpoint{4.779620in}{5.319658in}}{\pgfqpoint{4.771720in}{5.316385in}}{\pgfqpoint{4.765896in}{5.310561in}}%
\pgfpathcurveto{\pgfqpoint{4.760072in}{5.304738in}}{\pgfqpoint{4.756800in}{5.296837in}}{\pgfqpoint{4.756800in}{5.288601in}}%
\pgfpathcurveto{\pgfqpoint{4.756800in}{5.280365in}}{\pgfqpoint{4.760072in}{5.272465in}}{\pgfqpoint{4.765896in}{5.266641in}}%
\pgfpathcurveto{\pgfqpoint{4.771720in}{5.260817in}}{\pgfqpoint{4.779620in}{5.257545in}}{\pgfqpoint{4.787856in}{5.257545in}}%
\pgfpathclose%
\pgfusepath{stroke,fill}%
\end{pgfscope}%
\begin{pgfscope}%
\pgfpathrectangle{\pgfqpoint{0.894063in}{3.540000in}}{\pgfqpoint{6.713438in}{2.060556in}} %
\pgfusepath{clip}%
\pgfsetbuttcap%
\pgfsetroundjoin%
\definecolor{currentfill}{rgb}{0.501961,0.000000,0.501961}%
\pgfsetfillcolor{currentfill}%
\pgfsetlinewidth{1.003750pt}%
\definecolor{currentstroke}{rgb}{0.501961,0.000000,0.501961}%
\pgfsetstrokecolor{currentstroke}%
\pgfsetdash{}{0pt}%
\pgfpathmoveto{\pgfqpoint{4.922125in}{5.257545in}}%
\pgfpathcurveto{\pgfqpoint{4.930361in}{5.257545in}}{\pgfqpoint{4.938261in}{5.260817in}}{\pgfqpoint{4.944085in}{5.266641in}}%
\pgfpathcurveto{\pgfqpoint{4.949909in}{5.272465in}}{\pgfqpoint{4.953181in}{5.280365in}}{\pgfqpoint{4.953181in}{5.288601in}}%
\pgfpathcurveto{\pgfqpoint{4.953181in}{5.296837in}}{\pgfqpoint{4.949909in}{5.304738in}}{\pgfqpoint{4.944085in}{5.310561in}}%
\pgfpathcurveto{\pgfqpoint{4.938261in}{5.316385in}}{\pgfqpoint{4.930361in}{5.319658in}}{\pgfqpoint{4.922125in}{5.319658in}}%
\pgfpathcurveto{\pgfqpoint{4.913889in}{5.319658in}}{\pgfqpoint{4.905989in}{5.316385in}}{\pgfqpoint{4.900165in}{5.310561in}}%
\pgfpathcurveto{\pgfqpoint{4.894341in}{5.304738in}}{\pgfqpoint{4.891069in}{5.296837in}}{\pgfqpoint{4.891069in}{5.288601in}}%
\pgfpathcurveto{\pgfqpoint{4.891069in}{5.280365in}}{\pgfqpoint{4.894341in}{5.272465in}}{\pgfqpoint{4.900165in}{5.266641in}}%
\pgfpathcurveto{\pgfqpoint{4.905989in}{5.260817in}}{\pgfqpoint{4.913889in}{5.257545in}}{\pgfqpoint{4.922125in}{5.257545in}}%
\pgfpathclose%
\pgfusepath{stroke,fill}%
\end{pgfscope}%
\begin{pgfscope}%
\pgfpathrectangle{\pgfqpoint{0.894063in}{3.540000in}}{\pgfqpoint{6.713438in}{2.060556in}} %
\pgfusepath{clip}%
\pgfsetbuttcap%
\pgfsetroundjoin%
\definecolor{currentfill}{rgb}{0.501961,0.000000,0.501961}%
\pgfsetfillcolor{currentfill}%
\pgfsetlinewidth{1.003750pt}%
\definecolor{currentstroke}{rgb}{0.501961,0.000000,0.501961}%
\pgfsetstrokecolor{currentstroke}%
\pgfsetdash{}{0pt}%
\pgfpathmoveto{\pgfqpoint{6.130544in}{5.256832in}}%
\pgfpathcurveto{\pgfqpoint{6.138780in}{5.256832in}}{\pgfqpoint{6.146680in}{5.260104in}}{\pgfqpoint{6.152504in}{5.265928in}}%
\pgfpathcurveto{\pgfqpoint{6.158328in}{5.271752in}}{\pgfqpoint{6.161600in}{5.279652in}}{\pgfqpoint{6.161600in}{5.287888in}}%
\pgfpathcurveto{\pgfqpoint{6.161600in}{5.296125in}}{\pgfqpoint{6.158328in}{5.304025in}}{\pgfqpoint{6.152504in}{5.309848in}}%
\pgfpathcurveto{\pgfqpoint{6.146680in}{5.315672in}}{\pgfqpoint{6.138780in}{5.318945in}}{\pgfqpoint{6.130544in}{5.318945in}}%
\pgfpathcurveto{\pgfqpoint{6.122307in}{5.318945in}}{\pgfqpoint{6.114407in}{5.315672in}}{\pgfqpoint{6.108583in}{5.309848in}}%
\pgfpathcurveto{\pgfqpoint{6.102760in}{5.304025in}}{\pgfqpoint{6.099487in}{5.296125in}}{\pgfqpoint{6.099487in}{5.287888in}}%
\pgfpathcurveto{\pgfqpoint{6.099487in}{5.279652in}}{\pgfqpoint{6.102760in}{5.271752in}}{\pgfqpoint{6.108583in}{5.265928in}}%
\pgfpathcurveto{\pgfqpoint{6.114407in}{5.260104in}}{\pgfqpoint{6.122307in}{5.256832in}}{\pgfqpoint{6.130544in}{5.256832in}}%
\pgfpathclose%
\pgfusepath{stroke,fill}%
\end{pgfscope}%
\begin{pgfscope}%
\pgfpathrectangle{\pgfqpoint{0.894063in}{3.540000in}}{\pgfqpoint{6.713438in}{2.060556in}} %
\pgfusepath{clip}%
\pgfsetbuttcap%
\pgfsetroundjoin%
\definecolor{currentfill}{rgb}{0.501961,0.000000,0.501961}%
\pgfsetfillcolor{currentfill}%
\pgfsetlinewidth{1.003750pt}%
\definecolor{currentstroke}{rgb}{0.501961,0.000000,0.501961}%
\pgfsetstrokecolor{currentstroke}%
\pgfsetdash{}{0pt}%
\pgfpathmoveto{\pgfqpoint{5.727738in}{5.257530in}}%
\pgfpathcurveto{\pgfqpoint{5.735974in}{5.257530in}}{\pgfqpoint{5.743874in}{5.260802in}}{\pgfqpoint{5.749698in}{5.266626in}}%
\pgfpathcurveto{\pgfqpoint{5.755522in}{5.272450in}}{\pgfqpoint{5.758794in}{5.280350in}}{\pgfqpoint{5.758794in}{5.288586in}}%
\pgfpathcurveto{\pgfqpoint{5.758794in}{5.296822in}}{\pgfqpoint{5.755522in}{5.304722in}}{\pgfqpoint{5.749698in}{5.310546in}}%
\pgfpathcurveto{\pgfqpoint{5.743874in}{5.316370in}}{\pgfqpoint{5.735974in}{5.319643in}}{\pgfqpoint{5.727738in}{5.319643in}}%
\pgfpathcurveto{\pgfqpoint{5.719501in}{5.319643in}}{\pgfqpoint{5.711601in}{5.316370in}}{\pgfqpoint{5.705777in}{5.310546in}}%
\pgfpathcurveto{\pgfqpoint{5.699953in}{5.304722in}}{\pgfqpoint{5.696681in}{5.296822in}}{\pgfqpoint{5.696681in}{5.288586in}}%
\pgfpathcurveto{\pgfqpoint{5.696681in}{5.280350in}}{\pgfqpoint{5.699953in}{5.272450in}}{\pgfqpoint{5.705777in}{5.266626in}}%
\pgfpathcurveto{\pgfqpoint{5.711601in}{5.260802in}}{\pgfqpoint{5.719501in}{5.257530in}}{\pgfqpoint{5.727738in}{5.257530in}}%
\pgfpathclose%
\pgfusepath{stroke,fill}%
\end{pgfscope}%
\begin{pgfscope}%
\pgfpathrectangle{\pgfqpoint{0.894063in}{3.540000in}}{\pgfqpoint{6.713438in}{2.060556in}} %
\pgfusepath{clip}%
\pgfsetbuttcap%
\pgfsetroundjoin%
\definecolor{currentfill}{rgb}{0.501961,0.000000,0.501961}%
\pgfsetfillcolor{currentfill}%
\pgfsetlinewidth{1.003750pt}%
\definecolor{currentstroke}{rgb}{0.501961,0.000000,0.501961}%
\pgfsetstrokecolor{currentstroke}%
\pgfsetdash{}{0pt}%
\pgfpathmoveto{\pgfqpoint{1.028331in}{5.258352in}}%
\pgfpathcurveto{\pgfqpoint{1.036568in}{5.258352in}}{\pgfqpoint{1.044468in}{5.261625in}}{\pgfqpoint{1.050292in}{5.267449in}}%
\pgfpathcurveto{\pgfqpoint{1.056115in}{5.273273in}}{\pgfqpoint{1.059388in}{5.281173in}}{\pgfqpoint{1.059388in}{5.289409in}}%
\pgfpathcurveto{\pgfqpoint{1.059388in}{5.297645in}}{\pgfqpoint{1.056115in}{5.305545in}}{\pgfqpoint{1.050292in}{5.311369in}}%
\pgfpathcurveto{\pgfqpoint{1.044468in}{5.317193in}}{\pgfqpoint{1.036568in}{5.320465in}}{\pgfqpoint{1.028331in}{5.320465in}}%
\pgfpathcurveto{\pgfqpoint{1.020095in}{5.320465in}}{\pgfqpoint{1.012195in}{5.317193in}}{\pgfqpoint{1.006371in}{5.311369in}}%
\pgfpathcurveto{\pgfqpoint{1.000547in}{5.305545in}}{\pgfqpoint{0.997275in}{5.297645in}}{\pgfqpoint{0.997275in}{5.289409in}}%
\pgfpathcurveto{\pgfqpoint{0.997275in}{5.281173in}}{\pgfqpoint{1.000547in}{5.273273in}}{\pgfqpoint{1.006371in}{5.267449in}}%
\pgfpathcurveto{\pgfqpoint{1.012195in}{5.261625in}}{\pgfqpoint{1.020095in}{5.258352in}}{\pgfqpoint{1.028331in}{5.258352in}}%
\pgfpathclose%
\pgfusepath{stroke,fill}%
\end{pgfscope}%
\begin{pgfscope}%
\pgfpathrectangle{\pgfqpoint{0.894063in}{3.540000in}}{\pgfqpoint{6.713438in}{2.060556in}} %
\pgfusepath{clip}%
\pgfsetbuttcap%
\pgfsetroundjoin%
\definecolor{currentfill}{rgb}{0.501961,0.000000,0.501961}%
\pgfsetfillcolor{currentfill}%
\pgfsetlinewidth{1.003750pt}%
\definecolor{currentstroke}{rgb}{0.501961,0.000000,0.501961}%
\pgfsetstrokecolor{currentstroke}%
\pgfsetdash{}{0pt}%
\pgfpathmoveto{\pgfqpoint{5.324931in}{5.257534in}}%
\pgfpathcurveto{\pgfqpoint{5.333168in}{5.257534in}}{\pgfqpoint{5.341068in}{5.260806in}}{\pgfqpoint{5.346892in}{5.266630in}}%
\pgfpathcurveto{\pgfqpoint{5.352715in}{5.272454in}}{\pgfqpoint{5.355988in}{5.280354in}}{\pgfqpoint{5.355988in}{5.288590in}}%
\pgfpathcurveto{\pgfqpoint{5.355988in}{5.296826in}}{\pgfqpoint{5.352715in}{5.304727in}}{\pgfqpoint{5.346892in}{5.310550in}}%
\pgfpathcurveto{\pgfqpoint{5.341068in}{5.316374in}}{\pgfqpoint{5.333168in}{5.319647in}}{\pgfqpoint{5.324931in}{5.319647in}}%
\pgfpathcurveto{\pgfqpoint{5.316695in}{5.319647in}}{\pgfqpoint{5.308795in}{5.316374in}}{\pgfqpoint{5.302971in}{5.310550in}}%
\pgfpathcurveto{\pgfqpoint{5.297147in}{5.304727in}}{\pgfqpoint{5.293875in}{5.296826in}}{\pgfqpoint{5.293875in}{5.288590in}}%
\pgfpathcurveto{\pgfqpoint{5.293875in}{5.280354in}}{\pgfqpoint{5.297147in}{5.272454in}}{\pgfqpoint{5.302971in}{5.266630in}}%
\pgfpathcurveto{\pgfqpoint{5.308795in}{5.260806in}}{\pgfqpoint{5.316695in}{5.257534in}}{\pgfqpoint{5.324931in}{5.257534in}}%
\pgfpathclose%
\pgfusepath{stroke,fill}%
\end{pgfscope}%
\begin{pgfscope}%
\pgfpathrectangle{\pgfqpoint{0.894063in}{3.540000in}}{\pgfqpoint{6.713438in}{2.060556in}} %
\pgfusepath{clip}%
\pgfsetbuttcap%
\pgfsetroundjoin%
\definecolor{currentfill}{rgb}{0.501961,0.000000,0.501961}%
\pgfsetfillcolor{currentfill}%
\pgfsetlinewidth{1.003750pt}%
\definecolor{currentstroke}{rgb}{0.501961,0.000000,0.501961}%
\pgfsetstrokecolor{currentstroke}%
\pgfsetdash{}{0pt}%
\pgfpathmoveto{\pgfqpoint{7.338963in}{5.256049in}}%
\pgfpathcurveto{\pgfqpoint{7.347199in}{5.256049in}}{\pgfqpoint{7.355099in}{5.259321in}}{\pgfqpoint{7.360923in}{5.265145in}}%
\pgfpathcurveto{\pgfqpoint{7.366747in}{5.270969in}}{\pgfqpoint{7.370019in}{5.278869in}}{\pgfqpoint{7.370019in}{5.287105in}}%
\pgfpathcurveto{\pgfqpoint{7.370019in}{5.295341in}}{\pgfqpoint{7.366747in}{5.303242in}}{\pgfqpoint{7.360923in}{5.309065in}}%
\pgfpathcurveto{\pgfqpoint{7.355099in}{5.314889in}}{\pgfqpoint{7.347199in}{5.318162in}}{\pgfqpoint{7.338963in}{5.318162in}}%
\pgfpathcurveto{\pgfqpoint{7.330726in}{5.318162in}}{\pgfqpoint{7.322826in}{5.314889in}}{\pgfqpoint{7.317002in}{5.309065in}}%
\pgfpathcurveto{\pgfqpoint{7.311178in}{5.303242in}}{\pgfqpoint{7.307906in}{5.295341in}}{\pgfqpoint{7.307906in}{5.287105in}}%
\pgfpathcurveto{\pgfqpoint{7.307906in}{5.278869in}}{\pgfqpoint{7.311178in}{5.270969in}}{\pgfqpoint{7.317002in}{5.265145in}}%
\pgfpathcurveto{\pgfqpoint{7.322826in}{5.259321in}}{\pgfqpoint{7.330726in}{5.256049in}}{\pgfqpoint{7.338963in}{5.256049in}}%
\pgfpathclose%
\pgfusepath{stroke,fill}%
\end{pgfscope}%
\begin{pgfscope}%
\pgfpathrectangle{\pgfqpoint{0.894063in}{3.540000in}}{\pgfqpoint{6.713438in}{2.060556in}} %
\pgfusepath{clip}%
\pgfsetbuttcap%
\pgfsetroundjoin%
\definecolor{currentfill}{rgb}{0.501961,0.000000,0.501961}%
\pgfsetfillcolor{currentfill}%
\pgfsetlinewidth{1.003750pt}%
\definecolor{currentstroke}{rgb}{0.501961,0.000000,0.501961}%
\pgfsetstrokecolor{currentstroke}%
\pgfsetdash{}{0pt}%
\pgfpathmoveto{\pgfqpoint{7.204694in}{5.256058in}}%
\pgfpathcurveto{\pgfqpoint{7.212930in}{5.256058in}}{\pgfqpoint{7.220830in}{5.259331in}}{\pgfqpoint{7.226654in}{5.265155in}}%
\pgfpathcurveto{\pgfqpoint{7.232478in}{5.270979in}}{\pgfqpoint{7.235750in}{5.278879in}}{\pgfqpoint{7.235750in}{5.287115in}}%
\pgfpathcurveto{\pgfqpoint{7.235750in}{5.295351in}}{\pgfqpoint{7.232478in}{5.303251in}}{\pgfqpoint{7.226654in}{5.309075in}}%
\pgfpathcurveto{\pgfqpoint{7.220830in}{5.314899in}}{\pgfqpoint{7.212930in}{5.318171in}}{\pgfqpoint{7.204694in}{5.318171in}}%
\pgfpathcurveto{\pgfqpoint{7.196457in}{5.318171in}}{\pgfqpoint{7.188557in}{5.314899in}}{\pgfqpoint{7.182733in}{5.309075in}}%
\pgfpathcurveto{\pgfqpoint{7.176910in}{5.303251in}}{\pgfqpoint{7.173637in}{5.295351in}}{\pgfqpoint{7.173637in}{5.287115in}}%
\pgfpathcurveto{\pgfqpoint{7.173637in}{5.278879in}}{\pgfqpoint{7.176910in}{5.270979in}}{\pgfqpoint{7.182733in}{5.265155in}}%
\pgfpathcurveto{\pgfqpoint{7.188557in}{5.259331in}}{\pgfqpoint{7.196457in}{5.256058in}}{\pgfqpoint{7.204694in}{5.256058in}}%
\pgfpathclose%
\pgfusepath{stroke,fill}%
\end{pgfscope}%
\begin{pgfscope}%
\pgfpathrectangle{\pgfqpoint{0.894063in}{3.540000in}}{\pgfqpoint{6.713438in}{2.060556in}} %
\pgfusepath{clip}%
\pgfsetbuttcap%
\pgfsetroundjoin%
\definecolor{currentfill}{rgb}{0.501961,0.000000,0.501961}%
\pgfsetfillcolor{currentfill}%
\pgfsetlinewidth{1.003750pt}%
\definecolor{currentstroke}{rgb}{0.501961,0.000000,0.501961}%
\pgfsetstrokecolor{currentstroke}%
\pgfsetdash{}{0pt}%
\pgfpathmoveto{\pgfqpoint{6.264813in}{5.256811in}}%
\pgfpathcurveto{\pgfqpoint{6.273049in}{5.256811in}}{\pgfqpoint{6.280949in}{5.260083in}}{\pgfqpoint{6.286773in}{5.265907in}}%
\pgfpathcurveto{\pgfqpoint{6.292597in}{5.271731in}}{\pgfqpoint{6.295869in}{5.279631in}}{\pgfqpoint{6.295869in}{5.287868in}}%
\pgfpathcurveto{\pgfqpoint{6.295869in}{5.296104in}}{\pgfqpoint{6.292597in}{5.304004in}}{\pgfqpoint{6.286773in}{5.309828in}}%
\pgfpathcurveto{\pgfqpoint{6.280949in}{5.315652in}}{\pgfqpoint{6.273049in}{5.318924in}}{\pgfqpoint{6.264813in}{5.318924in}}%
\pgfpathcurveto{\pgfqpoint{6.256576in}{5.318924in}}{\pgfqpoint{6.248676in}{5.315652in}}{\pgfqpoint{6.242852in}{5.309828in}}%
\pgfpathcurveto{\pgfqpoint{6.237028in}{5.304004in}}{\pgfqpoint{6.233756in}{5.296104in}}{\pgfqpoint{6.233756in}{5.287868in}}%
\pgfpathcurveto{\pgfqpoint{6.233756in}{5.279631in}}{\pgfqpoint{6.237028in}{5.271731in}}{\pgfqpoint{6.242852in}{5.265907in}}%
\pgfpathcurveto{\pgfqpoint{6.248676in}{5.260083in}}{\pgfqpoint{6.256576in}{5.256811in}}{\pgfqpoint{6.264813in}{5.256811in}}%
\pgfpathclose%
\pgfusepath{stroke,fill}%
\end{pgfscope}%
\begin{pgfscope}%
\pgfpathrectangle{\pgfqpoint{0.894063in}{3.540000in}}{\pgfqpoint{6.713438in}{2.060556in}} %
\pgfusepath{clip}%
\pgfsetbuttcap%
\pgfsetroundjoin%
\definecolor{currentfill}{rgb}{0.501961,0.000000,0.501961}%
\pgfsetfillcolor{currentfill}%
\pgfsetlinewidth{1.003750pt}%
\definecolor{currentstroke}{rgb}{0.501961,0.000000,0.501961}%
\pgfsetstrokecolor{currentstroke}%
\pgfsetdash{}{0pt}%
\pgfpathmoveto{\pgfqpoint{7.473231in}{5.256043in}}%
\pgfpathcurveto{\pgfqpoint{7.481468in}{5.256043in}}{\pgfqpoint{7.489368in}{5.259316in}}{\pgfqpoint{7.495192in}{5.265139in}}%
\pgfpathcurveto{\pgfqpoint{7.501015in}{5.270963in}}{\pgfqpoint{7.504288in}{5.278863in}}{\pgfqpoint{7.504288in}{5.287100in}}%
\pgfpathcurveto{\pgfqpoint{7.504288in}{5.295336in}}{\pgfqpoint{7.501015in}{5.303236in}}{\pgfqpoint{7.495192in}{5.309060in}}%
\pgfpathcurveto{\pgfqpoint{7.489368in}{5.314884in}}{\pgfqpoint{7.481468in}{5.318156in}}{\pgfqpoint{7.473231in}{5.318156in}}%
\pgfpathcurveto{\pgfqpoint{7.464995in}{5.318156in}}{\pgfqpoint{7.457095in}{5.314884in}}{\pgfqpoint{7.451271in}{5.309060in}}%
\pgfpathcurveto{\pgfqpoint{7.445447in}{5.303236in}}{\pgfqpoint{7.442175in}{5.295336in}}{\pgfqpoint{7.442175in}{5.287100in}}%
\pgfpathcurveto{\pgfqpoint{7.442175in}{5.278863in}}{\pgfqpoint{7.445447in}{5.270963in}}{\pgfqpoint{7.451271in}{5.265139in}}%
\pgfpathcurveto{\pgfqpoint{7.457095in}{5.259316in}}{\pgfqpoint{7.464995in}{5.256043in}}{\pgfqpoint{7.473231in}{5.256043in}}%
\pgfpathclose%
\pgfusepath{stroke,fill}%
\end{pgfscope}%
\begin{pgfscope}%
\pgfpathrectangle{\pgfqpoint{0.894063in}{3.540000in}}{\pgfqpoint{6.713438in}{2.060556in}} %
\pgfusepath{clip}%
\pgfsetbuttcap%
\pgfsetroundjoin%
\definecolor{currentfill}{rgb}{0.501961,0.000000,0.501961}%
\pgfsetfillcolor{currentfill}%
\pgfsetlinewidth{1.003750pt}%
\definecolor{currentstroke}{rgb}{0.501961,0.000000,0.501961}%
\pgfsetstrokecolor{currentstroke}%
\pgfsetdash{}{0pt}%
\pgfpathmoveto{\pgfqpoint{5.056394in}{5.257542in}}%
\pgfpathcurveto{\pgfqpoint{5.064630in}{5.257542in}}{\pgfqpoint{5.072530in}{5.260814in}}{\pgfqpoint{5.078354in}{5.266638in}}%
\pgfpathcurveto{\pgfqpoint{5.084178in}{5.272462in}}{\pgfqpoint{5.087450in}{5.280362in}}{\pgfqpoint{5.087450in}{5.288598in}}%
\pgfpathcurveto{\pgfqpoint{5.087450in}{5.296835in}}{\pgfqpoint{5.084178in}{5.304735in}}{\pgfqpoint{5.078354in}{5.310559in}}%
\pgfpathcurveto{\pgfqpoint{5.072530in}{5.316383in}}{\pgfqpoint{5.064630in}{5.319655in}}{\pgfqpoint{5.056394in}{5.319655in}}%
\pgfpathcurveto{\pgfqpoint{5.048157in}{5.319655in}}{\pgfqpoint{5.040257in}{5.316383in}}{\pgfqpoint{5.034433in}{5.310559in}}%
\pgfpathcurveto{\pgfqpoint{5.028610in}{5.304735in}}{\pgfqpoint{5.025337in}{5.296835in}}{\pgfqpoint{5.025337in}{5.288598in}}%
\pgfpathcurveto{\pgfqpoint{5.025337in}{5.280362in}}{\pgfqpoint{5.028610in}{5.272462in}}{\pgfqpoint{5.034433in}{5.266638in}}%
\pgfpathcurveto{\pgfqpoint{5.040257in}{5.260814in}}{\pgfqpoint{5.048157in}{5.257542in}}{\pgfqpoint{5.056394in}{5.257542in}}%
\pgfpathclose%
\pgfusepath{stroke,fill}%
\end{pgfscope}%
\begin{pgfscope}%
\pgfpathrectangle{\pgfqpoint{0.894063in}{3.540000in}}{\pgfqpoint{6.713438in}{2.060556in}} %
\pgfusepath{clip}%
\pgfsetbuttcap%
\pgfsetroundjoin%
\definecolor{currentfill}{rgb}{0.501961,0.000000,0.501961}%
\pgfsetfillcolor{currentfill}%
\pgfsetlinewidth{1.003750pt}%
\definecolor{currentstroke}{rgb}{0.501961,0.000000,0.501961}%
\pgfsetstrokecolor{currentstroke}%
\pgfsetdash{}{0pt}%
\pgfpathmoveto{\pgfqpoint{2.908094in}{5.257692in}}%
\pgfpathcurveto{\pgfqpoint{2.916330in}{5.257692in}}{\pgfqpoint{2.924230in}{5.260964in}}{\pgfqpoint{2.930054in}{5.266788in}}%
\pgfpathcurveto{\pgfqpoint{2.935878in}{5.272612in}}{\pgfqpoint{2.939150in}{5.280512in}}{\pgfqpoint{2.939150in}{5.288748in}}%
\pgfpathcurveto{\pgfqpoint{2.939150in}{5.296984in}}{\pgfqpoint{2.935878in}{5.304884in}}{\pgfqpoint{2.930054in}{5.310708in}}%
\pgfpathcurveto{\pgfqpoint{2.924230in}{5.316532in}}{\pgfqpoint{2.916330in}{5.319805in}}{\pgfqpoint{2.908094in}{5.319805in}}%
\pgfpathcurveto{\pgfqpoint{2.899857in}{5.319805in}}{\pgfqpoint{2.891957in}{5.316532in}}{\pgfqpoint{2.886133in}{5.310708in}}%
\pgfpathcurveto{\pgfqpoint{2.880310in}{5.304884in}}{\pgfqpoint{2.877037in}{5.296984in}}{\pgfqpoint{2.877037in}{5.288748in}}%
\pgfpathcurveto{\pgfqpoint{2.877037in}{5.280512in}}{\pgfqpoint{2.880310in}{5.272612in}}{\pgfqpoint{2.886133in}{5.266788in}}%
\pgfpathcurveto{\pgfqpoint{2.891957in}{5.260964in}}{\pgfqpoint{2.899857in}{5.257692in}}{\pgfqpoint{2.908094in}{5.257692in}}%
\pgfpathclose%
\pgfusepath{stroke,fill}%
\end{pgfscope}%
\begin{pgfscope}%
\pgfpathrectangle{\pgfqpoint{0.894063in}{3.540000in}}{\pgfqpoint{6.713438in}{2.060556in}} %
\pgfusepath{clip}%
\pgfsetbuttcap%
\pgfsetroundjoin%
\definecolor{currentfill}{rgb}{0.501961,0.000000,0.501961}%
\pgfsetfillcolor{currentfill}%
\pgfsetlinewidth{1.003750pt}%
\definecolor{currentstroke}{rgb}{0.501961,0.000000,0.501961}%
\pgfsetstrokecolor{currentstroke}%
\pgfsetdash{}{0pt}%
\pgfpathmoveto{\pgfqpoint{3.445169in}{5.257611in}}%
\pgfpathcurveto{\pgfqpoint{3.453405in}{5.257611in}}{\pgfqpoint{3.461305in}{5.260883in}}{\pgfqpoint{3.467129in}{5.266707in}}%
\pgfpathcurveto{\pgfqpoint{3.472953in}{5.272531in}}{\pgfqpoint{3.476225in}{5.280431in}}{\pgfqpoint{3.476225in}{5.288667in}}%
\pgfpathcurveto{\pgfqpoint{3.476225in}{5.296903in}}{\pgfqpoint{3.472953in}{5.304803in}}{\pgfqpoint{3.467129in}{5.310627in}}%
\pgfpathcurveto{\pgfqpoint{3.461305in}{5.316451in}}{\pgfqpoint{3.453405in}{5.319724in}}{\pgfqpoint{3.445169in}{5.319724in}}%
\pgfpathcurveto{\pgfqpoint{3.436932in}{5.319724in}}{\pgfqpoint{3.429032in}{5.316451in}}{\pgfqpoint{3.423208in}{5.310627in}}%
\pgfpathcurveto{\pgfqpoint{3.417385in}{5.304803in}}{\pgfqpoint{3.414112in}{5.296903in}}{\pgfqpoint{3.414112in}{5.288667in}}%
\pgfpathcurveto{\pgfqpoint{3.414112in}{5.280431in}}{\pgfqpoint{3.417385in}{5.272531in}}{\pgfqpoint{3.423208in}{5.266707in}}%
\pgfpathcurveto{\pgfqpoint{3.429032in}{5.260883in}}{\pgfqpoint{3.436932in}{5.257611in}}{\pgfqpoint{3.445169in}{5.257611in}}%
\pgfpathclose%
\pgfusepath{stroke,fill}%
\end{pgfscope}%
\begin{pgfscope}%
\pgfpathrectangle{\pgfqpoint{0.894063in}{3.540000in}}{\pgfqpoint{6.713438in}{2.060556in}} %
\pgfusepath{clip}%
\pgfsetbuttcap%
\pgfsetroundjoin%
\definecolor{currentfill}{rgb}{0.501961,0.000000,0.501961}%
\pgfsetfillcolor{currentfill}%
\pgfsetlinewidth{1.003750pt}%
\definecolor{currentstroke}{rgb}{0.501961,0.000000,0.501961}%
\pgfsetstrokecolor{currentstroke}%
\pgfsetdash{}{0pt}%
\pgfpathmoveto{\pgfqpoint{4.116513in}{5.257576in}}%
\pgfpathcurveto{\pgfqpoint{4.124749in}{5.257576in}}{\pgfqpoint{4.132649in}{5.260849in}}{\pgfqpoint{4.138473in}{5.266673in}}%
\pgfpathcurveto{\pgfqpoint{4.144297in}{5.272496in}}{\pgfqpoint{4.147569in}{5.280396in}}{\pgfqpoint{4.147569in}{5.288633in}}%
\pgfpathcurveto{\pgfqpoint{4.147569in}{5.296869in}}{\pgfqpoint{4.144297in}{5.304769in}}{\pgfqpoint{4.138473in}{5.310593in}}%
\pgfpathcurveto{\pgfqpoint{4.132649in}{5.316417in}}{\pgfqpoint{4.124749in}{5.319689in}}{\pgfqpoint{4.116513in}{5.319689in}}%
\pgfpathcurveto{\pgfqpoint{4.108276in}{5.319689in}}{\pgfqpoint{4.100376in}{5.316417in}}{\pgfqpoint{4.094552in}{5.310593in}}%
\pgfpathcurveto{\pgfqpoint{4.088728in}{5.304769in}}{\pgfqpoint{4.085456in}{5.296869in}}{\pgfqpoint{4.085456in}{5.288633in}}%
\pgfpathcurveto{\pgfqpoint{4.085456in}{5.280396in}}{\pgfqpoint{4.088728in}{5.272496in}}{\pgfqpoint{4.094552in}{5.266673in}}%
\pgfpathcurveto{\pgfqpoint{4.100376in}{5.260849in}}{\pgfqpoint{4.108276in}{5.257576in}}{\pgfqpoint{4.116513in}{5.257576in}}%
\pgfpathclose%
\pgfusepath{stroke,fill}%
\end{pgfscope}%
\begin{pgfscope}%
\pgfpathrectangle{\pgfqpoint{0.894063in}{3.540000in}}{\pgfqpoint{6.713438in}{2.060556in}} %
\pgfusepath{clip}%
\pgfsetbuttcap%
\pgfsetroundjoin%
\definecolor{currentfill}{rgb}{0.501961,0.000000,0.501961}%
\pgfsetfillcolor{currentfill}%
\pgfsetlinewidth{1.003750pt}%
\definecolor{currentstroke}{rgb}{0.501961,0.000000,0.501961}%
\pgfsetstrokecolor{currentstroke}%
\pgfsetdash{}{0pt}%
\pgfpathmoveto{\pgfqpoint{1.431138in}{5.258199in}}%
\pgfpathcurveto{\pgfqpoint{1.439374in}{5.258199in}}{\pgfqpoint{1.447274in}{5.261471in}}{\pgfqpoint{1.453098in}{5.267295in}}%
\pgfpathcurveto{\pgfqpoint{1.458922in}{5.273119in}}{\pgfqpoint{1.462194in}{5.281019in}}{\pgfqpoint{1.462194in}{5.289255in}}%
\pgfpathcurveto{\pgfqpoint{1.462194in}{5.297491in}}{\pgfqpoint{1.458922in}{5.305391in}}{\pgfqpoint{1.453098in}{5.311215in}}%
\pgfpathcurveto{\pgfqpoint{1.447274in}{5.317039in}}{\pgfqpoint{1.439374in}{5.320312in}}{\pgfqpoint{1.431138in}{5.320312in}}%
\pgfpathcurveto{\pgfqpoint{1.422901in}{5.320312in}}{\pgfqpoint{1.415001in}{5.317039in}}{\pgfqpoint{1.409177in}{5.311215in}}%
\pgfpathcurveto{\pgfqpoint{1.403353in}{5.305391in}}{\pgfqpoint{1.400081in}{5.297491in}}{\pgfqpoint{1.400081in}{5.289255in}}%
\pgfpathcurveto{\pgfqpoint{1.400081in}{5.281019in}}{\pgfqpoint{1.403353in}{5.273119in}}{\pgfqpoint{1.409177in}{5.267295in}}%
\pgfpathcurveto{\pgfqpoint{1.415001in}{5.261471in}}{\pgfqpoint{1.422901in}{5.258199in}}{\pgfqpoint{1.431138in}{5.258199in}}%
\pgfpathclose%
\pgfusepath{stroke,fill}%
\end{pgfscope}%
\begin{pgfscope}%
\pgfpathrectangle{\pgfqpoint{0.894063in}{3.540000in}}{\pgfqpoint{6.713438in}{2.060556in}} %
\pgfusepath{clip}%
\pgfsetbuttcap%
\pgfsetroundjoin%
\definecolor{currentfill}{rgb}{0.501961,0.000000,0.501961}%
\pgfsetfillcolor{currentfill}%
\pgfsetlinewidth{1.003750pt}%
\definecolor{currentstroke}{rgb}{0.501961,0.000000,0.501961}%
\pgfsetstrokecolor{currentstroke}%
\pgfsetdash{}{0pt}%
\pgfpathmoveto{\pgfqpoint{2.773825in}{5.257688in}}%
\pgfpathcurveto{\pgfqpoint{2.782061in}{5.257688in}}{\pgfqpoint{2.789961in}{5.260960in}}{\pgfqpoint{2.795785in}{5.266784in}}%
\pgfpathcurveto{\pgfqpoint{2.801609in}{5.272608in}}{\pgfqpoint{2.804881in}{5.280508in}}{\pgfqpoint{2.804881in}{5.288744in}}%
\pgfpathcurveto{\pgfqpoint{2.804881in}{5.296980in}}{\pgfqpoint{2.801609in}{5.304880in}}{\pgfqpoint{2.795785in}{5.310704in}}%
\pgfpathcurveto{\pgfqpoint{2.789961in}{5.316528in}}{\pgfqpoint{2.782061in}{5.319801in}}{\pgfqpoint{2.773825in}{5.319801in}}%
\pgfpathcurveto{\pgfqpoint{2.765589in}{5.319801in}}{\pgfqpoint{2.757689in}{5.316528in}}{\pgfqpoint{2.751865in}{5.310704in}}%
\pgfpathcurveto{\pgfqpoint{2.746041in}{5.304880in}}{\pgfqpoint{2.742769in}{5.296980in}}{\pgfqpoint{2.742769in}{5.288744in}}%
\pgfpathcurveto{\pgfqpoint{2.742769in}{5.280508in}}{\pgfqpoint{2.746041in}{5.272608in}}{\pgfqpoint{2.751865in}{5.266784in}}%
\pgfpathcurveto{\pgfqpoint{2.757689in}{5.260960in}}{\pgfqpoint{2.765589in}{5.257688in}}{\pgfqpoint{2.773825in}{5.257688in}}%
\pgfpathclose%
\pgfusepath{stroke,fill}%
\end{pgfscope}%
\begin{pgfscope}%
\pgfpathrectangle{\pgfqpoint{0.894063in}{3.540000in}}{\pgfqpoint{6.713438in}{2.060556in}} %
\pgfusepath{clip}%
\pgfsetbuttcap%
\pgfsetroundjoin%
\definecolor{currentfill}{rgb}{0.501961,0.000000,0.501961}%
\pgfsetfillcolor{currentfill}%
\pgfsetlinewidth{1.003750pt}%
\definecolor{currentstroke}{rgb}{0.501961,0.000000,0.501961}%
\pgfsetstrokecolor{currentstroke}%
\pgfsetdash{}{0pt}%
\pgfpathmoveto{\pgfqpoint{1.565406in}{5.258181in}}%
\pgfpathcurveto{\pgfqpoint{1.573643in}{5.258181in}}{\pgfqpoint{1.581543in}{5.261453in}}{\pgfqpoint{1.587367in}{5.267277in}}%
\pgfpathcurveto{\pgfqpoint{1.593190in}{5.273101in}}{\pgfqpoint{1.596463in}{5.281001in}}{\pgfqpoint{1.596463in}{5.289237in}}%
\pgfpathcurveto{\pgfqpoint{1.596463in}{5.297473in}}{\pgfqpoint{1.593190in}{5.305374in}}{\pgfqpoint{1.587367in}{5.311197in}}%
\pgfpathcurveto{\pgfqpoint{1.581543in}{5.317021in}}{\pgfqpoint{1.573643in}{5.320294in}}{\pgfqpoint{1.565406in}{5.320294in}}%
\pgfpathcurveto{\pgfqpoint{1.557170in}{5.320294in}}{\pgfqpoint{1.549270in}{5.317021in}}{\pgfqpoint{1.543446in}{5.311197in}}%
\pgfpathcurveto{\pgfqpoint{1.537622in}{5.305374in}}{\pgfqpoint{1.534350in}{5.297473in}}{\pgfqpoint{1.534350in}{5.289237in}}%
\pgfpathcurveto{\pgfqpoint{1.534350in}{5.281001in}}{\pgfqpoint{1.537622in}{5.273101in}}{\pgfqpoint{1.543446in}{5.267277in}}%
\pgfpathcurveto{\pgfqpoint{1.549270in}{5.261453in}}{\pgfqpoint{1.557170in}{5.258181in}}{\pgfqpoint{1.565406in}{5.258181in}}%
\pgfpathclose%
\pgfusepath{stroke,fill}%
\end{pgfscope}%
\begin{pgfscope}%
\pgfpathrectangle{\pgfqpoint{0.894063in}{3.540000in}}{\pgfqpoint{6.713438in}{2.060556in}} %
\pgfusepath{clip}%
\pgfsetbuttcap%
\pgfsetroundjoin%
\definecolor{currentfill}{rgb}{0.501961,0.000000,0.501961}%
\pgfsetfillcolor{currentfill}%
\pgfsetlinewidth{1.003750pt}%
\definecolor{currentstroke}{rgb}{0.501961,0.000000,0.501961}%
\pgfsetstrokecolor{currentstroke}%
\pgfsetdash{}{0pt}%
\pgfpathmoveto{\pgfqpoint{4.250781in}{5.257572in}}%
\pgfpathcurveto{\pgfqpoint{4.259018in}{5.257572in}}{\pgfqpoint{4.266918in}{5.260844in}}{\pgfqpoint{4.272742in}{5.266668in}}%
\pgfpathcurveto{\pgfqpoint{4.278565in}{5.272492in}}{\pgfqpoint{4.281838in}{5.280392in}}{\pgfqpoint{4.281838in}{5.288629in}}%
\pgfpathcurveto{\pgfqpoint{4.281838in}{5.296865in}}{\pgfqpoint{4.278565in}{5.304765in}}{\pgfqpoint{4.272742in}{5.310589in}}%
\pgfpathcurveto{\pgfqpoint{4.266918in}{5.316413in}}{\pgfqpoint{4.259018in}{5.319685in}}{\pgfqpoint{4.250781in}{5.319685in}}%
\pgfpathcurveto{\pgfqpoint{4.242545in}{5.319685in}}{\pgfqpoint{4.234645in}{5.316413in}}{\pgfqpoint{4.228821in}{5.310589in}}%
\pgfpathcurveto{\pgfqpoint{4.222997in}{5.304765in}}{\pgfqpoint{4.219725in}{5.296865in}}{\pgfqpoint{4.219725in}{5.288629in}}%
\pgfpathcurveto{\pgfqpoint{4.219725in}{5.280392in}}{\pgfqpoint{4.222997in}{5.272492in}}{\pgfqpoint{4.228821in}{5.266668in}}%
\pgfpathcurveto{\pgfqpoint{4.234645in}{5.260844in}}{\pgfqpoint{4.242545in}{5.257572in}}{\pgfqpoint{4.250781in}{5.257572in}}%
\pgfpathclose%
\pgfusepath{stroke,fill}%
\end{pgfscope}%
\begin{pgfscope}%
\pgfpathrectangle{\pgfqpoint{0.894063in}{3.540000in}}{\pgfqpoint{6.713438in}{2.060556in}} %
\pgfusepath{clip}%
\pgfsetbuttcap%
\pgfsetroundjoin%
\definecolor{currentfill}{rgb}{0.501961,0.000000,0.501961}%
\pgfsetfillcolor{currentfill}%
\pgfsetlinewidth{1.003750pt}%
\definecolor{currentstroke}{rgb}{0.501961,0.000000,0.501961}%
\pgfsetstrokecolor{currentstroke}%
\pgfsetdash{}{0pt}%
\pgfpathmoveto{\pgfqpoint{3.847975in}{5.257601in}}%
\pgfpathcurveto{\pgfqpoint{3.856211in}{5.257601in}}{\pgfqpoint{3.864111in}{5.260873in}}{\pgfqpoint{3.869935in}{5.266697in}}%
\pgfpathcurveto{\pgfqpoint{3.875759in}{5.272521in}}{\pgfqpoint{3.879031in}{5.280421in}}{\pgfqpoint{3.879031in}{5.288658in}}%
\pgfpathcurveto{\pgfqpoint{3.879031in}{5.296894in}}{\pgfqpoint{3.875759in}{5.304794in}}{\pgfqpoint{3.869935in}{5.310618in}}%
\pgfpathcurveto{\pgfqpoint{3.864111in}{5.316442in}}{\pgfqpoint{3.856211in}{5.319714in}}{\pgfqpoint{3.847975in}{5.319714in}}%
\pgfpathcurveto{\pgfqpoint{3.839739in}{5.319714in}}{\pgfqpoint{3.831839in}{5.316442in}}{\pgfqpoint{3.826015in}{5.310618in}}%
\pgfpathcurveto{\pgfqpoint{3.820191in}{5.304794in}}{\pgfqpoint{3.816919in}{5.296894in}}{\pgfqpoint{3.816919in}{5.288658in}}%
\pgfpathcurveto{\pgfqpoint{3.816919in}{5.280421in}}{\pgfqpoint{3.820191in}{5.272521in}}{\pgfqpoint{3.826015in}{5.266697in}}%
\pgfpathcurveto{\pgfqpoint{3.831839in}{5.260873in}}{\pgfqpoint{3.839739in}{5.257601in}}{\pgfqpoint{3.847975in}{5.257601in}}%
\pgfpathclose%
\pgfusepath{stroke,fill}%
\end{pgfscope}%
\begin{pgfscope}%
\pgfpathrectangle{\pgfqpoint{0.894063in}{3.540000in}}{\pgfqpoint{6.713438in}{2.060556in}} %
\pgfusepath{clip}%
\pgfsetbuttcap%
\pgfsetroundjoin%
\definecolor{currentfill}{rgb}{0.501961,0.000000,0.501961}%
\pgfsetfillcolor{currentfill}%
\pgfsetlinewidth{1.003750pt}%
\definecolor{currentstroke}{rgb}{0.501961,0.000000,0.501961}%
\pgfsetstrokecolor{currentstroke}%
\pgfsetdash{}{0pt}%
\pgfpathmoveto{\pgfqpoint{7.607500in}{5.256042in}}%
\pgfpathcurveto{\pgfqpoint{7.615736in}{5.256042in}}{\pgfqpoint{7.623636in}{5.259314in}}{\pgfqpoint{7.629460in}{5.265138in}}%
\pgfpathcurveto{\pgfqpoint{7.635284in}{5.270962in}}{\pgfqpoint{7.638556in}{5.278862in}}{\pgfqpoint{7.638556in}{5.287098in}}%
\pgfpathcurveto{\pgfqpoint{7.638556in}{5.295335in}}{\pgfqpoint{7.635284in}{5.303235in}}{\pgfqpoint{7.629460in}{5.309059in}}%
\pgfpathcurveto{\pgfqpoint{7.623636in}{5.314883in}}{\pgfqpoint{7.615736in}{5.318155in}}{\pgfqpoint{7.607500in}{5.318155in}}%
\pgfpathcurveto{\pgfqpoint{7.599264in}{5.318155in}}{\pgfqpoint{7.591364in}{5.314883in}}{\pgfqpoint{7.585540in}{5.309059in}}%
\pgfpathcurveto{\pgfqpoint{7.579716in}{5.303235in}}{\pgfqpoint{7.576444in}{5.295335in}}{\pgfqpoint{7.576444in}{5.287098in}}%
\pgfpathcurveto{\pgfqpoint{7.576444in}{5.278862in}}{\pgfqpoint{7.579716in}{5.270962in}}{\pgfqpoint{7.585540in}{5.265138in}}%
\pgfpathcurveto{\pgfqpoint{7.591364in}{5.259314in}}{\pgfqpoint{7.599264in}{5.256042in}}{\pgfqpoint{7.607500in}{5.256042in}}%
\pgfpathclose%
\pgfusepath{stroke,fill}%
\end{pgfscope}%
\begin{pgfscope}%
\pgfpathrectangle{\pgfqpoint{0.894063in}{3.540000in}}{\pgfqpoint{6.713438in}{2.060556in}} %
\pgfusepath{clip}%
\pgfsetbuttcap%
\pgfsetroundjoin%
\definecolor{currentfill}{rgb}{0.501961,0.000000,0.501961}%
\pgfsetfillcolor{currentfill}%
\pgfsetlinewidth{1.003750pt}%
\definecolor{currentstroke}{rgb}{0.501961,0.000000,0.501961}%
\pgfsetstrokecolor{currentstroke}%
\pgfsetdash{}{0pt}%
\pgfpathmoveto{\pgfqpoint{4.385050in}{5.257561in}}%
\pgfpathcurveto{\pgfqpoint{4.393286in}{5.257561in}}{\pgfqpoint{4.401186in}{5.260833in}}{\pgfqpoint{4.407010in}{5.266657in}}%
\pgfpathcurveto{\pgfqpoint{4.412834in}{5.272481in}}{\pgfqpoint{4.416106in}{5.280381in}}{\pgfqpoint{4.416106in}{5.288618in}}%
\pgfpathcurveto{\pgfqpoint{4.416106in}{5.296854in}}{\pgfqpoint{4.412834in}{5.304754in}}{\pgfqpoint{4.407010in}{5.310578in}}%
\pgfpathcurveto{\pgfqpoint{4.401186in}{5.316402in}}{\pgfqpoint{4.393286in}{5.319674in}}{\pgfqpoint{4.385050in}{5.319674in}}%
\pgfpathcurveto{\pgfqpoint{4.376814in}{5.319674in}}{\pgfqpoint{4.368914in}{5.316402in}}{\pgfqpoint{4.363090in}{5.310578in}}%
\pgfpathcurveto{\pgfqpoint{4.357266in}{5.304754in}}{\pgfqpoint{4.353994in}{5.296854in}}{\pgfqpoint{4.353994in}{5.288618in}}%
\pgfpathcurveto{\pgfqpoint{4.353994in}{5.280381in}}{\pgfqpoint{4.357266in}{5.272481in}}{\pgfqpoint{4.363090in}{5.266657in}}%
\pgfpathcurveto{\pgfqpoint{4.368914in}{5.260833in}}{\pgfqpoint{4.376814in}{5.257561in}}{\pgfqpoint{4.385050in}{5.257561in}}%
\pgfpathclose%
\pgfusepath{stroke,fill}%
\end{pgfscope}%
\begin{pgfscope}%
\pgfpathrectangle{\pgfqpoint{0.894063in}{3.540000in}}{\pgfqpoint{6.713438in}{2.060556in}} %
\pgfusepath{clip}%
\pgfsetbuttcap%
\pgfsetroundjoin%
\definecolor{currentfill}{rgb}{0.501961,0.000000,0.501961}%
\pgfsetfillcolor{currentfill}%
\pgfsetlinewidth{1.003750pt}%
\definecolor{currentstroke}{rgb}{0.501961,0.000000,0.501961}%
\pgfsetstrokecolor{currentstroke}%
\pgfsetdash{}{0pt}%
\pgfpathmoveto{\pgfqpoint{6.533350in}{5.256102in}}%
\pgfpathcurveto{\pgfqpoint{6.541586in}{5.256102in}}{\pgfqpoint{6.549486in}{5.259375in}}{\pgfqpoint{6.555310in}{5.265199in}}%
\pgfpathcurveto{\pgfqpoint{6.561134in}{5.271022in}}{\pgfqpoint{6.564406in}{5.278923in}}{\pgfqpoint{6.564406in}{5.287159in}}%
\pgfpathcurveto{\pgfqpoint{6.564406in}{5.295395in}}{\pgfqpoint{6.561134in}{5.303295in}}{\pgfqpoint{6.555310in}{5.309119in}}%
\pgfpathcurveto{\pgfqpoint{6.549486in}{5.314943in}}{\pgfqpoint{6.541586in}{5.318215in}}{\pgfqpoint{6.533350in}{5.318215in}}%
\pgfpathcurveto{\pgfqpoint{6.525114in}{5.318215in}}{\pgfqpoint{6.517214in}{5.314943in}}{\pgfqpoint{6.511390in}{5.309119in}}%
\pgfpathcurveto{\pgfqpoint{6.505566in}{5.303295in}}{\pgfqpoint{6.502294in}{5.295395in}}{\pgfqpoint{6.502294in}{5.287159in}}%
\pgfpathcurveto{\pgfqpoint{6.502294in}{5.278923in}}{\pgfqpoint{6.505566in}{5.271022in}}{\pgfqpoint{6.511390in}{5.265199in}}%
\pgfpathcurveto{\pgfqpoint{6.517214in}{5.259375in}}{\pgfqpoint{6.525114in}{5.256102in}}{\pgfqpoint{6.533350in}{5.256102in}}%
\pgfpathclose%
\pgfusepath{stroke,fill}%
\end{pgfscope}%
\begin{pgfscope}%
\pgfpathrectangle{\pgfqpoint{0.894063in}{3.540000in}}{\pgfqpoint{6.713438in}{2.060556in}} %
\pgfusepath{clip}%
\pgfsetbuttcap%
\pgfsetroundjoin%
\definecolor{currentfill}{rgb}{0.501961,0.000000,0.501961}%
\pgfsetfillcolor{currentfill}%
\pgfsetlinewidth{1.003750pt}%
\definecolor{currentstroke}{rgb}{0.501961,0.000000,0.501961}%
\pgfsetstrokecolor{currentstroke}%
\pgfsetdash{}{0pt}%
\pgfpathmoveto{\pgfqpoint{1.296869in}{5.258241in}}%
\pgfpathcurveto{\pgfqpoint{1.305105in}{5.258241in}}{\pgfqpoint{1.313005in}{5.261513in}}{\pgfqpoint{1.318829in}{5.267337in}}%
\pgfpathcurveto{\pgfqpoint{1.324653in}{5.273161in}}{\pgfqpoint{1.327925in}{5.281061in}}{\pgfqpoint{1.327925in}{5.289298in}}%
\pgfpathcurveto{\pgfqpoint{1.327925in}{5.297534in}}{\pgfqpoint{1.324653in}{5.305434in}}{\pgfqpoint{1.318829in}{5.311258in}}%
\pgfpathcurveto{\pgfqpoint{1.313005in}{5.317082in}}{\pgfqpoint{1.305105in}{5.320354in}}{\pgfqpoint{1.296869in}{5.320354in}}%
\pgfpathcurveto{\pgfqpoint{1.288632in}{5.320354in}}{\pgfqpoint{1.280732in}{5.317082in}}{\pgfqpoint{1.274908in}{5.311258in}}%
\pgfpathcurveto{\pgfqpoint{1.269085in}{5.305434in}}{\pgfqpoint{1.265812in}{5.297534in}}{\pgfqpoint{1.265812in}{5.289298in}}%
\pgfpathcurveto{\pgfqpoint{1.265812in}{5.281061in}}{\pgfqpoint{1.269085in}{5.273161in}}{\pgfqpoint{1.274908in}{5.267337in}}%
\pgfpathcurveto{\pgfqpoint{1.280732in}{5.261513in}}{\pgfqpoint{1.288632in}{5.258241in}}{\pgfqpoint{1.296869in}{5.258241in}}%
\pgfpathclose%
\pgfusepath{stroke,fill}%
\end{pgfscope}%
\begin{pgfscope}%
\pgfpathrectangle{\pgfqpoint{0.894063in}{3.540000in}}{\pgfqpoint{6.713438in}{2.060556in}} %
\pgfusepath{clip}%
\pgfsetbuttcap%
\pgfsetroundjoin%
\definecolor{currentfill}{rgb}{0.501961,0.000000,0.501961}%
\pgfsetfillcolor{currentfill}%
\pgfsetlinewidth{1.003750pt}%
\definecolor{currentstroke}{rgb}{0.501961,0.000000,0.501961}%
\pgfsetstrokecolor{currentstroke}%
\pgfsetdash{}{0pt}%
\pgfpathmoveto{\pgfqpoint{4.519319in}{5.257558in}}%
\pgfpathcurveto{\pgfqpoint{4.527555in}{5.257558in}}{\pgfqpoint{4.535455in}{5.260831in}}{\pgfqpoint{4.541279in}{5.266655in}}%
\pgfpathcurveto{\pgfqpoint{4.547103in}{5.272479in}}{\pgfqpoint{4.550375in}{5.280379in}}{\pgfqpoint{4.550375in}{5.288615in}}%
\pgfpathcurveto{\pgfqpoint{4.550375in}{5.296851in}}{\pgfqpoint{4.547103in}{5.304751in}}{\pgfqpoint{4.541279in}{5.310575in}}%
\pgfpathcurveto{\pgfqpoint{4.535455in}{5.316399in}}{\pgfqpoint{4.527555in}{5.319671in}}{\pgfqpoint{4.519319in}{5.319671in}}%
\pgfpathcurveto{\pgfqpoint{4.511082in}{5.319671in}}{\pgfqpoint{4.503182in}{5.316399in}}{\pgfqpoint{4.497358in}{5.310575in}}%
\pgfpathcurveto{\pgfqpoint{4.491535in}{5.304751in}}{\pgfqpoint{4.488262in}{5.296851in}}{\pgfqpoint{4.488262in}{5.288615in}}%
\pgfpathcurveto{\pgfqpoint{4.488262in}{5.280379in}}{\pgfqpoint{4.491535in}{5.272479in}}{\pgfqpoint{4.497358in}{5.266655in}}%
\pgfpathcurveto{\pgfqpoint{4.503182in}{5.260831in}}{\pgfqpoint{4.511082in}{5.257558in}}{\pgfqpoint{4.519319in}{5.257558in}}%
\pgfpathclose%
\pgfusepath{stroke,fill}%
\end{pgfscope}%
\begin{pgfscope}%
\pgfpathrectangle{\pgfqpoint{0.894063in}{3.540000in}}{\pgfqpoint{6.713438in}{2.060556in}} %
\pgfusepath{clip}%
\pgfsetbuttcap%
\pgfsetroundjoin%
\definecolor{currentfill}{rgb}{0.501961,0.000000,0.501961}%
\pgfsetfillcolor{currentfill}%
\pgfsetlinewidth{1.003750pt}%
\definecolor{currentstroke}{rgb}{0.501961,0.000000,0.501961}%
\pgfsetstrokecolor{currentstroke}%
\pgfsetdash{}{0pt}%
\pgfpathmoveto{\pgfqpoint{2.505288in}{5.257701in}}%
\pgfpathcurveto{\pgfqpoint{2.513524in}{5.257701in}}{\pgfqpoint{2.521424in}{5.260974in}}{\pgfqpoint{2.527248in}{5.266798in}}%
\pgfpathcurveto{\pgfqpoint{2.533072in}{5.272621in}}{\pgfqpoint{2.536344in}{5.280522in}}{\pgfqpoint{2.536344in}{5.288758in}}%
\pgfpathcurveto{\pgfqpoint{2.536344in}{5.296994in}}{\pgfqpoint{2.533072in}{5.304894in}}{\pgfqpoint{2.527248in}{5.310718in}}%
\pgfpathcurveto{\pgfqpoint{2.521424in}{5.316542in}}{\pgfqpoint{2.513524in}{5.319814in}}{\pgfqpoint{2.505288in}{5.319814in}}%
\pgfpathcurveto{\pgfqpoint{2.497051in}{5.319814in}}{\pgfqpoint{2.489151in}{5.316542in}}{\pgfqpoint{2.483327in}{5.310718in}}%
\pgfpathcurveto{\pgfqpoint{2.477503in}{5.304894in}}{\pgfqpoint{2.474231in}{5.296994in}}{\pgfqpoint{2.474231in}{5.288758in}}%
\pgfpathcurveto{\pgfqpoint{2.474231in}{5.280522in}}{\pgfqpoint{2.477503in}{5.272621in}}{\pgfqpoint{2.483327in}{5.266798in}}%
\pgfpathcurveto{\pgfqpoint{2.489151in}{5.260974in}}{\pgfqpoint{2.497051in}{5.257701in}}{\pgfqpoint{2.505288in}{5.257701in}}%
\pgfpathclose%
\pgfusepath{stroke,fill}%
\end{pgfscope}%
\begin{pgfscope}%
\pgfpathrectangle{\pgfqpoint{0.894063in}{3.540000in}}{\pgfqpoint{6.713438in}{2.060556in}} %
\pgfusepath{clip}%
\pgfsetbuttcap%
\pgfsetroundjoin%
\definecolor{currentfill}{rgb}{0.501961,0.000000,0.501961}%
\pgfsetfillcolor{currentfill}%
\pgfsetlinewidth{1.003750pt}%
\definecolor{currentstroke}{rgb}{0.501961,0.000000,0.501961}%
\pgfsetstrokecolor{currentstroke}%
\pgfsetdash{}{0pt}%
\pgfpathmoveto{\pgfqpoint{5.459200in}{5.257532in}}%
\pgfpathcurveto{\pgfqpoint{5.467436in}{5.257532in}}{\pgfqpoint{5.475336in}{5.260805in}}{\pgfqpoint{5.481160in}{5.266629in}}%
\pgfpathcurveto{\pgfqpoint{5.486984in}{5.272452in}}{\pgfqpoint{5.490256in}{5.280353in}}{\pgfqpoint{5.490256in}{5.288589in}}%
\pgfpathcurveto{\pgfqpoint{5.490256in}{5.296825in}}{\pgfqpoint{5.486984in}{5.304725in}}{\pgfqpoint{5.481160in}{5.310549in}}%
\pgfpathcurveto{\pgfqpoint{5.475336in}{5.316373in}}{\pgfqpoint{5.467436in}{5.319645in}}{\pgfqpoint{5.459200in}{5.319645in}}%
\pgfpathcurveto{\pgfqpoint{5.450964in}{5.319645in}}{\pgfqpoint{5.443064in}{5.316373in}}{\pgfqpoint{5.437240in}{5.310549in}}%
\pgfpathcurveto{\pgfqpoint{5.431416in}{5.304725in}}{\pgfqpoint{5.428144in}{5.296825in}}{\pgfqpoint{5.428144in}{5.288589in}}%
\pgfpathcurveto{\pgfqpoint{5.428144in}{5.280353in}}{\pgfqpoint{5.431416in}{5.272452in}}{\pgfqpoint{5.437240in}{5.266629in}}%
\pgfpathcurveto{\pgfqpoint{5.443064in}{5.260805in}}{\pgfqpoint{5.450964in}{5.257532in}}{\pgfqpoint{5.459200in}{5.257532in}}%
\pgfpathclose%
\pgfusepath{stroke,fill}%
\end{pgfscope}%
\begin{pgfscope}%
\pgfpathrectangle{\pgfqpoint{0.894063in}{3.540000in}}{\pgfqpoint{6.713438in}{2.060556in}} %
\pgfusepath{clip}%
\pgfsetbuttcap%
\pgfsetroundjoin%
\definecolor{currentfill}{rgb}{0.501961,0.000000,0.501961}%
\pgfsetfillcolor{currentfill}%
\pgfsetlinewidth{1.003750pt}%
\definecolor{currentstroke}{rgb}{0.501961,0.000000,0.501961}%
\pgfsetstrokecolor{currentstroke}%
\pgfsetdash{}{0pt}%
\pgfpathmoveto{\pgfqpoint{6.936156in}{5.256093in}}%
\pgfpathcurveto{\pgfqpoint{6.944393in}{5.256093in}}{\pgfqpoint{6.952293in}{5.259365in}}{\pgfqpoint{6.958117in}{5.265189in}}%
\pgfpathcurveto{\pgfqpoint{6.963940in}{5.271013in}}{\pgfqpoint{6.967213in}{5.278913in}}{\pgfqpoint{6.967213in}{5.287149in}}%
\pgfpathcurveto{\pgfqpoint{6.967213in}{5.295385in}}{\pgfqpoint{6.963940in}{5.303286in}}{\pgfqpoint{6.958117in}{5.309109in}}%
\pgfpathcurveto{\pgfqpoint{6.952293in}{5.314933in}}{\pgfqpoint{6.944393in}{5.318206in}}{\pgfqpoint{6.936156in}{5.318206in}}%
\pgfpathcurveto{\pgfqpoint{6.927920in}{5.318206in}}{\pgfqpoint{6.920020in}{5.314933in}}{\pgfqpoint{6.914196in}{5.309109in}}%
\pgfpathcurveto{\pgfqpoint{6.908372in}{5.303286in}}{\pgfqpoint{6.905100in}{5.295385in}}{\pgfqpoint{6.905100in}{5.287149in}}%
\pgfpathcurveto{\pgfqpoint{6.905100in}{5.278913in}}{\pgfqpoint{6.908372in}{5.271013in}}{\pgfqpoint{6.914196in}{5.265189in}}%
\pgfpathcurveto{\pgfqpoint{6.920020in}{5.259365in}}{\pgfqpoint{6.927920in}{5.256093in}}{\pgfqpoint{6.936156in}{5.256093in}}%
\pgfpathclose%
\pgfusepath{stroke,fill}%
\end{pgfscope}%
\begin{pgfscope}%
\pgfpathrectangle{\pgfqpoint{0.894063in}{3.540000in}}{\pgfqpoint{6.713438in}{2.060556in}} %
\pgfusepath{clip}%
\pgfsetbuttcap%
\pgfsetroundjoin%
\definecolor{currentfill}{rgb}{0.501961,0.000000,0.501961}%
\pgfsetfillcolor{currentfill}%
\pgfsetlinewidth{1.003750pt}%
\definecolor{currentstroke}{rgb}{0.501961,0.000000,0.501961}%
\pgfsetstrokecolor{currentstroke}%
\pgfsetdash{}{0pt}%
\pgfpathmoveto{\pgfqpoint{5.862006in}{5.256819in}}%
\pgfpathcurveto{\pgfqpoint{5.870243in}{5.256819in}}{\pgfqpoint{5.878143in}{5.260092in}}{\pgfqpoint{5.883967in}{5.265916in}}%
\pgfpathcurveto{\pgfqpoint{5.889790in}{5.271740in}}{\pgfqpoint{5.893063in}{5.279640in}}{\pgfqpoint{5.893063in}{5.287876in}}%
\pgfpathcurveto{\pgfqpoint{5.893063in}{5.296112in}}{\pgfqpoint{5.889790in}{5.304012in}}{\pgfqpoint{5.883967in}{5.309836in}}%
\pgfpathcurveto{\pgfqpoint{5.878143in}{5.315660in}}{\pgfqpoint{5.870243in}{5.318932in}}{\pgfqpoint{5.862006in}{5.318932in}}%
\pgfpathcurveto{\pgfqpoint{5.853770in}{5.318932in}}{\pgfqpoint{5.845870in}{5.315660in}}{\pgfqpoint{5.840046in}{5.309836in}}%
\pgfpathcurveto{\pgfqpoint{5.834222in}{5.304012in}}{\pgfqpoint{5.830950in}{5.296112in}}{\pgfqpoint{5.830950in}{5.287876in}}%
\pgfpathcurveto{\pgfqpoint{5.830950in}{5.279640in}}{\pgfqpoint{5.834222in}{5.271740in}}{\pgfqpoint{5.840046in}{5.265916in}}%
\pgfpathcurveto{\pgfqpoint{5.845870in}{5.260092in}}{\pgfqpoint{5.853770in}{5.256819in}}{\pgfqpoint{5.862006in}{5.256819in}}%
\pgfpathclose%
\pgfusepath{stroke,fill}%
\end{pgfscope}%
\begin{pgfscope}%
\pgfpathrectangle{\pgfqpoint{0.894063in}{3.540000in}}{\pgfqpoint{6.713438in}{2.060556in}} %
\pgfusepath{clip}%
\pgfsetbuttcap%
\pgfsetroundjoin%
\definecolor{currentfill}{rgb}{0.501961,0.000000,0.501961}%
\pgfsetfillcolor{currentfill}%
\pgfsetlinewidth{1.003750pt}%
\definecolor{currentstroke}{rgb}{0.501961,0.000000,0.501961}%
\pgfsetstrokecolor{currentstroke}%
\pgfsetdash{}{0pt}%
\pgfpathmoveto{\pgfqpoint{7.070425in}{5.256067in}}%
\pgfpathcurveto{\pgfqpoint{7.078661in}{5.256067in}}{\pgfqpoint{7.086561in}{5.259339in}}{\pgfqpoint{7.092385in}{5.265163in}}%
\pgfpathcurveto{\pgfqpoint{7.098209in}{5.270987in}}{\pgfqpoint{7.101481in}{5.278887in}}{\pgfqpoint{7.101481in}{5.287123in}}%
\pgfpathcurveto{\pgfqpoint{7.101481in}{5.295359in}}{\pgfqpoint{7.098209in}{5.303259in}}{\pgfqpoint{7.092385in}{5.309083in}}%
\pgfpathcurveto{\pgfqpoint{7.086561in}{5.314907in}}{\pgfqpoint{7.078661in}{5.318180in}}{\pgfqpoint{7.070425in}{5.318180in}}%
\pgfpathcurveto{\pgfqpoint{7.062189in}{5.318180in}}{\pgfqpoint{7.054289in}{5.314907in}}{\pgfqpoint{7.048465in}{5.309083in}}%
\pgfpathcurveto{\pgfqpoint{7.042641in}{5.303259in}}{\pgfqpoint{7.039369in}{5.295359in}}{\pgfqpoint{7.039369in}{5.287123in}}%
\pgfpathcurveto{\pgfqpoint{7.039369in}{5.278887in}}{\pgfqpoint{7.042641in}{5.270987in}}{\pgfqpoint{7.048465in}{5.265163in}}%
\pgfpathcurveto{\pgfqpoint{7.054289in}{5.259339in}}{\pgfqpoint{7.062189in}{5.256067in}}{\pgfqpoint{7.070425in}{5.256067in}}%
\pgfpathclose%
\pgfusepath{stroke,fill}%
\end{pgfscope}%
\begin{pgfscope}%
\pgfpathrectangle{\pgfqpoint{0.894063in}{3.540000in}}{\pgfqpoint{6.713438in}{2.060556in}} %
\pgfusepath{clip}%
\pgfsetbuttcap%
\pgfsetroundjoin%
\definecolor{currentfill}{rgb}{0.501961,0.000000,0.501961}%
\pgfsetfillcolor{currentfill}%
\pgfsetlinewidth{1.003750pt}%
\definecolor{currentstroke}{rgb}{0.501961,0.000000,0.501961}%
\pgfsetstrokecolor{currentstroke}%
\pgfsetdash{}{0pt}%
\pgfpathmoveto{\pgfqpoint{3.176631in}{5.257666in}}%
\pgfpathcurveto{\pgfqpoint{3.184868in}{5.257666in}}{\pgfqpoint{3.192768in}{5.260938in}}{\pgfqpoint{3.198592in}{5.266762in}}%
\pgfpathcurveto{\pgfqpoint{3.204415in}{5.272586in}}{\pgfqpoint{3.207688in}{5.280486in}}{\pgfqpoint{3.207688in}{5.288722in}}%
\pgfpathcurveto{\pgfqpoint{3.207688in}{5.296958in}}{\pgfqpoint{3.204415in}{5.304858in}}{\pgfqpoint{3.198592in}{5.310682in}}%
\pgfpathcurveto{\pgfqpoint{3.192768in}{5.316506in}}{\pgfqpoint{3.184868in}{5.319779in}}{\pgfqpoint{3.176631in}{5.319779in}}%
\pgfpathcurveto{\pgfqpoint{3.168395in}{5.319779in}}{\pgfqpoint{3.160495in}{5.316506in}}{\pgfqpoint{3.154671in}{5.310682in}}%
\pgfpathcurveto{\pgfqpoint{3.148847in}{5.304858in}}{\pgfqpoint{3.145575in}{5.296958in}}{\pgfqpoint{3.145575in}{5.288722in}}%
\pgfpathcurveto{\pgfqpoint{3.145575in}{5.280486in}}{\pgfqpoint{3.148847in}{5.272586in}}{\pgfqpoint{3.154671in}{5.266762in}}%
\pgfpathcurveto{\pgfqpoint{3.160495in}{5.260938in}}{\pgfqpoint{3.168395in}{5.257666in}}{\pgfqpoint{3.176631in}{5.257666in}}%
\pgfpathclose%
\pgfusepath{stroke,fill}%
\end{pgfscope}%
\begin{pgfscope}%
\pgfpathrectangle{\pgfqpoint{0.894063in}{3.540000in}}{\pgfqpoint{6.713438in}{2.060556in}} %
\pgfusepath{clip}%
\pgfsetbuttcap%
\pgfsetroundjoin%
\definecolor{currentfill}{rgb}{0.501961,0.000000,0.501961}%
\pgfsetfillcolor{currentfill}%
\pgfsetlinewidth{1.003750pt}%
\definecolor{currentstroke}{rgb}{0.501961,0.000000,0.501961}%
\pgfsetstrokecolor{currentstroke}%
\pgfsetdash{}{0pt}%
\pgfpathmoveto{\pgfqpoint{2.102481in}{5.257844in}}%
\pgfpathcurveto{\pgfqpoint{2.110718in}{5.257844in}}{\pgfqpoint{2.118618in}{5.261116in}}{\pgfqpoint{2.124442in}{5.266940in}}%
\pgfpathcurveto{\pgfqpoint{2.130265in}{5.272764in}}{\pgfqpoint{2.133538in}{5.280664in}}{\pgfqpoint{2.133538in}{5.288901in}}%
\pgfpathcurveto{\pgfqpoint{2.133538in}{5.297137in}}{\pgfqpoint{2.130265in}{5.305037in}}{\pgfqpoint{2.124442in}{5.310861in}}%
\pgfpathcurveto{\pgfqpoint{2.118618in}{5.316685in}}{\pgfqpoint{2.110718in}{5.319957in}}{\pgfqpoint{2.102481in}{5.319957in}}%
\pgfpathcurveto{\pgfqpoint{2.094245in}{5.319957in}}{\pgfqpoint{2.086345in}{5.316685in}}{\pgfqpoint{2.080521in}{5.310861in}}%
\pgfpathcurveto{\pgfqpoint{2.074697in}{5.305037in}}{\pgfqpoint{2.071425in}{5.297137in}}{\pgfqpoint{2.071425in}{5.288901in}}%
\pgfpathcurveto{\pgfqpoint{2.071425in}{5.280664in}}{\pgfqpoint{2.074697in}{5.272764in}}{\pgfqpoint{2.080521in}{5.266940in}}%
\pgfpathcurveto{\pgfqpoint{2.086345in}{5.261116in}}{\pgfqpoint{2.094245in}{5.257844in}}{\pgfqpoint{2.102481in}{5.257844in}}%
\pgfpathclose%
\pgfusepath{stroke,fill}%
\end{pgfscope}%
\begin{pgfscope}%
\pgfpathrectangle{\pgfqpoint{0.894063in}{3.540000in}}{\pgfqpoint{6.713438in}{2.060556in}} %
\pgfusepath{clip}%
\pgfsetbuttcap%
\pgfsetroundjoin%
\definecolor{currentfill}{rgb}{0.501961,0.000000,0.501961}%
\pgfsetfillcolor{currentfill}%
\pgfsetlinewidth{1.003750pt}%
\definecolor{currentstroke}{rgb}{0.501961,0.000000,0.501961}%
\pgfsetstrokecolor{currentstroke}%
\pgfsetdash{}{0pt}%
\pgfpathmoveto{\pgfqpoint{1.968213in}{5.258111in}}%
\pgfpathcurveto{\pgfqpoint{1.976449in}{5.258111in}}{\pgfqpoint{1.984349in}{5.261383in}}{\pgfqpoint{1.990173in}{5.267207in}}%
\pgfpathcurveto{\pgfqpoint{1.995997in}{5.273031in}}{\pgfqpoint{1.999269in}{5.280931in}}{\pgfqpoint{1.999269in}{5.289167in}}%
\pgfpathcurveto{\pgfqpoint{1.999269in}{5.297403in}}{\pgfqpoint{1.995997in}{5.305303in}}{\pgfqpoint{1.990173in}{5.311127in}}%
\pgfpathcurveto{\pgfqpoint{1.984349in}{5.316951in}}{\pgfqpoint{1.976449in}{5.320224in}}{\pgfqpoint{1.968213in}{5.320224in}}%
\pgfpathcurveto{\pgfqpoint{1.959976in}{5.320224in}}{\pgfqpoint{1.952076in}{5.316951in}}{\pgfqpoint{1.946252in}{5.311127in}}%
\pgfpathcurveto{\pgfqpoint{1.940428in}{5.305303in}}{\pgfqpoint{1.937156in}{5.297403in}}{\pgfqpoint{1.937156in}{5.289167in}}%
\pgfpathcurveto{\pgfqpoint{1.937156in}{5.280931in}}{\pgfqpoint{1.940428in}{5.273031in}}{\pgfqpoint{1.946252in}{5.267207in}}%
\pgfpathcurveto{\pgfqpoint{1.952076in}{5.261383in}}{\pgfqpoint{1.959976in}{5.258111in}}{\pgfqpoint{1.968213in}{5.258111in}}%
\pgfpathclose%
\pgfusepath{stroke,fill}%
\end{pgfscope}%
\begin{pgfscope}%
\pgfpathrectangle{\pgfqpoint{0.894063in}{3.540000in}}{\pgfqpoint{6.713438in}{2.060556in}} %
\pgfusepath{clip}%
\pgfsetbuttcap%
\pgfsetroundjoin%
\definecolor{currentfill}{rgb}{0.501961,0.000000,0.501961}%
\pgfsetfillcolor{currentfill}%
\pgfsetlinewidth{1.003750pt}%
\definecolor{currentstroke}{rgb}{0.501961,0.000000,0.501961}%
\pgfsetstrokecolor{currentstroke}%
\pgfsetdash{}{0pt}%
\pgfpathmoveto{\pgfqpoint{3.310900in}{5.257650in}}%
\pgfpathcurveto{\pgfqpoint{3.319136in}{5.257650in}}{\pgfqpoint{3.327036in}{5.260923in}}{\pgfqpoint{3.332860in}{5.266747in}}%
\pgfpathcurveto{\pgfqpoint{3.338684in}{5.272571in}}{\pgfqpoint{3.341956in}{5.280471in}}{\pgfqpoint{3.341956in}{5.288707in}}%
\pgfpathcurveto{\pgfqpoint{3.341956in}{5.296943in}}{\pgfqpoint{3.338684in}{5.304843in}}{\pgfqpoint{3.332860in}{5.310667in}}%
\pgfpathcurveto{\pgfqpoint{3.327036in}{5.316491in}}{\pgfqpoint{3.319136in}{5.319763in}}{\pgfqpoint{3.310900in}{5.319763in}}%
\pgfpathcurveto{\pgfqpoint{3.302664in}{5.319763in}}{\pgfqpoint{3.294764in}{5.316491in}}{\pgfqpoint{3.288940in}{5.310667in}}%
\pgfpathcurveto{\pgfqpoint{3.283116in}{5.304843in}}{\pgfqpoint{3.279844in}{5.296943in}}{\pgfqpoint{3.279844in}{5.288707in}}%
\pgfpathcurveto{\pgfqpoint{3.279844in}{5.280471in}}{\pgfqpoint{3.283116in}{5.272571in}}{\pgfqpoint{3.288940in}{5.266747in}}%
\pgfpathcurveto{\pgfqpoint{3.294764in}{5.260923in}}{\pgfqpoint{3.302664in}{5.257650in}}{\pgfqpoint{3.310900in}{5.257650in}}%
\pgfpathclose%
\pgfusepath{stroke,fill}%
\end{pgfscope}%
\begin{pgfscope}%
\pgfpathrectangle{\pgfqpoint{0.894063in}{3.540000in}}{\pgfqpoint{6.713438in}{2.060556in}} %
\pgfusepath{clip}%
\pgfsetbuttcap%
\pgfsetroundjoin%
\definecolor{currentfill}{rgb}{0.501961,0.000000,0.501961}%
\pgfsetfillcolor{currentfill}%
\pgfsetlinewidth{1.003750pt}%
\definecolor{currentstroke}{rgb}{0.501961,0.000000,0.501961}%
\pgfsetstrokecolor{currentstroke}%
\pgfsetdash{}{0pt}%
\pgfpathmoveto{\pgfqpoint{5.593469in}{5.257530in}}%
\pgfpathcurveto{\pgfqpoint{5.601705in}{5.257530in}}{\pgfqpoint{5.609605in}{5.260802in}}{\pgfqpoint{5.615429in}{5.266626in}}%
\pgfpathcurveto{\pgfqpoint{5.621253in}{5.272450in}}{\pgfqpoint{5.624525in}{5.280350in}}{\pgfqpoint{5.624525in}{5.288586in}}%
\pgfpathcurveto{\pgfqpoint{5.624525in}{5.296822in}}{\pgfqpoint{5.621253in}{5.304722in}}{\pgfqpoint{5.615429in}{5.310546in}}%
\pgfpathcurveto{\pgfqpoint{5.609605in}{5.316370in}}{\pgfqpoint{5.601705in}{5.319643in}}{\pgfqpoint{5.593469in}{5.319643in}}%
\pgfpathcurveto{\pgfqpoint{5.585232in}{5.319643in}}{\pgfqpoint{5.577332in}{5.316370in}}{\pgfqpoint{5.571508in}{5.310546in}}%
\pgfpathcurveto{\pgfqpoint{5.565685in}{5.304722in}}{\pgfqpoint{5.562412in}{5.296822in}}{\pgfqpoint{5.562412in}{5.288586in}}%
\pgfpathcurveto{\pgfqpoint{5.562412in}{5.280350in}}{\pgfqpoint{5.565685in}{5.272450in}}{\pgfqpoint{5.571508in}{5.266626in}}%
\pgfpathcurveto{\pgfqpoint{5.577332in}{5.260802in}}{\pgfqpoint{5.585232in}{5.257530in}}{\pgfqpoint{5.593469in}{5.257530in}}%
\pgfpathclose%
\pgfusepath{stroke,fill}%
\end{pgfscope}%
\begin{pgfscope}%
\pgfpathrectangle{\pgfqpoint{0.894063in}{3.540000in}}{\pgfqpoint{6.713438in}{2.060556in}} %
\pgfusepath{clip}%
\pgfsetbuttcap%
\pgfsetroundjoin%
\definecolor{currentfill}{rgb}{0.501961,0.000000,0.501961}%
\pgfsetfillcolor{currentfill}%
\pgfsetlinewidth{1.003750pt}%
\definecolor{currentstroke}{rgb}{0.501961,0.000000,0.501961}%
\pgfsetstrokecolor{currentstroke}%
\pgfsetdash{}{0pt}%
\pgfpathmoveto{\pgfqpoint{3.042363in}{5.257688in}}%
\pgfpathcurveto{\pgfqpoint{3.050599in}{5.257688in}}{\pgfqpoint{3.058499in}{5.260960in}}{\pgfqpoint{3.064323in}{5.266784in}}%
\pgfpathcurveto{\pgfqpoint{3.070147in}{5.272608in}}{\pgfqpoint{3.073419in}{5.280508in}}{\pgfqpoint{3.073419in}{5.288744in}}%
\pgfpathcurveto{\pgfqpoint{3.073419in}{5.296980in}}{\pgfqpoint{3.070147in}{5.304880in}}{\pgfqpoint{3.064323in}{5.310704in}}%
\pgfpathcurveto{\pgfqpoint{3.058499in}{5.316528in}}{\pgfqpoint{3.050599in}{5.319801in}}{\pgfqpoint{3.042363in}{5.319801in}}%
\pgfpathcurveto{\pgfqpoint{3.034126in}{5.319801in}}{\pgfqpoint{3.026226in}{5.316528in}}{\pgfqpoint{3.020402in}{5.310704in}}%
\pgfpathcurveto{\pgfqpoint{3.014578in}{5.304880in}}{\pgfqpoint{3.011306in}{5.296980in}}{\pgfqpoint{3.011306in}{5.288744in}}%
\pgfpathcurveto{\pgfqpoint{3.011306in}{5.280508in}}{\pgfqpoint{3.014578in}{5.272608in}}{\pgfqpoint{3.020402in}{5.266784in}}%
\pgfpathcurveto{\pgfqpoint{3.026226in}{5.260960in}}{\pgfqpoint{3.034126in}{5.257688in}}{\pgfqpoint{3.042363in}{5.257688in}}%
\pgfpathclose%
\pgfusepath{stroke,fill}%
\end{pgfscope}%
\begin{pgfscope}%
\pgfpathrectangle{\pgfqpoint{0.894063in}{3.540000in}}{\pgfqpoint{6.713438in}{2.060556in}} %
\pgfusepath{clip}%
\pgfsetbuttcap%
\pgfsetroundjoin%
\definecolor{currentfill}{rgb}{0.501961,0.000000,0.501961}%
\pgfsetfillcolor{currentfill}%
\pgfsetlinewidth{1.003750pt}%
\definecolor{currentstroke}{rgb}{0.501961,0.000000,0.501961}%
\pgfsetstrokecolor{currentstroke}%
\pgfsetdash{}{0pt}%
\pgfpathmoveto{\pgfqpoint{5.190663in}{5.257542in}}%
\pgfpathcurveto{\pgfqpoint{5.198899in}{5.257542in}}{\pgfqpoint{5.206799in}{5.260814in}}{\pgfqpoint{5.212623in}{5.266638in}}%
\pgfpathcurveto{\pgfqpoint{5.218447in}{5.272462in}}{\pgfqpoint{5.221719in}{5.280362in}}{\pgfqpoint{5.221719in}{5.288598in}}%
\pgfpathcurveto{\pgfqpoint{5.221719in}{5.296835in}}{\pgfqpoint{5.218447in}{5.304735in}}{\pgfqpoint{5.212623in}{5.310559in}}%
\pgfpathcurveto{\pgfqpoint{5.206799in}{5.316383in}}{\pgfqpoint{5.198899in}{5.319655in}}{\pgfqpoint{5.190663in}{5.319655in}}%
\pgfpathcurveto{\pgfqpoint{5.182426in}{5.319655in}}{\pgfqpoint{5.174526in}{5.316383in}}{\pgfqpoint{5.168702in}{5.310559in}}%
\pgfpathcurveto{\pgfqpoint{5.162878in}{5.304735in}}{\pgfqpoint{5.159606in}{5.296835in}}{\pgfqpoint{5.159606in}{5.288598in}}%
\pgfpathcurveto{\pgfqpoint{5.159606in}{5.280362in}}{\pgfqpoint{5.162878in}{5.272462in}}{\pgfqpoint{5.168702in}{5.266638in}}%
\pgfpathcurveto{\pgfqpoint{5.174526in}{5.260814in}}{\pgfqpoint{5.182426in}{5.257542in}}{\pgfqpoint{5.190663in}{5.257542in}}%
\pgfpathclose%
\pgfusepath{stroke,fill}%
\end{pgfscope}%
\begin{pgfscope}%
\pgfpathrectangle{\pgfqpoint{0.894063in}{3.540000in}}{\pgfqpoint{6.713438in}{2.060556in}} %
\pgfusepath{clip}%
\pgfsetbuttcap%
\pgfsetroundjoin%
\definecolor{currentfill}{rgb}{0.501961,0.000000,0.501961}%
\pgfsetfillcolor{currentfill}%
\pgfsetlinewidth{1.003750pt}%
\definecolor{currentstroke}{rgb}{0.501961,0.000000,0.501961}%
\pgfsetstrokecolor{currentstroke}%
\pgfsetdash{}{0pt}%
\pgfpathmoveto{\pgfqpoint{6.801888in}{5.256095in}}%
\pgfpathcurveto{\pgfqpoint{6.810124in}{5.256095in}}{\pgfqpoint{6.818024in}{5.259368in}}{\pgfqpoint{6.823848in}{5.265192in}}%
\pgfpathcurveto{\pgfqpoint{6.829672in}{5.271016in}}{\pgfqpoint{6.832944in}{5.278916in}}{\pgfqpoint{6.832944in}{5.287152in}}%
\pgfpathcurveto{\pgfqpoint{6.832944in}{5.295388in}}{\pgfqpoint{6.829672in}{5.303288in}}{\pgfqpoint{6.823848in}{5.309112in}}%
\pgfpathcurveto{\pgfqpoint{6.818024in}{5.314936in}}{\pgfqpoint{6.810124in}{5.318208in}}{\pgfqpoint{6.801888in}{5.318208in}}%
\pgfpathcurveto{\pgfqpoint{6.793651in}{5.318208in}}{\pgfqpoint{6.785751in}{5.314936in}}{\pgfqpoint{6.779927in}{5.309112in}}%
\pgfpathcurveto{\pgfqpoint{6.774103in}{5.303288in}}{\pgfqpoint{6.770831in}{5.295388in}}{\pgfqpoint{6.770831in}{5.287152in}}%
\pgfpathcurveto{\pgfqpoint{6.770831in}{5.278916in}}{\pgfqpoint{6.774103in}{5.271016in}}{\pgfqpoint{6.779927in}{5.265192in}}%
\pgfpathcurveto{\pgfqpoint{6.785751in}{5.259368in}}{\pgfqpoint{6.793651in}{5.256095in}}{\pgfqpoint{6.801888in}{5.256095in}}%
\pgfpathclose%
\pgfusepath{stroke,fill}%
\end{pgfscope}%
\begin{pgfscope}%
\pgfpathrectangle{\pgfqpoint{0.894063in}{3.540000in}}{\pgfqpoint{6.713438in}{2.060556in}} %
\pgfusepath{clip}%
\pgfsetbuttcap%
\pgfsetroundjoin%
\definecolor{currentfill}{rgb}{0.501961,0.000000,0.501961}%
\pgfsetfillcolor{currentfill}%
\pgfsetlinewidth{1.003750pt}%
\definecolor{currentstroke}{rgb}{0.501961,0.000000,0.501961}%
\pgfsetstrokecolor{currentstroke}%
\pgfsetdash{}{0pt}%
\pgfpathmoveto{\pgfqpoint{3.579438in}{5.257608in}}%
\pgfpathcurveto{\pgfqpoint{3.587674in}{5.257608in}}{\pgfqpoint{3.595574in}{5.260880in}}{\pgfqpoint{3.601398in}{5.266704in}}%
\pgfpathcurveto{\pgfqpoint{3.607222in}{5.272528in}}{\pgfqpoint{3.610494in}{5.280428in}}{\pgfqpoint{3.610494in}{5.288664in}}%
\pgfpathcurveto{\pgfqpoint{3.610494in}{5.296901in}}{\pgfqpoint{3.607222in}{5.304801in}}{\pgfqpoint{3.601398in}{5.310625in}}%
\pgfpathcurveto{\pgfqpoint{3.595574in}{5.316449in}}{\pgfqpoint{3.587674in}{5.319721in}}{\pgfqpoint{3.579438in}{5.319721in}}%
\pgfpathcurveto{\pgfqpoint{3.571201in}{5.319721in}}{\pgfqpoint{3.563301in}{5.316449in}}{\pgfqpoint{3.557477in}{5.310625in}}%
\pgfpathcurveto{\pgfqpoint{3.551653in}{5.304801in}}{\pgfqpoint{3.548381in}{5.296901in}}{\pgfqpoint{3.548381in}{5.288664in}}%
\pgfpathcurveto{\pgfqpoint{3.548381in}{5.280428in}}{\pgfqpoint{3.551653in}{5.272528in}}{\pgfqpoint{3.557477in}{5.266704in}}%
\pgfpathcurveto{\pgfqpoint{3.563301in}{5.260880in}}{\pgfqpoint{3.571201in}{5.257608in}}{\pgfqpoint{3.579438in}{5.257608in}}%
\pgfpathclose%
\pgfusepath{stroke,fill}%
\end{pgfscope}%
\begin{pgfscope}%
\pgfpathrectangle{\pgfqpoint{0.894063in}{3.540000in}}{\pgfqpoint{6.713438in}{2.060556in}} %
\pgfusepath{clip}%
\pgfsetbuttcap%
\pgfsetroundjoin%
\definecolor{currentfill}{rgb}{0.501961,0.000000,0.501961}%
\pgfsetfillcolor{currentfill}%
\pgfsetlinewidth{1.003750pt}%
\definecolor{currentstroke}{rgb}{0.501961,0.000000,0.501961}%
\pgfsetstrokecolor{currentstroke}%
\pgfsetdash{}{0pt}%
\pgfpathmoveto{\pgfqpoint{2.371019in}{5.257708in}}%
\pgfpathcurveto{\pgfqpoint{2.379255in}{5.257708in}}{\pgfqpoint{2.387155in}{5.260980in}}{\pgfqpoint{2.392979in}{5.266804in}}%
\pgfpathcurveto{\pgfqpoint{2.398803in}{5.272628in}}{\pgfqpoint{2.402075in}{5.280528in}}{\pgfqpoint{2.402075in}{5.288765in}}%
\pgfpathcurveto{\pgfqpoint{2.402075in}{5.297001in}}{\pgfqpoint{2.398803in}{5.304901in}}{\pgfqpoint{2.392979in}{5.310725in}}%
\pgfpathcurveto{\pgfqpoint{2.387155in}{5.316549in}}{\pgfqpoint{2.379255in}{5.319821in}}{\pgfqpoint{2.371019in}{5.319821in}}%
\pgfpathcurveto{\pgfqpoint{2.362782in}{5.319821in}}{\pgfqpoint{2.354882in}{5.316549in}}{\pgfqpoint{2.349058in}{5.310725in}}%
\pgfpathcurveto{\pgfqpoint{2.343235in}{5.304901in}}{\pgfqpoint{2.339962in}{5.297001in}}{\pgfqpoint{2.339962in}{5.288765in}}%
\pgfpathcurveto{\pgfqpoint{2.339962in}{5.280528in}}{\pgfqpoint{2.343235in}{5.272628in}}{\pgfqpoint{2.349058in}{5.266804in}}%
\pgfpathcurveto{\pgfqpoint{2.354882in}{5.260980in}}{\pgfqpoint{2.362782in}{5.257708in}}{\pgfqpoint{2.371019in}{5.257708in}}%
\pgfpathclose%
\pgfusepath{stroke,fill}%
\end{pgfscope}%
\begin{pgfscope}%
\pgfpathrectangle{\pgfqpoint{0.894063in}{3.540000in}}{\pgfqpoint{6.713438in}{2.060556in}} %
\pgfusepath{clip}%
\pgfsetbuttcap%
\pgfsetroundjoin%
\definecolor{currentfill}{rgb}{0.501961,0.000000,0.501961}%
\pgfsetfillcolor{currentfill}%
\pgfsetlinewidth{1.003750pt}%
\definecolor{currentstroke}{rgb}{0.501961,0.000000,0.501961}%
\pgfsetstrokecolor{currentstroke}%
\pgfsetdash{}{0pt}%
\pgfpathmoveto{\pgfqpoint{3.982244in}{5.257600in}}%
\pgfpathcurveto{\pgfqpoint{3.990480in}{5.257600in}}{\pgfqpoint{3.998380in}{5.260872in}}{\pgfqpoint{4.004204in}{5.266696in}}%
\pgfpathcurveto{\pgfqpoint{4.010028in}{5.272520in}}{\pgfqpoint{4.013300in}{5.280420in}}{\pgfqpoint{4.013300in}{5.288656in}}%
\pgfpathcurveto{\pgfqpoint{4.013300in}{5.296892in}}{\pgfqpoint{4.010028in}{5.304792in}}{\pgfqpoint{4.004204in}{5.310616in}}%
\pgfpathcurveto{\pgfqpoint{3.998380in}{5.316440in}}{\pgfqpoint{3.990480in}{5.319713in}}{\pgfqpoint{3.982244in}{5.319713in}}%
\pgfpathcurveto{\pgfqpoint{3.974007in}{5.319713in}}{\pgfqpoint{3.966107in}{5.316440in}}{\pgfqpoint{3.960283in}{5.310616in}}%
\pgfpathcurveto{\pgfqpoint{3.954460in}{5.304792in}}{\pgfqpoint{3.951187in}{5.296892in}}{\pgfqpoint{3.951187in}{5.288656in}}%
\pgfpathcurveto{\pgfqpoint{3.951187in}{5.280420in}}{\pgfqpoint{3.954460in}{5.272520in}}{\pgfqpoint{3.960283in}{5.266696in}}%
\pgfpathcurveto{\pgfqpoint{3.966107in}{5.260872in}}{\pgfqpoint{3.974007in}{5.257600in}}{\pgfqpoint{3.982244in}{5.257600in}}%
\pgfpathclose%
\pgfusepath{stroke,fill}%
\end{pgfscope}%
\begin{pgfscope}%
\pgfpathrectangle{\pgfqpoint{0.894063in}{3.540000in}}{\pgfqpoint{6.713438in}{2.060556in}} %
\pgfusepath{clip}%
\pgfsetbuttcap%
\pgfsetroundjoin%
\definecolor{currentfill}{rgb}{0.501961,0.000000,0.501961}%
\pgfsetfillcolor{currentfill}%
\pgfsetlinewidth{1.003750pt}%
\definecolor{currentstroke}{rgb}{0.501961,0.000000,0.501961}%
\pgfsetstrokecolor{currentstroke}%
\pgfsetdash{}{0pt}%
\pgfpathmoveto{\pgfqpoint{4.653588in}{5.257546in}}%
\pgfpathcurveto{\pgfqpoint{4.661824in}{5.257546in}}{\pgfqpoint{4.669724in}{5.260818in}}{\pgfqpoint{4.675548in}{5.266642in}}%
\pgfpathcurveto{\pgfqpoint{4.681372in}{5.272466in}}{\pgfqpoint{4.684644in}{5.280366in}}{\pgfqpoint{4.684644in}{5.288603in}}%
\pgfpathcurveto{\pgfqpoint{4.684644in}{5.296839in}}{\pgfqpoint{4.681372in}{5.304739in}}{\pgfqpoint{4.675548in}{5.310563in}}%
\pgfpathcurveto{\pgfqpoint{4.669724in}{5.316387in}}{\pgfqpoint{4.661824in}{5.319659in}}{\pgfqpoint{4.653588in}{5.319659in}}%
\pgfpathcurveto{\pgfqpoint{4.645351in}{5.319659in}}{\pgfqpoint{4.637451in}{5.316387in}}{\pgfqpoint{4.631627in}{5.310563in}}%
\pgfpathcurveto{\pgfqpoint{4.625803in}{5.304739in}}{\pgfqpoint{4.622531in}{5.296839in}}{\pgfqpoint{4.622531in}{5.288603in}}%
\pgfpathcurveto{\pgfqpoint{4.622531in}{5.280366in}}{\pgfqpoint{4.625803in}{5.272466in}}{\pgfqpoint{4.631627in}{5.266642in}}%
\pgfpathcurveto{\pgfqpoint{4.637451in}{5.260818in}}{\pgfqpoint{4.645351in}{5.257546in}}{\pgfqpoint{4.653588in}{5.257546in}}%
\pgfpathclose%
\pgfusepath{stroke,fill}%
\end{pgfscope}%
\begin{pgfscope}%
\pgfpathrectangle{\pgfqpoint{0.894063in}{3.540000in}}{\pgfqpoint{6.713438in}{2.060556in}} %
\pgfusepath{clip}%
\pgfsetbuttcap%
\pgfsetroundjoin%
\definecolor{currentfill}{rgb}{0.501961,0.000000,0.501961}%
\pgfsetfillcolor{currentfill}%
\pgfsetlinewidth{1.003750pt}%
\definecolor{currentstroke}{rgb}{0.501961,0.000000,0.501961}%
\pgfsetstrokecolor{currentstroke}%
\pgfsetdash{}{0pt}%
\pgfpathmoveto{\pgfqpoint{3.713706in}{5.257605in}}%
\pgfpathcurveto{\pgfqpoint{3.721943in}{5.257605in}}{\pgfqpoint{3.729843in}{5.260877in}}{\pgfqpoint{3.735667in}{5.266701in}}%
\pgfpathcurveto{\pgfqpoint{3.741490in}{5.272525in}}{\pgfqpoint{3.744763in}{5.280425in}}{\pgfqpoint{3.744763in}{5.288662in}}%
\pgfpathcurveto{\pgfqpoint{3.744763in}{5.296898in}}{\pgfqpoint{3.741490in}{5.304798in}}{\pgfqpoint{3.735667in}{5.310622in}}%
\pgfpathcurveto{\pgfqpoint{3.729843in}{5.316446in}}{\pgfqpoint{3.721943in}{5.319718in}}{\pgfqpoint{3.713706in}{5.319718in}}%
\pgfpathcurveto{\pgfqpoint{3.705470in}{5.319718in}}{\pgfqpoint{3.697570in}{5.316446in}}{\pgfqpoint{3.691746in}{5.310622in}}%
\pgfpathcurveto{\pgfqpoint{3.685922in}{5.304798in}}{\pgfqpoint{3.682650in}{5.296898in}}{\pgfqpoint{3.682650in}{5.288662in}}%
\pgfpathcurveto{\pgfqpoint{3.682650in}{5.280425in}}{\pgfqpoint{3.685922in}{5.272525in}}{\pgfqpoint{3.691746in}{5.266701in}}%
\pgfpathcurveto{\pgfqpoint{3.697570in}{5.260877in}}{\pgfqpoint{3.705470in}{5.257605in}}{\pgfqpoint{3.713706in}{5.257605in}}%
\pgfpathclose%
\pgfusepath{stroke,fill}%
\end{pgfscope}%
\begin{pgfscope}%
\pgfpathrectangle{\pgfqpoint{0.894063in}{3.540000in}}{\pgfqpoint{6.713438in}{2.060556in}} %
\pgfusepath{clip}%
\pgfsetbuttcap%
\pgfsetroundjoin%
\definecolor{currentfill}{rgb}{0.501961,0.000000,0.501961}%
\pgfsetfillcolor{currentfill}%
\pgfsetlinewidth{1.003750pt}%
\definecolor{currentstroke}{rgb}{0.501961,0.000000,0.501961}%
\pgfsetstrokecolor{currentstroke}%
\pgfsetdash{}{0pt}%
\pgfpathmoveto{\pgfqpoint{2.236750in}{5.257839in}}%
\pgfpathcurveto{\pgfqpoint{2.244986in}{5.257839in}}{\pgfqpoint{2.252886in}{5.261111in}}{\pgfqpoint{2.258710in}{5.266935in}}%
\pgfpathcurveto{\pgfqpoint{2.264534in}{5.272759in}}{\pgfqpoint{2.267806in}{5.280659in}}{\pgfqpoint{2.267806in}{5.288895in}}%
\pgfpathcurveto{\pgfqpoint{2.267806in}{5.297131in}}{\pgfqpoint{2.264534in}{5.305031in}}{\pgfqpoint{2.258710in}{5.310855in}}%
\pgfpathcurveto{\pgfqpoint{2.252886in}{5.316679in}}{\pgfqpoint{2.244986in}{5.319952in}}{\pgfqpoint{2.236750in}{5.319952in}}%
\pgfpathcurveto{\pgfqpoint{2.228514in}{5.319952in}}{\pgfqpoint{2.220614in}{5.316679in}}{\pgfqpoint{2.214790in}{5.310855in}}%
\pgfpathcurveto{\pgfqpoint{2.208966in}{5.305031in}}{\pgfqpoint{2.205694in}{5.297131in}}{\pgfqpoint{2.205694in}{5.288895in}}%
\pgfpathcurveto{\pgfqpoint{2.205694in}{5.280659in}}{\pgfqpoint{2.208966in}{5.272759in}}{\pgfqpoint{2.214790in}{5.266935in}}%
\pgfpathcurveto{\pgfqpoint{2.220614in}{5.261111in}}{\pgfqpoint{2.228514in}{5.257839in}}{\pgfqpoint{2.236750in}{5.257839in}}%
\pgfpathclose%
\pgfusepath{stroke,fill}%
\end{pgfscope}%
\begin{pgfscope}%
\pgfsetrectcap%
\pgfsetmiterjoin%
\pgfsetlinewidth{1.003750pt}%
\definecolor{currentstroke}{rgb}{0.000000,0.000000,0.000000}%
\pgfsetstrokecolor{currentstroke}%
\pgfsetdash{}{0pt}%
\pgfpathmoveto{\pgfqpoint{0.894063in}{5.600556in}}%
\pgfpathlineto{\pgfqpoint{7.607500in}{5.600556in}}%
\pgfusepath{stroke}%
\end{pgfscope}%
\begin{pgfscope}%
\pgfsetrectcap%
\pgfsetmiterjoin%
\pgfsetlinewidth{1.003750pt}%
\definecolor{currentstroke}{rgb}{0.000000,0.000000,0.000000}%
\pgfsetstrokecolor{currentstroke}%
\pgfsetdash{}{0pt}%
\pgfpathmoveto{\pgfqpoint{7.607500in}{3.540000in}}%
\pgfpathlineto{\pgfqpoint{7.607500in}{5.600556in}}%
\pgfusepath{stroke}%
\end{pgfscope}%
\begin{pgfscope}%
\pgfsetrectcap%
\pgfsetmiterjoin%
\pgfsetlinewidth{1.003750pt}%
\definecolor{currentstroke}{rgb}{0.000000,0.000000,0.000000}%
\pgfsetstrokecolor{currentstroke}%
\pgfsetdash{}{0pt}%
\pgfpathmoveto{\pgfqpoint{0.894063in}{3.540000in}}%
\pgfpathlineto{\pgfqpoint{7.607500in}{3.540000in}}%
\pgfusepath{stroke}%
\end{pgfscope}%
\begin{pgfscope}%
\pgfsetrectcap%
\pgfsetmiterjoin%
\pgfsetlinewidth{1.003750pt}%
\definecolor{currentstroke}{rgb}{0.000000,0.000000,0.000000}%
\pgfsetstrokecolor{currentstroke}%
\pgfsetdash{}{0pt}%
\pgfpathmoveto{\pgfqpoint{0.894063in}{3.540000in}}%
\pgfpathlineto{\pgfqpoint{0.894063in}{5.600556in}}%
\pgfusepath{stroke}%
\end{pgfscope}%
\begin{pgfscope}%
\pgfsetbuttcap%
\pgfsetroundjoin%
\definecolor{currentfill}{rgb}{0.000000,0.000000,0.000000}%
\pgfsetfillcolor{currentfill}%
\pgfsetlinewidth{0.501875pt}%
\definecolor{currentstroke}{rgb}{0.000000,0.000000,0.000000}%
\pgfsetstrokecolor{currentstroke}%
\pgfsetdash{}{0pt}%
\pgfsys@defobject{currentmarker}{\pgfqpoint{0.000000in}{0.000000in}}{\pgfqpoint{0.000000in}{0.055556in}}{%
\pgfpathmoveto{\pgfqpoint{0.000000in}{0.000000in}}%
\pgfpathlineto{\pgfqpoint{0.000000in}{0.055556in}}%
\pgfusepath{stroke,fill}%
}%
\begin{pgfscope}%
\pgfsys@transformshift{0.894063in}{3.540000in}%
\pgfsys@useobject{currentmarker}{}%
\end{pgfscope}%
\end{pgfscope}%
\begin{pgfscope}%
\pgfsetbuttcap%
\pgfsetroundjoin%
\definecolor{currentfill}{rgb}{0.000000,0.000000,0.000000}%
\pgfsetfillcolor{currentfill}%
\pgfsetlinewidth{0.501875pt}%
\definecolor{currentstroke}{rgb}{0.000000,0.000000,0.000000}%
\pgfsetstrokecolor{currentstroke}%
\pgfsetdash{}{0pt}%
\pgfsys@defobject{currentmarker}{\pgfqpoint{0.000000in}{-0.055556in}}{\pgfqpoint{0.000000in}{0.000000in}}{%
\pgfpathmoveto{\pgfqpoint{0.000000in}{0.000000in}}%
\pgfpathlineto{\pgfqpoint{0.000000in}{-0.055556in}}%
\pgfusepath{stroke,fill}%
}%
\begin{pgfscope}%
\pgfsys@transformshift{0.894063in}{5.600556in}%
\pgfsys@useobject{currentmarker}{}%
\end{pgfscope}%
\end{pgfscope}%
\begin{pgfscope}%
\pgftext[x=0.894063in,y=3.484444in,,top]{\sffamily\fontsize{12.000000}{14.400000}\selectfont 0}%
\end{pgfscope}%
\begin{pgfscope}%
\pgfsetbuttcap%
\pgfsetroundjoin%
\definecolor{currentfill}{rgb}{0.000000,0.000000,0.000000}%
\pgfsetfillcolor{currentfill}%
\pgfsetlinewidth{0.501875pt}%
\definecolor{currentstroke}{rgb}{0.000000,0.000000,0.000000}%
\pgfsetstrokecolor{currentstroke}%
\pgfsetdash{}{0pt}%
\pgfsys@defobject{currentmarker}{\pgfqpoint{0.000000in}{0.000000in}}{\pgfqpoint{0.000000in}{0.055556in}}{%
\pgfpathmoveto{\pgfqpoint{0.000000in}{0.000000in}}%
\pgfpathlineto{\pgfqpoint{0.000000in}{0.055556in}}%
\pgfusepath{stroke,fill}%
}%
\begin{pgfscope}%
\pgfsys@transformshift{2.236750in}{3.540000in}%
\pgfsys@useobject{currentmarker}{}%
\end{pgfscope}%
\end{pgfscope}%
\begin{pgfscope}%
\pgfsetbuttcap%
\pgfsetroundjoin%
\definecolor{currentfill}{rgb}{0.000000,0.000000,0.000000}%
\pgfsetfillcolor{currentfill}%
\pgfsetlinewidth{0.501875pt}%
\definecolor{currentstroke}{rgb}{0.000000,0.000000,0.000000}%
\pgfsetstrokecolor{currentstroke}%
\pgfsetdash{}{0pt}%
\pgfsys@defobject{currentmarker}{\pgfqpoint{0.000000in}{-0.055556in}}{\pgfqpoint{0.000000in}{0.000000in}}{%
\pgfpathmoveto{\pgfqpoint{0.000000in}{0.000000in}}%
\pgfpathlineto{\pgfqpoint{0.000000in}{-0.055556in}}%
\pgfusepath{stroke,fill}%
}%
\begin{pgfscope}%
\pgfsys@transformshift{2.236750in}{5.600556in}%
\pgfsys@useobject{currentmarker}{}%
\end{pgfscope}%
\end{pgfscope}%
\begin{pgfscope}%
\pgftext[x=2.236750in,y=3.484444in,,top]{\sffamily\fontsize{12.000000}{14.400000}\selectfont 1000}%
\end{pgfscope}%
\begin{pgfscope}%
\pgfsetbuttcap%
\pgfsetroundjoin%
\definecolor{currentfill}{rgb}{0.000000,0.000000,0.000000}%
\pgfsetfillcolor{currentfill}%
\pgfsetlinewidth{0.501875pt}%
\definecolor{currentstroke}{rgb}{0.000000,0.000000,0.000000}%
\pgfsetstrokecolor{currentstroke}%
\pgfsetdash{}{0pt}%
\pgfsys@defobject{currentmarker}{\pgfqpoint{0.000000in}{0.000000in}}{\pgfqpoint{0.000000in}{0.055556in}}{%
\pgfpathmoveto{\pgfqpoint{0.000000in}{0.000000in}}%
\pgfpathlineto{\pgfqpoint{0.000000in}{0.055556in}}%
\pgfusepath{stroke,fill}%
}%
\begin{pgfscope}%
\pgfsys@transformshift{3.579438in}{3.540000in}%
\pgfsys@useobject{currentmarker}{}%
\end{pgfscope}%
\end{pgfscope}%
\begin{pgfscope}%
\pgfsetbuttcap%
\pgfsetroundjoin%
\definecolor{currentfill}{rgb}{0.000000,0.000000,0.000000}%
\pgfsetfillcolor{currentfill}%
\pgfsetlinewidth{0.501875pt}%
\definecolor{currentstroke}{rgb}{0.000000,0.000000,0.000000}%
\pgfsetstrokecolor{currentstroke}%
\pgfsetdash{}{0pt}%
\pgfsys@defobject{currentmarker}{\pgfqpoint{0.000000in}{-0.055556in}}{\pgfqpoint{0.000000in}{0.000000in}}{%
\pgfpathmoveto{\pgfqpoint{0.000000in}{0.000000in}}%
\pgfpathlineto{\pgfqpoint{0.000000in}{-0.055556in}}%
\pgfusepath{stroke,fill}%
}%
\begin{pgfscope}%
\pgfsys@transformshift{3.579438in}{5.600556in}%
\pgfsys@useobject{currentmarker}{}%
\end{pgfscope}%
\end{pgfscope}%
\begin{pgfscope}%
\pgftext[x=3.579438in,y=3.484444in,,top]{\sffamily\fontsize{12.000000}{14.400000}\selectfont 2000}%
\end{pgfscope}%
\begin{pgfscope}%
\pgfsetbuttcap%
\pgfsetroundjoin%
\definecolor{currentfill}{rgb}{0.000000,0.000000,0.000000}%
\pgfsetfillcolor{currentfill}%
\pgfsetlinewidth{0.501875pt}%
\definecolor{currentstroke}{rgb}{0.000000,0.000000,0.000000}%
\pgfsetstrokecolor{currentstroke}%
\pgfsetdash{}{0pt}%
\pgfsys@defobject{currentmarker}{\pgfqpoint{0.000000in}{0.000000in}}{\pgfqpoint{0.000000in}{0.055556in}}{%
\pgfpathmoveto{\pgfqpoint{0.000000in}{0.000000in}}%
\pgfpathlineto{\pgfqpoint{0.000000in}{0.055556in}}%
\pgfusepath{stroke,fill}%
}%
\begin{pgfscope}%
\pgfsys@transformshift{4.922125in}{3.540000in}%
\pgfsys@useobject{currentmarker}{}%
\end{pgfscope}%
\end{pgfscope}%
\begin{pgfscope}%
\pgfsetbuttcap%
\pgfsetroundjoin%
\definecolor{currentfill}{rgb}{0.000000,0.000000,0.000000}%
\pgfsetfillcolor{currentfill}%
\pgfsetlinewidth{0.501875pt}%
\definecolor{currentstroke}{rgb}{0.000000,0.000000,0.000000}%
\pgfsetstrokecolor{currentstroke}%
\pgfsetdash{}{0pt}%
\pgfsys@defobject{currentmarker}{\pgfqpoint{0.000000in}{-0.055556in}}{\pgfqpoint{0.000000in}{0.000000in}}{%
\pgfpathmoveto{\pgfqpoint{0.000000in}{0.000000in}}%
\pgfpathlineto{\pgfqpoint{0.000000in}{-0.055556in}}%
\pgfusepath{stroke,fill}%
}%
\begin{pgfscope}%
\pgfsys@transformshift{4.922125in}{5.600556in}%
\pgfsys@useobject{currentmarker}{}%
\end{pgfscope}%
\end{pgfscope}%
\begin{pgfscope}%
\pgftext[x=4.922125in,y=3.484444in,,top]{\sffamily\fontsize{12.000000}{14.400000}\selectfont 3000}%
\end{pgfscope}%
\begin{pgfscope}%
\pgfsetbuttcap%
\pgfsetroundjoin%
\definecolor{currentfill}{rgb}{0.000000,0.000000,0.000000}%
\pgfsetfillcolor{currentfill}%
\pgfsetlinewidth{0.501875pt}%
\definecolor{currentstroke}{rgb}{0.000000,0.000000,0.000000}%
\pgfsetstrokecolor{currentstroke}%
\pgfsetdash{}{0pt}%
\pgfsys@defobject{currentmarker}{\pgfqpoint{0.000000in}{0.000000in}}{\pgfqpoint{0.000000in}{0.055556in}}{%
\pgfpathmoveto{\pgfqpoint{0.000000in}{0.000000in}}%
\pgfpathlineto{\pgfqpoint{0.000000in}{0.055556in}}%
\pgfusepath{stroke,fill}%
}%
\begin{pgfscope}%
\pgfsys@transformshift{6.264813in}{3.540000in}%
\pgfsys@useobject{currentmarker}{}%
\end{pgfscope}%
\end{pgfscope}%
\begin{pgfscope}%
\pgfsetbuttcap%
\pgfsetroundjoin%
\definecolor{currentfill}{rgb}{0.000000,0.000000,0.000000}%
\pgfsetfillcolor{currentfill}%
\pgfsetlinewidth{0.501875pt}%
\definecolor{currentstroke}{rgb}{0.000000,0.000000,0.000000}%
\pgfsetstrokecolor{currentstroke}%
\pgfsetdash{}{0pt}%
\pgfsys@defobject{currentmarker}{\pgfqpoint{0.000000in}{-0.055556in}}{\pgfqpoint{0.000000in}{0.000000in}}{%
\pgfpathmoveto{\pgfqpoint{0.000000in}{0.000000in}}%
\pgfpathlineto{\pgfqpoint{0.000000in}{-0.055556in}}%
\pgfusepath{stroke,fill}%
}%
\begin{pgfscope}%
\pgfsys@transformshift{6.264813in}{5.600556in}%
\pgfsys@useobject{currentmarker}{}%
\end{pgfscope}%
\end{pgfscope}%
\begin{pgfscope}%
\pgftext[x=6.264813in,y=3.484444in,,top]{\sffamily\fontsize{12.000000}{14.400000}\selectfont 4000}%
\end{pgfscope}%
\begin{pgfscope}%
\pgfsetbuttcap%
\pgfsetroundjoin%
\definecolor{currentfill}{rgb}{0.000000,0.000000,0.000000}%
\pgfsetfillcolor{currentfill}%
\pgfsetlinewidth{0.501875pt}%
\definecolor{currentstroke}{rgb}{0.000000,0.000000,0.000000}%
\pgfsetstrokecolor{currentstroke}%
\pgfsetdash{}{0pt}%
\pgfsys@defobject{currentmarker}{\pgfqpoint{0.000000in}{0.000000in}}{\pgfqpoint{0.000000in}{0.055556in}}{%
\pgfpathmoveto{\pgfqpoint{0.000000in}{0.000000in}}%
\pgfpathlineto{\pgfqpoint{0.000000in}{0.055556in}}%
\pgfusepath{stroke,fill}%
}%
\begin{pgfscope}%
\pgfsys@transformshift{7.607500in}{3.540000in}%
\pgfsys@useobject{currentmarker}{}%
\end{pgfscope}%
\end{pgfscope}%
\begin{pgfscope}%
\pgfsetbuttcap%
\pgfsetroundjoin%
\definecolor{currentfill}{rgb}{0.000000,0.000000,0.000000}%
\pgfsetfillcolor{currentfill}%
\pgfsetlinewidth{0.501875pt}%
\definecolor{currentstroke}{rgb}{0.000000,0.000000,0.000000}%
\pgfsetstrokecolor{currentstroke}%
\pgfsetdash{}{0pt}%
\pgfsys@defobject{currentmarker}{\pgfqpoint{0.000000in}{-0.055556in}}{\pgfqpoint{0.000000in}{0.000000in}}{%
\pgfpathmoveto{\pgfqpoint{0.000000in}{0.000000in}}%
\pgfpathlineto{\pgfqpoint{0.000000in}{-0.055556in}}%
\pgfusepath{stroke,fill}%
}%
\begin{pgfscope}%
\pgfsys@transformshift{7.607500in}{5.600556in}%
\pgfsys@useobject{currentmarker}{}%
\end{pgfscope}%
\end{pgfscope}%
\begin{pgfscope}%
\pgftext[x=7.607500in,y=3.484444in,,top]{\sffamily\fontsize{12.000000}{14.400000}\selectfont 5000}%
\end{pgfscope}%
\begin{pgfscope}%
\pgftext[x=4.250781in,y=3.253705in,,top]{\sffamily\fontsize{12.000000}{14.400000}\selectfont Requests Served}%
\end{pgfscope}%
\begin{pgfscope}%
\pgfsetbuttcap%
\pgfsetroundjoin%
\definecolor{currentfill}{rgb}{0.000000,0.000000,0.000000}%
\pgfsetfillcolor{currentfill}%
\pgfsetlinewidth{0.501875pt}%
\definecolor{currentstroke}{rgb}{0.000000,0.000000,0.000000}%
\pgfsetstrokecolor{currentstroke}%
\pgfsetdash{}{0pt}%
\pgfsys@defobject{currentmarker}{\pgfqpoint{0.000000in}{0.000000in}}{\pgfqpoint{0.055556in}{0.000000in}}{%
\pgfpathmoveto{\pgfqpoint{0.000000in}{0.000000in}}%
\pgfpathlineto{\pgfqpoint{0.055556in}{0.000000in}}%
\pgfusepath{stroke,fill}%
}%
\begin{pgfscope}%
\pgfsys@transformshift{0.894063in}{3.540000in}%
\pgfsys@useobject{currentmarker}{}%
\end{pgfscope}%
\end{pgfscope}%
\begin{pgfscope}%
\pgfsetbuttcap%
\pgfsetroundjoin%
\definecolor{currentfill}{rgb}{0.000000,0.000000,0.000000}%
\pgfsetfillcolor{currentfill}%
\pgfsetlinewidth{0.501875pt}%
\definecolor{currentstroke}{rgb}{0.000000,0.000000,0.000000}%
\pgfsetstrokecolor{currentstroke}%
\pgfsetdash{}{0pt}%
\pgfsys@defobject{currentmarker}{\pgfqpoint{-0.055556in}{0.000000in}}{\pgfqpoint{0.000000in}{0.000000in}}{%
\pgfpathmoveto{\pgfqpoint{0.000000in}{0.000000in}}%
\pgfpathlineto{\pgfqpoint{-0.055556in}{0.000000in}}%
\pgfusepath{stroke,fill}%
}%
\begin{pgfscope}%
\pgfsys@transformshift{7.607500in}{3.540000in}%
\pgfsys@useobject{currentmarker}{}%
\end{pgfscope}%
\end{pgfscope}%
\begin{pgfscope}%
\pgftext[x=0.838507in,y=3.540000in,right,]{\sffamily\fontsize{12.000000}{14.400000}\selectfont 0}%
\end{pgfscope}%
\begin{pgfscope}%
\pgfsetbuttcap%
\pgfsetroundjoin%
\definecolor{currentfill}{rgb}{0.000000,0.000000,0.000000}%
\pgfsetfillcolor{currentfill}%
\pgfsetlinewidth{0.501875pt}%
\definecolor{currentstroke}{rgb}{0.000000,0.000000,0.000000}%
\pgfsetstrokecolor{currentstroke}%
\pgfsetdash{}{0pt}%
\pgfsys@defobject{currentmarker}{\pgfqpoint{0.000000in}{0.000000in}}{\pgfqpoint{0.055556in}{0.000000in}}{%
\pgfpathmoveto{\pgfqpoint{0.000000in}{0.000000in}}%
\pgfpathlineto{\pgfqpoint{0.055556in}{0.000000in}}%
\pgfusepath{stroke,fill}%
}%
\begin{pgfscope}%
\pgfsys@transformshift{0.894063in}{3.883426in}%
\pgfsys@useobject{currentmarker}{}%
\end{pgfscope}%
\end{pgfscope}%
\begin{pgfscope}%
\pgfsetbuttcap%
\pgfsetroundjoin%
\definecolor{currentfill}{rgb}{0.000000,0.000000,0.000000}%
\pgfsetfillcolor{currentfill}%
\pgfsetlinewidth{0.501875pt}%
\definecolor{currentstroke}{rgb}{0.000000,0.000000,0.000000}%
\pgfsetstrokecolor{currentstroke}%
\pgfsetdash{}{0pt}%
\pgfsys@defobject{currentmarker}{\pgfqpoint{-0.055556in}{0.000000in}}{\pgfqpoint{0.000000in}{0.000000in}}{%
\pgfpathmoveto{\pgfqpoint{0.000000in}{0.000000in}}%
\pgfpathlineto{\pgfqpoint{-0.055556in}{0.000000in}}%
\pgfusepath{stroke,fill}%
}%
\begin{pgfscope}%
\pgfsys@transformshift{7.607500in}{3.883426in}%
\pgfsys@useobject{currentmarker}{}%
\end{pgfscope}%
\end{pgfscope}%
\begin{pgfscope}%
\pgftext[x=0.838507in,y=3.883426in,right,]{\sffamily\fontsize{12.000000}{14.400000}\selectfont 1000}%
\end{pgfscope}%
\begin{pgfscope}%
\pgfsetbuttcap%
\pgfsetroundjoin%
\definecolor{currentfill}{rgb}{0.000000,0.000000,0.000000}%
\pgfsetfillcolor{currentfill}%
\pgfsetlinewidth{0.501875pt}%
\definecolor{currentstroke}{rgb}{0.000000,0.000000,0.000000}%
\pgfsetstrokecolor{currentstroke}%
\pgfsetdash{}{0pt}%
\pgfsys@defobject{currentmarker}{\pgfqpoint{0.000000in}{0.000000in}}{\pgfqpoint{0.055556in}{0.000000in}}{%
\pgfpathmoveto{\pgfqpoint{0.000000in}{0.000000in}}%
\pgfpathlineto{\pgfqpoint{0.055556in}{0.000000in}}%
\pgfusepath{stroke,fill}%
}%
\begin{pgfscope}%
\pgfsys@transformshift{0.894063in}{4.226852in}%
\pgfsys@useobject{currentmarker}{}%
\end{pgfscope}%
\end{pgfscope}%
\begin{pgfscope}%
\pgfsetbuttcap%
\pgfsetroundjoin%
\definecolor{currentfill}{rgb}{0.000000,0.000000,0.000000}%
\pgfsetfillcolor{currentfill}%
\pgfsetlinewidth{0.501875pt}%
\definecolor{currentstroke}{rgb}{0.000000,0.000000,0.000000}%
\pgfsetstrokecolor{currentstroke}%
\pgfsetdash{}{0pt}%
\pgfsys@defobject{currentmarker}{\pgfqpoint{-0.055556in}{0.000000in}}{\pgfqpoint{0.000000in}{0.000000in}}{%
\pgfpathmoveto{\pgfqpoint{0.000000in}{0.000000in}}%
\pgfpathlineto{\pgfqpoint{-0.055556in}{0.000000in}}%
\pgfusepath{stroke,fill}%
}%
\begin{pgfscope}%
\pgfsys@transformshift{7.607500in}{4.226852in}%
\pgfsys@useobject{currentmarker}{}%
\end{pgfscope}%
\end{pgfscope}%
\begin{pgfscope}%
\pgftext[x=0.838507in,y=4.226852in,right,]{\sffamily\fontsize{12.000000}{14.400000}\selectfont 2000}%
\end{pgfscope}%
\begin{pgfscope}%
\pgfsetbuttcap%
\pgfsetroundjoin%
\definecolor{currentfill}{rgb}{0.000000,0.000000,0.000000}%
\pgfsetfillcolor{currentfill}%
\pgfsetlinewidth{0.501875pt}%
\definecolor{currentstroke}{rgb}{0.000000,0.000000,0.000000}%
\pgfsetstrokecolor{currentstroke}%
\pgfsetdash{}{0pt}%
\pgfsys@defobject{currentmarker}{\pgfqpoint{0.000000in}{0.000000in}}{\pgfqpoint{0.055556in}{0.000000in}}{%
\pgfpathmoveto{\pgfqpoint{0.000000in}{0.000000in}}%
\pgfpathlineto{\pgfqpoint{0.055556in}{0.000000in}}%
\pgfusepath{stroke,fill}%
}%
\begin{pgfscope}%
\pgfsys@transformshift{0.894063in}{4.570278in}%
\pgfsys@useobject{currentmarker}{}%
\end{pgfscope}%
\end{pgfscope}%
\begin{pgfscope}%
\pgfsetbuttcap%
\pgfsetroundjoin%
\definecolor{currentfill}{rgb}{0.000000,0.000000,0.000000}%
\pgfsetfillcolor{currentfill}%
\pgfsetlinewidth{0.501875pt}%
\definecolor{currentstroke}{rgb}{0.000000,0.000000,0.000000}%
\pgfsetstrokecolor{currentstroke}%
\pgfsetdash{}{0pt}%
\pgfsys@defobject{currentmarker}{\pgfqpoint{-0.055556in}{0.000000in}}{\pgfqpoint{0.000000in}{0.000000in}}{%
\pgfpathmoveto{\pgfqpoint{0.000000in}{0.000000in}}%
\pgfpathlineto{\pgfqpoint{-0.055556in}{0.000000in}}%
\pgfusepath{stroke,fill}%
}%
\begin{pgfscope}%
\pgfsys@transformshift{7.607500in}{4.570278in}%
\pgfsys@useobject{currentmarker}{}%
\end{pgfscope}%
\end{pgfscope}%
\begin{pgfscope}%
\pgftext[x=0.838507in,y=4.570278in,right,]{\sffamily\fontsize{12.000000}{14.400000}\selectfont 3000}%
\end{pgfscope}%
\begin{pgfscope}%
\pgfsetbuttcap%
\pgfsetroundjoin%
\definecolor{currentfill}{rgb}{0.000000,0.000000,0.000000}%
\pgfsetfillcolor{currentfill}%
\pgfsetlinewidth{0.501875pt}%
\definecolor{currentstroke}{rgb}{0.000000,0.000000,0.000000}%
\pgfsetstrokecolor{currentstroke}%
\pgfsetdash{}{0pt}%
\pgfsys@defobject{currentmarker}{\pgfqpoint{0.000000in}{0.000000in}}{\pgfqpoint{0.055556in}{0.000000in}}{%
\pgfpathmoveto{\pgfqpoint{0.000000in}{0.000000in}}%
\pgfpathlineto{\pgfqpoint{0.055556in}{0.000000in}}%
\pgfusepath{stroke,fill}%
}%
\begin{pgfscope}%
\pgfsys@transformshift{0.894063in}{4.913704in}%
\pgfsys@useobject{currentmarker}{}%
\end{pgfscope}%
\end{pgfscope}%
\begin{pgfscope}%
\pgfsetbuttcap%
\pgfsetroundjoin%
\definecolor{currentfill}{rgb}{0.000000,0.000000,0.000000}%
\pgfsetfillcolor{currentfill}%
\pgfsetlinewidth{0.501875pt}%
\definecolor{currentstroke}{rgb}{0.000000,0.000000,0.000000}%
\pgfsetstrokecolor{currentstroke}%
\pgfsetdash{}{0pt}%
\pgfsys@defobject{currentmarker}{\pgfqpoint{-0.055556in}{0.000000in}}{\pgfqpoint{0.000000in}{0.000000in}}{%
\pgfpathmoveto{\pgfqpoint{0.000000in}{0.000000in}}%
\pgfpathlineto{\pgfqpoint{-0.055556in}{0.000000in}}%
\pgfusepath{stroke,fill}%
}%
\begin{pgfscope}%
\pgfsys@transformshift{7.607500in}{4.913704in}%
\pgfsys@useobject{currentmarker}{}%
\end{pgfscope}%
\end{pgfscope}%
\begin{pgfscope}%
\pgftext[x=0.838507in,y=4.913704in,right,]{\sffamily\fontsize{12.000000}{14.400000}\selectfont 4000}%
\end{pgfscope}%
\begin{pgfscope}%
\pgfsetbuttcap%
\pgfsetroundjoin%
\definecolor{currentfill}{rgb}{0.000000,0.000000,0.000000}%
\pgfsetfillcolor{currentfill}%
\pgfsetlinewidth{0.501875pt}%
\definecolor{currentstroke}{rgb}{0.000000,0.000000,0.000000}%
\pgfsetstrokecolor{currentstroke}%
\pgfsetdash{}{0pt}%
\pgfsys@defobject{currentmarker}{\pgfqpoint{0.000000in}{0.000000in}}{\pgfqpoint{0.055556in}{0.000000in}}{%
\pgfpathmoveto{\pgfqpoint{0.000000in}{0.000000in}}%
\pgfpathlineto{\pgfqpoint{0.055556in}{0.000000in}}%
\pgfusepath{stroke,fill}%
}%
\begin{pgfscope}%
\pgfsys@transformshift{0.894063in}{5.257130in}%
\pgfsys@useobject{currentmarker}{}%
\end{pgfscope}%
\end{pgfscope}%
\begin{pgfscope}%
\pgfsetbuttcap%
\pgfsetroundjoin%
\definecolor{currentfill}{rgb}{0.000000,0.000000,0.000000}%
\pgfsetfillcolor{currentfill}%
\pgfsetlinewidth{0.501875pt}%
\definecolor{currentstroke}{rgb}{0.000000,0.000000,0.000000}%
\pgfsetstrokecolor{currentstroke}%
\pgfsetdash{}{0pt}%
\pgfsys@defobject{currentmarker}{\pgfqpoint{-0.055556in}{0.000000in}}{\pgfqpoint{0.000000in}{0.000000in}}{%
\pgfpathmoveto{\pgfqpoint{0.000000in}{0.000000in}}%
\pgfpathlineto{\pgfqpoint{-0.055556in}{0.000000in}}%
\pgfusepath{stroke,fill}%
}%
\begin{pgfscope}%
\pgfsys@transformshift{7.607500in}{5.257130in}%
\pgfsys@useobject{currentmarker}{}%
\end{pgfscope}%
\end{pgfscope}%
\begin{pgfscope}%
\pgftext[x=0.838507in,y=5.257130in,right,]{\sffamily\fontsize{12.000000}{14.400000}\selectfont 5000}%
\end{pgfscope}%
\begin{pgfscope}%
\pgfsetbuttcap%
\pgfsetroundjoin%
\definecolor{currentfill}{rgb}{0.000000,0.000000,0.000000}%
\pgfsetfillcolor{currentfill}%
\pgfsetlinewidth{0.501875pt}%
\definecolor{currentstroke}{rgb}{0.000000,0.000000,0.000000}%
\pgfsetstrokecolor{currentstroke}%
\pgfsetdash{}{0pt}%
\pgfsys@defobject{currentmarker}{\pgfqpoint{0.000000in}{0.000000in}}{\pgfqpoint{0.055556in}{0.000000in}}{%
\pgfpathmoveto{\pgfqpoint{0.000000in}{0.000000in}}%
\pgfpathlineto{\pgfqpoint{0.055556in}{0.000000in}}%
\pgfusepath{stroke,fill}%
}%
\begin{pgfscope}%
\pgfsys@transformshift{0.894063in}{5.600556in}%
\pgfsys@useobject{currentmarker}{}%
\end{pgfscope}%
\end{pgfscope}%
\begin{pgfscope}%
\pgfsetbuttcap%
\pgfsetroundjoin%
\definecolor{currentfill}{rgb}{0.000000,0.000000,0.000000}%
\pgfsetfillcolor{currentfill}%
\pgfsetlinewidth{0.501875pt}%
\definecolor{currentstroke}{rgb}{0.000000,0.000000,0.000000}%
\pgfsetstrokecolor{currentstroke}%
\pgfsetdash{}{0pt}%
\pgfsys@defobject{currentmarker}{\pgfqpoint{-0.055556in}{0.000000in}}{\pgfqpoint{0.000000in}{0.000000in}}{%
\pgfpathmoveto{\pgfqpoint{0.000000in}{0.000000in}}%
\pgfpathlineto{\pgfqpoint{-0.055556in}{0.000000in}}%
\pgfusepath{stroke,fill}%
}%
\begin{pgfscope}%
\pgfsys@transformshift{7.607500in}{5.600556in}%
\pgfsys@useobject{currentmarker}{}%
\end{pgfscope}%
\end{pgfscope}%
\begin{pgfscope}%
\pgftext[x=0.838507in,y=5.600556in,right,]{\sffamily\fontsize{12.000000}{14.400000}\selectfont 6000}%
\end{pgfscope}%
\begin{pgfscope}%
\pgftext[x=0.344909in,y=4.570278in,,bottom,rotate=90.000000]{\sffamily\fontsize{12.000000}{14.400000}\selectfont Memory (Mb)}%
\end{pgfscope}%
\begin{pgfscope}%
\pgftext[x=4.250781in,y=5.670000in,,base]{\sffamily\fontsize{14.400000}{17.280000}\selectfont Shared memory}%
\end{pgfscope}%
\begin{pgfscope}%
\pgfsetbuttcap%
\pgfsetmiterjoin%
\definecolor{currentfill}{rgb}{1.000000,1.000000,1.000000}%
\pgfsetfillcolor{currentfill}%
\pgfsetlinewidth{0.000000pt}%
\definecolor{currentstroke}{rgb}{0.000000,0.000000,0.000000}%
\pgfsetstrokecolor{currentstroke}%
\pgfsetstrokeopacity{0.000000}%
\pgfsetdash{}{0pt}%
\pgfpathmoveto{\pgfqpoint{0.894063in}{0.630000in}}%
\pgfpathlineto{\pgfqpoint{7.607500in}{0.630000in}}%
\pgfpathlineto{\pgfqpoint{7.607500in}{2.690556in}}%
\pgfpathlineto{\pgfqpoint{0.894063in}{2.690556in}}%
\pgfpathclose%
\pgfusepath{fill}%
\end{pgfscope}%
\begin{pgfscope}%
\pgfpathrectangle{\pgfqpoint{0.894063in}{0.630000in}}{\pgfqpoint{6.713438in}{2.060556in}} %
\pgfusepath{clip}%
\pgfsetbuttcap%
\pgfsetroundjoin%
\definecolor{currentfill}{rgb}{0.000000,0.500000,0.000000}%
\pgfsetfillcolor{currentfill}%
\pgfsetlinewidth{1.003750pt}%
\definecolor{currentstroke}{rgb}{0.000000,0.500000,0.000000}%
\pgfsetstrokecolor{currentstroke}%
\pgfsetdash{}{0pt}%
\pgfpathmoveto{\pgfqpoint{6.667619in}{1.556484in}}%
\pgfpathcurveto{\pgfqpoint{6.675855in}{1.556484in}}{\pgfqpoint{6.683755in}{1.559756in}}{\pgfqpoint{6.689579in}{1.565580in}}%
\pgfpathcurveto{\pgfqpoint{6.695403in}{1.571404in}}{\pgfqpoint{6.698675in}{1.579304in}}{\pgfqpoint{6.698675in}{1.587540in}}%
\pgfpathcurveto{\pgfqpoint{6.698675in}{1.595776in}}{\pgfqpoint{6.695403in}{1.603676in}}{\pgfqpoint{6.689579in}{1.609500in}}%
\pgfpathcurveto{\pgfqpoint{6.683755in}{1.615324in}}{\pgfqpoint{6.675855in}{1.618597in}}{\pgfqpoint{6.667619in}{1.618597in}}%
\pgfpathcurveto{\pgfqpoint{6.659382in}{1.618597in}}{\pgfqpoint{6.651482in}{1.615324in}}{\pgfqpoint{6.645658in}{1.609500in}}%
\pgfpathcurveto{\pgfqpoint{6.639835in}{1.603676in}}{\pgfqpoint{6.636562in}{1.595776in}}{\pgfqpoint{6.636562in}{1.587540in}}%
\pgfpathcurveto{\pgfqpoint{6.636562in}{1.579304in}}{\pgfqpoint{6.639835in}{1.571404in}}{\pgfqpoint{6.645658in}{1.565580in}}%
\pgfpathcurveto{\pgfqpoint{6.651482in}{1.559756in}}{\pgfqpoint{6.659382in}{1.556484in}}{\pgfqpoint{6.667619in}{1.556484in}}%
\pgfpathclose%
\pgfusepath{stroke,fill}%
\end{pgfscope}%
\begin{pgfscope}%
\pgfpathrectangle{\pgfqpoint{0.894063in}{0.630000in}}{\pgfqpoint{6.713438in}{2.060556in}} %
\pgfusepath{clip}%
\pgfsetbuttcap%
\pgfsetroundjoin%
\definecolor{currentfill}{rgb}{0.000000,0.500000,0.000000}%
\pgfsetfillcolor{currentfill}%
\pgfsetlinewidth{1.003750pt}%
\definecolor{currentstroke}{rgb}{0.000000,0.500000,0.000000}%
\pgfsetstrokecolor{currentstroke}%
\pgfsetdash{}{0pt}%
\pgfpathmoveto{\pgfqpoint{2.639556in}{0.918547in}}%
\pgfpathcurveto{\pgfqpoint{2.647793in}{0.918547in}}{\pgfqpoint{2.655693in}{0.921820in}}{\pgfqpoint{2.661517in}{0.927644in}}%
\pgfpathcurveto{\pgfqpoint{2.667340in}{0.933468in}}{\pgfqpoint{2.670613in}{0.941368in}}{\pgfqpoint{2.670613in}{0.949604in}}%
\pgfpathcurveto{\pgfqpoint{2.670613in}{0.957840in}}{\pgfqpoint{2.667340in}{0.965740in}}{\pgfqpoint{2.661517in}{0.971564in}}%
\pgfpathcurveto{\pgfqpoint{2.655693in}{0.977388in}}{\pgfqpoint{2.647793in}{0.980660in}}{\pgfqpoint{2.639556in}{0.980660in}}%
\pgfpathcurveto{\pgfqpoint{2.631320in}{0.980660in}}{\pgfqpoint{2.623420in}{0.977388in}}{\pgfqpoint{2.617596in}{0.971564in}}%
\pgfpathcurveto{\pgfqpoint{2.611772in}{0.965740in}}{\pgfqpoint{2.608500in}{0.957840in}}{\pgfqpoint{2.608500in}{0.949604in}}%
\pgfpathcurveto{\pgfqpoint{2.608500in}{0.941368in}}{\pgfqpoint{2.611772in}{0.933468in}}{\pgfqpoint{2.617596in}{0.927644in}}%
\pgfpathcurveto{\pgfqpoint{2.623420in}{0.921820in}}{\pgfqpoint{2.631320in}{0.918547in}}{\pgfqpoint{2.639556in}{0.918547in}}%
\pgfpathclose%
\pgfusepath{stroke,fill}%
\end{pgfscope}%
\begin{pgfscope}%
\pgfpathrectangle{\pgfqpoint{0.894063in}{0.630000in}}{\pgfqpoint{6.713438in}{2.060556in}} %
\pgfusepath{clip}%
\pgfsetbuttcap%
\pgfsetroundjoin%
\definecolor{currentfill}{rgb}{0.000000,0.500000,0.000000}%
\pgfsetfillcolor{currentfill}%
\pgfsetlinewidth{1.003750pt}%
\definecolor{currentstroke}{rgb}{0.000000,0.500000,0.000000}%
\pgfsetstrokecolor{currentstroke}%
\pgfsetdash{}{0pt}%
\pgfpathmoveto{\pgfqpoint{1.699675in}{0.786419in}}%
\pgfpathcurveto{\pgfqpoint{1.707911in}{0.786419in}}{\pgfqpoint{1.715811in}{0.789691in}}{\pgfqpoint{1.721635in}{0.795515in}}%
\pgfpathcurveto{\pgfqpoint{1.727459in}{0.801339in}}{\pgfqpoint{1.730731in}{0.809239in}}{\pgfqpoint{1.730731in}{0.817475in}}%
\pgfpathcurveto{\pgfqpoint{1.730731in}{0.825712in}}{\pgfqpoint{1.727459in}{0.833612in}}{\pgfqpoint{1.721635in}{0.839435in}}%
\pgfpathcurveto{\pgfqpoint{1.715811in}{0.845259in}}{\pgfqpoint{1.707911in}{0.848532in}}{\pgfqpoint{1.699675in}{0.848532in}}%
\pgfpathcurveto{\pgfqpoint{1.691439in}{0.848532in}}{\pgfqpoint{1.683539in}{0.845259in}}{\pgfqpoint{1.677715in}{0.839435in}}%
\pgfpathcurveto{\pgfqpoint{1.671891in}{0.833612in}}{\pgfqpoint{1.668619in}{0.825712in}}{\pgfqpoint{1.668619in}{0.817475in}}%
\pgfpathcurveto{\pgfqpoint{1.668619in}{0.809239in}}{\pgfqpoint{1.671891in}{0.801339in}}{\pgfqpoint{1.677715in}{0.795515in}}%
\pgfpathcurveto{\pgfqpoint{1.683539in}{0.789691in}}{\pgfqpoint{1.691439in}{0.786419in}}{\pgfqpoint{1.699675in}{0.786419in}}%
\pgfpathclose%
\pgfusepath{stroke,fill}%
\end{pgfscope}%
\begin{pgfscope}%
\pgfpathrectangle{\pgfqpoint{0.894063in}{0.630000in}}{\pgfqpoint{6.713438in}{2.060556in}} %
\pgfusepath{clip}%
\pgfsetbuttcap%
\pgfsetroundjoin%
\definecolor{currentfill}{rgb}{0.000000,0.500000,0.000000}%
\pgfsetfillcolor{currentfill}%
\pgfsetlinewidth{1.003750pt}%
\definecolor{currentstroke}{rgb}{0.000000,0.500000,0.000000}%
\pgfsetstrokecolor{currentstroke}%
\pgfsetdash{}{0pt}%
\pgfpathmoveto{\pgfqpoint{1.162600in}{0.712504in}}%
\pgfpathcurveto{\pgfqpoint{1.170836in}{0.712504in}}{\pgfqpoint{1.178736in}{0.715776in}}{\pgfqpoint{1.184560in}{0.721600in}}%
\pgfpathcurveto{\pgfqpoint{1.190384in}{0.727424in}}{\pgfqpoint{1.193656in}{0.735324in}}{\pgfqpoint{1.193656in}{0.743560in}}%
\pgfpathcurveto{\pgfqpoint{1.193656in}{0.751796in}}{\pgfqpoint{1.190384in}{0.759696in}}{\pgfqpoint{1.184560in}{0.765520in}}%
\pgfpathcurveto{\pgfqpoint{1.178736in}{0.771344in}}{\pgfqpoint{1.170836in}{0.774617in}}{\pgfqpoint{1.162600in}{0.774617in}}%
\pgfpathcurveto{\pgfqpoint{1.154364in}{0.774617in}}{\pgfqpoint{1.146464in}{0.771344in}}{\pgfqpoint{1.140640in}{0.765520in}}%
\pgfpathcurveto{\pgfqpoint{1.134816in}{0.759696in}}{\pgfqpoint{1.131544in}{0.751796in}}{\pgfqpoint{1.131544in}{0.743560in}}%
\pgfpathcurveto{\pgfqpoint{1.131544in}{0.735324in}}{\pgfqpoint{1.134816in}{0.727424in}}{\pgfqpoint{1.140640in}{0.721600in}}%
\pgfpathcurveto{\pgfqpoint{1.146464in}{0.715776in}}{\pgfqpoint{1.154364in}{0.712504in}}{\pgfqpoint{1.162600in}{0.712504in}}%
\pgfpathclose%
\pgfusepath{stroke,fill}%
\end{pgfscope}%
\begin{pgfscope}%
\pgfpathrectangle{\pgfqpoint{0.894063in}{0.630000in}}{\pgfqpoint{6.713438in}{2.060556in}} %
\pgfusepath{clip}%
\pgfsetbuttcap%
\pgfsetroundjoin%
\definecolor{currentfill}{rgb}{0.000000,0.500000,0.000000}%
\pgfsetfillcolor{currentfill}%
\pgfsetlinewidth{1.003750pt}%
\definecolor{currentstroke}{rgb}{0.000000,0.500000,0.000000}%
\pgfsetstrokecolor{currentstroke}%
\pgfsetdash{}{0pt}%
\pgfpathmoveto{\pgfqpoint{1.833944in}{0.803627in}}%
\pgfpathcurveto{\pgfqpoint{1.842180in}{0.803627in}}{\pgfqpoint{1.850080in}{0.806900in}}{\pgfqpoint{1.855904in}{0.812724in}}%
\pgfpathcurveto{\pgfqpoint{1.861728in}{0.818547in}}{\pgfqpoint{1.865000in}{0.826448in}}{\pgfqpoint{1.865000in}{0.834684in}}%
\pgfpathcurveto{\pgfqpoint{1.865000in}{0.842920in}}{\pgfqpoint{1.861728in}{0.850820in}}{\pgfqpoint{1.855904in}{0.856644in}}%
\pgfpathcurveto{\pgfqpoint{1.850080in}{0.862468in}}{\pgfqpoint{1.842180in}{0.865740in}}{\pgfqpoint{1.833944in}{0.865740in}}%
\pgfpathcurveto{\pgfqpoint{1.825707in}{0.865740in}}{\pgfqpoint{1.817807in}{0.862468in}}{\pgfqpoint{1.811983in}{0.856644in}}%
\pgfpathcurveto{\pgfqpoint{1.806160in}{0.850820in}}{\pgfqpoint{1.802887in}{0.842920in}}{\pgfqpoint{1.802887in}{0.834684in}}%
\pgfpathcurveto{\pgfqpoint{1.802887in}{0.826448in}}{\pgfqpoint{1.806160in}{0.818547in}}{\pgfqpoint{1.811983in}{0.812724in}}%
\pgfpathcurveto{\pgfqpoint{1.817807in}{0.806900in}}{\pgfqpoint{1.825707in}{0.803627in}}{\pgfqpoint{1.833944in}{0.803627in}}%
\pgfpathclose%
\pgfusepath{stroke,fill}%
\end{pgfscope}%
\begin{pgfscope}%
\pgfpathrectangle{\pgfqpoint{0.894063in}{0.630000in}}{\pgfqpoint{6.713438in}{2.060556in}} %
\pgfusepath{clip}%
\pgfsetbuttcap%
\pgfsetroundjoin%
\definecolor{currentfill}{rgb}{0.000000,0.500000,0.000000}%
\pgfsetfillcolor{currentfill}%
\pgfsetlinewidth{1.003750pt}%
\definecolor{currentstroke}{rgb}{0.000000,0.500000,0.000000}%
\pgfsetstrokecolor{currentstroke}%
\pgfsetdash{}{0pt}%
\pgfpathmoveto{\pgfqpoint{5.996275in}{1.456076in}}%
\pgfpathcurveto{\pgfqpoint{6.004511in}{1.456076in}}{\pgfqpoint{6.012411in}{1.459348in}}{\pgfqpoint{6.018235in}{1.465172in}}%
\pgfpathcurveto{\pgfqpoint{6.024059in}{1.470996in}}{\pgfqpoint{6.027331in}{1.478896in}}{\pgfqpoint{6.027331in}{1.487132in}}%
\pgfpathcurveto{\pgfqpoint{6.027331in}{1.495369in}}{\pgfqpoint{6.024059in}{1.503269in}}{\pgfqpoint{6.018235in}{1.509092in}}%
\pgfpathcurveto{\pgfqpoint{6.012411in}{1.514916in}}{\pgfqpoint{6.004511in}{1.518189in}}{\pgfqpoint{5.996275in}{1.518189in}}%
\pgfpathcurveto{\pgfqpoint{5.988039in}{1.518189in}}{\pgfqpoint{5.980139in}{1.514916in}}{\pgfqpoint{5.974315in}{1.509092in}}%
\pgfpathcurveto{\pgfqpoint{5.968491in}{1.503269in}}{\pgfqpoint{5.965219in}{1.495369in}}{\pgfqpoint{5.965219in}{1.487132in}}%
\pgfpathcurveto{\pgfqpoint{5.965219in}{1.478896in}}{\pgfqpoint{5.968491in}{1.470996in}}{\pgfqpoint{5.974315in}{1.465172in}}%
\pgfpathcurveto{\pgfqpoint{5.980139in}{1.459348in}}{\pgfqpoint{5.988039in}{1.456076in}}{\pgfqpoint{5.996275in}{1.456076in}}%
\pgfpathclose%
\pgfusepath{stroke,fill}%
\end{pgfscope}%
\begin{pgfscope}%
\pgfpathrectangle{\pgfqpoint{0.894063in}{0.630000in}}{\pgfqpoint{6.713438in}{2.060556in}} %
\pgfusepath{clip}%
\pgfsetbuttcap%
\pgfsetroundjoin%
\definecolor{currentfill}{rgb}{0.000000,0.500000,0.000000}%
\pgfsetfillcolor{currentfill}%
\pgfsetlinewidth{1.003750pt}%
\definecolor{currentstroke}{rgb}{0.000000,0.500000,0.000000}%
\pgfsetstrokecolor{currentstroke}%
\pgfsetdash{}{0pt}%
\pgfpathmoveto{\pgfqpoint{6.399081in}{1.518393in}}%
\pgfpathcurveto{\pgfqpoint{6.407318in}{1.518393in}}{\pgfqpoint{6.415218in}{1.521665in}}{\pgfqpoint{6.421042in}{1.527489in}}%
\pgfpathcurveto{\pgfqpoint{6.426865in}{1.533313in}}{\pgfqpoint{6.430138in}{1.541213in}}{\pgfqpoint{6.430138in}{1.549449in}}%
\pgfpathcurveto{\pgfqpoint{6.430138in}{1.557686in}}{\pgfqpoint{6.426865in}{1.565586in}}{\pgfqpoint{6.421042in}{1.571410in}}%
\pgfpathcurveto{\pgfqpoint{6.415218in}{1.577234in}}{\pgfqpoint{6.407318in}{1.580506in}}{\pgfqpoint{6.399081in}{1.580506in}}%
\pgfpathcurveto{\pgfqpoint{6.390845in}{1.580506in}}{\pgfqpoint{6.382945in}{1.577234in}}{\pgfqpoint{6.377121in}{1.571410in}}%
\pgfpathcurveto{\pgfqpoint{6.371297in}{1.565586in}}{\pgfqpoint{6.368025in}{1.557686in}}{\pgfqpoint{6.368025in}{1.549449in}}%
\pgfpathcurveto{\pgfqpoint{6.368025in}{1.541213in}}{\pgfqpoint{6.371297in}{1.533313in}}{\pgfqpoint{6.377121in}{1.527489in}}%
\pgfpathcurveto{\pgfqpoint{6.382945in}{1.521665in}}{\pgfqpoint{6.390845in}{1.518393in}}{\pgfqpoint{6.399081in}{1.518393in}}%
\pgfpathclose%
\pgfusepath{stroke,fill}%
\end{pgfscope}%
\begin{pgfscope}%
\pgfpathrectangle{\pgfqpoint{0.894063in}{0.630000in}}{\pgfqpoint{6.713438in}{2.060556in}} %
\pgfusepath{clip}%
\pgfsetbuttcap%
\pgfsetroundjoin%
\definecolor{currentfill}{rgb}{0.000000,0.500000,0.000000}%
\pgfsetfillcolor{currentfill}%
\pgfsetlinewidth{1.003750pt}%
\definecolor{currentstroke}{rgb}{0.000000,0.500000,0.000000}%
\pgfsetstrokecolor{currentstroke}%
\pgfsetdash{}{0pt}%
\pgfpathmoveto{\pgfqpoint{4.787856in}{1.295429in}}%
\pgfpathcurveto{\pgfqpoint{4.796093in}{1.295429in}}{\pgfqpoint{4.803993in}{1.298701in}}{\pgfqpoint{4.809817in}{1.304525in}}%
\pgfpathcurveto{\pgfqpoint{4.815640in}{1.310349in}}{\pgfqpoint{4.818913in}{1.318249in}}{\pgfqpoint{4.818913in}{1.326485in}}%
\pgfpathcurveto{\pgfqpoint{4.818913in}{1.334722in}}{\pgfqpoint{4.815640in}{1.342622in}}{\pgfqpoint{4.809817in}{1.348446in}}%
\pgfpathcurveto{\pgfqpoint{4.803993in}{1.354270in}}{\pgfqpoint{4.796093in}{1.357542in}}{\pgfqpoint{4.787856in}{1.357542in}}%
\pgfpathcurveto{\pgfqpoint{4.779620in}{1.357542in}}{\pgfqpoint{4.771720in}{1.354270in}}{\pgfqpoint{4.765896in}{1.348446in}}%
\pgfpathcurveto{\pgfqpoint{4.760072in}{1.342622in}}{\pgfqpoint{4.756800in}{1.334722in}}{\pgfqpoint{4.756800in}{1.326485in}}%
\pgfpathcurveto{\pgfqpoint{4.756800in}{1.318249in}}{\pgfqpoint{4.760072in}{1.310349in}}{\pgfqpoint{4.765896in}{1.304525in}}%
\pgfpathcurveto{\pgfqpoint{4.771720in}{1.298701in}}{\pgfqpoint{4.779620in}{1.295429in}}{\pgfqpoint{4.787856in}{1.295429in}}%
\pgfpathclose%
\pgfusepath{stroke,fill}%
\end{pgfscope}%
\begin{pgfscope}%
\pgfpathrectangle{\pgfqpoint{0.894063in}{0.630000in}}{\pgfqpoint{6.713438in}{2.060556in}} %
\pgfusepath{clip}%
\pgfsetbuttcap%
\pgfsetroundjoin%
\definecolor{currentfill}{rgb}{0.000000,0.500000,0.000000}%
\pgfsetfillcolor{currentfill}%
\pgfsetlinewidth{1.003750pt}%
\definecolor{currentstroke}{rgb}{0.000000,0.500000,0.000000}%
\pgfsetstrokecolor{currentstroke}%
\pgfsetdash{}{0pt}%
\pgfpathmoveto{\pgfqpoint{4.922125in}{1.315134in}}%
\pgfpathcurveto{\pgfqpoint{4.930361in}{1.315134in}}{\pgfqpoint{4.938261in}{1.318406in}}{\pgfqpoint{4.944085in}{1.324230in}}%
\pgfpathcurveto{\pgfqpoint{4.949909in}{1.330054in}}{\pgfqpoint{4.953181in}{1.337954in}}{\pgfqpoint{4.953181in}{1.346190in}}%
\pgfpathcurveto{\pgfqpoint{4.953181in}{1.354427in}}{\pgfqpoint{4.949909in}{1.362327in}}{\pgfqpoint{4.944085in}{1.368150in}}%
\pgfpathcurveto{\pgfqpoint{4.938261in}{1.373974in}}{\pgfqpoint{4.930361in}{1.377247in}}{\pgfqpoint{4.922125in}{1.377247in}}%
\pgfpathcurveto{\pgfqpoint{4.913889in}{1.377247in}}{\pgfqpoint{4.905989in}{1.373974in}}{\pgfqpoint{4.900165in}{1.368150in}}%
\pgfpathcurveto{\pgfqpoint{4.894341in}{1.362327in}}{\pgfqpoint{4.891069in}{1.354427in}}{\pgfqpoint{4.891069in}{1.346190in}}%
\pgfpathcurveto{\pgfqpoint{4.891069in}{1.337954in}}{\pgfqpoint{4.894341in}{1.330054in}}{\pgfqpoint{4.900165in}{1.324230in}}%
\pgfpathcurveto{\pgfqpoint{4.905989in}{1.318406in}}{\pgfqpoint{4.913889in}{1.315134in}}{\pgfqpoint{4.922125in}{1.315134in}}%
\pgfpathclose%
\pgfusepath{stroke,fill}%
\end{pgfscope}%
\begin{pgfscope}%
\pgfpathrectangle{\pgfqpoint{0.894063in}{0.630000in}}{\pgfqpoint{6.713438in}{2.060556in}} %
\pgfusepath{clip}%
\pgfsetbuttcap%
\pgfsetroundjoin%
\definecolor{currentfill}{rgb}{0.000000,0.500000,0.000000}%
\pgfsetfillcolor{currentfill}%
\pgfsetlinewidth{1.003750pt}%
\definecolor{currentstroke}{rgb}{0.000000,0.500000,0.000000}%
\pgfsetstrokecolor{currentstroke}%
\pgfsetdash{}{0pt}%
\pgfpathmoveto{\pgfqpoint{6.130544in}{1.473967in}}%
\pgfpathcurveto{\pgfqpoint{6.138780in}{1.473967in}}{\pgfqpoint{6.146680in}{1.477240in}}{\pgfqpoint{6.152504in}{1.483063in}}%
\pgfpathcurveto{\pgfqpoint{6.158328in}{1.488887in}}{\pgfqpoint{6.161600in}{1.496787in}}{\pgfqpoint{6.161600in}{1.505024in}}%
\pgfpathcurveto{\pgfqpoint{6.161600in}{1.513260in}}{\pgfqpoint{6.158328in}{1.521160in}}{\pgfqpoint{6.152504in}{1.526984in}}%
\pgfpathcurveto{\pgfqpoint{6.146680in}{1.532808in}}{\pgfqpoint{6.138780in}{1.536080in}}{\pgfqpoint{6.130544in}{1.536080in}}%
\pgfpathcurveto{\pgfqpoint{6.122307in}{1.536080in}}{\pgfqpoint{6.114407in}{1.532808in}}{\pgfqpoint{6.108583in}{1.526984in}}%
\pgfpathcurveto{\pgfqpoint{6.102760in}{1.521160in}}{\pgfqpoint{6.099487in}{1.513260in}}{\pgfqpoint{6.099487in}{1.505024in}}%
\pgfpathcurveto{\pgfqpoint{6.099487in}{1.496787in}}{\pgfqpoint{6.102760in}{1.488887in}}{\pgfqpoint{6.108583in}{1.483063in}}%
\pgfpathcurveto{\pgfqpoint{6.114407in}{1.477240in}}{\pgfqpoint{6.122307in}{1.473967in}}{\pgfqpoint{6.130544in}{1.473967in}}%
\pgfpathclose%
\pgfusepath{stroke,fill}%
\end{pgfscope}%
\begin{pgfscope}%
\pgfpathrectangle{\pgfqpoint{0.894063in}{0.630000in}}{\pgfqpoint{6.713438in}{2.060556in}} %
\pgfusepath{clip}%
\pgfsetbuttcap%
\pgfsetroundjoin%
\definecolor{currentfill}{rgb}{0.000000,0.500000,0.000000}%
\pgfsetfillcolor{currentfill}%
\pgfsetlinewidth{1.003750pt}%
\definecolor{currentstroke}{rgb}{0.000000,0.500000,0.000000}%
\pgfsetstrokecolor{currentstroke}%
\pgfsetdash{}{0pt}%
\pgfpathmoveto{\pgfqpoint{5.727738in}{1.418020in}}%
\pgfpathcurveto{\pgfqpoint{5.735974in}{1.418020in}}{\pgfqpoint{5.743874in}{1.421293in}}{\pgfqpoint{5.749698in}{1.427116in}}%
\pgfpathcurveto{\pgfqpoint{5.755522in}{1.432940in}}{\pgfqpoint{5.758794in}{1.440840in}}{\pgfqpoint{5.758794in}{1.449077in}}%
\pgfpathcurveto{\pgfqpoint{5.758794in}{1.457313in}}{\pgfqpoint{5.755522in}{1.465213in}}{\pgfqpoint{5.749698in}{1.471037in}}%
\pgfpathcurveto{\pgfqpoint{5.743874in}{1.476861in}}{\pgfqpoint{5.735974in}{1.480133in}}{\pgfqpoint{5.727738in}{1.480133in}}%
\pgfpathcurveto{\pgfqpoint{5.719501in}{1.480133in}}{\pgfqpoint{5.711601in}{1.476861in}}{\pgfqpoint{5.705777in}{1.471037in}}%
\pgfpathcurveto{\pgfqpoint{5.699953in}{1.465213in}}{\pgfqpoint{5.696681in}{1.457313in}}{\pgfqpoint{5.696681in}{1.449077in}}%
\pgfpathcurveto{\pgfqpoint{5.696681in}{1.440840in}}{\pgfqpoint{5.699953in}{1.432940in}}{\pgfqpoint{5.705777in}{1.427116in}}%
\pgfpathcurveto{\pgfqpoint{5.711601in}{1.421293in}}{\pgfqpoint{5.719501in}{1.418020in}}{\pgfqpoint{5.727738in}{1.418020in}}%
\pgfpathclose%
\pgfusepath{stroke,fill}%
\end{pgfscope}%
\begin{pgfscope}%
\pgfpathrectangle{\pgfqpoint{0.894063in}{0.630000in}}{\pgfqpoint{6.713438in}{2.060556in}} %
\pgfusepath{clip}%
\pgfsetbuttcap%
\pgfsetroundjoin%
\definecolor{currentfill}{rgb}{0.000000,0.500000,0.000000}%
\pgfsetfillcolor{currentfill}%
\pgfsetlinewidth{1.003750pt}%
\definecolor{currentstroke}{rgb}{0.000000,0.500000,0.000000}%
\pgfsetstrokecolor{currentstroke}%
\pgfsetdash{}{0pt}%
\pgfpathmoveto{\pgfqpoint{1.028331in}{0.691769in}}%
\pgfpathcurveto{\pgfqpoint{1.036568in}{0.691769in}}{\pgfqpoint{1.044468in}{0.695041in}}{\pgfqpoint{1.050292in}{0.700865in}}%
\pgfpathcurveto{\pgfqpoint{1.056115in}{0.706689in}}{\pgfqpoint{1.059388in}{0.714589in}}{\pgfqpoint{1.059388in}{0.722825in}}%
\pgfpathcurveto{\pgfqpoint{1.059388in}{0.731061in}}{\pgfqpoint{1.056115in}{0.738961in}}{\pgfqpoint{1.050292in}{0.744785in}}%
\pgfpathcurveto{\pgfqpoint{1.044468in}{0.750609in}}{\pgfqpoint{1.036568in}{0.753882in}}{\pgfqpoint{1.028331in}{0.753882in}}%
\pgfpathcurveto{\pgfqpoint{1.020095in}{0.753882in}}{\pgfqpoint{1.012195in}{0.750609in}}{\pgfqpoint{1.006371in}{0.744785in}}%
\pgfpathcurveto{\pgfqpoint{1.000547in}{0.738961in}}{\pgfqpoint{0.997275in}{0.731061in}}{\pgfqpoint{0.997275in}{0.722825in}}%
\pgfpathcurveto{\pgfqpoint{0.997275in}{0.714589in}}{\pgfqpoint{1.000547in}{0.706689in}}{\pgfqpoint{1.006371in}{0.700865in}}%
\pgfpathcurveto{\pgfqpoint{1.012195in}{0.695041in}}{\pgfqpoint{1.020095in}{0.691769in}}{\pgfqpoint{1.028331in}{0.691769in}}%
\pgfpathclose%
\pgfusepath{stroke,fill}%
\end{pgfscope}%
\begin{pgfscope}%
\pgfpathrectangle{\pgfqpoint{0.894063in}{0.630000in}}{\pgfqpoint{6.713438in}{2.060556in}} %
\pgfusepath{clip}%
\pgfsetbuttcap%
\pgfsetroundjoin%
\definecolor{currentfill}{rgb}{0.000000,0.500000,0.000000}%
\pgfsetfillcolor{currentfill}%
\pgfsetlinewidth{1.003750pt}%
\definecolor{currentstroke}{rgb}{0.000000,0.500000,0.000000}%
\pgfsetstrokecolor{currentstroke}%
\pgfsetdash{}{0pt}%
\pgfpathmoveto{\pgfqpoint{5.324931in}{1.369061in}}%
\pgfpathcurveto{\pgfqpoint{5.333168in}{1.369061in}}{\pgfqpoint{5.341068in}{1.372334in}}{\pgfqpoint{5.346892in}{1.378158in}}%
\pgfpathcurveto{\pgfqpoint{5.352715in}{1.383982in}}{\pgfqpoint{5.355988in}{1.391882in}}{\pgfqpoint{5.355988in}{1.400118in}}%
\pgfpathcurveto{\pgfqpoint{5.355988in}{1.408354in}}{\pgfqpoint{5.352715in}{1.416254in}}{\pgfqpoint{5.346892in}{1.422078in}}%
\pgfpathcurveto{\pgfqpoint{5.341068in}{1.427902in}}{\pgfqpoint{5.333168in}{1.431174in}}{\pgfqpoint{5.324931in}{1.431174in}}%
\pgfpathcurveto{\pgfqpoint{5.316695in}{1.431174in}}{\pgfqpoint{5.308795in}{1.427902in}}{\pgfqpoint{5.302971in}{1.422078in}}%
\pgfpathcurveto{\pgfqpoint{5.297147in}{1.416254in}}{\pgfqpoint{5.293875in}{1.408354in}}{\pgfqpoint{5.293875in}{1.400118in}}%
\pgfpathcurveto{\pgfqpoint{5.293875in}{1.391882in}}{\pgfqpoint{5.297147in}{1.383982in}}{\pgfqpoint{5.302971in}{1.378158in}}%
\pgfpathcurveto{\pgfqpoint{5.308795in}{1.372334in}}{\pgfqpoint{5.316695in}{1.369061in}}{\pgfqpoint{5.324931in}{1.369061in}}%
\pgfpathclose%
\pgfusepath{stroke,fill}%
\end{pgfscope}%
\begin{pgfscope}%
\pgfpathrectangle{\pgfqpoint{0.894063in}{0.630000in}}{\pgfqpoint{6.713438in}{2.060556in}} %
\pgfusepath{clip}%
\pgfsetbuttcap%
\pgfsetroundjoin%
\definecolor{currentfill}{rgb}{0.000000,0.500000,0.000000}%
\pgfsetfillcolor{currentfill}%
\pgfsetlinewidth{1.003750pt}%
\definecolor{currentstroke}{rgb}{0.000000,0.500000,0.000000}%
\pgfsetstrokecolor{currentstroke}%
\pgfsetdash{}{0pt}%
\pgfpathmoveto{\pgfqpoint{7.338963in}{1.667412in}}%
\pgfpathcurveto{\pgfqpoint{7.347199in}{1.667412in}}{\pgfqpoint{7.355099in}{1.670685in}}{\pgfqpoint{7.360923in}{1.676508in}}%
\pgfpathcurveto{\pgfqpoint{7.366747in}{1.682332in}}{\pgfqpoint{7.370019in}{1.690232in}}{\pgfqpoint{7.370019in}{1.698469in}}%
\pgfpathcurveto{\pgfqpoint{7.370019in}{1.706705in}}{\pgfqpoint{7.366747in}{1.714605in}}{\pgfqpoint{7.360923in}{1.720429in}}%
\pgfpathcurveto{\pgfqpoint{7.355099in}{1.726253in}}{\pgfqpoint{7.347199in}{1.729525in}}{\pgfqpoint{7.338963in}{1.729525in}}%
\pgfpathcurveto{\pgfqpoint{7.330726in}{1.729525in}}{\pgfqpoint{7.322826in}{1.726253in}}{\pgfqpoint{7.317002in}{1.720429in}}%
\pgfpathcurveto{\pgfqpoint{7.311178in}{1.714605in}}{\pgfqpoint{7.307906in}{1.706705in}}{\pgfqpoint{7.307906in}{1.698469in}}%
\pgfpathcurveto{\pgfqpoint{7.307906in}{1.690232in}}{\pgfqpoint{7.311178in}{1.682332in}}{\pgfqpoint{7.317002in}{1.676508in}}%
\pgfpathcurveto{\pgfqpoint{7.322826in}{1.670685in}}{\pgfqpoint{7.330726in}{1.667412in}}{\pgfqpoint{7.338963in}{1.667412in}}%
\pgfpathclose%
\pgfusepath{stroke,fill}%
\end{pgfscope}%
\begin{pgfscope}%
\pgfpathrectangle{\pgfqpoint{0.894063in}{0.630000in}}{\pgfqpoint{6.713438in}{2.060556in}} %
\pgfusepath{clip}%
\pgfsetbuttcap%
\pgfsetroundjoin%
\definecolor{currentfill}{rgb}{0.000000,0.500000,0.000000}%
\pgfsetfillcolor{currentfill}%
\pgfsetlinewidth{1.003750pt}%
\definecolor{currentstroke}{rgb}{0.000000,0.500000,0.000000}%
\pgfsetstrokecolor{currentstroke}%
\pgfsetdash{}{0pt}%
\pgfpathmoveto{\pgfqpoint{7.204694in}{1.646601in}}%
\pgfpathcurveto{\pgfqpoint{7.212930in}{1.646601in}}{\pgfqpoint{7.220830in}{1.649873in}}{\pgfqpoint{7.226654in}{1.655697in}}%
\pgfpathcurveto{\pgfqpoint{7.232478in}{1.661521in}}{\pgfqpoint{7.235750in}{1.669421in}}{\pgfqpoint{7.235750in}{1.677657in}}%
\pgfpathcurveto{\pgfqpoint{7.235750in}{1.685893in}}{\pgfqpoint{7.232478in}{1.693793in}}{\pgfqpoint{7.226654in}{1.699617in}}%
\pgfpathcurveto{\pgfqpoint{7.220830in}{1.705441in}}{\pgfqpoint{7.212930in}{1.708714in}}{\pgfqpoint{7.204694in}{1.708714in}}%
\pgfpathcurveto{\pgfqpoint{7.196457in}{1.708714in}}{\pgfqpoint{7.188557in}{1.705441in}}{\pgfqpoint{7.182733in}{1.699617in}}%
\pgfpathcurveto{\pgfqpoint{7.176910in}{1.693793in}}{\pgfqpoint{7.173637in}{1.685893in}}{\pgfqpoint{7.173637in}{1.677657in}}%
\pgfpathcurveto{\pgfqpoint{7.173637in}{1.669421in}}{\pgfqpoint{7.176910in}{1.661521in}}{\pgfqpoint{7.182733in}{1.655697in}}%
\pgfpathcurveto{\pgfqpoint{7.188557in}{1.649873in}}{\pgfqpoint{7.196457in}{1.646601in}}{\pgfqpoint{7.204694in}{1.646601in}}%
\pgfpathclose%
\pgfusepath{stroke,fill}%
\end{pgfscope}%
\begin{pgfscope}%
\pgfpathrectangle{\pgfqpoint{0.894063in}{0.630000in}}{\pgfqpoint{6.713438in}{2.060556in}} %
\pgfusepath{clip}%
\pgfsetbuttcap%
\pgfsetroundjoin%
\definecolor{currentfill}{rgb}{0.000000,0.500000,0.000000}%
\pgfsetfillcolor{currentfill}%
\pgfsetlinewidth{1.003750pt}%
\definecolor{currentstroke}{rgb}{0.000000,0.500000,0.000000}%
\pgfsetstrokecolor{currentstroke}%
\pgfsetdash{}{0pt}%
\pgfpathmoveto{\pgfqpoint{6.264813in}{1.498458in}}%
\pgfpathcurveto{\pgfqpoint{6.273049in}{1.498458in}}{\pgfqpoint{6.280949in}{1.501731in}}{\pgfqpoint{6.286773in}{1.507555in}}%
\pgfpathcurveto{\pgfqpoint{6.292597in}{1.513379in}}{\pgfqpoint{6.295869in}{1.521279in}}{\pgfqpoint{6.295869in}{1.529515in}}%
\pgfpathcurveto{\pgfqpoint{6.295869in}{1.537751in}}{\pgfqpoint{6.292597in}{1.545651in}}{\pgfqpoint{6.286773in}{1.551475in}}%
\pgfpathcurveto{\pgfqpoint{6.280949in}{1.557299in}}{\pgfqpoint{6.273049in}{1.560571in}}{\pgfqpoint{6.264813in}{1.560571in}}%
\pgfpathcurveto{\pgfqpoint{6.256576in}{1.560571in}}{\pgfqpoint{6.248676in}{1.557299in}}{\pgfqpoint{6.242852in}{1.551475in}}%
\pgfpathcurveto{\pgfqpoint{6.237028in}{1.545651in}}{\pgfqpoint{6.233756in}{1.537751in}}{\pgfqpoint{6.233756in}{1.529515in}}%
\pgfpathcurveto{\pgfqpoint{6.233756in}{1.521279in}}{\pgfqpoint{6.237028in}{1.513379in}}{\pgfqpoint{6.242852in}{1.507555in}}%
\pgfpathcurveto{\pgfqpoint{6.248676in}{1.501731in}}{\pgfqpoint{6.256576in}{1.498458in}}{\pgfqpoint{6.264813in}{1.498458in}}%
\pgfpathclose%
\pgfusepath{stroke,fill}%
\end{pgfscope}%
\begin{pgfscope}%
\pgfpathrectangle{\pgfqpoint{0.894063in}{0.630000in}}{\pgfqpoint{6.713438in}{2.060556in}} %
\pgfusepath{clip}%
\pgfsetbuttcap%
\pgfsetroundjoin%
\definecolor{currentfill}{rgb}{0.000000,0.500000,0.000000}%
\pgfsetfillcolor{currentfill}%
\pgfsetlinewidth{1.003750pt}%
\definecolor{currentstroke}{rgb}{0.000000,0.500000,0.000000}%
\pgfsetstrokecolor{currentstroke}%
\pgfsetdash{}{0pt}%
\pgfpathmoveto{\pgfqpoint{7.473231in}{1.687718in}}%
\pgfpathcurveto{\pgfqpoint{7.481468in}{1.687718in}}{\pgfqpoint{7.489368in}{1.690990in}}{\pgfqpoint{7.495192in}{1.696814in}}%
\pgfpathcurveto{\pgfqpoint{7.501015in}{1.702638in}}{\pgfqpoint{7.504288in}{1.710538in}}{\pgfqpoint{7.504288in}{1.718774in}}%
\pgfpathcurveto{\pgfqpoint{7.504288in}{1.727010in}}{\pgfqpoint{7.501015in}{1.734910in}}{\pgfqpoint{7.495192in}{1.740734in}}%
\pgfpathcurveto{\pgfqpoint{7.489368in}{1.746558in}}{\pgfqpoint{7.481468in}{1.749831in}}{\pgfqpoint{7.473231in}{1.749831in}}%
\pgfpathcurveto{\pgfqpoint{7.464995in}{1.749831in}}{\pgfqpoint{7.457095in}{1.746558in}}{\pgfqpoint{7.451271in}{1.740734in}}%
\pgfpathcurveto{\pgfqpoint{7.445447in}{1.734910in}}{\pgfqpoint{7.442175in}{1.727010in}}{\pgfqpoint{7.442175in}{1.718774in}}%
\pgfpathcurveto{\pgfqpoint{7.442175in}{1.710538in}}{\pgfqpoint{7.445447in}{1.702638in}}{\pgfqpoint{7.451271in}{1.696814in}}%
\pgfpathcurveto{\pgfqpoint{7.457095in}{1.690990in}}{\pgfqpoint{7.464995in}{1.687718in}}{\pgfqpoint{7.473231in}{1.687718in}}%
\pgfpathclose%
\pgfusepath{stroke,fill}%
\end{pgfscope}%
\begin{pgfscope}%
\pgfpathrectangle{\pgfqpoint{0.894063in}{0.630000in}}{\pgfqpoint{6.713438in}{2.060556in}} %
\pgfusepath{clip}%
\pgfsetbuttcap%
\pgfsetroundjoin%
\definecolor{currentfill}{rgb}{0.000000,0.500000,0.000000}%
\pgfsetfillcolor{currentfill}%
\pgfsetlinewidth{1.003750pt}%
\definecolor{currentstroke}{rgb}{0.000000,0.500000,0.000000}%
\pgfsetstrokecolor{currentstroke}%
\pgfsetdash{}{0pt}%
\pgfpathmoveto{\pgfqpoint{5.056394in}{1.338212in}}%
\pgfpathcurveto{\pgfqpoint{5.064630in}{1.338212in}}{\pgfqpoint{5.072530in}{1.341484in}}{\pgfqpoint{5.078354in}{1.347308in}}%
\pgfpathcurveto{\pgfqpoint{5.084178in}{1.353132in}}{\pgfqpoint{5.087450in}{1.361032in}}{\pgfqpoint{5.087450in}{1.369268in}}%
\pgfpathcurveto{\pgfqpoint{5.087450in}{1.377505in}}{\pgfqpoint{5.084178in}{1.385405in}}{\pgfqpoint{5.078354in}{1.391229in}}%
\pgfpathcurveto{\pgfqpoint{5.072530in}{1.397053in}}{\pgfqpoint{5.064630in}{1.400325in}}{\pgfqpoint{5.056394in}{1.400325in}}%
\pgfpathcurveto{\pgfqpoint{5.048157in}{1.400325in}}{\pgfqpoint{5.040257in}{1.397053in}}{\pgfqpoint{5.034433in}{1.391229in}}%
\pgfpathcurveto{\pgfqpoint{5.028610in}{1.385405in}}{\pgfqpoint{5.025337in}{1.377505in}}{\pgfqpoint{5.025337in}{1.369268in}}%
\pgfpathcurveto{\pgfqpoint{5.025337in}{1.361032in}}{\pgfqpoint{5.028610in}{1.353132in}}{\pgfqpoint{5.034433in}{1.347308in}}%
\pgfpathcurveto{\pgfqpoint{5.040257in}{1.341484in}}{\pgfqpoint{5.048157in}{1.338212in}}{\pgfqpoint{5.056394in}{1.338212in}}%
\pgfpathclose%
\pgfusepath{stroke,fill}%
\end{pgfscope}%
\begin{pgfscope}%
\pgfpathrectangle{\pgfqpoint{0.894063in}{0.630000in}}{\pgfqpoint{6.713438in}{2.060556in}} %
\pgfusepath{clip}%
\pgfsetbuttcap%
\pgfsetroundjoin%
\definecolor{currentfill}{rgb}{0.000000,0.500000,0.000000}%
\pgfsetfillcolor{currentfill}%
\pgfsetlinewidth{1.003750pt}%
\definecolor{currentstroke}{rgb}{0.000000,0.500000,0.000000}%
\pgfsetstrokecolor{currentstroke}%
\pgfsetdash{}{0pt}%
\pgfpathmoveto{\pgfqpoint{2.908094in}{0.963962in}}%
\pgfpathcurveto{\pgfqpoint{2.916330in}{0.963962in}}{\pgfqpoint{2.924230in}{0.967234in}}{\pgfqpoint{2.930054in}{0.973058in}}%
\pgfpathcurveto{\pgfqpoint{2.935878in}{0.978882in}}{\pgfqpoint{2.939150in}{0.986782in}}{\pgfqpoint{2.939150in}{0.995019in}}%
\pgfpathcurveto{\pgfqpoint{2.939150in}{1.003255in}}{\pgfqpoint{2.935878in}{1.011155in}}{\pgfqpoint{2.930054in}{1.016979in}}%
\pgfpathcurveto{\pgfqpoint{2.924230in}{1.022803in}}{\pgfqpoint{2.916330in}{1.026075in}}{\pgfqpoint{2.908094in}{1.026075in}}%
\pgfpathcurveto{\pgfqpoint{2.899857in}{1.026075in}}{\pgfqpoint{2.891957in}{1.022803in}}{\pgfqpoint{2.886133in}{1.016979in}}%
\pgfpathcurveto{\pgfqpoint{2.880310in}{1.011155in}}{\pgfqpoint{2.877037in}{1.003255in}}{\pgfqpoint{2.877037in}{0.995019in}}%
\pgfpathcurveto{\pgfqpoint{2.877037in}{0.986782in}}{\pgfqpoint{2.880310in}{0.978882in}}{\pgfqpoint{2.886133in}{0.973058in}}%
\pgfpathcurveto{\pgfqpoint{2.891957in}{0.967234in}}{\pgfqpoint{2.899857in}{0.963962in}}{\pgfqpoint{2.908094in}{0.963962in}}%
\pgfpathclose%
\pgfusepath{stroke,fill}%
\end{pgfscope}%
\begin{pgfscope}%
\pgfpathrectangle{\pgfqpoint{0.894063in}{0.630000in}}{\pgfqpoint{6.713438in}{2.060556in}} %
\pgfusepath{clip}%
\pgfsetbuttcap%
\pgfsetroundjoin%
\definecolor{currentfill}{rgb}{0.000000,0.500000,0.000000}%
\pgfsetfillcolor{currentfill}%
\pgfsetlinewidth{1.003750pt}%
\definecolor{currentstroke}{rgb}{0.000000,0.500000,0.000000}%
\pgfsetstrokecolor{currentstroke}%
\pgfsetdash{}{0pt}%
\pgfpathmoveto{\pgfqpoint{3.445169in}{1.049057in}}%
\pgfpathcurveto{\pgfqpoint{3.453405in}{1.049057in}}{\pgfqpoint{3.461305in}{1.052329in}}{\pgfqpoint{3.467129in}{1.058153in}}%
\pgfpathcurveto{\pgfqpoint{3.472953in}{1.063977in}}{\pgfqpoint{3.476225in}{1.071877in}}{\pgfqpoint{3.476225in}{1.080114in}}%
\pgfpathcurveto{\pgfqpoint{3.476225in}{1.088350in}}{\pgfqpoint{3.472953in}{1.096250in}}{\pgfqpoint{3.467129in}{1.102074in}}%
\pgfpathcurveto{\pgfqpoint{3.461305in}{1.107898in}}{\pgfqpoint{3.453405in}{1.111170in}}{\pgfqpoint{3.445169in}{1.111170in}}%
\pgfpathcurveto{\pgfqpoint{3.436932in}{1.111170in}}{\pgfqpoint{3.429032in}{1.107898in}}{\pgfqpoint{3.423208in}{1.102074in}}%
\pgfpathcurveto{\pgfqpoint{3.417385in}{1.096250in}}{\pgfqpoint{3.414112in}{1.088350in}}{\pgfqpoint{3.414112in}{1.080114in}}%
\pgfpathcurveto{\pgfqpoint{3.414112in}{1.071877in}}{\pgfqpoint{3.417385in}{1.063977in}}{\pgfqpoint{3.423208in}{1.058153in}}%
\pgfpathcurveto{\pgfqpoint{3.429032in}{1.052329in}}{\pgfqpoint{3.436932in}{1.049057in}}{\pgfqpoint{3.445169in}{1.049057in}}%
\pgfpathclose%
\pgfusepath{stroke,fill}%
\end{pgfscope}%
\begin{pgfscope}%
\pgfpathrectangle{\pgfqpoint{0.894063in}{0.630000in}}{\pgfqpoint{6.713438in}{2.060556in}} %
\pgfusepath{clip}%
\pgfsetbuttcap%
\pgfsetroundjoin%
\definecolor{currentfill}{rgb}{0.000000,0.500000,0.000000}%
\pgfsetfillcolor{currentfill}%
\pgfsetlinewidth{1.003750pt}%
\definecolor{currentstroke}{rgb}{0.000000,0.500000,0.000000}%
\pgfsetstrokecolor{currentstroke}%
\pgfsetdash{}{0pt}%
\pgfpathmoveto{\pgfqpoint{4.116513in}{1.164460in}}%
\pgfpathcurveto{\pgfqpoint{4.124749in}{1.164460in}}{\pgfqpoint{4.132649in}{1.167732in}}{\pgfqpoint{4.138473in}{1.173556in}}%
\pgfpathcurveto{\pgfqpoint{4.144297in}{1.179380in}}{\pgfqpoint{4.147569in}{1.187280in}}{\pgfqpoint{4.147569in}{1.195517in}}%
\pgfpathcurveto{\pgfqpoint{4.147569in}{1.203753in}}{\pgfqpoint{4.144297in}{1.211653in}}{\pgfqpoint{4.138473in}{1.217477in}}%
\pgfpathcurveto{\pgfqpoint{4.132649in}{1.223301in}}{\pgfqpoint{4.124749in}{1.226573in}}{\pgfqpoint{4.116513in}{1.226573in}}%
\pgfpathcurveto{\pgfqpoint{4.108276in}{1.226573in}}{\pgfqpoint{4.100376in}{1.223301in}}{\pgfqpoint{4.094552in}{1.217477in}}%
\pgfpathcurveto{\pgfqpoint{4.088728in}{1.211653in}}{\pgfqpoint{4.085456in}{1.203753in}}{\pgfqpoint{4.085456in}{1.195517in}}%
\pgfpathcurveto{\pgfqpoint{4.085456in}{1.187280in}}{\pgfqpoint{4.088728in}{1.179380in}}{\pgfqpoint{4.094552in}{1.173556in}}%
\pgfpathcurveto{\pgfqpoint{4.100376in}{1.167732in}}{\pgfqpoint{4.108276in}{1.164460in}}{\pgfqpoint{4.116513in}{1.164460in}}%
\pgfpathclose%
\pgfusepath{stroke,fill}%
\end{pgfscope}%
\begin{pgfscope}%
\pgfpathrectangle{\pgfqpoint{0.894063in}{0.630000in}}{\pgfqpoint{6.713438in}{2.060556in}} %
\pgfusepath{clip}%
\pgfsetbuttcap%
\pgfsetroundjoin%
\definecolor{currentfill}{rgb}{0.000000,0.500000,0.000000}%
\pgfsetfillcolor{currentfill}%
\pgfsetlinewidth{1.003750pt}%
\definecolor{currentstroke}{rgb}{0.000000,0.500000,0.000000}%
\pgfsetstrokecolor{currentstroke}%
\pgfsetdash{}{0pt}%
\pgfpathmoveto{\pgfqpoint{1.431138in}{0.752013in}}%
\pgfpathcurveto{\pgfqpoint{1.439374in}{0.752013in}}{\pgfqpoint{1.447274in}{0.755286in}}{\pgfqpoint{1.453098in}{0.761110in}}%
\pgfpathcurveto{\pgfqpoint{1.458922in}{0.766934in}}{\pgfqpoint{1.462194in}{0.774834in}}{\pgfqpoint{1.462194in}{0.783070in}}%
\pgfpathcurveto{\pgfqpoint{1.462194in}{0.791306in}}{\pgfqpoint{1.458922in}{0.799206in}}{\pgfqpoint{1.453098in}{0.805030in}}%
\pgfpathcurveto{\pgfqpoint{1.447274in}{0.810854in}}{\pgfqpoint{1.439374in}{0.814126in}}{\pgfqpoint{1.431138in}{0.814126in}}%
\pgfpathcurveto{\pgfqpoint{1.422901in}{0.814126in}}{\pgfqpoint{1.415001in}{0.810854in}}{\pgfqpoint{1.409177in}{0.805030in}}%
\pgfpathcurveto{\pgfqpoint{1.403353in}{0.799206in}}{\pgfqpoint{1.400081in}{0.791306in}}{\pgfqpoint{1.400081in}{0.783070in}}%
\pgfpathcurveto{\pgfqpoint{1.400081in}{0.774834in}}{\pgfqpoint{1.403353in}{0.766934in}}{\pgfqpoint{1.409177in}{0.761110in}}%
\pgfpathcurveto{\pgfqpoint{1.415001in}{0.755286in}}{\pgfqpoint{1.422901in}{0.752013in}}{\pgfqpoint{1.431138in}{0.752013in}}%
\pgfpathclose%
\pgfusepath{stroke,fill}%
\end{pgfscope}%
\begin{pgfscope}%
\pgfpathrectangle{\pgfqpoint{0.894063in}{0.630000in}}{\pgfqpoint{6.713438in}{2.060556in}} %
\pgfusepath{clip}%
\pgfsetbuttcap%
\pgfsetroundjoin%
\definecolor{currentfill}{rgb}{0.000000,0.500000,0.000000}%
\pgfsetfillcolor{currentfill}%
\pgfsetlinewidth{1.003750pt}%
\definecolor{currentstroke}{rgb}{0.000000,0.500000,0.000000}%
\pgfsetstrokecolor{currentstroke}%
\pgfsetdash{}{0pt}%
\pgfpathmoveto{\pgfqpoint{2.773825in}{0.941838in}}%
\pgfpathcurveto{\pgfqpoint{2.782061in}{0.941838in}}{\pgfqpoint{2.789961in}{0.945110in}}{\pgfqpoint{2.795785in}{0.950934in}}%
\pgfpathcurveto{\pgfqpoint{2.801609in}{0.956758in}}{\pgfqpoint{2.804881in}{0.964658in}}{\pgfqpoint{2.804881in}{0.972894in}}%
\pgfpathcurveto{\pgfqpoint{2.804881in}{0.981130in}}{\pgfqpoint{2.801609in}{0.989030in}}{\pgfqpoint{2.795785in}{0.994854in}}%
\pgfpathcurveto{\pgfqpoint{2.789961in}{1.000678in}}{\pgfqpoint{2.782061in}{1.003951in}}{\pgfqpoint{2.773825in}{1.003951in}}%
\pgfpathcurveto{\pgfqpoint{2.765589in}{1.003951in}}{\pgfqpoint{2.757689in}{1.000678in}}{\pgfqpoint{2.751865in}{0.994854in}}%
\pgfpathcurveto{\pgfqpoint{2.746041in}{0.989030in}}{\pgfqpoint{2.742769in}{0.981130in}}{\pgfqpoint{2.742769in}{0.972894in}}%
\pgfpathcurveto{\pgfqpoint{2.742769in}{0.964658in}}{\pgfqpoint{2.746041in}{0.956758in}}{\pgfqpoint{2.751865in}{0.950934in}}%
\pgfpathcurveto{\pgfqpoint{2.757689in}{0.945110in}}{\pgfqpoint{2.765589in}{0.941838in}}{\pgfqpoint{2.773825in}{0.941838in}}%
\pgfpathclose%
\pgfusepath{stroke,fill}%
\end{pgfscope}%
\begin{pgfscope}%
\pgfpathrectangle{\pgfqpoint{0.894063in}{0.630000in}}{\pgfqpoint{6.713438in}{2.060556in}} %
\pgfusepath{clip}%
\pgfsetbuttcap%
\pgfsetroundjoin%
\definecolor{currentfill}{rgb}{0.000000,0.500000,0.000000}%
\pgfsetfillcolor{currentfill}%
\pgfsetlinewidth{1.003750pt}%
\definecolor{currentstroke}{rgb}{0.000000,0.500000,0.000000}%
\pgfsetstrokecolor{currentstroke}%
\pgfsetdash{}{0pt}%
\pgfpathmoveto{\pgfqpoint{1.565406in}{0.772054in}}%
\pgfpathcurveto{\pgfqpoint{1.573643in}{0.772054in}}{\pgfqpoint{1.581543in}{0.775326in}}{\pgfqpoint{1.587367in}{0.781150in}}%
\pgfpathcurveto{\pgfqpoint{1.593190in}{0.786974in}}{\pgfqpoint{1.596463in}{0.794874in}}{\pgfqpoint{1.596463in}{0.803110in}}%
\pgfpathcurveto{\pgfqpoint{1.596463in}{0.811346in}}{\pgfqpoint{1.593190in}{0.819247in}}{\pgfqpoint{1.587367in}{0.825070in}}%
\pgfpathcurveto{\pgfqpoint{1.581543in}{0.830894in}}{\pgfqpoint{1.573643in}{0.834167in}}{\pgfqpoint{1.565406in}{0.834167in}}%
\pgfpathcurveto{\pgfqpoint{1.557170in}{0.834167in}}{\pgfqpoint{1.549270in}{0.830894in}}{\pgfqpoint{1.543446in}{0.825070in}}%
\pgfpathcurveto{\pgfqpoint{1.537622in}{0.819247in}}{\pgfqpoint{1.534350in}{0.811346in}}{\pgfqpoint{1.534350in}{0.803110in}}%
\pgfpathcurveto{\pgfqpoint{1.534350in}{0.794874in}}{\pgfqpoint{1.537622in}{0.786974in}}{\pgfqpoint{1.543446in}{0.781150in}}%
\pgfpathcurveto{\pgfqpoint{1.549270in}{0.775326in}}{\pgfqpoint{1.557170in}{0.772054in}}{\pgfqpoint{1.565406in}{0.772054in}}%
\pgfpathclose%
\pgfusepath{stroke,fill}%
\end{pgfscope}%
\begin{pgfscope}%
\pgfpathrectangle{\pgfqpoint{0.894063in}{0.630000in}}{\pgfqpoint{6.713438in}{2.060556in}} %
\pgfusepath{clip}%
\pgfsetbuttcap%
\pgfsetroundjoin%
\definecolor{currentfill}{rgb}{0.000000,0.500000,0.000000}%
\pgfsetfillcolor{currentfill}%
\pgfsetlinewidth{1.003750pt}%
\definecolor{currentstroke}{rgb}{0.000000,0.500000,0.000000}%
\pgfsetstrokecolor{currentstroke}%
\pgfsetdash{}{0pt}%
\pgfpathmoveto{\pgfqpoint{4.250781in}{1.197588in}}%
\pgfpathcurveto{\pgfqpoint{4.259018in}{1.197588in}}{\pgfqpoint{4.266918in}{1.200860in}}{\pgfqpoint{4.272742in}{1.206684in}}%
\pgfpathcurveto{\pgfqpoint{4.278565in}{1.212508in}}{\pgfqpoint{4.281838in}{1.220408in}}{\pgfqpoint{4.281838in}{1.228644in}}%
\pgfpathcurveto{\pgfqpoint{4.281838in}{1.236881in}}{\pgfqpoint{4.278565in}{1.244781in}}{\pgfqpoint{4.272742in}{1.250605in}}%
\pgfpathcurveto{\pgfqpoint{4.266918in}{1.256429in}}{\pgfqpoint{4.259018in}{1.259701in}}{\pgfqpoint{4.250781in}{1.259701in}}%
\pgfpathcurveto{\pgfqpoint{4.242545in}{1.259701in}}{\pgfqpoint{4.234645in}{1.256429in}}{\pgfqpoint{4.228821in}{1.250605in}}%
\pgfpathcurveto{\pgfqpoint{4.222997in}{1.244781in}}{\pgfqpoint{4.219725in}{1.236881in}}{\pgfqpoint{4.219725in}{1.228644in}}%
\pgfpathcurveto{\pgfqpoint{4.219725in}{1.220408in}}{\pgfqpoint{4.222997in}{1.212508in}}{\pgfqpoint{4.228821in}{1.206684in}}%
\pgfpathcurveto{\pgfqpoint{4.234645in}{1.200860in}}{\pgfqpoint{4.242545in}{1.197588in}}{\pgfqpoint{4.250781in}{1.197588in}}%
\pgfpathclose%
\pgfusepath{stroke,fill}%
\end{pgfscope}%
\begin{pgfscope}%
\pgfpathrectangle{\pgfqpoint{0.894063in}{0.630000in}}{\pgfqpoint{6.713438in}{2.060556in}} %
\pgfusepath{clip}%
\pgfsetbuttcap%
\pgfsetroundjoin%
\definecolor{currentfill}{rgb}{0.000000,0.500000,0.000000}%
\pgfsetfillcolor{currentfill}%
\pgfsetlinewidth{1.003750pt}%
\definecolor{currentstroke}{rgb}{0.000000,0.500000,0.000000}%
\pgfsetstrokecolor{currentstroke}%
\pgfsetdash{}{0pt}%
\pgfpathmoveto{\pgfqpoint{3.847975in}{1.131462in}}%
\pgfpathcurveto{\pgfqpoint{3.856211in}{1.131462in}}{\pgfqpoint{3.864111in}{1.134734in}}{\pgfqpoint{3.869935in}{1.140558in}}%
\pgfpathcurveto{\pgfqpoint{3.875759in}{1.146382in}}{\pgfqpoint{3.879031in}{1.154282in}}{\pgfqpoint{3.879031in}{1.162518in}}%
\pgfpathcurveto{\pgfqpoint{3.879031in}{1.170754in}}{\pgfqpoint{3.875759in}{1.178655in}}{\pgfqpoint{3.869935in}{1.184478in}}%
\pgfpathcurveto{\pgfqpoint{3.864111in}{1.190302in}}{\pgfqpoint{3.856211in}{1.193575in}}{\pgfqpoint{3.847975in}{1.193575in}}%
\pgfpathcurveto{\pgfqpoint{3.839739in}{1.193575in}}{\pgfqpoint{3.831839in}{1.190302in}}{\pgfqpoint{3.826015in}{1.184478in}}%
\pgfpathcurveto{\pgfqpoint{3.820191in}{1.178655in}}{\pgfqpoint{3.816919in}{1.170754in}}{\pgfqpoint{3.816919in}{1.162518in}}%
\pgfpathcurveto{\pgfqpoint{3.816919in}{1.154282in}}{\pgfqpoint{3.820191in}{1.146382in}}{\pgfqpoint{3.826015in}{1.140558in}}%
\pgfpathcurveto{\pgfqpoint{3.831839in}{1.134734in}}{\pgfqpoint{3.839739in}{1.131462in}}{\pgfqpoint{3.847975in}{1.131462in}}%
\pgfpathclose%
\pgfusepath{stroke,fill}%
\end{pgfscope}%
\begin{pgfscope}%
\pgfpathrectangle{\pgfqpoint{0.894063in}{0.630000in}}{\pgfqpoint{6.713438in}{2.060556in}} %
\pgfusepath{clip}%
\pgfsetbuttcap%
\pgfsetroundjoin%
\definecolor{currentfill}{rgb}{0.000000,0.500000,0.000000}%
\pgfsetfillcolor{currentfill}%
\pgfsetlinewidth{1.003750pt}%
\definecolor{currentstroke}{rgb}{0.000000,0.500000,0.000000}%
\pgfsetstrokecolor{currentstroke}%
\pgfsetdash{}{0pt}%
\pgfpathmoveto{\pgfqpoint{7.607500in}{1.704090in}}%
\pgfpathcurveto{\pgfqpoint{7.615736in}{1.704090in}}{\pgfqpoint{7.623636in}{1.707362in}}{\pgfqpoint{7.629460in}{1.713186in}}%
\pgfpathcurveto{\pgfqpoint{7.635284in}{1.719010in}}{\pgfqpoint{7.638556in}{1.726910in}}{\pgfqpoint{7.638556in}{1.735147in}}%
\pgfpathcurveto{\pgfqpoint{7.638556in}{1.743383in}}{\pgfqpoint{7.635284in}{1.751283in}}{\pgfqpoint{7.629460in}{1.757107in}}%
\pgfpathcurveto{\pgfqpoint{7.623636in}{1.762931in}}{\pgfqpoint{7.615736in}{1.766203in}}{\pgfqpoint{7.607500in}{1.766203in}}%
\pgfpathcurveto{\pgfqpoint{7.599264in}{1.766203in}}{\pgfqpoint{7.591364in}{1.762931in}}{\pgfqpoint{7.585540in}{1.757107in}}%
\pgfpathcurveto{\pgfqpoint{7.579716in}{1.751283in}}{\pgfqpoint{7.576444in}{1.743383in}}{\pgfqpoint{7.576444in}{1.735147in}}%
\pgfpathcurveto{\pgfqpoint{7.576444in}{1.726910in}}{\pgfqpoint{7.579716in}{1.719010in}}{\pgfqpoint{7.585540in}{1.713186in}}%
\pgfpathcurveto{\pgfqpoint{7.591364in}{1.707362in}}{\pgfqpoint{7.599264in}{1.704090in}}{\pgfqpoint{7.607500in}{1.704090in}}%
\pgfpathclose%
\pgfusepath{stroke,fill}%
\end{pgfscope}%
\begin{pgfscope}%
\pgfpathrectangle{\pgfqpoint{0.894063in}{0.630000in}}{\pgfqpoint{6.713438in}{2.060556in}} %
\pgfusepath{clip}%
\pgfsetbuttcap%
\pgfsetroundjoin%
\definecolor{currentfill}{rgb}{0.000000,0.500000,0.000000}%
\pgfsetfillcolor{currentfill}%
\pgfsetlinewidth{1.003750pt}%
\definecolor{currentstroke}{rgb}{0.000000,0.500000,0.000000}%
\pgfsetstrokecolor{currentstroke}%
\pgfsetdash{}{0pt}%
\pgfpathmoveto{\pgfqpoint{4.385050in}{1.216922in}}%
\pgfpathcurveto{\pgfqpoint{4.393286in}{1.216922in}}{\pgfqpoint{4.401186in}{1.220194in}}{\pgfqpoint{4.407010in}{1.226018in}}%
\pgfpathcurveto{\pgfqpoint{4.412834in}{1.231842in}}{\pgfqpoint{4.416106in}{1.239742in}}{\pgfqpoint{4.416106in}{1.247978in}}%
\pgfpathcurveto{\pgfqpoint{4.416106in}{1.256215in}}{\pgfqpoint{4.412834in}{1.264115in}}{\pgfqpoint{4.407010in}{1.269939in}}%
\pgfpathcurveto{\pgfqpoint{4.401186in}{1.275762in}}{\pgfqpoint{4.393286in}{1.279035in}}{\pgfqpoint{4.385050in}{1.279035in}}%
\pgfpathcurveto{\pgfqpoint{4.376814in}{1.279035in}}{\pgfqpoint{4.368914in}{1.275762in}}{\pgfqpoint{4.363090in}{1.269939in}}%
\pgfpathcurveto{\pgfqpoint{4.357266in}{1.264115in}}{\pgfqpoint{4.353994in}{1.256215in}}{\pgfqpoint{4.353994in}{1.247978in}}%
\pgfpathcurveto{\pgfqpoint{4.353994in}{1.239742in}}{\pgfqpoint{4.357266in}{1.231842in}}{\pgfqpoint{4.363090in}{1.226018in}}%
\pgfpathcurveto{\pgfqpoint{4.368914in}{1.220194in}}{\pgfqpoint{4.376814in}{1.216922in}}{\pgfqpoint{4.385050in}{1.216922in}}%
\pgfpathclose%
\pgfusepath{stroke,fill}%
\end{pgfscope}%
\begin{pgfscope}%
\pgfpathrectangle{\pgfqpoint{0.894063in}{0.630000in}}{\pgfqpoint{6.713438in}{2.060556in}} %
\pgfusepath{clip}%
\pgfsetbuttcap%
\pgfsetroundjoin%
\definecolor{currentfill}{rgb}{0.000000,0.500000,0.000000}%
\pgfsetfillcolor{currentfill}%
\pgfsetlinewidth{1.003750pt}%
\definecolor{currentstroke}{rgb}{0.000000,0.500000,0.000000}%
\pgfsetstrokecolor{currentstroke}%
\pgfsetdash{}{0pt}%
\pgfpathmoveto{\pgfqpoint{6.533350in}{1.542649in}}%
\pgfpathcurveto{\pgfqpoint{6.541586in}{1.542649in}}{\pgfqpoint{6.549486in}{1.545921in}}{\pgfqpoint{6.555310in}{1.551745in}}%
\pgfpathcurveto{\pgfqpoint{6.561134in}{1.557569in}}{\pgfqpoint{6.564406in}{1.565469in}}{\pgfqpoint{6.564406in}{1.573705in}}%
\pgfpathcurveto{\pgfqpoint{6.564406in}{1.581941in}}{\pgfqpoint{6.561134in}{1.589841in}}{\pgfqpoint{6.555310in}{1.595665in}}%
\pgfpathcurveto{\pgfqpoint{6.549486in}{1.601489in}}{\pgfqpoint{6.541586in}{1.604762in}}{\pgfqpoint{6.533350in}{1.604762in}}%
\pgfpathcurveto{\pgfqpoint{6.525114in}{1.604762in}}{\pgfqpoint{6.517214in}{1.601489in}}{\pgfqpoint{6.511390in}{1.595665in}}%
\pgfpathcurveto{\pgfqpoint{6.505566in}{1.589841in}}{\pgfqpoint{6.502294in}{1.581941in}}{\pgfqpoint{6.502294in}{1.573705in}}%
\pgfpathcurveto{\pgfqpoint{6.502294in}{1.565469in}}{\pgfqpoint{6.505566in}{1.557569in}}{\pgfqpoint{6.511390in}{1.551745in}}%
\pgfpathcurveto{\pgfqpoint{6.517214in}{1.545921in}}{\pgfqpoint{6.525114in}{1.542649in}}{\pgfqpoint{6.533350in}{1.542649in}}%
\pgfpathclose%
\pgfusepath{stroke,fill}%
\end{pgfscope}%
\begin{pgfscope}%
\pgfpathrectangle{\pgfqpoint{0.894063in}{0.630000in}}{\pgfqpoint{6.713438in}{2.060556in}} %
\pgfusepath{clip}%
\pgfsetbuttcap%
\pgfsetroundjoin%
\definecolor{currentfill}{rgb}{0.000000,0.500000,0.000000}%
\pgfsetfillcolor{currentfill}%
\pgfsetlinewidth{1.003750pt}%
\definecolor{currentstroke}{rgb}{0.000000,0.500000,0.000000}%
\pgfsetstrokecolor{currentstroke}%
\pgfsetdash{}{0pt}%
\pgfpathmoveto{\pgfqpoint{1.296869in}{0.730313in}}%
\pgfpathcurveto{\pgfqpoint{1.305105in}{0.730313in}}{\pgfqpoint{1.313005in}{0.733585in}}{\pgfqpoint{1.318829in}{0.739409in}}%
\pgfpathcurveto{\pgfqpoint{1.324653in}{0.745233in}}{\pgfqpoint{1.327925in}{0.753133in}}{\pgfqpoint{1.327925in}{0.761369in}}%
\pgfpathcurveto{\pgfqpoint{1.327925in}{0.769606in}}{\pgfqpoint{1.324653in}{0.777506in}}{\pgfqpoint{1.318829in}{0.783330in}}%
\pgfpathcurveto{\pgfqpoint{1.313005in}{0.789153in}}{\pgfqpoint{1.305105in}{0.792426in}}{\pgfqpoint{1.296869in}{0.792426in}}%
\pgfpathcurveto{\pgfqpoint{1.288632in}{0.792426in}}{\pgfqpoint{1.280732in}{0.789153in}}{\pgfqpoint{1.274908in}{0.783330in}}%
\pgfpathcurveto{\pgfqpoint{1.269085in}{0.777506in}}{\pgfqpoint{1.265812in}{0.769606in}}{\pgfqpoint{1.265812in}{0.761369in}}%
\pgfpathcurveto{\pgfqpoint{1.265812in}{0.753133in}}{\pgfqpoint{1.269085in}{0.745233in}}{\pgfqpoint{1.274908in}{0.739409in}}%
\pgfpathcurveto{\pgfqpoint{1.280732in}{0.733585in}}{\pgfqpoint{1.288632in}{0.730313in}}{\pgfqpoint{1.296869in}{0.730313in}}%
\pgfpathclose%
\pgfusepath{stroke,fill}%
\end{pgfscope}%
\begin{pgfscope}%
\pgfpathrectangle{\pgfqpoint{0.894063in}{0.630000in}}{\pgfqpoint{6.713438in}{2.060556in}} %
\pgfusepath{clip}%
\pgfsetbuttcap%
\pgfsetroundjoin%
\definecolor{currentfill}{rgb}{0.000000,0.500000,0.000000}%
\pgfsetfillcolor{currentfill}%
\pgfsetlinewidth{1.003750pt}%
\definecolor{currentstroke}{rgb}{0.000000,0.500000,0.000000}%
\pgfsetstrokecolor{currentstroke}%
\pgfsetdash{}{0pt}%
\pgfpathmoveto{\pgfqpoint{4.519319in}{1.242643in}}%
\pgfpathcurveto{\pgfqpoint{4.527555in}{1.242643in}}{\pgfqpoint{4.535455in}{1.245916in}}{\pgfqpoint{4.541279in}{1.251740in}}%
\pgfpathcurveto{\pgfqpoint{4.547103in}{1.257564in}}{\pgfqpoint{4.550375in}{1.265464in}}{\pgfqpoint{4.550375in}{1.273700in}}%
\pgfpathcurveto{\pgfqpoint{4.550375in}{1.281936in}}{\pgfqpoint{4.547103in}{1.289836in}}{\pgfqpoint{4.541279in}{1.295660in}}%
\pgfpathcurveto{\pgfqpoint{4.535455in}{1.301484in}}{\pgfqpoint{4.527555in}{1.304756in}}{\pgfqpoint{4.519319in}{1.304756in}}%
\pgfpathcurveto{\pgfqpoint{4.511082in}{1.304756in}}{\pgfqpoint{4.503182in}{1.301484in}}{\pgfqpoint{4.497358in}{1.295660in}}%
\pgfpathcurveto{\pgfqpoint{4.491535in}{1.289836in}}{\pgfqpoint{4.488262in}{1.281936in}}{\pgfqpoint{4.488262in}{1.273700in}}%
\pgfpathcurveto{\pgfqpoint{4.488262in}{1.265464in}}{\pgfqpoint{4.491535in}{1.257564in}}{\pgfqpoint{4.497358in}{1.251740in}}%
\pgfpathcurveto{\pgfqpoint{4.503182in}{1.245916in}}{\pgfqpoint{4.511082in}{1.242643in}}{\pgfqpoint{4.519319in}{1.242643in}}%
\pgfpathclose%
\pgfusepath{stroke,fill}%
\end{pgfscope}%
\begin{pgfscope}%
\pgfpathrectangle{\pgfqpoint{0.894063in}{0.630000in}}{\pgfqpoint{6.713438in}{2.060556in}} %
\pgfusepath{clip}%
\pgfsetbuttcap%
\pgfsetroundjoin%
\definecolor{currentfill}{rgb}{0.000000,0.500000,0.000000}%
\pgfsetfillcolor{currentfill}%
\pgfsetlinewidth{1.003750pt}%
\definecolor{currentstroke}{rgb}{0.000000,0.500000,0.000000}%
\pgfsetstrokecolor{currentstroke}%
\pgfsetdash{}{0pt}%
\pgfpathmoveto{\pgfqpoint{2.505288in}{0.900032in}}%
\pgfpathcurveto{\pgfqpoint{2.513524in}{0.900032in}}{\pgfqpoint{2.521424in}{0.903304in}}{\pgfqpoint{2.527248in}{0.909128in}}%
\pgfpathcurveto{\pgfqpoint{2.533072in}{0.914952in}}{\pgfqpoint{2.536344in}{0.922852in}}{\pgfqpoint{2.536344in}{0.931088in}}%
\pgfpathcurveto{\pgfqpoint{2.536344in}{0.939325in}}{\pgfqpoint{2.533072in}{0.947225in}}{\pgfqpoint{2.527248in}{0.953049in}}%
\pgfpathcurveto{\pgfqpoint{2.521424in}{0.958873in}}{\pgfqpoint{2.513524in}{0.962145in}}{\pgfqpoint{2.505288in}{0.962145in}}%
\pgfpathcurveto{\pgfqpoint{2.497051in}{0.962145in}}{\pgfqpoint{2.489151in}{0.958873in}}{\pgfqpoint{2.483327in}{0.953049in}}%
\pgfpathcurveto{\pgfqpoint{2.477503in}{0.947225in}}{\pgfqpoint{2.474231in}{0.939325in}}{\pgfqpoint{2.474231in}{0.931088in}}%
\pgfpathcurveto{\pgfqpoint{2.474231in}{0.922852in}}{\pgfqpoint{2.477503in}{0.914952in}}{\pgfqpoint{2.483327in}{0.909128in}}%
\pgfpathcurveto{\pgfqpoint{2.489151in}{0.903304in}}{\pgfqpoint{2.497051in}{0.900032in}}{\pgfqpoint{2.505288in}{0.900032in}}%
\pgfpathclose%
\pgfusepath{stroke,fill}%
\end{pgfscope}%
\begin{pgfscope}%
\pgfpathrectangle{\pgfqpoint{0.894063in}{0.630000in}}{\pgfqpoint{6.713438in}{2.060556in}} %
\pgfusepath{clip}%
\pgfsetbuttcap%
\pgfsetroundjoin%
\definecolor{currentfill}{rgb}{0.000000,0.500000,0.000000}%
\pgfsetfillcolor{currentfill}%
\pgfsetlinewidth{1.003750pt}%
\definecolor{currentstroke}{rgb}{0.000000,0.500000,0.000000}%
\pgfsetstrokecolor{currentstroke}%
\pgfsetdash{}{0pt}%
\pgfpathmoveto{\pgfqpoint{5.459200in}{1.384262in}}%
\pgfpathcurveto{\pgfqpoint{5.467436in}{1.384262in}}{\pgfqpoint{5.475336in}{1.387535in}}{\pgfqpoint{5.481160in}{1.393359in}}%
\pgfpathcurveto{\pgfqpoint{5.486984in}{1.399183in}}{\pgfqpoint{5.490256in}{1.407083in}}{\pgfqpoint{5.490256in}{1.415319in}}%
\pgfpathcurveto{\pgfqpoint{5.490256in}{1.423555in}}{\pgfqpoint{5.486984in}{1.431455in}}{\pgfqpoint{5.481160in}{1.437279in}}%
\pgfpathcurveto{\pgfqpoint{5.475336in}{1.443103in}}{\pgfqpoint{5.467436in}{1.446375in}}{\pgfqpoint{5.459200in}{1.446375in}}%
\pgfpathcurveto{\pgfqpoint{5.450964in}{1.446375in}}{\pgfqpoint{5.443064in}{1.443103in}}{\pgfqpoint{5.437240in}{1.437279in}}%
\pgfpathcurveto{\pgfqpoint{5.431416in}{1.431455in}}{\pgfqpoint{5.428144in}{1.423555in}}{\pgfqpoint{5.428144in}{1.415319in}}%
\pgfpathcurveto{\pgfqpoint{5.428144in}{1.407083in}}{\pgfqpoint{5.431416in}{1.399183in}}{\pgfqpoint{5.437240in}{1.393359in}}%
\pgfpathcurveto{\pgfqpoint{5.443064in}{1.387535in}}{\pgfqpoint{5.450964in}{1.384262in}}{\pgfqpoint{5.459200in}{1.384262in}}%
\pgfpathclose%
\pgfusepath{stroke,fill}%
\end{pgfscope}%
\begin{pgfscope}%
\pgfpathrectangle{\pgfqpoint{0.894063in}{0.630000in}}{\pgfqpoint{6.713438in}{2.060556in}} %
\pgfusepath{clip}%
\pgfsetbuttcap%
\pgfsetroundjoin%
\definecolor{currentfill}{rgb}{0.000000,0.500000,0.000000}%
\pgfsetfillcolor{currentfill}%
\pgfsetlinewidth{1.003750pt}%
\definecolor{currentstroke}{rgb}{0.000000,0.500000,0.000000}%
\pgfsetstrokecolor{currentstroke}%
\pgfsetdash{}{0pt}%
\pgfpathmoveto{\pgfqpoint{6.936156in}{1.612154in}}%
\pgfpathcurveto{\pgfqpoint{6.944393in}{1.612154in}}{\pgfqpoint{6.952293in}{1.615426in}}{\pgfqpoint{6.958117in}{1.621250in}}%
\pgfpathcurveto{\pgfqpoint{6.963940in}{1.627074in}}{\pgfqpoint{6.967213in}{1.634974in}}{\pgfqpoint{6.967213in}{1.643210in}}%
\pgfpathcurveto{\pgfqpoint{6.967213in}{1.651447in}}{\pgfqpoint{6.963940in}{1.659347in}}{\pgfqpoint{6.958117in}{1.665171in}}%
\pgfpathcurveto{\pgfqpoint{6.952293in}{1.670995in}}{\pgfqpoint{6.944393in}{1.674267in}}{\pgfqpoint{6.936156in}{1.674267in}}%
\pgfpathcurveto{\pgfqpoint{6.927920in}{1.674267in}}{\pgfqpoint{6.920020in}{1.670995in}}{\pgfqpoint{6.914196in}{1.665171in}}%
\pgfpathcurveto{\pgfqpoint{6.908372in}{1.659347in}}{\pgfqpoint{6.905100in}{1.651447in}}{\pgfqpoint{6.905100in}{1.643210in}}%
\pgfpathcurveto{\pgfqpoint{6.905100in}{1.634974in}}{\pgfqpoint{6.908372in}{1.627074in}}{\pgfqpoint{6.914196in}{1.621250in}}%
\pgfpathcurveto{\pgfqpoint{6.920020in}{1.615426in}}{\pgfqpoint{6.927920in}{1.612154in}}{\pgfqpoint{6.936156in}{1.612154in}}%
\pgfpathclose%
\pgfusepath{stroke,fill}%
\end{pgfscope}%
\begin{pgfscope}%
\pgfpathrectangle{\pgfqpoint{0.894063in}{0.630000in}}{\pgfqpoint{6.713438in}{2.060556in}} %
\pgfusepath{clip}%
\pgfsetbuttcap%
\pgfsetroundjoin%
\definecolor{currentfill}{rgb}{0.000000,0.500000,0.000000}%
\pgfsetfillcolor{currentfill}%
\pgfsetlinewidth{1.003750pt}%
\definecolor{currentstroke}{rgb}{0.000000,0.500000,0.000000}%
\pgfsetstrokecolor{currentstroke}%
\pgfsetdash{}{0pt}%
\pgfpathmoveto{\pgfqpoint{5.862006in}{1.431667in}}%
\pgfpathcurveto{\pgfqpoint{5.870243in}{1.431667in}}{\pgfqpoint{5.878143in}{1.434939in}}{\pgfqpoint{5.883967in}{1.440763in}}%
\pgfpathcurveto{\pgfqpoint{5.889790in}{1.446587in}}{\pgfqpoint{5.893063in}{1.454487in}}{\pgfqpoint{5.893063in}{1.462723in}}%
\pgfpathcurveto{\pgfqpoint{5.893063in}{1.470960in}}{\pgfqpoint{5.889790in}{1.478860in}}{\pgfqpoint{5.883967in}{1.484684in}}%
\pgfpathcurveto{\pgfqpoint{5.878143in}{1.490508in}}{\pgfqpoint{5.870243in}{1.493780in}}{\pgfqpoint{5.862006in}{1.493780in}}%
\pgfpathcurveto{\pgfqpoint{5.853770in}{1.493780in}}{\pgfqpoint{5.845870in}{1.490508in}}{\pgfqpoint{5.840046in}{1.484684in}}%
\pgfpathcurveto{\pgfqpoint{5.834222in}{1.478860in}}{\pgfqpoint{5.830950in}{1.470960in}}{\pgfqpoint{5.830950in}{1.462723in}}%
\pgfpathcurveto{\pgfqpoint{5.830950in}{1.454487in}}{\pgfqpoint{5.834222in}{1.446587in}}{\pgfqpoint{5.840046in}{1.440763in}}%
\pgfpathcurveto{\pgfqpoint{5.845870in}{1.434939in}}{\pgfqpoint{5.853770in}{1.431667in}}{\pgfqpoint{5.862006in}{1.431667in}}%
\pgfpathclose%
\pgfusepath{stroke,fill}%
\end{pgfscope}%
\begin{pgfscope}%
\pgfpathrectangle{\pgfqpoint{0.894063in}{0.630000in}}{\pgfqpoint{6.713438in}{2.060556in}} %
\pgfusepath{clip}%
\pgfsetbuttcap%
\pgfsetroundjoin%
\definecolor{currentfill}{rgb}{0.000000,0.500000,0.000000}%
\pgfsetfillcolor{currentfill}%
\pgfsetlinewidth{1.003750pt}%
\definecolor{currentstroke}{rgb}{0.000000,0.500000,0.000000}%
\pgfsetstrokecolor{currentstroke}%
\pgfsetdash{}{0pt}%
\pgfpathmoveto{\pgfqpoint{7.070425in}{1.630793in}}%
\pgfpathcurveto{\pgfqpoint{7.078661in}{1.630793in}}{\pgfqpoint{7.086561in}{1.634065in}}{\pgfqpoint{7.092385in}{1.639889in}}%
\pgfpathcurveto{\pgfqpoint{7.098209in}{1.645713in}}{\pgfqpoint{7.101481in}{1.653613in}}{\pgfqpoint{7.101481in}{1.661850in}}%
\pgfpathcurveto{\pgfqpoint{7.101481in}{1.670086in}}{\pgfqpoint{7.098209in}{1.677986in}}{\pgfqpoint{7.092385in}{1.683810in}}%
\pgfpathcurveto{\pgfqpoint{7.086561in}{1.689634in}}{\pgfqpoint{7.078661in}{1.692906in}}{\pgfqpoint{7.070425in}{1.692906in}}%
\pgfpathcurveto{\pgfqpoint{7.062189in}{1.692906in}}{\pgfqpoint{7.054289in}{1.689634in}}{\pgfqpoint{7.048465in}{1.683810in}}%
\pgfpathcurveto{\pgfqpoint{7.042641in}{1.677986in}}{\pgfqpoint{7.039369in}{1.670086in}}{\pgfqpoint{7.039369in}{1.661850in}}%
\pgfpathcurveto{\pgfqpoint{7.039369in}{1.653613in}}{\pgfqpoint{7.042641in}{1.645713in}}{\pgfqpoint{7.048465in}{1.639889in}}%
\pgfpathcurveto{\pgfqpoint{7.054289in}{1.634065in}}{\pgfqpoint{7.062189in}{1.630793in}}{\pgfqpoint{7.070425in}{1.630793in}}%
\pgfpathclose%
\pgfusepath{stroke,fill}%
\end{pgfscope}%
\begin{pgfscope}%
\pgfpathrectangle{\pgfqpoint{0.894063in}{0.630000in}}{\pgfqpoint{6.713438in}{2.060556in}} %
\pgfusepath{clip}%
\pgfsetbuttcap%
\pgfsetroundjoin%
\definecolor{currentfill}{rgb}{0.000000,0.500000,0.000000}%
\pgfsetfillcolor{currentfill}%
\pgfsetlinewidth{1.003750pt}%
\definecolor{currentstroke}{rgb}{0.000000,0.500000,0.000000}%
\pgfsetstrokecolor{currentstroke}%
\pgfsetdash{}{0pt}%
\pgfpathmoveto{\pgfqpoint{3.176631in}{1.007140in}}%
\pgfpathcurveto{\pgfqpoint{3.184868in}{1.007140in}}{\pgfqpoint{3.192768in}{1.010412in}}{\pgfqpoint{3.198592in}{1.016236in}}%
\pgfpathcurveto{\pgfqpoint{3.204415in}{1.022060in}}{\pgfqpoint{3.207688in}{1.029960in}}{\pgfqpoint{3.207688in}{1.038196in}}%
\pgfpathcurveto{\pgfqpoint{3.207688in}{1.046432in}}{\pgfqpoint{3.204415in}{1.054332in}}{\pgfqpoint{3.198592in}{1.060156in}}%
\pgfpathcurveto{\pgfqpoint{3.192768in}{1.065980in}}{\pgfqpoint{3.184868in}{1.069253in}}{\pgfqpoint{3.176631in}{1.069253in}}%
\pgfpathcurveto{\pgfqpoint{3.168395in}{1.069253in}}{\pgfqpoint{3.160495in}{1.065980in}}{\pgfqpoint{3.154671in}{1.060156in}}%
\pgfpathcurveto{\pgfqpoint{3.148847in}{1.054332in}}{\pgfqpoint{3.145575in}{1.046432in}}{\pgfqpoint{3.145575in}{1.038196in}}%
\pgfpathcurveto{\pgfqpoint{3.145575in}{1.029960in}}{\pgfqpoint{3.148847in}{1.022060in}}{\pgfqpoint{3.154671in}{1.016236in}}%
\pgfpathcurveto{\pgfqpoint{3.160495in}{1.010412in}}{\pgfqpoint{3.168395in}{1.007140in}}{\pgfqpoint{3.176631in}{1.007140in}}%
\pgfpathclose%
\pgfusepath{stroke,fill}%
\end{pgfscope}%
\begin{pgfscope}%
\pgfpathrectangle{\pgfqpoint{0.894063in}{0.630000in}}{\pgfqpoint{6.713438in}{2.060556in}} %
\pgfusepath{clip}%
\pgfsetbuttcap%
\pgfsetroundjoin%
\definecolor{currentfill}{rgb}{0.000000,0.500000,0.000000}%
\pgfsetfillcolor{currentfill}%
\pgfsetlinewidth{1.003750pt}%
\definecolor{currentstroke}{rgb}{0.000000,0.500000,0.000000}%
\pgfsetstrokecolor{currentstroke}%
\pgfsetdash{}{0pt}%
\pgfpathmoveto{\pgfqpoint{2.102481in}{0.848335in}}%
\pgfpathcurveto{\pgfqpoint{2.110718in}{0.848335in}}{\pgfqpoint{2.118618in}{0.851608in}}{\pgfqpoint{2.124442in}{0.857432in}}%
\pgfpathcurveto{\pgfqpoint{2.130265in}{0.863256in}}{\pgfqpoint{2.133538in}{0.871156in}}{\pgfqpoint{2.133538in}{0.879392in}}%
\pgfpathcurveto{\pgfqpoint{2.133538in}{0.887628in}}{\pgfqpoint{2.130265in}{0.895528in}}{\pgfqpoint{2.124442in}{0.901352in}}%
\pgfpathcurveto{\pgfqpoint{2.118618in}{0.907176in}}{\pgfqpoint{2.110718in}{0.910448in}}{\pgfqpoint{2.102481in}{0.910448in}}%
\pgfpathcurveto{\pgfqpoint{2.094245in}{0.910448in}}{\pgfqpoint{2.086345in}{0.907176in}}{\pgfqpoint{2.080521in}{0.901352in}}%
\pgfpathcurveto{\pgfqpoint{2.074697in}{0.895528in}}{\pgfqpoint{2.071425in}{0.887628in}}{\pgfqpoint{2.071425in}{0.879392in}}%
\pgfpathcurveto{\pgfqpoint{2.071425in}{0.871156in}}{\pgfqpoint{2.074697in}{0.863256in}}{\pgfqpoint{2.080521in}{0.857432in}}%
\pgfpathcurveto{\pgfqpoint{2.086345in}{0.851608in}}{\pgfqpoint{2.094245in}{0.848335in}}{\pgfqpoint{2.102481in}{0.848335in}}%
\pgfpathclose%
\pgfusepath{stroke,fill}%
\end{pgfscope}%
\begin{pgfscope}%
\pgfpathrectangle{\pgfqpoint{0.894063in}{0.630000in}}{\pgfqpoint{6.713438in}{2.060556in}} %
\pgfusepath{clip}%
\pgfsetbuttcap%
\pgfsetroundjoin%
\definecolor{currentfill}{rgb}{0.000000,0.500000,0.000000}%
\pgfsetfillcolor{currentfill}%
\pgfsetlinewidth{1.003750pt}%
\definecolor{currentstroke}{rgb}{0.000000,0.500000,0.000000}%
\pgfsetstrokecolor{currentstroke}%
\pgfsetdash{}{0pt}%
\pgfpathmoveto{\pgfqpoint{1.968213in}{0.826034in}}%
\pgfpathcurveto{\pgfqpoint{1.976449in}{0.826034in}}{\pgfqpoint{1.984349in}{0.829307in}}{\pgfqpoint{1.990173in}{0.835131in}}%
\pgfpathcurveto{\pgfqpoint{1.995997in}{0.840955in}}{\pgfqpoint{1.999269in}{0.848855in}}{\pgfqpoint{1.999269in}{0.857091in}}%
\pgfpathcurveto{\pgfqpoint{1.999269in}{0.865327in}}{\pgfqpoint{1.995997in}{0.873227in}}{\pgfqpoint{1.990173in}{0.879051in}}%
\pgfpathcurveto{\pgfqpoint{1.984349in}{0.884875in}}{\pgfqpoint{1.976449in}{0.888147in}}{\pgfqpoint{1.968213in}{0.888147in}}%
\pgfpathcurveto{\pgfqpoint{1.959976in}{0.888147in}}{\pgfqpoint{1.952076in}{0.884875in}}{\pgfqpoint{1.946252in}{0.879051in}}%
\pgfpathcurveto{\pgfqpoint{1.940428in}{0.873227in}}{\pgfqpoint{1.937156in}{0.865327in}}{\pgfqpoint{1.937156in}{0.857091in}}%
\pgfpathcurveto{\pgfqpoint{1.937156in}{0.848855in}}{\pgfqpoint{1.940428in}{0.840955in}}{\pgfqpoint{1.946252in}{0.835131in}}%
\pgfpathcurveto{\pgfqpoint{1.952076in}{0.829307in}}{\pgfqpoint{1.959976in}{0.826034in}}{\pgfqpoint{1.968213in}{0.826034in}}%
\pgfpathclose%
\pgfusepath{stroke,fill}%
\end{pgfscope}%
\begin{pgfscope}%
\pgfpathrectangle{\pgfqpoint{0.894063in}{0.630000in}}{\pgfqpoint{6.713438in}{2.060556in}} %
\pgfusepath{clip}%
\pgfsetbuttcap%
\pgfsetroundjoin%
\definecolor{currentfill}{rgb}{0.000000,0.500000,0.000000}%
\pgfsetfillcolor{currentfill}%
\pgfsetlinewidth{1.003750pt}%
\definecolor{currentstroke}{rgb}{0.000000,0.500000,0.000000}%
\pgfsetstrokecolor{currentstroke}%
\pgfsetdash{}{0pt}%
\pgfpathmoveto{\pgfqpoint{3.310900in}{1.029706in}}%
\pgfpathcurveto{\pgfqpoint{3.319136in}{1.029706in}}{\pgfqpoint{3.327036in}{1.032978in}}{\pgfqpoint{3.332860in}{1.038802in}}%
\pgfpathcurveto{\pgfqpoint{3.338684in}{1.044626in}}{\pgfqpoint{3.341956in}{1.052526in}}{\pgfqpoint{3.341956in}{1.060762in}}%
\pgfpathcurveto{\pgfqpoint{3.341956in}{1.068998in}}{\pgfqpoint{3.338684in}{1.076898in}}{\pgfqpoint{3.332860in}{1.082722in}}%
\pgfpathcurveto{\pgfqpoint{3.327036in}{1.088546in}}{\pgfqpoint{3.319136in}{1.091819in}}{\pgfqpoint{3.310900in}{1.091819in}}%
\pgfpathcurveto{\pgfqpoint{3.302664in}{1.091819in}}{\pgfqpoint{3.294764in}{1.088546in}}{\pgfqpoint{3.288940in}{1.082722in}}%
\pgfpathcurveto{\pgfqpoint{3.283116in}{1.076898in}}{\pgfqpoint{3.279844in}{1.068998in}}{\pgfqpoint{3.279844in}{1.060762in}}%
\pgfpathcurveto{\pgfqpoint{3.279844in}{1.052526in}}{\pgfqpoint{3.283116in}{1.044626in}}{\pgfqpoint{3.288940in}{1.038802in}}%
\pgfpathcurveto{\pgfqpoint{3.294764in}{1.032978in}}{\pgfqpoint{3.302664in}{1.029706in}}{\pgfqpoint{3.310900in}{1.029706in}}%
\pgfpathclose%
\pgfusepath{stroke,fill}%
\end{pgfscope}%
\begin{pgfscope}%
\pgfpathrectangle{\pgfqpoint{0.894063in}{0.630000in}}{\pgfqpoint{6.713438in}{2.060556in}} %
\pgfusepath{clip}%
\pgfsetbuttcap%
\pgfsetroundjoin%
\definecolor{currentfill}{rgb}{0.000000,0.500000,0.000000}%
\pgfsetfillcolor{currentfill}%
\pgfsetlinewidth{1.003750pt}%
\definecolor{currentstroke}{rgb}{0.000000,0.500000,0.000000}%
\pgfsetstrokecolor{currentstroke}%
\pgfsetdash{}{0pt}%
\pgfpathmoveto{\pgfqpoint{5.593469in}{1.401748in}}%
\pgfpathcurveto{\pgfqpoint{5.601705in}{1.401748in}}{\pgfqpoint{5.609605in}{1.405020in}}{\pgfqpoint{5.615429in}{1.410844in}}%
\pgfpathcurveto{\pgfqpoint{5.621253in}{1.416668in}}{\pgfqpoint{5.624525in}{1.424568in}}{\pgfqpoint{5.624525in}{1.432804in}}%
\pgfpathcurveto{\pgfqpoint{5.624525in}{1.441040in}}{\pgfqpoint{5.621253in}{1.448941in}}{\pgfqpoint{5.615429in}{1.454764in}}%
\pgfpathcurveto{\pgfqpoint{5.609605in}{1.460588in}}{\pgfqpoint{5.601705in}{1.463861in}}{\pgfqpoint{5.593469in}{1.463861in}}%
\pgfpathcurveto{\pgfqpoint{5.585232in}{1.463861in}}{\pgfqpoint{5.577332in}{1.460588in}}{\pgfqpoint{5.571508in}{1.454764in}}%
\pgfpathcurveto{\pgfqpoint{5.565685in}{1.448941in}}{\pgfqpoint{5.562412in}{1.441040in}}{\pgfqpoint{5.562412in}{1.432804in}}%
\pgfpathcurveto{\pgfqpoint{5.562412in}{1.424568in}}{\pgfqpoint{5.565685in}{1.416668in}}{\pgfqpoint{5.571508in}{1.410844in}}%
\pgfpathcurveto{\pgfqpoint{5.577332in}{1.405020in}}{\pgfqpoint{5.585232in}{1.401748in}}{\pgfqpoint{5.593469in}{1.401748in}}%
\pgfpathclose%
\pgfusepath{stroke,fill}%
\end{pgfscope}%
\begin{pgfscope}%
\pgfpathrectangle{\pgfqpoint{0.894063in}{0.630000in}}{\pgfqpoint{6.713438in}{2.060556in}} %
\pgfusepath{clip}%
\pgfsetbuttcap%
\pgfsetroundjoin%
\definecolor{currentfill}{rgb}{0.000000,0.500000,0.000000}%
\pgfsetfillcolor{currentfill}%
\pgfsetlinewidth{1.003750pt}%
\definecolor{currentstroke}{rgb}{0.000000,0.500000,0.000000}%
\pgfsetstrokecolor{currentstroke}%
\pgfsetdash{}{0pt}%
\pgfpathmoveto{\pgfqpoint{3.042363in}{0.989360in}}%
\pgfpathcurveto{\pgfqpoint{3.050599in}{0.989360in}}{\pgfqpoint{3.058499in}{0.992632in}}{\pgfqpoint{3.064323in}{0.998456in}}%
\pgfpathcurveto{\pgfqpoint{3.070147in}{1.004280in}}{\pgfqpoint{3.073419in}{1.012180in}}{\pgfqpoint{3.073419in}{1.020416in}}%
\pgfpathcurveto{\pgfqpoint{3.073419in}{1.028653in}}{\pgfqpoint{3.070147in}{1.036553in}}{\pgfqpoint{3.064323in}{1.042377in}}%
\pgfpathcurveto{\pgfqpoint{3.058499in}{1.048201in}}{\pgfqpoint{3.050599in}{1.051473in}}{\pgfqpoint{3.042363in}{1.051473in}}%
\pgfpathcurveto{\pgfqpoint{3.034126in}{1.051473in}}{\pgfqpoint{3.026226in}{1.048201in}}{\pgfqpoint{3.020402in}{1.042377in}}%
\pgfpathcurveto{\pgfqpoint{3.014578in}{1.036553in}}{\pgfqpoint{3.011306in}{1.028653in}}{\pgfqpoint{3.011306in}{1.020416in}}%
\pgfpathcurveto{\pgfqpoint{3.011306in}{1.012180in}}{\pgfqpoint{3.014578in}{1.004280in}}{\pgfqpoint{3.020402in}{0.998456in}}%
\pgfpathcurveto{\pgfqpoint{3.026226in}{0.992632in}}{\pgfqpoint{3.034126in}{0.989360in}}{\pgfqpoint{3.042363in}{0.989360in}}%
\pgfpathclose%
\pgfusepath{stroke,fill}%
\end{pgfscope}%
\begin{pgfscope}%
\pgfpathrectangle{\pgfqpoint{0.894063in}{0.630000in}}{\pgfqpoint{6.713438in}{2.060556in}} %
\pgfusepath{clip}%
\pgfsetbuttcap%
\pgfsetroundjoin%
\definecolor{currentfill}{rgb}{0.000000,0.500000,0.000000}%
\pgfsetfillcolor{currentfill}%
\pgfsetlinewidth{1.003750pt}%
\definecolor{currentstroke}{rgb}{0.000000,0.500000,0.000000}%
\pgfsetstrokecolor{currentstroke}%
\pgfsetdash{}{0pt}%
\pgfpathmoveto{\pgfqpoint{5.190663in}{1.349433in}}%
\pgfpathcurveto{\pgfqpoint{5.198899in}{1.349433in}}{\pgfqpoint{5.206799in}{1.352705in}}{\pgfqpoint{5.212623in}{1.358529in}}%
\pgfpathcurveto{\pgfqpoint{5.218447in}{1.364353in}}{\pgfqpoint{5.221719in}{1.372253in}}{\pgfqpoint{5.221719in}{1.380490in}}%
\pgfpathcurveto{\pgfqpoint{5.221719in}{1.388726in}}{\pgfqpoint{5.218447in}{1.396626in}}{\pgfqpoint{5.212623in}{1.402450in}}%
\pgfpathcurveto{\pgfqpoint{5.206799in}{1.408274in}}{\pgfqpoint{5.198899in}{1.411546in}}{\pgfqpoint{5.190663in}{1.411546in}}%
\pgfpathcurveto{\pgfqpoint{5.182426in}{1.411546in}}{\pgfqpoint{5.174526in}{1.408274in}}{\pgfqpoint{5.168702in}{1.402450in}}%
\pgfpathcurveto{\pgfqpoint{5.162878in}{1.396626in}}{\pgfqpoint{5.159606in}{1.388726in}}{\pgfqpoint{5.159606in}{1.380490in}}%
\pgfpathcurveto{\pgfqpoint{5.159606in}{1.372253in}}{\pgfqpoint{5.162878in}{1.364353in}}{\pgfqpoint{5.168702in}{1.358529in}}%
\pgfpathcurveto{\pgfqpoint{5.174526in}{1.352705in}}{\pgfqpoint{5.182426in}{1.349433in}}{\pgfqpoint{5.190663in}{1.349433in}}%
\pgfpathclose%
\pgfusepath{stroke,fill}%
\end{pgfscope}%
\begin{pgfscope}%
\pgfpathrectangle{\pgfqpoint{0.894063in}{0.630000in}}{\pgfqpoint{6.713438in}{2.060556in}} %
\pgfusepath{clip}%
\pgfsetbuttcap%
\pgfsetroundjoin%
\definecolor{currentfill}{rgb}{0.000000,0.500000,0.000000}%
\pgfsetfillcolor{currentfill}%
\pgfsetlinewidth{1.003750pt}%
\definecolor{currentstroke}{rgb}{0.000000,0.500000,0.000000}%
\pgfsetstrokecolor{currentstroke}%
\pgfsetdash{}{0pt}%
\pgfpathmoveto{\pgfqpoint{6.801888in}{1.579444in}}%
\pgfpathcurveto{\pgfqpoint{6.810124in}{1.579444in}}{\pgfqpoint{6.818024in}{1.582716in}}{\pgfqpoint{6.823848in}{1.588540in}}%
\pgfpathcurveto{\pgfqpoint{6.829672in}{1.594364in}}{\pgfqpoint{6.832944in}{1.602264in}}{\pgfqpoint{6.832944in}{1.610501in}}%
\pgfpathcurveto{\pgfqpoint{6.832944in}{1.618737in}}{\pgfqpoint{6.829672in}{1.626637in}}{\pgfqpoint{6.823848in}{1.632461in}}%
\pgfpathcurveto{\pgfqpoint{6.818024in}{1.638285in}}{\pgfqpoint{6.810124in}{1.641557in}}{\pgfqpoint{6.801888in}{1.641557in}}%
\pgfpathcurveto{\pgfqpoint{6.793651in}{1.641557in}}{\pgfqpoint{6.785751in}{1.638285in}}{\pgfqpoint{6.779927in}{1.632461in}}%
\pgfpathcurveto{\pgfqpoint{6.774103in}{1.626637in}}{\pgfqpoint{6.770831in}{1.618737in}}{\pgfqpoint{6.770831in}{1.610501in}}%
\pgfpathcurveto{\pgfqpoint{6.770831in}{1.602264in}}{\pgfqpoint{6.774103in}{1.594364in}}{\pgfqpoint{6.779927in}{1.588540in}}%
\pgfpathcurveto{\pgfqpoint{6.785751in}{1.582716in}}{\pgfqpoint{6.793651in}{1.579444in}}{\pgfqpoint{6.801888in}{1.579444in}}%
\pgfpathclose%
\pgfusepath{stroke,fill}%
\end{pgfscope}%
\begin{pgfscope}%
\pgfpathrectangle{\pgfqpoint{0.894063in}{0.630000in}}{\pgfqpoint{6.713438in}{2.060556in}} %
\pgfusepath{clip}%
\pgfsetbuttcap%
\pgfsetroundjoin%
\definecolor{currentfill}{rgb}{0.000000,0.500000,0.000000}%
\pgfsetfillcolor{currentfill}%
\pgfsetlinewidth{1.003750pt}%
\definecolor{currentstroke}{rgb}{0.000000,0.500000,0.000000}%
\pgfsetstrokecolor{currentstroke}%
\pgfsetdash{}{0pt}%
\pgfpathmoveto{\pgfqpoint{3.579438in}{1.076539in}}%
\pgfpathcurveto{\pgfqpoint{3.587674in}{1.076539in}}{\pgfqpoint{3.595574in}{1.079811in}}{\pgfqpoint{3.601398in}{1.085635in}}%
\pgfpathcurveto{\pgfqpoint{3.607222in}{1.091459in}}{\pgfqpoint{3.610494in}{1.099359in}}{\pgfqpoint{3.610494in}{1.107596in}}%
\pgfpathcurveto{\pgfqpoint{3.610494in}{1.115832in}}{\pgfqpoint{3.607222in}{1.123732in}}{\pgfqpoint{3.601398in}{1.129556in}}%
\pgfpathcurveto{\pgfqpoint{3.595574in}{1.135380in}}{\pgfqpoint{3.587674in}{1.138652in}}{\pgfqpoint{3.579438in}{1.138652in}}%
\pgfpathcurveto{\pgfqpoint{3.571201in}{1.138652in}}{\pgfqpoint{3.563301in}{1.135380in}}{\pgfqpoint{3.557477in}{1.129556in}}%
\pgfpathcurveto{\pgfqpoint{3.551653in}{1.123732in}}{\pgfqpoint{3.548381in}{1.115832in}}{\pgfqpoint{3.548381in}{1.107596in}}%
\pgfpathcurveto{\pgfqpoint{3.548381in}{1.099359in}}{\pgfqpoint{3.551653in}{1.091459in}}{\pgfqpoint{3.557477in}{1.085635in}}%
\pgfpathcurveto{\pgfqpoint{3.563301in}{1.079811in}}{\pgfqpoint{3.571201in}{1.076539in}}{\pgfqpoint{3.579438in}{1.076539in}}%
\pgfpathclose%
\pgfusepath{stroke,fill}%
\end{pgfscope}%
\begin{pgfscope}%
\pgfpathrectangle{\pgfqpoint{0.894063in}{0.630000in}}{\pgfqpoint{6.713438in}{2.060556in}} %
\pgfusepath{clip}%
\pgfsetbuttcap%
\pgfsetroundjoin%
\definecolor{currentfill}{rgb}{0.000000,0.500000,0.000000}%
\pgfsetfillcolor{currentfill}%
\pgfsetlinewidth{1.003750pt}%
\definecolor{currentstroke}{rgb}{0.000000,0.500000,0.000000}%
\pgfsetstrokecolor{currentstroke}%
\pgfsetdash{}{0pt}%
\pgfpathmoveto{\pgfqpoint{2.371019in}{0.886632in}}%
\pgfpathcurveto{\pgfqpoint{2.379255in}{0.886632in}}{\pgfqpoint{2.387155in}{0.889905in}}{\pgfqpoint{2.392979in}{0.895729in}}%
\pgfpathcurveto{\pgfqpoint{2.398803in}{0.901553in}}{\pgfqpoint{2.402075in}{0.909453in}}{\pgfqpoint{2.402075in}{0.917689in}}%
\pgfpathcurveto{\pgfqpoint{2.402075in}{0.925925in}}{\pgfqpoint{2.398803in}{0.933825in}}{\pgfqpoint{2.392979in}{0.939649in}}%
\pgfpathcurveto{\pgfqpoint{2.387155in}{0.945473in}}{\pgfqpoint{2.379255in}{0.948745in}}{\pgfqpoint{2.371019in}{0.948745in}}%
\pgfpathcurveto{\pgfqpoint{2.362782in}{0.948745in}}{\pgfqpoint{2.354882in}{0.945473in}}{\pgfqpoint{2.349058in}{0.939649in}}%
\pgfpathcurveto{\pgfqpoint{2.343235in}{0.933825in}}{\pgfqpoint{2.339962in}{0.925925in}}{\pgfqpoint{2.339962in}{0.917689in}}%
\pgfpathcurveto{\pgfqpoint{2.339962in}{0.909453in}}{\pgfqpoint{2.343235in}{0.901553in}}{\pgfqpoint{2.349058in}{0.895729in}}%
\pgfpathcurveto{\pgfqpoint{2.354882in}{0.889905in}}{\pgfqpoint{2.362782in}{0.886632in}}{\pgfqpoint{2.371019in}{0.886632in}}%
\pgfpathclose%
\pgfusepath{stroke,fill}%
\end{pgfscope}%
\begin{pgfscope}%
\pgfpathrectangle{\pgfqpoint{0.894063in}{0.630000in}}{\pgfqpoint{6.713438in}{2.060556in}} %
\pgfusepath{clip}%
\pgfsetbuttcap%
\pgfsetroundjoin%
\definecolor{currentfill}{rgb}{0.000000,0.500000,0.000000}%
\pgfsetfillcolor{currentfill}%
\pgfsetlinewidth{1.003750pt}%
\definecolor{currentstroke}{rgb}{0.000000,0.500000,0.000000}%
\pgfsetstrokecolor{currentstroke}%
\pgfsetdash{}{0pt}%
\pgfpathmoveto{\pgfqpoint{3.982244in}{1.142542in}}%
\pgfpathcurveto{\pgfqpoint{3.990480in}{1.142542in}}{\pgfqpoint{3.998380in}{1.145814in}}{\pgfqpoint{4.004204in}{1.151638in}}%
\pgfpathcurveto{\pgfqpoint{4.010028in}{1.157462in}}{\pgfqpoint{4.013300in}{1.165362in}}{\pgfqpoint{4.013300in}{1.173598in}}%
\pgfpathcurveto{\pgfqpoint{4.013300in}{1.181834in}}{\pgfqpoint{4.010028in}{1.189734in}}{\pgfqpoint{4.004204in}{1.195558in}}%
\pgfpathcurveto{\pgfqpoint{3.998380in}{1.201382in}}{\pgfqpoint{3.990480in}{1.204655in}}{\pgfqpoint{3.982244in}{1.204655in}}%
\pgfpathcurveto{\pgfqpoint{3.974007in}{1.204655in}}{\pgfqpoint{3.966107in}{1.201382in}}{\pgfqpoint{3.960283in}{1.195558in}}%
\pgfpathcurveto{\pgfqpoint{3.954460in}{1.189734in}}{\pgfqpoint{3.951187in}{1.181834in}}{\pgfqpoint{3.951187in}{1.173598in}}%
\pgfpathcurveto{\pgfqpoint{3.951187in}{1.165362in}}{\pgfqpoint{3.954460in}{1.157462in}}{\pgfqpoint{3.960283in}{1.151638in}}%
\pgfpathcurveto{\pgfqpoint{3.966107in}{1.145814in}}{\pgfqpoint{3.974007in}{1.142542in}}{\pgfqpoint{3.982244in}{1.142542in}}%
\pgfpathclose%
\pgfusepath{stroke,fill}%
\end{pgfscope}%
\begin{pgfscope}%
\pgfpathrectangle{\pgfqpoint{0.894063in}{0.630000in}}{\pgfqpoint{6.713438in}{2.060556in}} %
\pgfusepath{clip}%
\pgfsetbuttcap%
\pgfsetroundjoin%
\definecolor{currentfill}{rgb}{0.000000,0.500000,0.000000}%
\pgfsetfillcolor{currentfill}%
\pgfsetlinewidth{1.003750pt}%
\definecolor{currentstroke}{rgb}{0.000000,0.500000,0.000000}%
\pgfsetstrokecolor{currentstroke}%
\pgfsetdash{}{0pt}%
\pgfpathmoveto{\pgfqpoint{4.653588in}{1.272174in}}%
\pgfpathcurveto{\pgfqpoint{4.661824in}{1.272174in}}{\pgfqpoint{4.669724in}{1.275446in}}{\pgfqpoint{4.675548in}{1.281270in}}%
\pgfpathcurveto{\pgfqpoint{4.681372in}{1.287094in}}{\pgfqpoint{4.684644in}{1.294994in}}{\pgfqpoint{4.684644in}{1.303231in}}%
\pgfpathcurveto{\pgfqpoint{4.684644in}{1.311467in}}{\pgfqpoint{4.681372in}{1.319367in}}{\pgfqpoint{4.675548in}{1.325191in}}%
\pgfpathcurveto{\pgfqpoint{4.669724in}{1.331015in}}{\pgfqpoint{4.661824in}{1.334287in}}{\pgfqpoint{4.653588in}{1.334287in}}%
\pgfpathcurveto{\pgfqpoint{4.645351in}{1.334287in}}{\pgfqpoint{4.637451in}{1.331015in}}{\pgfqpoint{4.631627in}{1.325191in}}%
\pgfpathcurveto{\pgfqpoint{4.625803in}{1.319367in}}{\pgfqpoint{4.622531in}{1.311467in}}{\pgfqpoint{4.622531in}{1.303231in}}%
\pgfpathcurveto{\pgfqpoint{4.622531in}{1.294994in}}{\pgfqpoint{4.625803in}{1.287094in}}{\pgfqpoint{4.631627in}{1.281270in}}%
\pgfpathcurveto{\pgfqpoint{4.637451in}{1.275446in}}{\pgfqpoint{4.645351in}{1.272174in}}{\pgfqpoint{4.653588in}{1.272174in}}%
\pgfpathclose%
\pgfusepath{stroke,fill}%
\end{pgfscope}%
\begin{pgfscope}%
\pgfpathrectangle{\pgfqpoint{0.894063in}{0.630000in}}{\pgfqpoint{6.713438in}{2.060556in}} %
\pgfusepath{clip}%
\pgfsetbuttcap%
\pgfsetroundjoin%
\definecolor{currentfill}{rgb}{0.000000,0.500000,0.000000}%
\pgfsetfillcolor{currentfill}%
\pgfsetlinewidth{1.003750pt}%
\definecolor{currentstroke}{rgb}{0.000000,0.500000,0.000000}%
\pgfsetstrokecolor{currentstroke}%
\pgfsetdash{}{0pt}%
\pgfpathmoveto{\pgfqpoint{3.713706in}{1.102378in}}%
\pgfpathcurveto{\pgfqpoint{3.721943in}{1.102378in}}{\pgfqpoint{3.729843in}{1.105651in}}{\pgfqpoint{3.735667in}{1.111475in}}%
\pgfpathcurveto{\pgfqpoint{3.741490in}{1.117299in}}{\pgfqpoint{3.744763in}{1.125199in}}{\pgfqpoint{3.744763in}{1.133435in}}%
\pgfpathcurveto{\pgfqpoint{3.744763in}{1.141671in}}{\pgfqpoint{3.741490in}{1.149571in}}{\pgfqpoint{3.735667in}{1.155395in}}%
\pgfpathcurveto{\pgfqpoint{3.729843in}{1.161219in}}{\pgfqpoint{3.721943in}{1.164491in}}{\pgfqpoint{3.713706in}{1.164491in}}%
\pgfpathcurveto{\pgfqpoint{3.705470in}{1.164491in}}{\pgfqpoint{3.697570in}{1.161219in}}{\pgfqpoint{3.691746in}{1.155395in}}%
\pgfpathcurveto{\pgfqpoint{3.685922in}{1.149571in}}{\pgfqpoint{3.682650in}{1.141671in}}{\pgfqpoint{3.682650in}{1.133435in}}%
\pgfpathcurveto{\pgfqpoint{3.682650in}{1.125199in}}{\pgfqpoint{3.685922in}{1.117299in}}{\pgfqpoint{3.691746in}{1.111475in}}%
\pgfpathcurveto{\pgfqpoint{3.697570in}{1.105651in}}{\pgfqpoint{3.705470in}{1.102378in}}{\pgfqpoint{3.713706in}{1.102378in}}%
\pgfpathclose%
\pgfusepath{stroke,fill}%
\end{pgfscope}%
\begin{pgfscope}%
\pgfpathrectangle{\pgfqpoint{0.894063in}{0.630000in}}{\pgfqpoint{6.713438in}{2.060556in}} %
\pgfusepath{clip}%
\pgfsetbuttcap%
\pgfsetroundjoin%
\definecolor{currentfill}{rgb}{0.000000,0.500000,0.000000}%
\pgfsetfillcolor{currentfill}%
\pgfsetlinewidth{1.003750pt}%
\definecolor{currentstroke}{rgb}{0.000000,0.500000,0.000000}%
\pgfsetstrokecolor{currentstroke}%
\pgfsetdash{}{0pt}%
\pgfpathmoveto{\pgfqpoint{2.236750in}{0.868040in}}%
\pgfpathcurveto{\pgfqpoint{2.244986in}{0.868040in}}{\pgfqpoint{2.252886in}{0.871313in}}{\pgfqpoint{2.258710in}{0.877137in}}%
\pgfpathcurveto{\pgfqpoint{2.264534in}{0.882960in}}{\pgfqpoint{2.267806in}{0.890861in}}{\pgfqpoint{2.267806in}{0.899097in}}%
\pgfpathcurveto{\pgfqpoint{2.267806in}{0.907333in}}{\pgfqpoint{2.264534in}{0.915233in}}{\pgfqpoint{2.258710in}{0.921057in}}%
\pgfpathcurveto{\pgfqpoint{2.252886in}{0.926881in}}{\pgfqpoint{2.244986in}{0.930153in}}{\pgfqpoint{2.236750in}{0.930153in}}%
\pgfpathcurveto{\pgfqpoint{2.228514in}{0.930153in}}{\pgfqpoint{2.220614in}{0.926881in}}{\pgfqpoint{2.214790in}{0.921057in}}%
\pgfpathcurveto{\pgfqpoint{2.208966in}{0.915233in}}{\pgfqpoint{2.205694in}{0.907333in}}{\pgfqpoint{2.205694in}{0.899097in}}%
\pgfpathcurveto{\pgfqpoint{2.205694in}{0.890861in}}{\pgfqpoint{2.208966in}{0.882960in}}{\pgfqpoint{2.214790in}{0.877137in}}%
\pgfpathcurveto{\pgfqpoint{2.220614in}{0.871313in}}{\pgfqpoint{2.228514in}{0.868040in}}{\pgfqpoint{2.236750in}{0.868040in}}%
\pgfpathclose%
\pgfusepath{stroke,fill}%
\end{pgfscope}%
\begin{pgfscope}%
\pgfpathrectangle{\pgfqpoint{0.894063in}{0.630000in}}{\pgfqpoint{6.713438in}{2.060556in}} %
\pgfusepath{clip}%
\pgfsetbuttcap%
\pgfsetroundjoin%
\definecolor{currentfill}{rgb}{0.000000,0.000000,0.000000}%
\pgfsetfillcolor{currentfill}%
\pgfsetlinewidth{1.003750pt}%
\definecolor{currentstroke}{rgb}{0.000000,0.000000,0.000000}%
\pgfsetstrokecolor{currentstroke}%
\pgfsetdash{}{0pt}%
\pgfpathmoveto{\pgfqpoint{6.667619in}{1.559445in}}%
\pgfpathcurveto{\pgfqpoint{6.675855in}{1.559445in}}{\pgfqpoint{6.683755in}{1.562717in}}{\pgfqpoint{6.689579in}{1.568541in}}%
\pgfpathcurveto{\pgfqpoint{6.695403in}{1.574365in}}{\pgfqpoint{6.698675in}{1.582265in}}{\pgfqpoint{6.698675in}{1.590501in}}%
\pgfpathcurveto{\pgfqpoint{6.698675in}{1.598738in}}{\pgfqpoint{6.695403in}{1.606638in}}{\pgfqpoint{6.689579in}{1.612462in}}%
\pgfpathcurveto{\pgfqpoint{6.683755in}{1.618286in}}{\pgfqpoint{6.675855in}{1.621558in}}{\pgfqpoint{6.667619in}{1.621558in}}%
\pgfpathcurveto{\pgfqpoint{6.659382in}{1.621558in}}{\pgfqpoint{6.651482in}{1.618286in}}{\pgfqpoint{6.645658in}{1.612462in}}%
\pgfpathcurveto{\pgfqpoint{6.639835in}{1.606638in}}{\pgfqpoint{6.636562in}{1.598738in}}{\pgfqpoint{6.636562in}{1.590501in}}%
\pgfpathcurveto{\pgfqpoint{6.636562in}{1.582265in}}{\pgfqpoint{6.639835in}{1.574365in}}{\pgfqpoint{6.645658in}{1.568541in}}%
\pgfpathcurveto{\pgfqpoint{6.651482in}{1.562717in}}{\pgfqpoint{6.659382in}{1.559445in}}{\pgfqpoint{6.667619in}{1.559445in}}%
\pgfpathclose%
\pgfusepath{stroke,fill}%
\end{pgfscope}%
\begin{pgfscope}%
\pgfpathrectangle{\pgfqpoint{0.894063in}{0.630000in}}{\pgfqpoint{6.713438in}{2.060556in}} %
\pgfusepath{clip}%
\pgfsetbuttcap%
\pgfsetroundjoin%
\definecolor{currentfill}{rgb}{0.000000,0.000000,0.000000}%
\pgfsetfillcolor{currentfill}%
\pgfsetlinewidth{1.003750pt}%
\definecolor{currentstroke}{rgb}{0.000000,0.000000,0.000000}%
\pgfsetstrokecolor{currentstroke}%
\pgfsetdash{}{0pt}%
\pgfpathmoveto{\pgfqpoint{2.639556in}{0.925424in}}%
\pgfpathcurveto{\pgfqpoint{2.647793in}{0.925424in}}{\pgfqpoint{2.655693in}{0.928696in}}{\pgfqpoint{2.661517in}{0.934520in}}%
\pgfpathcurveto{\pgfqpoint{2.667340in}{0.940344in}}{\pgfqpoint{2.670613in}{0.948244in}}{\pgfqpoint{2.670613in}{0.956480in}}%
\pgfpathcurveto{\pgfqpoint{2.670613in}{0.964717in}}{\pgfqpoint{2.667340in}{0.972617in}}{\pgfqpoint{2.661517in}{0.978441in}}%
\pgfpathcurveto{\pgfqpoint{2.655693in}{0.984265in}}{\pgfqpoint{2.647793in}{0.987537in}}{\pgfqpoint{2.639556in}{0.987537in}}%
\pgfpathcurveto{\pgfqpoint{2.631320in}{0.987537in}}{\pgfqpoint{2.623420in}{0.984265in}}{\pgfqpoint{2.617596in}{0.978441in}}%
\pgfpathcurveto{\pgfqpoint{2.611772in}{0.972617in}}{\pgfqpoint{2.608500in}{0.964717in}}{\pgfqpoint{2.608500in}{0.956480in}}%
\pgfpathcurveto{\pgfqpoint{2.608500in}{0.948244in}}{\pgfqpoint{2.611772in}{0.940344in}}{\pgfqpoint{2.617596in}{0.934520in}}%
\pgfpathcurveto{\pgfqpoint{2.623420in}{0.928696in}}{\pgfqpoint{2.631320in}{0.925424in}}{\pgfqpoint{2.639556in}{0.925424in}}%
\pgfpathclose%
\pgfusepath{stroke,fill}%
\end{pgfscope}%
\begin{pgfscope}%
\pgfpathrectangle{\pgfqpoint{0.894063in}{0.630000in}}{\pgfqpoint{6.713438in}{2.060556in}} %
\pgfusepath{clip}%
\pgfsetbuttcap%
\pgfsetroundjoin%
\definecolor{currentfill}{rgb}{0.000000,0.000000,0.000000}%
\pgfsetfillcolor{currentfill}%
\pgfsetlinewidth{1.003750pt}%
\definecolor{currentstroke}{rgb}{0.000000,0.000000,0.000000}%
\pgfsetstrokecolor{currentstroke}%
\pgfsetdash{}{0pt}%
\pgfpathmoveto{\pgfqpoint{1.699675in}{0.785536in}}%
\pgfpathcurveto{\pgfqpoint{1.707911in}{0.785536in}}{\pgfqpoint{1.715811in}{0.788808in}}{\pgfqpoint{1.721635in}{0.794632in}}%
\pgfpathcurveto{\pgfqpoint{1.727459in}{0.800456in}}{\pgfqpoint{1.730731in}{0.808356in}}{\pgfqpoint{1.730731in}{0.816592in}}%
\pgfpathcurveto{\pgfqpoint{1.730731in}{0.824828in}}{\pgfqpoint{1.727459in}{0.832728in}}{\pgfqpoint{1.721635in}{0.838552in}}%
\pgfpathcurveto{\pgfqpoint{1.715811in}{0.844376in}}{\pgfqpoint{1.707911in}{0.847649in}}{\pgfqpoint{1.699675in}{0.847649in}}%
\pgfpathcurveto{\pgfqpoint{1.691439in}{0.847649in}}{\pgfqpoint{1.683539in}{0.844376in}}{\pgfqpoint{1.677715in}{0.838552in}}%
\pgfpathcurveto{\pgfqpoint{1.671891in}{0.832728in}}{\pgfqpoint{1.668619in}{0.824828in}}{\pgfqpoint{1.668619in}{0.816592in}}%
\pgfpathcurveto{\pgfqpoint{1.668619in}{0.808356in}}{\pgfqpoint{1.671891in}{0.800456in}}{\pgfqpoint{1.677715in}{0.794632in}}%
\pgfpathcurveto{\pgfqpoint{1.683539in}{0.788808in}}{\pgfqpoint{1.691439in}{0.785536in}}{\pgfqpoint{1.699675in}{0.785536in}}%
\pgfpathclose%
\pgfusepath{stroke,fill}%
\end{pgfscope}%
\begin{pgfscope}%
\pgfpathrectangle{\pgfqpoint{0.894063in}{0.630000in}}{\pgfqpoint{6.713438in}{2.060556in}} %
\pgfusepath{clip}%
\pgfsetbuttcap%
\pgfsetroundjoin%
\definecolor{currentfill}{rgb}{0.000000,0.000000,0.000000}%
\pgfsetfillcolor{currentfill}%
\pgfsetlinewidth{1.003750pt}%
\definecolor{currentstroke}{rgb}{0.000000,0.000000,0.000000}%
\pgfsetstrokecolor{currentstroke}%
\pgfsetdash{}{0pt}%
\pgfpathmoveto{\pgfqpoint{1.162600in}{0.691780in}}%
\pgfpathcurveto{\pgfqpoint{1.170836in}{0.691780in}}{\pgfqpoint{1.178736in}{0.695053in}}{\pgfqpoint{1.184560in}{0.700877in}}%
\pgfpathcurveto{\pgfqpoint{1.190384in}{0.706701in}}{\pgfqpoint{1.193656in}{0.714601in}}{\pgfqpoint{1.193656in}{0.722837in}}%
\pgfpathcurveto{\pgfqpoint{1.193656in}{0.731073in}}{\pgfqpoint{1.190384in}{0.738973in}}{\pgfqpoint{1.184560in}{0.744797in}}%
\pgfpathcurveto{\pgfqpoint{1.178736in}{0.750621in}}{\pgfqpoint{1.170836in}{0.753893in}}{\pgfqpoint{1.162600in}{0.753893in}}%
\pgfpathcurveto{\pgfqpoint{1.154364in}{0.753893in}}{\pgfqpoint{1.146464in}{0.750621in}}{\pgfqpoint{1.140640in}{0.744797in}}%
\pgfpathcurveto{\pgfqpoint{1.134816in}{0.738973in}}{\pgfqpoint{1.131544in}{0.731073in}}{\pgfqpoint{1.131544in}{0.722837in}}%
\pgfpathcurveto{\pgfqpoint{1.131544in}{0.714601in}}{\pgfqpoint{1.134816in}{0.706701in}}{\pgfqpoint{1.140640in}{0.700877in}}%
\pgfpathcurveto{\pgfqpoint{1.146464in}{0.695053in}}{\pgfqpoint{1.154364in}{0.691780in}}{\pgfqpoint{1.162600in}{0.691780in}}%
\pgfpathclose%
\pgfusepath{stroke,fill}%
\end{pgfscope}%
\begin{pgfscope}%
\pgfpathrectangle{\pgfqpoint{0.894063in}{0.630000in}}{\pgfqpoint{6.713438in}{2.060556in}} %
\pgfusepath{clip}%
\pgfsetbuttcap%
\pgfsetroundjoin%
\definecolor{currentfill}{rgb}{0.000000,0.000000,0.000000}%
\pgfsetfillcolor{currentfill}%
\pgfsetlinewidth{1.003750pt}%
\definecolor{currentstroke}{rgb}{0.000000,0.000000,0.000000}%
\pgfsetstrokecolor{currentstroke}%
\pgfsetdash{}{0pt}%
\pgfpathmoveto{\pgfqpoint{1.833944in}{0.806124in}}%
\pgfpathcurveto{\pgfqpoint{1.842180in}{0.806124in}}{\pgfqpoint{1.850080in}{0.809396in}}{\pgfqpoint{1.855904in}{0.815220in}}%
\pgfpathcurveto{\pgfqpoint{1.861728in}{0.821044in}}{\pgfqpoint{1.865000in}{0.828944in}}{\pgfqpoint{1.865000in}{0.837180in}}%
\pgfpathcurveto{\pgfqpoint{1.865000in}{0.845416in}}{\pgfqpoint{1.861728in}{0.853316in}}{\pgfqpoint{1.855904in}{0.859140in}}%
\pgfpathcurveto{\pgfqpoint{1.850080in}{0.864964in}}{\pgfqpoint{1.842180in}{0.868237in}}{\pgfqpoint{1.833944in}{0.868237in}}%
\pgfpathcurveto{\pgfqpoint{1.825707in}{0.868237in}}{\pgfqpoint{1.817807in}{0.864964in}}{\pgfqpoint{1.811983in}{0.859140in}}%
\pgfpathcurveto{\pgfqpoint{1.806160in}{0.853316in}}{\pgfqpoint{1.802887in}{0.845416in}}{\pgfqpoint{1.802887in}{0.837180in}}%
\pgfpathcurveto{\pgfqpoint{1.802887in}{0.828944in}}{\pgfqpoint{1.806160in}{0.821044in}}{\pgfqpoint{1.811983in}{0.815220in}}%
\pgfpathcurveto{\pgfqpoint{1.817807in}{0.809396in}}{\pgfqpoint{1.825707in}{0.806124in}}{\pgfqpoint{1.833944in}{0.806124in}}%
\pgfpathclose%
\pgfusepath{stroke,fill}%
\end{pgfscope}%
\begin{pgfscope}%
\pgfpathrectangle{\pgfqpoint{0.894063in}{0.630000in}}{\pgfqpoint{6.713438in}{2.060556in}} %
\pgfusepath{clip}%
\pgfsetbuttcap%
\pgfsetroundjoin%
\definecolor{currentfill}{rgb}{0.000000,0.000000,0.000000}%
\pgfsetfillcolor{currentfill}%
\pgfsetlinewidth{1.003750pt}%
\definecolor{currentstroke}{rgb}{0.000000,0.000000,0.000000}%
\pgfsetstrokecolor{currentstroke}%
\pgfsetdash{}{0pt}%
\pgfpathmoveto{\pgfqpoint{5.996275in}{1.470812in}}%
\pgfpathcurveto{\pgfqpoint{6.004511in}{1.470812in}}{\pgfqpoint{6.012411in}{1.474084in}}{\pgfqpoint{6.018235in}{1.479908in}}%
\pgfpathcurveto{\pgfqpoint{6.024059in}{1.485732in}}{\pgfqpoint{6.027331in}{1.493632in}}{\pgfqpoint{6.027331in}{1.501868in}}%
\pgfpathcurveto{\pgfqpoint{6.027331in}{1.510104in}}{\pgfqpoint{6.024059in}{1.518004in}}{\pgfqpoint{6.018235in}{1.523828in}}%
\pgfpathcurveto{\pgfqpoint{6.012411in}{1.529652in}}{\pgfqpoint{6.004511in}{1.532925in}}{\pgfqpoint{5.996275in}{1.532925in}}%
\pgfpathcurveto{\pgfqpoint{5.988039in}{1.532925in}}{\pgfqpoint{5.980139in}{1.529652in}}{\pgfqpoint{5.974315in}{1.523828in}}%
\pgfpathcurveto{\pgfqpoint{5.968491in}{1.518004in}}{\pgfqpoint{5.965219in}{1.510104in}}{\pgfqpoint{5.965219in}{1.501868in}}%
\pgfpathcurveto{\pgfqpoint{5.965219in}{1.493632in}}{\pgfqpoint{5.968491in}{1.485732in}}{\pgfqpoint{5.974315in}{1.479908in}}%
\pgfpathcurveto{\pgfqpoint{5.980139in}{1.474084in}}{\pgfqpoint{5.988039in}{1.470812in}}{\pgfqpoint{5.996275in}{1.470812in}}%
\pgfpathclose%
\pgfusepath{stroke,fill}%
\end{pgfscope}%
\begin{pgfscope}%
\pgfpathrectangle{\pgfqpoint{0.894063in}{0.630000in}}{\pgfqpoint{6.713438in}{2.060556in}} %
\pgfusepath{clip}%
\pgfsetbuttcap%
\pgfsetroundjoin%
\definecolor{currentfill}{rgb}{0.000000,0.000000,0.000000}%
\pgfsetfillcolor{currentfill}%
\pgfsetlinewidth{1.003750pt}%
\definecolor{currentstroke}{rgb}{0.000000,0.000000,0.000000}%
\pgfsetstrokecolor{currentstroke}%
\pgfsetdash{}{0pt}%
\pgfpathmoveto{\pgfqpoint{6.399081in}{1.521348in}}%
\pgfpathcurveto{\pgfqpoint{6.407318in}{1.521348in}}{\pgfqpoint{6.415218in}{1.524621in}}{\pgfqpoint{6.421042in}{1.530444in}}%
\pgfpathcurveto{\pgfqpoint{6.426865in}{1.536268in}}{\pgfqpoint{6.430138in}{1.544168in}}{\pgfqpoint{6.430138in}{1.552405in}}%
\pgfpathcurveto{\pgfqpoint{6.430138in}{1.560641in}}{\pgfqpoint{6.426865in}{1.568541in}}{\pgfqpoint{6.421042in}{1.574365in}}%
\pgfpathcurveto{\pgfqpoint{6.415218in}{1.580189in}}{\pgfqpoint{6.407318in}{1.583461in}}{\pgfqpoint{6.399081in}{1.583461in}}%
\pgfpathcurveto{\pgfqpoint{6.390845in}{1.583461in}}{\pgfqpoint{6.382945in}{1.580189in}}{\pgfqpoint{6.377121in}{1.574365in}}%
\pgfpathcurveto{\pgfqpoint{6.371297in}{1.568541in}}{\pgfqpoint{6.368025in}{1.560641in}}{\pgfqpoint{6.368025in}{1.552405in}}%
\pgfpathcurveto{\pgfqpoint{6.368025in}{1.544168in}}{\pgfqpoint{6.371297in}{1.536268in}}{\pgfqpoint{6.377121in}{1.530444in}}%
\pgfpathcurveto{\pgfqpoint{6.382945in}{1.524621in}}{\pgfqpoint{6.390845in}{1.521348in}}{\pgfqpoint{6.399081in}{1.521348in}}%
\pgfpathclose%
\pgfusepath{stroke,fill}%
\end{pgfscope}%
\begin{pgfscope}%
\pgfpathrectangle{\pgfqpoint{0.894063in}{0.630000in}}{\pgfqpoint{6.713438in}{2.060556in}} %
\pgfusepath{clip}%
\pgfsetbuttcap%
\pgfsetroundjoin%
\definecolor{currentfill}{rgb}{0.000000,0.000000,0.000000}%
\pgfsetfillcolor{currentfill}%
\pgfsetlinewidth{1.003750pt}%
\definecolor{currentstroke}{rgb}{0.000000,0.000000,0.000000}%
\pgfsetstrokecolor{currentstroke}%
\pgfsetdash{}{0pt}%
\pgfpathmoveto{\pgfqpoint{4.787856in}{1.273923in}}%
\pgfpathcurveto{\pgfqpoint{4.796093in}{1.273923in}}{\pgfqpoint{4.803993in}{1.277195in}}{\pgfqpoint{4.809817in}{1.283019in}}%
\pgfpathcurveto{\pgfqpoint{4.815640in}{1.288843in}}{\pgfqpoint{4.818913in}{1.296743in}}{\pgfqpoint{4.818913in}{1.304979in}}%
\pgfpathcurveto{\pgfqpoint{4.818913in}{1.313215in}}{\pgfqpoint{4.815640in}{1.321115in}}{\pgfqpoint{4.809817in}{1.326939in}}%
\pgfpathcurveto{\pgfqpoint{4.803993in}{1.332763in}}{\pgfqpoint{4.796093in}{1.336036in}}{\pgfqpoint{4.787856in}{1.336036in}}%
\pgfpathcurveto{\pgfqpoint{4.779620in}{1.336036in}}{\pgfqpoint{4.771720in}{1.332763in}}{\pgfqpoint{4.765896in}{1.326939in}}%
\pgfpathcurveto{\pgfqpoint{4.760072in}{1.321115in}}{\pgfqpoint{4.756800in}{1.313215in}}{\pgfqpoint{4.756800in}{1.304979in}}%
\pgfpathcurveto{\pgfqpoint{4.756800in}{1.296743in}}{\pgfqpoint{4.760072in}{1.288843in}}{\pgfqpoint{4.765896in}{1.283019in}}%
\pgfpathcurveto{\pgfqpoint{4.771720in}{1.277195in}}{\pgfqpoint{4.779620in}{1.273923in}}{\pgfqpoint{4.787856in}{1.273923in}}%
\pgfpathclose%
\pgfusepath{stroke,fill}%
\end{pgfscope}%
\begin{pgfscope}%
\pgfpathrectangle{\pgfqpoint{0.894063in}{0.630000in}}{\pgfqpoint{6.713438in}{2.060556in}} %
\pgfusepath{clip}%
\pgfsetbuttcap%
\pgfsetroundjoin%
\definecolor{currentfill}{rgb}{0.000000,0.000000,0.000000}%
\pgfsetfillcolor{currentfill}%
\pgfsetlinewidth{1.003750pt}%
\definecolor{currentstroke}{rgb}{0.000000,0.000000,0.000000}%
\pgfsetstrokecolor{currentstroke}%
\pgfsetdash{}{0pt}%
\pgfpathmoveto{\pgfqpoint{4.922125in}{1.295429in}}%
\pgfpathcurveto{\pgfqpoint{4.930361in}{1.295429in}}{\pgfqpoint{4.938261in}{1.298701in}}{\pgfqpoint{4.944085in}{1.304525in}}%
\pgfpathcurveto{\pgfqpoint{4.949909in}{1.310349in}}{\pgfqpoint{4.953181in}{1.318249in}}{\pgfqpoint{4.953181in}{1.326485in}}%
\pgfpathcurveto{\pgfqpoint{4.953181in}{1.334722in}}{\pgfqpoint{4.949909in}{1.342622in}}{\pgfqpoint{4.944085in}{1.348446in}}%
\pgfpathcurveto{\pgfqpoint{4.938261in}{1.354270in}}{\pgfqpoint{4.930361in}{1.357542in}}{\pgfqpoint{4.922125in}{1.357542in}}%
\pgfpathcurveto{\pgfqpoint{4.913889in}{1.357542in}}{\pgfqpoint{4.905989in}{1.354270in}}{\pgfqpoint{4.900165in}{1.348446in}}%
\pgfpathcurveto{\pgfqpoint{4.894341in}{1.342622in}}{\pgfqpoint{4.891069in}{1.334722in}}{\pgfqpoint{4.891069in}{1.326485in}}%
\pgfpathcurveto{\pgfqpoint{4.891069in}{1.318249in}}{\pgfqpoint{4.894341in}{1.310349in}}{\pgfqpoint{4.900165in}{1.304525in}}%
\pgfpathcurveto{\pgfqpoint{4.905989in}{1.298701in}}{\pgfqpoint{4.913889in}{1.295429in}}{\pgfqpoint{4.922125in}{1.295429in}}%
\pgfpathclose%
\pgfusepath{stroke,fill}%
\end{pgfscope}%
\begin{pgfscope}%
\pgfpathrectangle{\pgfqpoint{0.894063in}{0.630000in}}{\pgfqpoint{6.713438in}{2.060556in}} %
\pgfusepath{clip}%
\pgfsetbuttcap%
\pgfsetroundjoin%
\definecolor{currentfill}{rgb}{0.000000,0.000000,0.000000}%
\pgfsetfillcolor{currentfill}%
\pgfsetlinewidth{1.003750pt}%
\definecolor{currentstroke}{rgb}{0.000000,0.000000,0.000000}%
\pgfsetstrokecolor{currentstroke}%
\pgfsetdash{}{0pt}%
\pgfpathmoveto{\pgfqpoint{6.130544in}{1.494390in}}%
\pgfpathcurveto{\pgfqpoint{6.138780in}{1.494390in}}{\pgfqpoint{6.146680in}{1.497663in}}{\pgfqpoint{6.152504in}{1.503487in}}%
\pgfpathcurveto{\pgfqpoint{6.158328in}{1.509310in}}{\pgfqpoint{6.161600in}{1.517211in}}{\pgfqpoint{6.161600in}{1.525447in}}%
\pgfpathcurveto{\pgfqpoint{6.161600in}{1.533683in}}{\pgfqpoint{6.158328in}{1.541583in}}{\pgfqpoint{6.152504in}{1.547407in}}%
\pgfpathcurveto{\pgfqpoint{6.146680in}{1.553231in}}{\pgfqpoint{6.138780in}{1.556503in}}{\pgfqpoint{6.130544in}{1.556503in}}%
\pgfpathcurveto{\pgfqpoint{6.122307in}{1.556503in}}{\pgfqpoint{6.114407in}{1.553231in}}{\pgfqpoint{6.108583in}{1.547407in}}%
\pgfpathcurveto{\pgfqpoint{6.102760in}{1.541583in}}{\pgfqpoint{6.099487in}{1.533683in}}{\pgfqpoint{6.099487in}{1.525447in}}%
\pgfpathcurveto{\pgfqpoint{6.099487in}{1.517211in}}{\pgfqpoint{6.102760in}{1.509310in}}{\pgfqpoint{6.108583in}{1.503487in}}%
\pgfpathcurveto{\pgfqpoint{6.114407in}{1.497663in}}{\pgfqpoint{6.122307in}{1.494390in}}{\pgfqpoint{6.130544in}{1.494390in}}%
\pgfpathclose%
\pgfusepath{stroke,fill}%
\end{pgfscope}%
\begin{pgfscope}%
\pgfpathrectangle{\pgfqpoint{0.894063in}{0.630000in}}{\pgfqpoint{6.713438in}{2.060556in}} %
\pgfusepath{clip}%
\pgfsetbuttcap%
\pgfsetroundjoin%
\definecolor{currentfill}{rgb}{0.000000,0.000000,0.000000}%
\pgfsetfillcolor{currentfill}%
\pgfsetlinewidth{1.003750pt}%
\definecolor{currentstroke}{rgb}{0.000000,0.000000,0.000000}%
\pgfsetstrokecolor{currentstroke}%
\pgfsetdash{}{0pt}%
\pgfpathmoveto{\pgfqpoint{5.727738in}{1.428294in}}%
\pgfpathcurveto{\pgfqpoint{5.735974in}{1.428294in}}{\pgfqpoint{5.743874in}{1.431566in}}{\pgfqpoint{5.749698in}{1.437390in}}%
\pgfpathcurveto{\pgfqpoint{5.755522in}{1.443214in}}{\pgfqpoint{5.758794in}{1.451114in}}{\pgfqpoint{5.758794in}{1.459350in}}%
\pgfpathcurveto{\pgfqpoint{5.758794in}{1.467586in}}{\pgfqpoint{5.755522in}{1.475486in}}{\pgfqpoint{5.749698in}{1.481310in}}%
\pgfpathcurveto{\pgfqpoint{5.743874in}{1.487134in}}{\pgfqpoint{5.735974in}{1.490407in}}{\pgfqpoint{5.727738in}{1.490407in}}%
\pgfpathcurveto{\pgfqpoint{5.719501in}{1.490407in}}{\pgfqpoint{5.711601in}{1.487134in}}{\pgfqpoint{5.705777in}{1.481310in}}%
\pgfpathcurveto{\pgfqpoint{5.699953in}{1.475486in}}{\pgfqpoint{5.696681in}{1.467586in}}{\pgfqpoint{5.696681in}{1.459350in}}%
\pgfpathcurveto{\pgfqpoint{5.696681in}{1.451114in}}{\pgfqpoint{5.699953in}{1.443214in}}{\pgfqpoint{5.705777in}{1.437390in}}%
\pgfpathcurveto{\pgfqpoint{5.711601in}{1.431566in}}{\pgfqpoint{5.719501in}{1.428294in}}{\pgfqpoint{5.727738in}{1.428294in}}%
\pgfpathclose%
\pgfusepath{stroke,fill}%
\end{pgfscope}%
\begin{pgfscope}%
\pgfpathrectangle{\pgfqpoint{0.894063in}{0.630000in}}{\pgfqpoint{6.713438in}{2.060556in}} %
\pgfusepath{clip}%
\pgfsetbuttcap%
\pgfsetroundjoin%
\definecolor{currentfill}{rgb}{0.000000,0.000000,0.000000}%
\pgfsetfillcolor{currentfill}%
\pgfsetlinewidth{1.003750pt}%
\definecolor{currentstroke}{rgb}{0.000000,0.000000,0.000000}%
\pgfsetstrokecolor{currentstroke}%
\pgfsetdash{}{0pt}%
\pgfpathmoveto{\pgfqpoint{1.028331in}{0.673965in}}%
\pgfpathcurveto{\pgfqpoint{1.036568in}{0.673965in}}{\pgfqpoint{1.044468in}{0.677238in}}{\pgfqpoint{1.050292in}{0.683062in}}%
\pgfpathcurveto{\pgfqpoint{1.056115in}{0.688886in}}{\pgfqpoint{1.059388in}{0.696786in}}{\pgfqpoint{1.059388in}{0.705022in}}%
\pgfpathcurveto{\pgfqpoint{1.059388in}{0.713258in}}{\pgfqpoint{1.056115in}{0.721158in}}{\pgfqpoint{1.050292in}{0.726982in}}%
\pgfpathcurveto{\pgfqpoint{1.044468in}{0.732806in}}{\pgfqpoint{1.036568in}{0.736078in}}{\pgfqpoint{1.028331in}{0.736078in}}%
\pgfpathcurveto{\pgfqpoint{1.020095in}{0.736078in}}{\pgfqpoint{1.012195in}{0.732806in}}{\pgfqpoint{1.006371in}{0.726982in}}%
\pgfpathcurveto{\pgfqpoint{1.000547in}{0.721158in}}{\pgfqpoint{0.997275in}{0.713258in}}{\pgfqpoint{0.997275in}{0.705022in}}%
\pgfpathcurveto{\pgfqpoint{0.997275in}{0.696786in}}{\pgfqpoint{1.000547in}{0.688886in}}{\pgfqpoint{1.006371in}{0.683062in}}%
\pgfpathcurveto{\pgfqpoint{1.012195in}{0.677238in}}{\pgfqpoint{1.020095in}{0.673965in}}{\pgfqpoint{1.028331in}{0.673965in}}%
\pgfpathclose%
\pgfusepath{stroke,fill}%
\end{pgfscope}%
\begin{pgfscope}%
\pgfpathrectangle{\pgfqpoint{0.894063in}{0.630000in}}{\pgfqpoint{6.713438in}{2.060556in}} %
\pgfusepath{clip}%
\pgfsetbuttcap%
\pgfsetroundjoin%
\definecolor{currentfill}{rgb}{0.000000,0.000000,0.000000}%
\pgfsetfillcolor{currentfill}%
\pgfsetlinewidth{1.003750pt}%
\definecolor{currentstroke}{rgb}{0.000000,0.000000,0.000000}%
\pgfsetstrokecolor{currentstroke}%
\pgfsetdash{}{0pt}%
\pgfpathmoveto{\pgfqpoint{5.324931in}{1.354561in}}%
\pgfpathcurveto{\pgfqpoint{5.333168in}{1.354561in}}{\pgfqpoint{5.341068in}{1.357833in}}{\pgfqpoint{5.346892in}{1.363657in}}%
\pgfpathcurveto{\pgfqpoint{5.352715in}{1.369481in}}{\pgfqpoint{5.355988in}{1.377381in}}{\pgfqpoint{5.355988in}{1.385617in}}%
\pgfpathcurveto{\pgfqpoint{5.355988in}{1.393854in}}{\pgfqpoint{5.352715in}{1.401754in}}{\pgfqpoint{5.346892in}{1.407578in}}%
\pgfpathcurveto{\pgfqpoint{5.341068in}{1.413402in}}{\pgfqpoint{5.333168in}{1.416674in}}{\pgfqpoint{5.324931in}{1.416674in}}%
\pgfpathcurveto{\pgfqpoint{5.316695in}{1.416674in}}{\pgfqpoint{5.308795in}{1.413402in}}{\pgfqpoint{5.302971in}{1.407578in}}%
\pgfpathcurveto{\pgfqpoint{5.297147in}{1.401754in}}{\pgfqpoint{5.293875in}{1.393854in}}{\pgfqpoint{5.293875in}{1.385617in}}%
\pgfpathcurveto{\pgfqpoint{5.293875in}{1.377381in}}{\pgfqpoint{5.297147in}{1.369481in}}{\pgfqpoint{5.302971in}{1.363657in}}%
\pgfpathcurveto{\pgfqpoint{5.308795in}{1.357833in}}{\pgfqpoint{5.316695in}{1.354561in}}{\pgfqpoint{5.324931in}{1.354561in}}%
\pgfpathclose%
\pgfusepath{stroke,fill}%
\end{pgfscope}%
\begin{pgfscope}%
\pgfpathrectangle{\pgfqpoint{0.894063in}{0.630000in}}{\pgfqpoint{6.713438in}{2.060556in}} %
\pgfusepath{clip}%
\pgfsetbuttcap%
\pgfsetroundjoin%
\definecolor{currentfill}{rgb}{0.000000,0.000000,0.000000}%
\pgfsetfillcolor{currentfill}%
\pgfsetlinewidth{1.003750pt}%
\definecolor{currentstroke}{rgb}{0.000000,0.000000,0.000000}%
\pgfsetstrokecolor{currentstroke}%
\pgfsetdash{}{0pt}%
\pgfpathmoveto{\pgfqpoint{7.338963in}{1.670091in}}%
\pgfpathcurveto{\pgfqpoint{7.347199in}{1.670091in}}{\pgfqpoint{7.355099in}{1.673363in}}{\pgfqpoint{7.360923in}{1.679187in}}%
\pgfpathcurveto{\pgfqpoint{7.366747in}{1.685011in}}{\pgfqpoint{7.370019in}{1.692911in}}{\pgfqpoint{7.370019in}{1.701147in}}%
\pgfpathcurveto{\pgfqpoint{7.370019in}{1.709384in}}{\pgfqpoint{7.366747in}{1.717284in}}{\pgfqpoint{7.360923in}{1.723108in}}%
\pgfpathcurveto{\pgfqpoint{7.355099in}{1.728932in}}{\pgfqpoint{7.347199in}{1.732204in}}{\pgfqpoint{7.338963in}{1.732204in}}%
\pgfpathcurveto{\pgfqpoint{7.330726in}{1.732204in}}{\pgfqpoint{7.322826in}{1.728932in}}{\pgfqpoint{7.317002in}{1.723108in}}%
\pgfpathcurveto{\pgfqpoint{7.311178in}{1.717284in}}{\pgfqpoint{7.307906in}{1.709384in}}{\pgfqpoint{7.307906in}{1.701147in}}%
\pgfpathcurveto{\pgfqpoint{7.307906in}{1.692911in}}{\pgfqpoint{7.311178in}{1.685011in}}{\pgfqpoint{7.317002in}{1.679187in}}%
\pgfpathcurveto{\pgfqpoint{7.322826in}{1.673363in}}{\pgfqpoint{7.330726in}{1.670091in}}{\pgfqpoint{7.338963in}{1.670091in}}%
\pgfpathclose%
\pgfusepath{stroke,fill}%
\end{pgfscope}%
\begin{pgfscope}%
\pgfpathrectangle{\pgfqpoint{0.894063in}{0.630000in}}{\pgfqpoint{6.713438in}{2.060556in}} %
\pgfusepath{clip}%
\pgfsetbuttcap%
\pgfsetroundjoin%
\definecolor{currentfill}{rgb}{0.000000,0.000000,0.000000}%
\pgfsetfillcolor{currentfill}%
\pgfsetlinewidth{1.003750pt}%
\definecolor{currentstroke}{rgb}{0.000000,0.000000,0.000000}%
\pgfsetstrokecolor{currentstroke}%
\pgfsetdash{}{0pt}%
\pgfpathmoveto{\pgfqpoint{7.204694in}{1.646601in}}%
\pgfpathcurveto{\pgfqpoint{7.212930in}{1.646601in}}{\pgfqpoint{7.220830in}{1.649873in}}{\pgfqpoint{7.226654in}{1.655697in}}%
\pgfpathcurveto{\pgfqpoint{7.232478in}{1.661521in}}{\pgfqpoint{7.235750in}{1.669421in}}{\pgfqpoint{7.235750in}{1.677657in}}%
\pgfpathcurveto{\pgfqpoint{7.235750in}{1.685893in}}{\pgfqpoint{7.232478in}{1.693793in}}{\pgfqpoint{7.226654in}{1.699617in}}%
\pgfpathcurveto{\pgfqpoint{7.220830in}{1.705441in}}{\pgfqpoint{7.212930in}{1.708714in}}{\pgfqpoint{7.204694in}{1.708714in}}%
\pgfpathcurveto{\pgfqpoint{7.196457in}{1.708714in}}{\pgfqpoint{7.188557in}{1.705441in}}{\pgfqpoint{7.182733in}{1.699617in}}%
\pgfpathcurveto{\pgfqpoint{7.176910in}{1.693793in}}{\pgfqpoint{7.173637in}{1.685893in}}{\pgfqpoint{7.173637in}{1.677657in}}%
\pgfpathcurveto{\pgfqpoint{7.173637in}{1.669421in}}{\pgfqpoint{7.176910in}{1.661521in}}{\pgfqpoint{7.182733in}{1.655697in}}%
\pgfpathcurveto{\pgfqpoint{7.188557in}{1.649873in}}{\pgfqpoint{7.196457in}{1.646601in}}{\pgfqpoint{7.204694in}{1.646601in}}%
\pgfpathclose%
\pgfusepath{stroke,fill}%
\end{pgfscope}%
\begin{pgfscope}%
\pgfpathrectangle{\pgfqpoint{0.894063in}{0.630000in}}{\pgfqpoint{6.713438in}{2.060556in}} %
\pgfusepath{clip}%
\pgfsetbuttcap%
\pgfsetroundjoin%
\definecolor{currentfill}{rgb}{0.000000,0.000000,0.000000}%
\pgfsetfillcolor{currentfill}%
\pgfsetlinewidth{1.003750pt}%
\definecolor{currentstroke}{rgb}{0.000000,0.000000,0.000000}%
\pgfsetstrokecolor{currentstroke}%
\pgfsetdash{}{0pt}%
\pgfpathmoveto{\pgfqpoint{6.264813in}{1.498458in}}%
\pgfpathcurveto{\pgfqpoint{6.273049in}{1.498458in}}{\pgfqpoint{6.280949in}{1.501731in}}{\pgfqpoint{6.286773in}{1.507555in}}%
\pgfpathcurveto{\pgfqpoint{6.292597in}{1.513379in}}{\pgfqpoint{6.295869in}{1.521279in}}{\pgfqpoint{6.295869in}{1.529515in}}%
\pgfpathcurveto{\pgfqpoint{6.295869in}{1.537751in}}{\pgfqpoint{6.292597in}{1.545651in}}{\pgfqpoint{6.286773in}{1.551475in}}%
\pgfpathcurveto{\pgfqpoint{6.280949in}{1.557299in}}{\pgfqpoint{6.273049in}{1.560571in}}{\pgfqpoint{6.264813in}{1.560571in}}%
\pgfpathcurveto{\pgfqpoint{6.256576in}{1.560571in}}{\pgfqpoint{6.248676in}{1.557299in}}{\pgfqpoint{6.242852in}{1.551475in}}%
\pgfpathcurveto{\pgfqpoint{6.237028in}{1.545651in}}{\pgfqpoint{6.233756in}{1.537751in}}{\pgfqpoint{6.233756in}{1.529515in}}%
\pgfpathcurveto{\pgfqpoint{6.233756in}{1.521279in}}{\pgfqpoint{6.237028in}{1.513379in}}{\pgfqpoint{6.242852in}{1.507555in}}%
\pgfpathcurveto{\pgfqpoint{6.248676in}{1.501731in}}{\pgfqpoint{6.256576in}{1.498458in}}{\pgfqpoint{6.264813in}{1.498458in}}%
\pgfpathclose%
\pgfusepath{stroke,fill}%
\end{pgfscope}%
\begin{pgfscope}%
\pgfpathrectangle{\pgfqpoint{0.894063in}{0.630000in}}{\pgfqpoint{6.713438in}{2.060556in}} %
\pgfusepath{clip}%
\pgfsetbuttcap%
\pgfsetroundjoin%
\definecolor{currentfill}{rgb}{0.000000,0.000000,0.000000}%
\pgfsetfillcolor{currentfill}%
\pgfsetlinewidth{1.003750pt}%
\definecolor{currentstroke}{rgb}{0.000000,0.000000,0.000000}%
\pgfsetstrokecolor{currentstroke}%
\pgfsetdash{}{0pt}%
\pgfpathmoveto{\pgfqpoint{7.473231in}{1.689384in}}%
\pgfpathcurveto{\pgfqpoint{7.481468in}{1.689384in}}{\pgfqpoint{7.489368in}{1.692656in}}{\pgfqpoint{7.495192in}{1.698480in}}%
\pgfpathcurveto{\pgfqpoint{7.501015in}{1.704304in}}{\pgfqpoint{7.504288in}{1.712204in}}{\pgfqpoint{7.504288in}{1.720440in}}%
\pgfpathcurveto{\pgfqpoint{7.504288in}{1.728676in}}{\pgfqpoint{7.501015in}{1.736576in}}{\pgfqpoint{7.495192in}{1.742400in}}%
\pgfpathcurveto{\pgfqpoint{7.489368in}{1.748224in}}{\pgfqpoint{7.481468in}{1.751497in}}{\pgfqpoint{7.473231in}{1.751497in}}%
\pgfpathcurveto{\pgfqpoint{7.464995in}{1.751497in}}{\pgfqpoint{7.457095in}{1.748224in}}{\pgfqpoint{7.451271in}{1.742400in}}%
\pgfpathcurveto{\pgfqpoint{7.445447in}{1.736576in}}{\pgfqpoint{7.442175in}{1.728676in}}{\pgfqpoint{7.442175in}{1.720440in}}%
\pgfpathcurveto{\pgfqpoint{7.442175in}{1.712204in}}{\pgfqpoint{7.445447in}{1.704304in}}{\pgfqpoint{7.451271in}{1.698480in}}%
\pgfpathcurveto{\pgfqpoint{7.457095in}{1.692656in}}{\pgfqpoint{7.464995in}{1.689384in}}{\pgfqpoint{7.473231in}{1.689384in}}%
\pgfpathclose%
\pgfusepath{stroke,fill}%
\end{pgfscope}%
\begin{pgfscope}%
\pgfpathrectangle{\pgfqpoint{0.894063in}{0.630000in}}{\pgfqpoint{6.713438in}{2.060556in}} %
\pgfusepath{clip}%
\pgfsetbuttcap%
\pgfsetroundjoin%
\definecolor{currentfill}{rgb}{0.000000,0.000000,0.000000}%
\pgfsetfillcolor{currentfill}%
\pgfsetlinewidth{1.003750pt}%
\definecolor{currentstroke}{rgb}{0.000000,0.000000,0.000000}%
\pgfsetstrokecolor{currentstroke}%
\pgfsetdash{}{0pt}%
\pgfpathmoveto{\pgfqpoint{5.056394in}{1.314951in}}%
\pgfpathcurveto{\pgfqpoint{5.064630in}{1.314951in}}{\pgfqpoint{5.072530in}{1.318224in}}{\pgfqpoint{5.078354in}{1.324047in}}%
\pgfpathcurveto{\pgfqpoint{5.084178in}{1.329871in}}{\pgfqpoint{5.087450in}{1.337771in}}{\pgfqpoint{5.087450in}{1.346008in}}%
\pgfpathcurveto{\pgfqpoint{5.087450in}{1.354244in}}{\pgfqpoint{5.084178in}{1.362144in}}{\pgfqpoint{5.078354in}{1.367968in}}%
\pgfpathcurveto{\pgfqpoint{5.072530in}{1.373792in}}{\pgfqpoint{5.064630in}{1.377064in}}{\pgfqpoint{5.056394in}{1.377064in}}%
\pgfpathcurveto{\pgfqpoint{5.048157in}{1.377064in}}{\pgfqpoint{5.040257in}{1.373792in}}{\pgfqpoint{5.034433in}{1.367968in}}%
\pgfpathcurveto{\pgfqpoint{5.028610in}{1.362144in}}{\pgfqpoint{5.025337in}{1.354244in}}{\pgfqpoint{5.025337in}{1.346008in}}%
\pgfpathcurveto{\pgfqpoint{5.025337in}{1.337771in}}{\pgfqpoint{5.028610in}{1.329871in}}{\pgfqpoint{5.034433in}{1.324047in}}%
\pgfpathcurveto{\pgfqpoint{5.040257in}{1.318224in}}{\pgfqpoint{5.048157in}{1.314951in}}{\pgfqpoint{5.056394in}{1.314951in}}%
\pgfpathclose%
\pgfusepath{stroke,fill}%
\end{pgfscope}%
\begin{pgfscope}%
\pgfpathrectangle{\pgfqpoint{0.894063in}{0.630000in}}{\pgfqpoint{6.713438in}{2.060556in}} %
\pgfusepath{clip}%
\pgfsetbuttcap%
\pgfsetroundjoin%
\definecolor{currentfill}{rgb}{0.000000,0.000000,0.000000}%
\pgfsetfillcolor{currentfill}%
\pgfsetlinewidth{1.003750pt}%
\definecolor{currentstroke}{rgb}{0.000000,0.000000,0.000000}%
\pgfsetstrokecolor{currentstroke}%
\pgfsetdash{}{0pt}%
\pgfpathmoveto{\pgfqpoint{2.908094in}{0.959706in}}%
\pgfpathcurveto{\pgfqpoint{2.916330in}{0.959706in}}{\pgfqpoint{2.924230in}{0.962978in}}{\pgfqpoint{2.930054in}{0.968802in}}%
\pgfpathcurveto{\pgfqpoint{2.935878in}{0.974626in}}{\pgfqpoint{2.939150in}{0.982526in}}{\pgfqpoint{2.939150in}{0.990762in}}%
\pgfpathcurveto{\pgfqpoint{2.939150in}{0.998998in}}{\pgfqpoint{2.935878in}{1.006898in}}{\pgfqpoint{2.930054in}{1.012722in}}%
\pgfpathcurveto{\pgfqpoint{2.924230in}{1.018546in}}{\pgfqpoint{2.916330in}{1.021819in}}{\pgfqpoint{2.908094in}{1.021819in}}%
\pgfpathcurveto{\pgfqpoint{2.899857in}{1.021819in}}{\pgfqpoint{2.891957in}{1.018546in}}{\pgfqpoint{2.886133in}{1.012722in}}%
\pgfpathcurveto{\pgfqpoint{2.880310in}{1.006898in}}{\pgfqpoint{2.877037in}{0.998998in}}{\pgfqpoint{2.877037in}{0.990762in}}%
\pgfpathcurveto{\pgfqpoint{2.877037in}{0.982526in}}{\pgfqpoint{2.880310in}{0.974626in}}{\pgfqpoint{2.886133in}{0.968802in}}%
\pgfpathcurveto{\pgfqpoint{2.891957in}{0.962978in}}{\pgfqpoint{2.899857in}{0.959706in}}{\pgfqpoint{2.908094in}{0.959706in}}%
\pgfpathclose%
\pgfusepath{stroke,fill}%
\end{pgfscope}%
\begin{pgfscope}%
\pgfpathrectangle{\pgfqpoint{0.894063in}{0.630000in}}{\pgfqpoint{6.713438in}{2.060556in}} %
\pgfusepath{clip}%
\pgfsetbuttcap%
\pgfsetroundjoin%
\definecolor{currentfill}{rgb}{0.000000,0.000000,0.000000}%
\pgfsetfillcolor{currentfill}%
\pgfsetlinewidth{1.003750pt}%
\definecolor{currentstroke}{rgb}{0.000000,0.000000,0.000000}%
\pgfsetstrokecolor{currentstroke}%
\pgfsetdash{}{0pt}%
\pgfpathmoveto{\pgfqpoint{3.445169in}{1.055875in}}%
\pgfpathcurveto{\pgfqpoint{3.453405in}{1.055875in}}{\pgfqpoint{3.461305in}{1.059147in}}{\pgfqpoint{3.467129in}{1.064971in}}%
\pgfpathcurveto{\pgfqpoint{3.472953in}{1.070795in}}{\pgfqpoint{3.476225in}{1.078695in}}{\pgfqpoint{3.476225in}{1.086931in}}%
\pgfpathcurveto{\pgfqpoint{3.476225in}{1.095167in}}{\pgfqpoint{3.472953in}{1.103067in}}{\pgfqpoint{3.467129in}{1.108891in}}%
\pgfpathcurveto{\pgfqpoint{3.461305in}{1.114715in}}{\pgfqpoint{3.453405in}{1.117988in}}{\pgfqpoint{3.445169in}{1.117988in}}%
\pgfpathcurveto{\pgfqpoint{3.436932in}{1.117988in}}{\pgfqpoint{3.429032in}{1.114715in}}{\pgfqpoint{3.423208in}{1.108891in}}%
\pgfpathcurveto{\pgfqpoint{3.417385in}{1.103067in}}{\pgfqpoint{3.414112in}{1.095167in}}{\pgfqpoint{3.414112in}{1.086931in}}%
\pgfpathcurveto{\pgfqpoint{3.414112in}{1.078695in}}{\pgfqpoint{3.417385in}{1.070795in}}{\pgfqpoint{3.423208in}{1.064971in}}%
\pgfpathcurveto{\pgfqpoint{3.429032in}{1.059147in}}{\pgfqpoint{3.436932in}{1.055875in}}{\pgfqpoint{3.445169in}{1.055875in}}%
\pgfpathclose%
\pgfusepath{stroke,fill}%
\end{pgfscope}%
\begin{pgfscope}%
\pgfpathrectangle{\pgfqpoint{0.894063in}{0.630000in}}{\pgfqpoint{6.713438in}{2.060556in}} %
\pgfusepath{clip}%
\pgfsetbuttcap%
\pgfsetroundjoin%
\definecolor{currentfill}{rgb}{0.000000,0.000000,0.000000}%
\pgfsetfillcolor{currentfill}%
\pgfsetlinewidth{1.003750pt}%
\definecolor{currentstroke}{rgb}{0.000000,0.000000,0.000000}%
\pgfsetstrokecolor{currentstroke}%
\pgfsetdash{}{0pt}%
\pgfpathmoveto{\pgfqpoint{4.116513in}{1.175010in}}%
\pgfpathcurveto{\pgfqpoint{4.124749in}{1.175010in}}{\pgfqpoint{4.132649in}{1.178282in}}{\pgfqpoint{4.138473in}{1.184106in}}%
\pgfpathcurveto{\pgfqpoint{4.144297in}{1.189930in}}{\pgfqpoint{4.147569in}{1.197830in}}{\pgfqpoint{4.147569in}{1.206067in}}%
\pgfpathcurveto{\pgfqpoint{4.147569in}{1.214303in}}{\pgfqpoint{4.144297in}{1.222203in}}{\pgfqpoint{4.138473in}{1.228027in}}%
\pgfpathcurveto{\pgfqpoint{4.132649in}{1.233851in}}{\pgfqpoint{4.124749in}{1.237123in}}{\pgfqpoint{4.116513in}{1.237123in}}%
\pgfpathcurveto{\pgfqpoint{4.108276in}{1.237123in}}{\pgfqpoint{4.100376in}{1.233851in}}{\pgfqpoint{4.094552in}{1.228027in}}%
\pgfpathcurveto{\pgfqpoint{4.088728in}{1.222203in}}{\pgfqpoint{4.085456in}{1.214303in}}{\pgfqpoint{4.085456in}{1.206067in}}%
\pgfpathcurveto{\pgfqpoint{4.085456in}{1.197830in}}{\pgfqpoint{4.088728in}{1.189930in}}{\pgfqpoint{4.094552in}{1.184106in}}%
\pgfpathcurveto{\pgfqpoint{4.100376in}{1.178282in}}{\pgfqpoint{4.108276in}{1.175010in}}{\pgfqpoint{4.116513in}{1.175010in}}%
\pgfpathclose%
\pgfusepath{stroke,fill}%
\end{pgfscope}%
\begin{pgfscope}%
\pgfpathrectangle{\pgfqpoint{0.894063in}{0.630000in}}{\pgfqpoint{6.713438in}{2.060556in}} %
\pgfusepath{clip}%
\pgfsetbuttcap%
\pgfsetroundjoin%
\definecolor{currentfill}{rgb}{0.000000,0.000000,0.000000}%
\pgfsetfillcolor{currentfill}%
\pgfsetlinewidth{1.003750pt}%
\definecolor{currentstroke}{rgb}{0.000000,0.000000,0.000000}%
\pgfsetstrokecolor{currentstroke}%
\pgfsetdash{}{0pt}%
\pgfpathmoveto{\pgfqpoint{1.431138in}{0.747233in}}%
\pgfpathcurveto{\pgfqpoint{1.439374in}{0.747233in}}{\pgfqpoint{1.447274in}{0.750505in}}{\pgfqpoint{1.453098in}{0.756329in}}%
\pgfpathcurveto{\pgfqpoint{1.458922in}{0.762153in}}{\pgfqpoint{1.462194in}{0.770053in}}{\pgfqpoint{1.462194in}{0.778289in}}%
\pgfpathcurveto{\pgfqpoint{1.462194in}{0.786526in}}{\pgfqpoint{1.458922in}{0.794426in}}{\pgfqpoint{1.453098in}{0.800250in}}%
\pgfpathcurveto{\pgfqpoint{1.447274in}{0.806074in}}{\pgfqpoint{1.439374in}{0.809346in}}{\pgfqpoint{1.431138in}{0.809346in}}%
\pgfpathcurveto{\pgfqpoint{1.422901in}{0.809346in}}{\pgfqpoint{1.415001in}{0.806074in}}{\pgfqpoint{1.409177in}{0.800250in}}%
\pgfpathcurveto{\pgfqpoint{1.403353in}{0.794426in}}{\pgfqpoint{1.400081in}{0.786526in}}{\pgfqpoint{1.400081in}{0.778289in}}%
\pgfpathcurveto{\pgfqpoint{1.400081in}{0.770053in}}{\pgfqpoint{1.403353in}{0.762153in}}{\pgfqpoint{1.409177in}{0.756329in}}%
\pgfpathcurveto{\pgfqpoint{1.415001in}{0.750505in}}{\pgfqpoint{1.422901in}{0.747233in}}{\pgfqpoint{1.431138in}{0.747233in}}%
\pgfpathclose%
\pgfusepath{stroke,fill}%
\end{pgfscope}%
\begin{pgfscope}%
\pgfpathrectangle{\pgfqpoint{0.894063in}{0.630000in}}{\pgfqpoint{6.713438in}{2.060556in}} %
\pgfusepath{clip}%
\pgfsetbuttcap%
\pgfsetroundjoin%
\definecolor{currentfill}{rgb}{0.000000,0.000000,0.000000}%
\pgfsetfillcolor{currentfill}%
\pgfsetlinewidth{1.003750pt}%
\definecolor{currentstroke}{rgb}{0.000000,0.000000,0.000000}%
\pgfsetstrokecolor{currentstroke}%
\pgfsetdash{}{0pt}%
\pgfpathmoveto{\pgfqpoint{2.773825in}{0.946589in}}%
\pgfpathcurveto{\pgfqpoint{2.782061in}{0.946589in}}{\pgfqpoint{2.789961in}{0.949861in}}{\pgfqpoint{2.795785in}{0.955685in}}%
\pgfpathcurveto{\pgfqpoint{2.801609in}{0.961509in}}{\pgfqpoint{2.804881in}{0.969409in}}{\pgfqpoint{2.804881in}{0.977645in}}%
\pgfpathcurveto{\pgfqpoint{2.804881in}{0.985881in}}{\pgfqpoint{2.801609in}{0.993781in}}{\pgfqpoint{2.795785in}{0.999605in}}%
\pgfpathcurveto{\pgfqpoint{2.789961in}{1.005429in}}{\pgfqpoint{2.782061in}{1.008702in}}{\pgfqpoint{2.773825in}{1.008702in}}%
\pgfpathcurveto{\pgfqpoint{2.765589in}{1.008702in}}{\pgfqpoint{2.757689in}{1.005429in}}{\pgfqpoint{2.751865in}{0.999605in}}%
\pgfpathcurveto{\pgfqpoint{2.746041in}{0.993781in}}{\pgfqpoint{2.742769in}{0.985881in}}{\pgfqpoint{2.742769in}{0.977645in}}%
\pgfpathcurveto{\pgfqpoint{2.742769in}{0.969409in}}{\pgfqpoint{2.746041in}{0.961509in}}{\pgfqpoint{2.751865in}{0.955685in}}%
\pgfpathcurveto{\pgfqpoint{2.757689in}{0.949861in}}{\pgfqpoint{2.765589in}{0.946589in}}{\pgfqpoint{2.773825in}{0.946589in}}%
\pgfpathclose%
\pgfusepath{stroke,fill}%
\end{pgfscope}%
\begin{pgfscope}%
\pgfpathrectangle{\pgfqpoint{0.894063in}{0.630000in}}{\pgfqpoint{6.713438in}{2.060556in}} %
\pgfusepath{clip}%
\pgfsetbuttcap%
\pgfsetroundjoin%
\definecolor{currentfill}{rgb}{0.000000,0.000000,0.000000}%
\pgfsetfillcolor{currentfill}%
\pgfsetlinewidth{1.003750pt}%
\definecolor{currentstroke}{rgb}{0.000000,0.000000,0.000000}%
\pgfsetstrokecolor{currentstroke}%
\pgfsetdash{}{0pt}%
\pgfpathmoveto{\pgfqpoint{1.565406in}{0.765354in}}%
\pgfpathcurveto{\pgfqpoint{1.573643in}{0.765354in}}{\pgfqpoint{1.581543in}{0.768626in}}{\pgfqpoint{1.587367in}{0.774450in}}%
\pgfpathcurveto{\pgfqpoint{1.593190in}{0.780274in}}{\pgfqpoint{1.596463in}{0.788174in}}{\pgfqpoint{1.596463in}{0.796410in}}%
\pgfpathcurveto{\pgfqpoint{1.596463in}{0.804647in}}{\pgfqpoint{1.593190in}{0.812547in}}{\pgfqpoint{1.587367in}{0.818371in}}%
\pgfpathcurveto{\pgfqpoint{1.581543in}{0.824195in}}{\pgfqpoint{1.573643in}{0.827467in}}{\pgfqpoint{1.565406in}{0.827467in}}%
\pgfpathcurveto{\pgfqpoint{1.557170in}{0.827467in}}{\pgfqpoint{1.549270in}{0.824195in}}{\pgfqpoint{1.543446in}{0.818371in}}%
\pgfpathcurveto{\pgfqpoint{1.537622in}{0.812547in}}{\pgfqpoint{1.534350in}{0.804647in}}{\pgfqpoint{1.534350in}{0.796410in}}%
\pgfpathcurveto{\pgfqpoint{1.534350in}{0.788174in}}{\pgfqpoint{1.537622in}{0.780274in}}{\pgfqpoint{1.543446in}{0.774450in}}%
\pgfpathcurveto{\pgfqpoint{1.549270in}{0.768626in}}{\pgfqpoint{1.557170in}{0.765354in}}{\pgfqpoint{1.565406in}{0.765354in}}%
\pgfpathclose%
\pgfusepath{stroke,fill}%
\end{pgfscope}%
\begin{pgfscope}%
\pgfpathrectangle{\pgfqpoint{0.894063in}{0.630000in}}{\pgfqpoint{6.713438in}{2.060556in}} %
\pgfusepath{clip}%
\pgfsetbuttcap%
\pgfsetroundjoin%
\definecolor{currentfill}{rgb}{0.000000,0.000000,0.000000}%
\pgfsetfillcolor{currentfill}%
\pgfsetlinewidth{1.003750pt}%
\definecolor{currentstroke}{rgb}{0.000000,0.000000,0.000000}%
\pgfsetstrokecolor{currentstroke}%
\pgfsetdash{}{0pt}%
\pgfpathmoveto{\pgfqpoint{4.250781in}{1.191653in}}%
\pgfpathcurveto{\pgfqpoint{4.259018in}{1.191653in}}{\pgfqpoint{4.266918in}{1.194926in}}{\pgfqpoint{4.272742in}{1.200750in}}%
\pgfpathcurveto{\pgfqpoint{4.278565in}{1.206574in}}{\pgfqpoint{4.281838in}{1.214474in}}{\pgfqpoint{4.281838in}{1.222710in}}%
\pgfpathcurveto{\pgfqpoint{4.281838in}{1.230946in}}{\pgfqpoint{4.278565in}{1.238846in}}{\pgfqpoint{4.272742in}{1.244670in}}%
\pgfpathcurveto{\pgfqpoint{4.266918in}{1.250494in}}{\pgfqpoint{4.259018in}{1.253766in}}{\pgfqpoint{4.250781in}{1.253766in}}%
\pgfpathcurveto{\pgfqpoint{4.242545in}{1.253766in}}{\pgfqpoint{4.234645in}{1.250494in}}{\pgfqpoint{4.228821in}{1.244670in}}%
\pgfpathcurveto{\pgfqpoint{4.222997in}{1.238846in}}{\pgfqpoint{4.219725in}{1.230946in}}{\pgfqpoint{4.219725in}{1.222710in}}%
\pgfpathcurveto{\pgfqpoint{4.219725in}{1.214474in}}{\pgfqpoint{4.222997in}{1.206574in}}{\pgfqpoint{4.228821in}{1.200750in}}%
\pgfpathcurveto{\pgfqpoint{4.234645in}{1.194926in}}{\pgfqpoint{4.242545in}{1.191653in}}{\pgfqpoint{4.250781in}{1.191653in}}%
\pgfpathclose%
\pgfusepath{stroke,fill}%
\end{pgfscope}%
\begin{pgfscope}%
\pgfpathrectangle{\pgfqpoint{0.894063in}{0.630000in}}{\pgfqpoint{6.713438in}{2.060556in}} %
\pgfusepath{clip}%
\pgfsetbuttcap%
\pgfsetroundjoin%
\definecolor{currentfill}{rgb}{0.000000,0.000000,0.000000}%
\pgfsetfillcolor{currentfill}%
\pgfsetlinewidth{1.003750pt}%
\definecolor{currentstroke}{rgb}{0.000000,0.000000,0.000000}%
\pgfsetstrokecolor{currentstroke}%
\pgfsetdash{}{0pt}%
\pgfpathmoveto{\pgfqpoint{3.847975in}{1.116249in}}%
\pgfpathcurveto{\pgfqpoint{3.856211in}{1.116249in}}{\pgfqpoint{3.864111in}{1.119521in}}{\pgfqpoint{3.869935in}{1.125345in}}%
\pgfpathcurveto{\pgfqpoint{3.875759in}{1.131169in}}{\pgfqpoint{3.879031in}{1.139069in}}{\pgfqpoint{3.879031in}{1.147305in}}%
\pgfpathcurveto{\pgfqpoint{3.879031in}{1.155542in}}{\pgfqpoint{3.875759in}{1.163442in}}{\pgfqpoint{3.869935in}{1.169266in}}%
\pgfpathcurveto{\pgfqpoint{3.864111in}{1.175090in}}{\pgfqpoint{3.856211in}{1.178362in}}{\pgfqpoint{3.847975in}{1.178362in}}%
\pgfpathcurveto{\pgfqpoint{3.839739in}{1.178362in}}{\pgfqpoint{3.831839in}{1.175090in}}{\pgfqpoint{3.826015in}{1.169266in}}%
\pgfpathcurveto{\pgfqpoint{3.820191in}{1.163442in}}{\pgfqpoint{3.816919in}{1.155542in}}{\pgfqpoint{3.816919in}{1.147305in}}%
\pgfpathcurveto{\pgfqpoint{3.816919in}{1.139069in}}{\pgfqpoint{3.820191in}{1.131169in}}{\pgfqpoint{3.826015in}{1.125345in}}%
\pgfpathcurveto{\pgfqpoint{3.831839in}{1.119521in}}{\pgfqpoint{3.839739in}{1.116249in}}{\pgfqpoint{3.847975in}{1.116249in}}%
\pgfpathclose%
\pgfusepath{stroke,fill}%
\end{pgfscope}%
\begin{pgfscope}%
\pgfpathrectangle{\pgfqpoint{0.894063in}{0.630000in}}{\pgfqpoint{6.713438in}{2.060556in}} %
\pgfusepath{clip}%
\pgfsetbuttcap%
\pgfsetroundjoin%
\definecolor{currentfill}{rgb}{0.000000,0.000000,0.000000}%
\pgfsetfillcolor{currentfill}%
\pgfsetlinewidth{1.003750pt}%
\definecolor{currentstroke}{rgb}{0.000000,0.000000,0.000000}%
\pgfsetstrokecolor{currentstroke}%
\pgfsetdash{}{0pt}%
\pgfpathmoveto{\pgfqpoint{7.607500in}{1.722382in}}%
\pgfpathcurveto{\pgfqpoint{7.615736in}{1.722382in}}{\pgfqpoint{7.623636in}{1.725654in}}{\pgfqpoint{7.629460in}{1.731478in}}%
\pgfpathcurveto{\pgfqpoint{7.635284in}{1.737302in}}{\pgfqpoint{7.638556in}{1.745202in}}{\pgfqpoint{7.638556in}{1.753438in}}%
\pgfpathcurveto{\pgfqpoint{7.638556in}{1.761675in}}{\pgfqpoint{7.635284in}{1.769575in}}{\pgfqpoint{7.629460in}{1.775399in}}%
\pgfpathcurveto{\pgfqpoint{7.623636in}{1.781223in}}{\pgfqpoint{7.615736in}{1.784495in}}{\pgfqpoint{7.607500in}{1.784495in}}%
\pgfpathcurveto{\pgfqpoint{7.599264in}{1.784495in}}{\pgfqpoint{7.591364in}{1.781223in}}{\pgfqpoint{7.585540in}{1.775399in}}%
\pgfpathcurveto{\pgfqpoint{7.579716in}{1.769575in}}{\pgfqpoint{7.576444in}{1.761675in}}{\pgfqpoint{7.576444in}{1.753438in}}%
\pgfpathcurveto{\pgfqpoint{7.576444in}{1.745202in}}{\pgfqpoint{7.579716in}{1.737302in}}{\pgfqpoint{7.585540in}{1.731478in}}%
\pgfpathcurveto{\pgfqpoint{7.591364in}{1.725654in}}{\pgfqpoint{7.599264in}{1.722382in}}{\pgfqpoint{7.607500in}{1.722382in}}%
\pgfpathclose%
\pgfusepath{stroke,fill}%
\end{pgfscope}%
\begin{pgfscope}%
\pgfpathrectangle{\pgfqpoint{0.894063in}{0.630000in}}{\pgfqpoint{6.713438in}{2.060556in}} %
\pgfusepath{clip}%
\pgfsetbuttcap%
\pgfsetroundjoin%
\definecolor{currentfill}{rgb}{0.000000,0.000000,0.000000}%
\pgfsetfillcolor{currentfill}%
\pgfsetlinewidth{1.003750pt}%
\definecolor{currentstroke}{rgb}{0.000000,0.000000,0.000000}%
\pgfsetstrokecolor{currentstroke}%
\pgfsetdash{}{0pt}%
\pgfpathmoveto{\pgfqpoint{4.385050in}{1.212942in}}%
\pgfpathcurveto{\pgfqpoint{4.393286in}{1.212942in}}{\pgfqpoint{4.401186in}{1.216214in}}{\pgfqpoint{4.407010in}{1.222038in}}%
\pgfpathcurveto{\pgfqpoint{4.412834in}{1.227862in}}{\pgfqpoint{4.416106in}{1.235762in}}{\pgfqpoint{4.416106in}{1.243998in}}%
\pgfpathcurveto{\pgfqpoint{4.416106in}{1.252235in}}{\pgfqpoint{4.412834in}{1.260135in}}{\pgfqpoint{4.407010in}{1.265959in}}%
\pgfpathcurveto{\pgfqpoint{4.401186in}{1.271783in}}{\pgfqpoint{4.393286in}{1.275055in}}{\pgfqpoint{4.385050in}{1.275055in}}%
\pgfpathcurveto{\pgfqpoint{4.376814in}{1.275055in}}{\pgfqpoint{4.368914in}{1.271783in}}{\pgfqpoint{4.363090in}{1.265959in}}%
\pgfpathcurveto{\pgfqpoint{4.357266in}{1.260135in}}{\pgfqpoint{4.353994in}{1.252235in}}{\pgfqpoint{4.353994in}{1.243998in}}%
\pgfpathcurveto{\pgfqpoint{4.353994in}{1.235762in}}{\pgfqpoint{4.357266in}{1.227862in}}{\pgfqpoint{4.363090in}{1.222038in}}%
\pgfpathcurveto{\pgfqpoint{4.368914in}{1.216214in}}{\pgfqpoint{4.376814in}{1.212942in}}{\pgfqpoint{4.385050in}{1.212942in}}%
\pgfpathclose%
\pgfusepath{stroke,fill}%
\end{pgfscope}%
\begin{pgfscope}%
\pgfpathrectangle{\pgfqpoint{0.894063in}{0.630000in}}{\pgfqpoint{6.713438in}{2.060556in}} %
\pgfusepath{clip}%
\pgfsetbuttcap%
\pgfsetroundjoin%
\definecolor{currentfill}{rgb}{0.000000,0.000000,0.000000}%
\pgfsetfillcolor{currentfill}%
\pgfsetlinewidth{1.003750pt}%
\definecolor{currentstroke}{rgb}{0.000000,0.000000,0.000000}%
\pgfsetstrokecolor{currentstroke}%
\pgfsetdash{}{0pt}%
\pgfpathmoveto{\pgfqpoint{6.533350in}{1.533912in}}%
\pgfpathcurveto{\pgfqpoint{6.541586in}{1.533912in}}{\pgfqpoint{6.549486in}{1.537184in}}{\pgfqpoint{6.555310in}{1.543008in}}%
\pgfpathcurveto{\pgfqpoint{6.561134in}{1.548832in}}{\pgfqpoint{6.564406in}{1.556732in}}{\pgfqpoint{6.564406in}{1.564968in}}%
\pgfpathcurveto{\pgfqpoint{6.564406in}{1.573205in}}{\pgfqpoint{6.561134in}{1.581105in}}{\pgfqpoint{6.555310in}{1.586929in}}%
\pgfpathcurveto{\pgfqpoint{6.549486in}{1.592752in}}{\pgfqpoint{6.541586in}{1.596025in}}{\pgfqpoint{6.533350in}{1.596025in}}%
\pgfpathcurveto{\pgfqpoint{6.525114in}{1.596025in}}{\pgfqpoint{6.517214in}{1.592752in}}{\pgfqpoint{6.511390in}{1.586929in}}%
\pgfpathcurveto{\pgfqpoint{6.505566in}{1.581105in}}{\pgfqpoint{6.502294in}{1.573205in}}{\pgfqpoint{6.502294in}{1.564968in}}%
\pgfpathcurveto{\pgfqpoint{6.502294in}{1.556732in}}{\pgfqpoint{6.505566in}{1.548832in}}{\pgfqpoint{6.511390in}{1.543008in}}%
\pgfpathcurveto{\pgfqpoint{6.517214in}{1.537184in}}{\pgfqpoint{6.525114in}{1.533912in}}{\pgfqpoint{6.533350in}{1.533912in}}%
\pgfpathclose%
\pgfusepath{stroke,fill}%
\end{pgfscope}%
\begin{pgfscope}%
\pgfpathrectangle{\pgfqpoint{0.894063in}{0.630000in}}{\pgfqpoint{6.713438in}{2.060556in}} %
\pgfusepath{clip}%
\pgfsetbuttcap%
\pgfsetroundjoin%
\definecolor{currentfill}{rgb}{0.000000,0.000000,0.000000}%
\pgfsetfillcolor{currentfill}%
\pgfsetlinewidth{1.003750pt}%
\definecolor{currentstroke}{rgb}{0.000000,0.000000,0.000000}%
\pgfsetstrokecolor{currentstroke}%
\pgfsetdash{}{0pt}%
\pgfpathmoveto{\pgfqpoint{1.296869in}{0.722536in}}%
\pgfpathcurveto{\pgfqpoint{1.305105in}{0.722536in}}{\pgfqpoint{1.313005in}{0.725808in}}{\pgfqpoint{1.318829in}{0.731632in}}%
\pgfpathcurveto{\pgfqpoint{1.324653in}{0.737456in}}{\pgfqpoint{1.327925in}{0.745356in}}{\pgfqpoint{1.327925in}{0.753592in}}%
\pgfpathcurveto{\pgfqpoint{1.327925in}{0.761828in}}{\pgfqpoint{1.324653in}{0.769728in}}{\pgfqpoint{1.318829in}{0.775552in}}%
\pgfpathcurveto{\pgfqpoint{1.313005in}{0.781376in}}{\pgfqpoint{1.305105in}{0.784649in}}{\pgfqpoint{1.296869in}{0.784649in}}%
\pgfpathcurveto{\pgfqpoint{1.288632in}{0.784649in}}{\pgfqpoint{1.280732in}{0.781376in}}{\pgfqpoint{1.274908in}{0.775552in}}%
\pgfpathcurveto{\pgfqpoint{1.269085in}{0.769728in}}{\pgfqpoint{1.265812in}{0.761828in}}{\pgfqpoint{1.265812in}{0.753592in}}%
\pgfpathcurveto{\pgfqpoint{1.265812in}{0.745356in}}{\pgfqpoint{1.269085in}{0.737456in}}{\pgfqpoint{1.274908in}{0.731632in}}%
\pgfpathcurveto{\pgfqpoint{1.280732in}{0.725808in}}{\pgfqpoint{1.288632in}{0.722536in}}{\pgfqpoint{1.296869in}{0.722536in}}%
\pgfpathclose%
\pgfusepath{stroke,fill}%
\end{pgfscope}%
\begin{pgfscope}%
\pgfpathrectangle{\pgfqpoint{0.894063in}{0.630000in}}{\pgfqpoint{6.713438in}{2.060556in}} %
\pgfusepath{clip}%
\pgfsetbuttcap%
\pgfsetroundjoin%
\definecolor{currentfill}{rgb}{0.000000,0.000000,0.000000}%
\pgfsetfillcolor{currentfill}%
\pgfsetlinewidth{1.003750pt}%
\definecolor{currentstroke}{rgb}{0.000000,0.000000,0.000000}%
\pgfsetstrokecolor{currentstroke}%
\pgfsetdash{}{0pt}%
\pgfpathmoveto{\pgfqpoint{4.519319in}{1.242832in}}%
\pgfpathcurveto{\pgfqpoint{4.527555in}{1.242832in}}{\pgfqpoint{4.535455in}{1.246104in}}{\pgfqpoint{4.541279in}{1.251928in}}%
\pgfpathcurveto{\pgfqpoint{4.547103in}{1.257752in}}{\pgfqpoint{4.550375in}{1.265652in}}{\pgfqpoint{4.550375in}{1.273888in}}%
\pgfpathcurveto{\pgfqpoint{4.550375in}{1.282125in}}{\pgfqpoint{4.547103in}{1.290025in}}{\pgfqpoint{4.541279in}{1.295849in}}%
\pgfpathcurveto{\pgfqpoint{4.535455in}{1.301672in}}{\pgfqpoint{4.527555in}{1.304945in}}{\pgfqpoint{4.519319in}{1.304945in}}%
\pgfpathcurveto{\pgfqpoint{4.511082in}{1.304945in}}{\pgfqpoint{4.503182in}{1.301672in}}{\pgfqpoint{4.497358in}{1.295849in}}%
\pgfpathcurveto{\pgfqpoint{4.491535in}{1.290025in}}{\pgfqpoint{4.488262in}{1.282125in}}{\pgfqpoint{4.488262in}{1.273888in}}%
\pgfpathcurveto{\pgfqpoint{4.488262in}{1.265652in}}{\pgfqpoint{4.491535in}{1.257752in}}{\pgfqpoint{4.497358in}{1.251928in}}%
\pgfpathcurveto{\pgfqpoint{4.503182in}{1.246104in}}{\pgfqpoint{4.511082in}{1.242832in}}{\pgfqpoint{4.519319in}{1.242832in}}%
\pgfpathclose%
\pgfusepath{stroke,fill}%
\end{pgfscope}%
\begin{pgfscope}%
\pgfpathrectangle{\pgfqpoint{0.894063in}{0.630000in}}{\pgfqpoint{6.713438in}{2.060556in}} %
\pgfusepath{clip}%
\pgfsetbuttcap%
\pgfsetroundjoin%
\definecolor{currentfill}{rgb}{0.000000,0.000000,0.000000}%
\pgfsetfillcolor{currentfill}%
\pgfsetlinewidth{1.003750pt}%
\definecolor{currentstroke}{rgb}{0.000000,0.000000,0.000000}%
\pgfsetstrokecolor{currentstroke}%
\pgfsetdash{}{0pt}%
\pgfpathmoveto{\pgfqpoint{2.505288in}{0.911224in}}%
\pgfpathcurveto{\pgfqpoint{2.513524in}{0.911224in}}{\pgfqpoint{2.521424in}{0.914496in}}{\pgfqpoint{2.527248in}{0.920320in}}%
\pgfpathcurveto{\pgfqpoint{2.533072in}{0.926144in}}{\pgfqpoint{2.536344in}{0.934044in}}{\pgfqpoint{2.536344in}{0.942280in}}%
\pgfpathcurveto{\pgfqpoint{2.536344in}{0.950516in}}{\pgfqpoint{2.533072in}{0.958416in}}{\pgfqpoint{2.527248in}{0.964240in}}%
\pgfpathcurveto{\pgfqpoint{2.521424in}{0.970064in}}{\pgfqpoint{2.513524in}{0.973337in}}{\pgfqpoint{2.505288in}{0.973337in}}%
\pgfpathcurveto{\pgfqpoint{2.497051in}{0.973337in}}{\pgfqpoint{2.489151in}{0.970064in}}{\pgfqpoint{2.483327in}{0.964240in}}%
\pgfpathcurveto{\pgfqpoint{2.477503in}{0.958416in}}{\pgfqpoint{2.474231in}{0.950516in}}{\pgfqpoint{2.474231in}{0.942280in}}%
\pgfpathcurveto{\pgfqpoint{2.474231in}{0.934044in}}{\pgfqpoint{2.477503in}{0.926144in}}{\pgfqpoint{2.483327in}{0.920320in}}%
\pgfpathcurveto{\pgfqpoint{2.489151in}{0.914496in}}{\pgfqpoint{2.497051in}{0.911224in}}{\pgfqpoint{2.505288in}{0.911224in}}%
\pgfpathclose%
\pgfusepath{stroke,fill}%
\end{pgfscope}%
\begin{pgfscope}%
\pgfpathrectangle{\pgfqpoint{0.894063in}{0.630000in}}{\pgfqpoint{6.713438in}{2.060556in}} %
\pgfusepath{clip}%
\pgfsetbuttcap%
\pgfsetroundjoin%
\definecolor{currentfill}{rgb}{0.000000,0.000000,0.000000}%
\pgfsetfillcolor{currentfill}%
\pgfsetlinewidth{1.003750pt}%
\definecolor{currentstroke}{rgb}{0.000000,0.000000,0.000000}%
\pgfsetstrokecolor{currentstroke}%
\pgfsetdash{}{0pt}%
\pgfpathmoveto{\pgfqpoint{5.459200in}{1.380712in}}%
\pgfpathcurveto{\pgfqpoint{5.467436in}{1.380712in}}{\pgfqpoint{5.475336in}{1.383985in}}{\pgfqpoint{5.481160in}{1.389809in}}%
\pgfpathcurveto{\pgfqpoint{5.486984in}{1.395633in}}{\pgfqpoint{5.490256in}{1.403533in}}{\pgfqpoint{5.490256in}{1.411769in}}%
\pgfpathcurveto{\pgfqpoint{5.490256in}{1.420005in}}{\pgfqpoint{5.486984in}{1.427905in}}{\pgfqpoint{5.481160in}{1.433729in}}%
\pgfpathcurveto{\pgfqpoint{5.475336in}{1.439553in}}{\pgfqpoint{5.467436in}{1.442825in}}{\pgfqpoint{5.459200in}{1.442825in}}%
\pgfpathcurveto{\pgfqpoint{5.450964in}{1.442825in}}{\pgfqpoint{5.443064in}{1.439553in}}{\pgfqpoint{5.437240in}{1.433729in}}%
\pgfpathcurveto{\pgfqpoint{5.431416in}{1.427905in}}{\pgfqpoint{5.428144in}{1.420005in}}{\pgfqpoint{5.428144in}{1.411769in}}%
\pgfpathcurveto{\pgfqpoint{5.428144in}{1.403533in}}{\pgfqpoint{5.431416in}{1.395633in}}{\pgfqpoint{5.437240in}{1.389809in}}%
\pgfpathcurveto{\pgfqpoint{5.443064in}{1.383985in}}{\pgfqpoint{5.450964in}{1.380712in}}{\pgfqpoint{5.459200in}{1.380712in}}%
\pgfpathclose%
\pgfusepath{stroke,fill}%
\end{pgfscope}%
\begin{pgfscope}%
\pgfpathrectangle{\pgfqpoint{0.894063in}{0.630000in}}{\pgfqpoint{6.713438in}{2.060556in}} %
\pgfusepath{clip}%
\pgfsetbuttcap%
\pgfsetroundjoin%
\definecolor{currentfill}{rgb}{0.000000,0.000000,0.000000}%
\pgfsetfillcolor{currentfill}%
\pgfsetlinewidth{1.003750pt}%
\definecolor{currentstroke}{rgb}{0.000000,0.000000,0.000000}%
\pgfsetstrokecolor{currentstroke}%
\pgfsetdash{}{0pt}%
\pgfpathmoveto{\pgfqpoint{6.936156in}{1.610523in}}%
\pgfpathcurveto{\pgfqpoint{6.944393in}{1.610523in}}{\pgfqpoint{6.952293in}{1.613796in}}{\pgfqpoint{6.958117in}{1.619619in}}%
\pgfpathcurveto{\pgfqpoint{6.963940in}{1.625443in}}{\pgfqpoint{6.967213in}{1.633343in}}{\pgfqpoint{6.967213in}{1.641580in}}%
\pgfpathcurveto{\pgfqpoint{6.967213in}{1.649816in}}{\pgfqpoint{6.963940in}{1.657716in}}{\pgfqpoint{6.958117in}{1.663540in}}%
\pgfpathcurveto{\pgfqpoint{6.952293in}{1.669364in}}{\pgfqpoint{6.944393in}{1.672636in}}{\pgfqpoint{6.936156in}{1.672636in}}%
\pgfpathcurveto{\pgfqpoint{6.927920in}{1.672636in}}{\pgfqpoint{6.920020in}{1.669364in}}{\pgfqpoint{6.914196in}{1.663540in}}%
\pgfpathcurveto{\pgfqpoint{6.908372in}{1.657716in}}{\pgfqpoint{6.905100in}{1.649816in}}{\pgfqpoint{6.905100in}{1.641580in}}%
\pgfpathcurveto{\pgfqpoint{6.905100in}{1.633343in}}{\pgfqpoint{6.908372in}{1.625443in}}{\pgfqpoint{6.914196in}{1.619619in}}%
\pgfpathcurveto{\pgfqpoint{6.920020in}{1.613796in}}{\pgfqpoint{6.927920in}{1.610523in}}{\pgfqpoint{6.936156in}{1.610523in}}%
\pgfpathclose%
\pgfusepath{stroke,fill}%
\end{pgfscope}%
\begin{pgfscope}%
\pgfpathrectangle{\pgfqpoint{0.894063in}{0.630000in}}{\pgfqpoint{6.713438in}{2.060556in}} %
\pgfusepath{clip}%
\pgfsetbuttcap%
\pgfsetroundjoin%
\definecolor{currentfill}{rgb}{0.000000,0.000000,0.000000}%
\pgfsetfillcolor{currentfill}%
\pgfsetlinewidth{1.003750pt}%
\definecolor{currentstroke}{rgb}{0.000000,0.000000,0.000000}%
\pgfsetstrokecolor{currentstroke}%
\pgfsetdash{}{0pt}%
\pgfpathmoveto{\pgfqpoint{5.862006in}{1.453397in}}%
\pgfpathcurveto{\pgfqpoint{5.870243in}{1.453397in}}{\pgfqpoint{5.878143in}{1.456669in}}{\pgfqpoint{5.883967in}{1.462493in}}%
\pgfpathcurveto{\pgfqpoint{5.889790in}{1.468317in}}{\pgfqpoint{5.893063in}{1.476217in}}{\pgfqpoint{5.893063in}{1.484454in}}%
\pgfpathcurveto{\pgfqpoint{5.893063in}{1.492690in}}{\pgfqpoint{5.889790in}{1.500590in}}{\pgfqpoint{5.883967in}{1.506414in}}%
\pgfpathcurveto{\pgfqpoint{5.878143in}{1.512238in}}{\pgfqpoint{5.870243in}{1.515510in}}{\pgfqpoint{5.862006in}{1.515510in}}%
\pgfpathcurveto{\pgfqpoint{5.853770in}{1.515510in}}{\pgfqpoint{5.845870in}{1.512238in}}{\pgfqpoint{5.840046in}{1.506414in}}%
\pgfpathcurveto{\pgfqpoint{5.834222in}{1.500590in}}{\pgfqpoint{5.830950in}{1.492690in}}{\pgfqpoint{5.830950in}{1.484454in}}%
\pgfpathcurveto{\pgfqpoint{5.830950in}{1.476217in}}{\pgfqpoint{5.834222in}{1.468317in}}{\pgfqpoint{5.840046in}{1.462493in}}%
\pgfpathcurveto{\pgfqpoint{5.845870in}{1.456669in}}{\pgfqpoint{5.853770in}{1.453397in}}{\pgfqpoint{5.862006in}{1.453397in}}%
\pgfpathclose%
\pgfusepath{stroke,fill}%
\end{pgfscope}%
\begin{pgfscope}%
\pgfpathrectangle{\pgfqpoint{0.894063in}{0.630000in}}{\pgfqpoint{6.713438in}{2.060556in}} %
\pgfusepath{clip}%
\pgfsetbuttcap%
\pgfsetroundjoin%
\definecolor{currentfill}{rgb}{0.000000,0.000000,0.000000}%
\pgfsetfillcolor{currentfill}%
\pgfsetlinewidth{1.003750pt}%
\definecolor{currentstroke}{rgb}{0.000000,0.000000,0.000000}%
\pgfsetstrokecolor{currentstroke}%
\pgfsetdash{}{0pt}%
\pgfpathmoveto{\pgfqpoint{7.070425in}{1.625712in}}%
\pgfpathcurveto{\pgfqpoint{7.078661in}{1.625712in}}{\pgfqpoint{7.086561in}{1.628985in}}{\pgfqpoint{7.092385in}{1.634809in}}%
\pgfpathcurveto{\pgfqpoint{7.098209in}{1.640633in}}{\pgfqpoint{7.101481in}{1.648533in}}{\pgfqpoint{7.101481in}{1.656769in}}%
\pgfpathcurveto{\pgfqpoint{7.101481in}{1.665005in}}{\pgfqpoint{7.098209in}{1.672905in}}{\pgfqpoint{7.092385in}{1.678729in}}%
\pgfpathcurveto{\pgfqpoint{7.086561in}{1.684553in}}{\pgfqpoint{7.078661in}{1.687825in}}{\pgfqpoint{7.070425in}{1.687825in}}%
\pgfpathcurveto{\pgfqpoint{7.062189in}{1.687825in}}{\pgfqpoint{7.054289in}{1.684553in}}{\pgfqpoint{7.048465in}{1.678729in}}%
\pgfpathcurveto{\pgfqpoint{7.042641in}{1.672905in}}{\pgfqpoint{7.039369in}{1.665005in}}{\pgfqpoint{7.039369in}{1.656769in}}%
\pgfpathcurveto{\pgfqpoint{7.039369in}{1.648533in}}{\pgfqpoint{7.042641in}{1.640633in}}{\pgfqpoint{7.048465in}{1.634809in}}%
\pgfpathcurveto{\pgfqpoint{7.054289in}{1.628985in}}{\pgfqpoint{7.062189in}{1.625712in}}{\pgfqpoint{7.070425in}{1.625712in}}%
\pgfpathclose%
\pgfusepath{stroke,fill}%
\end{pgfscope}%
\begin{pgfscope}%
\pgfpathrectangle{\pgfqpoint{0.894063in}{0.630000in}}{\pgfqpoint{6.713438in}{2.060556in}} %
\pgfusepath{clip}%
\pgfsetbuttcap%
\pgfsetroundjoin%
\definecolor{currentfill}{rgb}{0.000000,0.000000,0.000000}%
\pgfsetfillcolor{currentfill}%
\pgfsetlinewidth{1.003750pt}%
\definecolor{currentstroke}{rgb}{0.000000,0.000000,0.000000}%
\pgfsetstrokecolor{currentstroke}%
\pgfsetdash{}{0pt}%
\pgfpathmoveto{\pgfqpoint{3.176631in}{1.015399in}}%
\pgfpathcurveto{\pgfqpoint{3.184868in}{1.015399in}}{\pgfqpoint{3.192768in}{1.018672in}}{\pgfqpoint{3.198592in}{1.024496in}}%
\pgfpathcurveto{\pgfqpoint{3.204415in}{1.030320in}}{\pgfqpoint{3.207688in}{1.038220in}}{\pgfqpoint{3.207688in}{1.046456in}}%
\pgfpathcurveto{\pgfqpoint{3.207688in}{1.054692in}}{\pgfqpoint{3.204415in}{1.062592in}}{\pgfqpoint{3.198592in}{1.068416in}}%
\pgfpathcurveto{\pgfqpoint{3.192768in}{1.074240in}}{\pgfqpoint{3.184868in}{1.077512in}}{\pgfqpoint{3.176631in}{1.077512in}}%
\pgfpathcurveto{\pgfqpoint{3.168395in}{1.077512in}}{\pgfqpoint{3.160495in}{1.074240in}}{\pgfqpoint{3.154671in}{1.068416in}}%
\pgfpathcurveto{\pgfqpoint{3.148847in}{1.062592in}}{\pgfqpoint{3.145575in}{1.054692in}}{\pgfqpoint{3.145575in}{1.046456in}}%
\pgfpathcurveto{\pgfqpoint{3.145575in}{1.038220in}}{\pgfqpoint{3.148847in}{1.030320in}}{\pgfqpoint{3.154671in}{1.024496in}}%
\pgfpathcurveto{\pgfqpoint{3.160495in}{1.018672in}}{\pgfqpoint{3.168395in}{1.015399in}}{\pgfqpoint{3.176631in}{1.015399in}}%
\pgfpathclose%
\pgfusepath{stroke,fill}%
\end{pgfscope}%
\begin{pgfscope}%
\pgfpathrectangle{\pgfqpoint{0.894063in}{0.630000in}}{\pgfqpoint{6.713438in}{2.060556in}} %
\pgfusepath{clip}%
\pgfsetbuttcap%
\pgfsetroundjoin%
\definecolor{currentfill}{rgb}{0.000000,0.000000,0.000000}%
\pgfsetfillcolor{currentfill}%
\pgfsetlinewidth{1.003750pt}%
\definecolor{currentstroke}{rgb}{0.000000,0.000000,0.000000}%
\pgfsetstrokecolor{currentstroke}%
\pgfsetdash{}{0pt}%
\pgfpathmoveto{\pgfqpoint{2.102481in}{0.850272in}}%
\pgfpathcurveto{\pgfqpoint{2.110718in}{0.850272in}}{\pgfqpoint{2.118618in}{0.853545in}}{\pgfqpoint{2.124442in}{0.859369in}}%
\pgfpathcurveto{\pgfqpoint{2.130265in}{0.865193in}}{\pgfqpoint{2.133538in}{0.873093in}}{\pgfqpoint{2.133538in}{0.881329in}}%
\pgfpathcurveto{\pgfqpoint{2.133538in}{0.889565in}}{\pgfqpoint{2.130265in}{0.897465in}}{\pgfqpoint{2.124442in}{0.903289in}}%
\pgfpathcurveto{\pgfqpoint{2.118618in}{0.909113in}}{\pgfqpoint{2.110718in}{0.912385in}}{\pgfqpoint{2.102481in}{0.912385in}}%
\pgfpathcurveto{\pgfqpoint{2.094245in}{0.912385in}}{\pgfqpoint{2.086345in}{0.909113in}}{\pgfqpoint{2.080521in}{0.903289in}}%
\pgfpathcurveto{\pgfqpoint{2.074697in}{0.897465in}}{\pgfqpoint{2.071425in}{0.889565in}}{\pgfqpoint{2.071425in}{0.881329in}}%
\pgfpathcurveto{\pgfqpoint{2.071425in}{0.873093in}}{\pgfqpoint{2.074697in}{0.865193in}}{\pgfqpoint{2.080521in}{0.859369in}}%
\pgfpathcurveto{\pgfqpoint{2.086345in}{0.853545in}}{\pgfqpoint{2.094245in}{0.850272in}}{\pgfqpoint{2.102481in}{0.850272in}}%
\pgfpathclose%
\pgfusepath{stroke,fill}%
\end{pgfscope}%
\begin{pgfscope}%
\pgfpathrectangle{\pgfqpoint{0.894063in}{0.630000in}}{\pgfqpoint{6.713438in}{2.060556in}} %
\pgfusepath{clip}%
\pgfsetbuttcap%
\pgfsetroundjoin%
\definecolor{currentfill}{rgb}{0.000000,0.000000,0.000000}%
\pgfsetfillcolor{currentfill}%
\pgfsetlinewidth{1.003750pt}%
\definecolor{currentstroke}{rgb}{0.000000,0.000000,0.000000}%
\pgfsetstrokecolor{currentstroke}%
\pgfsetdash{}{0pt}%
\pgfpathmoveto{\pgfqpoint{1.968213in}{0.826034in}}%
\pgfpathcurveto{\pgfqpoint{1.976449in}{0.826034in}}{\pgfqpoint{1.984349in}{0.829307in}}{\pgfqpoint{1.990173in}{0.835131in}}%
\pgfpathcurveto{\pgfqpoint{1.995997in}{0.840955in}}{\pgfqpoint{1.999269in}{0.848855in}}{\pgfqpoint{1.999269in}{0.857091in}}%
\pgfpathcurveto{\pgfqpoint{1.999269in}{0.865327in}}{\pgfqpoint{1.995997in}{0.873227in}}{\pgfqpoint{1.990173in}{0.879051in}}%
\pgfpathcurveto{\pgfqpoint{1.984349in}{0.884875in}}{\pgfqpoint{1.976449in}{0.888147in}}{\pgfqpoint{1.968213in}{0.888147in}}%
\pgfpathcurveto{\pgfqpoint{1.959976in}{0.888147in}}{\pgfqpoint{1.952076in}{0.884875in}}{\pgfqpoint{1.946252in}{0.879051in}}%
\pgfpathcurveto{\pgfqpoint{1.940428in}{0.873227in}}{\pgfqpoint{1.937156in}{0.865327in}}{\pgfqpoint{1.937156in}{0.857091in}}%
\pgfpathcurveto{\pgfqpoint{1.937156in}{0.848855in}}{\pgfqpoint{1.940428in}{0.840955in}}{\pgfqpoint{1.946252in}{0.835131in}}%
\pgfpathcurveto{\pgfqpoint{1.952076in}{0.829307in}}{\pgfqpoint{1.959976in}{0.826034in}}{\pgfqpoint{1.968213in}{0.826034in}}%
\pgfpathclose%
\pgfusepath{stroke,fill}%
\end{pgfscope}%
\begin{pgfscope}%
\pgfpathrectangle{\pgfqpoint{0.894063in}{0.630000in}}{\pgfqpoint{6.713438in}{2.060556in}} %
\pgfusepath{clip}%
\pgfsetbuttcap%
\pgfsetroundjoin%
\definecolor{currentfill}{rgb}{0.000000,0.000000,0.000000}%
\pgfsetfillcolor{currentfill}%
\pgfsetlinewidth{1.003750pt}%
\definecolor{currentstroke}{rgb}{0.000000,0.000000,0.000000}%
\pgfsetstrokecolor{currentstroke}%
\pgfsetdash{}{0pt}%
\pgfpathmoveto{\pgfqpoint{3.310900in}{1.035222in}}%
\pgfpathcurveto{\pgfqpoint{3.319136in}{1.035222in}}{\pgfqpoint{3.327036in}{1.038494in}}{\pgfqpoint{3.332860in}{1.044318in}}%
\pgfpathcurveto{\pgfqpoint{3.338684in}{1.050142in}}{\pgfqpoint{3.341956in}{1.058042in}}{\pgfqpoint{3.341956in}{1.066278in}}%
\pgfpathcurveto{\pgfqpoint{3.341956in}{1.074515in}}{\pgfqpoint{3.338684in}{1.082415in}}{\pgfqpoint{3.332860in}{1.088239in}}%
\pgfpathcurveto{\pgfqpoint{3.327036in}{1.094063in}}{\pgfqpoint{3.319136in}{1.097335in}}{\pgfqpoint{3.310900in}{1.097335in}}%
\pgfpathcurveto{\pgfqpoint{3.302664in}{1.097335in}}{\pgfqpoint{3.294764in}{1.094063in}}{\pgfqpoint{3.288940in}{1.088239in}}%
\pgfpathcurveto{\pgfqpoint{3.283116in}{1.082415in}}{\pgfqpoint{3.279844in}{1.074515in}}{\pgfqpoint{3.279844in}{1.066278in}}%
\pgfpathcurveto{\pgfqpoint{3.279844in}{1.058042in}}{\pgfqpoint{3.283116in}{1.050142in}}{\pgfqpoint{3.288940in}{1.044318in}}%
\pgfpathcurveto{\pgfqpoint{3.294764in}{1.038494in}}{\pgfqpoint{3.302664in}{1.035222in}}{\pgfqpoint{3.310900in}{1.035222in}}%
\pgfpathclose%
\pgfusepath{stroke,fill}%
\end{pgfscope}%
\begin{pgfscope}%
\pgfpathrectangle{\pgfqpoint{0.894063in}{0.630000in}}{\pgfqpoint{6.713438in}{2.060556in}} %
\pgfusepath{clip}%
\pgfsetbuttcap%
\pgfsetroundjoin%
\definecolor{currentfill}{rgb}{0.000000,0.000000,0.000000}%
\pgfsetfillcolor{currentfill}%
\pgfsetlinewidth{1.003750pt}%
\definecolor{currentstroke}{rgb}{0.000000,0.000000,0.000000}%
\pgfsetstrokecolor{currentstroke}%
\pgfsetdash{}{0pt}%
\pgfpathmoveto{\pgfqpoint{5.593469in}{1.409154in}}%
\pgfpathcurveto{\pgfqpoint{5.601705in}{1.409154in}}{\pgfqpoint{5.609605in}{1.412426in}}{\pgfqpoint{5.615429in}{1.418250in}}%
\pgfpathcurveto{\pgfqpoint{5.621253in}{1.424074in}}{\pgfqpoint{5.624525in}{1.431974in}}{\pgfqpoint{5.624525in}{1.440210in}}%
\pgfpathcurveto{\pgfqpoint{5.624525in}{1.448447in}}{\pgfqpoint{5.621253in}{1.456347in}}{\pgfqpoint{5.615429in}{1.462171in}}%
\pgfpathcurveto{\pgfqpoint{5.609605in}{1.467995in}}{\pgfqpoint{5.601705in}{1.471267in}}{\pgfqpoint{5.593469in}{1.471267in}}%
\pgfpathcurveto{\pgfqpoint{5.585232in}{1.471267in}}{\pgfqpoint{5.577332in}{1.467995in}}{\pgfqpoint{5.571508in}{1.462171in}}%
\pgfpathcurveto{\pgfqpoint{5.565685in}{1.456347in}}{\pgfqpoint{5.562412in}{1.448447in}}{\pgfqpoint{5.562412in}{1.440210in}}%
\pgfpathcurveto{\pgfqpoint{5.562412in}{1.431974in}}{\pgfqpoint{5.565685in}{1.424074in}}{\pgfqpoint{5.571508in}{1.418250in}}%
\pgfpathcurveto{\pgfqpoint{5.577332in}{1.412426in}}{\pgfqpoint{5.585232in}{1.409154in}}{\pgfqpoint{5.593469in}{1.409154in}}%
\pgfpathclose%
\pgfusepath{stroke,fill}%
\end{pgfscope}%
\begin{pgfscope}%
\pgfpathrectangle{\pgfqpoint{0.894063in}{0.630000in}}{\pgfqpoint{6.713438in}{2.060556in}} %
\pgfusepath{clip}%
\pgfsetbuttcap%
\pgfsetroundjoin%
\definecolor{currentfill}{rgb}{0.000000,0.000000,0.000000}%
\pgfsetfillcolor{currentfill}%
\pgfsetlinewidth{1.003750pt}%
\definecolor{currentstroke}{rgb}{0.000000,0.000000,0.000000}%
\pgfsetstrokecolor{currentstroke}%
\pgfsetdash{}{0pt}%
\pgfpathmoveto{\pgfqpoint{3.042363in}{0.982236in}}%
\pgfpathcurveto{\pgfqpoint{3.050599in}{0.982236in}}{\pgfqpoint{3.058499in}{0.985509in}}{\pgfqpoint{3.064323in}{0.991333in}}%
\pgfpathcurveto{\pgfqpoint{3.070147in}{0.997156in}}{\pgfqpoint{3.073419in}{1.005056in}}{\pgfqpoint{3.073419in}{1.013293in}}%
\pgfpathcurveto{\pgfqpoint{3.073419in}{1.021529in}}{\pgfqpoint{3.070147in}{1.029429in}}{\pgfqpoint{3.064323in}{1.035253in}}%
\pgfpathcurveto{\pgfqpoint{3.058499in}{1.041077in}}{\pgfqpoint{3.050599in}{1.044349in}}{\pgfqpoint{3.042363in}{1.044349in}}%
\pgfpathcurveto{\pgfqpoint{3.034126in}{1.044349in}}{\pgfqpoint{3.026226in}{1.041077in}}{\pgfqpoint{3.020402in}{1.035253in}}%
\pgfpathcurveto{\pgfqpoint{3.014578in}{1.029429in}}{\pgfqpoint{3.011306in}{1.021529in}}{\pgfqpoint{3.011306in}{1.013293in}}%
\pgfpathcurveto{\pgfqpoint{3.011306in}{1.005056in}}{\pgfqpoint{3.014578in}{0.997156in}}{\pgfqpoint{3.020402in}{0.991333in}}%
\pgfpathcurveto{\pgfqpoint{3.026226in}{0.985509in}}{\pgfqpoint{3.034126in}{0.982236in}}{\pgfqpoint{3.042363in}{0.982236in}}%
\pgfpathclose%
\pgfusepath{stroke,fill}%
\end{pgfscope}%
\begin{pgfscope}%
\pgfpathrectangle{\pgfqpoint{0.894063in}{0.630000in}}{\pgfqpoint{6.713438in}{2.060556in}} %
\pgfusepath{clip}%
\pgfsetbuttcap%
\pgfsetroundjoin%
\definecolor{currentfill}{rgb}{0.000000,0.000000,0.000000}%
\pgfsetfillcolor{currentfill}%
\pgfsetlinewidth{1.003750pt}%
\definecolor{currentstroke}{rgb}{0.000000,0.000000,0.000000}%
\pgfsetstrokecolor{currentstroke}%
\pgfsetdash{}{0pt}%
\pgfpathmoveto{\pgfqpoint{5.190663in}{1.333308in}}%
\pgfpathcurveto{\pgfqpoint{5.198899in}{1.333308in}}{\pgfqpoint{5.206799in}{1.336580in}}{\pgfqpoint{5.212623in}{1.342404in}}%
\pgfpathcurveto{\pgfqpoint{5.218447in}{1.348228in}}{\pgfqpoint{5.221719in}{1.356128in}}{\pgfqpoint{5.221719in}{1.364364in}}%
\pgfpathcurveto{\pgfqpoint{5.221719in}{1.372601in}}{\pgfqpoint{5.218447in}{1.380501in}}{\pgfqpoint{5.212623in}{1.386325in}}%
\pgfpathcurveto{\pgfqpoint{5.206799in}{1.392149in}}{\pgfqpoint{5.198899in}{1.395421in}}{\pgfqpoint{5.190663in}{1.395421in}}%
\pgfpathcurveto{\pgfqpoint{5.182426in}{1.395421in}}{\pgfqpoint{5.174526in}{1.392149in}}{\pgfqpoint{5.168702in}{1.386325in}}%
\pgfpathcurveto{\pgfqpoint{5.162878in}{1.380501in}}{\pgfqpoint{5.159606in}{1.372601in}}{\pgfqpoint{5.159606in}{1.364364in}}%
\pgfpathcurveto{\pgfqpoint{5.159606in}{1.356128in}}{\pgfqpoint{5.162878in}{1.348228in}}{\pgfqpoint{5.168702in}{1.342404in}}%
\pgfpathcurveto{\pgfqpoint{5.174526in}{1.336580in}}{\pgfqpoint{5.182426in}{1.333308in}}{\pgfqpoint{5.190663in}{1.333308in}}%
\pgfpathclose%
\pgfusepath{stroke,fill}%
\end{pgfscope}%
\begin{pgfscope}%
\pgfpathrectangle{\pgfqpoint{0.894063in}{0.630000in}}{\pgfqpoint{6.713438in}{2.060556in}} %
\pgfusepath{clip}%
\pgfsetbuttcap%
\pgfsetroundjoin%
\definecolor{currentfill}{rgb}{0.000000,0.000000,0.000000}%
\pgfsetfillcolor{currentfill}%
\pgfsetlinewidth{1.003750pt}%
\definecolor{currentstroke}{rgb}{0.000000,0.000000,0.000000}%
\pgfsetstrokecolor{currentstroke}%
\pgfsetdash{}{0pt}%
\pgfpathmoveto{\pgfqpoint{6.801888in}{1.583718in}}%
\pgfpathcurveto{\pgfqpoint{6.810124in}{1.583718in}}{\pgfqpoint{6.818024in}{1.586991in}}{\pgfqpoint{6.823848in}{1.592815in}}%
\pgfpathcurveto{\pgfqpoint{6.829672in}{1.598638in}}{\pgfqpoint{6.832944in}{1.606539in}}{\pgfqpoint{6.832944in}{1.614775in}}%
\pgfpathcurveto{\pgfqpoint{6.832944in}{1.623011in}}{\pgfqpoint{6.829672in}{1.630911in}}{\pgfqpoint{6.823848in}{1.636735in}}%
\pgfpathcurveto{\pgfqpoint{6.818024in}{1.642559in}}{\pgfqpoint{6.810124in}{1.645831in}}{\pgfqpoint{6.801888in}{1.645831in}}%
\pgfpathcurveto{\pgfqpoint{6.793651in}{1.645831in}}{\pgfqpoint{6.785751in}{1.642559in}}{\pgfqpoint{6.779927in}{1.636735in}}%
\pgfpathcurveto{\pgfqpoint{6.774103in}{1.630911in}}{\pgfqpoint{6.770831in}{1.623011in}}{\pgfqpoint{6.770831in}{1.614775in}}%
\pgfpathcurveto{\pgfqpoint{6.770831in}{1.606539in}}{\pgfqpoint{6.774103in}{1.598638in}}{\pgfqpoint{6.779927in}{1.592815in}}%
\pgfpathcurveto{\pgfqpoint{6.785751in}{1.586991in}}{\pgfqpoint{6.793651in}{1.583718in}}{\pgfqpoint{6.801888in}{1.583718in}}%
\pgfpathclose%
\pgfusepath{stroke,fill}%
\end{pgfscope}%
\begin{pgfscope}%
\pgfpathrectangle{\pgfqpoint{0.894063in}{0.630000in}}{\pgfqpoint{6.713438in}{2.060556in}} %
\pgfusepath{clip}%
\pgfsetbuttcap%
\pgfsetroundjoin%
\definecolor{currentfill}{rgb}{0.000000,0.000000,0.000000}%
\pgfsetfillcolor{currentfill}%
\pgfsetlinewidth{1.003750pt}%
\definecolor{currentstroke}{rgb}{0.000000,0.000000,0.000000}%
\pgfsetstrokecolor{currentstroke}%
\pgfsetdash{}{0pt}%
\pgfpathmoveto{\pgfqpoint{3.579438in}{1.076539in}}%
\pgfpathcurveto{\pgfqpoint{3.587674in}{1.076539in}}{\pgfqpoint{3.595574in}{1.079811in}}{\pgfqpoint{3.601398in}{1.085635in}}%
\pgfpathcurveto{\pgfqpoint{3.607222in}{1.091459in}}{\pgfqpoint{3.610494in}{1.099359in}}{\pgfqpoint{3.610494in}{1.107596in}}%
\pgfpathcurveto{\pgfqpoint{3.610494in}{1.115832in}}{\pgfqpoint{3.607222in}{1.123732in}}{\pgfqpoint{3.601398in}{1.129556in}}%
\pgfpathcurveto{\pgfqpoint{3.595574in}{1.135380in}}{\pgfqpoint{3.587674in}{1.138652in}}{\pgfqpoint{3.579438in}{1.138652in}}%
\pgfpathcurveto{\pgfqpoint{3.571201in}{1.138652in}}{\pgfqpoint{3.563301in}{1.135380in}}{\pgfqpoint{3.557477in}{1.129556in}}%
\pgfpathcurveto{\pgfqpoint{3.551653in}{1.123732in}}{\pgfqpoint{3.548381in}{1.115832in}}{\pgfqpoint{3.548381in}{1.107596in}}%
\pgfpathcurveto{\pgfqpoint{3.548381in}{1.099359in}}{\pgfqpoint{3.551653in}{1.091459in}}{\pgfqpoint{3.557477in}{1.085635in}}%
\pgfpathcurveto{\pgfqpoint{3.563301in}{1.079811in}}{\pgfqpoint{3.571201in}{1.076539in}}{\pgfqpoint{3.579438in}{1.076539in}}%
\pgfpathclose%
\pgfusepath{stroke,fill}%
\end{pgfscope}%
\begin{pgfscope}%
\pgfpathrectangle{\pgfqpoint{0.894063in}{0.630000in}}{\pgfqpoint{6.713438in}{2.060556in}} %
\pgfusepath{clip}%
\pgfsetbuttcap%
\pgfsetroundjoin%
\definecolor{currentfill}{rgb}{0.000000,0.000000,0.000000}%
\pgfsetfillcolor{currentfill}%
\pgfsetlinewidth{1.003750pt}%
\definecolor{currentstroke}{rgb}{0.000000,0.000000,0.000000}%
\pgfsetstrokecolor{currentstroke}%
\pgfsetdash{}{0pt}%
\pgfpathmoveto{\pgfqpoint{2.371019in}{0.891113in}}%
\pgfpathcurveto{\pgfqpoint{2.379255in}{0.891113in}}{\pgfqpoint{2.387155in}{0.894385in}}{\pgfqpoint{2.392979in}{0.900209in}}%
\pgfpathcurveto{\pgfqpoint{2.398803in}{0.906033in}}{\pgfqpoint{2.402075in}{0.913933in}}{\pgfqpoint{2.402075in}{0.922169in}}%
\pgfpathcurveto{\pgfqpoint{2.402075in}{0.930405in}}{\pgfqpoint{2.398803in}{0.938305in}}{\pgfqpoint{2.392979in}{0.944129in}}%
\pgfpathcurveto{\pgfqpoint{2.387155in}{0.949953in}}{\pgfqpoint{2.379255in}{0.953226in}}{\pgfqpoint{2.371019in}{0.953226in}}%
\pgfpathcurveto{\pgfqpoint{2.362782in}{0.953226in}}{\pgfqpoint{2.354882in}{0.949953in}}{\pgfqpoint{2.349058in}{0.944129in}}%
\pgfpathcurveto{\pgfqpoint{2.343235in}{0.938305in}}{\pgfqpoint{2.339962in}{0.930405in}}{\pgfqpoint{2.339962in}{0.922169in}}%
\pgfpathcurveto{\pgfqpoint{2.339962in}{0.913933in}}{\pgfqpoint{2.343235in}{0.906033in}}{\pgfqpoint{2.349058in}{0.900209in}}%
\pgfpathcurveto{\pgfqpoint{2.354882in}{0.894385in}}{\pgfqpoint{2.362782in}{0.891113in}}{\pgfqpoint{2.371019in}{0.891113in}}%
\pgfpathclose%
\pgfusepath{stroke,fill}%
\end{pgfscope}%
\begin{pgfscope}%
\pgfpathrectangle{\pgfqpoint{0.894063in}{0.630000in}}{\pgfqpoint{6.713438in}{2.060556in}} %
\pgfusepath{clip}%
\pgfsetbuttcap%
\pgfsetroundjoin%
\definecolor{currentfill}{rgb}{0.000000,0.000000,0.000000}%
\pgfsetfillcolor{currentfill}%
\pgfsetlinewidth{1.003750pt}%
\definecolor{currentstroke}{rgb}{0.000000,0.000000,0.000000}%
\pgfsetstrokecolor{currentstroke}%
\pgfsetdash{}{0pt}%
\pgfpathmoveto{\pgfqpoint{3.982244in}{1.143960in}}%
\pgfpathcurveto{\pgfqpoint{3.990480in}{1.143960in}}{\pgfqpoint{3.998380in}{1.147233in}}{\pgfqpoint{4.004204in}{1.153057in}}%
\pgfpathcurveto{\pgfqpoint{4.010028in}{1.158881in}}{\pgfqpoint{4.013300in}{1.166781in}}{\pgfqpoint{4.013300in}{1.175017in}}%
\pgfpathcurveto{\pgfqpoint{4.013300in}{1.183253in}}{\pgfqpoint{4.010028in}{1.191153in}}{\pgfqpoint{4.004204in}{1.196977in}}%
\pgfpathcurveto{\pgfqpoint{3.998380in}{1.202801in}}{\pgfqpoint{3.990480in}{1.206073in}}{\pgfqpoint{3.982244in}{1.206073in}}%
\pgfpathcurveto{\pgfqpoint{3.974007in}{1.206073in}}{\pgfqpoint{3.966107in}{1.202801in}}{\pgfqpoint{3.960283in}{1.196977in}}%
\pgfpathcurveto{\pgfqpoint{3.954460in}{1.191153in}}{\pgfqpoint{3.951187in}{1.183253in}}{\pgfqpoint{3.951187in}{1.175017in}}%
\pgfpathcurveto{\pgfqpoint{3.951187in}{1.166781in}}{\pgfqpoint{3.954460in}{1.158881in}}{\pgfqpoint{3.960283in}{1.153057in}}%
\pgfpathcurveto{\pgfqpoint{3.966107in}{1.147233in}}{\pgfqpoint{3.974007in}{1.143960in}}{\pgfqpoint{3.982244in}{1.143960in}}%
\pgfpathclose%
\pgfusepath{stroke,fill}%
\end{pgfscope}%
\begin{pgfscope}%
\pgfpathrectangle{\pgfqpoint{0.894063in}{0.630000in}}{\pgfqpoint{6.713438in}{2.060556in}} %
\pgfusepath{clip}%
\pgfsetbuttcap%
\pgfsetroundjoin%
\definecolor{currentfill}{rgb}{0.000000,0.000000,0.000000}%
\pgfsetfillcolor{currentfill}%
\pgfsetlinewidth{1.003750pt}%
\definecolor{currentstroke}{rgb}{0.000000,0.000000,0.000000}%
\pgfsetstrokecolor{currentstroke}%
\pgfsetdash{}{0pt}%
\pgfpathmoveto{\pgfqpoint{4.653588in}{1.256979in}}%
\pgfpathcurveto{\pgfqpoint{4.661824in}{1.256979in}}{\pgfqpoint{4.669724in}{1.260251in}}{\pgfqpoint{4.675548in}{1.266075in}}%
\pgfpathcurveto{\pgfqpoint{4.681372in}{1.271899in}}{\pgfqpoint{4.684644in}{1.279799in}}{\pgfqpoint{4.684644in}{1.288035in}}%
\pgfpathcurveto{\pgfqpoint{4.684644in}{1.296272in}}{\pgfqpoint{4.681372in}{1.304172in}}{\pgfqpoint{4.675548in}{1.309996in}}%
\pgfpathcurveto{\pgfqpoint{4.669724in}{1.315820in}}{\pgfqpoint{4.661824in}{1.319092in}}{\pgfqpoint{4.653588in}{1.319092in}}%
\pgfpathcurveto{\pgfqpoint{4.645351in}{1.319092in}}{\pgfqpoint{4.637451in}{1.315820in}}{\pgfqpoint{4.631627in}{1.309996in}}%
\pgfpathcurveto{\pgfqpoint{4.625803in}{1.304172in}}{\pgfqpoint{4.622531in}{1.296272in}}{\pgfqpoint{4.622531in}{1.288035in}}%
\pgfpathcurveto{\pgfqpoint{4.622531in}{1.279799in}}{\pgfqpoint{4.625803in}{1.271899in}}{\pgfqpoint{4.631627in}{1.266075in}}%
\pgfpathcurveto{\pgfqpoint{4.637451in}{1.260251in}}{\pgfqpoint{4.645351in}{1.256979in}}{\pgfqpoint{4.653588in}{1.256979in}}%
\pgfpathclose%
\pgfusepath{stroke,fill}%
\end{pgfscope}%
\begin{pgfscope}%
\pgfpathrectangle{\pgfqpoint{0.894063in}{0.630000in}}{\pgfqpoint{6.713438in}{2.060556in}} %
\pgfusepath{clip}%
\pgfsetbuttcap%
\pgfsetroundjoin%
\definecolor{currentfill}{rgb}{0.000000,0.000000,0.000000}%
\pgfsetfillcolor{currentfill}%
\pgfsetlinewidth{1.003750pt}%
\definecolor{currentstroke}{rgb}{0.000000,0.000000,0.000000}%
\pgfsetstrokecolor{currentstroke}%
\pgfsetdash{}{0pt}%
\pgfpathmoveto{\pgfqpoint{3.713706in}{1.094949in}}%
\pgfpathcurveto{\pgfqpoint{3.721943in}{1.094949in}}{\pgfqpoint{3.729843in}{1.098221in}}{\pgfqpoint{3.735667in}{1.104045in}}%
\pgfpathcurveto{\pgfqpoint{3.741490in}{1.109869in}}{\pgfqpoint{3.744763in}{1.117769in}}{\pgfqpoint{3.744763in}{1.126005in}}%
\pgfpathcurveto{\pgfqpoint{3.744763in}{1.134241in}}{\pgfqpoint{3.741490in}{1.142141in}}{\pgfqpoint{3.735667in}{1.147965in}}%
\pgfpathcurveto{\pgfqpoint{3.729843in}{1.153789in}}{\pgfqpoint{3.721943in}{1.157062in}}{\pgfqpoint{3.713706in}{1.157062in}}%
\pgfpathcurveto{\pgfqpoint{3.705470in}{1.157062in}}{\pgfqpoint{3.697570in}{1.153789in}}{\pgfqpoint{3.691746in}{1.147965in}}%
\pgfpathcurveto{\pgfqpoint{3.685922in}{1.142141in}}{\pgfqpoint{3.682650in}{1.134241in}}{\pgfqpoint{3.682650in}{1.126005in}}%
\pgfpathcurveto{\pgfqpoint{3.682650in}{1.117769in}}{\pgfqpoint{3.685922in}{1.109869in}}{\pgfqpoint{3.691746in}{1.104045in}}%
\pgfpathcurveto{\pgfqpoint{3.697570in}{1.098221in}}{\pgfqpoint{3.705470in}{1.094949in}}{\pgfqpoint{3.713706in}{1.094949in}}%
\pgfpathclose%
\pgfusepath{stroke,fill}%
\end{pgfscope}%
\begin{pgfscope}%
\pgfpathrectangle{\pgfqpoint{0.894063in}{0.630000in}}{\pgfqpoint{6.713438in}{2.060556in}} %
\pgfusepath{clip}%
\pgfsetbuttcap%
\pgfsetroundjoin%
\definecolor{currentfill}{rgb}{0.000000,0.000000,0.000000}%
\pgfsetfillcolor{currentfill}%
\pgfsetlinewidth{1.003750pt}%
\definecolor{currentstroke}{rgb}{0.000000,0.000000,0.000000}%
\pgfsetstrokecolor{currentstroke}%
\pgfsetdash{}{0pt}%
\pgfpathmoveto{\pgfqpoint{2.236750in}{0.872815in}}%
\pgfpathcurveto{\pgfqpoint{2.244986in}{0.872815in}}{\pgfqpoint{2.252886in}{0.876087in}}{\pgfqpoint{2.258710in}{0.881911in}}%
\pgfpathcurveto{\pgfqpoint{2.264534in}{0.887735in}}{\pgfqpoint{2.267806in}{0.895635in}}{\pgfqpoint{2.267806in}{0.903871in}}%
\pgfpathcurveto{\pgfqpoint{2.267806in}{0.912108in}}{\pgfqpoint{2.264534in}{0.920008in}}{\pgfqpoint{2.258710in}{0.925832in}}%
\pgfpathcurveto{\pgfqpoint{2.252886in}{0.931656in}}{\pgfqpoint{2.244986in}{0.934928in}}{\pgfqpoint{2.236750in}{0.934928in}}%
\pgfpathcurveto{\pgfqpoint{2.228514in}{0.934928in}}{\pgfqpoint{2.220614in}{0.931656in}}{\pgfqpoint{2.214790in}{0.925832in}}%
\pgfpathcurveto{\pgfqpoint{2.208966in}{0.920008in}}{\pgfqpoint{2.205694in}{0.912108in}}{\pgfqpoint{2.205694in}{0.903871in}}%
\pgfpathcurveto{\pgfqpoint{2.205694in}{0.895635in}}{\pgfqpoint{2.208966in}{0.887735in}}{\pgfqpoint{2.214790in}{0.881911in}}%
\pgfpathcurveto{\pgfqpoint{2.220614in}{0.876087in}}{\pgfqpoint{2.228514in}{0.872815in}}{\pgfqpoint{2.236750in}{0.872815in}}%
\pgfpathclose%
\pgfusepath{stroke,fill}%
\end{pgfscope}%
\begin{pgfscope}%
\pgfpathrectangle{\pgfqpoint{0.894063in}{0.630000in}}{\pgfqpoint{6.713438in}{2.060556in}} %
\pgfusepath{clip}%
\pgfsetbuttcap%
\pgfsetroundjoin%
\definecolor{currentfill}{rgb}{0.000000,0.750000,0.750000}%
\pgfsetfillcolor{currentfill}%
\pgfsetlinewidth{1.003750pt}%
\definecolor{currentstroke}{rgb}{0.000000,0.750000,0.750000}%
\pgfsetstrokecolor{currentstroke}%
\pgfsetdash{}{0pt}%
\pgfpathmoveto{\pgfqpoint{6.667619in}{1.630793in}}%
\pgfpathcurveto{\pgfqpoint{6.675855in}{1.630793in}}{\pgfqpoint{6.683755in}{1.634065in}}{\pgfqpoint{6.689579in}{1.639889in}}%
\pgfpathcurveto{\pgfqpoint{6.695403in}{1.645713in}}{\pgfqpoint{6.698675in}{1.653613in}}{\pgfqpoint{6.698675in}{1.661850in}}%
\pgfpathcurveto{\pgfqpoint{6.698675in}{1.670086in}}{\pgfqpoint{6.695403in}{1.677986in}}{\pgfqpoint{6.689579in}{1.683810in}}%
\pgfpathcurveto{\pgfqpoint{6.683755in}{1.689634in}}{\pgfqpoint{6.675855in}{1.692906in}}{\pgfqpoint{6.667619in}{1.692906in}}%
\pgfpathcurveto{\pgfqpoint{6.659382in}{1.692906in}}{\pgfqpoint{6.651482in}{1.689634in}}{\pgfqpoint{6.645658in}{1.683810in}}%
\pgfpathcurveto{\pgfqpoint{6.639835in}{1.677986in}}{\pgfqpoint{6.636562in}{1.670086in}}{\pgfqpoint{6.636562in}{1.661850in}}%
\pgfpathcurveto{\pgfqpoint{6.636562in}{1.653613in}}{\pgfqpoint{6.639835in}{1.645713in}}{\pgfqpoint{6.645658in}{1.639889in}}%
\pgfpathcurveto{\pgfqpoint{6.651482in}{1.634065in}}{\pgfqpoint{6.659382in}{1.630793in}}{\pgfqpoint{6.667619in}{1.630793in}}%
\pgfpathclose%
\pgfusepath{stroke,fill}%
\end{pgfscope}%
\begin{pgfscope}%
\pgfpathrectangle{\pgfqpoint{0.894063in}{0.630000in}}{\pgfqpoint{6.713438in}{2.060556in}} %
\pgfusepath{clip}%
\pgfsetbuttcap%
\pgfsetroundjoin%
\definecolor{currentfill}{rgb}{0.000000,0.750000,0.750000}%
\pgfsetfillcolor{currentfill}%
\pgfsetlinewidth{1.003750pt}%
\definecolor{currentstroke}{rgb}{0.000000,0.750000,0.750000}%
\pgfsetstrokecolor{currentstroke}%
\pgfsetdash{}{0pt}%
\pgfpathmoveto{\pgfqpoint{2.639556in}{0.946600in}}%
\pgfpathcurveto{\pgfqpoint{2.647793in}{0.946600in}}{\pgfqpoint{2.655693in}{0.949873in}}{\pgfqpoint{2.661517in}{0.955697in}}%
\pgfpathcurveto{\pgfqpoint{2.667340in}{0.961521in}}{\pgfqpoint{2.670613in}{0.969421in}}{\pgfqpoint{2.670613in}{0.977657in}}%
\pgfpathcurveto{\pgfqpoint{2.670613in}{0.985893in}}{\pgfqpoint{2.667340in}{0.993793in}}{\pgfqpoint{2.661517in}{0.999617in}}%
\pgfpathcurveto{\pgfqpoint{2.655693in}{1.005441in}}{\pgfqpoint{2.647793in}{1.008713in}}{\pgfqpoint{2.639556in}{1.008713in}}%
\pgfpathcurveto{\pgfqpoint{2.631320in}{1.008713in}}{\pgfqpoint{2.623420in}{1.005441in}}{\pgfqpoint{2.617596in}{0.999617in}}%
\pgfpathcurveto{\pgfqpoint{2.611772in}{0.993793in}}{\pgfqpoint{2.608500in}{0.985893in}}{\pgfqpoint{2.608500in}{0.977657in}}%
\pgfpathcurveto{\pgfqpoint{2.608500in}{0.969421in}}{\pgfqpoint{2.611772in}{0.961521in}}{\pgfqpoint{2.617596in}{0.955697in}}%
\pgfpathcurveto{\pgfqpoint{2.623420in}{0.949873in}}{\pgfqpoint{2.631320in}{0.946600in}}{\pgfqpoint{2.639556in}{0.946600in}}%
\pgfpathclose%
\pgfusepath{stroke,fill}%
\end{pgfscope}%
\begin{pgfscope}%
\pgfpathrectangle{\pgfqpoint{0.894063in}{0.630000in}}{\pgfqpoint{6.713438in}{2.060556in}} %
\pgfusepath{clip}%
\pgfsetbuttcap%
\pgfsetroundjoin%
\definecolor{currentfill}{rgb}{0.000000,0.750000,0.750000}%
\pgfsetfillcolor{currentfill}%
\pgfsetlinewidth{1.003750pt}%
\definecolor{currentstroke}{rgb}{0.000000,0.750000,0.750000}%
\pgfsetstrokecolor{currentstroke}%
\pgfsetdash{}{0pt}%
\pgfpathmoveto{\pgfqpoint{1.699675in}{0.769357in}}%
\pgfpathcurveto{\pgfqpoint{1.707911in}{0.769357in}}{\pgfqpoint{1.715811in}{0.772630in}}{\pgfqpoint{1.721635in}{0.778454in}}%
\pgfpathcurveto{\pgfqpoint{1.727459in}{0.784277in}}{\pgfqpoint{1.730731in}{0.792178in}}{\pgfqpoint{1.730731in}{0.800414in}}%
\pgfpathcurveto{\pgfqpoint{1.730731in}{0.808650in}}{\pgfqpoint{1.727459in}{0.816550in}}{\pgfqpoint{1.721635in}{0.822374in}}%
\pgfpathcurveto{\pgfqpoint{1.715811in}{0.828198in}}{\pgfqpoint{1.707911in}{0.831470in}}{\pgfqpoint{1.699675in}{0.831470in}}%
\pgfpathcurveto{\pgfqpoint{1.691439in}{0.831470in}}{\pgfqpoint{1.683539in}{0.828198in}}{\pgfqpoint{1.677715in}{0.822374in}}%
\pgfpathcurveto{\pgfqpoint{1.671891in}{0.816550in}}{\pgfqpoint{1.668619in}{0.808650in}}{\pgfqpoint{1.668619in}{0.800414in}}%
\pgfpathcurveto{\pgfqpoint{1.668619in}{0.792178in}}{\pgfqpoint{1.671891in}{0.784277in}}{\pgfqpoint{1.677715in}{0.778454in}}%
\pgfpathcurveto{\pgfqpoint{1.683539in}{0.772630in}}{\pgfqpoint{1.691439in}{0.769357in}}{\pgfqpoint{1.699675in}{0.769357in}}%
\pgfpathclose%
\pgfusepath{stroke,fill}%
\end{pgfscope}%
\begin{pgfscope}%
\pgfpathrectangle{\pgfqpoint{0.894063in}{0.630000in}}{\pgfqpoint{6.713438in}{2.060556in}} %
\pgfusepath{clip}%
\pgfsetbuttcap%
\pgfsetroundjoin%
\definecolor{currentfill}{rgb}{0.000000,0.750000,0.750000}%
\pgfsetfillcolor{currentfill}%
\pgfsetlinewidth{1.003750pt}%
\definecolor{currentstroke}{rgb}{0.000000,0.750000,0.750000}%
\pgfsetstrokecolor{currentstroke}%
\pgfsetdash{}{0pt}%
\pgfpathmoveto{\pgfqpoint{1.162600in}{0.688018in}}%
\pgfpathcurveto{\pgfqpoint{1.170836in}{0.688018in}}{\pgfqpoint{1.178736in}{0.691291in}}{\pgfqpoint{1.184560in}{0.697115in}}%
\pgfpathcurveto{\pgfqpoint{1.190384in}{0.702939in}}{\pgfqpoint{1.193656in}{0.710839in}}{\pgfqpoint{1.193656in}{0.719075in}}%
\pgfpathcurveto{\pgfqpoint{1.193656in}{0.727311in}}{\pgfqpoint{1.190384in}{0.735211in}}{\pgfqpoint{1.184560in}{0.741035in}}%
\pgfpathcurveto{\pgfqpoint{1.178736in}{0.746859in}}{\pgfqpoint{1.170836in}{0.750131in}}{\pgfqpoint{1.162600in}{0.750131in}}%
\pgfpathcurveto{\pgfqpoint{1.154364in}{0.750131in}}{\pgfqpoint{1.146464in}{0.746859in}}{\pgfqpoint{1.140640in}{0.741035in}}%
\pgfpathcurveto{\pgfqpoint{1.134816in}{0.735211in}}{\pgfqpoint{1.131544in}{0.727311in}}{\pgfqpoint{1.131544in}{0.719075in}}%
\pgfpathcurveto{\pgfqpoint{1.131544in}{0.710839in}}{\pgfqpoint{1.134816in}{0.702939in}}{\pgfqpoint{1.140640in}{0.697115in}}%
\pgfpathcurveto{\pgfqpoint{1.146464in}{0.691291in}}{\pgfqpoint{1.154364in}{0.688018in}}{\pgfqpoint{1.162600in}{0.688018in}}%
\pgfpathclose%
\pgfusepath{stroke,fill}%
\end{pgfscope}%
\begin{pgfscope}%
\pgfpathrectangle{\pgfqpoint{0.894063in}{0.630000in}}{\pgfqpoint{6.713438in}{2.060556in}} %
\pgfusepath{clip}%
\pgfsetbuttcap%
\pgfsetroundjoin%
\definecolor{currentfill}{rgb}{0.000000,0.750000,0.750000}%
\pgfsetfillcolor{currentfill}%
\pgfsetlinewidth{1.003750pt}%
\definecolor{currentstroke}{rgb}{0.000000,0.750000,0.750000}%
\pgfsetstrokecolor{currentstroke}%
\pgfsetdash{}{0pt}%
\pgfpathmoveto{\pgfqpoint{1.833944in}{0.796633in}}%
\pgfpathcurveto{\pgfqpoint{1.842180in}{0.796633in}}{\pgfqpoint{1.850080in}{0.799906in}}{\pgfqpoint{1.855904in}{0.805729in}}%
\pgfpathcurveto{\pgfqpoint{1.861728in}{0.811553in}}{\pgfqpoint{1.865000in}{0.819453in}}{\pgfqpoint{1.865000in}{0.827690in}}%
\pgfpathcurveto{\pgfqpoint{1.865000in}{0.835926in}}{\pgfqpoint{1.861728in}{0.843826in}}{\pgfqpoint{1.855904in}{0.849650in}}%
\pgfpathcurveto{\pgfqpoint{1.850080in}{0.855474in}}{\pgfqpoint{1.842180in}{0.858746in}}{\pgfqpoint{1.833944in}{0.858746in}}%
\pgfpathcurveto{\pgfqpoint{1.825707in}{0.858746in}}{\pgfqpoint{1.817807in}{0.855474in}}{\pgfqpoint{1.811983in}{0.849650in}}%
\pgfpathcurveto{\pgfqpoint{1.806160in}{0.843826in}}{\pgfqpoint{1.802887in}{0.835926in}}{\pgfqpoint{1.802887in}{0.827690in}}%
\pgfpathcurveto{\pgfqpoint{1.802887in}{0.819453in}}{\pgfqpoint{1.806160in}{0.811553in}}{\pgfqpoint{1.811983in}{0.805729in}}%
\pgfpathcurveto{\pgfqpoint{1.817807in}{0.799906in}}{\pgfqpoint{1.825707in}{0.796633in}}{\pgfqpoint{1.833944in}{0.796633in}}%
\pgfpathclose%
\pgfusepath{stroke,fill}%
\end{pgfscope}%
\begin{pgfscope}%
\pgfpathrectangle{\pgfqpoint{0.894063in}{0.630000in}}{\pgfqpoint{6.713438in}{2.060556in}} %
\pgfusepath{clip}%
\pgfsetbuttcap%
\pgfsetroundjoin%
\definecolor{currentfill}{rgb}{0.000000,0.750000,0.750000}%
\pgfsetfillcolor{currentfill}%
\pgfsetlinewidth{1.003750pt}%
\definecolor{currentstroke}{rgb}{0.000000,0.750000,0.750000}%
\pgfsetstrokecolor{currentstroke}%
\pgfsetdash{}{0pt}%
\pgfpathmoveto{\pgfqpoint{5.996275in}{1.511204in}}%
\pgfpathcurveto{\pgfqpoint{6.004511in}{1.511204in}}{\pgfqpoint{6.012411in}{1.514477in}}{\pgfqpoint{6.018235in}{1.520301in}}%
\pgfpathcurveto{\pgfqpoint{6.024059in}{1.526125in}}{\pgfqpoint{6.027331in}{1.534025in}}{\pgfqpoint{6.027331in}{1.542261in}}%
\pgfpathcurveto{\pgfqpoint{6.027331in}{1.550497in}}{\pgfqpoint{6.024059in}{1.558397in}}{\pgfqpoint{6.018235in}{1.564221in}}%
\pgfpathcurveto{\pgfqpoint{6.012411in}{1.570045in}}{\pgfqpoint{6.004511in}{1.573317in}}{\pgfqpoint{5.996275in}{1.573317in}}%
\pgfpathcurveto{\pgfqpoint{5.988039in}{1.573317in}}{\pgfqpoint{5.980139in}{1.570045in}}{\pgfqpoint{5.974315in}{1.564221in}}%
\pgfpathcurveto{\pgfqpoint{5.968491in}{1.558397in}}{\pgfqpoint{5.965219in}{1.550497in}}{\pgfqpoint{5.965219in}{1.542261in}}%
\pgfpathcurveto{\pgfqpoint{5.965219in}{1.534025in}}{\pgfqpoint{5.968491in}{1.526125in}}{\pgfqpoint{5.974315in}{1.520301in}}%
\pgfpathcurveto{\pgfqpoint{5.980139in}{1.514477in}}{\pgfqpoint{5.988039in}{1.511204in}}{\pgfqpoint{5.996275in}{1.511204in}}%
\pgfpathclose%
\pgfusepath{stroke,fill}%
\end{pgfscope}%
\begin{pgfscope}%
\pgfpathrectangle{\pgfqpoint{0.894063in}{0.630000in}}{\pgfqpoint{6.713438in}{2.060556in}} %
\pgfusepath{clip}%
\pgfsetbuttcap%
\pgfsetroundjoin%
\definecolor{currentfill}{rgb}{0.000000,0.750000,0.750000}%
\pgfsetfillcolor{currentfill}%
\pgfsetlinewidth{1.003750pt}%
\definecolor{currentstroke}{rgb}{0.000000,0.750000,0.750000}%
\pgfsetstrokecolor{currentstroke}%
\pgfsetdash{}{0pt}%
\pgfpathmoveto{\pgfqpoint{6.399081in}{1.576183in}}%
\pgfpathcurveto{\pgfqpoint{6.407318in}{1.576183in}}{\pgfqpoint{6.415218in}{1.579455in}}{\pgfqpoint{6.421042in}{1.585279in}}%
\pgfpathcurveto{\pgfqpoint{6.426865in}{1.591103in}}{\pgfqpoint{6.430138in}{1.599003in}}{\pgfqpoint{6.430138in}{1.607239in}}%
\pgfpathcurveto{\pgfqpoint{6.430138in}{1.615475in}}{\pgfqpoint{6.426865in}{1.623375in}}{\pgfqpoint{6.421042in}{1.629199in}}%
\pgfpathcurveto{\pgfqpoint{6.415218in}{1.635023in}}{\pgfqpoint{6.407318in}{1.638296in}}{\pgfqpoint{6.399081in}{1.638296in}}%
\pgfpathcurveto{\pgfqpoint{6.390845in}{1.638296in}}{\pgfqpoint{6.382945in}{1.635023in}}{\pgfqpoint{6.377121in}{1.629199in}}%
\pgfpathcurveto{\pgfqpoint{6.371297in}{1.623375in}}{\pgfqpoint{6.368025in}{1.615475in}}{\pgfqpoint{6.368025in}{1.607239in}}%
\pgfpathcurveto{\pgfqpoint{6.368025in}{1.599003in}}{\pgfqpoint{6.371297in}{1.591103in}}{\pgfqpoint{6.377121in}{1.585279in}}%
\pgfpathcurveto{\pgfqpoint{6.382945in}{1.579455in}}{\pgfqpoint{6.390845in}{1.576183in}}{\pgfqpoint{6.399081in}{1.576183in}}%
\pgfpathclose%
\pgfusepath{stroke,fill}%
\end{pgfscope}%
\begin{pgfscope}%
\pgfpathrectangle{\pgfqpoint{0.894063in}{0.630000in}}{\pgfqpoint{6.713438in}{2.060556in}} %
\pgfusepath{clip}%
\pgfsetbuttcap%
\pgfsetroundjoin%
\definecolor{currentfill}{rgb}{0.000000,0.750000,0.750000}%
\pgfsetfillcolor{currentfill}%
\pgfsetlinewidth{1.003750pt}%
\definecolor{currentstroke}{rgb}{0.000000,0.750000,0.750000}%
\pgfsetstrokecolor{currentstroke}%
\pgfsetdash{}{0pt}%
\pgfpathmoveto{\pgfqpoint{4.787856in}{1.295429in}}%
\pgfpathcurveto{\pgfqpoint{4.796093in}{1.295429in}}{\pgfqpoint{4.803993in}{1.298701in}}{\pgfqpoint{4.809817in}{1.304525in}}%
\pgfpathcurveto{\pgfqpoint{4.815640in}{1.310349in}}{\pgfqpoint{4.818913in}{1.318249in}}{\pgfqpoint{4.818913in}{1.326485in}}%
\pgfpathcurveto{\pgfqpoint{4.818913in}{1.334722in}}{\pgfqpoint{4.815640in}{1.342622in}}{\pgfqpoint{4.809817in}{1.348446in}}%
\pgfpathcurveto{\pgfqpoint{4.803993in}{1.354270in}}{\pgfqpoint{4.796093in}{1.357542in}}{\pgfqpoint{4.787856in}{1.357542in}}%
\pgfpathcurveto{\pgfqpoint{4.779620in}{1.357542in}}{\pgfqpoint{4.771720in}{1.354270in}}{\pgfqpoint{4.765896in}{1.348446in}}%
\pgfpathcurveto{\pgfqpoint{4.760072in}{1.342622in}}{\pgfqpoint{4.756800in}{1.334722in}}{\pgfqpoint{4.756800in}{1.326485in}}%
\pgfpathcurveto{\pgfqpoint{4.756800in}{1.318249in}}{\pgfqpoint{4.760072in}{1.310349in}}{\pgfqpoint{4.765896in}{1.304525in}}%
\pgfpathcurveto{\pgfqpoint{4.771720in}{1.298701in}}{\pgfqpoint{4.779620in}{1.295429in}}{\pgfqpoint{4.787856in}{1.295429in}}%
\pgfpathclose%
\pgfusepath{stroke,fill}%
\end{pgfscope}%
\begin{pgfscope}%
\pgfpathrectangle{\pgfqpoint{0.894063in}{0.630000in}}{\pgfqpoint{6.713438in}{2.060556in}} %
\pgfusepath{clip}%
\pgfsetbuttcap%
\pgfsetroundjoin%
\definecolor{currentfill}{rgb}{0.000000,0.750000,0.750000}%
\pgfsetfillcolor{currentfill}%
\pgfsetlinewidth{1.003750pt}%
\definecolor{currentstroke}{rgb}{0.000000,0.750000,0.750000}%
\pgfsetstrokecolor{currentstroke}%
\pgfsetdash{}{0pt}%
\pgfpathmoveto{\pgfqpoint{4.922125in}{1.319732in}}%
\pgfpathcurveto{\pgfqpoint{4.930361in}{1.319732in}}{\pgfqpoint{4.938261in}{1.323004in}}{\pgfqpoint{4.944085in}{1.328828in}}%
\pgfpathcurveto{\pgfqpoint{4.949909in}{1.334652in}}{\pgfqpoint{4.953181in}{1.342552in}}{\pgfqpoint{4.953181in}{1.350788in}}%
\pgfpathcurveto{\pgfqpoint{4.953181in}{1.359025in}}{\pgfqpoint{4.949909in}{1.366925in}}{\pgfqpoint{4.944085in}{1.372748in}}%
\pgfpathcurveto{\pgfqpoint{4.938261in}{1.378572in}}{\pgfqpoint{4.930361in}{1.381845in}}{\pgfqpoint{4.922125in}{1.381845in}}%
\pgfpathcurveto{\pgfqpoint{4.913889in}{1.381845in}}{\pgfqpoint{4.905989in}{1.378572in}}{\pgfqpoint{4.900165in}{1.372748in}}%
\pgfpathcurveto{\pgfqpoint{4.894341in}{1.366925in}}{\pgfqpoint{4.891069in}{1.359025in}}{\pgfqpoint{4.891069in}{1.350788in}}%
\pgfpathcurveto{\pgfqpoint{4.891069in}{1.342552in}}{\pgfqpoint{4.894341in}{1.334652in}}{\pgfqpoint{4.900165in}{1.328828in}}%
\pgfpathcurveto{\pgfqpoint{4.905989in}{1.323004in}}{\pgfqpoint{4.913889in}{1.319732in}}{\pgfqpoint{4.922125in}{1.319732in}}%
\pgfpathclose%
\pgfusepath{stroke,fill}%
\end{pgfscope}%
\begin{pgfscope}%
\pgfpathrectangle{\pgfqpoint{0.894063in}{0.630000in}}{\pgfqpoint{6.713438in}{2.060556in}} %
\pgfusepath{clip}%
\pgfsetbuttcap%
\pgfsetroundjoin%
\definecolor{currentfill}{rgb}{0.000000,0.750000,0.750000}%
\pgfsetfillcolor{currentfill}%
\pgfsetlinewidth{1.003750pt}%
\definecolor{currentstroke}{rgb}{0.000000,0.750000,0.750000}%
\pgfsetstrokecolor{currentstroke}%
\pgfsetdash{}{0pt}%
\pgfpathmoveto{\pgfqpoint{6.130544in}{1.532146in}}%
\pgfpathcurveto{\pgfqpoint{6.138780in}{1.532146in}}{\pgfqpoint{6.146680in}{1.535418in}}{\pgfqpoint{6.152504in}{1.541242in}}%
\pgfpathcurveto{\pgfqpoint{6.158328in}{1.547066in}}{\pgfqpoint{6.161600in}{1.554966in}}{\pgfqpoint{6.161600in}{1.563202in}}%
\pgfpathcurveto{\pgfqpoint{6.161600in}{1.571438in}}{\pgfqpoint{6.158328in}{1.579338in}}{\pgfqpoint{6.152504in}{1.585162in}}%
\pgfpathcurveto{\pgfqpoint{6.146680in}{1.590986in}}{\pgfqpoint{6.138780in}{1.594259in}}{\pgfqpoint{6.130544in}{1.594259in}}%
\pgfpathcurveto{\pgfqpoint{6.122307in}{1.594259in}}{\pgfqpoint{6.114407in}{1.590986in}}{\pgfqpoint{6.108583in}{1.585162in}}%
\pgfpathcurveto{\pgfqpoint{6.102760in}{1.579338in}}{\pgfqpoint{6.099487in}{1.571438in}}{\pgfqpoint{6.099487in}{1.563202in}}%
\pgfpathcurveto{\pgfqpoint{6.099487in}{1.554966in}}{\pgfqpoint{6.102760in}{1.547066in}}{\pgfqpoint{6.108583in}{1.541242in}}%
\pgfpathcurveto{\pgfqpoint{6.114407in}{1.535418in}}{\pgfqpoint{6.122307in}{1.532146in}}{\pgfqpoint{6.130544in}{1.532146in}}%
\pgfpathclose%
\pgfusepath{stroke,fill}%
\end{pgfscope}%
\begin{pgfscope}%
\pgfpathrectangle{\pgfqpoint{0.894063in}{0.630000in}}{\pgfqpoint{6.713438in}{2.060556in}} %
\pgfusepath{clip}%
\pgfsetbuttcap%
\pgfsetroundjoin%
\definecolor{currentfill}{rgb}{0.000000,0.750000,0.750000}%
\pgfsetfillcolor{currentfill}%
\pgfsetlinewidth{1.003750pt}%
\definecolor{currentstroke}{rgb}{0.000000,0.750000,0.750000}%
\pgfsetstrokecolor{currentstroke}%
\pgfsetdash{}{0pt}%
\pgfpathmoveto{\pgfqpoint{5.727738in}{1.463635in}}%
\pgfpathcurveto{\pgfqpoint{5.735974in}{1.463635in}}{\pgfqpoint{5.743874in}{1.466907in}}{\pgfqpoint{5.749698in}{1.472731in}}%
\pgfpathcurveto{\pgfqpoint{5.755522in}{1.478555in}}{\pgfqpoint{5.758794in}{1.486455in}}{\pgfqpoint{5.758794in}{1.494692in}}%
\pgfpathcurveto{\pgfqpoint{5.758794in}{1.502928in}}{\pgfqpoint{5.755522in}{1.510828in}}{\pgfqpoint{5.749698in}{1.516652in}}%
\pgfpathcurveto{\pgfqpoint{5.743874in}{1.522476in}}{\pgfqpoint{5.735974in}{1.525748in}}{\pgfqpoint{5.727738in}{1.525748in}}%
\pgfpathcurveto{\pgfqpoint{5.719501in}{1.525748in}}{\pgfqpoint{5.711601in}{1.522476in}}{\pgfqpoint{5.705777in}{1.516652in}}%
\pgfpathcurveto{\pgfqpoint{5.699953in}{1.510828in}}{\pgfqpoint{5.696681in}{1.502928in}}{\pgfqpoint{5.696681in}{1.494692in}}%
\pgfpathcurveto{\pgfqpoint{5.696681in}{1.486455in}}{\pgfqpoint{5.699953in}{1.478555in}}{\pgfqpoint{5.705777in}{1.472731in}}%
\pgfpathcurveto{\pgfqpoint{5.711601in}{1.466907in}}{\pgfqpoint{5.719501in}{1.463635in}}{\pgfqpoint{5.727738in}{1.463635in}}%
\pgfpathclose%
\pgfusepath{stroke,fill}%
\end{pgfscope}%
\begin{pgfscope}%
\pgfpathrectangle{\pgfqpoint{0.894063in}{0.630000in}}{\pgfqpoint{6.713438in}{2.060556in}} %
\pgfusepath{clip}%
\pgfsetbuttcap%
\pgfsetroundjoin%
\definecolor{currentfill}{rgb}{0.000000,0.750000,0.750000}%
\pgfsetfillcolor{currentfill}%
\pgfsetlinewidth{1.003750pt}%
\definecolor{currentstroke}{rgb}{0.000000,0.750000,0.750000}%
\pgfsetstrokecolor{currentstroke}%
\pgfsetdash{}{0pt}%
\pgfpathmoveto{\pgfqpoint{1.028331in}{0.666106in}}%
\pgfpathcurveto{\pgfqpoint{1.036568in}{0.666106in}}{\pgfqpoint{1.044468in}{0.669378in}}{\pgfqpoint{1.050292in}{0.675202in}}%
\pgfpathcurveto{\pgfqpoint{1.056115in}{0.681026in}}{\pgfqpoint{1.059388in}{0.688926in}}{\pgfqpoint{1.059388in}{0.697162in}}%
\pgfpathcurveto{\pgfqpoint{1.059388in}{0.705399in}}{\pgfqpoint{1.056115in}{0.713299in}}{\pgfqpoint{1.050292in}{0.719123in}}%
\pgfpathcurveto{\pgfqpoint{1.044468in}{0.724947in}}{\pgfqpoint{1.036568in}{0.728219in}}{\pgfqpoint{1.028331in}{0.728219in}}%
\pgfpathcurveto{\pgfqpoint{1.020095in}{0.728219in}}{\pgfqpoint{1.012195in}{0.724947in}}{\pgfqpoint{1.006371in}{0.719123in}}%
\pgfpathcurveto{\pgfqpoint{1.000547in}{0.713299in}}{\pgfqpoint{0.997275in}{0.705399in}}{\pgfqpoint{0.997275in}{0.697162in}}%
\pgfpathcurveto{\pgfqpoint{0.997275in}{0.688926in}}{\pgfqpoint{1.000547in}{0.681026in}}{\pgfqpoint{1.006371in}{0.675202in}}%
\pgfpathcurveto{\pgfqpoint{1.012195in}{0.669378in}}{\pgfqpoint{1.020095in}{0.666106in}}{\pgfqpoint{1.028331in}{0.666106in}}%
\pgfpathclose%
\pgfusepath{stroke,fill}%
\end{pgfscope}%
\begin{pgfscope}%
\pgfpathrectangle{\pgfqpoint{0.894063in}{0.630000in}}{\pgfqpoint{6.713438in}{2.060556in}} %
\pgfusepath{clip}%
\pgfsetbuttcap%
\pgfsetroundjoin%
\definecolor{currentfill}{rgb}{0.000000,0.750000,0.750000}%
\pgfsetfillcolor{currentfill}%
\pgfsetlinewidth{1.003750pt}%
\definecolor{currentstroke}{rgb}{0.000000,0.750000,0.750000}%
\pgfsetstrokecolor{currentstroke}%
\pgfsetdash{}{0pt}%
\pgfpathmoveto{\pgfqpoint{5.324931in}{1.395749in}}%
\pgfpathcurveto{\pgfqpoint{5.333168in}{1.395749in}}{\pgfqpoint{5.341068in}{1.399021in}}{\pgfqpoint{5.346892in}{1.404845in}}%
\pgfpathcurveto{\pgfqpoint{5.352715in}{1.410669in}}{\pgfqpoint{5.355988in}{1.418569in}}{\pgfqpoint{5.355988in}{1.426805in}}%
\pgfpathcurveto{\pgfqpoint{5.355988in}{1.435041in}}{\pgfqpoint{5.352715in}{1.442941in}}{\pgfqpoint{5.346892in}{1.448765in}}%
\pgfpathcurveto{\pgfqpoint{5.341068in}{1.454589in}}{\pgfqpoint{5.333168in}{1.457862in}}{\pgfqpoint{5.324931in}{1.457862in}}%
\pgfpathcurveto{\pgfqpoint{5.316695in}{1.457862in}}{\pgfqpoint{5.308795in}{1.454589in}}{\pgfqpoint{5.302971in}{1.448765in}}%
\pgfpathcurveto{\pgfqpoint{5.297147in}{1.442941in}}{\pgfqpoint{5.293875in}{1.435041in}}{\pgfqpoint{5.293875in}{1.426805in}}%
\pgfpathcurveto{\pgfqpoint{5.293875in}{1.418569in}}{\pgfqpoint{5.297147in}{1.410669in}}{\pgfqpoint{5.302971in}{1.404845in}}%
\pgfpathcurveto{\pgfqpoint{5.308795in}{1.399021in}}{\pgfqpoint{5.316695in}{1.395749in}}{\pgfqpoint{5.324931in}{1.395749in}}%
\pgfpathclose%
\pgfusepath{stroke,fill}%
\end{pgfscope}%
\begin{pgfscope}%
\pgfpathrectangle{\pgfqpoint{0.894063in}{0.630000in}}{\pgfqpoint{6.713438in}{2.060556in}} %
\pgfusepath{clip}%
\pgfsetbuttcap%
\pgfsetroundjoin%
\definecolor{currentfill}{rgb}{0.000000,0.750000,0.750000}%
\pgfsetfillcolor{currentfill}%
\pgfsetlinewidth{1.003750pt}%
\definecolor{currentstroke}{rgb}{0.000000,0.750000,0.750000}%
\pgfsetstrokecolor{currentstroke}%
\pgfsetdash{}{0pt}%
\pgfpathmoveto{\pgfqpoint{7.338963in}{1.738637in}}%
\pgfpathcurveto{\pgfqpoint{7.347199in}{1.738637in}}{\pgfqpoint{7.355099in}{1.741909in}}{\pgfqpoint{7.360923in}{1.747733in}}%
\pgfpathcurveto{\pgfqpoint{7.366747in}{1.753557in}}{\pgfqpoint{7.370019in}{1.761457in}}{\pgfqpoint{7.370019in}{1.769693in}}%
\pgfpathcurveto{\pgfqpoint{7.370019in}{1.777930in}}{\pgfqpoint{7.366747in}{1.785830in}}{\pgfqpoint{7.360923in}{1.791654in}}%
\pgfpathcurveto{\pgfqpoint{7.355099in}{1.797477in}}{\pgfqpoint{7.347199in}{1.800750in}}{\pgfqpoint{7.338963in}{1.800750in}}%
\pgfpathcurveto{\pgfqpoint{7.330726in}{1.800750in}}{\pgfqpoint{7.322826in}{1.797477in}}{\pgfqpoint{7.317002in}{1.791654in}}%
\pgfpathcurveto{\pgfqpoint{7.311178in}{1.785830in}}{\pgfqpoint{7.307906in}{1.777930in}}{\pgfqpoint{7.307906in}{1.769693in}}%
\pgfpathcurveto{\pgfqpoint{7.307906in}{1.761457in}}{\pgfqpoint{7.311178in}{1.753557in}}{\pgfqpoint{7.317002in}{1.747733in}}%
\pgfpathcurveto{\pgfqpoint{7.322826in}{1.741909in}}{\pgfqpoint{7.330726in}{1.738637in}}{\pgfqpoint{7.338963in}{1.738637in}}%
\pgfpathclose%
\pgfusepath{stroke,fill}%
\end{pgfscope}%
\begin{pgfscope}%
\pgfpathrectangle{\pgfqpoint{0.894063in}{0.630000in}}{\pgfqpoint{6.713438in}{2.060556in}} %
\pgfusepath{clip}%
\pgfsetbuttcap%
\pgfsetroundjoin%
\definecolor{currentfill}{rgb}{0.000000,0.750000,0.750000}%
\pgfsetfillcolor{currentfill}%
\pgfsetlinewidth{1.003750pt}%
\definecolor{currentstroke}{rgb}{0.000000,0.750000,0.750000}%
\pgfsetstrokecolor{currentstroke}%
\pgfsetdash{}{0pt}%
\pgfpathmoveto{\pgfqpoint{7.204694in}{1.714505in}}%
\pgfpathcurveto{\pgfqpoint{7.212930in}{1.714505in}}{\pgfqpoint{7.220830in}{1.717777in}}{\pgfqpoint{7.226654in}{1.723601in}}%
\pgfpathcurveto{\pgfqpoint{7.232478in}{1.729425in}}{\pgfqpoint{7.235750in}{1.737325in}}{\pgfqpoint{7.235750in}{1.745561in}}%
\pgfpathcurveto{\pgfqpoint{7.235750in}{1.753798in}}{\pgfqpoint{7.232478in}{1.761698in}}{\pgfqpoint{7.226654in}{1.767521in}}%
\pgfpathcurveto{\pgfqpoint{7.220830in}{1.773345in}}{\pgfqpoint{7.212930in}{1.776618in}}{\pgfqpoint{7.204694in}{1.776618in}}%
\pgfpathcurveto{\pgfqpoint{7.196457in}{1.776618in}}{\pgfqpoint{7.188557in}{1.773345in}}{\pgfqpoint{7.182733in}{1.767521in}}%
\pgfpathcurveto{\pgfqpoint{7.176910in}{1.761698in}}{\pgfqpoint{7.173637in}{1.753798in}}{\pgfqpoint{7.173637in}{1.745561in}}%
\pgfpathcurveto{\pgfqpoint{7.173637in}{1.737325in}}{\pgfqpoint{7.176910in}{1.729425in}}{\pgfqpoint{7.182733in}{1.723601in}}%
\pgfpathcurveto{\pgfqpoint{7.188557in}{1.717777in}}{\pgfqpoint{7.196457in}{1.714505in}}{\pgfqpoint{7.204694in}{1.714505in}}%
\pgfpathclose%
\pgfusepath{stroke,fill}%
\end{pgfscope}%
\begin{pgfscope}%
\pgfpathrectangle{\pgfqpoint{0.894063in}{0.630000in}}{\pgfqpoint{6.713438in}{2.060556in}} %
\pgfusepath{clip}%
\pgfsetbuttcap%
\pgfsetroundjoin%
\definecolor{currentfill}{rgb}{0.000000,0.750000,0.750000}%
\pgfsetfillcolor{currentfill}%
\pgfsetlinewidth{1.003750pt}%
\definecolor{currentstroke}{rgb}{0.000000,0.750000,0.750000}%
\pgfsetstrokecolor{currentstroke}%
\pgfsetdash{}{0pt}%
\pgfpathmoveto{\pgfqpoint{6.264813in}{1.555742in}}%
\pgfpathcurveto{\pgfqpoint{6.273049in}{1.555742in}}{\pgfqpoint{6.280949in}{1.559014in}}{\pgfqpoint{6.286773in}{1.564838in}}%
\pgfpathcurveto{\pgfqpoint{6.292597in}{1.570662in}}{\pgfqpoint{6.295869in}{1.578562in}}{\pgfqpoint{6.295869in}{1.586798in}}%
\pgfpathcurveto{\pgfqpoint{6.295869in}{1.595035in}}{\pgfqpoint{6.292597in}{1.602935in}}{\pgfqpoint{6.286773in}{1.608759in}}%
\pgfpathcurveto{\pgfqpoint{6.280949in}{1.614583in}}{\pgfqpoint{6.273049in}{1.617855in}}{\pgfqpoint{6.264813in}{1.617855in}}%
\pgfpathcurveto{\pgfqpoint{6.256576in}{1.617855in}}{\pgfqpoint{6.248676in}{1.614583in}}{\pgfqpoint{6.242852in}{1.608759in}}%
\pgfpathcurveto{\pgfqpoint{6.237028in}{1.602935in}}{\pgfqpoint{6.233756in}{1.595035in}}{\pgfqpoint{6.233756in}{1.586798in}}%
\pgfpathcurveto{\pgfqpoint{6.233756in}{1.578562in}}{\pgfqpoint{6.237028in}{1.570662in}}{\pgfqpoint{6.242852in}{1.564838in}}%
\pgfpathcurveto{\pgfqpoint{6.248676in}{1.559014in}}{\pgfqpoint{6.256576in}{1.555742in}}{\pgfqpoint{6.264813in}{1.555742in}}%
\pgfpathclose%
\pgfusepath{stroke,fill}%
\end{pgfscope}%
\begin{pgfscope}%
\pgfpathrectangle{\pgfqpoint{0.894063in}{0.630000in}}{\pgfqpoint{6.713438in}{2.060556in}} %
\pgfusepath{clip}%
\pgfsetbuttcap%
\pgfsetroundjoin%
\definecolor{currentfill}{rgb}{0.000000,0.750000,0.750000}%
\pgfsetfillcolor{currentfill}%
\pgfsetlinewidth{1.003750pt}%
\definecolor{currentstroke}{rgb}{0.000000,0.750000,0.750000}%
\pgfsetstrokecolor{currentstroke}%
\pgfsetdash{}{0pt}%
\pgfpathmoveto{\pgfqpoint{7.473231in}{1.747491in}}%
\pgfpathcurveto{\pgfqpoint{7.481468in}{1.747491in}}{\pgfqpoint{7.489368in}{1.750764in}}{\pgfqpoint{7.495192in}{1.756588in}}%
\pgfpathcurveto{\pgfqpoint{7.501015in}{1.762411in}}{\pgfqpoint{7.504288in}{1.770311in}}{\pgfqpoint{7.504288in}{1.778548in}}%
\pgfpathcurveto{\pgfqpoint{7.504288in}{1.786784in}}{\pgfqpoint{7.501015in}{1.794684in}}{\pgfqpoint{7.495192in}{1.800508in}}%
\pgfpathcurveto{\pgfqpoint{7.489368in}{1.806332in}}{\pgfqpoint{7.481468in}{1.809604in}}{\pgfqpoint{7.473231in}{1.809604in}}%
\pgfpathcurveto{\pgfqpoint{7.464995in}{1.809604in}}{\pgfqpoint{7.457095in}{1.806332in}}{\pgfqpoint{7.451271in}{1.800508in}}%
\pgfpathcurveto{\pgfqpoint{7.445447in}{1.794684in}}{\pgfqpoint{7.442175in}{1.786784in}}{\pgfqpoint{7.442175in}{1.778548in}}%
\pgfpathcurveto{\pgfqpoint{7.442175in}{1.770311in}}{\pgfqpoint{7.445447in}{1.762411in}}{\pgfqpoint{7.451271in}{1.756588in}}%
\pgfpathcurveto{\pgfqpoint{7.457095in}{1.750764in}}{\pgfqpoint{7.464995in}{1.747491in}}{\pgfqpoint{7.473231in}{1.747491in}}%
\pgfpathclose%
\pgfusepath{stroke,fill}%
\end{pgfscope}%
\begin{pgfscope}%
\pgfpathrectangle{\pgfqpoint{0.894063in}{0.630000in}}{\pgfqpoint{6.713438in}{2.060556in}} %
\pgfusepath{clip}%
\pgfsetbuttcap%
\pgfsetroundjoin%
\definecolor{currentfill}{rgb}{0.000000,0.750000,0.750000}%
\pgfsetfillcolor{currentfill}%
\pgfsetlinewidth{1.003750pt}%
\definecolor{currentstroke}{rgb}{0.000000,0.750000,0.750000}%
\pgfsetstrokecolor{currentstroke}%
\pgfsetdash{}{0pt}%
\pgfpathmoveto{\pgfqpoint{5.056394in}{1.340738in}}%
\pgfpathcurveto{\pgfqpoint{5.064630in}{1.340738in}}{\pgfqpoint{5.072530in}{1.344010in}}{\pgfqpoint{5.078354in}{1.349834in}}%
\pgfpathcurveto{\pgfqpoint{5.084178in}{1.355658in}}{\pgfqpoint{5.087450in}{1.363558in}}{\pgfqpoint{5.087450in}{1.371794in}}%
\pgfpathcurveto{\pgfqpoint{5.087450in}{1.380030in}}{\pgfqpoint{5.084178in}{1.387930in}}{\pgfqpoint{5.078354in}{1.393754in}}%
\pgfpathcurveto{\pgfqpoint{5.072530in}{1.399578in}}{\pgfqpoint{5.064630in}{1.402851in}}{\pgfqpoint{5.056394in}{1.402851in}}%
\pgfpathcurveto{\pgfqpoint{5.048157in}{1.402851in}}{\pgfqpoint{5.040257in}{1.399578in}}{\pgfqpoint{5.034433in}{1.393754in}}%
\pgfpathcurveto{\pgfqpoint{5.028610in}{1.387930in}}{\pgfqpoint{5.025337in}{1.380030in}}{\pgfqpoint{5.025337in}{1.371794in}}%
\pgfpathcurveto{\pgfqpoint{5.025337in}{1.363558in}}{\pgfqpoint{5.028610in}{1.355658in}}{\pgfqpoint{5.034433in}{1.349834in}}%
\pgfpathcurveto{\pgfqpoint{5.040257in}{1.344010in}}{\pgfqpoint{5.048157in}{1.340738in}}{\pgfqpoint{5.056394in}{1.340738in}}%
\pgfpathclose%
\pgfusepath{stroke,fill}%
\end{pgfscope}%
\begin{pgfscope}%
\pgfpathrectangle{\pgfqpoint{0.894063in}{0.630000in}}{\pgfqpoint{6.713438in}{2.060556in}} %
\pgfusepath{clip}%
\pgfsetbuttcap%
\pgfsetroundjoin%
\definecolor{currentfill}{rgb}{0.000000,0.750000,0.750000}%
\pgfsetfillcolor{currentfill}%
\pgfsetlinewidth{1.003750pt}%
\definecolor{currentstroke}{rgb}{0.000000,0.750000,0.750000}%
\pgfsetstrokecolor{currentstroke}%
\pgfsetdash{}{0pt}%
\pgfpathmoveto{\pgfqpoint{2.908094in}{0.989354in}}%
\pgfpathcurveto{\pgfqpoint{2.916330in}{0.989354in}}{\pgfqpoint{2.924230in}{0.992626in}}{\pgfqpoint{2.930054in}{0.998450in}}%
\pgfpathcurveto{\pgfqpoint{2.935878in}{1.004274in}}{\pgfqpoint{2.939150in}{1.012174in}}{\pgfqpoint{2.939150in}{1.020411in}}%
\pgfpathcurveto{\pgfqpoint{2.939150in}{1.028647in}}{\pgfqpoint{2.935878in}{1.036547in}}{\pgfqpoint{2.930054in}{1.042371in}}%
\pgfpathcurveto{\pgfqpoint{2.924230in}{1.048195in}}{\pgfqpoint{2.916330in}{1.051467in}}{\pgfqpoint{2.908094in}{1.051467in}}%
\pgfpathcurveto{\pgfqpoint{2.899857in}{1.051467in}}{\pgfqpoint{2.891957in}{1.048195in}}{\pgfqpoint{2.886133in}{1.042371in}}%
\pgfpathcurveto{\pgfqpoint{2.880310in}{1.036547in}}{\pgfqpoint{2.877037in}{1.028647in}}{\pgfqpoint{2.877037in}{1.020411in}}%
\pgfpathcurveto{\pgfqpoint{2.877037in}{1.012174in}}{\pgfqpoint{2.880310in}{1.004274in}}{\pgfqpoint{2.886133in}{0.998450in}}%
\pgfpathcurveto{\pgfqpoint{2.891957in}{0.992626in}}{\pgfqpoint{2.899857in}{0.989354in}}{\pgfqpoint{2.908094in}{0.989354in}}%
\pgfpathclose%
\pgfusepath{stroke,fill}%
\end{pgfscope}%
\begin{pgfscope}%
\pgfpathrectangle{\pgfqpoint{0.894063in}{0.630000in}}{\pgfqpoint{6.713438in}{2.060556in}} %
\pgfusepath{clip}%
\pgfsetbuttcap%
\pgfsetroundjoin%
\definecolor{currentfill}{rgb}{0.000000,0.750000,0.750000}%
\pgfsetfillcolor{currentfill}%
\pgfsetlinewidth{1.003750pt}%
\definecolor{currentstroke}{rgb}{0.000000,0.750000,0.750000}%
\pgfsetstrokecolor{currentstroke}%
\pgfsetdash{}{0pt}%
\pgfpathmoveto{\pgfqpoint{3.445169in}{1.074019in}}%
\pgfpathcurveto{\pgfqpoint{3.453405in}{1.074019in}}{\pgfqpoint{3.461305in}{1.077292in}}{\pgfqpoint{3.467129in}{1.083116in}}%
\pgfpathcurveto{\pgfqpoint{3.472953in}{1.088939in}}{\pgfqpoint{3.476225in}{1.096840in}}{\pgfqpoint{3.476225in}{1.105076in}}%
\pgfpathcurveto{\pgfqpoint{3.476225in}{1.113312in}}{\pgfqpoint{3.472953in}{1.121212in}}{\pgfqpoint{3.467129in}{1.127036in}}%
\pgfpathcurveto{\pgfqpoint{3.461305in}{1.132860in}}{\pgfqpoint{3.453405in}{1.136132in}}{\pgfqpoint{3.445169in}{1.136132in}}%
\pgfpathcurveto{\pgfqpoint{3.436932in}{1.136132in}}{\pgfqpoint{3.429032in}{1.132860in}}{\pgfqpoint{3.423208in}{1.127036in}}%
\pgfpathcurveto{\pgfqpoint{3.417385in}{1.121212in}}{\pgfqpoint{3.414112in}{1.113312in}}{\pgfqpoint{3.414112in}{1.105076in}}%
\pgfpathcurveto{\pgfqpoint{3.414112in}{1.096840in}}{\pgfqpoint{3.417385in}{1.088939in}}{\pgfqpoint{3.423208in}{1.083116in}}%
\pgfpathcurveto{\pgfqpoint{3.429032in}{1.077292in}}{\pgfqpoint{3.436932in}{1.074019in}}{\pgfqpoint{3.445169in}{1.074019in}}%
\pgfpathclose%
\pgfusepath{stroke,fill}%
\end{pgfscope}%
\begin{pgfscope}%
\pgfpathrectangle{\pgfqpoint{0.894063in}{0.630000in}}{\pgfqpoint{6.713438in}{2.060556in}} %
\pgfusepath{clip}%
\pgfsetbuttcap%
\pgfsetroundjoin%
\definecolor{currentfill}{rgb}{0.000000,0.750000,0.750000}%
\pgfsetfillcolor{currentfill}%
\pgfsetlinewidth{1.003750pt}%
\definecolor{currentstroke}{rgb}{0.000000,0.750000,0.750000}%
\pgfsetstrokecolor{currentstroke}%
\pgfsetdash{}{0pt}%
\pgfpathmoveto{\pgfqpoint{4.116513in}{1.181998in}}%
\pgfpathcurveto{\pgfqpoint{4.124749in}{1.181998in}}{\pgfqpoint{4.132649in}{1.185271in}}{\pgfqpoint{4.138473in}{1.191095in}}%
\pgfpathcurveto{\pgfqpoint{4.144297in}{1.196918in}}{\pgfqpoint{4.147569in}{1.204819in}}{\pgfqpoint{4.147569in}{1.213055in}}%
\pgfpathcurveto{\pgfqpoint{4.147569in}{1.221291in}}{\pgfqpoint{4.144297in}{1.229191in}}{\pgfqpoint{4.138473in}{1.235015in}}%
\pgfpathcurveto{\pgfqpoint{4.132649in}{1.240839in}}{\pgfqpoint{4.124749in}{1.244111in}}{\pgfqpoint{4.116513in}{1.244111in}}%
\pgfpathcurveto{\pgfqpoint{4.108276in}{1.244111in}}{\pgfqpoint{4.100376in}{1.240839in}}{\pgfqpoint{4.094552in}{1.235015in}}%
\pgfpathcurveto{\pgfqpoint{4.088728in}{1.229191in}}{\pgfqpoint{4.085456in}{1.221291in}}{\pgfqpoint{4.085456in}{1.213055in}}%
\pgfpathcurveto{\pgfqpoint{4.085456in}{1.204819in}}{\pgfqpoint{4.088728in}{1.196918in}}{\pgfqpoint{4.094552in}{1.191095in}}%
\pgfpathcurveto{\pgfqpoint{4.100376in}{1.185271in}}{\pgfqpoint{4.108276in}{1.181998in}}{\pgfqpoint{4.116513in}{1.181998in}}%
\pgfpathclose%
\pgfusepath{stroke,fill}%
\end{pgfscope}%
\begin{pgfscope}%
\pgfpathrectangle{\pgfqpoint{0.894063in}{0.630000in}}{\pgfqpoint{6.713438in}{2.060556in}} %
\pgfusepath{clip}%
\pgfsetbuttcap%
\pgfsetroundjoin%
\definecolor{currentfill}{rgb}{0.000000,0.750000,0.750000}%
\pgfsetfillcolor{currentfill}%
\pgfsetlinewidth{1.003750pt}%
\definecolor{currentstroke}{rgb}{0.000000,0.750000,0.750000}%
\pgfsetstrokecolor{currentstroke}%
\pgfsetdash{}{0pt}%
\pgfpathmoveto{\pgfqpoint{1.431138in}{0.724231in}}%
\pgfpathcurveto{\pgfqpoint{1.439374in}{0.724231in}}{\pgfqpoint{1.447274in}{0.727503in}}{\pgfqpoint{1.453098in}{0.733327in}}%
\pgfpathcurveto{\pgfqpoint{1.458922in}{0.739151in}}{\pgfqpoint{1.462194in}{0.747051in}}{\pgfqpoint{1.462194in}{0.755288in}}%
\pgfpathcurveto{\pgfqpoint{1.462194in}{0.763524in}}{\pgfqpoint{1.458922in}{0.771424in}}{\pgfqpoint{1.453098in}{0.777248in}}%
\pgfpathcurveto{\pgfqpoint{1.447274in}{0.783072in}}{\pgfqpoint{1.439374in}{0.786344in}}{\pgfqpoint{1.431138in}{0.786344in}}%
\pgfpathcurveto{\pgfqpoint{1.422901in}{0.786344in}}{\pgfqpoint{1.415001in}{0.783072in}}{\pgfqpoint{1.409177in}{0.777248in}}%
\pgfpathcurveto{\pgfqpoint{1.403353in}{0.771424in}}{\pgfqpoint{1.400081in}{0.763524in}}{\pgfqpoint{1.400081in}{0.755288in}}%
\pgfpathcurveto{\pgfqpoint{1.400081in}{0.747051in}}{\pgfqpoint{1.403353in}{0.739151in}}{\pgfqpoint{1.409177in}{0.733327in}}%
\pgfpathcurveto{\pgfqpoint{1.415001in}{0.727503in}}{\pgfqpoint{1.422901in}{0.724231in}}{\pgfqpoint{1.431138in}{0.724231in}}%
\pgfpathclose%
\pgfusepath{stroke,fill}%
\end{pgfscope}%
\begin{pgfscope}%
\pgfpathrectangle{\pgfqpoint{0.894063in}{0.630000in}}{\pgfqpoint{6.713438in}{2.060556in}} %
\pgfusepath{clip}%
\pgfsetbuttcap%
\pgfsetroundjoin%
\definecolor{currentfill}{rgb}{0.000000,0.750000,0.750000}%
\pgfsetfillcolor{currentfill}%
\pgfsetlinewidth{1.003750pt}%
\definecolor{currentstroke}{rgb}{0.000000,0.750000,0.750000}%
\pgfsetstrokecolor{currentstroke}%
\pgfsetdash{}{0pt}%
\pgfpathmoveto{\pgfqpoint{2.773825in}{0.963838in}}%
\pgfpathcurveto{\pgfqpoint{2.782061in}{0.963838in}}{\pgfqpoint{2.789961in}{0.967111in}}{\pgfqpoint{2.795785in}{0.972935in}}%
\pgfpathcurveto{\pgfqpoint{2.801609in}{0.978759in}}{\pgfqpoint{2.804881in}{0.986659in}}{\pgfqpoint{2.804881in}{0.994895in}}%
\pgfpathcurveto{\pgfqpoint{2.804881in}{1.003131in}}{\pgfqpoint{2.801609in}{1.011031in}}{\pgfqpoint{2.795785in}{1.016855in}}%
\pgfpathcurveto{\pgfqpoint{2.789961in}{1.022679in}}{\pgfqpoint{2.782061in}{1.025951in}}{\pgfqpoint{2.773825in}{1.025951in}}%
\pgfpathcurveto{\pgfqpoint{2.765589in}{1.025951in}}{\pgfqpoint{2.757689in}{1.022679in}}{\pgfqpoint{2.751865in}{1.016855in}}%
\pgfpathcurveto{\pgfqpoint{2.746041in}{1.011031in}}{\pgfqpoint{2.742769in}{1.003131in}}{\pgfqpoint{2.742769in}{0.994895in}}%
\pgfpathcurveto{\pgfqpoint{2.742769in}{0.986659in}}{\pgfqpoint{2.746041in}{0.978759in}}{\pgfqpoint{2.751865in}{0.972935in}}%
\pgfpathcurveto{\pgfqpoint{2.757689in}{0.967111in}}{\pgfqpoint{2.765589in}{0.963838in}}{\pgfqpoint{2.773825in}{0.963838in}}%
\pgfpathclose%
\pgfusepath{stroke,fill}%
\end{pgfscope}%
\begin{pgfscope}%
\pgfpathrectangle{\pgfqpoint{0.894063in}{0.630000in}}{\pgfqpoint{6.713438in}{2.060556in}} %
\pgfusepath{clip}%
\pgfsetbuttcap%
\pgfsetroundjoin%
\definecolor{currentfill}{rgb}{0.000000,0.750000,0.750000}%
\pgfsetfillcolor{currentfill}%
\pgfsetlinewidth{1.003750pt}%
\definecolor{currentstroke}{rgb}{0.000000,0.750000,0.750000}%
\pgfsetstrokecolor{currentstroke}%
\pgfsetdash{}{0pt}%
\pgfpathmoveto{\pgfqpoint{1.565406in}{0.747233in}}%
\pgfpathcurveto{\pgfqpoint{1.573643in}{0.747233in}}{\pgfqpoint{1.581543in}{0.750505in}}{\pgfqpoint{1.587367in}{0.756329in}}%
\pgfpathcurveto{\pgfqpoint{1.593190in}{0.762153in}}{\pgfqpoint{1.596463in}{0.770053in}}{\pgfqpoint{1.596463in}{0.778289in}}%
\pgfpathcurveto{\pgfqpoint{1.596463in}{0.786526in}}{\pgfqpoint{1.593190in}{0.794426in}}{\pgfqpoint{1.587367in}{0.800250in}}%
\pgfpathcurveto{\pgfqpoint{1.581543in}{0.806074in}}{\pgfqpoint{1.573643in}{0.809346in}}{\pgfqpoint{1.565406in}{0.809346in}}%
\pgfpathcurveto{\pgfqpoint{1.557170in}{0.809346in}}{\pgfqpoint{1.549270in}{0.806074in}}{\pgfqpoint{1.543446in}{0.800250in}}%
\pgfpathcurveto{\pgfqpoint{1.537622in}{0.794426in}}{\pgfqpoint{1.534350in}{0.786526in}}{\pgfqpoint{1.534350in}{0.778289in}}%
\pgfpathcurveto{\pgfqpoint{1.534350in}{0.770053in}}{\pgfqpoint{1.537622in}{0.762153in}}{\pgfqpoint{1.543446in}{0.756329in}}%
\pgfpathcurveto{\pgfqpoint{1.549270in}{0.750505in}}{\pgfqpoint{1.557170in}{0.747233in}}{\pgfqpoint{1.565406in}{0.747233in}}%
\pgfpathclose%
\pgfusepath{stroke,fill}%
\end{pgfscope}%
\begin{pgfscope}%
\pgfpathrectangle{\pgfqpoint{0.894063in}{0.630000in}}{\pgfqpoint{6.713438in}{2.060556in}} %
\pgfusepath{clip}%
\pgfsetbuttcap%
\pgfsetroundjoin%
\definecolor{currentfill}{rgb}{0.000000,0.750000,0.750000}%
\pgfsetfillcolor{currentfill}%
\pgfsetlinewidth{1.003750pt}%
\definecolor{currentstroke}{rgb}{0.000000,0.750000,0.750000}%
\pgfsetstrokecolor{currentstroke}%
\pgfsetdash{}{0pt}%
\pgfpathmoveto{\pgfqpoint{4.250781in}{1.200278in}}%
\pgfpathcurveto{\pgfqpoint{4.259018in}{1.200278in}}{\pgfqpoint{4.266918in}{1.203551in}}{\pgfqpoint{4.272742in}{1.209375in}}%
\pgfpathcurveto{\pgfqpoint{4.278565in}{1.215199in}}{\pgfqpoint{4.281838in}{1.223099in}}{\pgfqpoint{4.281838in}{1.231335in}}%
\pgfpathcurveto{\pgfqpoint{4.281838in}{1.239571in}}{\pgfqpoint{4.278565in}{1.247471in}}{\pgfqpoint{4.272742in}{1.253295in}}%
\pgfpathcurveto{\pgfqpoint{4.266918in}{1.259119in}}{\pgfqpoint{4.259018in}{1.262391in}}{\pgfqpoint{4.250781in}{1.262391in}}%
\pgfpathcurveto{\pgfqpoint{4.242545in}{1.262391in}}{\pgfqpoint{4.234645in}{1.259119in}}{\pgfqpoint{4.228821in}{1.253295in}}%
\pgfpathcurveto{\pgfqpoint{4.222997in}{1.247471in}}{\pgfqpoint{4.219725in}{1.239571in}}{\pgfqpoint{4.219725in}{1.231335in}}%
\pgfpathcurveto{\pgfqpoint{4.219725in}{1.223099in}}{\pgfqpoint{4.222997in}{1.215199in}}{\pgfqpoint{4.228821in}{1.209375in}}%
\pgfpathcurveto{\pgfqpoint{4.234645in}{1.203551in}}{\pgfqpoint{4.242545in}{1.200278in}}{\pgfqpoint{4.250781in}{1.200278in}}%
\pgfpathclose%
\pgfusepath{stroke,fill}%
\end{pgfscope}%
\begin{pgfscope}%
\pgfpathrectangle{\pgfqpoint{0.894063in}{0.630000in}}{\pgfqpoint{6.713438in}{2.060556in}} %
\pgfusepath{clip}%
\pgfsetbuttcap%
\pgfsetroundjoin%
\definecolor{currentfill}{rgb}{0.000000,0.750000,0.750000}%
\pgfsetfillcolor{currentfill}%
\pgfsetlinewidth{1.003750pt}%
\definecolor{currentstroke}{rgb}{0.000000,0.750000,0.750000}%
\pgfsetstrokecolor{currentstroke}%
\pgfsetdash{}{0pt}%
\pgfpathmoveto{\pgfqpoint{3.847975in}{1.138668in}}%
\pgfpathcurveto{\pgfqpoint{3.856211in}{1.138668in}}{\pgfqpoint{3.864111in}{1.141940in}}{\pgfqpoint{3.869935in}{1.147764in}}%
\pgfpathcurveto{\pgfqpoint{3.875759in}{1.153588in}}{\pgfqpoint{3.879031in}{1.161488in}}{\pgfqpoint{3.879031in}{1.169724in}}%
\pgfpathcurveto{\pgfqpoint{3.879031in}{1.177961in}}{\pgfqpoint{3.875759in}{1.185861in}}{\pgfqpoint{3.869935in}{1.191685in}}%
\pgfpathcurveto{\pgfqpoint{3.864111in}{1.197508in}}{\pgfqpoint{3.856211in}{1.200781in}}{\pgfqpoint{3.847975in}{1.200781in}}%
\pgfpathcurveto{\pgfqpoint{3.839739in}{1.200781in}}{\pgfqpoint{3.831839in}{1.197508in}}{\pgfqpoint{3.826015in}{1.191685in}}%
\pgfpathcurveto{\pgfqpoint{3.820191in}{1.185861in}}{\pgfqpoint{3.816919in}{1.177961in}}{\pgfqpoint{3.816919in}{1.169724in}}%
\pgfpathcurveto{\pgfqpoint{3.816919in}{1.161488in}}{\pgfqpoint{3.820191in}{1.153588in}}{\pgfqpoint{3.826015in}{1.147764in}}%
\pgfpathcurveto{\pgfqpoint{3.831839in}{1.141940in}}{\pgfqpoint{3.839739in}{1.138668in}}{\pgfqpoint{3.847975in}{1.138668in}}%
\pgfpathclose%
\pgfusepath{stroke,fill}%
\end{pgfscope}%
\begin{pgfscope}%
\pgfpathrectangle{\pgfqpoint{0.894063in}{0.630000in}}{\pgfqpoint{6.713438in}{2.060556in}} %
\pgfusepath{clip}%
\pgfsetbuttcap%
\pgfsetroundjoin%
\definecolor{currentfill}{rgb}{0.000000,0.750000,0.750000}%
\pgfsetfillcolor{currentfill}%
\pgfsetlinewidth{1.003750pt}%
\definecolor{currentstroke}{rgb}{0.000000,0.750000,0.750000}%
\pgfsetstrokecolor{currentstroke}%
\pgfsetdash{}{0pt}%
\pgfpathmoveto{\pgfqpoint{7.607500in}{1.770216in}}%
\pgfpathcurveto{\pgfqpoint{7.615736in}{1.770216in}}{\pgfqpoint{7.623636in}{1.773489in}}{\pgfqpoint{7.629460in}{1.779313in}}%
\pgfpathcurveto{\pgfqpoint{7.635284in}{1.785136in}}{\pgfqpoint{7.638556in}{1.793036in}}{\pgfqpoint{7.638556in}{1.801273in}}%
\pgfpathcurveto{\pgfqpoint{7.638556in}{1.809509in}}{\pgfqpoint{7.635284in}{1.817409in}}{\pgfqpoint{7.629460in}{1.823233in}}%
\pgfpathcurveto{\pgfqpoint{7.623636in}{1.829057in}}{\pgfqpoint{7.615736in}{1.832329in}}{\pgfqpoint{7.607500in}{1.832329in}}%
\pgfpathcurveto{\pgfqpoint{7.599264in}{1.832329in}}{\pgfqpoint{7.591364in}{1.829057in}}{\pgfqpoint{7.585540in}{1.823233in}}%
\pgfpathcurveto{\pgfqpoint{7.579716in}{1.817409in}}{\pgfqpoint{7.576444in}{1.809509in}}{\pgfqpoint{7.576444in}{1.801273in}}%
\pgfpathcurveto{\pgfqpoint{7.576444in}{1.793036in}}{\pgfqpoint{7.579716in}{1.785136in}}{\pgfqpoint{7.585540in}{1.779313in}}%
\pgfpathcurveto{\pgfqpoint{7.591364in}{1.773489in}}{\pgfqpoint{7.599264in}{1.770216in}}{\pgfqpoint{7.607500in}{1.770216in}}%
\pgfpathclose%
\pgfusepath{stroke,fill}%
\end{pgfscope}%
\begin{pgfscope}%
\pgfpathrectangle{\pgfqpoint{0.894063in}{0.630000in}}{\pgfqpoint{6.713438in}{2.060556in}} %
\pgfusepath{clip}%
\pgfsetbuttcap%
\pgfsetroundjoin%
\definecolor{currentfill}{rgb}{0.000000,0.750000,0.750000}%
\pgfsetfillcolor{currentfill}%
\pgfsetlinewidth{1.003750pt}%
\definecolor{currentstroke}{rgb}{0.000000,0.750000,0.750000}%
\pgfsetstrokecolor{currentstroke}%
\pgfsetdash{}{0pt}%
\pgfpathmoveto{\pgfqpoint{4.385050in}{1.220990in}}%
\pgfpathcurveto{\pgfqpoint{4.393286in}{1.220990in}}{\pgfqpoint{4.401186in}{1.224262in}}{\pgfqpoint{4.407010in}{1.230086in}}%
\pgfpathcurveto{\pgfqpoint{4.412834in}{1.235910in}}{\pgfqpoint{4.416106in}{1.243810in}}{\pgfqpoint{4.416106in}{1.252046in}}%
\pgfpathcurveto{\pgfqpoint{4.416106in}{1.260283in}}{\pgfqpoint{4.412834in}{1.268183in}}{\pgfqpoint{4.407010in}{1.274007in}}%
\pgfpathcurveto{\pgfqpoint{4.401186in}{1.279831in}}{\pgfqpoint{4.393286in}{1.283103in}}{\pgfqpoint{4.385050in}{1.283103in}}%
\pgfpathcurveto{\pgfqpoint{4.376814in}{1.283103in}}{\pgfqpoint{4.368914in}{1.279831in}}{\pgfqpoint{4.363090in}{1.274007in}}%
\pgfpathcurveto{\pgfqpoint{4.357266in}{1.268183in}}{\pgfqpoint{4.353994in}{1.260283in}}{\pgfqpoint{4.353994in}{1.252046in}}%
\pgfpathcurveto{\pgfqpoint{4.353994in}{1.243810in}}{\pgfqpoint{4.357266in}{1.235910in}}{\pgfqpoint{4.363090in}{1.230086in}}%
\pgfpathcurveto{\pgfqpoint{4.368914in}{1.224262in}}{\pgfqpoint{4.376814in}{1.220990in}}{\pgfqpoint{4.385050in}{1.220990in}}%
\pgfpathclose%
\pgfusepath{stroke,fill}%
\end{pgfscope}%
\begin{pgfscope}%
\pgfpathrectangle{\pgfqpoint{0.894063in}{0.630000in}}{\pgfqpoint{6.713438in}{2.060556in}} %
\pgfusepath{clip}%
\pgfsetbuttcap%
\pgfsetroundjoin%
\definecolor{currentfill}{rgb}{0.000000,0.750000,0.750000}%
\pgfsetfillcolor{currentfill}%
\pgfsetlinewidth{1.003750pt}%
\definecolor{currentstroke}{rgb}{0.000000,0.750000,0.750000}%
\pgfsetstrokecolor{currentstroke}%
\pgfsetdash{}{0pt}%
\pgfpathmoveto{\pgfqpoint{6.533350in}{1.600144in}}%
\pgfpathcurveto{\pgfqpoint{6.541586in}{1.600144in}}{\pgfqpoint{6.549486in}{1.603416in}}{\pgfqpoint{6.555310in}{1.609240in}}%
\pgfpathcurveto{\pgfqpoint{6.561134in}{1.615064in}}{\pgfqpoint{6.564406in}{1.622964in}}{\pgfqpoint{6.564406in}{1.631200in}}%
\pgfpathcurveto{\pgfqpoint{6.564406in}{1.639437in}}{\pgfqpoint{6.561134in}{1.647337in}}{\pgfqpoint{6.555310in}{1.653161in}}%
\pgfpathcurveto{\pgfqpoint{6.549486in}{1.658985in}}{\pgfqpoint{6.541586in}{1.662257in}}{\pgfqpoint{6.533350in}{1.662257in}}%
\pgfpathcurveto{\pgfqpoint{6.525114in}{1.662257in}}{\pgfqpoint{6.517214in}{1.658985in}}{\pgfqpoint{6.511390in}{1.653161in}}%
\pgfpathcurveto{\pgfqpoint{6.505566in}{1.647337in}}{\pgfqpoint{6.502294in}{1.639437in}}{\pgfqpoint{6.502294in}{1.631200in}}%
\pgfpathcurveto{\pgfqpoint{6.502294in}{1.622964in}}{\pgfqpoint{6.505566in}{1.615064in}}{\pgfqpoint{6.511390in}{1.609240in}}%
\pgfpathcurveto{\pgfqpoint{6.517214in}{1.603416in}}{\pgfqpoint{6.525114in}{1.600144in}}{\pgfqpoint{6.533350in}{1.600144in}}%
\pgfpathclose%
\pgfusepath{stroke,fill}%
\end{pgfscope}%
\begin{pgfscope}%
\pgfpathrectangle{\pgfqpoint{0.894063in}{0.630000in}}{\pgfqpoint{6.713438in}{2.060556in}} %
\pgfusepath{clip}%
\pgfsetbuttcap%
\pgfsetroundjoin%
\definecolor{currentfill}{rgb}{0.000000,0.750000,0.750000}%
\pgfsetfillcolor{currentfill}%
\pgfsetlinewidth{1.003750pt}%
\definecolor{currentstroke}{rgb}{0.000000,0.750000,0.750000}%
\pgfsetstrokecolor{currentstroke}%
\pgfsetdash{}{0pt}%
\pgfpathmoveto{\pgfqpoint{1.296869in}{0.703961in}}%
\pgfpathcurveto{\pgfqpoint{1.305105in}{0.703961in}}{\pgfqpoint{1.313005in}{0.707233in}}{\pgfqpoint{1.318829in}{0.713057in}}%
\pgfpathcurveto{\pgfqpoint{1.324653in}{0.718881in}}{\pgfqpoint{1.327925in}{0.726781in}}{\pgfqpoint{1.327925in}{0.735018in}}%
\pgfpathcurveto{\pgfqpoint{1.327925in}{0.743254in}}{\pgfqpoint{1.324653in}{0.751154in}}{\pgfqpoint{1.318829in}{0.756978in}}%
\pgfpathcurveto{\pgfqpoint{1.313005in}{0.762802in}}{\pgfqpoint{1.305105in}{0.766074in}}{\pgfqpoint{1.296869in}{0.766074in}}%
\pgfpathcurveto{\pgfqpoint{1.288632in}{0.766074in}}{\pgfqpoint{1.280732in}{0.762802in}}{\pgfqpoint{1.274908in}{0.756978in}}%
\pgfpathcurveto{\pgfqpoint{1.269085in}{0.751154in}}{\pgfqpoint{1.265812in}{0.743254in}}{\pgfqpoint{1.265812in}{0.735018in}}%
\pgfpathcurveto{\pgfqpoint{1.265812in}{0.726781in}}{\pgfqpoint{1.269085in}{0.718881in}}{\pgfqpoint{1.274908in}{0.713057in}}%
\pgfpathcurveto{\pgfqpoint{1.280732in}{0.707233in}}{\pgfqpoint{1.288632in}{0.703961in}}{\pgfqpoint{1.296869in}{0.703961in}}%
\pgfpathclose%
\pgfusepath{stroke,fill}%
\end{pgfscope}%
\begin{pgfscope}%
\pgfpathrectangle{\pgfqpoint{0.894063in}{0.630000in}}{\pgfqpoint{6.713438in}{2.060556in}} %
\pgfusepath{clip}%
\pgfsetbuttcap%
\pgfsetroundjoin%
\definecolor{currentfill}{rgb}{0.000000,0.750000,0.750000}%
\pgfsetfillcolor{currentfill}%
\pgfsetlinewidth{1.003750pt}%
\definecolor{currentstroke}{rgb}{0.000000,0.750000,0.750000}%
\pgfsetstrokecolor{currentstroke}%
\pgfsetdash{}{0pt}%
\pgfpathmoveto{\pgfqpoint{4.519319in}{1.239682in}}%
\pgfpathcurveto{\pgfqpoint{4.527555in}{1.239682in}}{\pgfqpoint{4.535455in}{1.242954in}}{\pgfqpoint{4.541279in}{1.248778in}}%
\pgfpathcurveto{\pgfqpoint{4.547103in}{1.254602in}}{\pgfqpoint{4.550375in}{1.262502in}}{\pgfqpoint{4.550375in}{1.270739in}}%
\pgfpathcurveto{\pgfqpoint{4.550375in}{1.278975in}}{\pgfqpoint{4.547103in}{1.286875in}}{\pgfqpoint{4.541279in}{1.292699in}}%
\pgfpathcurveto{\pgfqpoint{4.535455in}{1.298523in}}{\pgfqpoint{4.527555in}{1.301795in}}{\pgfqpoint{4.519319in}{1.301795in}}%
\pgfpathcurveto{\pgfqpoint{4.511082in}{1.301795in}}{\pgfqpoint{4.503182in}{1.298523in}}{\pgfqpoint{4.497358in}{1.292699in}}%
\pgfpathcurveto{\pgfqpoint{4.491535in}{1.286875in}}{\pgfqpoint{4.488262in}{1.278975in}}{\pgfqpoint{4.488262in}{1.270739in}}%
\pgfpathcurveto{\pgfqpoint{4.488262in}{1.262502in}}{\pgfqpoint{4.491535in}{1.254602in}}{\pgfqpoint{4.497358in}{1.248778in}}%
\pgfpathcurveto{\pgfqpoint{4.503182in}{1.242954in}}{\pgfqpoint{4.511082in}{1.239682in}}{\pgfqpoint{4.519319in}{1.239682in}}%
\pgfpathclose%
\pgfusepath{stroke,fill}%
\end{pgfscope}%
\begin{pgfscope}%
\pgfpathrectangle{\pgfqpoint{0.894063in}{0.630000in}}{\pgfqpoint{6.713438in}{2.060556in}} %
\pgfusepath{clip}%
\pgfsetbuttcap%
\pgfsetroundjoin%
\definecolor{currentfill}{rgb}{0.000000,0.750000,0.750000}%
\pgfsetfillcolor{currentfill}%
\pgfsetlinewidth{1.003750pt}%
\definecolor{currentstroke}{rgb}{0.000000,0.750000,0.750000}%
\pgfsetstrokecolor{currentstroke}%
\pgfsetdash{}{0pt}%
\pgfpathmoveto{\pgfqpoint{2.505288in}{0.917447in}}%
\pgfpathcurveto{\pgfqpoint{2.513524in}{0.917447in}}{\pgfqpoint{2.521424in}{0.920719in}}{\pgfqpoint{2.527248in}{0.926543in}}%
\pgfpathcurveto{\pgfqpoint{2.533072in}{0.932367in}}{\pgfqpoint{2.536344in}{0.940267in}}{\pgfqpoint{2.536344in}{0.948503in}}%
\pgfpathcurveto{\pgfqpoint{2.536344in}{0.956739in}}{\pgfqpoint{2.533072in}{0.964639in}}{\pgfqpoint{2.527248in}{0.970463in}}%
\pgfpathcurveto{\pgfqpoint{2.521424in}{0.976287in}}{\pgfqpoint{2.513524in}{0.979560in}}{\pgfqpoint{2.505288in}{0.979560in}}%
\pgfpathcurveto{\pgfqpoint{2.497051in}{0.979560in}}{\pgfqpoint{2.489151in}{0.976287in}}{\pgfqpoint{2.483327in}{0.970463in}}%
\pgfpathcurveto{\pgfqpoint{2.477503in}{0.964639in}}{\pgfqpoint{2.474231in}{0.956739in}}{\pgfqpoint{2.474231in}{0.948503in}}%
\pgfpathcurveto{\pgfqpoint{2.474231in}{0.940267in}}{\pgfqpoint{2.477503in}{0.932367in}}{\pgfqpoint{2.483327in}{0.926543in}}%
\pgfpathcurveto{\pgfqpoint{2.489151in}{0.920719in}}{\pgfqpoint{2.497051in}{0.917447in}}{\pgfqpoint{2.505288in}{0.917447in}}%
\pgfpathclose%
\pgfusepath{stroke,fill}%
\end{pgfscope}%
\begin{pgfscope}%
\pgfpathrectangle{\pgfqpoint{0.894063in}{0.630000in}}{\pgfqpoint{6.713438in}{2.060556in}} %
\pgfusepath{clip}%
\pgfsetbuttcap%
\pgfsetroundjoin%
\definecolor{currentfill}{rgb}{0.000000,0.750000,0.750000}%
\pgfsetfillcolor{currentfill}%
\pgfsetlinewidth{1.003750pt}%
\definecolor{currentstroke}{rgb}{0.000000,0.750000,0.750000}%
\pgfsetstrokecolor{currentstroke}%
\pgfsetdash{}{0pt}%
\pgfpathmoveto{\pgfqpoint{5.459200in}{1.421871in}}%
\pgfpathcurveto{\pgfqpoint{5.467436in}{1.421871in}}{\pgfqpoint{5.475336in}{1.425143in}}{\pgfqpoint{5.481160in}{1.430967in}}%
\pgfpathcurveto{\pgfqpoint{5.486984in}{1.436791in}}{\pgfqpoint{5.490256in}{1.444691in}}{\pgfqpoint{5.490256in}{1.452927in}}%
\pgfpathcurveto{\pgfqpoint{5.490256in}{1.461163in}}{\pgfqpoint{5.486984in}{1.469063in}}{\pgfqpoint{5.481160in}{1.474887in}}%
\pgfpathcurveto{\pgfqpoint{5.475336in}{1.480711in}}{\pgfqpoint{5.467436in}{1.483984in}}{\pgfqpoint{5.459200in}{1.483984in}}%
\pgfpathcurveto{\pgfqpoint{5.450964in}{1.483984in}}{\pgfqpoint{5.443064in}{1.480711in}}{\pgfqpoint{5.437240in}{1.474887in}}%
\pgfpathcurveto{\pgfqpoint{5.431416in}{1.469063in}}{\pgfqpoint{5.428144in}{1.461163in}}{\pgfqpoint{5.428144in}{1.452927in}}%
\pgfpathcurveto{\pgfqpoint{5.428144in}{1.444691in}}{\pgfqpoint{5.431416in}{1.436791in}}{\pgfqpoint{5.437240in}{1.430967in}}%
\pgfpathcurveto{\pgfqpoint{5.443064in}{1.425143in}}{\pgfqpoint{5.450964in}{1.421871in}}{\pgfqpoint{5.459200in}{1.421871in}}%
\pgfpathclose%
\pgfusepath{stroke,fill}%
\end{pgfscope}%
\begin{pgfscope}%
\pgfpathrectangle{\pgfqpoint{0.894063in}{0.630000in}}{\pgfqpoint{6.713438in}{2.060556in}} %
\pgfusepath{clip}%
\pgfsetbuttcap%
\pgfsetroundjoin%
\definecolor{currentfill}{rgb}{0.000000,0.750000,0.750000}%
\pgfsetfillcolor{currentfill}%
\pgfsetlinewidth{1.003750pt}%
\definecolor{currentstroke}{rgb}{0.000000,0.750000,0.750000}%
\pgfsetstrokecolor{currentstroke}%
\pgfsetdash{}{0pt}%
\pgfpathmoveto{\pgfqpoint{6.936156in}{1.661607in}}%
\pgfpathcurveto{\pgfqpoint{6.944393in}{1.661607in}}{\pgfqpoint{6.952293in}{1.664880in}}{\pgfqpoint{6.958117in}{1.670704in}}%
\pgfpathcurveto{\pgfqpoint{6.963940in}{1.676527in}}{\pgfqpoint{6.967213in}{1.684428in}}{\pgfqpoint{6.967213in}{1.692664in}}%
\pgfpathcurveto{\pgfqpoint{6.967213in}{1.700900in}}{\pgfqpoint{6.963940in}{1.708800in}}{\pgfqpoint{6.958117in}{1.714624in}}%
\pgfpathcurveto{\pgfqpoint{6.952293in}{1.720448in}}{\pgfqpoint{6.944393in}{1.723720in}}{\pgfqpoint{6.936156in}{1.723720in}}%
\pgfpathcurveto{\pgfqpoint{6.927920in}{1.723720in}}{\pgfqpoint{6.920020in}{1.720448in}}{\pgfqpoint{6.914196in}{1.714624in}}%
\pgfpathcurveto{\pgfqpoint{6.908372in}{1.708800in}}{\pgfqpoint{6.905100in}{1.700900in}}{\pgfqpoint{6.905100in}{1.692664in}}%
\pgfpathcurveto{\pgfqpoint{6.905100in}{1.684428in}}{\pgfqpoint{6.908372in}{1.676527in}}{\pgfqpoint{6.914196in}{1.670704in}}%
\pgfpathcurveto{\pgfqpoint{6.920020in}{1.664880in}}{\pgfqpoint{6.927920in}{1.661607in}}{\pgfqpoint{6.936156in}{1.661607in}}%
\pgfpathclose%
\pgfusepath{stroke,fill}%
\end{pgfscope}%
\begin{pgfscope}%
\pgfpathrectangle{\pgfqpoint{0.894063in}{0.630000in}}{\pgfqpoint{6.713438in}{2.060556in}} %
\pgfusepath{clip}%
\pgfsetbuttcap%
\pgfsetroundjoin%
\definecolor{currentfill}{rgb}{0.000000,0.750000,0.750000}%
\pgfsetfillcolor{currentfill}%
\pgfsetlinewidth{1.003750pt}%
\definecolor{currentstroke}{rgb}{0.000000,0.750000,0.750000}%
\pgfsetstrokecolor{currentstroke}%
\pgfsetdash{}{0pt}%
\pgfpathmoveto{\pgfqpoint{5.862006in}{1.487608in}}%
\pgfpathcurveto{\pgfqpoint{5.870243in}{1.487608in}}{\pgfqpoint{5.878143in}{1.490880in}}{\pgfqpoint{5.883967in}{1.496704in}}%
\pgfpathcurveto{\pgfqpoint{5.889790in}{1.502528in}}{\pgfqpoint{5.893063in}{1.510428in}}{\pgfqpoint{5.893063in}{1.518665in}}%
\pgfpathcurveto{\pgfqpoint{5.893063in}{1.526901in}}{\pgfqpoint{5.889790in}{1.534801in}}{\pgfqpoint{5.883967in}{1.540625in}}%
\pgfpathcurveto{\pgfqpoint{5.878143in}{1.546449in}}{\pgfqpoint{5.870243in}{1.549721in}}{\pgfqpoint{5.862006in}{1.549721in}}%
\pgfpathcurveto{\pgfqpoint{5.853770in}{1.549721in}}{\pgfqpoint{5.845870in}{1.546449in}}{\pgfqpoint{5.840046in}{1.540625in}}%
\pgfpathcurveto{\pgfqpoint{5.834222in}{1.534801in}}{\pgfqpoint{5.830950in}{1.526901in}}{\pgfqpoint{5.830950in}{1.518665in}}%
\pgfpathcurveto{\pgfqpoint{5.830950in}{1.510428in}}{\pgfqpoint{5.834222in}{1.502528in}}{\pgfqpoint{5.840046in}{1.496704in}}%
\pgfpathcurveto{\pgfqpoint{5.845870in}{1.490880in}}{\pgfqpoint{5.853770in}{1.487608in}}{\pgfqpoint{5.862006in}{1.487608in}}%
\pgfpathclose%
\pgfusepath{stroke,fill}%
\end{pgfscope}%
\begin{pgfscope}%
\pgfpathrectangle{\pgfqpoint{0.894063in}{0.630000in}}{\pgfqpoint{6.713438in}{2.060556in}} %
\pgfusepath{clip}%
\pgfsetbuttcap%
\pgfsetroundjoin%
\definecolor{currentfill}{rgb}{0.000000,0.750000,0.750000}%
\pgfsetfillcolor{currentfill}%
\pgfsetlinewidth{1.003750pt}%
\definecolor{currentstroke}{rgb}{0.000000,0.750000,0.750000}%
\pgfsetstrokecolor{currentstroke}%
\pgfsetdash{}{0pt}%
\pgfpathmoveto{\pgfqpoint{7.070425in}{1.687718in}}%
\pgfpathcurveto{\pgfqpoint{7.078661in}{1.687718in}}{\pgfqpoint{7.086561in}{1.690990in}}{\pgfqpoint{7.092385in}{1.696814in}}%
\pgfpathcurveto{\pgfqpoint{7.098209in}{1.702638in}}{\pgfqpoint{7.101481in}{1.710538in}}{\pgfqpoint{7.101481in}{1.718774in}}%
\pgfpathcurveto{\pgfqpoint{7.101481in}{1.727010in}}{\pgfqpoint{7.098209in}{1.734910in}}{\pgfqpoint{7.092385in}{1.740734in}}%
\pgfpathcurveto{\pgfqpoint{7.086561in}{1.746558in}}{\pgfqpoint{7.078661in}{1.749831in}}{\pgfqpoint{7.070425in}{1.749831in}}%
\pgfpathcurveto{\pgfqpoint{7.062189in}{1.749831in}}{\pgfqpoint{7.054289in}{1.746558in}}{\pgfqpoint{7.048465in}{1.740734in}}%
\pgfpathcurveto{\pgfqpoint{7.042641in}{1.734910in}}{\pgfqpoint{7.039369in}{1.727010in}}{\pgfqpoint{7.039369in}{1.718774in}}%
\pgfpathcurveto{\pgfqpoint{7.039369in}{1.710538in}}{\pgfqpoint{7.042641in}{1.702638in}}{\pgfqpoint{7.048465in}{1.696814in}}%
\pgfpathcurveto{\pgfqpoint{7.054289in}{1.690990in}}{\pgfqpoint{7.062189in}{1.687718in}}{\pgfqpoint{7.070425in}{1.687718in}}%
\pgfpathclose%
\pgfusepath{stroke,fill}%
\end{pgfscope}%
\begin{pgfscope}%
\pgfpathrectangle{\pgfqpoint{0.894063in}{0.630000in}}{\pgfqpoint{6.713438in}{2.060556in}} %
\pgfusepath{clip}%
\pgfsetbuttcap%
\pgfsetroundjoin%
\definecolor{currentfill}{rgb}{0.000000,0.750000,0.750000}%
\pgfsetfillcolor{currentfill}%
\pgfsetlinewidth{1.003750pt}%
\definecolor{currentstroke}{rgb}{0.000000,0.750000,0.750000}%
\pgfsetstrokecolor{currentstroke}%
\pgfsetdash{}{0pt}%
\pgfpathmoveto{\pgfqpoint{3.176631in}{1.035034in}}%
\pgfpathcurveto{\pgfqpoint{3.184868in}{1.035034in}}{\pgfqpoint{3.192768in}{1.038306in}}{\pgfqpoint{3.198592in}{1.044130in}}%
\pgfpathcurveto{\pgfqpoint{3.204415in}{1.049954in}}{\pgfqpoint{3.207688in}{1.057854in}}{\pgfqpoint{3.207688in}{1.066090in}}%
\pgfpathcurveto{\pgfqpoint{3.207688in}{1.074326in}}{\pgfqpoint{3.204415in}{1.082226in}}{\pgfqpoint{3.198592in}{1.088050in}}%
\pgfpathcurveto{\pgfqpoint{3.192768in}{1.093874in}}{\pgfqpoint{3.184868in}{1.097147in}}{\pgfqpoint{3.176631in}{1.097147in}}%
\pgfpathcurveto{\pgfqpoint{3.168395in}{1.097147in}}{\pgfqpoint{3.160495in}{1.093874in}}{\pgfqpoint{3.154671in}{1.088050in}}%
\pgfpathcurveto{\pgfqpoint{3.148847in}{1.082226in}}{\pgfqpoint{3.145575in}{1.074326in}}{\pgfqpoint{3.145575in}{1.066090in}}%
\pgfpathcurveto{\pgfqpoint{3.145575in}{1.057854in}}{\pgfqpoint{3.148847in}{1.049954in}}{\pgfqpoint{3.154671in}{1.044130in}}%
\pgfpathcurveto{\pgfqpoint{3.160495in}{1.038306in}}{\pgfqpoint{3.168395in}{1.035034in}}{\pgfqpoint{3.176631in}{1.035034in}}%
\pgfpathclose%
\pgfusepath{stroke,fill}%
\end{pgfscope}%
\begin{pgfscope}%
\pgfpathrectangle{\pgfqpoint{0.894063in}{0.630000in}}{\pgfqpoint{6.713438in}{2.060556in}} %
\pgfusepath{clip}%
\pgfsetbuttcap%
\pgfsetroundjoin%
\definecolor{currentfill}{rgb}{0.000000,0.750000,0.750000}%
\pgfsetfillcolor{currentfill}%
\pgfsetlinewidth{1.003750pt}%
\definecolor{currentstroke}{rgb}{0.000000,0.750000,0.750000}%
\pgfsetstrokecolor{currentstroke}%
\pgfsetdash{}{0pt}%
\pgfpathmoveto{\pgfqpoint{2.102481in}{0.845833in}}%
\pgfpathcurveto{\pgfqpoint{2.110718in}{0.845833in}}{\pgfqpoint{2.118618in}{0.849106in}}{\pgfqpoint{2.124442in}{0.854930in}}%
\pgfpathcurveto{\pgfqpoint{2.130265in}{0.860754in}}{\pgfqpoint{2.133538in}{0.868654in}}{\pgfqpoint{2.133538in}{0.876890in}}%
\pgfpathcurveto{\pgfqpoint{2.133538in}{0.885126in}}{\pgfqpoint{2.130265in}{0.893026in}}{\pgfqpoint{2.124442in}{0.898850in}}%
\pgfpathcurveto{\pgfqpoint{2.118618in}{0.904674in}}{\pgfqpoint{2.110718in}{0.907946in}}{\pgfqpoint{2.102481in}{0.907946in}}%
\pgfpathcurveto{\pgfqpoint{2.094245in}{0.907946in}}{\pgfqpoint{2.086345in}{0.904674in}}{\pgfqpoint{2.080521in}{0.898850in}}%
\pgfpathcurveto{\pgfqpoint{2.074697in}{0.893026in}}{\pgfqpoint{2.071425in}{0.885126in}}{\pgfqpoint{2.071425in}{0.876890in}}%
\pgfpathcurveto{\pgfqpoint{2.071425in}{0.868654in}}{\pgfqpoint{2.074697in}{0.860754in}}{\pgfqpoint{2.080521in}{0.854930in}}%
\pgfpathcurveto{\pgfqpoint{2.086345in}{0.849106in}}{\pgfqpoint{2.094245in}{0.845833in}}{\pgfqpoint{2.102481in}{0.845833in}}%
\pgfpathclose%
\pgfusepath{stroke,fill}%
\end{pgfscope}%
\begin{pgfscope}%
\pgfpathrectangle{\pgfqpoint{0.894063in}{0.630000in}}{\pgfqpoint{6.713438in}{2.060556in}} %
\pgfusepath{clip}%
\pgfsetbuttcap%
\pgfsetroundjoin%
\definecolor{currentfill}{rgb}{0.000000,0.750000,0.750000}%
\pgfsetfillcolor{currentfill}%
\pgfsetlinewidth{1.003750pt}%
\definecolor{currentstroke}{rgb}{0.000000,0.750000,0.750000}%
\pgfsetstrokecolor{currentstroke}%
\pgfsetdash{}{0pt}%
\pgfpathmoveto{\pgfqpoint{1.968213in}{0.820995in}}%
\pgfpathcurveto{\pgfqpoint{1.976449in}{0.820995in}}{\pgfqpoint{1.984349in}{0.824267in}}{\pgfqpoint{1.990173in}{0.830091in}}%
\pgfpathcurveto{\pgfqpoint{1.995997in}{0.835915in}}{\pgfqpoint{1.999269in}{0.843815in}}{\pgfqpoint{1.999269in}{0.852051in}}%
\pgfpathcurveto{\pgfqpoint{1.999269in}{0.860288in}}{\pgfqpoint{1.995997in}{0.868188in}}{\pgfqpoint{1.990173in}{0.874012in}}%
\pgfpathcurveto{\pgfqpoint{1.984349in}{0.879836in}}{\pgfqpoint{1.976449in}{0.883108in}}{\pgfqpoint{1.968213in}{0.883108in}}%
\pgfpathcurveto{\pgfqpoint{1.959976in}{0.883108in}}{\pgfqpoint{1.952076in}{0.879836in}}{\pgfqpoint{1.946252in}{0.874012in}}%
\pgfpathcurveto{\pgfqpoint{1.940428in}{0.868188in}}{\pgfqpoint{1.937156in}{0.860288in}}{\pgfqpoint{1.937156in}{0.852051in}}%
\pgfpathcurveto{\pgfqpoint{1.937156in}{0.843815in}}{\pgfqpoint{1.940428in}{0.835915in}}{\pgfqpoint{1.946252in}{0.830091in}}%
\pgfpathcurveto{\pgfqpoint{1.952076in}{0.824267in}}{\pgfqpoint{1.959976in}{0.820995in}}{\pgfqpoint{1.968213in}{0.820995in}}%
\pgfpathclose%
\pgfusepath{stroke,fill}%
\end{pgfscope}%
\begin{pgfscope}%
\pgfpathrectangle{\pgfqpoint{0.894063in}{0.630000in}}{\pgfqpoint{6.713438in}{2.060556in}} %
\pgfusepath{clip}%
\pgfsetbuttcap%
\pgfsetroundjoin%
\definecolor{currentfill}{rgb}{0.000000,0.750000,0.750000}%
\pgfsetfillcolor{currentfill}%
\pgfsetlinewidth{1.003750pt}%
\definecolor{currentstroke}{rgb}{0.000000,0.750000,0.750000}%
\pgfsetstrokecolor{currentstroke}%
\pgfsetdash{}{0pt}%
\pgfpathmoveto{\pgfqpoint{3.310900in}{1.056787in}}%
\pgfpathcurveto{\pgfqpoint{3.319136in}{1.056787in}}{\pgfqpoint{3.327036in}{1.060059in}}{\pgfqpoint{3.332860in}{1.065883in}}%
\pgfpathcurveto{\pgfqpoint{3.338684in}{1.071707in}}{\pgfqpoint{3.341956in}{1.079607in}}{\pgfqpoint{3.341956in}{1.087844in}}%
\pgfpathcurveto{\pgfqpoint{3.341956in}{1.096080in}}{\pgfqpoint{3.338684in}{1.103980in}}{\pgfqpoint{3.332860in}{1.109804in}}%
\pgfpathcurveto{\pgfqpoint{3.327036in}{1.115628in}}{\pgfqpoint{3.319136in}{1.118900in}}{\pgfqpoint{3.310900in}{1.118900in}}%
\pgfpathcurveto{\pgfqpoint{3.302664in}{1.118900in}}{\pgfqpoint{3.294764in}{1.115628in}}{\pgfqpoint{3.288940in}{1.109804in}}%
\pgfpathcurveto{\pgfqpoint{3.283116in}{1.103980in}}{\pgfqpoint{3.279844in}{1.096080in}}{\pgfqpoint{3.279844in}{1.087844in}}%
\pgfpathcurveto{\pgfqpoint{3.279844in}{1.079607in}}{\pgfqpoint{3.283116in}{1.071707in}}{\pgfqpoint{3.288940in}{1.065883in}}%
\pgfpathcurveto{\pgfqpoint{3.294764in}{1.060059in}}{\pgfqpoint{3.302664in}{1.056787in}}{\pgfqpoint{3.310900in}{1.056787in}}%
\pgfpathclose%
\pgfusepath{stroke,fill}%
\end{pgfscope}%
\begin{pgfscope}%
\pgfpathrectangle{\pgfqpoint{0.894063in}{0.630000in}}{\pgfqpoint{6.713438in}{2.060556in}} %
\pgfusepath{clip}%
\pgfsetbuttcap%
\pgfsetroundjoin%
\definecolor{currentfill}{rgb}{0.000000,0.750000,0.750000}%
\pgfsetfillcolor{currentfill}%
\pgfsetlinewidth{1.003750pt}%
\definecolor{currentstroke}{rgb}{0.000000,0.750000,0.750000}%
\pgfsetstrokecolor{currentstroke}%
\pgfsetdash{}{0pt}%
\pgfpathmoveto{\pgfqpoint{5.593469in}{1.441334in}}%
\pgfpathcurveto{\pgfqpoint{5.601705in}{1.441334in}}{\pgfqpoint{5.609605in}{1.444606in}}{\pgfqpoint{5.615429in}{1.450430in}}%
\pgfpathcurveto{\pgfqpoint{5.621253in}{1.456254in}}{\pgfqpoint{5.624525in}{1.464154in}}{\pgfqpoint{5.624525in}{1.472390in}}%
\pgfpathcurveto{\pgfqpoint{5.624525in}{1.480627in}}{\pgfqpoint{5.621253in}{1.488527in}}{\pgfqpoint{5.615429in}{1.494351in}}%
\pgfpathcurveto{\pgfqpoint{5.609605in}{1.500175in}}{\pgfqpoint{5.601705in}{1.503447in}}{\pgfqpoint{5.593469in}{1.503447in}}%
\pgfpathcurveto{\pgfqpoint{5.585232in}{1.503447in}}{\pgfqpoint{5.577332in}{1.500175in}}{\pgfqpoint{5.571508in}{1.494351in}}%
\pgfpathcurveto{\pgfqpoint{5.565685in}{1.488527in}}{\pgfqpoint{5.562412in}{1.480627in}}{\pgfqpoint{5.562412in}{1.472390in}}%
\pgfpathcurveto{\pgfqpoint{5.562412in}{1.464154in}}{\pgfqpoint{5.565685in}{1.456254in}}{\pgfqpoint{5.571508in}{1.450430in}}%
\pgfpathcurveto{\pgfqpoint{5.577332in}{1.444606in}}{\pgfqpoint{5.585232in}{1.441334in}}{\pgfqpoint{5.593469in}{1.441334in}}%
\pgfpathclose%
\pgfusepath{stroke,fill}%
\end{pgfscope}%
\begin{pgfscope}%
\pgfpathrectangle{\pgfqpoint{0.894063in}{0.630000in}}{\pgfqpoint{6.713438in}{2.060556in}} %
\pgfusepath{clip}%
\pgfsetbuttcap%
\pgfsetroundjoin%
\definecolor{currentfill}{rgb}{0.000000,0.750000,0.750000}%
\pgfsetfillcolor{currentfill}%
\pgfsetlinewidth{1.003750pt}%
\definecolor{currentstroke}{rgb}{0.000000,0.750000,0.750000}%
\pgfsetstrokecolor{currentstroke}%
\pgfsetdash{}{0pt}%
\pgfpathmoveto{\pgfqpoint{3.042363in}{1.013922in}}%
\pgfpathcurveto{\pgfqpoint{3.050599in}{1.013922in}}{\pgfqpoint{3.058499in}{1.017194in}}{\pgfqpoint{3.064323in}{1.023018in}}%
\pgfpathcurveto{\pgfqpoint{3.070147in}{1.028842in}}{\pgfqpoint{3.073419in}{1.036742in}}{\pgfqpoint{3.073419in}{1.044978in}}%
\pgfpathcurveto{\pgfqpoint{3.073419in}{1.053215in}}{\pgfqpoint{3.070147in}{1.061115in}}{\pgfqpoint{3.064323in}{1.066938in}}%
\pgfpathcurveto{\pgfqpoint{3.058499in}{1.072762in}}{\pgfqpoint{3.050599in}{1.076035in}}{\pgfqpoint{3.042363in}{1.076035in}}%
\pgfpathcurveto{\pgfqpoint{3.034126in}{1.076035in}}{\pgfqpoint{3.026226in}{1.072762in}}{\pgfqpoint{3.020402in}{1.066938in}}%
\pgfpathcurveto{\pgfqpoint{3.014578in}{1.061115in}}{\pgfqpoint{3.011306in}{1.053215in}}{\pgfqpoint{3.011306in}{1.044978in}}%
\pgfpathcurveto{\pgfqpoint{3.011306in}{1.036742in}}{\pgfqpoint{3.014578in}{1.028842in}}{\pgfqpoint{3.020402in}{1.023018in}}%
\pgfpathcurveto{\pgfqpoint{3.026226in}{1.017194in}}{\pgfqpoint{3.034126in}{1.013922in}}{\pgfqpoint{3.042363in}{1.013922in}}%
\pgfpathclose%
\pgfusepath{stroke,fill}%
\end{pgfscope}%
\begin{pgfscope}%
\pgfpathrectangle{\pgfqpoint{0.894063in}{0.630000in}}{\pgfqpoint{6.713438in}{2.060556in}} %
\pgfusepath{clip}%
\pgfsetbuttcap%
\pgfsetroundjoin%
\definecolor{currentfill}{rgb}{0.000000,0.750000,0.750000}%
\pgfsetfillcolor{currentfill}%
\pgfsetlinewidth{1.003750pt}%
\definecolor{currentstroke}{rgb}{0.000000,0.750000,0.750000}%
\pgfsetstrokecolor{currentstroke}%
\pgfsetdash{}{0pt}%
\pgfpathmoveto{\pgfqpoint{5.190663in}{1.372770in}}%
\pgfpathcurveto{\pgfqpoint{5.198899in}{1.372770in}}{\pgfqpoint{5.206799in}{1.376043in}}{\pgfqpoint{5.212623in}{1.381867in}}%
\pgfpathcurveto{\pgfqpoint{5.218447in}{1.387691in}}{\pgfqpoint{5.221719in}{1.395591in}}{\pgfqpoint{5.221719in}{1.403827in}}%
\pgfpathcurveto{\pgfqpoint{5.221719in}{1.412063in}}{\pgfqpoint{5.218447in}{1.419963in}}{\pgfqpoint{5.212623in}{1.425787in}}%
\pgfpathcurveto{\pgfqpoint{5.206799in}{1.431611in}}{\pgfqpoint{5.198899in}{1.434883in}}{\pgfqpoint{5.190663in}{1.434883in}}%
\pgfpathcurveto{\pgfqpoint{5.182426in}{1.434883in}}{\pgfqpoint{5.174526in}{1.431611in}}{\pgfqpoint{5.168702in}{1.425787in}}%
\pgfpathcurveto{\pgfqpoint{5.162878in}{1.419963in}}{\pgfqpoint{5.159606in}{1.412063in}}{\pgfqpoint{5.159606in}{1.403827in}}%
\pgfpathcurveto{\pgfqpoint{5.159606in}{1.395591in}}{\pgfqpoint{5.162878in}{1.387691in}}{\pgfqpoint{5.168702in}{1.381867in}}%
\pgfpathcurveto{\pgfqpoint{5.174526in}{1.376043in}}{\pgfqpoint{5.182426in}{1.372770in}}{\pgfqpoint{5.190663in}{1.372770in}}%
\pgfpathclose%
\pgfusepath{stroke,fill}%
\end{pgfscope}%
\begin{pgfscope}%
\pgfpathrectangle{\pgfqpoint{0.894063in}{0.630000in}}{\pgfqpoint{6.713438in}{2.060556in}} %
\pgfusepath{clip}%
\pgfsetbuttcap%
\pgfsetroundjoin%
\definecolor{currentfill}{rgb}{0.000000,0.750000,0.750000}%
\pgfsetfillcolor{currentfill}%
\pgfsetlinewidth{1.003750pt}%
\definecolor{currentstroke}{rgb}{0.000000,0.750000,0.750000}%
\pgfsetstrokecolor{currentstroke}%
\pgfsetdash{}{0pt}%
\pgfpathmoveto{\pgfqpoint{6.801888in}{1.653177in}}%
\pgfpathcurveto{\pgfqpoint{6.810124in}{1.653177in}}{\pgfqpoint{6.818024in}{1.656449in}}{\pgfqpoint{6.823848in}{1.662273in}}%
\pgfpathcurveto{\pgfqpoint{6.829672in}{1.668097in}}{\pgfqpoint{6.832944in}{1.675997in}}{\pgfqpoint{6.832944in}{1.684233in}}%
\pgfpathcurveto{\pgfqpoint{6.832944in}{1.692469in}}{\pgfqpoint{6.829672in}{1.700370in}}{\pgfqpoint{6.823848in}{1.706193in}}%
\pgfpathcurveto{\pgfqpoint{6.818024in}{1.712017in}}{\pgfqpoint{6.810124in}{1.715290in}}{\pgfqpoint{6.801888in}{1.715290in}}%
\pgfpathcurveto{\pgfqpoint{6.793651in}{1.715290in}}{\pgfqpoint{6.785751in}{1.712017in}}{\pgfqpoint{6.779927in}{1.706193in}}%
\pgfpathcurveto{\pgfqpoint{6.774103in}{1.700370in}}{\pgfqpoint{6.770831in}{1.692469in}}{\pgfqpoint{6.770831in}{1.684233in}}%
\pgfpathcurveto{\pgfqpoint{6.770831in}{1.675997in}}{\pgfqpoint{6.774103in}{1.668097in}}{\pgfqpoint{6.779927in}{1.662273in}}%
\pgfpathcurveto{\pgfqpoint{6.785751in}{1.656449in}}{\pgfqpoint{6.793651in}{1.653177in}}{\pgfqpoint{6.801888in}{1.653177in}}%
\pgfpathclose%
\pgfusepath{stroke,fill}%
\end{pgfscope}%
\begin{pgfscope}%
\pgfpathrectangle{\pgfqpoint{0.894063in}{0.630000in}}{\pgfqpoint{6.713438in}{2.060556in}} %
\pgfusepath{clip}%
\pgfsetbuttcap%
\pgfsetroundjoin%
\definecolor{currentfill}{rgb}{0.000000,0.750000,0.750000}%
\pgfsetfillcolor{currentfill}%
\pgfsetlinewidth{1.003750pt}%
\definecolor{currentstroke}{rgb}{0.000000,0.750000,0.750000}%
\pgfsetstrokecolor{currentstroke}%
\pgfsetdash{}{0pt}%
\pgfpathmoveto{\pgfqpoint{3.579438in}{1.097103in}}%
\pgfpathcurveto{\pgfqpoint{3.587674in}{1.097103in}}{\pgfqpoint{3.595574in}{1.100376in}}{\pgfqpoint{3.601398in}{1.106200in}}%
\pgfpathcurveto{\pgfqpoint{3.607222in}{1.112024in}}{\pgfqpoint{3.610494in}{1.119924in}}{\pgfqpoint{3.610494in}{1.128160in}}%
\pgfpathcurveto{\pgfqpoint{3.610494in}{1.136396in}}{\pgfqpoint{3.607222in}{1.144296in}}{\pgfqpoint{3.601398in}{1.150120in}}%
\pgfpathcurveto{\pgfqpoint{3.595574in}{1.155944in}}{\pgfqpoint{3.587674in}{1.159216in}}{\pgfqpoint{3.579438in}{1.159216in}}%
\pgfpathcurveto{\pgfqpoint{3.571201in}{1.159216in}}{\pgfqpoint{3.563301in}{1.155944in}}{\pgfqpoint{3.557477in}{1.150120in}}%
\pgfpathcurveto{\pgfqpoint{3.551653in}{1.144296in}}{\pgfqpoint{3.548381in}{1.136396in}}{\pgfqpoint{3.548381in}{1.128160in}}%
\pgfpathcurveto{\pgfqpoint{3.548381in}{1.119924in}}{\pgfqpoint{3.551653in}{1.112024in}}{\pgfqpoint{3.557477in}{1.106200in}}%
\pgfpathcurveto{\pgfqpoint{3.563301in}{1.100376in}}{\pgfqpoint{3.571201in}{1.097103in}}{\pgfqpoint{3.579438in}{1.097103in}}%
\pgfpathclose%
\pgfusepath{stroke,fill}%
\end{pgfscope}%
\begin{pgfscope}%
\pgfpathrectangle{\pgfqpoint{0.894063in}{0.630000in}}{\pgfqpoint{6.713438in}{2.060556in}} %
\pgfusepath{clip}%
\pgfsetbuttcap%
\pgfsetroundjoin%
\definecolor{currentfill}{rgb}{0.000000,0.750000,0.750000}%
\pgfsetfillcolor{currentfill}%
\pgfsetlinewidth{1.003750pt}%
\definecolor{currentstroke}{rgb}{0.000000,0.750000,0.750000}%
\pgfsetstrokecolor{currentstroke}%
\pgfsetdash{}{0pt}%
\pgfpathmoveto{\pgfqpoint{2.371019in}{0.897335in}}%
\pgfpathcurveto{\pgfqpoint{2.379255in}{0.897335in}}{\pgfqpoint{2.387155in}{0.900608in}}{\pgfqpoint{2.392979in}{0.906432in}}%
\pgfpathcurveto{\pgfqpoint{2.398803in}{0.912256in}}{\pgfqpoint{2.402075in}{0.920156in}}{\pgfqpoint{2.402075in}{0.928392in}}%
\pgfpathcurveto{\pgfqpoint{2.402075in}{0.936628in}}{\pgfqpoint{2.398803in}{0.944528in}}{\pgfqpoint{2.392979in}{0.950352in}}%
\pgfpathcurveto{\pgfqpoint{2.387155in}{0.956176in}}{\pgfqpoint{2.379255in}{0.959448in}}{\pgfqpoint{2.371019in}{0.959448in}}%
\pgfpathcurveto{\pgfqpoint{2.362782in}{0.959448in}}{\pgfqpoint{2.354882in}{0.956176in}}{\pgfqpoint{2.349058in}{0.950352in}}%
\pgfpathcurveto{\pgfqpoint{2.343235in}{0.944528in}}{\pgfqpoint{2.339962in}{0.936628in}}{\pgfqpoint{2.339962in}{0.928392in}}%
\pgfpathcurveto{\pgfqpoint{2.339962in}{0.920156in}}{\pgfqpoint{2.343235in}{0.912256in}}{\pgfqpoint{2.349058in}{0.906432in}}%
\pgfpathcurveto{\pgfqpoint{2.354882in}{0.900608in}}{\pgfqpoint{2.362782in}{0.897335in}}{\pgfqpoint{2.371019in}{0.897335in}}%
\pgfpathclose%
\pgfusepath{stroke,fill}%
\end{pgfscope}%
\begin{pgfscope}%
\pgfpathrectangle{\pgfqpoint{0.894063in}{0.630000in}}{\pgfqpoint{6.713438in}{2.060556in}} %
\pgfusepath{clip}%
\pgfsetbuttcap%
\pgfsetroundjoin%
\definecolor{currentfill}{rgb}{0.000000,0.750000,0.750000}%
\pgfsetfillcolor{currentfill}%
\pgfsetlinewidth{1.003750pt}%
\definecolor{currentstroke}{rgb}{0.000000,0.750000,0.750000}%
\pgfsetstrokecolor{currentstroke}%
\pgfsetdash{}{0pt}%
\pgfpathmoveto{\pgfqpoint{3.982244in}{1.164460in}}%
\pgfpathcurveto{\pgfqpoint{3.990480in}{1.164460in}}{\pgfqpoint{3.998380in}{1.167732in}}{\pgfqpoint{4.004204in}{1.173556in}}%
\pgfpathcurveto{\pgfqpoint{4.010028in}{1.179380in}}{\pgfqpoint{4.013300in}{1.187280in}}{\pgfqpoint{4.013300in}{1.195517in}}%
\pgfpathcurveto{\pgfqpoint{4.013300in}{1.203753in}}{\pgfqpoint{4.010028in}{1.211653in}}{\pgfqpoint{4.004204in}{1.217477in}}%
\pgfpathcurveto{\pgfqpoint{3.998380in}{1.223301in}}{\pgfqpoint{3.990480in}{1.226573in}}{\pgfqpoint{3.982244in}{1.226573in}}%
\pgfpathcurveto{\pgfqpoint{3.974007in}{1.226573in}}{\pgfqpoint{3.966107in}{1.223301in}}{\pgfqpoint{3.960283in}{1.217477in}}%
\pgfpathcurveto{\pgfqpoint{3.954460in}{1.211653in}}{\pgfqpoint{3.951187in}{1.203753in}}{\pgfqpoint{3.951187in}{1.195517in}}%
\pgfpathcurveto{\pgfqpoint{3.951187in}{1.187280in}}{\pgfqpoint{3.954460in}{1.179380in}}{\pgfqpoint{3.960283in}{1.173556in}}%
\pgfpathcurveto{\pgfqpoint{3.966107in}{1.167732in}}{\pgfqpoint{3.974007in}{1.164460in}}{\pgfqpoint{3.982244in}{1.164460in}}%
\pgfpathclose%
\pgfusepath{stroke,fill}%
\end{pgfscope}%
\begin{pgfscope}%
\pgfpathrectangle{\pgfqpoint{0.894063in}{0.630000in}}{\pgfqpoint{6.713438in}{2.060556in}} %
\pgfusepath{clip}%
\pgfsetbuttcap%
\pgfsetroundjoin%
\definecolor{currentfill}{rgb}{0.000000,0.750000,0.750000}%
\pgfsetfillcolor{currentfill}%
\pgfsetlinewidth{1.003750pt}%
\definecolor{currentstroke}{rgb}{0.000000,0.750000,0.750000}%
\pgfsetstrokecolor{currentstroke}%
\pgfsetdash{}{0pt}%
\pgfpathmoveto{\pgfqpoint{4.653588in}{1.265268in}}%
\pgfpathcurveto{\pgfqpoint{4.661824in}{1.265268in}}{\pgfqpoint{4.669724in}{1.268541in}}{\pgfqpoint{4.675548in}{1.274365in}}%
\pgfpathcurveto{\pgfqpoint{4.681372in}{1.280188in}}{\pgfqpoint{4.684644in}{1.288089in}}{\pgfqpoint{4.684644in}{1.296325in}}%
\pgfpathcurveto{\pgfqpoint{4.684644in}{1.304561in}}{\pgfqpoint{4.681372in}{1.312461in}}{\pgfqpoint{4.675548in}{1.318285in}}%
\pgfpathcurveto{\pgfqpoint{4.669724in}{1.324109in}}{\pgfqpoint{4.661824in}{1.327381in}}{\pgfqpoint{4.653588in}{1.327381in}}%
\pgfpathcurveto{\pgfqpoint{4.645351in}{1.327381in}}{\pgfqpoint{4.637451in}{1.324109in}}{\pgfqpoint{4.631627in}{1.318285in}}%
\pgfpathcurveto{\pgfqpoint{4.625803in}{1.312461in}}{\pgfqpoint{4.622531in}{1.304561in}}{\pgfqpoint{4.622531in}{1.296325in}}%
\pgfpathcurveto{\pgfqpoint{4.622531in}{1.288089in}}{\pgfqpoint{4.625803in}{1.280188in}}{\pgfqpoint{4.631627in}{1.274365in}}%
\pgfpathcurveto{\pgfqpoint{4.637451in}{1.268541in}}{\pgfqpoint{4.645351in}{1.265268in}}{\pgfqpoint{4.653588in}{1.265268in}}%
\pgfpathclose%
\pgfusepath{stroke,fill}%
\end{pgfscope}%
\begin{pgfscope}%
\pgfpathrectangle{\pgfqpoint{0.894063in}{0.630000in}}{\pgfqpoint{6.713438in}{2.060556in}} %
\pgfusepath{clip}%
\pgfsetbuttcap%
\pgfsetroundjoin%
\definecolor{currentfill}{rgb}{0.000000,0.750000,0.750000}%
\pgfsetfillcolor{currentfill}%
\pgfsetlinewidth{1.003750pt}%
\definecolor{currentstroke}{rgb}{0.000000,0.750000,0.750000}%
\pgfsetstrokecolor{currentstroke}%
\pgfsetdash{}{0pt}%
\pgfpathmoveto{\pgfqpoint{3.713706in}{1.113558in}}%
\pgfpathcurveto{\pgfqpoint{3.721943in}{1.113558in}}{\pgfqpoint{3.729843in}{1.116831in}}{\pgfqpoint{3.735667in}{1.122655in}}%
\pgfpathcurveto{\pgfqpoint{3.741490in}{1.128479in}}{\pgfqpoint{3.744763in}{1.136379in}}{\pgfqpoint{3.744763in}{1.144615in}}%
\pgfpathcurveto{\pgfqpoint{3.744763in}{1.152851in}}{\pgfqpoint{3.741490in}{1.160751in}}{\pgfqpoint{3.735667in}{1.166575in}}%
\pgfpathcurveto{\pgfqpoint{3.729843in}{1.172399in}}{\pgfqpoint{3.721943in}{1.175671in}}{\pgfqpoint{3.713706in}{1.175671in}}%
\pgfpathcurveto{\pgfqpoint{3.705470in}{1.175671in}}{\pgfqpoint{3.697570in}{1.172399in}}{\pgfqpoint{3.691746in}{1.166575in}}%
\pgfpathcurveto{\pgfqpoint{3.685922in}{1.160751in}}{\pgfqpoint{3.682650in}{1.152851in}}{\pgfqpoint{3.682650in}{1.144615in}}%
\pgfpathcurveto{\pgfqpoint{3.682650in}{1.136379in}}{\pgfqpoint{3.685922in}{1.128479in}}{\pgfqpoint{3.691746in}{1.122655in}}%
\pgfpathcurveto{\pgfqpoint{3.697570in}{1.116831in}}{\pgfqpoint{3.705470in}{1.113558in}}{\pgfqpoint{3.713706in}{1.113558in}}%
\pgfpathclose%
\pgfusepath{stroke,fill}%
\end{pgfscope}%
\begin{pgfscope}%
\pgfpathrectangle{\pgfqpoint{0.894063in}{0.630000in}}{\pgfqpoint{6.713438in}{2.060556in}} %
\pgfusepath{clip}%
\pgfsetbuttcap%
\pgfsetroundjoin%
\definecolor{currentfill}{rgb}{0.000000,0.750000,0.750000}%
\pgfsetfillcolor{currentfill}%
\pgfsetlinewidth{1.003750pt}%
\definecolor{currentstroke}{rgb}{0.000000,0.750000,0.750000}%
\pgfsetstrokecolor{currentstroke}%
\pgfsetdash{}{0pt}%
\pgfpathmoveto{\pgfqpoint{2.236750in}{0.868040in}}%
\pgfpathcurveto{\pgfqpoint{2.244986in}{0.868040in}}{\pgfqpoint{2.252886in}{0.871313in}}{\pgfqpoint{2.258710in}{0.877137in}}%
\pgfpathcurveto{\pgfqpoint{2.264534in}{0.882960in}}{\pgfqpoint{2.267806in}{0.890861in}}{\pgfqpoint{2.267806in}{0.899097in}}%
\pgfpathcurveto{\pgfqpoint{2.267806in}{0.907333in}}{\pgfqpoint{2.264534in}{0.915233in}}{\pgfqpoint{2.258710in}{0.921057in}}%
\pgfpathcurveto{\pgfqpoint{2.252886in}{0.926881in}}{\pgfqpoint{2.244986in}{0.930153in}}{\pgfqpoint{2.236750in}{0.930153in}}%
\pgfpathcurveto{\pgfqpoint{2.228514in}{0.930153in}}{\pgfqpoint{2.220614in}{0.926881in}}{\pgfqpoint{2.214790in}{0.921057in}}%
\pgfpathcurveto{\pgfqpoint{2.208966in}{0.915233in}}{\pgfqpoint{2.205694in}{0.907333in}}{\pgfqpoint{2.205694in}{0.899097in}}%
\pgfpathcurveto{\pgfqpoint{2.205694in}{0.890861in}}{\pgfqpoint{2.208966in}{0.882960in}}{\pgfqpoint{2.214790in}{0.877137in}}%
\pgfpathcurveto{\pgfqpoint{2.220614in}{0.871313in}}{\pgfqpoint{2.228514in}{0.868040in}}{\pgfqpoint{2.236750in}{0.868040in}}%
\pgfpathclose%
\pgfusepath{stroke,fill}%
\end{pgfscope}%
\begin{pgfscope}%
\pgfpathrectangle{\pgfqpoint{0.894063in}{0.630000in}}{\pgfqpoint{6.713438in}{2.060556in}} %
\pgfusepath{clip}%
\pgfsetbuttcap%
\pgfsetroundjoin%
\definecolor{currentfill}{rgb}{0.501961,0.000000,0.501961}%
\pgfsetfillcolor{currentfill}%
\pgfsetlinewidth{1.003750pt}%
\definecolor{currentstroke}{rgb}{0.501961,0.000000,0.501961}%
\pgfsetstrokecolor{currentstroke}%
\pgfsetdash{}{0pt}%
\pgfpathmoveto{\pgfqpoint{6.667619in}{1.761762in}}%
\pgfpathcurveto{\pgfqpoint{6.675855in}{1.761762in}}{\pgfqpoint{6.683755in}{1.765034in}}{\pgfqpoint{6.689579in}{1.770858in}}%
\pgfpathcurveto{\pgfqpoint{6.695403in}{1.776682in}}{\pgfqpoint{6.698675in}{1.784582in}}{\pgfqpoint{6.698675in}{1.792819in}}%
\pgfpathcurveto{\pgfqpoint{6.698675in}{1.801055in}}{\pgfqpoint{6.695403in}{1.808955in}}{\pgfqpoint{6.689579in}{1.814779in}}%
\pgfpathcurveto{\pgfqpoint{6.683755in}{1.820603in}}{\pgfqpoint{6.675855in}{1.823875in}}{\pgfqpoint{6.667619in}{1.823875in}}%
\pgfpathcurveto{\pgfqpoint{6.659382in}{1.823875in}}{\pgfqpoint{6.651482in}{1.820603in}}{\pgfqpoint{6.645658in}{1.814779in}}%
\pgfpathcurveto{\pgfqpoint{6.639835in}{1.808955in}}{\pgfqpoint{6.636562in}{1.801055in}}{\pgfqpoint{6.636562in}{1.792819in}}%
\pgfpathcurveto{\pgfqpoint{6.636562in}{1.784582in}}{\pgfqpoint{6.639835in}{1.776682in}}{\pgfqpoint{6.645658in}{1.770858in}}%
\pgfpathcurveto{\pgfqpoint{6.651482in}{1.765034in}}{\pgfqpoint{6.659382in}{1.761762in}}{\pgfqpoint{6.667619in}{1.761762in}}%
\pgfpathclose%
\pgfusepath{stroke,fill}%
\end{pgfscope}%
\begin{pgfscope}%
\pgfpathrectangle{\pgfqpoint{0.894063in}{0.630000in}}{\pgfqpoint{6.713438in}{2.060556in}} %
\pgfusepath{clip}%
\pgfsetbuttcap%
\pgfsetroundjoin%
\definecolor{currentfill}{rgb}{0.501961,0.000000,0.501961}%
\pgfsetfillcolor{currentfill}%
\pgfsetlinewidth{1.003750pt}%
\definecolor{currentstroke}{rgb}{0.501961,0.000000,0.501961}%
\pgfsetstrokecolor{currentstroke}%
\pgfsetdash{}{0pt}%
\pgfpathmoveto{\pgfqpoint{2.639556in}{0.992910in}}%
\pgfpathcurveto{\pgfqpoint{2.647793in}{0.992910in}}{\pgfqpoint{2.655693in}{0.996182in}}{\pgfqpoint{2.661517in}{1.002006in}}%
\pgfpathcurveto{\pgfqpoint{2.667340in}{1.007830in}}{\pgfqpoint{2.670613in}{1.015730in}}{\pgfqpoint{2.670613in}{1.023966in}}%
\pgfpathcurveto{\pgfqpoint{2.670613in}{1.032203in}}{\pgfqpoint{2.667340in}{1.040103in}}{\pgfqpoint{2.661517in}{1.045927in}}%
\pgfpathcurveto{\pgfqpoint{2.655693in}{1.051751in}}{\pgfqpoint{2.647793in}{1.055023in}}{\pgfqpoint{2.639556in}{1.055023in}}%
\pgfpathcurveto{\pgfqpoint{2.631320in}{1.055023in}}{\pgfqpoint{2.623420in}{1.051751in}}{\pgfqpoint{2.617596in}{1.045927in}}%
\pgfpathcurveto{\pgfqpoint{2.611772in}{1.040103in}}{\pgfqpoint{2.608500in}{1.032203in}}{\pgfqpoint{2.608500in}{1.023966in}}%
\pgfpathcurveto{\pgfqpoint{2.608500in}{1.015730in}}{\pgfqpoint{2.611772in}{1.007830in}}{\pgfqpoint{2.617596in}{1.002006in}}%
\pgfpathcurveto{\pgfqpoint{2.623420in}{0.996182in}}{\pgfqpoint{2.631320in}{0.992910in}}{\pgfqpoint{2.639556in}{0.992910in}}%
\pgfpathclose%
\pgfusepath{stroke,fill}%
\end{pgfscope}%
\begin{pgfscope}%
\pgfpathrectangle{\pgfqpoint{0.894063in}{0.630000in}}{\pgfqpoint{6.713438in}{2.060556in}} %
\pgfusepath{clip}%
\pgfsetbuttcap%
\pgfsetroundjoin%
\definecolor{currentfill}{rgb}{0.501961,0.000000,0.501961}%
\pgfsetfillcolor{currentfill}%
\pgfsetlinewidth{1.003750pt}%
\definecolor{currentstroke}{rgb}{0.501961,0.000000,0.501961}%
\pgfsetstrokecolor{currentstroke}%
\pgfsetdash{}{0pt}%
\pgfpathmoveto{\pgfqpoint{1.699675in}{0.800195in}}%
\pgfpathcurveto{\pgfqpoint{1.707911in}{0.800195in}}{\pgfqpoint{1.715811in}{0.803467in}}{\pgfqpoint{1.721635in}{0.809291in}}%
\pgfpathcurveto{\pgfqpoint{1.727459in}{0.815115in}}{\pgfqpoint{1.730731in}{0.823015in}}{\pgfqpoint{1.730731in}{0.831252in}}%
\pgfpathcurveto{\pgfqpoint{1.730731in}{0.839488in}}{\pgfqpoint{1.727459in}{0.847388in}}{\pgfqpoint{1.721635in}{0.853212in}}%
\pgfpathcurveto{\pgfqpoint{1.715811in}{0.859036in}}{\pgfqpoint{1.707911in}{0.862308in}}{\pgfqpoint{1.699675in}{0.862308in}}%
\pgfpathcurveto{\pgfqpoint{1.691439in}{0.862308in}}{\pgfqpoint{1.683539in}{0.859036in}}{\pgfqpoint{1.677715in}{0.853212in}}%
\pgfpathcurveto{\pgfqpoint{1.671891in}{0.847388in}}{\pgfqpoint{1.668619in}{0.839488in}}{\pgfqpoint{1.668619in}{0.831252in}}%
\pgfpathcurveto{\pgfqpoint{1.668619in}{0.823015in}}{\pgfqpoint{1.671891in}{0.815115in}}{\pgfqpoint{1.677715in}{0.809291in}}%
\pgfpathcurveto{\pgfqpoint{1.683539in}{0.803467in}}{\pgfqpoint{1.691439in}{0.800195in}}{\pgfqpoint{1.699675in}{0.800195in}}%
\pgfpathclose%
\pgfusepath{stroke,fill}%
\end{pgfscope}%
\begin{pgfscope}%
\pgfpathrectangle{\pgfqpoint{0.894063in}{0.630000in}}{\pgfqpoint{6.713438in}{2.060556in}} %
\pgfusepath{clip}%
\pgfsetbuttcap%
\pgfsetroundjoin%
\definecolor{currentfill}{rgb}{0.501961,0.000000,0.501961}%
\pgfsetfillcolor{currentfill}%
\pgfsetlinewidth{1.003750pt}%
\definecolor{currentstroke}{rgb}{0.501961,0.000000,0.501961}%
\pgfsetstrokecolor{currentstroke}%
\pgfsetdash{}{0pt}%
\pgfpathmoveto{\pgfqpoint{1.162600in}{0.699616in}}%
\pgfpathcurveto{\pgfqpoint{1.170836in}{0.699616in}}{\pgfqpoint{1.178736in}{0.702889in}}{\pgfqpoint{1.184560in}{0.708713in}}%
\pgfpathcurveto{\pgfqpoint{1.190384in}{0.714537in}}{\pgfqpoint{1.193656in}{0.722437in}}{\pgfqpoint{1.193656in}{0.730673in}}%
\pgfpathcurveto{\pgfqpoint{1.193656in}{0.738909in}}{\pgfqpoint{1.190384in}{0.746809in}}{\pgfqpoint{1.184560in}{0.752633in}}%
\pgfpathcurveto{\pgfqpoint{1.178736in}{0.758457in}}{\pgfqpoint{1.170836in}{0.761729in}}{\pgfqpoint{1.162600in}{0.761729in}}%
\pgfpathcurveto{\pgfqpoint{1.154364in}{0.761729in}}{\pgfqpoint{1.146464in}{0.758457in}}{\pgfqpoint{1.140640in}{0.752633in}}%
\pgfpathcurveto{\pgfqpoint{1.134816in}{0.746809in}}{\pgfqpoint{1.131544in}{0.738909in}}{\pgfqpoint{1.131544in}{0.730673in}}%
\pgfpathcurveto{\pgfqpoint{1.131544in}{0.722437in}}{\pgfqpoint{1.134816in}{0.714537in}}{\pgfqpoint{1.140640in}{0.708713in}}%
\pgfpathcurveto{\pgfqpoint{1.146464in}{0.702889in}}{\pgfqpoint{1.154364in}{0.699616in}}{\pgfqpoint{1.162600in}{0.699616in}}%
\pgfpathclose%
\pgfusepath{stroke,fill}%
\end{pgfscope}%
\begin{pgfscope}%
\pgfpathrectangle{\pgfqpoint{0.894063in}{0.630000in}}{\pgfqpoint{6.713438in}{2.060556in}} %
\pgfusepath{clip}%
\pgfsetbuttcap%
\pgfsetroundjoin%
\definecolor{currentfill}{rgb}{0.501961,0.000000,0.501961}%
\pgfsetfillcolor{currentfill}%
\pgfsetlinewidth{1.003750pt}%
\definecolor{currentstroke}{rgb}{0.501961,0.000000,0.501961}%
\pgfsetstrokecolor{currentstroke}%
\pgfsetdash{}{0pt}%
\pgfpathmoveto{\pgfqpoint{1.833944in}{0.826034in}}%
\pgfpathcurveto{\pgfqpoint{1.842180in}{0.826034in}}{\pgfqpoint{1.850080in}{0.829307in}}{\pgfqpoint{1.855904in}{0.835131in}}%
\pgfpathcurveto{\pgfqpoint{1.861728in}{0.840955in}}{\pgfqpoint{1.865000in}{0.848855in}}{\pgfqpoint{1.865000in}{0.857091in}}%
\pgfpathcurveto{\pgfqpoint{1.865000in}{0.865327in}}{\pgfqpoint{1.861728in}{0.873227in}}{\pgfqpoint{1.855904in}{0.879051in}}%
\pgfpathcurveto{\pgfqpoint{1.850080in}{0.884875in}}{\pgfqpoint{1.842180in}{0.888147in}}{\pgfqpoint{1.833944in}{0.888147in}}%
\pgfpathcurveto{\pgfqpoint{1.825707in}{0.888147in}}{\pgfqpoint{1.817807in}{0.884875in}}{\pgfqpoint{1.811983in}{0.879051in}}%
\pgfpathcurveto{\pgfqpoint{1.806160in}{0.873227in}}{\pgfqpoint{1.802887in}{0.865327in}}{\pgfqpoint{1.802887in}{0.857091in}}%
\pgfpathcurveto{\pgfqpoint{1.802887in}{0.848855in}}{\pgfqpoint{1.806160in}{0.840955in}}{\pgfqpoint{1.811983in}{0.835131in}}%
\pgfpathcurveto{\pgfqpoint{1.817807in}{0.829307in}}{\pgfqpoint{1.825707in}{0.826034in}}{\pgfqpoint{1.833944in}{0.826034in}}%
\pgfpathclose%
\pgfusepath{stroke,fill}%
\end{pgfscope}%
\begin{pgfscope}%
\pgfpathrectangle{\pgfqpoint{0.894063in}{0.630000in}}{\pgfqpoint{6.713438in}{2.060556in}} %
\pgfusepath{clip}%
\pgfsetbuttcap%
\pgfsetroundjoin%
\definecolor{currentfill}{rgb}{0.501961,0.000000,0.501961}%
\pgfsetfillcolor{currentfill}%
\pgfsetlinewidth{1.003750pt}%
\definecolor{currentstroke}{rgb}{0.501961,0.000000,0.501961}%
\pgfsetstrokecolor{currentstroke}%
\pgfsetdash{}{0pt}%
\pgfpathmoveto{\pgfqpoint{5.996275in}{1.617977in}}%
\pgfpathcurveto{\pgfqpoint{6.004511in}{1.617977in}}{\pgfqpoint{6.012411in}{1.621249in}}{\pgfqpoint{6.018235in}{1.627073in}}%
\pgfpathcurveto{\pgfqpoint{6.024059in}{1.632897in}}{\pgfqpoint{6.027331in}{1.640797in}}{\pgfqpoint{6.027331in}{1.649033in}}%
\pgfpathcurveto{\pgfqpoint{6.027331in}{1.657269in}}{\pgfqpoint{6.024059in}{1.665169in}}{\pgfqpoint{6.018235in}{1.670993in}}%
\pgfpathcurveto{\pgfqpoint{6.012411in}{1.676817in}}{\pgfqpoint{6.004511in}{1.680090in}}{\pgfqpoint{5.996275in}{1.680090in}}%
\pgfpathcurveto{\pgfqpoint{5.988039in}{1.680090in}}{\pgfqpoint{5.980139in}{1.676817in}}{\pgfqpoint{5.974315in}{1.670993in}}%
\pgfpathcurveto{\pgfqpoint{5.968491in}{1.665169in}}{\pgfqpoint{5.965219in}{1.657269in}}{\pgfqpoint{5.965219in}{1.649033in}}%
\pgfpathcurveto{\pgfqpoint{5.965219in}{1.640797in}}{\pgfqpoint{5.968491in}{1.632897in}}{\pgfqpoint{5.974315in}{1.627073in}}%
\pgfpathcurveto{\pgfqpoint{5.980139in}{1.621249in}}{\pgfqpoint{5.988039in}{1.617977in}}{\pgfqpoint{5.996275in}{1.617977in}}%
\pgfpathclose%
\pgfusepath{stroke,fill}%
\end{pgfscope}%
\begin{pgfscope}%
\pgfpathrectangle{\pgfqpoint{0.894063in}{0.630000in}}{\pgfqpoint{6.713438in}{2.060556in}} %
\pgfusepath{clip}%
\pgfsetbuttcap%
\pgfsetroundjoin%
\definecolor{currentfill}{rgb}{0.501961,0.000000,0.501961}%
\pgfsetfillcolor{currentfill}%
\pgfsetlinewidth{1.003750pt}%
\definecolor{currentstroke}{rgb}{0.501961,0.000000,0.501961}%
\pgfsetstrokecolor{currentstroke}%
\pgfsetdash{}{0pt}%
\pgfpathmoveto{\pgfqpoint{6.399081in}{1.708217in}}%
\pgfpathcurveto{\pgfqpoint{6.407318in}{1.708217in}}{\pgfqpoint{6.415218in}{1.711489in}}{\pgfqpoint{6.421042in}{1.717313in}}%
\pgfpathcurveto{\pgfqpoint{6.426865in}{1.723137in}}{\pgfqpoint{6.430138in}{1.731037in}}{\pgfqpoint{6.430138in}{1.739274in}}%
\pgfpathcurveto{\pgfqpoint{6.430138in}{1.747510in}}{\pgfqpoint{6.426865in}{1.755410in}}{\pgfqpoint{6.421042in}{1.761234in}}%
\pgfpathcurveto{\pgfqpoint{6.415218in}{1.767058in}}{\pgfqpoint{6.407318in}{1.770330in}}{\pgfqpoint{6.399081in}{1.770330in}}%
\pgfpathcurveto{\pgfqpoint{6.390845in}{1.770330in}}{\pgfqpoint{6.382945in}{1.767058in}}{\pgfqpoint{6.377121in}{1.761234in}}%
\pgfpathcurveto{\pgfqpoint{6.371297in}{1.755410in}}{\pgfqpoint{6.368025in}{1.747510in}}{\pgfqpoint{6.368025in}{1.739274in}}%
\pgfpathcurveto{\pgfqpoint{6.368025in}{1.731037in}}{\pgfqpoint{6.371297in}{1.723137in}}{\pgfqpoint{6.377121in}{1.717313in}}%
\pgfpathcurveto{\pgfqpoint{6.382945in}{1.711489in}}{\pgfqpoint{6.390845in}{1.708217in}}{\pgfqpoint{6.399081in}{1.708217in}}%
\pgfpathclose%
\pgfusepath{stroke,fill}%
\end{pgfscope}%
\begin{pgfscope}%
\pgfpathrectangle{\pgfqpoint{0.894063in}{0.630000in}}{\pgfqpoint{6.713438in}{2.060556in}} %
\pgfusepath{clip}%
\pgfsetbuttcap%
\pgfsetroundjoin%
\definecolor{currentfill}{rgb}{0.501961,0.000000,0.501961}%
\pgfsetfillcolor{currentfill}%
\pgfsetlinewidth{1.003750pt}%
\definecolor{currentstroke}{rgb}{0.501961,0.000000,0.501961}%
\pgfsetstrokecolor{currentstroke}%
\pgfsetdash{}{0pt}%
\pgfpathmoveto{\pgfqpoint{4.787856in}{1.384262in}}%
\pgfpathcurveto{\pgfqpoint{4.796093in}{1.384262in}}{\pgfqpoint{4.803993in}{1.387535in}}{\pgfqpoint{4.809817in}{1.393359in}}%
\pgfpathcurveto{\pgfqpoint{4.815640in}{1.399183in}}{\pgfqpoint{4.818913in}{1.407083in}}{\pgfqpoint{4.818913in}{1.415319in}}%
\pgfpathcurveto{\pgfqpoint{4.818913in}{1.423555in}}{\pgfqpoint{4.815640in}{1.431455in}}{\pgfqpoint{4.809817in}{1.437279in}}%
\pgfpathcurveto{\pgfqpoint{4.803993in}{1.443103in}}{\pgfqpoint{4.796093in}{1.446375in}}{\pgfqpoint{4.787856in}{1.446375in}}%
\pgfpathcurveto{\pgfqpoint{4.779620in}{1.446375in}}{\pgfqpoint{4.771720in}{1.443103in}}{\pgfqpoint{4.765896in}{1.437279in}}%
\pgfpathcurveto{\pgfqpoint{4.760072in}{1.431455in}}{\pgfqpoint{4.756800in}{1.423555in}}{\pgfqpoint{4.756800in}{1.415319in}}%
\pgfpathcurveto{\pgfqpoint{4.756800in}{1.407083in}}{\pgfqpoint{4.760072in}{1.399183in}}{\pgfqpoint{4.765896in}{1.393359in}}%
\pgfpathcurveto{\pgfqpoint{4.771720in}{1.387535in}}{\pgfqpoint{4.779620in}{1.384262in}}{\pgfqpoint{4.787856in}{1.384262in}}%
\pgfpathclose%
\pgfusepath{stroke,fill}%
\end{pgfscope}%
\begin{pgfscope}%
\pgfpathrectangle{\pgfqpoint{0.894063in}{0.630000in}}{\pgfqpoint{6.713438in}{2.060556in}} %
\pgfusepath{clip}%
\pgfsetbuttcap%
\pgfsetroundjoin%
\definecolor{currentfill}{rgb}{0.501961,0.000000,0.501961}%
\pgfsetfillcolor{currentfill}%
\pgfsetlinewidth{1.003750pt}%
\definecolor{currentstroke}{rgb}{0.501961,0.000000,0.501961}%
\pgfsetstrokecolor{currentstroke}%
\pgfsetdash{}{0pt}%
\pgfpathmoveto{\pgfqpoint{4.922125in}{1.411668in}}%
\pgfpathcurveto{\pgfqpoint{4.930361in}{1.411668in}}{\pgfqpoint{4.938261in}{1.414940in}}{\pgfqpoint{4.944085in}{1.420764in}}%
\pgfpathcurveto{\pgfqpoint{4.949909in}{1.426588in}}{\pgfqpoint{4.953181in}{1.434488in}}{\pgfqpoint{4.953181in}{1.442724in}}%
\pgfpathcurveto{\pgfqpoint{4.953181in}{1.450961in}}{\pgfqpoint{4.949909in}{1.458861in}}{\pgfqpoint{4.944085in}{1.464685in}}%
\pgfpathcurveto{\pgfqpoint{4.938261in}{1.470509in}}{\pgfqpoint{4.930361in}{1.473781in}}{\pgfqpoint{4.922125in}{1.473781in}}%
\pgfpathcurveto{\pgfqpoint{4.913889in}{1.473781in}}{\pgfqpoint{4.905989in}{1.470509in}}{\pgfqpoint{4.900165in}{1.464685in}}%
\pgfpathcurveto{\pgfqpoint{4.894341in}{1.458861in}}{\pgfqpoint{4.891069in}{1.450961in}}{\pgfqpoint{4.891069in}{1.442724in}}%
\pgfpathcurveto{\pgfqpoint{4.891069in}{1.434488in}}{\pgfqpoint{4.894341in}{1.426588in}}{\pgfqpoint{4.900165in}{1.420764in}}%
\pgfpathcurveto{\pgfqpoint{4.905989in}{1.414940in}}{\pgfqpoint{4.913889in}{1.411668in}}{\pgfqpoint{4.922125in}{1.411668in}}%
\pgfpathclose%
\pgfusepath{stroke,fill}%
\end{pgfscope}%
\begin{pgfscope}%
\pgfpathrectangle{\pgfqpoint{0.894063in}{0.630000in}}{\pgfqpoint{6.713438in}{2.060556in}} %
\pgfusepath{clip}%
\pgfsetbuttcap%
\pgfsetroundjoin%
\definecolor{currentfill}{rgb}{0.501961,0.000000,0.501961}%
\pgfsetfillcolor{currentfill}%
\pgfsetlinewidth{1.003750pt}%
\definecolor{currentstroke}{rgb}{0.501961,0.000000,0.501961}%
\pgfsetstrokecolor{currentstroke}%
\pgfsetdash{}{0pt}%
\pgfpathmoveto{\pgfqpoint{6.130544in}{1.646465in}}%
\pgfpathcurveto{\pgfqpoint{6.138780in}{1.646465in}}{\pgfqpoint{6.146680in}{1.649737in}}{\pgfqpoint{6.152504in}{1.655561in}}%
\pgfpathcurveto{\pgfqpoint{6.158328in}{1.661385in}}{\pgfqpoint{6.161600in}{1.669285in}}{\pgfqpoint{6.161600in}{1.677522in}}%
\pgfpathcurveto{\pgfqpoint{6.161600in}{1.685758in}}{\pgfqpoint{6.158328in}{1.693658in}}{\pgfqpoint{6.152504in}{1.699482in}}%
\pgfpathcurveto{\pgfqpoint{6.146680in}{1.705306in}}{\pgfqpoint{6.138780in}{1.708578in}}{\pgfqpoint{6.130544in}{1.708578in}}%
\pgfpathcurveto{\pgfqpoint{6.122307in}{1.708578in}}{\pgfqpoint{6.114407in}{1.705306in}}{\pgfqpoint{6.108583in}{1.699482in}}%
\pgfpathcurveto{\pgfqpoint{6.102760in}{1.693658in}}{\pgfqpoint{6.099487in}{1.685758in}}{\pgfqpoint{6.099487in}{1.677522in}}%
\pgfpathcurveto{\pgfqpoint{6.099487in}{1.669285in}}{\pgfqpoint{6.102760in}{1.661385in}}{\pgfqpoint{6.108583in}{1.655561in}}%
\pgfpathcurveto{\pgfqpoint{6.114407in}{1.649737in}}{\pgfqpoint{6.122307in}{1.646465in}}{\pgfqpoint{6.130544in}{1.646465in}}%
\pgfpathclose%
\pgfusepath{stroke,fill}%
\end{pgfscope}%
\begin{pgfscope}%
\pgfpathrectangle{\pgfqpoint{0.894063in}{0.630000in}}{\pgfqpoint{6.713438in}{2.060556in}} %
\pgfusepath{clip}%
\pgfsetbuttcap%
\pgfsetroundjoin%
\definecolor{currentfill}{rgb}{0.501961,0.000000,0.501961}%
\pgfsetfillcolor{currentfill}%
\pgfsetlinewidth{1.003750pt}%
\definecolor{currentstroke}{rgb}{0.501961,0.000000,0.501961}%
\pgfsetstrokecolor{currentstroke}%
\pgfsetdash{}{0pt}%
\pgfpathmoveto{\pgfqpoint{5.727738in}{1.562106in}}%
\pgfpathcurveto{\pgfqpoint{5.735974in}{1.562106in}}{\pgfqpoint{5.743874in}{1.565378in}}{\pgfqpoint{5.749698in}{1.571202in}}%
\pgfpathcurveto{\pgfqpoint{5.755522in}{1.577026in}}{\pgfqpoint{5.758794in}{1.584926in}}{\pgfqpoint{5.758794in}{1.593163in}}%
\pgfpathcurveto{\pgfqpoint{5.758794in}{1.601399in}}{\pgfqpoint{5.755522in}{1.609299in}}{\pgfqpoint{5.749698in}{1.615123in}}%
\pgfpathcurveto{\pgfqpoint{5.743874in}{1.620947in}}{\pgfqpoint{5.735974in}{1.624219in}}{\pgfqpoint{5.727738in}{1.624219in}}%
\pgfpathcurveto{\pgfqpoint{5.719501in}{1.624219in}}{\pgfqpoint{5.711601in}{1.620947in}}{\pgfqpoint{5.705777in}{1.615123in}}%
\pgfpathcurveto{\pgfqpoint{5.699953in}{1.609299in}}{\pgfqpoint{5.696681in}{1.601399in}}{\pgfqpoint{5.696681in}{1.593163in}}%
\pgfpathcurveto{\pgfqpoint{5.696681in}{1.584926in}}{\pgfqpoint{5.699953in}{1.577026in}}{\pgfqpoint{5.705777in}{1.571202in}}%
\pgfpathcurveto{\pgfqpoint{5.711601in}{1.565378in}}{\pgfqpoint{5.719501in}{1.562106in}}{\pgfqpoint{5.727738in}{1.562106in}}%
\pgfpathclose%
\pgfusepath{stroke,fill}%
\end{pgfscope}%
\begin{pgfscope}%
\pgfpathrectangle{\pgfqpoint{0.894063in}{0.630000in}}{\pgfqpoint{6.713438in}{2.060556in}} %
\pgfusepath{clip}%
\pgfsetbuttcap%
\pgfsetroundjoin%
\definecolor{currentfill}{rgb}{0.501961,0.000000,0.501961}%
\pgfsetfillcolor{currentfill}%
\pgfsetlinewidth{1.003750pt}%
\definecolor{currentstroke}{rgb}{0.501961,0.000000,0.501961}%
\pgfsetstrokecolor{currentstroke}%
\pgfsetdash{}{0pt}%
\pgfpathmoveto{\pgfqpoint{1.028331in}{0.674148in}}%
\pgfpathcurveto{\pgfqpoint{1.036568in}{0.674148in}}{\pgfqpoint{1.044468in}{0.677420in}}{\pgfqpoint{1.050292in}{0.683244in}}%
\pgfpathcurveto{\pgfqpoint{1.056115in}{0.689068in}}{\pgfqpoint{1.059388in}{0.696968in}}{\pgfqpoint{1.059388in}{0.705204in}}%
\pgfpathcurveto{\pgfqpoint{1.059388in}{0.713441in}}{\pgfqpoint{1.056115in}{0.721341in}}{\pgfqpoint{1.050292in}{0.727165in}}%
\pgfpathcurveto{\pgfqpoint{1.044468in}{0.732989in}}{\pgfqpoint{1.036568in}{0.736261in}}{\pgfqpoint{1.028331in}{0.736261in}}%
\pgfpathcurveto{\pgfqpoint{1.020095in}{0.736261in}}{\pgfqpoint{1.012195in}{0.732989in}}{\pgfqpoint{1.006371in}{0.727165in}}%
\pgfpathcurveto{\pgfqpoint{1.000547in}{0.721341in}}{\pgfqpoint{0.997275in}{0.713441in}}{\pgfqpoint{0.997275in}{0.705204in}}%
\pgfpathcurveto{\pgfqpoint{0.997275in}{0.696968in}}{\pgfqpoint{1.000547in}{0.689068in}}{\pgfqpoint{1.006371in}{0.683244in}}%
\pgfpathcurveto{\pgfqpoint{1.012195in}{0.677420in}}{\pgfqpoint{1.020095in}{0.674148in}}{\pgfqpoint{1.028331in}{0.674148in}}%
\pgfpathclose%
\pgfusepath{stroke,fill}%
\end{pgfscope}%
\begin{pgfscope}%
\pgfpathrectangle{\pgfqpoint{0.894063in}{0.630000in}}{\pgfqpoint{6.713438in}{2.060556in}} %
\pgfusepath{clip}%
\pgfsetbuttcap%
\pgfsetroundjoin%
\definecolor{currentfill}{rgb}{0.501961,0.000000,0.501961}%
\pgfsetfillcolor{currentfill}%
\pgfsetlinewidth{1.003750pt}%
\definecolor{currentstroke}{rgb}{0.501961,0.000000,0.501961}%
\pgfsetstrokecolor{currentstroke}%
\pgfsetdash{}{0pt}%
\pgfpathmoveto{\pgfqpoint{5.324931in}{1.485247in}}%
\pgfpathcurveto{\pgfqpoint{5.333168in}{1.485247in}}{\pgfqpoint{5.341068in}{1.488520in}}{\pgfqpoint{5.346892in}{1.494344in}}%
\pgfpathcurveto{\pgfqpoint{5.352715in}{1.500167in}}{\pgfqpoint{5.355988in}{1.508068in}}{\pgfqpoint{5.355988in}{1.516304in}}%
\pgfpathcurveto{\pgfqpoint{5.355988in}{1.524540in}}{\pgfqpoint{5.352715in}{1.532440in}}{\pgfqpoint{5.346892in}{1.538264in}}%
\pgfpathcurveto{\pgfqpoint{5.341068in}{1.544088in}}{\pgfqpoint{5.333168in}{1.547360in}}{\pgfqpoint{5.324931in}{1.547360in}}%
\pgfpathcurveto{\pgfqpoint{5.316695in}{1.547360in}}{\pgfqpoint{5.308795in}{1.544088in}}{\pgfqpoint{5.302971in}{1.538264in}}%
\pgfpathcurveto{\pgfqpoint{5.297147in}{1.532440in}}{\pgfqpoint{5.293875in}{1.524540in}}{\pgfqpoint{5.293875in}{1.516304in}}%
\pgfpathcurveto{\pgfqpoint{5.293875in}{1.508068in}}{\pgfqpoint{5.297147in}{1.500167in}}{\pgfqpoint{5.302971in}{1.494344in}}%
\pgfpathcurveto{\pgfqpoint{5.308795in}{1.488520in}}{\pgfqpoint{5.316695in}{1.485247in}}{\pgfqpoint{5.324931in}{1.485247in}}%
\pgfpathclose%
\pgfusepath{stroke,fill}%
\end{pgfscope}%
\begin{pgfscope}%
\pgfpathrectangle{\pgfqpoint{0.894063in}{0.630000in}}{\pgfqpoint{6.713438in}{2.060556in}} %
\pgfusepath{clip}%
\pgfsetbuttcap%
\pgfsetroundjoin%
\definecolor{currentfill}{rgb}{0.501961,0.000000,0.501961}%
\pgfsetfillcolor{currentfill}%
\pgfsetlinewidth{1.003750pt}%
\definecolor{currentstroke}{rgb}{0.501961,0.000000,0.501961}%
\pgfsetstrokecolor{currentstroke}%
\pgfsetdash{}{0pt}%
\pgfpathmoveto{\pgfqpoint{7.338963in}{1.889246in}}%
\pgfpathcurveto{\pgfqpoint{7.347199in}{1.889246in}}{\pgfqpoint{7.355099in}{1.892518in}}{\pgfqpoint{7.360923in}{1.898342in}}%
\pgfpathcurveto{\pgfqpoint{7.366747in}{1.904166in}}{\pgfqpoint{7.370019in}{1.912066in}}{\pgfqpoint{7.370019in}{1.920302in}}%
\pgfpathcurveto{\pgfqpoint{7.370019in}{1.928539in}}{\pgfqpoint{7.366747in}{1.936439in}}{\pgfqpoint{7.360923in}{1.942262in}}%
\pgfpathcurveto{\pgfqpoint{7.355099in}{1.948086in}}{\pgfqpoint{7.347199in}{1.951359in}}{\pgfqpoint{7.338963in}{1.951359in}}%
\pgfpathcurveto{\pgfqpoint{7.330726in}{1.951359in}}{\pgfqpoint{7.322826in}{1.948086in}}{\pgfqpoint{7.317002in}{1.942262in}}%
\pgfpathcurveto{\pgfqpoint{7.311178in}{1.936439in}}{\pgfqpoint{7.307906in}{1.928539in}}{\pgfqpoint{7.307906in}{1.920302in}}%
\pgfpathcurveto{\pgfqpoint{7.307906in}{1.912066in}}{\pgfqpoint{7.311178in}{1.904166in}}{\pgfqpoint{7.317002in}{1.898342in}}%
\pgfpathcurveto{\pgfqpoint{7.322826in}{1.892518in}}{\pgfqpoint{7.330726in}{1.889246in}}{\pgfqpoint{7.338963in}{1.889246in}}%
\pgfpathclose%
\pgfusepath{stroke,fill}%
\end{pgfscope}%
\begin{pgfscope}%
\pgfpathrectangle{\pgfqpoint{0.894063in}{0.630000in}}{\pgfqpoint{6.713438in}{2.060556in}} %
\pgfusepath{clip}%
\pgfsetbuttcap%
\pgfsetroundjoin%
\definecolor{currentfill}{rgb}{0.501961,0.000000,0.501961}%
\pgfsetfillcolor{currentfill}%
\pgfsetlinewidth{1.003750pt}%
\definecolor{currentstroke}{rgb}{0.501961,0.000000,0.501961}%
\pgfsetstrokecolor{currentstroke}%
\pgfsetdash{}{0pt}%
\pgfpathmoveto{\pgfqpoint{7.204694in}{1.863024in}}%
\pgfpathcurveto{\pgfqpoint{7.212930in}{1.863024in}}{\pgfqpoint{7.220830in}{1.866296in}}{\pgfqpoint{7.226654in}{1.872120in}}%
\pgfpathcurveto{\pgfqpoint{7.232478in}{1.877944in}}{\pgfqpoint{7.235750in}{1.885844in}}{\pgfqpoint{7.235750in}{1.894080in}}%
\pgfpathcurveto{\pgfqpoint{7.235750in}{1.902316in}}{\pgfqpoint{7.232478in}{1.910217in}}{\pgfqpoint{7.226654in}{1.916040in}}%
\pgfpathcurveto{\pgfqpoint{7.220830in}{1.921864in}}{\pgfqpoint{7.212930in}{1.925137in}}{\pgfqpoint{7.204694in}{1.925137in}}%
\pgfpathcurveto{\pgfqpoint{7.196457in}{1.925137in}}{\pgfqpoint{7.188557in}{1.921864in}}{\pgfqpoint{7.182733in}{1.916040in}}%
\pgfpathcurveto{\pgfqpoint{7.176910in}{1.910217in}}{\pgfqpoint{7.173637in}{1.902316in}}{\pgfqpoint{7.173637in}{1.894080in}}%
\pgfpathcurveto{\pgfqpoint{7.173637in}{1.885844in}}{\pgfqpoint{7.176910in}{1.877944in}}{\pgfqpoint{7.182733in}{1.872120in}}%
\pgfpathcurveto{\pgfqpoint{7.188557in}{1.866296in}}{\pgfqpoint{7.196457in}{1.863024in}}{\pgfqpoint{7.204694in}{1.863024in}}%
\pgfpathclose%
\pgfusepath{stroke,fill}%
\end{pgfscope}%
\begin{pgfscope}%
\pgfpathrectangle{\pgfqpoint{0.894063in}{0.630000in}}{\pgfqpoint{6.713438in}{2.060556in}} %
\pgfusepath{clip}%
\pgfsetbuttcap%
\pgfsetroundjoin%
\definecolor{currentfill}{rgb}{0.501961,0.000000,0.501961}%
\pgfsetfillcolor{currentfill}%
\pgfsetlinewidth{1.003750pt}%
\definecolor{currentstroke}{rgb}{0.501961,0.000000,0.501961}%
\pgfsetstrokecolor{currentstroke}%
\pgfsetdash{}{0pt}%
\pgfpathmoveto{\pgfqpoint{6.264813in}{1.682366in}}%
\pgfpathcurveto{\pgfqpoint{6.273049in}{1.682366in}}{\pgfqpoint{6.280949in}{1.685638in}}{\pgfqpoint{6.286773in}{1.691462in}}%
\pgfpathcurveto{\pgfqpoint{6.292597in}{1.697286in}}{\pgfqpoint{6.295869in}{1.705186in}}{\pgfqpoint{6.295869in}{1.713422in}}%
\pgfpathcurveto{\pgfqpoint{6.295869in}{1.721659in}}{\pgfqpoint{6.292597in}{1.729559in}}{\pgfqpoint{6.286773in}{1.735383in}}%
\pgfpathcurveto{\pgfqpoint{6.280949in}{1.741207in}}{\pgfqpoint{6.273049in}{1.744479in}}{\pgfqpoint{6.264813in}{1.744479in}}%
\pgfpathcurveto{\pgfqpoint{6.256576in}{1.744479in}}{\pgfqpoint{6.248676in}{1.741207in}}{\pgfqpoint{6.242852in}{1.735383in}}%
\pgfpathcurveto{\pgfqpoint{6.237028in}{1.729559in}}{\pgfqpoint{6.233756in}{1.721659in}}{\pgfqpoint{6.233756in}{1.713422in}}%
\pgfpathcurveto{\pgfqpoint{6.233756in}{1.705186in}}{\pgfqpoint{6.237028in}{1.697286in}}{\pgfqpoint{6.242852in}{1.691462in}}%
\pgfpathcurveto{\pgfqpoint{6.248676in}{1.685638in}}{\pgfqpoint{6.256576in}{1.682366in}}{\pgfqpoint{6.264813in}{1.682366in}}%
\pgfpathclose%
\pgfusepath{stroke,fill}%
\end{pgfscope}%
\begin{pgfscope}%
\pgfpathrectangle{\pgfqpoint{0.894063in}{0.630000in}}{\pgfqpoint{6.713438in}{2.060556in}} %
\pgfusepath{clip}%
\pgfsetbuttcap%
\pgfsetroundjoin%
\definecolor{currentfill}{rgb}{0.501961,0.000000,0.501961}%
\pgfsetfillcolor{currentfill}%
\pgfsetlinewidth{1.003750pt}%
\definecolor{currentstroke}{rgb}{0.501961,0.000000,0.501961}%
\pgfsetstrokecolor{currentstroke}%
\pgfsetdash{}{0pt}%
\pgfpathmoveto{\pgfqpoint{7.473231in}{1.931605in}}%
\pgfpathcurveto{\pgfqpoint{7.481468in}{1.931605in}}{\pgfqpoint{7.489368in}{1.934877in}}{\pgfqpoint{7.495192in}{1.940701in}}%
\pgfpathcurveto{\pgfqpoint{7.501015in}{1.946525in}}{\pgfqpoint{7.504288in}{1.954425in}}{\pgfqpoint{7.504288in}{1.962661in}}%
\pgfpathcurveto{\pgfqpoint{7.504288in}{1.970898in}}{\pgfqpoint{7.501015in}{1.978798in}}{\pgfqpoint{7.495192in}{1.984622in}}%
\pgfpathcurveto{\pgfqpoint{7.489368in}{1.990446in}}{\pgfqpoint{7.481468in}{1.993718in}}{\pgfqpoint{7.473231in}{1.993718in}}%
\pgfpathcurveto{\pgfqpoint{7.464995in}{1.993718in}}{\pgfqpoint{7.457095in}{1.990446in}}{\pgfqpoint{7.451271in}{1.984622in}}%
\pgfpathcurveto{\pgfqpoint{7.445447in}{1.978798in}}{\pgfqpoint{7.442175in}{1.970898in}}{\pgfqpoint{7.442175in}{1.962661in}}%
\pgfpathcurveto{\pgfqpoint{7.442175in}{1.954425in}}{\pgfqpoint{7.445447in}{1.946525in}}{\pgfqpoint{7.451271in}{1.940701in}}%
\pgfpathcurveto{\pgfqpoint{7.457095in}{1.934877in}}{\pgfqpoint{7.464995in}{1.931605in}}{\pgfqpoint{7.473231in}{1.931605in}}%
\pgfpathclose%
\pgfusepath{stroke,fill}%
\end{pgfscope}%
\begin{pgfscope}%
\pgfpathrectangle{\pgfqpoint{0.894063in}{0.630000in}}{\pgfqpoint{6.713438in}{2.060556in}} %
\pgfusepath{clip}%
\pgfsetbuttcap%
\pgfsetroundjoin%
\definecolor{currentfill}{rgb}{0.501961,0.000000,0.501961}%
\pgfsetfillcolor{currentfill}%
\pgfsetlinewidth{1.003750pt}%
\definecolor{currentstroke}{rgb}{0.501961,0.000000,0.501961}%
\pgfsetstrokecolor{currentstroke}%
\pgfsetdash{}{0pt}%
\pgfpathmoveto{\pgfqpoint{5.056394in}{1.434369in}}%
\pgfpathcurveto{\pgfqpoint{5.064630in}{1.434369in}}{\pgfqpoint{5.072530in}{1.437642in}}{\pgfqpoint{5.078354in}{1.443465in}}%
\pgfpathcurveto{\pgfqpoint{5.084178in}{1.449289in}}{\pgfqpoint{5.087450in}{1.457189in}}{\pgfqpoint{5.087450in}{1.465426in}}%
\pgfpathcurveto{\pgfqpoint{5.087450in}{1.473662in}}{\pgfqpoint{5.084178in}{1.481562in}}{\pgfqpoint{5.078354in}{1.487386in}}%
\pgfpathcurveto{\pgfqpoint{5.072530in}{1.493210in}}{\pgfqpoint{5.064630in}{1.496482in}}{\pgfqpoint{5.056394in}{1.496482in}}%
\pgfpathcurveto{\pgfqpoint{5.048157in}{1.496482in}}{\pgfqpoint{5.040257in}{1.493210in}}{\pgfqpoint{5.034433in}{1.487386in}}%
\pgfpathcurveto{\pgfqpoint{5.028610in}{1.481562in}}{\pgfqpoint{5.025337in}{1.473662in}}{\pgfqpoint{5.025337in}{1.465426in}}%
\pgfpathcurveto{\pgfqpoint{5.025337in}{1.457189in}}{\pgfqpoint{5.028610in}{1.449289in}}{\pgfqpoint{5.034433in}{1.443465in}}%
\pgfpathcurveto{\pgfqpoint{5.040257in}{1.437642in}}{\pgfqpoint{5.048157in}{1.434369in}}{\pgfqpoint{5.056394in}{1.434369in}}%
\pgfpathclose%
\pgfusepath{stroke,fill}%
\end{pgfscope}%
\begin{pgfscope}%
\pgfpathrectangle{\pgfqpoint{0.894063in}{0.630000in}}{\pgfqpoint{6.713438in}{2.060556in}} %
\pgfusepath{clip}%
\pgfsetbuttcap%
\pgfsetroundjoin%
\definecolor{currentfill}{rgb}{0.501961,0.000000,0.501961}%
\pgfsetfillcolor{currentfill}%
\pgfsetlinewidth{1.003750pt}%
\definecolor{currentstroke}{rgb}{0.501961,0.000000,0.501961}%
\pgfsetstrokecolor{currentstroke}%
\pgfsetdash{}{0pt}%
\pgfpathmoveto{\pgfqpoint{2.908094in}{1.051412in}}%
\pgfpathcurveto{\pgfqpoint{2.916330in}{1.051412in}}{\pgfqpoint{2.924230in}{1.054684in}}{\pgfqpoint{2.930054in}{1.060508in}}%
\pgfpathcurveto{\pgfqpoint{2.935878in}{1.066332in}}{\pgfqpoint{2.939150in}{1.074232in}}{\pgfqpoint{2.939150in}{1.082469in}}%
\pgfpathcurveto{\pgfqpoint{2.939150in}{1.090705in}}{\pgfqpoint{2.935878in}{1.098605in}}{\pgfqpoint{2.930054in}{1.104429in}}%
\pgfpathcurveto{\pgfqpoint{2.924230in}{1.110253in}}{\pgfqpoint{2.916330in}{1.113525in}}{\pgfqpoint{2.908094in}{1.113525in}}%
\pgfpathcurveto{\pgfqpoint{2.899857in}{1.113525in}}{\pgfqpoint{2.891957in}{1.110253in}}{\pgfqpoint{2.886133in}{1.104429in}}%
\pgfpathcurveto{\pgfqpoint{2.880310in}{1.098605in}}{\pgfqpoint{2.877037in}{1.090705in}}{\pgfqpoint{2.877037in}{1.082469in}}%
\pgfpathcurveto{\pgfqpoint{2.877037in}{1.074232in}}{\pgfqpoint{2.880310in}{1.066332in}}{\pgfqpoint{2.886133in}{1.060508in}}%
\pgfpathcurveto{\pgfqpoint{2.891957in}{1.054684in}}{\pgfqpoint{2.899857in}{1.051412in}}{\pgfqpoint{2.908094in}{1.051412in}}%
\pgfpathclose%
\pgfusepath{stroke,fill}%
\end{pgfscope}%
\begin{pgfscope}%
\pgfpathrectangle{\pgfqpoint{0.894063in}{0.630000in}}{\pgfqpoint{6.713438in}{2.060556in}} %
\pgfusepath{clip}%
\pgfsetbuttcap%
\pgfsetroundjoin%
\definecolor{currentfill}{rgb}{0.501961,0.000000,0.501961}%
\pgfsetfillcolor{currentfill}%
\pgfsetlinewidth{1.003750pt}%
\definecolor{currentstroke}{rgb}{0.501961,0.000000,0.501961}%
\pgfsetstrokecolor{currentstroke}%
\pgfsetdash{}{0pt}%
\pgfpathmoveto{\pgfqpoint{3.445169in}{1.161463in}}%
\pgfpathcurveto{\pgfqpoint{3.453405in}{1.161463in}}{\pgfqpoint{3.461305in}{1.164736in}}{\pgfqpoint{3.467129in}{1.170560in}}%
\pgfpathcurveto{\pgfqpoint{3.472953in}{1.176384in}}{\pgfqpoint{3.476225in}{1.184284in}}{\pgfqpoint{3.476225in}{1.192520in}}%
\pgfpathcurveto{\pgfqpoint{3.476225in}{1.200756in}}{\pgfqpoint{3.472953in}{1.208656in}}{\pgfqpoint{3.467129in}{1.214480in}}%
\pgfpathcurveto{\pgfqpoint{3.461305in}{1.220304in}}{\pgfqpoint{3.453405in}{1.223576in}}{\pgfqpoint{3.445169in}{1.223576in}}%
\pgfpathcurveto{\pgfqpoint{3.436932in}{1.223576in}}{\pgfqpoint{3.429032in}{1.220304in}}{\pgfqpoint{3.423208in}{1.214480in}}%
\pgfpathcurveto{\pgfqpoint{3.417385in}{1.208656in}}{\pgfqpoint{3.414112in}{1.200756in}}{\pgfqpoint{3.414112in}{1.192520in}}%
\pgfpathcurveto{\pgfqpoint{3.414112in}{1.184284in}}{\pgfqpoint{3.417385in}{1.176384in}}{\pgfqpoint{3.423208in}{1.170560in}}%
\pgfpathcurveto{\pgfqpoint{3.429032in}{1.164736in}}{\pgfqpoint{3.436932in}{1.161463in}}{\pgfqpoint{3.445169in}{1.161463in}}%
\pgfpathclose%
\pgfusepath{stroke,fill}%
\end{pgfscope}%
\begin{pgfscope}%
\pgfpathrectangle{\pgfqpoint{0.894063in}{0.630000in}}{\pgfqpoint{6.713438in}{2.060556in}} %
\pgfusepath{clip}%
\pgfsetbuttcap%
\pgfsetroundjoin%
\definecolor{currentfill}{rgb}{0.501961,0.000000,0.501961}%
\pgfsetfillcolor{currentfill}%
\pgfsetlinewidth{1.003750pt}%
\definecolor{currentstroke}{rgb}{0.501961,0.000000,0.501961}%
\pgfsetstrokecolor{currentstroke}%
\pgfsetdash{}{0pt}%
\pgfpathmoveto{\pgfqpoint{4.116513in}{1.272174in}}%
\pgfpathcurveto{\pgfqpoint{4.124749in}{1.272174in}}{\pgfqpoint{4.132649in}{1.275446in}}{\pgfqpoint{4.138473in}{1.281270in}}%
\pgfpathcurveto{\pgfqpoint{4.144297in}{1.287094in}}{\pgfqpoint{4.147569in}{1.294994in}}{\pgfqpoint{4.147569in}{1.303231in}}%
\pgfpathcurveto{\pgfqpoint{4.147569in}{1.311467in}}{\pgfqpoint{4.144297in}{1.319367in}}{\pgfqpoint{4.138473in}{1.325191in}}%
\pgfpathcurveto{\pgfqpoint{4.132649in}{1.331015in}}{\pgfqpoint{4.124749in}{1.334287in}}{\pgfqpoint{4.116513in}{1.334287in}}%
\pgfpathcurveto{\pgfqpoint{4.108276in}{1.334287in}}{\pgfqpoint{4.100376in}{1.331015in}}{\pgfqpoint{4.094552in}{1.325191in}}%
\pgfpathcurveto{\pgfqpoint{4.088728in}{1.319367in}}{\pgfqpoint{4.085456in}{1.311467in}}{\pgfqpoint{4.085456in}{1.303231in}}%
\pgfpathcurveto{\pgfqpoint{4.085456in}{1.294994in}}{\pgfqpoint{4.088728in}{1.287094in}}{\pgfqpoint{4.094552in}{1.281270in}}%
\pgfpathcurveto{\pgfqpoint{4.100376in}{1.275446in}}{\pgfqpoint{4.108276in}{1.272174in}}{\pgfqpoint{4.116513in}{1.272174in}}%
\pgfpathclose%
\pgfusepath{stroke,fill}%
\end{pgfscope}%
\begin{pgfscope}%
\pgfpathrectangle{\pgfqpoint{0.894063in}{0.630000in}}{\pgfqpoint{6.713438in}{2.060556in}} %
\pgfusepath{clip}%
\pgfsetbuttcap%
\pgfsetroundjoin%
\definecolor{currentfill}{rgb}{0.501961,0.000000,0.501961}%
\pgfsetfillcolor{currentfill}%
\pgfsetlinewidth{1.003750pt}%
\definecolor{currentstroke}{rgb}{0.501961,0.000000,0.501961}%
\pgfsetstrokecolor{currentstroke}%
\pgfsetdash{}{0pt}%
\pgfpathmoveto{\pgfqpoint{1.431138in}{0.744171in}}%
\pgfpathcurveto{\pgfqpoint{1.439374in}{0.744171in}}{\pgfqpoint{1.447274in}{0.747444in}}{\pgfqpoint{1.453098in}{0.753268in}}%
\pgfpathcurveto{\pgfqpoint{1.458922in}{0.759092in}}{\pgfqpoint{1.462194in}{0.766992in}}{\pgfqpoint{1.462194in}{0.775228in}}%
\pgfpathcurveto{\pgfqpoint{1.462194in}{0.783464in}}{\pgfqpoint{1.458922in}{0.791364in}}{\pgfqpoint{1.453098in}{0.797188in}}%
\pgfpathcurveto{\pgfqpoint{1.447274in}{0.803012in}}{\pgfqpoint{1.439374in}{0.806284in}}{\pgfqpoint{1.431138in}{0.806284in}}%
\pgfpathcurveto{\pgfqpoint{1.422901in}{0.806284in}}{\pgfqpoint{1.415001in}{0.803012in}}{\pgfqpoint{1.409177in}{0.797188in}}%
\pgfpathcurveto{\pgfqpoint{1.403353in}{0.791364in}}{\pgfqpoint{1.400081in}{0.783464in}}{\pgfqpoint{1.400081in}{0.775228in}}%
\pgfpathcurveto{\pgfqpoint{1.400081in}{0.766992in}}{\pgfqpoint{1.403353in}{0.759092in}}{\pgfqpoint{1.409177in}{0.753268in}}%
\pgfpathcurveto{\pgfqpoint{1.415001in}{0.747444in}}{\pgfqpoint{1.422901in}{0.744171in}}{\pgfqpoint{1.431138in}{0.744171in}}%
\pgfpathclose%
\pgfusepath{stroke,fill}%
\end{pgfscope}%
\begin{pgfscope}%
\pgfpathrectangle{\pgfqpoint{0.894063in}{0.630000in}}{\pgfqpoint{6.713438in}{2.060556in}} %
\pgfusepath{clip}%
\pgfsetbuttcap%
\pgfsetroundjoin%
\definecolor{currentfill}{rgb}{0.501961,0.000000,0.501961}%
\pgfsetfillcolor{currentfill}%
\pgfsetlinewidth{1.003750pt}%
\definecolor{currentstroke}{rgb}{0.501961,0.000000,0.501961}%
\pgfsetstrokecolor{currentstroke}%
\pgfsetdash{}{0pt}%
\pgfpathmoveto{\pgfqpoint{2.773825in}{1.029706in}}%
\pgfpathcurveto{\pgfqpoint{2.782061in}{1.029706in}}{\pgfqpoint{2.789961in}{1.032978in}}{\pgfqpoint{2.795785in}{1.038802in}}%
\pgfpathcurveto{\pgfqpoint{2.801609in}{1.044626in}}{\pgfqpoint{2.804881in}{1.052526in}}{\pgfqpoint{2.804881in}{1.060762in}}%
\pgfpathcurveto{\pgfqpoint{2.804881in}{1.068998in}}{\pgfqpoint{2.801609in}{1.076898in}}{\pgfqpoint{2.795785in}{1.082722in}}%
\pgfpathcurveto{\pgfqpoint{2.789961in}{1.088546in}}{\pgfqpoint{2.782061in}{1.091819in}}{\pgfqpoint{2.773825in}{1.091819in}}%
\pgfpathcurveto{\pgfqpoint{2.765589in}{1.091819in}}{\pgfqpoint{2.757689in}{1.088546in}}{\pgfqpoint{2.751865in}{1.082722in}}%
\pgfpathcurveto{\pgfqpoint{2.746041in}{1.076898in}}{\pgfqpoint{2.742769in}{1.068998in}}{\pgfqpoint{2.742769in}{1.060762in}}%
\pgfpathcurveto{\pgfqpoint{2.742769in}{1.052526in}}{\pgfqpoint{2.746041in}{1.044626in}}{\pgfqpoint{2.751865in}{1.038802in}}%
\pgfpathcurveto{\pgfqpoint{2.757689in}{1.032978in}}{\pgfqpoint{2.765589in}{1.029706in}}{\pgfqpoint{2.773825in}{1.029706in}}%
\pgfpathclose%
\pgfusepath{stroke,fill}%
\end{pgfscope}%
\begin{pgfscope}%
\pgfpathrectangle{\pgfqpoint{0.894063in}{0.630000in}}{\pgfqpoint{6.713438in}{2.060556in}} %
\pgfusepath{clip}%
\pgfsetbuttcap%
\pgfsetroundjoin%
\definecolor{currentfill}{rgb}{0.501961,0.000000,0.501961}%
\pgfsetfillcolor{currentfill}%
\pgfsetlinewidth{1.003750pt}%
\definecolor{currentstroke}{rgb}{0.501961,0.000000,0.501961}%
\pgfsetstrokecolor{currentstroke}%
\pgfsetdash{}{0pt}%
\pgfpathmoveto{\pgfqpoint{1.565406in}{0.772925in}}%
\pgfpathcurveto{\pgfqpoint{1.573643in}{0.772925in}}{\pgfqpoint{1.581543in}{0.776197in}}{\pgfqpoint{1.587367in}{0.782021in}}%
\pgfpathcurveto{\pgfqpoint{1.593190in}{0.787845in}}{\pgfqpoint{1.596463in}{0.795745in}}{\pgfqpoint{1.596463in}{0.803982in}}%
\pgfpathcurveto{\pgfqpoint{1.596463in}{0.812218in}}{\pgfqpoint{1.593190in}{0.820118in}}{\pgfqpoint{1.587367in}{0.825942in}}%
\pgfpathcurveto{\pgfqpoint{1.581543in}{0.831766in}}{\pgfqpoint{1.573643in}{0.835038in}}{\pgfqpoint{1.565406in}{0.835038in}}%
\pgfpathcurveto{\pgfqpoint{1.557170in}{0.835038in}}{\pgfqpoint{1.549270in}{0.831766in}}{\pgfqpoint{1.543446in}{0.825942in}}%
\pgfpathcurveto{\pgfqpoint{1.537622in}{0.820118in}}{\pgfqpoint{1.534350in}{0.812218in}}{\pgfqpoint{1.534350in}{0.803982in}}%
\pgfpathcurveto{\pgfqpoint{1.534350in}{0.795745in}}{\pgfqpoint{1.537622in}{0.787845in}}{\pgfqpoint{1.543446in}{0.782021in}}%
\pgfpathcurveto{\pgfqpoint{1.549270in}{0.776197in}}{\pgfqpoint{1.557170in}{0.772925in}}{\pgfqpoint{1.565406in}{0.772925in}}%
\pgfpathclose%
\pgfusepath{stroke,fill}%
\end{pgfscope}%
\begin{pgfscope}%
\pgfpathrectangle{\pgfqpoint{0.894063in}{0.630000in}}{\pgfqpoint{6.713438in}{2.060556in}} %
\pgfusepath{clip}%
\pgfsetbuttcap%
\pgfsetroundjoin%
\definecolor{currentfill}{rgb}{0.501961,0.000000,0.501961}%
\pgfsetfillcolor{currentfill}%
\pgfsetlinewidth{1.003750pt}%
\definecolor{currentstroke}{rgb}{0.501961,0.000000,0.501961}%
\pgfsetstrokecolor{currentstroke}%
\pgfsetdash{}{0pt}%
\pgfpathmoveto{\pgfqpoint{4.250781in}{1.295429in}}%
\pgfpathcurveto{\pgfqpoint{4.259018in}{1.295429in}}{\pgfqpoint{4.266918in}{1.298701in}}{\pgfqpoint{4.272742in}{1.304525in}}%
\pgfpathcurveto{\pgfqpoint{4.278565in}{1.310349in}}{\pgfqpoint{4.281838in}{1.318249in}}{\pgfqpoint{4.281838in}{1.326485in}}%
\pgfpathcurveto{\pgfqpoint{4.281838in}{1.334722in}}{\pgfqpoint{4.278565in}{1.342622in}}{\pgfqpoint{4.272742in}{1.348446in}}%
\pgfpathcurveto{\pgfqpoint{4.266918in}{1.354270in}}{\pgfqpoint{4.259018in}{1.357542in}}{\pgfqpoint{4.250781in}{1.357542in}}%
\pgfpathcurveto{\pgfqpoint{4.242545in}{1.357542in}}{\pgfqpoint{4.234645in}{1.354270in}}{\pgfqpoint{4.228821in}{1.348446in}}%
\pgfpathcurveto{\pgfqpoint{4.222997in}{1.342622in}}{\pgfqpoint{4.219725in}{1.334722in}}{\pgfqpoint{4.219725in}{1.326485in}}%
\pgfpathcurveto{\pgfqpoint{4.219725in}{1.318249in}}{\pgfqpoint{4.222997in}{1.310349in}}{\pgfqpoint{4.228821in}{1.304525in}}%
\pgfpathcurveto{\pgfqpoint{4.234645in}{1.298701in}}{\pgfqpoint{4.242545in}{1.295429in}}{\pgfqpoint{4.250781in}{1.295429in}}%
\pgfpathclose%
\pgfusepath{stroke,fill}%
\end{pgfscope}%
\begin{pgfscope}%
\pgfpathrectangle{\pgfqpoint{0.894063in}{0.630000in}}{\pgfqpoint{6.713438in}{2.060556in}} %
\pgfusepath{clip}%
\pgfsetbuttcap%
\pgfsetroundjoin%
\definecolor{currentfill}{rgb}{0.501961,0.000000,0.501961}%
\pgfsetfillcolor{currentfill}%
\pgfsetlinewidth{1.003750pt}%
\definecolor{currentstroke}{rgb}{0.501961,0.000000,0.501961}%
\pgfsetstrokecolor{currentstroke}%
\pgfsetdash{}{0pt}%
\pgfpathmoveto{\pgfqpoint{3.847975in}{1.229338in}}%
\pgfpathcurveto{\pgfqpoint{3.856211in}{1.229338in}}{\pgfqpoint{3.864111in}{1.232610in}}{\pgfqpoint{3.869935in}{1.238434in}}%
\pgfpathcurveto{\pgfqpoint{3.875759in}{1.244258in}}{\pgfqpoint{3.879031in}{1.252158in}}{\pgfqpoint{3.879031in}{1.260395in}}%
\pgfpathcurveto{\pgfqpoint{3.879031in}{1.268631in}}{\pgfqpoint{3.875759in}{1.276531in}}{\pgfqpoint{3.869935in}{1.282355in}}%
\pgfpathcurveto{\pgfqpoint{3.864111in}{1.288179in}}{\pgfqpoint{3.856211in}{1.291451in}}{\pgfqpoint{3.847975in}{1.291451in}}%
\pgfpathcurveto{\pgfqpoint{3.839739in}{1.291451in}}{\pgfqpoint{3.831839in}{1.288179in}}{\pgfqpoint{3.826015in}{1.282355in}}%
\pgfpathcurveto{\pgfqpoint{3.820191in}{1.276531in}}{\pgfqpoint{3.816919in}{1.268631in}}{\pgfqpoint{3.816919in}{1.260395in}}%
\pgfpathcurveto{\pgfqpoint{3.816919in}{1.252158in}}{\pgfqpoint{3.820191in}{1.244258in}}{\pgfqpoint{3.826015in}{1.238434in}}%
\pgfpathcurveto{\pgfqpoint{3.831839in}{1.232610in}}{\pgfqpoint{3.839739in}{1.229338in}}{\pgfqpoint{3.847975in}{1.229338in}}%
\pgfpathclose%
\pgfusepath{stroke,fill}%
\end{pgfscope}%
\begin{pgfscope}%
\pgfpathrectangle{\pgfqpoint{0.894063in}{0.630000in}}{\pgfqpoint{6.713438in}{2.060556in}} %
\pgfusepath{clip}%
\pgfsetbuttcap%
\pgfsetroundjoin%
\definecolor{currentfill}{rgb}{0.501961,0.000000,0.501961}%
\pgfsetfillcolor{currentfill}%
\pgfsetlinewidth{1.003750pt}%
\definecolor{currentstroke}{rgb}{0.501961,0.000000,0.501961}%
\pgfsetstrokecolor{currentstroke}%
\pgfsetdash{}{0pt}%
\pgfpathmoveto{\pgfqpoint{7.607500in}{1.951663in}}%
\pgfpathcurveto{\pgfqpoint{7.615736in}{1.951663in}}{\pgfqpoint{7.623636in}{1.954935in}}{\pgfqpoint{7.629460in}{1.960759in}}%
\pgfpathcurveto{\pgfqpoint{7.635284in}{1.966583in}}{\pgfqpoint{7.638556in}{1.974483in}}{\pgfqpoint{7.638556in}{1.982719in}}%
\pgfpathcurveto{\pgfqpoint{7.638556in}{1.990956in}}{\pgfqpoint{7.635284in}{1.998856in}}{\pgfqpoint{7.629460in}{2.004680in}}%
\pgfpathcurveto{\pgfqpoint{7.623636in}{2.010504in}}{\pgfqpoint{7.615736in}{2.013776in}}{\pgfqpoint{7.607500in}{2.013776in}}%
\pgfpathcurveto{\pgfqpoint{7.599264in}{2.013776in}}{\pgfqpoint{7.591364in}{2.010504in}}{\pgfqpoint{7.585540in}{2.004680in}}%
\pgfpathcurveto{\pgfqpoint{7.579716in}{1.998856in}}{\pgfqpoint{7.576444in}{1.990956in}}{\pgfqpoint{7.576444in}{1.982719in}}%
\pgfpathcurveto{\pgfqpoint{7.576444in}{1.974483in}}{\pgfqpoint{7.579716in}{1.966583in}}{\pgfqpoint{7.585540in}{1.960759in}}%
\pgfpathcurveto{\pgfqpoint{7.591364in}{1.954935in}}{\pgfqpoint{7.599264in}{1.951663in}}{\pgfqpoint{7.607500in}{1.951663in}}%
\pgfpathclose%
\pgfusepath{stroke,fill}%
\end{pgfscope}%
\begin{pgfscope}%
\pgfpathrectangle{\pgfqpoint{0.894063in}{0.630000in}}{\pgfqpoint{6.713438in}{2.060556in}} %
\pgfusepath{clip}%
\pgfsetbuttcap%
\pgfsetroundjoin%
\definecolor{currentfill}{rgb}{0.501961,0.000000,0.501961}%
\pgfsetfillcolor{currentfill}%
\pgfsetlinewidth{1.003750pt}%
\definecolor{currentstroke}{rgb}{0.501961,0.000000,0.501961}%
\pgfsetstrokecolor{currentstroke}%
\pgfsetdash{}{0pt}%
\pgfpathmoveto{\pgfqpoint{4.385050in}{1.322234in}}%
\pgfpathcurveto{\pgfqpoint{4.393286in}{1.322234in}}{\pgfqpoint{4.401186in}{1.325506in}}{\pgfqpoint{4.407010in}{1.331330in}}%
\pgfpathcurveto{\pgfqpoint{4.412834in}{1.337154in}}{\pgfqpoint{4.416106in}{1.345054in}}{\pgfqpoint{4.416106in}{1.353290in}}%
\pgfpathcurveto{\pgfqpoint{4.416106in}{1.361527in}}{\pgfqpoint{4.412834in}{1.369427in}}{\pgfqpoint{4.407010in}{1.375251in}}%
\pgfpathcurveto{\pgfqpoint{4.401186in}{1.381075in}}{\pgfqpoint{4.393286in}{1.384347in}}{\pgfqpoint{4.385050in}{1.384347in}}%
\pgfpathcurveto{\pgfqpoint{4.376814in}{1.384347in}}{\pgfqpoint{4.368914in}{1.381075in}}{\pgfqpoint{4.363090in}{1.375251in}}%
\pgfpathcurveto{\pgfqpoint{4.357266in}{1.369427in}}{\pgfqpoint{4.353994in}{1.361527in}}{\pgfqpoint{4.353994in}{1.353290in}}%
\pgfpathcurveto{\pgfqpoint{4.353994in}{1.345054in}}{\pgfqpoint{4.357266in}{1.337154in}}{\pgfqpoint{4.363090in}{1.331330in}}%
\pgfpathcurveto{\pgfqpoint{4.368914in}{1.325506in}}{\pgfqpoint{4.376814in}{1.322234in}}{\pgfqpoint{4.385050in}{1.322234in}}%
\pgfpathclose%
\pgfusepath{stroke,fill}%
\end{pgfscope}%
\begin{pgfscope}%
\pgfpathrectangle{\pgfqpoint{0.894063in}{0.630000in}}{\pgfqpoint{6.713438in}{2.060556in}} %
\pgfusepath{clip}%
\pgfsetbuttcap%
\pgfsetroundjoin%
\definecolor{currentfill}{rgb}{0.501961,0.000000,0.501961}%
\pgfsetfillcolor{currentfill}%
\pgfsetlinewidth{1.003750pt}%
\definecolor{currentstroke}{rgb}{0.501961,0.000000,0.501961}%
\pgfsetstrokecolor{currentstroke}%
\pgfsetdash{}{0pt}%
\pgfpathmoveto{\pgfqpoint{6.533350in}{1.741139in}}%
\pgfpathcurveto{\pgfqpoint{6.541586in}{1.741139in}}{\pgfqpoint{6.549486in}{1.744411in}}{\pgfqpoint{6.555310in}{1.750235in}}%
\pgfpathcurveto{\pgfqpoint{6.561134in}{1.756059in}}{\pgfqpoint{6.564406in}{1.763959in}}{\pgfqpoint{6.564406in}{1.772195in}}%
\pgfpathcurveto{\pgfqpoint{6.564406in}{1.780432in}}{\pgfqpoint{6.561134in}{1.788332in}}{\pgfqpoint{6.555310in}{1.794156in}}%
\pgfpathcurveto{\pgfqpoint{6.549486in}{1.799980in}}{\pgfqpoint{6.541586in}{1.803252in}}{\pgfqpoint{6.533350in}{1.803252in}}%
\pgfpathcurveto{\pgfqpoint{6.525114in}{1.803252in}}{\pgfqpoint{6.517214in}{1.799980in}}{\pgfqpoint{6.511390in}{1.794156in}}%
\pgfpathcurveto{\pgfqpoint{6.505566in}{1.788332in}}{\pgfqpoint{6.502294in}{1.780432in}}{\pgfqpoint{6.502294in}{1.772195in}}%
\pgfpathcurveto{\pgfqpoint{6.502294in}{1.763959in}}{\pgfqpoint{6.505566in}{1.756059in}}{\pgfqpoint{6.511390in}{1.750235in}}%
\pgfpathcurveto{\pgfqpoint{6.517214in}{1.744411in}}{\pgfqpoint{6.525114in}{1.741139in}}{\pgfqpoint{6.533350in}{1.741139in}}%
\pgfpathclose%
\pgfusepath{stroke,fill}%
\end{pgfscope}%
\begin{pgfscope}%
\pgfpathrectangle{\pgfqpoint{0.894063in}{0.630000in}}{\pgfqpoint{6.713438in}{2.060556in}} %
\pgfusepath{clip}%
\pgfsetbuttcap%
\pgfsetroundjoin%
\definecolor{currentfill}{rgb}{0.501961,0.000000,0.501961}%
\pgfsetfillcolor{currentfill}%
\pgfsetlinewidth{1.003750pt}%
\definecolor{currentstroke}{rgb}{0.501961,0.000000,0.501961}%
\pgfsetstrokecolor{currentstroke}%
\pgfsetdash{}{0pt}%
\pgfpathmoveto{\pgfqpoint{1.296869in}{0.719816in}}%
\pgfpathcurveto{\pgfqpoint{1.305105in}{0.719816in}}{\pgfqpoint{1.313005in}{0.723088in}}{\pgfqpoint{1.318829in}{0.728912in}}%
\pgfpathcurveto{\pgfqpoint{1.324653in}{0.734736in}}{\pgfqpoint{1.327925in}{0.742636in}}{\pgfqpoint{1.327925in}{0.750872in}}%
\pgfpathcurveto{\pgfqpoint{1.327925in}{0.759108in}}{\pgfqpoint{1.324653in}{0.767009in}}{\pgfqpoint{1.318829in}{0.772832in}}%
\pgfpathcurveto{\pgfqpoint{1.313005in}{0.778656in}}{\pgfqpoint{1.305105in}{0.781929in}}{\pgfqpoint{1.296869in}{0.781929in}}%
\pgfpathcurveto{\pgfqpoint{1.288632in}{0.781929in}}{\pgfqpoint{1.280732in}{0.778656in}}{\pgfqpoint{1.274908in}{0.772832in}}%
\pgfpathcurveto{\pgfqpoint{1.269085in}{0.767009in}}{\pgfqpoint{1.265812in}{0.759108in}}{\pgfqpoint{1.265812in}{0.750872in}}%
\pgfpathcurveto{\pgfqpoint{1.265812in}{0.742636in}}{\pgfqpoint{1.269085in}{0.734736in}}{\pgfqpoint{1.274908in}{0.728912in}}%
\pgfpathcurveto{\pgfqpoint{1.280732in}{0.723088in}}{\pgfqpoint{1.288632in}{0.719816in}}{\pgfqpoint{1.296869in}{0.719816in}}%
\pgfpathclose%
\pgfusepath{stroke,fill}%
\end{pgfscope}%
\begin{pgfscope}%
\pgfpathrectangle{\pgfqpoint{0.894063in}{0.630000in}}{\pgfqpoint{6.713438in}{2.060556in}} %
\pgfusepath{clip}%
\pgfsetbuttcap%
\pgfsetroundjoin%
\definecolor{currentfill}{rgb}{0.501961,0.000000,0.501961}%
\pgfsetfillcolor{currentfill}%
\pgfsetlinewidth{1.003750pt}%
\definecolor{currentstroke}{rgb}{0.501961,0.000000,0.501961}%
\pgfsetstrokecolor{currentstroke}%
\pgfsetdash{}{0pt}%
\pgfpathmoveto{\pgfqpoint{4.519319in}{1.338212in}}%
\pgfpathcurveto{\pgfqpoint{4.527555in}{1.338212in}}{\pgfqpoint{4.535455in}{1.341484in}}{\pgfqpoint{4.541279in}{1.347308in}}%
\pgfpathcurveto{\pgfqpoint{4.547103in}{1.353132in}}{\pgfqpoint{4.550375in}{1.361032in}}{\pgfqpoint{4.550375in}{1.369268in}}%
\pgfpathcurveto{\pgfqpoint{4.550375in}{1.377505in}}{\pgfqpoint{4.547103in}{1.385405in}}{\pgfqpoint{4.541279in}{1.391229in}}%
\pgfpathcurveto{\pgfqpoint{4.535455in}{1.397053in}}{\pgfqpoint{4.527555in}{1.400325in}}{\pgfqpoint{4.519319in}{1.400325in}}%
\pgfpathcurveto{\pgfqpoint{4.511082in}{1.400325in}}{\pgfqpoint{4.503182in}{1.397053in}}{\pgfqpoint{4.497358in}{1.391229in}}%
\pgfpathcurveto{\pgfqpoint{4.491535in}{1.385405in}}{\pgfqpoint{4.488262in}{1.377505in}}{\pgfqpoint{4.488262in}{1.369268in}}%
\pgfpathcurveto{\pgfqpoint{4.488262in}{1.361032in}}{\pgfqpoint{4.491535in}{1.353132in}}{\pgfqpoint{4.497358in}{1.347308in}}%
\pgfpathcurveto{\pgfqpoint{4.503182in}{1.341484in}}{\pgfqpoint{4.511082in}{1.338212in}}{\pgfqpoint{4.519319in}{1.338212in}}%
\pgfpathclose%
\pgfusepath{stroke,fill}%
\end{pgfscope}%
\begin{pgfscope}%
\pgfpathrectangle{\pgfqpoint{0.894063in}{0.630000in}}{\pgfqpoint{6.713438in}{2.060556in}} %
\pgfusepath{clip}%
\pgfsetbuttcap%
\pgfsetroundjoin%
\definecolor{currentfill}{rgb}{0.501961,0.000000,0.501961}%
\pgfsetfillcolor{currentfill}%
\pgfsetlinewidth{1.003750pt}%
\definecolor{currentstroke}{rgb}{0.501961,0.000000,0.501961}%
\pgfsetstrokecolor{currentstroke}%
\pgfsetdash{}{0pt}%
\pgfpathmoveto{\pgfqpoint{2.505288in}{0.963962in}}%
\pgfpathcurveto{\pgfqpoint{2.513524in}{0.963962in}}{\pgfqpoint{2.521424in}{0.967234in}}{\pgfqpoint{2.527248in}{0.973058in}}%
\pgfpathcurveto{\pgfqpoint{2.533072in}{0.978882in}}{\pgfqpoint{2.536344in}{0.986782in}}{\pgfqpoint{2.536344in}{0.995019in}}%
\pgfpathcurveto{\pgfqpoint{2.536344in}{1.003255in}}{\pgfqpoint{2.533072in}{1.011155in}}{\pgfqpoint{2.527248in}{1.016979in}}%
\pgfpathcurveto{\pgfqpoint{2.521424in}{1.022803in}}{\pgfqpoint{2.513524in}{1.026075in}}{\pgfqpoint{2.505288in}{1.026075in}}%
\pgfpathcurveto{\pgfqpoint{2.497051in}{1.026075in}}{\pgfqpoint{2.489151in}{1.022803in}}{\pgfqpoint{2.483327in}{1.016979in}}%
\pgfpathcurveto{\pgfqpoint{2.477503in}{1.011155in}}{\pgfqpoint{2.474231in}{1.003255in}}{\pgfqpoint{2.474231in}{0.995019in}}%
\pgfpathcurveto{\pgfqpoint{2.474231in}{0.986782in}}{\pgfqpoint{2.477503in}{0.978882in}}{\pgfqpoint{2.483327in}{0.973058in}}%
\pgfpathcurveto{\pgfqpoint{2.489151in}{0.967234in}}{\pgfqpoint{2.497051in}{0.963962in}}{\pgfqpoint{2.505288in}{0.963962in}}%
\pgfpathclose%
\pgfusepath{stroke,fill}%
\end{pgfscope}%
\begin{pgfscope}%
\pgfpathrectangle{\pgfqpoint{0.894063in}{0.630000in}}{\pgfqpoint{6.713438in}{2.060556in}} %
\pgfusepath{clip}%
\pgfsetbuttcap%
\pgfsetroundjoin%
\definecolor{currentfill}{rgb}{0.501961,0.000000,0.501961}%
\pgfsetfillcolor{currentfill}%
\pgfsetlinewidth{1.003750pt}%
\definecolor{currentstroke}{rgb}{0.501961,0.000000,0.501961}%
\pgfsetstrokecolor{currentstroke}%
\pgfsetdash{}{0pt}%
\pgfpathmoveto{\pgfqpoint{5.459200in}{1.508526in}}%
\pgfpathcurveto{\pgfqpoint{5.467436in}{1.508526in}}{\pgfqpoint{5.475336in}{1.511798in}}{\pgfqpoint{5.481160in}{1.517622in}}%
\pgfpathcurveto{\pgfqpoint{5.486984in}{1.523446in}}{\pgfqpoint{5.490256in}{1.531346in}}{\pgfqpoint{5.490256in}{1.539582in}}%
\pgfpathcurveto{\pgfqpoint{5.490256in}{1.547818in}}{\pgfqpoint{5.486984in}{1.555719in}}{\pgfqpoint{5.481160in}{1.561542in}}%
\pgfpathcurveto{\pgfqpoint{5.475336in}{1.567366in}}{\pgfqpoint{5.467436in}{1.570639in}}{\pgfqpoint{5.459200in}{1.570639in}}%
\pgfpathcurveto{\pgfqpoint{5.450964in}{1.570639in}}{\pgfqpoint{5.443064in}{1.567366in}}{\pgfqpoint{5.437240in}{1.561542in}}%
\pgfpathcurveto{\pgfqpoint{5.431416in}{1.555719in}}{\pgfqpoint{5.428144in}{1.547818in}}{\pgfqpoint{5.428144in}{1.539582in}}%
\pgfpathcurveto{\pgfqpoint{5.428144in}{1.531346in}}{\pgfqpoint{5.431416in}{1.523446in}}{\pgfqpoint{5.437240in}{1.517622in}}%
\pgfpathcurveto{\pgfqpoint{5.443064in}{1.511798in}}{\pgfqpoint{5.450964in}{1.508526in}}{\pgfqpoint{5.459200in}{1.508526in}}%
\pgfpathclose%
\pgfusepath{stroke,fill}%
\end{pgfscope}%
\begin{pgfscope}%
\pgfpathrectangle{\pgfqpoint{0.894063in}{0.630000in}}{\pgfqpoint{6.713438in}{2.060556in}} %
\pgfusepath{clip}%
\pgfsetbuttcap%
\pgfsetroundjoin%
\definecolor{currentfill}{rgb}{0.501961,0.000000,0.501961}%
\pgfsetfillcolor{currentfill}%
\pgfsetlinewidth{1.003750pt}%
\definecolor{currentstroke}{rgb}{0.501961,0.000000,0.501961}%
\pgfsetstrokecolor{currentstroke}%
\pgfsetdash{}{0pt}%
\pgfpathmoveto{\pgfqpoint{6.936156in}{1.808254in}}%
\pgfpathcurveto{\pgfqpoint{6.944393in}{1.808254in}}{\pgfqpoint{6.952293in}{1.811526in}}{\pgfqpoint{6.958117in}{1.817350in}}%
\pgfpathcurveto{\pgfqpoint{6.963940in}{1.823174in}}{\pgfqpoint{6.967213in}{1.831074in}}{\pgfqpoint{6.967213in}{1.839311in}}%
\pgfpathcurveto{\pgfqpoint{6.967213in}{1.847547in}}{\pgfqpoint{6.963940in}{1.855447in}}{\pgfqpoint{6.958117in}{1.861271in}}%
\pgfpathcurveto{\pgfqpoint{6.952293in}{1.867095in}}{\pgfqpoint{6.944393in}{1.870367in}}{\pgfqpoint{6.936156in}{1.870367in}}%
\pgfpathcurveto{\pgfqpoint{6.927920in}{1.870367in}}{\pgfqpoint{6.920020in}{1.867095in}}{\pgfqpoint{6.914196in}{1.861271in}}%
\pgfpathcurveto{\pgfqpoint{6.908372in}{1.855447in}}{\pgfqpoint{6.905100in}{1.847547in}}{\pgfqpoint{6.905100in}{1.839311in}}%
\pgfpathcurveto{\pgfqpoint{6.905100in}{1.831074in}}{\pgfqpoint{6.908372in}{1.823174in}}{\pgfqpoint{6.914196in}{1.817350in}}%
\pgfpathcurveto{\pgfqpoint{6.920020in}{1.811526in}}{\pgfqpoint{6.927920in}{1.808254in}}{\pgfqpoint{6.936156in}{1.808254in}}%
\pgfpathclose%
\pgfusepath{stroke,fill}%
\end{pgfscope}%
\begin{pgfscope}%
\pgfpathrectangle{\pgfqpoint{0.894063in}{0.630000in}}{\pgfqpoint{6.713438in}{2.060556in}} %
\pgfusepath{clip}%
\pgfsetbuttcap%
\pgfsetroundjoin%
\definecolor{currentfill}{rgb}{0.501961,0.000000,0.501961}%
\pgfsetfillcolor{currentfill}%
\pgfsetlinewidth{1.003750pt}%
\definecolor{currentstroke}{rgb}{0.501961,0.000000,0.501961}%
\pgfsetstrokecolor{currentstroke}%
\pgfsetdash{}{0pt}%
\pgfpathmoveto{\pgfqpoint{5.862006in}{1.594398in}}%
\pgfpathcurveto{\pgfqpoint{5.870243in}{1.594398in}}{\pgfqpoint{5.878143in}{1.597670in}}{\pgfqpoint{5.883967in}{1.603494in}}%
\pgfpathcurveto{\pgfqpoint{5.889790in}{1.609318in}}{\pgfqpoint{5.893063in}{1.617218in}}{\pgfqpoint{5.893063in}{1.625454in}}%
\pgfpathcurveto{\pgfqpoint{5.893063in}{1.633691in}}{\pgfqpoint{5.889790in}{1.641591in}}{\pgfqpoint{5.883967in}{1.647415in}}%
\pgfpathcurveto{\pgfqpoint{5.878143in}{1.653239in}}{\pgfqpoint{5.870243in}{1.656511in}}{\pgfqpoint{5.862006in}{1.656511in}}%
\pgfpathcurveto{\pgfqpoint{5.853770in}{1.656511in}}{\pgfqpoint{5.845870in}{1.653239in}}{\pgfqpoint{5.840046in}{1.647415in}}%
\pgfpathcurveto{\pgfqpoint{5.834222in}{1.641591in}}{\pgfqpoint{5.830950in}{1.633691in}}{\pgfqpoint{5.830950in}{1.625454in}}%
\pgfpathcurveto{\pgfqpoint{5.830950in}{1.617218in}}{\pgfqpoint{5.834222in}{1.609318in}}{\pgfqpoint{5.840046in}{1.603494in}}%
\pgfpathcurveto{\pgfqpoint{5.845870in}{1.597670in}}{\pgfqpoint{5.853770in}{1.594398in}}{\pgfqpoint{5.862006in}{1.594398in}}%
\pgfpathclose%
\pgfusepath{stroke,fill}%
\end{pgfscope}%
\begin{pgfscope}%
\pgfpathrectangle{\pgfqpoint{0.894063in}{0.630000in}}{\pgfqpoint{6.713438in}{2.060556in}} %
\pgfusepath{clip}%
\pgfsetbuttcap%
\pgfsetroundjoin%
\definecolor{currentfill}{rgb}{0.501961,0.000000,0.501961}%
\pgfsetfillcolor{currentfill}%
\pgfsetlinewidth{1.003750pt}%
\definecolor{currentstroke}{rgb}{0.501961,0.000000,0.501961}%
\pgfsetstrokecolor{currentstroke}%
\pgfsetdash{}{0pt}%
\pgfpathmoveto{\pgfqpoint{7.070425in}{1.836784in}}%
\pgfpathcurveto{\pgfqpoint{7.078661in}{1.836784in}}{\pgfqpoint{7.086561in}{1.840056in}}{\pgfqpoint{7.092385in}{1.845880in}}%
\pgfpathcurveto{\pgfqpoint{7.098209in}{1.851704in}}{\pgfqpoint{7.101481in}{1.859604in}}{\pgfqpoint{7.101481in}{1.867840in}}%
\pgfpathcurveto{\pgfqpoint{7.101481in}{1.876077in}}{\pgfqpoint{7.098209in}{1.883977in}}{\pgfqpoint{7.092385in}{1.889801in}}%
\pgfpathcurveto{\pgfqpoint{7.086561in}{1.895625in}}{\pgfqpoint{7.078661in}{1.898897in}}{\pgfqpoint{7.070425in}{1.898897in}}%
\pgfpathcurveto{\pgfqpoint{7.062189in}{1.898897in}}{\pgfqpoint{7.054289in}{1.895625in}}{\pgfqpoint{7.048465in}{1.889801in}}%
\pgfpathcurveto{\pgfqpoint{7.042641in}{1.883977in}}{\pgfqpoint{7.039369in}{1.876077in}}{\pgfqpoint{7.039369in}{1.867840in}}%
\pgfpathcurveto{\pgfqpoint{7.039369in}{1.859604in}}{\pgfqpoint{7.042641in}{1.851704in}}{\pgfqpoint{7.048465in}{1.845880in}}%
\pgfpathcurveto{\pgfqpoint{7.054289in}{1.840056in}}{\pgfqpoint{7.062189in}{1.836784in}}{\pgfqpoint{7.070425in}{1.836784in}}%
\pgfpathclose%
\pgfusepath{stroke,fill}%
\end{pgfscope}%
\begin{pgfscope}%
\pgfpathrectangle{\pgfqpoint{0.894063in}{0.630000in}}{\pgfqpoint{6.713438in}{2.060556in}} %
\pgfusepath{clip}%
\pgfsetbuttcap%
\pgfsetroundjoin%
\definecolor{currentfill}{rgb}{0.501961,0.000000,0.501961}%
\pgfsetfillcolor{currentfill}%
\pgfsetlinewidth{1.003750pt}%
\definecolor{currentstroke}{rgb}{0.501961,0.000000,0.501961}%
\pgfsetstrokecolor{currentstroke}%
\pgfsetdash{}{0pt}%
\pgfpathmoveto{\pgfqpoint{3.176631in}{1.092264in}}%
\pgfpathcurveto{\pgfqpoint{3.184868in}{1.092264in}}{\pgfqpoint{3.192768in}{1.095536in}}{\pgfqpoint{3.198592in}{1.101360in}}%
\pgfpathcurveto{\pgfqpoint{3.204415in}{1.107184in}}{\pgfqpoint{3.207688in}{1.115084in}}{\pgfqpoint{3.207688in}{1.123321in}}%
\pgfpathcurveto{\pgfqpoint{3.207688in}{1.131557in}}{\pgfqpoint{3.204415in}{1.139457in}}{\pgfqpoint{3.198592in}{1.145281in}}%
\pgfpathcurveto{\pgfqpoint{3.192768in}{1.151105in}}{\pgfqpoint{3.184868in}{1.154377in}}{\pgfqpoint{3.176631in}{1.154377in}}%
\pgfpathcurveto{\pgfqpoint{3.168395in}{1.154377in}}{\pgfqpoint{3.160495in}{1.151105in}}{\pgfqpoint{3.154671in}{1.145281in}}%
\pgfpathcurveto{\pgfqpoint{3.148847in}{1.139457in}}{\pgfqpoint{3.145575in}{1.131557in}}{\pgfqpoint{3.145575in}{1.123321in}}%
\pgfpathcurveto{\pgfqpoint{3.145575in}{1.115084in}}{\pgfqpoint{3.148847in}{1.107184in}}{\pgfqpoint{3.154671in}{1.101360in}}%
\pgfpathcurveto{\pgfqpoint{3.160495in}{1.095536in}}{\pgfqpoint{3.168395in}{1.092264in}}{\pgfqpoint{3.176631in}{1.092264in}}%
\pgfpathclose%
\pgfusepath{stroke,fill}%
\end{pgfscope}%
\begin{pgfscope}%
\pgfpathrectangle{\pgfqpoint{0.894063in}{0.630000in}}{\pgfqpoint{6.713438in}{2.060556in}} %
\pgfusepath{clip}%
\pgfsetbuttcap%
\pgfsetroundjoin%
\definecolor{currentfill}{rgb}{0.501961,0.000000,0.501961}%
\pgfsetfillcolor{currentfill}%
\pgfsetlinewidth{1.003750pt}%
\definecolor{currentstroke}{rgb}{0.501961,0.000000,0.501961}%
\pgfsetstrokecolor{currentstroke}%
\pgfsetdash{}{0pt}%
\pgfpathmoveto{\pgfqpoint{2.102481in}{0.877066in}}%
\pgfpathcurveto{\pgfqpoint{2.110718in}{0.877066in}}{\pgfqpoint{2.118618in}{0.880338in}}{\pgfqpoint{2.124442in}{0.886162in}}%
\pgfpathcurveto{\pgfqpoint{2.130265in}{0.891986in}}{\pgfqpoint{2.133538in}{0.899886in}}{\pgfqpoint{2.133538in}{0.908122in}}%
\pgfpathcurveto{\pgfqpoint{2.133538in}{0.916358in}}{\pgfqpoint{2.130265in}{0.924258in}}{\pgfqpoint{2.124442in}{0.930082in}}%
\pgfpathcurveto{\pgfqpoint{2.118618in}{0.935906in}}{\pgfqpoint{2.110718in}{0.939179in}}{\pgfqpoint{2.102481in}{0.939179in}}%
\pgfpathcurveto{\pgfqpoint{2.094245in}{0.939179in}}{\pgfqpoint{2.086345in}{0.935906in}}{\pgfqpoint{2.080521in}{0.930082in}}%
\pgfpathcurveto{\pgfqpoint{2.074697in}{0.924258in}}{\pgfqpoint{2.071425in}{0.916358in}}{\pgfqpoint{2.071425in}{0.908122in}}%
\pgfpathcurveto{\pgfqpoint{2.071425in}{0.899886in}}{\pgfqpoint{2.074697in}{0.891986in}}{\pgfqpoint{2.080521in}{0.886162in}}%
\pgfpathcurveto{\pgfqpoint{2.086345in}{0.880338in}}{\pgfqpoint{2.094245in}{0.877066in}}{\pgfqpoint{2.102481in}{0.877066in}}%
\pgfpathclose%
\pgfusepath{stroke,fill}%
\end{pgfscope}%
\begin{pgfscope}%
\pgfpathrectangle{\pgfqpoint{0.894063in}{0.630000in}}{\pgfqpoint{6.713438in}{2.060556in}} %
\pgfusepath{clip}%
\pgfsetbuttcap%
\pgfsetroundjoin%
\definecolor{currentfill}{rgb}{0.501961,0.000000,0.501961}%
\pgfsetfillcolor{currentfill}%
\pgfsetlinewidth{1.003750pt}%
\definecolor{currentstroke}{rgb}{0.501961,0.000000,0.501961}%
\pgfsetstrokecolor{currentstroke}%
\pgfsetdash{}{0pt}%
\pgfpathmoveto{\pgfqpoint{1.968213in}{0.852775in}}%
\pgfpathcurveto{\pgfqpoint{1.976449in}{0.852775in}}{\pgfqpoint{1.984349in}{0.856047in}}{\pgfqpoint{1.990173in}{0.861871in}}%
\pgfpathcurveto{\pgfqpoint{1.995997in}{0.867695in}}{\pgfqpoint{1.999269in}{0.875595in}}{\pgfqpoint{1.999269in}{0.883831in}}%
\pgfpathcurveto{\pgfqpoint{1.999269in}{0.892067in}}{\pgfqpoint{1.995997in}{0.899967in}}{\pgfqpoint{1.990173in}{0.905791in}}%
\pgfpathcurveto{\pgfqpoint{1.984349in}{0.911615in}}{\pgfqpoint{1.976449in}{0.914888in}}{\pgfqpoint{1.968213in}{0.914888in}}%
\pgfpathcurveto{\pgfqpoint{1.959976in}{0.914888in}}{\pgfqpoint{1.952076in}{0.911615in}}{\pgfqpoint{1.946252in}{0.905791in}}%
\pgfpathcurveto{\pgfqpoint{1.940428in}{0.899967in}}{\pgfqpoint{1.937156in}{0.892067in}}{\pgfqpoint{1.937156in}{0.883831in}}%
\pgfpathcurveto{\pgfqpoint{1.937156in}{0.875595in}}{\pgfqpoint{1.940428in}{0.867695in}}{\pgfqpoint{1.946252in}{0.861871in}}%
\pgfpathcurveto{\pgfqpoint{1.952076in}{0.856047in}}{\pgfqpoint{1.959976in}{0.852775in}}{\pgfqpoint{1.968213in}{0.852775in}}%
\pgfpathclose%
\pgfusepath{stroke,fill}%
\end{pgfscope}%
\begin{pgfscope}%
\pgfpathrectangle{\pgfqpoint{0.894063in}{0.630000in}}{\pgfqpoint{6.713438in}{2.060556in}} %
\pgfusepath{clip}%
\pgfsetbuttcap%
\pgfsetroundjoin%
\definecolor{currentfill}{rgb}{0.501961,0.000000,0.501961}%
\pgfsetfillcolor{currentfill}%
\pgfsetlinewidth{1.003750pt}%
\definecolor{currentstroke}{rgb}{0.501961,0.000000,0.501961}%
\pgfsetstrokecolor{currentstroke}%
\pgfsetdash{}{0pt}%
\pgfpathmoveto{\pgfqpoint{3.310900in}{1.122949in}}%
\pgfpathcurveto{\pgfqpoint{3.319136in}{1.122949in}}{\pgfqpoint{3.327036in}{1.126221in}}{\pgfqpoint{3.332860in}{1.132045in}}%
\pgfpathcurveto{\pgfqpoint{3.338684in}{1.137869in}}{\pgfqpoint{3.341956in}{1.145769in}}{\pgfqpoint{3.341956in}{1.154005in}}%
\pgfpathcurveto{\pgfqpoint{3.341956in}{1.162241in}}{\pgfqpoint{3.338684in}{1.170141in}}{\pgfqpoint{3.332860in}{1.175965in}}%
\pgfpathcurveto{\pgfqpoint{3.327036in}{1.181789in}}{\pgfqpoint{3.319136in}{1.185062in}}{\pgfqpoint{3.310900in}{1.185062in}}%
\pgfpathcurveto{\pgfqpoint{3.302664in}{1.185062in}}{\pgfqpoint{3.294764in}{1.181789in}}{\pgfqpoint{3.288940in}{1.175965in}}%
\pgfpathcurveto{\pgfqpoint{3.283116in}{1.170141in}}{\pgfqpoint{3.279844in}{1.162241in}}{\pgfqpoint{3.279844in}{1.154005in}}%
\pgfpathcurveto{\pgfqpoint{3.279844in}{1.145769in}}{\pgfqpoint{3.283116in}{1.137869in}}{\pgfqpoint{3.288940in}{1.132045in}}%
\pgfpathcurveto{\pgfqpoint{3.294764in}{1.126221in}}{\pgfqpoint{3.302664in}{1.122949in}}{\pgfqpoint{3.310900in}{1.122949in}}%
\pgfpathclose%
\pgfusepath{stroke,fill}%
\end{pgfscope}%
\begin{pgfscope}%
\pgfpathrectangle{\pgfqpoint{0.894063in}{0.630000in}}{\pgfqpoint{6.713438in}{2.060556in}} %
\pgfusepath{clip}%
\pgfsetbuttcap%
\pgfsetroundjoin%
\definecolor{currentfill}{rgb}{0.501961,0.000000,0.501961}%
\pgfsetfillcolor{currentfill}%
\pgfsetlinewidth{1.003750pt}%
\definecolor{currentstroke}{rgb}{0.501961,0.000000,0.501961}%
\pgfsetstrokecolor{currentstroke}%
\pgfsetdash{}{0pt}%
\pgfpathmoveto{\pgfqpoint{5.593469in}{1.533912in}}%
\pgfpathcurveto{\pgfqpoint{5.601705in}{1.533912in}}{\pgfqpoint{5.609605in}{1.537184in}}{\pgfqpoint{5.615429in}{1.543008in}}%
\pgfpathcurveto{\pgfqpoint{5.621253in}{1.548832in}}{\pgfqpoint{5.624525in}{1.556732in}}{\pgfqpoint{5.624525in}{1.564968in}}%
\pgfpathcurveto{\pgfqpoint{5.624525in}{1.573205in}}{\pgfqpoint{5.621253in}{1.581105in}}{\pgfqpoint{5.615429in}{1.586929in}}%
\pgfpathcurveto{\pgfqpoint{5.609605in}{1.592752in}}{\pgfqpoint{5.601705in}{1.596025in}}{\pgfqpoint{5.593469in}{1.596025in}}%
\pgfpathcurveto{\pgfqpoint{5.585232in}{1.596025in}}{\pgfqpoint{5.577332in}{1.592752in}}{\pgfqpoint{5.571508in}{1.586929in}}%
\pgfpathcurveto{\pgfqpoint{5.565685in}{1.581105in}}{\pgfqpoint{5.562412in}{1.573205in}}{\pgfqpoint{5.562412in}{1.564968in}}%
\pgfpathcurveto{\pgfqpoint{5.562412in}{1.556732in}}{\pgfqpoint{5.565685in}{1.548832in}}{\pgfqpoint{5.571508in}{1.543008in}}%
\pgfpathcurveto{\pgfqpoint{5.577332in}{1.537184in}}{\pgfqpoint{5.585232in}{1.533912in}}{\pgfqpoint{5.593469in}{1.533912in}}%
\pgfpathclose%
\pgfusepath{stroke,fill}%
\end{pgfscope}%
\begin{pgfscope}%
\pgfpathrectangle{\pgfqpoint{0.894063in}{0.630000in}}{\pgfqpoint{6.713438in}{2.060556in}} %
\pgfusepath{clip}%
\pgfsetbuttcap%
\pgfsetroundjoin%
\definecolor{currentfill}{rgb}{0.501961,0.000000,0.501961}%
\pgfsetfillcolor{currentfill}%
\pgfsetlinewidth{1.003750pt}%
\definecolor{currentstroke}{rgb}{0.501961,0.000000,0.501961}%
\pgfsetstrokecolor{currentstroke}%
\pgfsetdash{}{0pt}%
\pgfpathmoveto{\pgfqpoint{3.042363in}{1.070446in}}%
\pgfpathcurveto{\pgfqpoint{3.050599in}{1.070446in}}{\pgfqpoint{3.058499in}{1.073718in}}{\pgfqpoint{3.064323in}{1.079542in}}%
\pgfpathcurveto{\pgfqpoint{3.070147in}{1.085366in}}{\pgfqpoint{3.073419in}{1.093266in}}{\pgfqpoint{3.073419in}{1.101502in}}%
\pgfpathcurveto{\pgfqpoint{3.073419in}{1.109738in}}{\pgfqpoint{3.070147in}{1.117639in}}{\pgfqpoint{3.064323in}{1.123462in}}%
\pgfpathcurveto{\pgfqpoint{3.058499in}{1.129286in}}{\pgfqpoint{3.050599in}{1.132559in}}{\pgfqpoint{3.042363in}{1.132559in}}%
\pgfpathcurveto{\pgfqpoint{3.034126in}{1.132559in}}{\pgfqpoint{3.026226in}{1.129286in}}{\pgfqpoint{3.020402in}{1.123462in}}%
\pgfpathcurveto{\pgfqpoint{3.014578in}{1.117639in}}{\pgfqpoint{3.011306in}{1.109738in}}{\pgfqpoint{3.011306in}{1.101502in}}%
\pgfpathcurveto{\pgfqpoint{3.011306in}{1.093266in}}{\pgfqpoint{3.014578in}{1.085366in}}{\pgfqpoint{3.020402in}{1.079542in}}%
\pgfpathcurveto{\pgfqpoint{3.026226in}{1.073718in}}{\pgfqpoint{3.034126in}{1.070446in}}{\pgfqpoint{3.042363in}{1.070446in}}%
\pgfpathclose%
\pgfusepath{stroke,fill}%
\end{pgfscope}%
\begin{pgfscope}%
\pgfpathrectangle{\pgfqpoint{0.894063in}{0.630000in}}{\pgfqpoint{6.713438in}{2.060556in}} %
\pgfusepath{clip}%
\pgfsetbuttcap%
\pgfsetroundjoin%
\definecolor{currentfill}{rgb}{0.501961,0.000000,0.501961}%
\pgfsetfillcolor{currentfill}%
\pgfsetlinewidth{1.003750pt}%
\definecolor{currentstroke}{rgb}{0.501961,0.000000,0.501961}%
\pgfsetstrokecolor{currentstroke}%
\pgfsetdash{}{0pt}%
\pgfpathmoveto{\pgfqpoint{5.190663in}{1.462770in}}%
\pgfpathcurveto{\pgfqpoint{5.198899in}{1.462770in}}{\pgfqpoint{5.206799in}{1.466042in}}{\pgfqpoint{5.212623in}{1.471866in}}%
\pgfpathcurveto{\pgfqpoint{5.218447in}{1.477690in}}{\pgfqpoint{5.221719in}{1.485590in}}{\pgfqpoint{5.221719in}{1.493826in}}%
\pgfpathcurveto{\pgfqpoint{5.221719in}{1.502062in}}{\pgfqpoint{5.218447in}{1.509962in}}{\pgfqpoint{5.212623in}{1.515786in}}%
\pgfpathcurveto{\pgfqpoint{5.206799in}{1.521610in}}{\pgfqpoint{5.198899in}{1.524883in}}{\pgfqpoint{5.190663in}{1.524883in}}%
\pgfpathcurveto{\pgfqpoint{5.182426in}{1.524883in}}{\pgfqpoint{5.174526in}{1.521610in}}{\pgfqpoint{5.168702in}{1.515786in}}%
\pgfpathcurveto{\pgfqpoint{5.162878in}{1.509962in}}{\pgfqpoint{5.159606in}{1.502062in}}{\pgfqpoint{5.159606in}{1.493826in}}%
\pgfpathcurveto{\pgfqpoint{5.159606in}{1.485590in}}{\pgfqpoint{5.162878in}{1.477690in}}{\pgfqpoint{5.168702in}{1.471866in}}%
\pgfpathcurveto{\pgfqpoint{5.174526in}{1.466042in}}{\pgfqpoint{5.182426in}{1.462770in}}{\pgfqpoint{5.190663in}{1.462770in}}%
\pgfpathclose%
\pgfusepath{stroke,fill}%
\end{pgfscope}%
\begin{pgfscope}%
\pgfpathrectangle{\pgfqpoint{0.894063in}{0.630000in}}{\pgfqpoint{6.713438in}{2.060556in}} %
\pgfusepath{clip}%
\pgfsetbuttcap%
\pgfsetroundjoin%
\definecolor{currentfill}{rgb}{0.501961,0.000000,0.501961}%
\pgfsetfillcolor{currentfill}%
\pgfsetlinewidth{1.003750pt}%
\definecolor{currentstroke}{rgb}{0.501961,0.000000,0.501961}%
\pgfsetstrokecolor{currentstroke}%
\pgfsetdash{}{0pt}%
\pgfpathmoveto{\pgfqpoint{6.801888in}{1.781243in}}%
\pgfpathcurveto{\pgfqpoint{6.810124in}{1.781243in}}{\pgfqpoint{6.818024in}{1.784515in}}{\pgfqpoint{6.823848in}{1.790339in}}%
\pgfpathcurveto{\pgfqpoint{6.829672in}{1.796163in}}{\pgfqpoint{6.832944in}{1.804063in}}{\pgfqpoint{6.832944in}{1.812300in}}%
\pgfpathcurveto{\pgfqpoint{6.832944in}{1.820536in}}{\pgfqpoint{6.829672in}{1.828436in}}{\pgfqpoint{6.823848in}{1.834260in}}%
\pgfpathcurveto{\pgfqpoint{6.818024in}{1.840084in}}{\pgfqpoint{6.810124in}{1.843356in}}{\pgfqpoint{6.801888in}{1.843356in}}%
\pgfpathcurveto{\pgfqpoint{6.793651in}{1.843356in}}{\pgfqpoint{6.785751in}{1.840084in}}{\pgfqpoint{6.779927in}{1.834260in}}%
\pgfpathcurveto{\pgfqpoint{6.774103in}{1.828436in}}{\pgfqpoint{6.770831in}{1.820536in}}{\pgfqpoint{6.770831in}{1.812300in}}%
\pgfpathcurveto{\pgfqpoint{6.770831in}{1.804063in}}{\pgfqpoint{6.774103in}{1.796163in}}{\pgfqpoint{6.779927in}{1.790339in}}%
\pgfpathcurveto{\pgfqpoint{6.785751in}{1.784515in}}{\pgfqpoint{6.793651in}{1.781243in}}{\pgfqpoint{6.801888in}{1.781243in}}%
\pgfpathclose%
\pgfusepath{stroke,fill}%
\end{pgfscope}%
\begin{pgfscope}%
\pgfpathrectangle{\pgfqpoint{0.894063in}{0.630000in}}{\pgfqpoint{6.713438in}{2.060556in}} %
\pgfusepath{clip}%
\pgfsetbuttcap%
\pgfsetroundjoin%
\definecolor{currentfill}{rgb}{0.501961,0.000000,0.501961}%
\pgfsetfillcolor{currentfill}%
\pgfsetlinewidth{1.003750pt}%
\definecolor{currentstroke}{rgb}{0.501961,0.000000,0.501961}%
\pgfsetstrokecolor{currentstroke}%
\pgfsetdash{}{0pt}%
\pgfpathmoveto{\pgfqpoint{3.579438in}{1.179308in}}%
\pgfpathcurveto{\pgfqpoint{3.587674in}{1.179308in}}{\pgfqpoint{3.595574in}{1.182580in}}{\pgfqpoint{3.601398in}{1.188404in}}%
\pgfpathcurveto{\pgfqpoint{3.607222in}{1.194228in}}{\pgfqpoint{3.610494in}{1.202128in}}{\pgfqpoint{3.610494in}{1.210364in}}%
\pgfpathcurveto{\pgfqpoint{3.610494in}{1.218601in}}{\pgfqpoint{3.607222in}{1.226501in}}{\pgfqpoint{3.601398in}{1.232325in}}%
\pgfpathcurveto{\pgfqpoint{3.595574in}{1.238148in}}{\pgfqpoint{3.587674in}{1.241421in}}{\pgfqpoint{3.579438in}{1.241421in}}%
\pgfpathcurveto{\pgfqpoint{3.571201in}{1.241421in}}{\pgfqpoint{3.563301in}{1.238148in}}{\pgfqpoint{3.557477in}{1.232325in}}%
\pgfpathcurveto{\pgfqpoint{3.551653in}{1.226501in}}{\pgfqpoint{3.548381in}{1.218601in}}{\pgfqpoint{3.548381in}{1.210364in}}%
\pgfpathcurveto{\pgfqpoint{3.548381in}{1.202128in}}{\pgfqpoint{3.551653in}{1.194228in}}{\pgfqpoint{3.557477in}{1.188404in}}%
\pgfpathcurveto{\pgfqpoint{3.563301in}{1.182580in}}{\pgfqpoint{3.571201in}{1.179308in}}{\pgfqpoint{3.579438in}{1.179308in}}%
\pgfpathclose%
\pgfusepath{stroke,fill}%
\end{pgfscope}%
\begin{pgfscope}%
\pgfpathrectangle{\pgfqpoint{0.894063in}{0.630000in}}{\pgfqpoint{6.713438in}{2.060556in}} %
\pgfusepath{clip}%
\pgfsetbuttcap%
\pgfsetroundjoin%
\definecolor{currentfill}{rgb}{0.501961,0.000000,0.501961}%
\pgfsetfillcolor{currentfill}%
\pgfsetlinewidth{1.003750pt}%
\definecolor{currentstroke}{rgb}{0.501961,0.000000,0.501961}%
\pgfsetstrokecolor{currentstroke}%
\pgfsetdash{}{0pt}%
\pgfpathmoveto{\pgfqpoint{2.371019in}{0.935103in}}%
\pgfpathcurveto{\pgfqpoint{2.379255in}{0.935103in}}{\pgfqpoint{2.387155in}{0.938375in}}{\pgfqpoint{2.392979in}{0.944199in}}%
\pgfpathcurveto{\pgfqpoint{2.398803in}{0.950023in}}{\pgfqpoint{2.402075in}{0.957923in}}{\pgfqpoint{2.402075in}{0.966159in}}%
\pgfpathcurveto{\pgfqpoint{2.402075in}{0.974395in}}{\pgfqpoint{2.398803in}{0.982295in}}{\pgfqpoint{2.392979in}{0.988119in}}%
\pgfpathcurveto{\pgfqpoint{2.387155in}{0.993943in}}{\pgfqpoint{2.379255in}{0.997216in}}{\pgfqpoint{2.371019in}{0.997216in}}%
\pgfpathcurveto{\pgfqpoint{2.362782in}{0.997216in}}{\pgfqpoint{2.354882in}{0.993943in}}{\pgfqpoint{2.349058in}{0.988119in}}%
\pgfpathcurveto{\pgfqpoint{2.343235in}{0.982295in}}{\pgfqpoint{2.339962in}{0.974395in}}{\pgfqpoint{2.339962in}{0.966159in}}%
\pgfpathcurveto{\pgfqpoint{2.339962in}{0.957923in}}{\pgfqpoint{2.343235in}{0.950023in}}{\pgfqpoint{2.349058in}{0.944199in}}%
\pgfpathcurveto{\pgfqpoint{2.354882in}{0.938375in}}{\pgfqpoint{2.362782in}{0.935103in}}{\pgfqpoint{2.371019in}{0.935103in}}%
\pgfpathclose%
\pgfusepath{stroke,fill}%
\end{pgfscope}%
\begin{pgfscope}%
\pgfpathrectangle{\pgfqpoint{0.894063in}{0.630000in}}{\pgfqpoint{6.713438in}{2.060556in}} %
\pgfusepath{clip}%
\pgfsetbuttcap%
\pgfsetroundjoin%
\definecolor{currentfill}{rgb}{0.501961,0.000000,0.501961}%
\pgfsetfillcolor{currentfill}%
\pgfsetlinewidth{1.003750pt}%
\definecolor{currentstroke}{rgb}{0.501961,0.000000,0.501961}%
\pgfsetstrokecolor{currentstroke}%
\pgfsetdash{}{0pt}%
\pgfpathmoveto{\pgfqpoint{3.982244in}{1.252958in}}%
\pgfpathcurveto{\pgfqpoint{3.990480in}{1.252958in}}{\pgfqpoint{3.998380in}{1.256230in}}{\pgfqpoint{4.004204in}{1.262054in}}%
\pgfpathcurveto{\pgfqpoint{4.010028in}{1.267878in}}{\pgfqpoint{4.013300in}{1.275778in}}{\pgfqpoint{4.013300in}{1.284014in}}%
\pgfpathcurveto{\pgfqpoint{4.013300in}{1.292251in}}{\pgfqpoint{4.010028in}{1.300151in}}{\pgfqpoint{4.004204in}{1.305975in}}%
\pgfpathcurveto{\pgfqpoint{3.998380in}{1.311799in}}{\pgfqpoint{3.990480in}{1.315071in}}{\pgfqpoint{3.982244in}{1.315071in}}%
\pgfpathcurveto{\pgfqpoint{3.974007in}{1.315071in}}{\pgfqpoint{3.966107in}{1.311799in}}{\pgfqpoint{3.960283in}{1.305975in}}%
\pgfpathcurveto{\pgfqpoint{3.954460in}{1.300151in}}{\pgfqpoint{3.951187in}{1.292251in}}{\pgfqpoint{3.951187in}{1.284014in}}%
\pgfpathcurveto{\pgfqpoint{3.951187in}{1.275778in}}{\pgfqpoint{3.954460in}{1.267878in}}{\pgfqpoint{3.960283in}{1.262054in}}%
\pgfpathcurveto{\pgfqpoint{3.966107in}{1.256230in}}{\pgfqpoint{3.974007in}{1.252958in}}{\pgfqpoint{3.982244in}{1.252958in}}%
\pgfpathclose%
\pgfusepath{stroke,fill}%
\end{pgfscope}%
\begin{pgfscope}%
\pgfpathrectangle{\pgfqpoint{0.894063in}{0.630000in}}{\pgfqpoint{6.713438in}{2.060556in}} %
\pgfusepath{clip}%
\pgfsetbuttcap%
\pgfsetroundjoin%
\definecolor{currentfill}{rgb}{0.501961,0.000000,0.501961}%
\pgfsetfillcolor{currentfill}%
\pgfsetlinewidth{1.003750pt}%
\definecolor{currentstroke}{rgb}{0.501961,0.000000,0.501961}%
\pgfsetstrokecolor{currentstroke}%
\pgfsetdash{}{0pt}%
\pgfpathmoveto{\pgfqpoint{4.653588in}{1.358847in}}%
\pgfpathcurveto{\pgfqpoint{4.661824in}{1.358847in}}{\pgfqpoint{4.669724in}{1.362119in}}{\pgfqpoint{4.675548in}{1.367943in}}%
\pgfpathcurveto{\pgfqpoint{4.681372in}{1.373767in}}{\pgfqpoint{4.684644in}{1.381667in}}{\pgfqpoint{4.684644in}{1.389903in}}%
\pgfpathcurveto{\pgfqpoint{4.684644in}{1.398140in}}{\pgfqpoint{4.681372in}{1.406040in}}{\pgfqpoint{4.675548in}{1.411864in}}%
\pgfpathcurveto{\pgfqpoint{4.669724in}{1.417688in}}{\pgfqpoint{4.661824in}{1.420960in}}{\pgfqpoint{4.653588in}{1.420960in}}%
\pgfpathcurveto{\pgfqpoint{4.645351in}{1.420960in}}{\pgfqpoint{4.637451in}{1.417688in}}{\pgfqpoint{4.631627in}{1.411864in}}%
\pgfpathcurveto{\pgfqpoint{4.625803in}{1.406040in}}{\pgfqpoint{4.622531in}{1.398140in}}{\pgfqpoint{4.622531in}{1.389903in}}%
\pgfpathcurveto{\pgfqpoint{4.622531in}{1.381667in}}{\pgfqpoint{4.625803in}{1.373767in}}{\pgfqpoint{4.631627in}{1.367943in}}%
\pgfpathcurveto{\pgfqpoint{4.637451in}{1.362119in}}{\pgfqpoint{4.645351in}{1.358847in}}{\pgfqpoint{4.653588in}{1.358847in}}%
\pgfpathclose%
\pgfusepath{stroke,fill}%
\end{pgfscope}%
\begin{pgfscope}%
\pgfpathrectangle{\pgfqpoint{0.894063in}{0.630000in}}{\pgfqpoint{6.713438in}{2.060556in}} %
\pgfusepath{clip}%
\pgfsetbuttcap%
\pgfsetroundjoin%
\definecolor{currentfill}{rgb}{0.501961,0.000000,0.501961}%
\pgfsetfillcolor{currentfill}%
\pgfsetlinewidth{1.003750pt}%
\definecolor{currentstroke}{rgb}{0.501961,0.000000,0.501961}%
\pgfsetstrokecolor{currentstroke}%
\pgfsetdash{}{0pt}%
\pgfpathmoveto{\pgfqpoint{3.713706in}{1.210110in}}%
\pgfpathcurveto{\pgfqpoint{3.721943in}{1.210110in}}{\pgfqpoint{3.729843in}{1.213382in}}{\pgfqpoint{3.735667in}{1.219206in}}%
\pgfpathcurveto{\pgfqpoint{3.741490in}{1.225030in}}{\pgfqpoint{3.744763in}{1.232930in}}{\pgfqpoint{3.744763in}{1.241167in}}%
\pgfpathcurveto{\pgfqpoint{3.744763in}{1.249403in}}{\pgfqpoint{3.741490in}{1.257303in}}{\pgfqpoint{3.735667in}{1.263127in}}%
\pgfpathcurveto{\pgfqpoint{3.729843in}{1.268951in}}{\pgfqpoint{3.721943in}{1.272223in}}{\pgfqpoint{3.713706in}{1.272223in}}%
\pgfpathcurveto{\pgfqpoint{3.705470in}{1.272223in}}{\pgfqpoint{3.697570in}{1.268951in}}{\pgfqpoint{3.691746in}{1.263127in}}%
\pgfpathcurveto{\pgfqpoint{3.685922in}{1.257303in}}{\pgfqpoint{3.682650in}{1.249403in}}{\pgfqpoint{3.682650in}{1.241167in}}%
\pgfpathcurveto{\pgfqpoint{3.682650in}{1.232930in}}{\pgfqpoint{3.685922in}{1.225030in}}{\pgfqpoint{3.691746in}{1.219206in}}%
\pgfpathcurveto{\pgfqpoint{3.697570in}{1.213382in}}{\pgfqpoint{3.705470in}{1.210110in}}{\pgfqpoint{3.713706in}{1.210110in}}%
\pgfpathclose%
\pgfusepath{stroke,fill}%
\end{pgfscope}%
\begin{pgfscope}%
\pgfpathrectangle{\pgfqpoint{0.894063in}{0.630000in}}{\pgfqpoint{6.713438in}{2.060556in}} %
\pgfusepath{clip}%
\pgfsetbuttcap%
\pgfsetroundjoin%
\definecolor{currentfill}{rgb}{0.501961,0.000000,0.501961}%
\pgfsetfillcolor{currentfill}%
\pgfsetlinewidth{1.003750pt}%
\definecolor{currentstroke}{rgb}{0.501961,0.000000,0.501961}%
\pgfsetstrokecolor{currentstroke}%
\pgfsetdash{}{0pt}%
\pgfpathmoveto{\pgfqpoint{2.236750in}{0.906714in}}%
\pgfpathcurveto{\pgfqpoint{2.244986in}{0.906714in}}{\pgfqpoint{2.252886in}{0.909986in}}{\pgfqpoint{2.258710in}{0.915810in}}%
\pgfpathcurveto{\pgfqpoint{2.264534in}{0.921634in}}{\pgfqpoint{2.267806in}{0.929534in}}{\pgfqpoint{2.267806in}{0.937770in}}%
\pgfpathcurveto{\pgfqpoint{2.267806in}{0.946007in}}{\pgfqpoint{2.264534in}{0.953907in}}{\pgfqpoint{2.258710in}{0.959731in}}%
\pgfpathcurveto{\pgfqpoint{2.252886in}{0.965555in}}{\pgfqpoint{2.244986in}{0.968827in}}{\pgfqpoint{2.236750in}{0.968827in}}%
\pgfpathcurveto{\pgfqpoint{2.228514in}{0.968827in}}{\pgfqpoint{2.220614in}{0.965555in}}{\pgfqpoint{2.214790in}{0.959731in}}%
\pgfpathcurveto{\pgfqpoint{2.208966in}{0.953907in}}{\pgfqpoint{2.205694in}{0.946007in}}{\pgfqpoint{2.205694in}{0.937770in}}%
\pgfpathcurveto{\pgfqpoint{2.205694in}{0.929534in}}{\pgfqpoint{2.208966in}{0.921634in}}{\pgfqpoint{2.214790in}{0.915810in}}%
\pgfpathcurveto{\pgfqpoint{2.220614in}{0.909986in}}{\pgfqpoint{2.228514in}{0.906714in}}{\pgfqpoint{2.236750in}{0.906714in}}%
\pgfpathclose%
\pgfusepath{stroke,fill}%
\end{pgfscope}%
\begin{pgfscope}%
\pgfsetrectcap%
\pgfsetmiterjoin%
\pgfsetlinewidth{1.003750pt}%
\definecolor{currentstroke}{rgb}{0.000000,0.000000,0.000000}%
\pgfsetstrokecolor{currentstroke}%
\pgfsetdash{}{0pt}%
\pgfpathmoveto{\pgfqpoint{0.894063in}{2.690556in}}%
\pgfpathlineto{\pgfqpoint{7.607500in}{2.690556in}}%
\pgfusepath{stroke}%
\end{pgfscope}%
\begin{pgfscope}%
\pgfsetrectcap%
\pgfsetmiterjoin%
\pgfsetlinewidth{1.003750pt}%
\definecolor{currentstroke}{rgb}{0.000000,0.000000,0.000000}%
\pgfsetstrokecolor{currentstroke}%
\pgfsetdash{}{0pt}%
\pgfpathmoveto{\pgfqpoint{7.607500in}{0.630000in}}%
\pgfpathlineto{\pgfqpoint{7.607500in}{2.690556in}}%
\pgfusepath{stroke}%
\end{pgfscope}%
\begin{pgfscope}%
\pgfsetrectcap%
\pgfsetmiterjoin%
\pgfsetlinewidth{1.003750pt}%
\definecolor{currentstroke}{rgb}{0.000000,0.000000,0.000000}%
\pgfsetstrokecolor{currentstroke}%
\pgfsetdash{}{0pt}%
\pgfpathmoveto{\pgfqpoint{0.894063in}{0.630000in}}%
\pgfpathlineto{\pgfqpoint{7.607500in}{0.630000in}}%
\pgfusepath{stroke}%
\end{pgfscope}%
\begin{pgfscope}%
\pgfsetrectcap%
\pgfsetmiterjoin%
\pgfsetlinewidth{1.003750pt}%
\definecolor{currentstroke}{rgb}{0.000000,0.000000,0.000000}%
\pgfsetstrokecolor{currentstroke}%
\pgfsetdash{}{0pt}%
\pgfpathmoveto{\pgfqpoint{0.894063in}{0.630000in}}%
\pgfpathlineto{\pgfqpoint{0.894063in}{2.690556in}}%
\pgfusepath{stroke}%
\end{pgfscope}%
\begin{pgfscope}%
\pgfsetbuttcap%
\pgfsetroundjoin%
\definecolor{currentfill}{rgb}{0.000000,0.000000,0.000000}%
\pgfsetfillcolor{currentfill}%
\pgfsetlinewidth{0.501875pt}%
\definecolor{currentstroke}{rgb}{0.000000,0.000000,0.000000}%
\pgfsetstrokecolor{currentstroke}%
\pgfsetdash{}{0pt}%
\pgfsys@defobject{currentmarker}{\pgfqpoint{0.000000in}{0.000000in}}{\pgfqpoint{0.000000in}{0.055556in}}{%
\pgfpathmoveto{\pgfqpoint{0.000000in}{0.000000in}}%
\pgfpathlineto{\pgfqpoint{0.000000in}{0.055556in}}%
\pgfusepath{stroke,fill}%
}%
\begin{pgfscope}%
\pgfsys@transformshift{0.894063in}{0.630000in}%
\pgfsys@useobject{currentmarker}{}%
\end{pgfscope}%
\end{pgfscope}%
\begin{pgfscope}%
\pgfsetbuttcap%
\pgfsetroundjoin%
\definecolor{currentfill}{rgb}{0.000000,0.000000,0.000000}%
\pgfsetfillcolor{currentfill}%
\pgfsetlinewidth{0.501875pt}%
\definecolor{currentstroke}{rgb}{0.000000,0.000000,0.000000}%
\pgfsetstrokecolor{currentstroke}%
\pgfsetdash{}{0pt}%
\pgfsys@defobject{currentmarker}{\pgfqpoint{0.000000in}{-0.055556in}}{\pgfqpoint{0.000000in}{0.000000in}}{%
\pgfpathmoveto{\pgfqpoint{0.000000in}{0.000000in}}%
\pgfpathlineto{\pgfqpoint{0.000000in}{-0.055556in}}%
\pgfusepath{stroke,fill}%
}%
\begin{pgfscope}%
\pgfsys@transformshift{0.894063in}{2.690556in}%
\pgfsys@useobject{currentmarker}{}%
\end{pgfscope}%
\end{pgfscope}%
\begin{pgfscope}%
\pgftext[x=0.894063in,y=0.574444in,,top]{\sffamily\fontsize{12.000000}{14.400000}\selectfont 0}%
\end{pgfscope}%
\begin{pgfscope}%
\pgfsetbuttcap%
\pgfsetroundjoin%
\definecolor{currentfill}{rgb}{0.000000,0.000000,0.000000}%
\pgfsetfillcolor{currentfill}%
\pgfsetlinewidth{0.501875pt}%
\definecolor{currentstroke}{rgb}{0.000000,0.000000,0.000000}%
\pgfsetstrokecolor{currentstroke}%
\pgfsetdash{}{0pt}%
\pgfsys@defobject{currentmarker}{\pgfqpoint{0.000000in}{0.000000in}}{\pgfqpoint{0.000000in}{0.055556in}}{%
\pgfpathmoveto{\pgfqpoint{0.000000in}{0.000000in}}%
\pgfpathlineto{\pgfqpoint{0.000000in}{0.055556in}}%
\pgfusepath{stroke,fill}%
}%
\begin{pgfscope}%
\pgfsys@transformshift{2.236750in}{0.630000in}%
\pgfsys@useobject{currentmarker}{}%
\end{pgfscope}%
\end{pgfscope}%
\begin{pgfscope}%
\pgfsetbuttcap%
\pgfsetroundjoin%
\definecolor{currentfill}{rgb}{0.000000,0.000000,0.000000}%
\pgfsetfillcolor{currentfill}%
\pgfsetlinewidth{0.501875pt}%
\definecolor{currentstroke}{rgb}{0.000000,0.000000,0.000000}%
\pgfsetstrokecolor{currentstroke}%
\pgfsetdash{}{0pt}%
\pgfsys@defobject{currentmarker}{\pgfqpoint{0.000000in}{-0.055556in}}{\pgfqpoint{0.000000in}{0.000000in}}{%
\pgfpathmoveto{\pgfqpoint{0.000000in}{0.000000in}}%
\pgfpathlineto{\pgfqpoint{0.000000in}{-0.055556in}}%
\pgfusepath{stroke,fill}%
}%
\begin{pgfscope}%
\pgfsys@transformshift{2.236750in}{2.690556in}%
\pgfsys@useobject{currentmarker}{}%
\end{pgfscope}%
\end{pgfscope}%
\begin{pgfscope}%
\pgftext[x=2.236750in,y=0.574444in,,top]{\sffamily\fontsize{12.000000}{14.400000}\selectfont 1000}%
\end{pgfscope}%
\begin{pgfscope}%
\pgfsetbuttcap%
\pgfsetroundjoin%
\definecolor{currentfill}{rgb}{0.000000,0.000000,0.000000}%
\pgfsetfillcolor{currentfill}%
\pgfsetlinewidth{0.501875pt}%
\definecolor{currentstroke}{rgb}{0.000000,0.000000,0.000000}%
\pgfsetstrokecolor{currentstroke}%
\pgfsetdash{}{0pt}%
\pgfsys@defobject{currentmarker}{\pgfqpoint{0.000000in}{0.000000in}}{\pgfqpoint{0.000000in}{0.055556in}}{%
\pgfpathmoveto{\pgfqpoint{0.000000in}{0.000000in}}%
\pgfpathlineto{\pgfqpoint{0.000000in}{0.055556in}}%
\pgfusepath{stroke,fill}%
}%
\begin{pgfscope}%
\pgfsys@transformshift{3.579438in}{0.630000in}%
\pgfsys@useobject{currentmarker}{}%
\end{pgfscope}%
\end{pgfscope}%
\begin{pgfscope}%
\pgfsetbuttcap%
\pgfsetroundjoin%
\definecolor{currentfill}{rgb}{0.000000,0.000000,0.000000}%
\pgfsetfillcolor{currentfill}%
\pgfsetlinewidth{0.501875pt}%
\definecolor{currentstroke}{rgb}{0.000000,0.000000,0.000000}%
\pgfsetstrokecolor{currentstroke}%
\pgfsetdash{}{0pt}%
\pgfsys@defobject{currentmarker}{\pgfqpoint{0.000000in}{-0.055556in}}{\pgfqpoint{0.000000in}{0.000000in}}{%
\pgfpathmoveto{\pgfqpoint{0.000000in}{0.000000in}}%
\pgfpathlineto{\pgfqpoint{0.000000in}{-0.055556in}}%
\pgfusepath{stroke,fill}%
}%
\begin{pgfscope}%
\pgfsys@transformshift{3.579438in}{2.690556in}%
\pgfsys@useobject{currentmarker}{}%
\end{pgfscope}%
\end{pgfscope}%
\begin{pgfscope}%
\pgftext[x=3.579438in,y=0.574444in,,top]{\sffamily\fontsize{12.000000}{14.400000}\selectfont 2000}%
\end{pgfscope}%
\begin{pgfscope}%
\pgfsetbuttcap%
\pgfsetroundjoin%
\definecolor{currentfill}{rgb}{0.000000,0.000000,0.000000}%
\pgfsetfillcolor{currentfill}%
\pgfsetlinewidth{0.501875pt}%
\definecolor{currentstroke}{rgb}{0.000000,0.000000,0.000000}%
\pgfsetstrokecolor{currentstroke}%
\pgfsetdash{}{0pt}%
\pgfsys@defobject{currentmarker}{\pgfqpoint{0.000000in}{0.000000in}}{\pgfqpoint{0.000000in}{0.055556in}}{%
\pgfpathmoveto{\pgfqpoint{0.000000in}{0.000000in}}%
\pgfpathlineto{\pgfqpoint{0.000000in}{0.055556in}}%
\pgfusepath{stroke,fill}%
}%
\begin{pgfscope}%
\pgfsys@transformshift{4.922125in}{0.630000in}%
\pgfsys@useobject{currentmarker}{}%
\end{pgfscope}%
\end{pgfscope}%
\begin{pgfscope}%
\pgfsetbuttcap%
\pgfsetroundjoin%
\definecolor{currentfill}{rgb}{0.000000,0.000000,0.000000}%
\pgfsetfillcolor{currentfill}%
\pgfsetlinewidth{0.501875pt}%
\definecolor{currentstroke}{rgb}{0.000000,0.000000,0.000000}%
\pgfsetstrokecolor{currentstroke}%
\pgfsetdash{}{0pt}%
\pgfsys@defobject{currentmarker}{\pgfqpoint{0.000000in}{-0.055556in}}{\pgfqpoint{0.000000in}{0.000000in}}{%
\pgfpathmoveto{\pgfqpoint{0.000000in}{0.000000in}}%
\pgfpathlineto{\pgfqpoint{0.000000in}{-0.055556in}}%
\pgfusepath{stroke,fill}%
}%
\begin{pgfscope}%
\pgfsys@transformshift{4.922125in}{2.690556in}%
\pgfsys@useobject{currentmarker}{}%
\end{pgfscope}%
\end{pgfscope}%
\begin{pgfscope}%
\pgftext[x=4.922125in,y=0.574444in,,top]{\sffamily\fontsize{12.000000}{14.400000}\selectfont 3000}%
\end{pgfscope}%
\begin{pgfscope}%
\pgfsetbuttcap%
\pgfsetroundjoin%
\definecolor{currentfill}{rgb}{0.000000,0.000000,0.000000}%
\pgfsetfillcolor{currentfill}%
\pgfsetlinewidth{0.501875pt}%
\definecolor{currentstroke}{rgb}{0.000000,0.000000,0.000000}%
\pgfsetstrokecolor{currentstroke}%
\pgfsetdash{}{0pt}%
\pgfsys@defobject{currentmarker}{\pgfqpoint{0.000000in}{0.000000in}}{\pgfqpoint{0.000000in}{0.055556in}}{%
\pgfpathmoveto{\pgfqpoint{0.000000in}{0.000000in}}%
\pgfpathlineto{\pgfqpoint{0.000000in}{0.055556in}}%
\pgfusepath{stroke,fill}%
}%
\begin{pgfscope}%
\pgfsys@transformshift{6.264813in}{0.630000in}%
\pgfsys@useobject{currentmarker}{}%
\end{pgfscope}%
\end{pgfscope}%
\begin{pgfscope}%
\pgfsetbuttcap%
\pgfsetroundjoin%
\definecolor{currentfill}{rgb}{0.000000,0.000000,0.000000}%
\pgfsetfillcolor{currentfill}%
\pgfsetlinewidth{0.501875pt}%
\definecolor{currentstroke}{rgb}{0.000000,0.000000,0.000000}%
\pgfsetstrokecolor{currentstroke}%
\pgfsetdash{}{0pt}%
\pgfsys@defobject{currentmarker}{\pgfqpoint{0.000000in}{-0.055556in}}{\pgfqpoint{0.000000in}{0.000000in}}{%
\pgfpathmoveto{\pgfqpoint{0.000000in}{0.000000in}}%
\pgfpathlineto{\pgfqpoint{0.000000in}{-0.055556in}}%
\pgfusepath{stroke,fill}%
}%
\begin{pgfscope}%
\pgfsys@transformshift{6.264813in}{2.690556in}%
\pgfsys@useobject{currentmarker}{}%
\end{pgfscope}%
\end{pgfscope}%
\begin{pgfscope}%
\pgftext[x=6.264813in,y=0.574444in,,top]{\sffamily\fontsize{12.000000}{14.400000}\selectfont 4000}%
\end{pgfscope}%
\begin{pgfscope}%
\pgfsetbuttcap%
\pgfsetroundjoin%
\definecolor{currentfill}{rgb}{0.000000,0.000000,0.000000}%
\pgfsetfillcolor{currentfill}%
\pgfsetlinewidth{0.501875pt}%
\definecolor{currentstroke}{rgb}{0.000000,0.000000,0.000000}%
\pgfsetstrokecolor{currentstroke}%
\pgfsetdash{}{0pt}%
\pgfsys@defobject{currentmarker}{\pgfqpoint{0.000000in}{0.000000in}}{\pgfqpoint{0.000000in}{0.055556in}}{%
\pgfpathmoveto{\pgfqpoint{0.000000in}{0.000000in}}%
\pgfpathlineto{\pgfqpoint{0.000000in}{0.055556in}}%
\pgfusepath{stroke,fill}%
}%
\begin{pgfscope}%
\pgfsys@transformshift{7.607500in}{0.630000in}%
\pgfsys@useobject{currentmarker}{}%
\end{pgfscope}%
\end{pgfscope}%
\begin{pgfscope}%
\pgfsetbuttcap%
\pgfsetroundjoin%
\definecolor{currentfill}{rgb}{0.000000,0.000000,0.000000}%
\pgfsetfillcolor{currentfill}%
\pgfsetlinewidth{0.501875pt}%
\definecolor{currentstroke}{rgb}{0.000000,0.000000,0.000000}%
\pgfsetstrokecolor{currentstroke}%
\pgfsetdash{}{0pt}%
\pgfsys@defobject{currentmarker}{\pgfqpoint{0.000000in}{-0.055556in}}{\pgfqpoint{0.000000in}{0.000000in}}{%
\pgfpathmoveto{\pgfqpoint{0.000000in}{0.000000in}}%
\pgfpathlineto{\pgfqpoint{0.000000in}{-0.055556in}}%
\pgfusepath{stroke,fill}%
}%
\begin{pgfscope}%
\pgfsys@transformshift{7.607500in}{2.690556in}%
\pgfsys@useobject{currentmarker}{}%
\end{pgfscope}%
\end{pgfscope}%
\begin{pgfscope}%
\pgftext[x=7.607500in,y=0.574444in,,top]{\sffamily\fontsize{12.000000}{14.400000}\selectfont 5000}%
\end{pgfscope}%
\begin{pgfscope}%
\pgftext[x=4.250781in,y=0.343705in,,top]{\sffamily\fontsize{12.000000}{14.400000}\selectfont Requests Served}%
\end{pgfscope}%
\begin{pgfscope}%
\pgfsetbuttcap%
\pgfsetroundjoin%
\definecolor{currentfill}{rgb}{0.000000,0.000000,0.000000}%
\pgfsetfillcolor{currentfill}%
\pgfsetlinewidth{0.501875pt}%
\definecolor{currentstroke}{rgb}{0.000000,0.000000,0.000000}%
\pgfsetstrokecolor{currentstroke}%
\pgfsetdash{}{0pt}%
\pgfsys@defobject{currentmarker}{\pgfqpoint{0.000000in}{0.000000in}}{\pgfqpoint{0.055556in}{0.000000in}}{%
\pgfpathmoveto{\pgfqpoint{0.000000in}{0.000000in}}%
\pgfpathlineto{\pgfqpoint{0.055556in}{0.000000in}}%
\pgfusepath{stroke,fill}%
}%
\begin{pgfscope}%
\pgfsys@transformshift{0.894063in}{0.630000in}%
\pgfsys@useobject{currentmarker}{}%
\end{pgfscope}%
\end{pgfscope}%
\begin{pgfscope}%
\pgfsetbuttcap%
\pgfsetroundjoin%
\definecolor{currentfill}{rgb}{0.000000,0.000000,0.000000}%
\pgfsetfillcolor{currentfill}%
\pgfsetlinewidth{0.501875pt}%
\definecolor{currentstroke}{rgb}{0.000000,0.000000,0.000000}%
\pgfsetstrokecolor{currentstroke}%
\pgfsetdash{}{0pt}%
\pgfsys@defobject{currentmarker}{\pgfqpoint{-0.055556in}{0.000000in}}{\pgfqpoint{0.000000in}{0.000000in}}{%
\pgfpathmoveto{\pgfqpoint{0.000000in}{0.000000in}}%
\pgfpathlineto{\pgfqpoint{-0.055556in}{0.000000in}}%
\pgfusepath{stroke,fill}%
}%
\begin{pgfscope}%
\pgfsys@transformshift{7.607500in}{0.630000in}%
\pgfsys@useobject{currentmarker}{}%
\end{pgfscope}%
\end{pgfscope}%
\begin{pgfscope}%
\pgftext[x=0.838507in,y=0.630000in,right,]{\sffamily\fontsize{12.000000}{14.400000}\selectfont 0}%
\end{pgfscope}%
\begin{pgfscope}%
\pgfsetbuttcap%
\pgfsetroundjoin%
\definecolor{currentfill}{rgb}{0.000000,0.000000,0.000000}%
\pgfsetfillcolor{currentfill}%
\pgfsetlinewidth{0.501875pt}%
\definecolor{currentstroke}{rgb}{0.000000,0.000000,0.000000}%
\pgfsetstrokecolor{currentstroke}%
\pgfsetdash{}{0pt}%
\pgfsys@defobject{currentmarker}{\pgfqpoint{0.000000in}{0.000000in}}{\pgfqpoint{0.055556in}{0.000000in}}{%
\pgfpathmoveto{\pgfqpoint{0.000000in}{0.000000in}}%
\pgfpathlineto{\pgfqpoint{0.055556in}{0.000000in}}%
\pgfusepath{stroke,fill}%
}%
\begin{pgfscope}%
\pgfsys@transformshift{0.894063in}{0.924365in}%
\pgfsys@useobject{currentmarker}{}%
\end{pgfscope}%
\end{pgfscope}%
\begin{pgfscope}%
\pgfsetbuttcap%
\pgfsetroundjoin%
\definecolor{currentfill}{rgb}{0.000000,0.000000,0.000000}%
\pgfsetfillcolor{currentfill}%
\pgfsetlinewidth{0.501875pt}%
\definecolor{currentstroke}{rgb}{0.000000,0.000000,0.000000}%
\pgfsetstrokecolor{currentstroke}%
\pgfsetdash{}{0pt}%
\pgfsys@defobject{currentmarker}{\pgfqpoint{-0.055556in}{0.000000in}}{\pgfqpoint{0.000000in}{0.000000in}}{%
\pgfpathmoveto{\pgfqpoint{0.000000in}{0.000000in}}%
\pgfpathlineto{\pgfqpoint{-0.055556in}{0.000000in}}%
\pgfusepath{stroke,fill}%
}%
\begin{pgfscope}%
\pgfsys@transformshift{7.607500in}{0.924365in}%
\pgfsys@useobject{currentmarker}{}%
\end{pgfscope}%
\end{pgfscope}%
\begin{pgfscope}%
\pgftext[x=0.838507in,y=0.924365in,right,]{\sffamily\fontsize{12.000000}{14.400000}\selectfont 200}%
\end{pgfscope}%
\begin{pgfscope}%
\pgfsetbuttcap%
\pgfsetroundjoin%
\definecolor{currentfill}{rgb}{0.000000,0.000000,0.000000}%
\pgfsetfillcolor{currentfill}%
\pgfsetlinewidth{0.501875pt}%
\definecolor{currentstroke}{rgb}{0.000000,0.000000,0.000000}%
\pgfsetstrokecolor{currentstroke}%
\pgfsetdash{}{0pt}%
\pgfsys@defobject{currentmarker}{\pgfqpoint{0.000000in}{0.000000in}}{\pgfqpoint{0.055556in}{0.000000in}}{%
\pgfpathmoveto{\pgfqpoint{0.000000in}{0.000000in}}%
\pgfpathlineto{\pgfqpoint{0.055556in}{0.000000in}}%
\pgfusepath{stroke,fill}%
}%
\begin{pgfscope}%
\pgfsys@transformshift{0.894063in}{1.218730in}%
\pgfsys@useobject{currentmarker}{}%
\end{pgfscope}%
\end{pgfscope}%
\begin{pgfscope}%
\pgfsetbuttcap%
\pgfsetroundjoin%
\definecolor{currentfill}{rgb}{0.000000,0.000000,0.000000}%
\pgfsetfillcolor{currentfill}%
\pgfsetlinewidth{0.501875pt}%
\definecolor{currentstroke}{rgb}{0.000000,0.000000,0.000000}%
\pgfsetstrokecolor{currentstroke}%
\pgfsetdash{}{0pt}%
\pgfsys@defobject{currentmarker}{\pgfqpoint{-0.055556in}{0.000000in}}{\pgfqpoint{0.000000in}{0.000000in}}{%
\pgfpathmoveto{\pgfqpoint{0.000000in}{0.000000in}}%
\pgfpathlineto{\pgfqpoint{-0.055556in}{0.000000in}}%
\pgfusepath{stroke,fill}%
}%
\begin{pgfscope}%
\pgfsys@transformshift{7.607500in}{1.218730in}%
\pgfsys@useobject{currentmarker}{}%
\end{pgfscope}%
\end{pgfscope}%
\begin{pgfscope}%
\pgftext[x=0.838507in,y=1.218730in,right,]{\sffamily\fontsize{12.000000}{14.400000}\selectfont 400}%
\end{pgfscope}%
\begin{pgfscope}%
\pgfsetbuttcap%
\pgfsetroundjoin%
\definecolor{currentfill}{rgb}{0.000000,0.000000,0.000000}%
\pgfsetfillcolor{currentfill}%
\pgfsetlinewidth{0.501875pt}%
\definecolor{currentstroke}{rgb}{0.000000,0.000000,0.000000}%
\pgfsetstrokecolor{currentstroke}%
\pgfsetdash{}{0pt}%
\pgfsys@defobject{currentmarker}{\pgfqpoint{0.000000in}{0.000000in}}{\pgfqpoint{0.055556in}{0.000000in}}{%
\pgfpathmoveto{\pgfqpoint{0.000000in}{0.000000in}}%
\pgfpathlineto{\pgfqpoint{0.055556in}{0.000000in}}%
\pgfusepath{stroke,fill}%
}%
\begin{pgfscope}%
\pgfsys@transformshift{0.894063in}{1.513095in}%
\pgfsys@useobject{currentmarker}{}%
\end{pgfscope}%
\end{pgfscope}%
\begin{pgfscope}%
\pgfsetbuttcap%
\pgfsetroundjoin%
\definecolor{currentfill}{rgb}{0.000000,0.000000,0.000000}%
\pgfsetfillcolor{currentfill}%
\pgfsetlinewidth{0.501875pt}%
\definecolor{currentstroke}{rgb}{0.000000,0.000000,0.000000}%
\pgfsetstrokecolor{currentstroke}%
\pgfsetdash{}{0pt}%
\pgfsys@defobject{currentmarker}{\pgfqpoint{-0.055556in}{0.000000in}}{\pgfqpoint{0.000000in}{0.000000in}}{%
\pgfpathmoveto{\pgfqpoint{0.000000in}{0.000000in}}%
\pgfpathlineto{\pgfqpoint{-0.055556in}{0.000000in}}%
\pgfusepath{stroke,fill}%
}%
\begin{pgfscope}%
\pgfsys@transformshift{7.607500in}{1.513095in}%
\pgfsys@useobject{currentmarker}{}%
\end{pgfscope}%
\end{pgfscope}%
\begin{pgfscope}%
\pgftext[x=0.838507in,y=1.513095in,right,]{\sffamily\fontsize{12.000000}{14.400000}\selectfont 600}%
\end{pgfscope}%
\begin{pgfscope}%
\pgfsetbuttcap%
\pgfsetroundjoin%
\definecolor{currentfill}{rgb}{0.000000,0.000000,0.000000}%
\pgfsetfillcolor{currentfill}%
\pgfsetlinewidth{0.501875pt}%
\definecolor{currentstroke}{rgb}{0.000000,0.000000,0.000000}%
\pgfsetstrokecolor{currentstroke}%
\pgfsetdash{}{0pt}%
\pgfsys@defobject{currentmarker}{\pgfqpoint{0.000000in}{0.000000in}}{\pgfqpoint{0.055556in}{0.000000in}}{%
\pgfpathmoveto{\pgfqpoint{0.000000in}{0.000000in}}%
\pgfpathlineto{\pgfqpoint{0.055556in}{0.000000in}}%
\pgfusepath{stroke,fill}%
}%
\begin{pgfscope}%
\pgfsys@transformshift{0.894063in}{1.807460in}%
\pgfsys@useobject{currentmarker}{}%
\end{pgfscope}%
\end{pgfscope}%
\begin{pgfscope}%
\pgfsetbuttcap%
\pgfsetroundjoin%
\definecolor{currentfill}{rgb}{0.000000,0.000000,0.000000}%
\pgfsetfillcolor{currentfill}%
\pgfsetlinewidth{0.501875pt}%
\definecolor{currentstroke}{rgb}{0.000000,0.000000,0.000000}%
\pgfsetstrokecolor{currentstroke}%
\pgfsetdash{}{0pt}%
\pgfsys@defobject{currentmarker}{\pgfqpoint{-0.055556in}{0.000000in}}{\pgfqpoint{0.000000in}{0.000000in}}{%
\pgfpathmoveto{\pgfqpoint{0.000000in}{0.000000in}}%
\pgfpathlineto{\pgfqpoint{-0.055556in}{0.000000in}}%
\pgfusepath{stroke,fill}%
}%
\begin{pgfscope}%
\pgfsys@transformshift{7.607500in}{1.807460in}%
\pgfsys@useobject{currentmarker}{}%
\end{pgfscope}%
\end{pgfscope}%
\begin{pgfscope}%
\pgftext[x=0.838507in,y=1.807460in,right,]{\sffamily\fontsize{12.000000}{14.400000}\selectfont 800}%
\end{pgfscope}%
\begin{pgfscope}%
\pgfsetbuttcap%
\pgfsetroundjoin%
\definecolor{currentfill}{rgb}{0.000000,0.000000,0.000000}%
\pgfsetfillcolor{currentfill}%
\pgfsetlinewidth{0.501875pt}%
\definecolor{currentstroke}{rgb}{0.000000,0.000000,0.000000}%
\pgfsetstrokecolor{currentstroke}%
\pgfsetdash{}{0pt}%
\pgfsys@defobject{currentmarker}{\pgfqpoint{0.000000in}{0.000000in}}{\pgfqpoint{0.055556in}{0.000000in}}{%
\pgfpathmoveto{\pgfqpoint{0.000000in}{0.000000in}}%
\pgfpathlineto{\pgfqpoint{0.055556in}{0.000000in}}%
\pgfusepath{stroke,fill}%
}%
\begin{pgfscope}%
\pgfsys@transformshift{0.894063in}{2.101825in}%
\pgfsys@useobject{currentmarker}{}%
\end{pgfscope}%
\end{pgfscope}%
\begin{pgfscope}%
\pgfsetbuttcap%
\pgfsetroundjoin%
\definecolor{currentfill}{rgb}{0.000000,0.000000,0.000000}%
\pgfsetfillcolor{currentfill}%
\pgfsetlinewidth{0.501875pt}%
\definecolor{currentstroke}{rgb}{0.000000,0.000000,0.000000}%
\pgfsetstrokecolor{currentstroke}%
\pgfsetdash{}{0pt}%
\pgfsys@defobject{currentmarker}{\pgfqpoint{-0.055556in}{0.000000in}}{\pgfqpoint{0.000000in}{0.000000in}}{%
\pgfpathmoveto{\pgfqpoint{0.000000in}{0.000000in}}%
\pgfpathlineto{\pgfqpoint{-0.055556in}{0.000000in}}%
\pgfusepath{stroke,fill}%
}%
\begin{pgfscope}%
\pgfsys@transformshift{7.607500in}{2.101825in}%
\pgfsys@useobject{currentmarker}{}%
\end{pgfscope}%
\end{pgfscope}%
\begin{pgfscope}%
\pgftext[x=0.838507in,y=2.101825in,right,]{\sffamily\fontsize{12.000000}{14.400000}\selectfont 1000}%
\end{pgfscope}%
\begin{pgfscope}%
\pgfsetbuttcap%
\pgfsetroundjoin%
\definecolor{currentfill}{rgb}{0.000000,0.000000,0.000000}%
\pgfsetfillcolor{currentfill}%
\pgfsetlinewidth{0.501875pt}%
\definecolor{currentstroke}{rgb}{0.000000,0.000000,0.000000}%
\pgfsetstrokecolor{currentstroke}%
\pgfsetdash{}{0pt}%
\pgfsys@defobject{currentmarker}{\pgfqpoint{0.000000in}{0.000000in}}{\pgfqpoint{0.055556in}{0.000000in}}{%
\pgfpathmoveto{\pgfqpoint{0.000000in}{0.000000in}}%
\pgfpathlineto{\pgfqpoint{0.055556in}{0.000000in}}%
\pgfusepath{stroke,fill}%
}%
\begin{pgfscope}%
\pgfsys@transformshift{0.894063in}{2.396190in}%
\pgfsys@useobject{currentmarker}{}%
\end{pgfscope}%
\end{pgfscope}%
\begin{pgfscope}%
\pgfsetbuttcap%
\pgfsetroundjoin%
\definecolor{currentfill}{rgb}{0.000000,0.000000,0.000000}%
\pgfsetfillcolor{currentfill}%
\pgfsetlinewidth{0.501875pt}%
\definecolor{currentstroke}{rgb}{0.000000,0.000000,0.000000}%
\pgfsetstrokecolor{currentstroke}%
\pgfsetdash{}{0pt}%
\pgfsys@defobject{currentmarker}{\pgfqpoint{-0.055556in}{0.000000in}}{\pgfqpoint{0.000000in}{0.000000in}}{%
\pgfpathmoveto{\pgfqpoint{0.000000in}{0.000000in}}%
\pgfpathlineto{\pgfqpoint{-0.055556in}{0.000000in}}%
\pgfusepath{stroke,fill}%
}%
\begin{pgfscope}%
\pgfsys@transformshift{7.607500in}{2.396190in}%
\pgfsys@useobject{currentmarker}{}%
\end{pgfscope}%
\end{pgfscope}%
\begin{pgfscope}%
\pgftext[x=0.838507in,y=2.396190in,right,]{\sffamily\fontsize{12.000000}{14.400000}\selectfont 1200}%
\end{pgfscope}%
\begin{pgfscope}%
\pgfsetbuttcap%
\pgfsetroundjoin%
\definecolor{currentfill}{rgb}{0.000000,0.000000,0.000000}%
\pgfsetfillcolor{currentfill}%
\pgfsetlinewidth{0.501875pt}%
\definecolor{currentstroke}{rgb}{0.000000,0.000000,0.000000}%
\pgfsetstrokecolor{currentstroke}%
\pgfsetdash{}{0pt}%
\pgfsys@defobject{currentmarker}{\pgfqpoint{0.000000in}{0.000000in}}{\pgfqpoint{0.055556in}{0.000000in}}{%
\pgfpathmoveto{\pgfqpoint{0.000000in}{0.000000in}}%
\pgfpathlineto{\pgfqpoint{0.055556in}{0.000000in}}%
\pgfusepath{stroke,fill}%
}%
\begin{pgfscope}%
\pgfsys@transformshift{0.894063in}{2.690556in}%
\pgfsys@useobject{currentmarker}{}%
\end{pgfscope}%
\end{pgfscope}%
\begin{pgfscope}%
\pgfsetbuttcap%
\pgfsetroundjoin%
\definecolor{currentfill}{rgb}{0.000000,0.000000,0.000000}%
\pgfsetfillcolor{currentfill}%
\pgfsetlinewidth{0.501875pt}%
\definecolor{currentstroke}{rgb}{0.000000,0.000000,0.000000}%
\pgfsetstrokecolor{currentstroke}%
\pgfsetdash{}{0pt}%
\pgfsys@defobject{currentmarker}{\pgfqpoint{-0.055556in}{0.000000in}}{\pgfqpoint{0.000000in}{0.000000in}}{%
\pgfpathmoveto{\pgfqpoint{0.000000in}{0.000000in}}%
\pgfpathlineto{\pgfqpoint{-0.055556in}{0.000000in}}%
\pgfusepath{stroke,fill}%
}%
\begin{pgfscope}%
\pgfsys@transformshift{7.607500in}{2.690556in}%
\pgfsys@useobject{currentmarker}{}%
\end{pgfscope}%
\end{pgfscope}%
\begin{pgfscope}%
\pgftext[x=0.838507in,y=2.690556in,right,]{\sffamily\fontsize{12.000000}{14.400000}\selectfont 1400}%
\end{pgfscope}%
\begin{pgfscope}%
\pgftext[x=0.344909in,y=1.660278in,,bottom,rotate=90.000000]{\sffamily\fontsize{12.000000}{14.400000}\selectfont Memory (Mb)}%
\end{pgfscope}%
\begin{pgfscope}%
\pgftext[x=4.250781in,y=2.760000in,,base]{\sffamily\fontsize{14.400000}{17.280000}\selectfont Nonshared memory}%
\end{pgfscope}%
\end{pgfpicture}%
\makeatother%
\endgroup%
}
    \caption{Memory usage of patched Garbage Collector running 4 workers after $X$ requests.  Each dot represents a worker's memory consumption after $X$ requests.}\label{fig:patched_mem}
\end{figure}

\begin{figure}
    \begin{center}\textbf{Memory Usage of Standard GC}\end{center}
    \resizebox{0.5\textwidth}{!}{%% Creator: Matplotlib, PGF backend
%%
%% To include the figure in your LaTeX document, write
%%   \input{<filename>.pgf}
%%
%% Make sure the required packages are loaded in your preamble
%%   \usepackage{pgf}
%%
%% Figures using additional raster images can only be included by \input if
%% they are in the same directory as the main LaTeX file. For loading figures
%% from other directories you can use the `import` package
%%   \usepackage{import}
%% and then include the figures with
%%   \import{<path to file>}{<filename>.pgf}
%%
%% Matplotlib used the following preamble
%%   \usepackage{fontspec}
%%   \setmainfont{Bitstream Vera Serif}
%%   \setsansfont{Bitstream Vera Sans}
%%   \setmonofont{Bitstream Vera Sans Mono}
%%
\begingroup%
\makeatletter%
\begin{pgfpicture}%
\pgfpathrectangle{\pgfpointorigin}{\pgfqpoint{8.000000in}{6.000000in}}%
\pgfusepath{use as bounding box, clip}%
\begin{pgfscope}%
\pgfsetbuttcap%
\pgfsetmiterjoin%
\definecolor{currentfill}{rgb}{1.000000,1.000000,1.000000}%
\pgfsetfillcolor{currentfill}%
\pgfsetlinewidth{0.000000pt}%
\definecolor{currentstroke}{rgb}{1.000000,1.000000,1.000000}%
\pgfsetstrokecolor{currentstroke}%
\pgfsetdash{}{0pt}%
\pgfpathmoveto{\pgfqpoint{0.000000in}{0.000000in}}%
\pgfpathlineto{\pgfqpoint{8.000000in}{0.000000in}}%
\pgfpathlineto{\pgfqpoint{8.000000in}{6.000000in}}%
\pgfpathlineto{\pgfqpoint{0.000000in}{6.000000in}}%
\pgfpathclose%
\pgfusepath{fill}%
\end{pgfscope}%
\begin{pgfscope}%
\pgfsetbuttcap%
\pgfsetmiterjoin%
\definecolor{currentfill}{rgb}{1.000000,1.000000,1.000000}%
\pgfsetfillcolor{currentfill}%
\pgfsetlinewidth{0.000000pt}%
\definecolor{currentstroke}{rgb}{0.000000,0.000000,0.000000}%
\pgfsetstrokecolor{currentstroke}%
\pgfsetstrokeopacity{0.000000}%
\pgfsetdash{}{0pt}%
\pgfpathmoveto{\pgfqpoint{0.894063in}{3.540000in}}%
\pgfpathlineto{\pgfqpoint{7.607500in}{3.540000in}}%
\pgfpathlineto{\pgfqpoint{7.607500in}{5.600556in}}%
\pgfpathlineto{\pgfqpoint{0.894063in}{5.600556in}}%
\pgfpathclose%
\pgfusepath{fill}%
\end{pgfscope}%
\begin{pgfscope}%
\pgfpathrectangle{\pgfqpoint{0.894063in}{3.540000in}}{\pgfqpoint{6.713438in}{2.060556in}} %
\pgfusepath{clip}%
\pgfsetbuttcap%
\pgfsetroundjoin%
\definecolor{currentfill}{rgb}{0.000000,0.000000,1.000000}%
\pgfsetfillcolor{currentfill}%
\pgfsetlinewidth{1.003750pt}%
\definecolor{currentstroke}{rgb}{0.000000,0.000000,0.000000}%
\pgfsetstrokecolor{currentstroke}%
\pgfsetdash{}{0pt}%
\pgfpathmoveto{\pgfqpoint{6.667619in}{4.862362in}}%
\pgfpathcurveto{\pgfqpoint{6.675855in}{4.862362in}}{\pgfqpoint{6.683755in}{4.865634in}}{\pgfqpoint{6.689579in}{4.871458in}}%
\pgfpathcurveto{\pgfqpoint{6.695403in}{4.877282in}}{\pgfqpoint{6.698675in}{4.885182in}}{\pgfqpoint{6.698675in}{4.893418in}}%
\pgfpathcurveto{\pgfqpoint{6.698675in}{4.901655in}}{\pgfqpoint{6.695403in}{4.909555in}}{\pgfqpoint{6.689579in}{4.915378in}}%
\pgfpathcurveto{\pgfqpoint{6.683755in}{4.921202in}}{\pgfqpoint{6.675855in}{4.924475in}}{\pgfqpoint{6.667619in}{4.924475in}}%
\pgfpathcurveto{\pgfqpoint{6.659382in}{4.924475in}}{\pgfqpoint{6.651482in}{4.921202in}}{\pgfqpoint{6.645658in}{4.915378in}}%
\pgfpathcurveto{\pgfqpoint{6.639835in}{4.909555in}}{\pgfqpoint{6.636562in}{4.901655in}}{\pgfqpoint{6.636562in}{4.893418in}}%
\pgfpathcurveto{\pgfqpoint{6.636562in}{4.885182in}}{\pgfqpoint{6.639835in}{4.877282in}}{\pgfqpoint{6.645658in}{4.871458in}}%
\pgfpathcurveto{\pgfqpoint{6.651482in}{4.865634in}}{\pgfqpoint{6.659382in}{4.862362in}}{\pgfqpoint{6.667619in}{4.862362in}}%
\pgfpathclose%
\pgfusepath{stroke,fill}%
\end{pgfscope}%
\begin{pgfscope}%
\pgfpathrectangle{\pgfqpoint{0.894063in}{3.540000in}}{\pgfqpoint{6.713438in}{2.060556in}} %
\pgfusepath{clip}%
\pgfsetbuttcap%
\pgfsetroundjoin%
\definecolor{currentfill}{rgb}{0.000000,0.000000,1.000000}%
\pgfsetfillcolor{currentfill}%
\pgfsetlinewidth{1.003750pt}%
\definecolor{currentstroke}{rgb}{0.000000,0.000000,0.000000}%
\pgfsetstrokecolor{currentstroke}%
\pgfsetdash{}{0pt}%
\pgfpathmoveto{\pgfqpoint{2.639556in}{5.126980in}}%
\pgfpathcurveto{\pgfqpoint{2.647793in}{5.126980in}}{\pgfqpoint{2.655693in}{5.130252in}}{\pgfqpoint{2.661517in}{5.136076in}}%
\pgfpathcurveto{\pgfqpoint{2.667340in}{5.141900in}}{\pgfqpoint{2.670613in}{5.149800in}}{\pgfqpoint{2.670613in}{5.158036in}}%
\pgfpathcurveto{\pgfqpoint{2.670613in}{5.166272in}}{\pgfqpoint{2.667340in}{5.174172in}}{\pgfqpoint{2.661517in}{5.179996in}}%
\pgfpathcurveto{\pgfqpoint{2.655693in}{5.185820in}}{\pgfqpoint{2.647793in}{5.189093in}}{\pgfqpoint{2.639556in}{5.189093in}}%
\pgfpathcurveto{\pgfqpoint{2.631320in}{5.189093in}}{\pgfqpoint{2.623420in}{5.185820in}}{\pgfqpoint{2.617596in}{5.179996in}}%
\pgfpathcurveto{\pgfqpoint{2.611772in}{5.174172in}}{\pgfqpoint{2.608500in}{5.166272in}}{\pgfqpoint{2.608500in}{5.158036in}}%
\pgfpathcurveto{\pgfqpoint{2.608500in}{5.149800in}}{\pgfqpoint{2.611772in}{5.141900in}}{\pgfqpoint{2.617596in}{5.136076in}}%
\pgfpathcurveto{\pgfqpoint{2.623420in}{5.130252in}}{\pgfqpoint{2.631320in}{5.126980in}}{\pgfqpoint{2.639556in}{5.126980in}}%
\pgfpathclose%
\pgfusepath{stroke,fill}%
\end{pgfscope}%
\begin{pgfscope}%
\pgfpathrectangle{\pgfqpoint{0.894063in}{3.540000in}}{\pgfqpoint{6.713438in}{2.060556in}} %
\pgfusepath{clip}%
\pgfsetbuttcap%
\pgfsetroundjoin%
\definecolor{currentfill}{rgb}{0.000000,0.000000,1.000000}%
\pgfsetfillcolor{currentfill}%
\pgfsetlinewidth{1.003750pt}%
\definecolor{currentstroke}{rgb}{0.000000,0.000000,0.000000}%
\pgfsetstrokecolor{currentstroke}%
\pgfsetdash{}{0pt}%
\pgfpathmoveto{\pgfqpoint{1.699675in}{5.132491in}}%
\pgfpathcurveto{\pgfqpoint{1.707911in}{5.132491in}}{\pgfqpoint{1.715811in}{5.135763in}}{\pgfqpoint{1.721635in}{5.141587in}}%
\pgfpathcurveto{\pgfqpoint{1.727459in}{5.147411in}}{\pgfqpoint{1.730731in}{5.155311in}}{\pgfqpoint{1.730731in}{5.163547in}}%
\pgfpathcurveto{\pgfqpoint{1.730731in}{5.171784in}}{\pgfqpoint{1.727459in}{5.179684in}}{\pgfqpoint{1.721635in}{5.185508in}}%
\pgfpathcurveto{\pgfqpoint{1.715811in}{5.191332in}}{\pgfqpoint{1.707911in}{5.194604in}}{\pgfqpoint{1.699675in}{5.194604in}}%
\pgfpathcurveto{\pgfqpoint{1.691439in}{5.194604in}}{\pgfqpoint{1.683539in}{5.191332in}}{\pgfqpoint{1.677715in}{5.185508in}}%
\pgfpathcurveto{\pgfqpoint{1.671891in}{5.179684in}}{\pgfqpoint{1.668619in}{5.171784in}}{\pgfqpoint{1.668619in}{5.163547in}}%
\pgfpathcurveto{\pgfqpoint{1.668619in}{5.155311in}}{\pgfqpoint{1.671891in}{5.147411in}}{\pgfqpoint{1.677715in}{5.141587in}}%
\pgfpathcurveto{\pgfqpoint{1.683539in}{5.135763in}}{\pgfqpoint{1.691439in}{5.132491in}}{\pgfqpoint{1.699675in}{5.132491in}}%
\pgfpathclose%
\pgfusepath{stroke,fill}%
\end{pgfscope}%
\begin{pgfscope}%
\pgfpathrectangle{\pgfqpoint{0.894063in}{3.540000in}}{\pgfqpoint{6.713438in}{2.060556in}} %
\pgfusepath{clip}%
\pgfsetbuttcap%
\pgfsetroundjoin%
\definecolor{currentfill}{rgb}{0.000000,0.000000,1.000000}%
\pgfsetfillcolor{currentfill}%
\pgfsetlinewidth{1.003750pt}%
\definecolor{currentstroke}{rgb}{0.000000,0.000000,0.000000}%
\pgfsetstrokecolor{currentstroke}%
\pgfsetdash{}{0pt}%
\pgfpathmoveto{\pgfqpoint{1.162600in}{5.264666in}}%
\pgfpathcurveto{\pgfqpoint{1.170836in}{5.264666in}}{\pgfqpoint{1.178736in}{5.267938in}}{\pgfqpoint{1.184560in}{5.273762in}}%
\pgfpathcurveto{\pgfqpoint{1.190384in}{5.279586in}}{\pgfqpoint{1.193656in}{5.287486in}}{\pgfqpoint{1.193656in}{5.295722in}}%
\pgfpathcurveto{\pgfqpoint{1.193656in}{5.303959in}}{\pgfqpoint{1.190384in}{5.311859in}}{\pgfqpoint{1.184560in}{5.317683in}}%
\pgfpathcurveto{\pgfqpoint{1.178736in}{5.323507in}}{\pgfqpoint{1.170836in}{5.326779in}}{\pgfqpoint{1.162600in}{5.326779in}}%
\pgfpathcurveto{\pgfqpoint{1.154364in}{5.326779in}}{\pgfqpoint{1.146464in}{5.323507in}}{\pgfqpoint{1.140640in}{5.317683in}}%
\pgfpathcurveto{\pgfqpoint{1.134816in}{5.311859in}}{\pgfqpoint{1.131544in}{5.303959in}}{\pgfqpoint{1.131544in}{5.295722in}}%
\pgfpathcurveto{\pgfqpoint{1.131544in}{5.287486in}}{\pgfqpoint{1.134816in}{5.279586in}}{\pgfqpoint{1.140640in}{5.273762in}}%
\pgfpathcurveto{\pgfqpoint{1.146464in}{5.267938in}}{\pgfqpoint{1.154364in}{5.264666in}}{\pgfqpoint{1.162600in}{5.264666in}}%
\pgfpathclose%
\pgfusepath{stroke,fill}%
\end{pgfscope}%
\begin{pgfscope}%
\pgfpathrectangle{\pgfqpoint{0.894063in}{3.540000in}}{\pgfqpoint{6.713438in}{2.060556in}} %
\pgfusepath{clip}%
\pgfsetbuttcap%
\pgfsetroundjoin%
\definecolor{currentfill}{rgb}{0.000000,0.000000,1.000000}%
\pgfsetfillcolor{currentfill}%
\pgfsetlinewidth{1.003750pt}%
\definecolor{currentstroke}{rgb}{0.000000,0.000000,0.000000}%
\pgfsetstrokecolor{currentstroke}%
\pgfsetdash{}{0pt}%
\pgfpathmoveto{\pgfqpoint{1.833944in}{5.130395in}}%
\pgfpathcurveto{\pgfqpoint{1.842180in}{5.130395in}}{\pgfqpoint{1.850080in}{5.133667in}}{\pgfqpoint{1.855904in}{5.139491in}}%
\pgfpathcurveto{\pgfqpoint{1.861728in}{5.145315in}}{\pgfqpoint{1.865000in}{5.153215in}}{\pgfqpoint{1.865000in}{5.161451in}}%
\pgfpathcurveto{\pgfqpoint{1.865000in}{5.169687in}}{\pgfqpoint{1.861728in}{5.177587in}}{\pgfqpoint{1.855904in}{5.183411in}}%
\pgfpathcurveto{\pgfqpoint{1.850080in}{5.189235in}}{\pgfqpoint{1.842180in}{5.192508in}}{\pgfqpoint{1.833944in}{5.192508in}}%
\pgfpathcurveto{\pgfqpoint{1.825707in}{5.192508in}}{\pgfqpoint{1.817807in}{5.189235in}}{\pgfqpoint{1.811983in}{5.183411in}}%
\pgfpathcurveto{\pgfqpoint{1.806160in}{5.177587in}}{\pgfqpoint{1.802887in}{5.169687in}}{\pgfqpoint{1.802887in}{5.161451in}}%
\pgfpathcurveto{\pgfqpoint{1.802887in}{5.153215in}}{\pgfqpoint{1.806160in}{5.145315in}}{\pgfqpoint{1.811983in}{5.139491in}}%
\pgfpathcurveto{\pgfqpoint{1.817807in}{5.133667in}}{\pgfqpoint{1.825707in}{5.130395in}}{\pgfqpoint{1.833944in}{5.130395in}}%
\pgfpathclose%
\pgfusepath{stroke,fill}%
\end{pgfscope}%
\begin{pgfscope}%
\pgfpathrectangle{\pgfqpoint{0.894063in}{3.540000in}}{\pgfqpoint{6.713438in}{2.060556in}} %
\pgfusepath{clip}%
\pgfsetbuttcap%
\pgfsetroundjoin%
\definecolor{currentfill}{rgb}{0.000000,0.000000,1.000000}%
\pgfsetfillcolor{currentfill}%
\pgfsetlinewidth{1.003750pt}%
\definecolor{currentstroke}{rgb}{0.000000,0.000000,0.000000}%
\pgfsetstrokecolor{currentstroke}%
\pgfsetdash{}{0pt}%
\pgfpathmoveto{\pgfqpoint{5.996275in}{4.978040in}}%
\pgfpathcurveto{\pgfqpoint{6.004511in}{4.978040in}}{\pgfqpoint{6.012411in}{4.981312in}}{\pgfqpoint{6.018235in}{4.987136in}}%
\pgfpathcurveto{\pgfqpoint{6.024059in}{4.992960in}}{\pgfqpoint{6.027331in}{5.000860in}}{\pgfqpoint{6.027331in}{5.009096in}}%
\pgfpathcurveto{\pgfqpoint{6.027331in}{5.017333in}}{\pgfqpoint{6.024059in}{5.025233in}}{\pgfqpoint{6.018235in}{5.031057in}}%
\pgfpathcurveto{\pgfqpoint{6.012411in}{5.036881in}}{\pgfqpoint{6.004511in}{5.040153in}}{\pgfqpoint{5.996275in}{5.040153in}}%
\pgfpathcurveto{\pgfqpoint{5.988039in}{5.040153in}}{\pgfqpoint{5.980139in}{5.036881in}}{\pgfqpoint{5.974315in}{5.031057in}}%
\pgfpathcurveto{\pgfqpoint{5.968491in}{5.025233in}}{\pgfqpoint{5.965219in}{5.017333in}}{\pgfqpoint{5.965219in}{5.009096in}}%
\pgfpathcurveto{\pgfqpoint{5.965219in}{5.000860in}}{\pgfqpoint{5.968491in}{4.992960in}}{\pgfqpoint{5.974315in}{4.987136in}}%
\pgfpathcurveto{\pgfqpoint{5.980139in}{4.981312in}}{\pgfqpoint{5.988039in}{4.978040in}}{\pgfqpoint{5.996275in}{4.978040in}}%
\pgfpathclose%
\pgfusepath{stroke,fill}%
\end{pgfscope}%
\begin{pgfscope}%
\pgfpathrectangle{\pgfqpoint{0.894063in}{3.540000in}}{\pgfqpoint{6.713438in}{2.060556in}} %
\pgfusepath{clip}%
\pgfsetbuttcap%
\pgfsetroundjoin%
\definecolor{currentfill}{rgb}{0.000000,0.000000,1.000000}%
\pgfsetfillcolor{currentfill}%
\pgfsetlinewidth{1.003750pt}%
\definecolor{currentstroke}{rgb}{0.000000,0.000000,0.000000}%
\pgfsetstrokecolor{currentstroke}%
\pgfsetdash{}{0pt}%
\pgfpathmoveto{\pgfqpoint{6.399081in}{4.900983in}}%
\pgfpathcurveto{\pgfqpoint{6.407318in}{4.900983in}}{\pgfqpoint{6.415218in}{4.904256in}}{\pgfqpoint{6.421042in}{4.910080in}}%
\pgfpathcurveto{\pgfqpoint{6.426865in}{4.915904in}}{\pgfqpoint{6.430138in}{4.923804in}}{\pgfqpoint{6.430138in}{4.932040in}}%
\pgfpathcurveto{\pgfqpoint{6.430138in}{4.940276in}}{\pgfqpoint{6.426865in}{4.948176in}}{\pgfqpoint{6.421042in}{4.954000in}}%
\pgfpathcurveto{\pgfqpoint{6.415218in}{4.959824in}}{\pgfqpoint{6.407318in}{4.963096in}}{\pgfqpoint{6.399081in}{4.963096in}}%
\pgfpathcurveto{\pgfqpoint{6.390845in}{4.963096in}}{\pgfqpoint{6.382945in}{4.959824in}}{\pgfqpoint{6.377121in}{4.954000in}}%
\pgfpathcurveto{\pgfqpoint{6.371297in}{4.948176in}}{\pgfqpoint{6.368025in}{4.940276in}}{\pgfqpoint{6.368025in}{4.932040in}}%
\pgfpathcurveto{\pgfqpoint{6.368025in}{4.923804in}}{\pgfqpoint{6.371297in}{4.915904in}}{\pgfqpoint{6.377121in}{4.910080in}}%
\pgfpathcurveto{\pgfqpoint{6.382945in}{4.904256in}}{\pgfqpoint{6.390845in}{4.900983in}}{\pgfqpoint{6.399081in}{4.900983in}}%
\pgfpathclose%
\pgfusepath{stroke,fill}%
\end{pgfscope}%
\begin{pgfscope}%
\pgfpathrectangle{\pgfqpoint{0.894063in}{3.540000in}}{\pgfqpoint{6.713438in}{2.060556in}} %
\pgfusepath{clip}%
\pgfsetbuttcap%
\pgfsetroundjoin%
\definecolor{currentfill}{rgb}{0.000000,0.000000,1.000000}%
\pgfsetfillcolor{currentfill}%
\pgfsetlinewidth{1.003750pt}%
\definecolor{currentstroke}{rgb}{0.000000,0.000000,0.000000}%
\pgfsetstrokecolor{currentstroke}%
\pgfsetdash{}{0pt}%
\pgfpathmoveto{\pgfqpoint{4.787856in}{5.126938in}}%
\pgfpathcurveto{\pgfqpoint{4.796093in}{5.126938in}}{\pgfqpoint{4.803993in}{5.130211in}}{\pgfqpoint{4.809817in}{5.136035in}}%
\pgfpathcurveto{\pgfqpoint{4.815640in}{5.141859in}}{\pgfqpoint{4.818913in}{5.149759in}}{\pgfqpoint{4.818913in}{5.157995in}}%
\pgfpathcurveto{\pgfqpoint{4.818913in}{5.166231in}}{\pgfqpoint{4.815640in}{5.174131in}}{\pgfqpoint{4.809817in}{5.179955in}}%
\pgfpathcurveto{\pgfqpoint{4.803993in}{5.185779in}}{\pgfqpoint{4.796093in}{5.189051in}}{\pgfqpoint{4.787856in}{5.189051in}}%
\pgfpathcurveto{\pgfqpoint{4.779620in}{5.189051in}}{\pgfqpoint{4.771720in}{5.185779in}}{\pgfqpoint{4.765896in}{5.179955in}}%
\pgfpathcurveto{\pgfqpoint{4.760072in}{5.174131in}}{\pgfqpoint{4.756800in}{5.166231in}}{\pgfqpoint{4.756800in}{5.157995in}}%
\pgfpathcurveto{\pgfqpoint{4.756800in}{5.149759in}}{\pgfqpoint{4.760072in}{5.141859in}}{\pgfqpoint{4.765896in}{5.136035in}}%
\pgfpathcurveto{\pgfqpoint{4.771720in}{5.130211in}}{\pgfqpoint{4.779620in}{5.126938in}}{\pgfqpoint{4.787856in}{5.126938in}}%
\pgfpathclose%
\pgfusepath{stroke,fill}%
\end{pgfscope}%
\begin{pgfscope}%
\pgfpathrectangle{\pgfqpoint{0.894063in}{3.540000in}}{\pgfqpoint{6.713438in}{2.060556in}} %
\pgfusepath{clip}%
\pgfsetbuttcap%
\pgfsetroundjoin%
\definecolor{currentfill}{rgb}{0.000000,0.000000,1.000000}%
\pgfsetfillcolor{currentfill}%
\pgfsetlinewidth{1.003750pt}%
\definecolor{currentstroke}{rgb}{0.000000,0.000000,0.000000}%
\pgfsetstrokecolor{currentstroke}%
\pgfsetdash{}{0pt}%
\pgfpathmoveto{\pgfqpoint{4.922125in}{5.123229in}}%
\pgfpathcurveto{\pgfqpoint{4.930361in}{5.123229in}}{\pgfqpoint{4.938261in}{5.126502in}}{\pgfqpoint{4.944085in}{5.132326in}}%
\pgfpathcurveto{\pgfqpoint{4.949909in}{5.138150in}}{\pgfqpoint{4.953181in}{5.146050in}}{\pgfqpoint{4.953181in}{5.154286in}}%
\pgfpathcurveto{\pgfqpoint{4.953181in}{5.162522in}}{\pgfqpoint{4.949909in}{5.170422in}}{\pgfqpoint{4.944085in}{5.176246in}}%
\pgfpathcurveto{\pgfqpoint{4.938261in}{5.182070in}}{\pgfqpoint{4.930361in}{5.185342in}}{\pgfqpoint{4.922125in}{5.185342in}}%
\pgfpathcurveto{\pgfqpoint{4.913889in}{5.185342in}}{\pgfqpoint{4.905989in}{5.182070in}}{\pgfqpoint{4.900165in}{5.176246in}}%
\pgfpathcurveto{\pgfqpoint{4.894341in}{5.170422in}}{\pgfqpoint{4.891069in}{5.162522in}}{\pgfqpoint{4.891069in}{5.154286in}}%
\pgfpathcurveto{\pgfqpoint{4.891069in}{5.146050in}}{\pgfqpoint{4.894341in}{5.138150in}}{\pgfqpoint{4.900165in}{5.132326in}}%
\pgfpathcurveto{\pgfqpoint{4.905989in}{5.126502in}}{\pgfqpoint{4.913889in}{5.123229in}}{\pgfqpoint{4.922125in}{5.123229in}}%
\pgfpathclose%
\pgfusepath{stroke,fill}%
\end{pgfscope}%
\begin{pgfscope}%
\pgfpathrectangle{\pgfqpoint{0.894063in}{3.540000in}}{\pgfqpoint{6.713438in}{2.060556in}} %
\pgfusepath{clip}%
\pgfsetbuttcap%
\pgfsetroundjoin%
\definecolor{currentfill}{rgb}{0.000000,0.000000,1.000000}%
\pgfsetfillcolor{currentfill}%
\pgfsetlinewidth{1.003750pt}%
\definecolor{currentstroke}{rgb}{0.000000,0.000000,0.000000}%
\pgfsetstrokecolor{currentstroke}%
\pgfsetdash{}{0pt}%
\pgfpathmoveto{\pgfqpoint{6.130544in}{4.950448in}}%
\pgfpathcurveto{\pgfqpoint{6.138780in}{4.950448in}}{\pgfqpoint{6.146680in}{4.953720in}}{\pgfqpoint{6.152504in}{4.959544in}}%
\pgfpathcurveto{\pgfqpoint{6.158328in}{4.965368in}}{\pgfqpoint{6.161600in}{4.973268in}}{\pgfqpoint{6.161600in}{4.981504in}}%
\pgfpathcurveto{\pgfqpoint{6.161600in}{4.989741in}}{\pgfqpoint{6.158328in}{4.997641in}}{\pgfqpoint{6.152504in}{5.003464in}}%
\pgfpathcurveto{\pgfqpoint{6.146680in}{5.009288in}}{\pgfqpoint{6.138780in}{5.012561in}}{\pgfqpoint{6.130544in}{5.012561in}}%
\pgfpathcurveto{\pgfqpoint{6.122307in}{5.012561in}}{\pgfqpoint{6.114407in}{5.009288in}}{\pgfqpoint{6.108583in}{5.003464in}}%
\pgfpathcurveto{\pgfqpoint{6.102760in}{4.997641in}}{\pgfqpoint{6.099487in}{4.989741in}}{\pgfqpoint{6.099487in}{4.981504in}}%
\pgfpathcurveto{\pgfqpoint{6.099487in}{4.973268in}}{\pgfqpoint{6.102760in}{4.965368in}}{\pgfqpoint{6.108583in}{4.959544in}}%
\pgfpathcurveto{\pgfqpoint{6.114407in}{4.953720in}}{\pgfqpoint{6.122307in}{4.950448in}}{\pgfqpoint{6.130544in}{4.950448in}}%
\pgfpathclose%
\pgfusepath{stroke,fill}%
\end{pgfscope}%
\begin{pgfscope}%
\pgfpathrectangle{\pgfqpoint{0.894063in}{3.540000in}}{\pgfqpoint{6.713438in}{2.060556in}} %
\pgfusepath{clip}%
\pgfsetbuttcap%
\pgfsetroundjoin%
\definecolor{currentfill}{rgb}{0.000000,0.000000,1.000000}%
\pgfsetfillcolor{currentfill}%
\pgfsetlinewidth{1.003750pt}%
\definecolor{currentstroke}{rgb}{0.000000,0.000000,0.000000}%
\pgfsetstrokecolor{currentstroke}%
\pgfsetdash{}{0pt}%
\pgfpathmoveto{\pgfqpoint{5.727738in}{5.013609in}}%
\pgfpathcurveto{\pgfqpoint{5.735974in}{5.013609in}}{\pgfqpoint{5.743874in}{5.016882in}}{\pgfqpoint{5.749698in}{5.022705in}}%
\pgfpathcurveto{\pgfqpoint{5.755522in}{5.028529in}}{\pgfqpoint{5.758794in}{5.036429in}}{\pgfqpoint{5.758794in}{5.044666in}}%
\pgfpathcurveto{\pgfqpoint{5.758794in}{5.052902in}}{\pgfqpoint{5.755522in}{5.060802in}}{\pgfqpoint{5.749698in}{5.066626in}}%
\pgfpathcurveto{\pgfqpoint{5.743874in}{5.072450in}}{\pgfqpoint{5.735974in}{5.075722in}}{\pgfqpoint{5.727738in}{5.075722in}}%
\pgfpathcurveto{\pgfqpoint{5.719501in}{5.075722in}}{\pgfqpoint{5.711601in}{5.072450in}}{\pgfqpoint{5.705777in}{5.066626in}}%
\pgfpathcurveto{\pgfqpoint{5.699953in}{5.060802in}}{\pgfqpoint{5.696681in}{5.052902in}}{\pgfqpoint{5.696681in}{5.044666in}}%
\pgfpathcurveto{\pgfqpoint{5.696681in}{5.036429in}}{\pgfqpoint{5.699953in}{5.028529in}}{\pgfqpoint{5.705777in}{5.022705in}}%
\pgfpathcurveto{\pgfqpoint{5.711601in}{5.016882in}}{\pgfqpoint{5.719501in}{5.013609in}}{\pgfqpoint{5.727738in}{5.013609in}}%
\pgfpathclose%
\pgfusepath{stroke,fill}%
\end{pgfscope}%
\begin{pgfscope}%
\pgfpathrectangle{\pgfqpoint{0.894063in}{3.540000in}}{\pgfqpoint{6.713438in}{2.060556in}} %
\pgfusepath{clip}%
\pgfsetbuttcap%
\pgfsetroundjoin%
\definecolor{currentfill}{rgb}{0.000000,0.000000,1.000000}%
\pgfsetfillcolor{currentfill}%
\pgfsetlinewidth{1.003750pt}%
\definecolor{currentstroke}{rgb}{0.000000,0.000000,0.000000}%
\pgfsetstrokecolor{currentstroke}%
\pgfsetdash{}{0pt}%
\pgfpathmoveto{\pgfqpoint{1.028331in}{5.264744in}}%
\pgfpathcurveto{\pgfqpoint{1.036568in}{5.264744in}}{\pgfqpoint{1.044468in}{5.268017in}}{\pgfqpoint{1.050292in}{5.273841in}}%
\pgfpathcurveto{\pgfqpoint{1.056115in}{5.279664in}}{\pgfqpoint{1.059388in}{5.287564in}}{\pgfqpoint{1.059388in}{5.295801in}}%
\pgfpathcurveto{\pgfqpoint{1.059388in}{5.304037in}}{\pgfqpoint{1.056115in}{5.311937in}}{\pgfqpoint{1.050292in}{5.317761in}}%
\pgfpathcurveto{\pgfqpoint{1.044468in}{5.323585in}}{\pgfqpoint{1.036568in}{5.326857in}}{\pgfqpoint{1.028331in}{5.326857in}}%
\pgfpathcurveto{\pgfqpoint{1.020095in}{5.326857in}}{\pgfqpoint{1.012195in}{5.323585in}}{\pgfqpoint{1.006371in}{5.317761in}}%
\pgfpathcurveto{\pgfqpoint{1.000547in}{5.311937in}}{\pgfqpoint{0.997275in}{5.304037in}}{\pgfqpoint{0.997275in}{5.295801in}}%
\pgfpathcurveto{\pgfqpoint{0.997275in}{5.287564in}}{\pgfqpoint{1.000547in}{5.279664in}}{\pgfqpoint{1.006371in}{5.273841in}}%
\pgfpathcurveto{\pgfqpoint{1.012195in}{5.268017in}}{\pgfqpoint{1.020095in}{5.264744in}}{\pgfqpoint{1.028331in}{5.264744in}}%
\pgfpathclose%
\pgfusepath{stroke,fill}%
\end{pgfscope}%
\begin{pgfscope}%
\pgfpathrectangle{\pgfqpoint{0.894063in}{3.540000in}}{\pgfqpoint{6.713438in}{2.060556in}} %
\pgfusepath{clip}%
\pgfsetbuttcap%
\pgfsetroundjoin%
\definecolor{currentfill}{rgb}{0.000000,0.000000,1.000000}%
\pgfsetfillcolor{currentfill}%
\pgfsetlinewidth{1.003750pt}%
\definecolor{currentstroke}{rgb}{0.000000,0.000000,0.000000}%
\pgfsetstrokecolor{currentstroke}%
\pgfsetdash{}{0pt}%
\pgfpathmoveto{\pgfqpoint{5.324931in}{5.075298in}}%
\pgfpathcurveto{\pgfqpoint{5.333168in}{5.075298in}}{\pgfqpoint{5.341068in}{5.078570in}}{\pgfqpoint{5.346892in}{5.084394in}}%
\pgfpathcurveto{\pgfqpoint{5.352715in}{5.090218in}}{\pgfqpoint{5.355988in}{5.098118in}}{\pgfqpoint{5.355988in}{5.106355in}}%
\pgfpathcurveto{\pgfqpoint{5.355988in}{5.114591in}}{\pgfqpoint{5.352715in}{5.122491in}}{\pgfqpoint{5.346892in}{5.128315in}}%
\pgfpathcurveto{\pgfqpoint{5.341068in}{5.134139in}}{\pgfqpoint{5.333168in}{5.137411in}}{\pgfqpoint{5.324931in}{5.137411in}}%
\pgfpathcurveto{\pgfqpoint{5.316695in}{5.137411in}}{\pgfqpoint{5.308795in}{5.134139in}}{\pgfqpoint{5.302971in}{5.128315in}}%
\pgfpathcurveto{\pgfqpoint{5.297147in}{5.122491in}}{\pgfqpoint{5.293875in}{5.114591in}}{\pgfqpoint{5.293875in}{5.106355in}}%
\pgfpathcurveto{\pgfqpoint{5.293875in}{5.098118in}}{\pgfqpoint{5.297147in}{5.090218in}}{\pgfqpoint{5.302971in}{5.084394in}}%
\pgfpathcurveto{\pgfqpoint{5.308795in}{5.078570in}}{\pgfqpoint{5.316695in}{5.075298in}}{\pgfqpoint{5.324931in}{5.075298in}}%
\pgfpathclose%
\pgfusepath{stroke,fill}%
\end{pgfscope}%
\begin{pgfscope}%
\pgfpathrectangle{\pgfqpoint{0.894063in}{3.540000in}}{\pgfqpoint{6.713438in}{2.060556in}} %
\pgfusepath{clip}%
\pgfsetbuttcap%
\pgfsetroundjoin%
\definecolor{currentfill}{rgb}{0.000000,0.000000,1.000000}%
\pgfsetfillcolor{currentfill}%
\pgfsetlinewidth{1.003750pt}%
\definecolor{currentstroke}{rgb}{0.000000,0.000000,0.000000}%
\pgfsetstrokecolor{currentstroke}%
\pgfsetdash{}{0pt}%
\pgfpathmoveto{\pgfqpoint{7.338963in}{4.758800in}}%
\pgfpathcurveto{\pgfqpoint{7.347199in}{4.758800in}}{\pgfqpoint{7.355099in}{4.762072in}}{\pgfqpoint{7.360923in}{4.767896in}}%
\pgfpathcurveto{\pgfqpoint{7.366747in}{4.773720in}}{\pgfqpoint{7.370019in}{4.781620in}}{\pgfqpoint{7.370019in}{4.789856in}}%
\pgfpathcurveto{\pgfqpoint{7.370019in}{4.798092in}}{\pgfqpoint{7.366747in}{4.805992in}}{\pgfqpoint{7.360923in}{4.811816in}}%
\pgfpathcurveto{\pgfqpoint{7.355099in}{4.817640in}}{\pgfqpoint{7.347199in}{4.820913in}}{\pgfqpoint{7.338963in}{4.820913in}}%
\pgfpathcurveto{\pgfqpoint{7.330726in}{4.820913in}}{\pgfqpoint{7.322826in}{4.817640in}}{\pgfqpoint{7.317002in}{4.811816in}}%
\pgfpathcurveto{\pgfqpoint{7.311178in}{4.805992in}}{\pgfqpoint{7.307906in}{4.798092in}}{\pgfqpoint{7.307906in}{4.789856in}}%
\pgfpathcurveto{\pgfqpoint{7.307906in}{4.781620in}}{\pgfqpoint{7.311178in}{4.773720in}}{\pgfqpoint{7.317002in}{4.767896in}}%
\pgfpathcurveto{\pgfqpoint{7.322826in}{4.762072in}}{\pgfqpoint{7.330726in}{4.758800in}}{\pgfqpoint{7.338963in}{4.758800in}}%
\pgfpathclose%
\pgfusepath{stroke,fill}%
\end{pgfscope}%
\begin{pgfscope}%
\pgfpathrectangle{\pgfqpoint{0.894063in}{3.540000in}}{\pgfqpoint{6.713438in}{2.060556in}} %
\pgfusepath{clip}%
\pgfsetbuttcap%
\pgfsetroundjoin%
\definecolor{currentfill}{rgb}{0.000000,0.000000,1.000000}%
\pgfsetfillcolor{currentfill}%
\pgfsetlinewidth{1.003750pt}%
\definecolor{currentstroke}{rgb}{0.000000,0.000000,0.000000}%
\pgfsetstrokecolor{currentstroke}%
\pgfsetdash{}{0pt}%
\pgfpathmoveto{\pgfqpoint{7.204694in}{4.804795in}}%
\pgfpathcurveto{\pgfqpoint{7.212930in}{4.804795in}}{\pgfqpoint{7.220830in}{4.808068in}}{\pgfqpoint{7.226654in}{4.813892in}}%
\pgfpathcurveto{\pgfqpoint{7.232478in}{4.819715in}}{\pgfqpoint{7.235750in}{4.827616in}}{\pgfqpoint{7.235750in}{4.835852in}}%
\pgfpathcurveto{\pgfqpoint{7.235750in}{4.844088in}}{\pgfqpoint{7.232478in}{4.851988in}}{\pgfqpoint{7.226654in}{4.857812in}}%
\pgfpathcurveto{\pgfqpoint{7.220830in}{4.863636in}}{\pgfqpoint{7.212930in}{4.866908in}}{\pgfqpoint{7.204694in}{4.866908in}}%
\pgfpathcurveto{\pgfqpoint{7.196457in}{4.866908in}}{\pgfqpoint{7.188557in}{4.863636in}}{\pgfqpoint{7.182733in}{4.857812in}}%
\pgfpathcurveto{\pgfqpoint{7.176910in}{4.851988in}}{\pgfqpoint{7.173637in}{4.844088in}}{\pgfqpoint{7.173637in}{4.835852in}}%
\pgfpathcurveto{\pgfqpoint{7.173637in}{4.827616in}}{\pgfqpoint{7.176910in}{4.819715in}}{\pgfqpoint{7.182733in}{4.813892in}}%
\pgfpathcurveto{\pgfqpoint{7.188557in}{4.808068in}}{\pgfqpoint{7.196457in}{4.804795in}}{\pgfqpoint{7.204694in}{4.804795in}}%
\pgfpathclose%
\pgfusepath{stroke,fill}%
\end{pgfscope}%
\begin{pgfscope}%
\pgfpathrectangle{\pgfqpoint{0.894063in}{3.540000in}}{\pgfqpoint{6.713438in}{2.060556in}} %
\pgfusepath{clip}%
\pgfsetbuttcap%
\pgfsetroundjoin%
\definecolor{currentfill}{rgb}{0.000000,0.000000,1.000000}%
\pgfsetfillcolor{currentfill}%
\pgfsetlinewidth{1.003750pt}%
\definecolor{currentstroke}{rgb}{0.000000,0.000000,0.000000}%
\pgfsetstrokecolor{currentstroke}%
\pgfsetdash{}{0pt}%
\pgfpathmoveto{\pgfqpoint{6.264813in}{4.925213in}}%
\pgfpathcurveto{\pgfqpoint{6.273049in}{4.925213in}}{\pgfqpoint{6.280949in}{4.928485in}}{\pgfqpoint{6.286773in}{4.934309in}}%
\pgfpathcurveto{\pgfqpoint{6.292597in}{4.940133in}}{\pgfqpoint{6.295869in}{4.948033in}}{\pgfqpoint{6.295869in}{4.956269in}}%
\pgfpathcurveto{\pgfqpoint{6.295869in}{4.964506in}}{\pgfqpoint{6.292597in}{4.972406in}}{\pgfqpoint{6.286773in}{4.978230in}}%
\pgfpathcurveto{\pgfqpoint{6.280949in}{4.984053in}}{\pgfqpoint{6.273049in}{4.987326in}}{\pgfqpoint{6.264813in}{4.987326in}}%
\pgfpathcurveto{\pgfqpoint{6.256576in}{4.987326in}}{\pgfqpoint{6.248676in}{4.984053in}}{\pgfqpoint{6.242852in}{4.978230in}}%
\pgfpathcurveto{\pgfqpoint{6.237028in}{4.972406in}}{\pgfqpoint{6.233756in}{4.964506in}}{\pgfqpoint{6.233756in}{4.956269in}}%
\pgfpathcurveto{\pgfqpoint{6.233756in}{4.948033in}}{\pgfqpoint{6.237028in}{4.940133in}}{\pgfqpoint{6.242852in}{4.934309in}}%
\pgfpathcurveto{\pgfqpoint{6.248676in}{4.928485in}}{\pgfqpoint{6.256576in}{4.925213in}}{\pgfqpoint{6.264813in}{4.925213in}}%
\pgfpathclose%
\pgfusepath{stroke,fill}%
\end{pgfscope}%
\begin{pgfscope}%
\pgfpathrectangle{\pgfqpoint{0.894063in}{3.540000in}}{\pgfqpoint{6.713438in}{2.060556in}} %
\pgfusepath{clip}%
\pgfsetbuttcap%
\pgfsetroundjoin%
\definecolor{currentfill}{rgb}{0.000000,0.000000,1.000000}%
\pgfsetfillcolor{currentfill}%
\pgfsetlinewidth{1.003750pt}%
\definecolor{currentstroke}{rgb}{0.000000,0.000000,0.000000}%
\pgfsetstrokecolor{currentstroke}%
\pgfsetdash{}{0pt}%
\pgfpathmoveto{\pgfqpoint{7.473231in}{4.740588in}}%
\pgfpathcurveto{\pgfqpoint{7.481468in}{4.740588in}}{\pgfqpoint{7.489368in}{4.743861in}}{\pgfqpoint{7.495192in}{4.749685in}}%
\pgfpathcurveto{\pgfqpoint{7.501015in}{4.755509in}}{\pgfqpoint{7.504288in}{4.763409in}}{\pgfqpoint{7.504288in}{4.771645in}}%
\pgfpathcurveto{\pgfqpoint{7.504288in}{4.779881in}}{\pgfqpoint{7.501015in}{4.787781in}}{\pgfqpoint{7.495192in}{4.793605in}}%
\pgfpathcurveto{\pgfqpoint{7.489368in}{4.799429in}}{\pgfqpoint{7.481468in}{4.802701in}}{\pgfqpoint{7.473231in}{4.802701in}}%
\pgfpathcurveto{\pgfqpoint{7.464995in}{4.802701in}}{\pgfqpoint{7.457095in}{4.799429in}}{\pgfqpoint{7.451271in}{4.793605in}}%
\pgfpathcurveto{\pgfqpoint{7.445447in}{4.787781in}}{\pgfqpoint{7.442175in}{4.779881in}}{\pgfqpoint{7.442175in}{4.771645in}}%
\pgfpathcurveto{\pgfqpoint{7.442175in}{4.763409in}}{\pgfqpoint{7.445447in}{4.755509in}}{\pgfqpoint{7.451271in}{4.749685in}}%
\pgfpathcurveto{\pgfqpoint{7.457095in}{4.743861in}}{\pgfqpoint{7.464995in}{4.740588in}}{\pgfqpoint{7.473231in}{4.740588in}}%
\pgfpathclose%
\pgfusepath{stroke,fill}%
\end{pgfscope}%
\begin{pgfscope}%
\pgfpathrectangle{\pgfqpoint{0.894063in}{3.540000in}}{\pgfqpoint{6.713438in}{2.060556in}} %
\pgfusepath{clip}%
\pgfsetbuttcap%
\pgfsetroundjoin%
\definecolor{currentfill}{rgb}{0.000000,0.000000,1.000000}%
\pgfsetfillcolor{currentfill}%
\pgfsetlinewidth{1.003750pt}%
\definecolor{currentstroke}{rgb}{0.000000,0.000000,0.000000}%
\pgfsetstrokecolor{currentstroke}%
\pgfsetdash{}{0pt}%
\pgfpathmoveto{\pgfqpoint{5.056394in}{5.101326in}}%
\pgfpathcurveto{\pgfqpoint{5.064630in}{5.101326in}}{\pgfqpoint{5.072530in}{5.104598in}}{\pgfqpoint{5.078354in}{5.110422in}}%
\pgfpathcurveto{\pgfqpoint{5.084178in}{5.116246in}}{\pgfqpoint{5.087450in}{5.124146in}}{\pgfqpoint{5.087450in}{5.132382in}}%
\pgfpathcurveto{\pgfqpoint{5.087450in}{5.140619in}}{\pgfqpoint{5.084178in}{5.148519in}}{\pgfqpoint{5.078354in}{5.154342in}}%
\pgfpathcurveto{\pgfqpoint{5.072530in}{5.160166in}}{\pgfqpoint{5.064630in}{5.163439in}}{\pgfqpoint{5.056394in}{5.163439in}}%
\pgfpathcurveto{\pgfqpoint{5.048157in}{5.163439in}}{\pgfqpoint{5.040257in}{5.160166in}}{\pgfqpoint{5.034433in}{5.154342in}}%
\pgfpathcurveto{\pgfqpoint{5.028610in}{5.148519in}}{\pgfqpoint{5.025337in}{5.140619in}}{\pgfqpoint{5.025337in}{5.132382in}}%
\pgfpathcurveto{\pgfqpoint{5.025337in}{5.124146in}}{\pgfqpoint{5.028610in}{5.116246in}}{\pgfqpoint{5.034433in}{5.110422in}}%
\pgfpathcurveto{\pgfqpoint{5.040257in}{5.104598in}}{\pgfqpoint{5.048157in}{5.101326in}}{\pgfqpoint{5.056394in}{5.101326in}}%
\pgfpathclose%
\pgfusepath{stroke,fill}%
\end{pgfscope}%
\begin{pgfscope}%
\pgfpathrectangle{\pgfqpoint{0.894063in}{3.540000in}}{\pgfqpoint{6.713438in}{2.060556in}} %
\pgfusepath{clip}%
\pgfsetbuttcap%
\pgfsetroundjoin%
\definecolor{currentfill}{rgb}{0.000000,0.000000,1.000000}%
\pgfsetfillcolor{currentfill}%
\pgfsetlinewidth{1.003750pt}%
\definecolor{currentstroke}{rgb}{0.000000,0.000000,0.000000}%
\pgfsetstrokecolor{currentstroke}%
\pgfsetdash{}{0pt}%
\pgfpathmoveto{\pgfqpoint{2.908094in}{5.126971in}}%
\pgfpathcurveto{\pgfqpoint{2.916330in}{5.126971in}}{\pgfqpoint{2.924230in}{5.130244in}}{\pgfqpoint{2.930054in}{5.136068in}}%
\pgfpathcurveto{\pgfqpoint{2.935878in}{5.141892in}}{\pgfqpoint{2.939150in}{5.149792in}}{\pgfqpoint{2.939150in}{5.158028in}}%
\pgfpathcurveto{\pgfqpoint{2.939150in}{5.166264in}}{\pgfqpoint{2.935878in}{5.174164in}}{\pgfqpoint{2.930054in}{5.179988in}}%
\pgfpathcurveto{\pgfqpoint{2.924230in}{5.185812in}}{\pgfqpoint{2.916330in}{5.189084in}}{\pgfqpoint{2.908094in}{5.189084in}}%
\pgfpathcurveto{\pgfqpoint{2.899857in}{5.189084in}}{\pgfqpoint{2.891957in}{5.185812in}}{\pgfqpoint{2.886133in}{5.179988in}}%
\pgfpathcurveto{\pgfqpoint{2.880310in}{5.174164in}}{\pgfqpoint{2.877037in}{5.166264in}}{\pgfqpoint{2.877037in}{5.158028in}}%
\pgfpathcurveto{\pgfqpoint{2.877037in}{5.149792in}}{\pgfqpoint{2.880310in}{5.141892in}}{\pgfqpoint{2.886133in}{5.136068in}}%
\pgfpathcurveto{\pgfqpoint{2.891957in}{5.130244in}}{\pgfqpoint{2.899857in}{5.126971in}}{\pgfqpoint{2.908094in}{5.126971in}}%
\pgfpathclose%
\pgfusepath{stroke,fill}%
\end{pgfscope}%
\begin{pgfscope}%
\pgfpathrectangle{\pgfqpoint{0.894063in}{3.540000in}}{\pgfqpoint{6.713438in}{2.060556in}} %
\pgfusepath{clip}%
\pgfsetbuttcap%
\pgfsetroundjoin%
\definecolor{currentfill}{rgb}{0.000000,0.000000,1.000000}%
\pgfsetfillcolor{currentfill}%
\pgfsetlinewidth{1.003750pt}%
\definecolor{currentstroke}{rgb}{0.000000,0.000000,0.000000}%
\pgfsetstrokecolor{currentstroke}%
\pgfsetdash{}{0pt}%
\pgfpathmoveto{\pgfqpoint{3.445169in}{5.126965in}}%
\pgfpathcurveto{\pgfqpoint{3.453405in}{5.126965in}}{\pgfqpoint{3.461305in}{5.130237in}}{\pgfqpoint{3.467129in}{5.136061in}}%
\pgfpathcurveto{\pgfqpoint{3.472953in}{5.141885in}}{\pgfqpoint{3.476225in}{5.149785in}}{\pgfqpoint{3.476225in}{5.158021in}}%
\pgfpathcurveto{\pgfqpoint{3.476225in}{5.166257in}}{\pgfqpoint{3.472953in}{5.174157in}}{\pgfqpoint{3.467129in}{5.179981in}}%
\pgfpathcurveto{\pgfqpoint{3.461305in}{5.185805in}}{\pgfqpoint{3.453405in}{5.189078in}}{\pgfqpoint{3.445169in}{5.189078in}}%
\pgfpathcurveto{\pgfqpoint{3.436932in}{5.189078in}}{\pgfqpoint{3.429032in}{5.185805in}}{\pgfqpoint{3.423208in}{5.179981in}}%
\pgfpathcurveto{\pgfqpoint{3.417385in}{5.174157in}}{\pgfqpoint{3.414112in}{5.166257in}}{\pgfqpoint{3.414112in}{5.158021in}}%
\pgfpathcurveto{\pgfqpoint{3.414112in}{5.149785in}}{\pgfqpoint{3.417385in}{5.141885in}}{\pgfqpoint{3.423208in}{5.136061in}}%
\pgfpathcurveto{\pgfqpoint{3.429032in}{5.130237in}}{\pgfqpoint{3.436932in}{5.126965in}}{\pgfqpoint{3.445169in}{5.126965in}}%
\pgfpathclose%
\pgfusepath{stroke,fill}%
\end{pgfscope}%
\begin{pgfscope}%
\pgfpathrectangle{\pgfqpoint{0.894063in}{3.540000in}}{\pgfqpoint{6.713438in}{2.060556in}} %
\pgfusepath{clip}%
\pgfsetbuttcap%
\pgfsetroundjoin%
\definecolor{currentfill}{rgb}{0.000000,0.000000,1.000000}%
\pgfsetfillcolor{currentfill}%
\pgfsetlinewidth{1.003750pt}%
\definecolor{currentstroke}{rgb}{0.000000,0.000000,0.000000}%
\pgfsetstrokecolor{currentstroke}%
\pgfsetdash{}{0pt}%
\pgfpathmoveto{\pgfqpoint{4.116513in}{5.126943in}}%
\pgfpathcurveto{\pgfqpoint{4.124749in}{5.126943in}}{\pgfqpoint{4.132649in}{5.130215in}}{\pgfqpoint{4.138473in}{5.136039in}}%
\pgfpathcurveto{\pgfqpoint{4.144297in}{5.141863in}}{\pgfqpoint{4.147569in}{5.149763in}}{\pgfqpoint{4.147569in}{5.157999in}}%
\pgfpathcurveto{\pgfqpoint{4.147569in}{5.166235in}}{\pgfqpoint{4.144297in}{5.174135in}}{\pgfqpoint{4.138473in}{5.179959in}}%
\pgfpathcurveto{\pgfqpoint{4.132649in}{5.185783in}}{\pgfqpoint{4.124749in}{5.189056in}}{\pgfqpoint{4.116513in}{5.189056in}}%
\pgfpathcurveto{\pgfqpoint{4.108276in}{5.189056in}}{\pgfqpoint{4.100376in}{5.185783in}}{\pgfqpoint{4.094552in}{5.179959in}}%
\pgfpathcurveto{\pgfqpoint{4.088728in}{5.174135in}}{\pgfqpoint{4.085456in}{5.166235in}}{\pgfqpoint{4.085456in}{5.157999in}}%
\pgfpathcurveto{\pgfqpoint{4.085456in}{5.149763in}}{\pgfqpoint{4.088728in}{5.141863in}}{\pgfqpoint{4.094552in}{5.136039in}}%
\pgfpathcurveto{\pgfqpoint{4.100376in}{5.130215in}}{\pgfqpoint{4.108276in}{5.126943in}}{\pgfqpoint{4.116513in}{5.126943in}}%
\pgfpathclose%
\pgfusepath{stroke,fill}%
\end{pgfscope}%
\begin{pgfscope}%
\pgfpathrectangle{\pgfqpoint{0.894063in}{3.540000in}}{\pgfqpoint{6.713438in}{2.060556in}} %
\pgfusepath{clip}%
\pgfsetbuttcap%
\pgfsetroundjoin%
\definecolor{currentfill}{rgb}{0.000000,0.000000,1.000000}%
\pgfsetfillcolor{currentfill}%
\pgfsetlinewidth{1.003750pt}%
\definecolor{currentstroke}{rgb}{0.000000,0.000000,0.000000}%
\pgfsetstrokecolor{currentstroke}%
\pgfsetdash{}{0pt}%
\pgfpathmoveto{\pgfqpoint{1.431138in}{5.139366in}}%
\pgfpathcurveto{\pgfqpoint{1.439374in}{5.139366in}}{\pgfqpoint{1.447274in}{5.142639in}}{\pgfqpoint{1.453098in}{5.148463in}}%
\pgfpathcurveto{\pgfqpoint{1.458922in}{5.154286in}}{\pgfqpoint{1.462194in}{5.162187in}}{\pgfqpoint{1.462194in}{5.170423in}}%
\pgfpathcurveto{\pgfqpoint{1.462194in}{5.178659in}}{\pgfqpoint{1.458922in}{5.186559in}}{\pgfqpoint{1.453098in}{5.192383in}}%
\pgfpathcurveto{\pgfqpoint{1.447274in}{5.198207in}}{\pgfqpoint{1.439374in}{5.201479in}}{\pgfqpoint{1.431138in}{5.201479in}}%
\pgfpathcurveto{\pgfqpoint{1.422901in}{5.201479in}}{\pgfqpoint{1.415001in}{5.198207in}}{\pgfqpoint{1.409177in}{5.192383in}}%
\pgfpathcurveto{\pgfqpoint{1.403353in}{5.186559in}}{\pgfqpoint{1.400081in}{5.178659in}}{\pgfqpoint{1.400081in}{5.170423in}}%
\pgfpathcurveto{\pgfqpoint{1.400081in}{5.162187in}}{\pgfqpoint{1.403353in}{5.154286in}}{\pgfqpoint{1.409177in}{5.148463in}}%
\pgfpathcurveto{\pgfqpoint{1.415001in}{5.142639in}}{\pgfqpoint{1.422901in}{5.139366in}}{\pgfqpoint{1.431138in}{5.139366in}}%
\pgfpathclose%
\pgfusepath{stroke,fill}%
\end{pgfscope}%
\begin{pgfscope}%
\pgfpathrectangle{\pgfqpoint{0.894063in}{3.540000in}}{\pgfqpoint{6.713438in}{2.060556in}} %
\pgfusepath{clip}%
\pgfsetbuttcap%
\pgfsetroundjoin%
\definecolor{currentfill}{rgb}{0.000000,0.000000,1.000000}%
\pgfsetfillcolor{currentfill}%
\pgfsetlinewidth{1.003750pt}%
\definecolor{currentstroke}{rgb}{0.000000,0.000000,0.000000}%
\pgfsetstrokecolor{currentstroke}%
\pgfsetdash{}{0pt}%
\pgfpathmoveto{\pgfqpoint{2.773825in}{5.126978in}}%
\pgfpathcurveto{\pgfqpoint{2.782061in}{5.126978in}}{\pgfqpoint{2.789961in}{5.130251in}}{\pgfqpoint{2.795785in}{5.136075in}}%
\pgfpathcurveto{\pgfqpoint{2.801609in}{5.141898in}}{\pgfqpoint{2.804881in}{5.149798in}}{\pgfqpoint{2.804881in}{5.158035in}}%
\pgfpathcurveto{\pgfqpoint{2.804881in}{5.166271in}}{\pgfqpoint{2.801609in}{5.174171in}}{\pgfqpoint{2.795785in}{5.179995in}}%
\pgfpathcurveto{\pgfqpoint{2.789961in}{5.185819in}}{\pgfqpoint{2.782061in}{5.189091in}}{\pgfqpoint{2.773825in}{5.189091in}}%
\pgfpathcurveto{\pgfqpoint{2.765589in}{5.189091in}}{\pgfqpoint{2.757689in}{5.185819in}}{\pgfqpoint{2.751865in}{5.179995in}}%
\pgfpathcurveto{\pgfqpoint{2.746041in}{5.174171in}}{\pgfqpoint{2.742769in}{5.166271in}}{\pgfqpoint{2.742769in}{5.158035in}}%
\pgfpathcurveto{\pgfqpoint{2.742769in}{5.149798in}}{\pgfqpoint{2.746041in}{5.141898in}}{\pgfqpoint{2.751865in}{5.136075in}}%
\pgfpathcurveto{\pgfqpoint{2.757689in}{5.130251in}}{\pgfqpoint{2.765589in}{5.126978in}}{\pgfqpoint{2.773825in}{5.126978in}}%
\pgfpathclose%
\pgfusepath{stroke,fill}%
\end{pgfscope}%
\begin{pgfscope}%
\pgfpathrectangle{\pgfqpoint{0.894063in}{3.540000in}}{\pgfqpoint{6.713438in}{2.060556in}} %
\pgfusepath{clip}%
\pgfsetbuttcap%
\pgfsetroundjoin%
\definecolor{currentfill}{rgb}{0.000000,0.000000,1.000000}%
\pgfsetfillcolor{currentfill}%
\pgfsetlinewidth{1.003750pt}%
\definecolor{currentstroke}{rgb}{0.000000,0.000000,0.000000}%
\pgfsetstrokecolor{currentstroke}%
\pgfsetdash{}{0pt}%
\pgfpathmoveto{\pgfqpoint{1.565406in}{5.136835in}}%
\pgfpathcurveto{\pgfqpoint{1.573643in}{5.136835in}}{\pgfqpoint{1.581543in}{5.140107in}}{\pgfqpoint{1.587367in}{5.145931in}}%
\pgfpathcurveto{\pgfqpoint{1.593190in}{5.151755in}}{\pgfqpoint{1.596463in}{5.159655in}}{\pgfqpoint{1.596463in}{5.167891in}}%
\pgfpathcurveto{\pgfqpoint{1.596463in}{5.176127in}}{\pgfqpoint{1.593190in}{5.184027in}}{\pgfqpoint{1.587367in}{5.189851in}}%
\pgfpathcurveto{\pgfqpoint{1.581543in}{5.195675in}}{\pgfqpoint{1.573643in}{5.198948in}}{\pgfqpoint{1.565406in}{5.198948in}}%
\pgfpathcurveto{\pgfqpoint{1.557170in}{5.198948in}}{\pgfqpoint{1.549270in}{5.195675in}}{\pgfqpoint{1.543446in}{5.189851in}}%
\pgfpathcurveto{\pgfqpoint{1.537622in}{5.184027in}}{\pgfqpoint{1.534350in}{5.176127in}}{\pgfqpoint{1.534350in}{5.167891in}}%
\pgfpathcurveto{\pgfqpoint{1.534350in}{5.159655in}}{\pgfqpoint{1.537622in}{5.151755in}}{\pgfqpoint{1.543446in}{5.145931in}}%
\pgfpathcurveto{\pgfqpoint{1.549270in}{5.140107in}}{\pgfqpoint{1.557170in}{5.136835in}}{\pgfqpoint{1.565406in}{5.136835in}}%
\pgfpathclose%
\pgfusepath{stroke,fill}%
\end{pgfscope}%
\begin{pgfscope}%
\pgfpathrectangle{\pgfqpoint{0.894063in}{3.540000in}}{\pgfqpoint{6.713438in}{2.060556in}} %
\pgfusepath{clip}%
\pgfsetbuttcap%
\pgfsetroundjoin%
\definecolor{currentfill}{rgb}{0.000000,0.000000,1.000000}%
\pgfsetfillcolor{currentfill}%
\pgfsetlinewidth{1.003750pt}%
\definecolor{currentstroke}{rgb}{0.000000,0.000000,0.000000}%
\pgfsetstrokecolor{currentstroke}%
\pgfsetdash{}{0pt}%
\pgfpathmoveto{\pgfqpoint{4.250781in}{5.126943in}}%
\pgfpathcurveto{\pgfqpoint{4.259018in}{5.126943in}}{\pgfqpoint{4.266918in}{5.130215in}}{\pgfqpoint{4.272742in}{5.136039in}}%
\pgfpathcurveto{\pgfqpoint{4.278565in}{5.141863in}}{\pgfqpoint{4.281838in}{5.149763in}}{\pgfqpoint{4.281838in}{5.157999in}}%
\pgfpathcurveto{\pgfqpoint{4.281838in}{5.166235in}}{\pgfqpoint{4.278565in}{5.174135in}}{\pgfqpoint{4.272742in}{5.179959in}}%
\pgfpathcurveto{\pgfqpoint{4.266918in}{5.185783in}}{\pgfqpoint{4.259018in}{5.189056in}}{\pgfqpoint{4.250781in}{5.189056in}}%
\pgfpathcurveto{\pgfqpoint{4.242545in}{5.189056in}}{\pgfqpoint{4.234645in}{5.185783in}}{\pgfqpoint{4.228821in}{5.179959in}}%
\pgfpathcurveto{\pgfqpoint{4.222997in}{5.174135in}}{\pgfqpoint{4.219725in}{5.166235in}}{\pgfqpoint{4.219725in}{5.157999in}}%
\pgfpathcurveto{\pgfqpoint{4.219725in}{5.149763in}}{\pgfqpoint{4.222997in}{5.141863in}}{\pgfqpoint{4.228821in}{5.136039in}}%
\pgfpathcurveto{\pgfqpoint{4.234645in}{5.130215in}}{\pgfqpoint{4.242545in}{5.126943in}}{\pgfqpoint{4.250781in}{5.126943in}}%
\pgfpathclose%
\pgfusepath{stroke,fill}%
\end{pgfscope}%
\begin{pgfscope}%
\pgfpathrectangle{\pgfqpoint{0.894063in}{3.540000in}}{\pgfqpoint{6.713438in}{2.060556in}} %
\pgfusepath{clip}%
\pgfsetbuttcap%
\pgfsetroundjoin%
\definecolor{currentfill}{rgb}{0.000000,0.000000,1.000000}%
\pgfsetfillcolor{currentfill}%
\pgfsetlinewidth{1.003750pt}%
\definecolor{currentstroke}{rgb}{0.000000,0.000000,0.000000}%
\pgfsetstrokecolor{currentstroke}%
\pgfsetdash{}{0pt}%
\pgfpathmoveto{\pgfqpoint{3.847975in}{5.126955in}}%
\pgfpathcurveto{\pgfqpoint{3.856211in}{5.126955in}}{\pgfqpoint{3.864111in}{5.130227in}}{\pgfqpoint{3.869935in}{5.136051in}}%
\pgfpathcurveto{\pgfqpoint{3.875759in}{5.141875in}}{\pgfqpoint{3.879031in}{5.149775in}}{\pgfqpoint{3.879031in}{5.158011in}}%
\pgfpathcurveto{\pgfqpoint{3.879031in}{5.166248in}}{\pgfqpoint{3.875759in}{5.174148in}}{\pgfqpoint{3.869935in}{5.179972in}}%
\pgfpathcurveto{\pgfqpoint{3.864111in}{5.185796in}}{\pgfqpoint{3.856211in}{5.189068in}}{\pgfqpoint{3.847975in}{5.189068in}}%
\pgfpathcurveto{\pgfqpoint{3.839739in}{5.189068in}}{\pgfqpoint{3.831839in}{5.185796in}}{\pgfqpoint{3.826015in}{5.179972in}}%
\pgfpathcurveto{\pgfqpoint{3.820191in}{5.174148in}}{\pgfqpoint{3.816919in}{5.166248in}}{\pgfqpoint{3.816919in}{5.158011in}}%
\pgfpathcurveto{\pgfqpoint{3.816919in}{5.149775in}}{\pgfqpoint{3.820191in}{5.141875in}}{\pgfqpoint{3.826015in}{5.136051in}}%
\pgfpathcurveto{\pgfqpoint{3.831839in}{5.130227in}}{\pgfqpoint{3.839739in}{5.126955in}}{\pgfqpoint{3.847975in}{5.126955in}}%
\pgfpathclose%
\pgfusepath{stroke,fill}%
\end{pgfscope}%
\begin{pgfscope}%
\pgfpathrectangle{\pgfqpoint{0.894063in}{3.540000in}}{\pgfqpoint{6.713438in}{2.060556in}} %
\pgfusepath{clip}%
\pgfsetbuttcap%
\pgfsetroundjoin%
\definecolor{currentfill}{rgb}{0.000000,0.000000,1.000000}%
\pgfsetfillcolor{currentfill}%
\pgfsetlinewidth{1.003750pt}%
\definecolor{currentstroke}{rgb}{0.000000,0.000000,0.000000}%
\pgfsetstrokecolor{currentstroke}%
\pgfsetdash{}{0pt}%
\pgfpathmoveto{\pgfqpoint{7.607500in}{4.725278in}}%
\pgfpathcurveto{\pgfqpoint{7.615736in}{4.725278in}}{\pgfqpoint{7.623636in}{4.728551in}}{\pgfqpoint{7.629460in}{4.734375in}}%
\pgfpathcurveto{\pgfqpoint{7.635284in}{4.740199in}}{\pgfqpoint{7.638556in}{4.748099in}}{\pgfqpoint{7.638556in}{4.756335in}}%
\pgfpathcurveto{\pgfqpoint{7.638556in}{4.764571in}}{\pgfqpoint{7.635284in}{4.772471in}}{\pgfqpoint{7.629460in}{4.778295in}}%
\pgfpathcurveto{\pgfqpoint{7.623636in}{4.784119in}}{\pgfqpoint{7.615736in}{4.787391in}}{\pgfqpoint{7.607500in}{4.787391in}}%
\pgfpathcurveto{\pgfqpoint{7.599264in}{4.787391in}}{\pgfqpoint{7.591364in}{4.784119in}}{\pgfqpoint{7.585540in}{4.778295in}}%
\pgfpathcurveto{\pgfqpoint{7.579716in}{4.772471in}}{\pgfqpoint{7.576444in}{4.764571in}}{\pgfqpoint{7.576444in}{4.756335in}}%
\pgfpathcurveto{\pgfqpoint{7.576444in}{4.748099in}}{\pgfqpoint{7.579716in}{4.740199in}}{\pgfqpoint{7.585540in}{4.734375in}}%
\pgfpathcurveto{\pgfqpoint{7.591364in}{4.728551in}}{\pgfqpoint{7.599264in}{4.725278in}}{\pgfqpoint{7.607500in}{4.725278in}}%
\pgfpathclose%
\pgfusepath{stroke,fill}%
\end{pgfscope}%
\begin{pgfscope}%
\pgfpathrectangle{\pgfqpoint{0.894063in}{3.540000in}}{\pgfqpoint{6.713438in}{2.060556in}} %
\pgfusepath{clip}%
\pgfsetbuttcap%
\pgfsetroundjoin%
\definecolor{currentfill}{rgb}{0.000000,0.000000,1.000000}%
\pgfsetfillcolor{currentfill}%
\pgfsetlinewidth{1.003750pt}%
\definecolor{currentstroke}{rgb}{0.000000,0.000000,0.000000}%
\pgfsetstrokecolor{currentstroke}%
\pgfsetdash{}{0pt}%
\pgfpathmoveto{\pgfqpoint{4.385050in}{5.126938in}}%
\pgfpathcurveto{\pgfqpoint{4.393286in}{5.126938in}}{\pgfqpoint{4.401186in}{5.130211in}}{\pgfqpoint{4.407010in}{5.136035in}}%
\pgfpathcurveto{\pgfqpoint{4.412834in}{5.141859in}}{\pgfqpoint{4.416106in}{5.149759in}}{\pgfqpoint{4.416106in}{5.157995in}}%
\pgfpathcurveto{\pgfqpoint{4.416106in}{5.166231in}}{\pgfqpoint{4.412834in}{5.174131in}}{\pgfqpoint{4.407010in}{5.179955in}}%
\pgfpathcurveto{\pgfqpoint{4.401186in}{5.185779in}}{\pgfqpoint{4.393286in}{5.189051in}}{\pgfqpoint{4.385050in}{5.189051in}}%
\pgfpathcurveto{\pgfqpoint{4.376814in}{5.189051in}}{\pgfqpoint{4.368914in}{5.185779in}}{\pgfqpoint{4.363090in}{5.179955in}}%
\pgfpathcurveto{\pgfqpoint{4.357266in}{5.174131in}}{\pgfqpoint{4.353994in}{5.166231in}}{\pgfqpoint{4.353994in}{5.157995in}}%
\pgfpathcurveto{\pgfqpoint{4.353994in}{5.149759in}}{\pgfqpoint{4.357266in}{5.141859in}}{\pgfqpoint{4.363090in}{5.136035in}}%
\pgfpathcurveto{\pgfqpoint{4.368914in}{5.130211in}}{\pgfqpoint{4.376814in}{5.126938in}}{\pgfqpoint{4.385050in}{5.126938in}}%
\pgfpathclose%
\pgfusepath{stroke,fill}%
\end{pgfscope}%
\begin{pgfscope}%
\pgfpathrectangle{\pgfqpoint{0.894063in}{3.540000in}}{\pgfqpoint{6.713438in}{2.060556in}} %
\pgfusepath{clip}%
\pgfsetbuttcap%
\pgfsetroundjoin%
\definecolor{currentfill}{rgb}{0.000000,0.000000,1.000000}%
\pgfsetfillcolor{currentfill}%
\pgfsetlinewidth{1.003750pt}%
\definecolor{currentstroke}{rgb}{0.000000,0.000000,0.000000}%
\pgfsetstrokecolor{currentstroke}%
\pgfsetdash{}{0pt}%
\pgfpathmoveto{\pgfqpoint{6.533350in}{4.882511in}}%
\pgfpathcurveto{\pgfqpoint{6.541586in}{4.882511in}}{\pgfqpoint{6.549486in}{4.885784in}}{\pgfqpoint{6.555310in}{4.891607in}}%
\pgfpathcurveto{\pgfqpoint{6.561134in}{4.897431in}}{\pgfqpoint{6.564406in}{4.905331in}}{\pgfqpoint{6.564406in}{4.913568in}}%
\pgfpathcurveto{\pgfqpoint{6.564406in}{4.921804in}}{\pgfqpoint{6.561134in}{4.929704in}}{\pgfqpoint{6.555310in}{4.935528in}}%
\pgfpathcurveto{\pgfqpoint{6.549486in}{4.941352in}}{\pgfqpoint{6.541586in}{4.944624in}}{\pgfqpoint{6.533350in}{4.944624in}}%
\pgfpathcurveto{\pgfqpoint{6.525114in}{4.944624in}}{\pgfqpoint{6.517214in}{4.941352in}}{\pgfqpoint{6.511390in}{4.935528in}}%
\pgfpathcurveto{\pgfqpoint{6.505566in}{4.929704in}}{\pgfqpoint{6.502294in}{4.921804in}}{\pgfqpoint{6.502294in}{4.913568in}}%
\pgfpathcurveto{\pgfqpoint{6.502294in}{4.905331in}}{\pgfqpoint{6.505566in}{4.897431in}}{\pgfqpoint{6.511390in}{4.891607in}}%
\pgfpathcurveto{\pgfqpoint{6.517214in}{4.885784in}}{\pgfqpoint{6.525114in}{4.882511in}}{\pgfqpoint{6.533350in}{4.882511in}}%
\pgfpathclose%
\pgfusepath{stroke,fill}%
\end{pgfscope}%
\begin{pgfscope}%
\pgfpathrectangle{\pgfqpoint{0.894063in}{3.540000in}}{\pgfqpoint{6.713438in}{2.060556in}} %
\pgfusepath{clip}%
\pgfsetbuttcap%
\pgfsetroundjoin%
\definecolor{currentfill}{rgb}{0.000000,0.000000,1.000000}%
\pgfsetfillcolor{currentfill}%
\pgfsetlinewidth{1.003750pt}%
\definecolor{currentstroke}{rgb}{0.000000,0.000000,0.000000}%
\pgfsetstrokecolor{currentstroke}%
\pgfsetdash{}{0pt}%
\pgfpathmoveto{\pgfqpoint{1.296869in}{5.142420in}}%
\pgfpathcurveto{\pgfqpoint{1.305105in}{5.142420in}}{\pgfqpoint{1.313005in}{5.145692in}}{\pgfqpoint{1.318829in}{5.151516in}}%
\pgfpathcurveto{\pgfqpoint{1.324653in}{5.157340in}}{\pgfqpoint{1.327925in}{5.165240in}}{\pgfqpoint{1.327925in}{5.173477in}}%
\pgfpathcurveto{\pgfqpoint{1.327925in}{5.181713in}}{\pgfqpoint{1.324653in}{5.189613in}}{\pgfqpoint{1.318829in}{5.195437in}}%
\pgfpathcurveto{\pgfqpoint{1.313005in}{5.201261in}}{\pgfqpoint{1.305105in}{5.204533in}}{\pgfqpoint{1.296869in}{5.204533in}}%
\pgfpathcurveto{\pgfqpoint{1.288632in}{5.204533in}}{\pgfqpoint{1.280732in}{5.201261in}}{\pgfqpoint{1.274908in}{5.195437in}}%
\pgfpathcurveto{\pgfqpoint{1.269085in}{5.189613in}}{\pgfqpoint{1.265812in}{5.181713in}}{\pgfqpoint{1.265812in}{5.173477in}}%
\pgfpathcurveto{\pgfqpoint{1.265812in}{5.165240in}}{\pgfqpoint{1.269085in}{5.157340in}}{\pgfqpoint{1.274908in}{5.151516in}}%
\pgfpathcurveto{\pgfqpoint{1.280732in}{5.145692in}}{\pgfqpoint{1.288632in}{5.142420in}}{\pgfqpoint{1.296869in}{5.142420in}}%
\pgfpathclose%
\pgfusepath{stroke,fill}%
\end{pgfscope}%
\begin{pgfscope}%
\pgfpathrectangle{\pgfqpoint{0.894063in}{3.540000in}}{\pgfqpoint{6.713438in}{2.060556in}} %
\pgfusepath{clip}%
\pgfsetbuttcap%
\pgfsetroundjoin%
\definecolor{currentfill}{rgb}{0.000000,0.000000,1.000000}%
\pgfsetfillcolor{currentfill}%
\pgfsetlinewidth{1.003750pt}%
\definecolor{currentstroke}{rgb}{0.000000,0.000000,0.000000}%
\pgfsetstrokecolor{currentstroke}%
\pgfsetdash{}{0pt}%
\pgfpathmoveto{\pgfqpoint{4.519319in}{5.126938in}}%
\pgfpathcurveto{\pgfqpoint{4.527555in}{5.126938in}}{\pgfqpoint{4.535455in}{5.130211in}}{\pgfqpoint{4.541279in}{5.136035in}}%
\pgfpathcurveto{\pgfqpoint{4.547103in}{5.141859in}}{\pgfqpoint{4.550375in}{5.149759in}}{\pgfqpoint{4.550375in}{5.157995in}}%
\pgfpathcurveto{\pgfqpoint{4.550375in}{5.166231in}}{\pgfqpoint{4.547103in}{5.174131in}}{\pgfqpoint{4.541279in}{5.179955in}}%
\pgfpathcurveto{\pgfqpoint{4.535455in}{5.185779in}}{\pgfqpoint{4.527555in}{5.189051in}}{\pgfqpoint{4.519319in}{5.189051in}}%
\pgfpathcurveto{\pgfqpoint{4.511082in}{5.189051in}}{\pgfqpoint{4.503182in}{5.185779in}}{\pgfqpoint{4.497358in}{5.179955in}}%
\pgfpathcurveto{\pgfqpoint{4.491535in}{5.174131in}}{\pgfqpoint{4.488262in}{5.166231in}}{\pgfqpoint{4.488262in}{5.157995in}}%
\pgfpathcurveto{\pgfqpoint{4.488262in}{5.149759in}}{\pgfqpoint{4.491535in}{5.141859in}}{\pgfqpoint{4.497358in}{5.136035in}}%
\pgfpathcurveto{\pgfqpoint{4.503182in}{5.130211in}}{\pgfqpoint{4.511082in}{5.126938in}}{\pgfqpoint{4.519319in}{5.126938in}}%
\pgfpathclose%
\pgfusepath{stroke,fill}%
\end{pgfscope}%
\begin{pgfscope}%
\pgfpathrectangle{\pgfqpoint{0.894063in}{3.540000in}}{\pgfqpoint{6.713438in}{2.060556in}} %
\pgfusepath{clip}%
\pgfsetbuttcap%
\pgfsetroundjoin%
\definecolor{currentfill}{rgb}{0.000000,0.000000,1.000000}%
\pgfsetfillcolor{currentfill}%
\pgfsetlinewidth{1.003750pt}%
\definecolor{currentstroke}{rgb}{0.000000,0.000000,0.000000}%
\pgfsetstrokecolor{currentstroke}%
\pgfsetdash{}{0pt}%
\pgfpathmoveto{\pgfqpoint{2.505288in}{5.126980in}}%
\pgfpathcurveto{\pgfqpoint{2.513524in}{5.126980in}}{\pgfqpoint{2.521424in}{5.130252in}}{\pgfqpoint{2.527248in}{5.136076in}}%
\pgfpathcurveto{\pgfqpoint{2.533072in}{5.141900in}}{\pgfqpoint{2.536344in}{5.149800in}}{\pgfqpoint{2.536344in}{5.158036in}}%
\pgfpathcurveto{\pgfqpoint{2.536344in}{5.166272in}}{\pgfqpoint{2.533072in}{5.174172in}}{\pgfqpoint{2.527248in}{5.179996in}}%
\pgfpathcurveto{\pgfqpoint{2.521424in}{5.185820in}}{\pgfqpoint{2.513524in}{5.189093in}}{\pgfqpoint{2.505288in}{5.189093in}}%
\pgfpathcurveto{\pgfqpoint{2.497051in}{5.189093in}}{\pgfqpoint{2.489151in}{5.185820in}}{\pgfqpoint{2.483327in}{5.179996in}}%
\pgfpathcurveto{\pgfqpoint{2.477503in}{5.174172in}}{\pgfqpoint{2.474231in}{5.166272in}}{\pgfqpoint{2.474231in}{5.158036in}}%
\pgfpathcurveto{\pgfqpoint{2.474231in}{5.149800in}}{\pgfqpoint{2.477503in}{5.141900in}}{\pgfqpoint{2.483327in}{5.136076in}}%
\pgfpathcurveto{\pgfqpoint{2.489151in}{5.130252in}}{\pgfqpoint{2.497051in}{5.126980in}}{\pgfqpoint{2.505288in}{5.126980in}}%
\pgfpathclose%
\pgfusepath{stroke,fill}%
\end{pgfscope}%
\begin{pgfscope}%
\pgfpathrectangle{\pgfqpoint{0.894063in}{3.540000in}}{\pgfqpoint{6.713438in}{2.060556in}} %
\pgfusepath{clip}%
\pgfsetbuttcap%
\pgfsetroundjoin%
\definecolor{currentfill}{rgb}{0.000000,0.000000,1.000000}%
\pgfsetfillcolor{currentfill}%
\pgfsetlinewidth{1.003750pt}%
\definecolor{currentstroke}{rgb}{0.000000,0.000000,0.000000}%
\pgfsetstrokecolor{currentstroke}%
\pgfsetdash{}{0pt}%
\pgfpathmoveto{\pgfqpoint{5.459200in}{5.048666in}}%
\pgfpathcurveto{\pgfqpoint{5.467436in}{5.048666in}}{\pgfqpoint{5.475336in}{5.051938in}}{\pgfqpoint{5.481160in}{5.057762in}}%
\pgfpathcurveto{\pgfqpoint{5.486984in}{5.063586in}}{\pgfqpoint{5.490256in}{5.071486in}}{\pgfqpoint{5.490256in}{5.079723in}}%
\pgfpathcurveto{\pgfqpoint{5.490256in}{5.087959in}}{\pgfqpoint{5.486984in}{5.095859in}}{\pgfqpoint{5.481160in}{5.101683in}}%
\pgfpathcurveto{\pgfqpoint{5.475336in}{5.107507in}}{\pgfqpoint{5.467436in}{5.110779in}}{\pgfqpoint{5.459200in}{5.110779in}}%
\pgfpathcurveto{\pgfqpoint{5.450964in}{5.110779in}}{\pgfqpoint{5.443064in}{5.107507in}}{\pgfqpoint{5.437240in}{5.101683in}}%
\pgfpathcurveto{\pgfqpoint{5.431416in}{5.095859in}}{\pgfqpoint{5.428144in}{5.087959in}}{\pgfqpoint{5.428144in}{5.079723in}}%
\pgfpathcurveto{\pgfqpoint{5.428144in}{5.071486in}}{\pgfqpoint{5.431416in}{5.063586in}}{\pgfqpoint{5.437240in}{5.057762in}}%
\pgfpathcurveto{\pgfqpoint{5.443064in}{5.051938in}}{\pgfqpoint{5.450964in}{5.048666in}}{\pgfqpoint{5.459200in}{5.048666in}}%
\pgfpathclose%
\pgfusepath{stroke,fill}%
\end{pgfscope}%
\begin{pgfscope}%
\pgfpathrectangle{\pgfqpoint{0.894063in}{3.540000in}}{\pgfqpoint{6.713438in}{2.060556in}} %
\pgfusepath{clip}%
\pgfsetbuttcap%
\pgfsetroundjoin%
\definecolor{currentfill}{rgb}{0.000000,0.000000,1.000000}%
\pgfsetfillcolor{currentfill}%
\pgfsetlinewidth{1.003750pt}%
\definecolor{currentstroke}{rgb}{0.000000,0.000000,0.000000}%
\pgfsetstrokecolor{currentstroke}%
\pgfsetdash{}{0pt}%
\pgfpathmoveto{\pgfqpoint{6.936156in}{4.824000in}}%
\pgfpathcurveto{\pgfqpoint{6.944393in}{4.824000in}}{\pgfqpoint{6.952293in}{4.827272in}}{\pgfqpoint{6.958117in}{4.833096in}}%
\pgfpathcurveto{\pgfqpoint{6.963940in}{4.838920in}}{\pgfqpoint{6.967213in}{4.846820in}}{\pgfqpoint{6.967213in}{4.855056in}}%
\pgfpathcurveto{\pgfqpoint{6.967213in}{4.863292in}}{\pgfqpoint{6.963940in}{4.871193in}}{\pgfqpoint{6.958117in}{4.877016in}}%
\pgfpathcurveto{\pgfqpoint{6.952293in}{4.882840in}}{\pgfqpoint{6.944393in}{4.886113in}}{\pgfqpoint{6.936156in}{4.886113in}}%
\pgfpathcurveto{\pgfqpoint{6.927920in}{4.886113in}}{\pgfqpoint{6.920020in}{4.882840in}}{\pgfqpoint{6.914196in}{4.877016in}}%
\pgfpathcurveto{\pgfqpoint{6.908372in}{4.871193in}}{\pgfqpoint{6.905100in}{4.863292in}}{\pgfqpoint{6.905100in}{4.855056in}}%
\pgfpathcurveto{\pgfqpoint{6.905100in}{4.846820in}}{\pgfqpoint{6.908372in}{4.838920in}}{\pgfqpoint{6.914196in}{4.833096in}}%
\pgfpathcurveto{\pgfqpoint{6.920020in}{4.827272in}}{\pgfqpoint{6.927920in}{4.824000in}}{\pgfqpoint{6.936156in}{4.824000in}}%
\pgfpathclose%
\pgfusepath{stroke,fill}%
\end{pgfscope}%
\begin{pgfscope}%
\pgfpathrectangle{\pgfqpoint{0.894063in}{3.540000in}}{\pgfqpoint{6.713438in}{2.060556in}} %
\pgfusepath{clip}%
\pgfsetbuttcap%
\pgfsetroundjoin%
\definecolor{currentfill}{rgb}{0.000000,0.000000,1.000000}%
\pgfsetfillcolor{currentfill}%
\pgfsetlinewidth{1.003750pt}%
\definecolor{currentstroke}{rgb}{0.000000,0.000000,0.000000}%
\pgfsetstrokecolor{currentstroke}%
\pgfsetdash{}{0pt}%
\pgfpathmoveto{\pgfqpoint{5.862006in}{4.977959in}}%
\pgfpathcurveto{\pgfqpoint{5.870243in}{4.977959in}}{\pgfqpoint{5.878143in}{4.981231in}}{\pgfqpoint{5.883967in}{4.987055in}}%
\pgfpathcurveto{\pgfqpoint{5.889790in}{4.992879in}}{\pgfqpoint{5.893063in}{5.000779in}}{\pgfqpoint{5.893063in}{5.009015in}}%
\pgfpathcurveto{\pgfqpoint{5.893063in}{5.017252in}}{\pgfqpoint{5.889790in}{5.025152in}}{\pgfqpoint{5.883967in}{5.030976in}}%
\pgfpathcurveto{\pgfqpoint{5.878143in}{5.036800in}}{\pgfqpoint{5.870243in}{5.040072in}}{\pgfqpoint{5.862006in}{5.040072in}}%
\pgfpathcurveto{\pgfqpoint{5.853770in}{5.040072in}}{\pgfqpoint{5.845870in}{5.036800in}}{\pgfqpoint{5.840046in}{5.030976in}}%
\pgfpathcurveto{\pgfqpoint{5.834222in}{5.025152in}}{\pgfqpoint{5.830950in}{5.017252in}}{\pgfqpoint{5.830950in}{5.009015in}}%
\pgfpathcurveto{\pgfqpoint{5.830950in}{5.000779in}}{\pgfqpoint{5.834222in}{4.992879in}}{\pgfqpoint{5.840046in}{4.987055in}}%
\pgfpathcurveto{\pgfqpoint{5.845870in}{4.981231in}}{\pgfqpoint{5.853770in}{4.977959in}}{\pgfqpoint{5.862006in}{4.977959in}}%
\pgfpathclose%
\pgfusepath{stroke,fill}%
\end{pgfscope}%
\begin{pgfscope}%
\pgfpathrectangle{\pgfqpoint{0.894063in}{3.540000in}}{\pgfqpoint{6.713438in}{2.060556in}} %
\pgfusepath{clip}%
\pgfsetbuttcap%
\pgfsetroundjoin%
\definecolor{currentfill}{rgb}{0.000000,0.000000,1.000000}%
\pgfsetfillcolor{currentfill}%
\pgfsetlinewidth{1.003750pt}%
\definecolor{currentstroke}{rgb}{0.000000,0.000000,0.000000}%
\pgfsetstrokecolor{currentstroke}%
\pgfsetdash{}{0pt}%
\pgfpathmoveto{\pgfqpoint{7.070425in}{4.804795in}}%
\pgfpathcurveto{\pgfqpoint{7.078661in}{4.804795in}}{\pgfqpoint{7.086561in}{4.808068in}}{\pgfqpoint{7.092385in}{4.813892in}}%
\pgfpathcurveto{\pgfqpoint{7.098209in}{4.819715in}}{\pgfqpoint{7.101481in}{4.827616in}}{\pgfqpoint{7.101481in}{4.835852in}}%
\pgfpathcurveto{\pgfqpoint{7.101481in}{4.844088in}}{\pgfqpoint{7.098209in}{4.851988in}}{\pgfqpoint{7.092385in}{4.857812in}}%
\pgfpathcurveto{\pgfqpoint{7.086561in}{4.863636in}}{\pgfqpoint{7.078661in}{4.866908in}}{\pgfqpoint{7.070425in}{4.866908in}}%
\pgfpathcurveto{\pgfqpoint{7.062189in}{4.866908in}}{\pgfqpoint{7.054289in}{4.863636in}}{\pgfqpoint{7.048465in}{4.857812in}}%
\pgfpathcurveto{\pgfqpoint{7.042641in}{4.851988in}}{\pgfqpoint{7.039369in}{4.844088in}}{\pgfqpoint{7.039369in}{4.835852in}}%
\pgfpathcurveto{\pgfqpoint{7.039369in}{4.827616in}}{\pgfqpoint{7.042641in}{4.819715in}}{\pgfqpoint{7.048465in}{4.813892in}}%
\pgfpathcurveto{\pgfqpoint{7.054289in}{4.808068in}}{\pgfqpoint{7.062189in}{4.804795in}}{\pgfqpoint{7.070425in}{4.804795in}}%
\pgfpathclose%
\pgfusepath{stroke,fill}%
\end{pgfscope}%
\begin{pgfscope}%
\pgfpathrectangle{\pgfqpoint{0.894063in}{3.540000in}}{\pgfqpoint{6.713438in}{2.060556in}} %
\pgfusepath{clip}%
\pgfsetbuttcap%
\pgfsetroundjoin%
\definecolor{currentfill}{rgb}{0.000000,0.000000,1.000000}%
\pgfsetfillcolor{currentfill}%
\pgfsetlinewidth{1.003750pt}%
\definecolor{currentstroke}{rgb}{0.000000,0.000000,0.000000}%
\pgfsetstrokecolor{currentstroke}%
\pgfsetdash{}{0pt}%
\pgfpathmoveto{\pgfqpoint{3.176631in}{5.126971in}}%
\pgfpathcurveto{\pgfqpoint{3.184868in}{5.126971in}}{\pgfqpoint{3.192768in}{5.130244in}}{\pgfqpoint{3.198592in}{5.136068in}}%
\pgfpathcurveto{\pgfqpoint{3.204415in}{5.141892in}}{\pgfqpoint{3.207688in}{5.149792in}}{\pgfqpoint{3.207688in}{5.158028in}}%
\pgfpathcurveto{\pgfqpoint{3.207688in}{5.166264in}}{\pgfqpoint{3.204415in}{5.174164in}}{\pgfqpoint{3.198592in}{5.179988in}}%
\pgfpathcurveto{\pgfqpoint{3.192768in}{5.185812in}}{\pgfqpoint{3.184868in}{5.189084in}}{\pgfqpoint{3.176631in}{5.189084in}}%
\pgfpathcurveto{\pgfqpoint{3.168395in}{5.189084in}}{\pgfqpoint{3.160495in}{5.185812in}}{\pgfqpoint{3.154671in}{5.179988in}}%
\pgfpathcurveto{\pgfqpoint{3.148847in}{5.174164in}}{\pgfqpoint{3.145575in}{5.166264in}}{\pgfqpoint{3.145575in}{5.158028in}}%
\pgfpathcurveto{\pgfqpoint{3.145575in}{5.149792in}}{\pgfqpoint{3.148847in}{5.141892in}}{\pgfqpoint{3.154671in}{5.136068in}}%
\pgfpathcurveto{\pgfqpoint{3.160495in}{5.130244in}}{\pgfqpoint{3.168395in}{5.126971in}}{\pgfqpoint{3.176631in}{5.126971in}}%
\pgfpathclose%
\pgfusepath{stroke,fill}%
\end{pgfscope}%
\begin{pgfscope}%
\pgfpathrectangle{\pgfqpoint{0.894063in}{3.540000in}}{\pgfqpoint{6.713438in}{2.060556in}} %
\pgfusepath{clip}%
\pgfsetbuttcap%
\pgfsetroundjoin%
\definecolor{currentfill}{rgb}{0.000000,0.000000,1.000000}%
\pgfsetfillcolor{currentfill}%
\pgfsetlinewidth{1.003750pt}%
\definecolor{currentstroke}{rgb}{0.000000,0.000000,0.000000}%
\pgfsetstrokecolor{currentstroke}%
\pgfsetdash{}{0pt}%
\pgfpathmoveto{\pgfqpoint{2.102481in}{5.127017in}}%
\pgfpathcurveto{\pgfqpoint{2.110718in}{5.127017in}}{\pgfqpoint{2.118618in}{5.130289in}}{\pgfqpoint{2.124442in}{5.136113in}}%
\pgfpathcurveto{\pgfqpoint{2.130265in}{5.141937in}}{\pgfqpoint{2.133538in}{5.149837in}}{\pgfqpoint{2.133538in}{5.158073in}}%
\pgfpathcurveto{\pgfqpoint{2.133538in}{5.166310in}}{\pgfqpoint{2.130265in}{5.174210in}}{\pgfqpoint{2.124442in}{5.180033in}}%
\pgfpathcurveto{\pgfqpoint{2.118618in}{5.185857in}}{\pgfqpoint{2.110718in}{5.189130in}}{\pgfqpoint{2.102481in}{5.189130in}}%
\pgfpathcurveto{\pgfqpoint{2.094245in}{5.189130in}}{\pgfqpoint{2.086345in}{5.185857in}}{\pgfqpoint{2.080521in}{5.180033in}}%
\pgfpathcurveto{\pgfqpoint{2.074697in}{5.174210in}}{\pgfqpoint{2.071425in}{5.166310in}}{\pgfqpoint{2.071425in}{5.158073in}}%
\pgfpathcurveto{\pgfqpoint{2.071425in}{5.149837in}}{\pgfqpoint{2.074697in}{5.141937in}}{\pgfqpoint{2.080521in}{5.136113in}}%
\pgfpathcurveto{\pgfqpoint{2.086345in}{5.130289in}}{\pgfqpoint{2.094245in}{5.127017in}}{\pgfqpoint{2.102481in}{5.127017in}}%
\pgfpathclose%
\pgfusepath{stroke,fill}%
\end{pgfscope}%
\begin{pgfscope}%
\pgfpathrectangle{\pgfqpoint{0.894063in}{3.540000in}}{\pgfqpoint{6.713438in}{2.060556in}} %
\pgfusepath{clip}%
\pgfsetbuttcap%
\pgfsetroundjoin%
\definecolor{currentfill}{rgb}{0.000000,0.000000,1.000000}%
\pgfsetfillcolor{currentfill}%
\pgfsetlinewidth{1.003750pt}%
\definecolor{currentstroke}{rgb}{0.000000,0.000000,0.000000}%
\pgfsetstrokecolor{currentstroke}%
\pgfsetdash{}{0pt}%
\pgfpathmoveto{\pgfqpoint{1.968213in}{5.127150in}}%
\pgfpathcurveto{\pgfqpoint{1.976449in}{5.127150in}}{\pgfqpoint{1.984349in}{5.130422in}}{\pgfqpoint{1.990173in}{5.136246in}}%
\pgfpathcurveto{\pgfqpoint{1.995997in}{5.142070in}}{\pgfqpoint{1.999269in}{5.149970in}}{\pgfqpoint{1.999269in}{5.158206in}}%
\pgfpathcurveto{\pgfqpoint{1.999269in}{5.166443in}}{\pgfqpoint{1.995997in}{5.174343in}}{\pgfqpoint{1.990173in}{5.180167in}}%
\pgfpathcurveto{\pgfqpoint{1.984349in}{5.185991in}}{\pgfqpoint{1.976449in}{5.189263in}}{\pgfqpoint{1.968213in}{5.189263in}}%
\pgfpathcurveto{\pgfqpoint{1.959976in}{5.189263in}}{\pgfqpoint{1.952076in}{5.185991in}}{\pgfqpoint{1.946252in}{5.180167in}}%
\pgfpathcurveto{\pgfqpoint{1.940428in}{5.174343in}}{\pgfqpoint{1.937156in}{5.166443in}}{\pgfqpoint{1.937156in}{5.158206in}}%
\pgfpathcurveto{\pgfqpoint{1.937156in}{5.149970in}}{\pgfqpoint{1.940428in}{5.142070in}}{\pgfqpoint{1.946252in}{5.136246in}}%
\pgfpathcurveto{\pgfqpoint{1.952076in}{5.130422in}}{\pgfqpoint{1.959976in}{5.127150in}}{\pgfqpoint{1.968213in}{5.127150in}}%
\pgfpathclose%
\pgfusepath{stroke,fill}%
\end{pgfscope}%
\begin{pgfscope}%
\pgfpathrectangle{\pgfqpoint{0.894063in}{3.540000in}}{\pgfqpoint{6.713438in}{2.060556in}} %
\pgfusepath{clip}%
\pgfsetbuttcap%
\pgfsetroundjoin%
\definecolor{currentfill}{rgb}{0.000000,0.000000,1.000000}%
\pgfsetfillcolor{currentfill}%
\pgfsetlinewidth{1.003750pt}%
\definecolor{currentstroke}{rgb}{0.000000,0.000000,0.000000}%
\pgfsetstrokecolor{currentstroke}%
\pgfsetdash{}{0pt}%
\pgfpathmoveto{\pgfqpoint{3.310900in}{5.126971in}}%
\pgfpathcurveto{\pgfqpoint{3.319136in}{5.126971in}}{\pgfqpoint{3.327036in}{5.130244in}}{\pgfqpoint{3.332860in}{5.136068in}}%
\pgfpathcurveto{\pgfqpoint{3.338684in}{5.141892in}}{\pgfqpoint{3.341956in}{5.149792in}}{\pgfqpoint{3.341956in}{5.158028in}}%
\pgfpathcurveto{\pgfqpoint{3.341956in}{5.166264in}}{\pgfqpoint{3.338684in}{5.174164in}}{\pgfqpoint{3.332860in}{5.179988in}}%
\pgfpathcurveto{\pgfqpoint{3.327036in}{5.185812in}}{\pgfqpoint{3.319136in}{5.189084in}}{\pgfqpoint{3.310900in}{5.189084in}}%
\pgfpathcurveto{\pgfqpoint{3.302664in}{5.189084in}}{\pgfqpoint{3.294764in}{5.185812in}}{\pgfqpoint{3.288940in}{5.179988in}}%
\pgfpathcurveto{\pgfqpoint{3.283116in}{5.174164in}}{\pgfqpoint{3.279844in}{5.166264in}}{\pgfqpoint{3.279844in}{5.158028in}}%
\pgfpathcurveto{\pgfqpoint{3.279844in}{5.149792in}}{\pgfqpoint{3.283116in}{5.141892in}}{\pgfqpoint{3.288940in}{5.136068in}}%
\pgfpathcurveto{\pgfqpoint{3.294764in}{5.130244in}}{\pgfqpoint{3.302664in}{5.126971in}}{\pgfqpoint{3.310900in}{5.126971in}}%
\pgfpathclose%
\pgfusepath{stroke,fill}%
\end{pgfscope}%
\begin{pgfscope}%
\pgfpathrectangle{\pgfqpoint{0.894063in}{3.540000in}}{\pgfqpoint{6.713438in}{2.060556in}} %
\pgfusepath{clip}%
\pgfsetbuttcap%
\pgfsetroundjoin%
\definecolor{currentfill}{rgb}{0.000000,0.000000,1.000000}%
\pgfsetfillcolor{currentfill}%
\pgfsetlinewidth{1.003750pt}%
\definecolor{currentstroke}{rgb}{0.000000,0.000000,0.000000}%
\pgfsetstrokecolor{currentstroke}%
\pgfsetdash{}{0pt}%
\pgfpathmoveto{\pgfqpoint{5.593469in}{5.016054in}}%
\pgfpathcurveto{\pgfqpoint{5.601705in}{5.016054in}}{\pgfqpoint{5.609605in}{5.019327in}}{\pgfqpoint{5.615429in}{5.025151in}}%
\pgfpathcurveto{\pgfqpoint{5.621253in}{5.030975in}}{\pgfqpoint{5.624525in}{5.038875in}}{\pgfqpoint{5.624525in}{5.047111in}}%
\pgfpathcurveto{\pgfqpoint{5.624525in}{5.055347in}}{\pgfqpoint{5.621253in}{5.063247in}}{\pgfqpoint{5.615429in}{5.069071in}}%
\pgfpathcurveto{\pgfqpoint{5.609605in}{5.074895in}}{\pgfqpoint{5.601705in}{5.078167in}}{\pgfqpoint{5.593469in}{5.078167in}}%
\pgfpathcurveto{\pgfqpoint{5.585232in}{5.078167in}}{\pgfqpoint{5.577332in}{5.074895in}}{\pgfqpoint{5.571508in}{5.069071in}}%
\pgfpathcurveto{\pgfqpoint{5.565685in}{5.063247in}}{\pgfqpoint{5.562412in}{5.055347in}}{\pgfqpoint{5.562412in}{5.047111in}}%
\pgfpathcurveto{\pgfqpoint{5.562412in}{5.038875in}}{\pgfqpoint{5.565685in}{5.030975in}}{\pgfqpoint{5.571508in}{5.025151in}}%
\pgfpathcurveto{\pgfqpoint{5.577332in}{5.019327in}}{\pgfqpoint{5.585232in}{5.016054in}}{\pgfqpoint{5.593469in}{5.016054in}}%
\pgfpathclose%
\pgfusepath{stroke,fill}%
\end{pgfscope}%
\begin{pgfscope}%
\pgfpathrectangle{\pgfqpoint{0.894063in}{3.540000in}}{\pgfqpoint{6.713438in}{2.060556in}} %
\pgfusepath{clip}%
\pgfsetbuttcap%
\pgfsetroundjoin%
\definecolor{currentfill}{rgb}{0.000000,0.000000,1.000000}%
\pgfsetfillcolor{currentfill}%
\pgfsetlinewidth{1.003750pt}%
\definecolor{currentstroke}{rgb}{0.000000,0.000000,0.000000}%
\pgfsetstrokecolor{currentstroke}%
\pgfsetdash{}{0pt}%
\pgfpathmoveto{\pgfqpoint{3.042363in}{5.126971in}}%
\pgfpathcurveto{\pgfqpoint{3.050599in}{5.126971in}}{\pgfqpoint{3.058499in}{5.130244in}}{\pgfqpoint{3.064323in}{5.136068in}}%
\pgfpathcurveto{\pgfqpoint{3.070147in}{5.141892in}}{\pgfqpoint{3.073419in}{5.149792in}}{\pgfqpoint{3.073419in}{5.158028in}}%
\pgfpathcurveto{\pgfqpoint{3.073419in}{5.166264in}}{\pgfqpoint{3.070147in}{5.174164in}}{\pgfqpoint{3.064323in}{5.179988in}}%
\pgfpathcurveto{\pgfqpoint{3.058499in}{5.185812in}}{\pgfqpoint{3.050599in}{5.189084in}}{\pgfqpoint{3.042363in}{5.189084in}}%
\pgfpathcurveto{\pgfqpoint{3.034126in}{5.189084in}}{\pgfqpoint{3.026226in}{5.185812in}}{\pgfqpoint{3.020402in}{5.179988in}}%
\pgfpathcurveto{\pgfqpoint{3.014578in}{5.174164in}}{\pgfqpoint{3.011306in}{5.166264in}}{\pgfqpoint{3.011306in}{5.158028in}}%
\pgfpathcurveto{\pgfqpoint{3.011306in}{5.149792in}}{\pgfqpoint{3.014578in}{5.141892in}}{\pgfqpoint{3.020402in}{5.136068in}}%
\pgfpathcurveto{\pgfqpoint{3.026226in}{5.130244in}}{\pgfqpoint{3.034126in}{5.126971in}}{\pgfqpoint{3.042363in}{5.126971in}}%
\pgfpathclose%
\pgfusepath{stroke,fill}%
\end{pgfscope}%
\begin{pgfscope}%
\pgfpathrectangle{\pgfqpoint{0.894063in}{3.540000in}}{\pgfqpoint{6.713438in}{2.060556in}} %
\pgfusepath{clip}%
\pgfsetbuttcap%
\pgfsetroundjoin%
\definecolor{currentfill}{rgb}{0.000000,0.000000,1.000000}%
\pgfsetfillcolor{currentfill}%
\pgfsetlinewidth{1.003750pt}%
\definecolor{currentstroke}{rgb}{0.000000,0.000000,0.000000}%
\pgfsetstrokecolor{currentstroke}%
\pgfsetdash{}{0pt}%
\pgfpathmoveto{\pgfqpoint{5.190663in}{5.075824in}}%
\pgfpathcurveto{\pgfqpoint{5.198899in}{5.075824in}}{\pgfqpoint{5.206799in}{5.079097in}}{\pgfqpoint{5.212623in}{5.084921in}}%
\pgfpathcurveto{\pgfqpoint{5.218447in}{5.090744in}}{\pgfqpoint{5.221719in}{5.098645in}}{\pgfqpoint{5.221719in}{5.106881in}}%
\pgfpathcurveto{\pgfqpoint{5.221719in}{5.115117in}}{\pgfqpoint{5.218447in}{5.123017in}}{\pgfqpoint{5.212623in}{5.128841in}}%
\pgfpathcurveto{\pgfqpoint{5.206799in}{5.134665in}}{\pgfqpoint{5.198899in}{5.137937in}}{\pgfqpoint{5.190663in}{5.137937in}}%
\pgfpathcurveto{\pgfqpoint{5.182426in}{5.137937in}}{\pgfqpoint{5.174526in}{5.134665in}}{\pgfqpoint{5.168702in}{5.128841in}}%
\pgfpathcurveto{\pgfqpoint{5.162878in}{5.123017in}}{\pgfqpoint{5.159606in}{5.115117in}}{\pgfqpoint{5.159606in}{5.106881in}}%
\pgfpathcurveto{\pgfqpoint{5.159606in}{5.098645in}}{\pgfqpoint{5.162878in}{5.090744in}}{\pgfqpoint{5.168702in}{5.084921in}}%
\pgfpathcurveto{\pgfqpoint{5.174526in}{5.079097in}}{\pgfqpoint{5.182426in}{5.075824in}}{\pgfqpoint{5.190663in}{5.075824in}}%
\pgfpathclose%
\pgfusepath{stroke,fill}%
\end{pgfscope}%
\begin{pgfscope}%
\pgfpathrectangle{\pgfqpoint{0.894063in}{3.540000in}}{\pgfqpoint{6.713438in}{2.060556in}} %
\pgfusepath{clip}%
\pgfsetbuttcap%
\pgfsetroundjoin%
\definecolor{currentfill}{rgb}{0.000000,0.000000,1.000000}%
\pgfsetfillcolor{currentfill}%
\pgfsetlinewidth{1.003750pt}%
\definecolor{currentstroke}{rgb}{0.000000,0.000000,0.000000}%
\pgfsetstrokecolor{currentstroke}%
\pgfsetdash{}{0pt}%
\pgfpathmoveto{\pgfqpoint{6.801888in}{4.842870in}}%
\pgfpathcurveto{\pgfqpoint{6.810124in}{4.842870in}}{\pgfqpoint{6.818024in}{4.846143in}}{\pgfqpoint{6.823848in}{4.851966in}}%
\pgfpathcurveto{\pgfqpoint{6.829672in}{4.857790in}}{\pgfqpoint{6.832944in}{4.865690in}}{\pgfqpoint{6.832944in}{4.873927in}}%
\pgfpathcurveto{\pgfqpoint{6.832944in}{4.882163in}}{\pgfqpoint{6.829672in}{4.890063in}}{\pgfqpoint{6.823848in}{4.895887in}}%
\pgfpathcurveto{\pgfqpoint{6.818024in}{4.901711in}}{\pgfqpoint{6.810124in}{4.904983in}}{\pgfqpoint{6.801888in}{4.904983in}}%
\pgfpathcurveto{\pgfqpoint{6.793651in}{4.904983in}}{\pgfqpoint{6.785751in}{4.901711in}}{\pgfqpoint{6.779927in}{4.895887in}}%
\pgfpathcurveto{\pgfqpoint{6.774103in}{4.890063in}}{\pgfqpoint{6.770831in}{4.882163in}}{\pgfqpoint{6.770831in}{4.873927in}}%
\pgfpathcurveto{\pgfqpoint{6.770831in}{4.865690in}}{\pgfqpoint{6.774103in}{4.857790in}}{\pgfqpoint{6.779927in}{4.851966in}}%
\pgfpathcurveto{\pgfqpoint{6.785751in}{4.846143in}}{\pgfqpoint{6.793651in}{4.842870in}}{\pgfqpoint{6.801888in}{4.842870in}}%
\pgfpathclose%
\pgfusepath{stroke,fill}%
\end{pgfscope}%
\begin{pgfscope}%
\pgfpathrectangle{\pgfqpoint{0.894063in}{3.540000in}}{\pgfqpoint{6.713438in}{2.060556in}} %
\pgfusepath{clip}%
\pgfsetbuttcap%
\pgfsetroundjoin%
\definecolor{currentfill}{rgb}{0.000000,0.000000,1.000000}%
\pgfsetfillcolor{currentfill}%
\pgfsetlinewidth{1.003750pt}%
\definecolor{currentstroke}{rgb}{0.000000,0.000000,0.000000}%
\pgfsetstrokecolor{currentstroke}%
\pgfsetdash{}{0pt}%
\pgfpathmoveto{\pgfqpoint{3.579438in}{5.126955in}}%
\pgfpathcurveto{\pgfqpoint{3.587674in}{5.126955in}}{\pgfqpoint{3.595574in}{5.130227in}}{\pgfqpoint{3.601398in}{5.136051in}}%
\pgfpathcurveto{\pgfqpoint{3.607222in}{5.141875in}}{\pgfqpoint{3.610494in}{5.149775in}}{\pgfqpoint{3.610494in}{5.158011in}}%
\pgfpathcurveto{\pgfqpoint{3.610494in}{5.166248in}}{\pgfqpoint{3.607222in}{5.174148in}}{\pgfqpoint{3.601398in}{5.179972in}}%
\pgfpathcurveto{\pgfqpoint{3.595574in}{5.185796in}}{\pgfqpoint{3.587674in}{5.189068in}}{\pgfqpoint{3.579438in}{5.189068in}}%
\pgfpathcurveto{\pgfqpoint{3.571201in}{5.189068in}}{\pgfqpoint{3.563301in}{5.185796in}}{\pgfqpoint{3.557477in}{5.179972in}}%
\pgfpathcurveto{\pgfqpoint{3.551653in}{5.174148in}}{\pgfqpoint{3.548381in}{5.166248in}}{\pgfqpoint{3.548381in}{5.158011in}}%
\pgfpathcurveto{\pgfqpoint{3.548381in}{5.149775in}}{\pgfqpoint{3.551653in}{5.141875in}}{\pgfqpoint{3.557477in}{5.136051in}}%
\pgfpathcurveto{\pgfqpoint{3.563301in}{5.130227in}}{\pgfqpoint{3.571201in}{5.126955in}}{\pgfqpoint{3.579438in}{5.126955in}}%
\pgfpathclose%
\pgfusepath{stroke,fill}%
\end{pgfscope}%
\begin{pgfscope}%
\pgfpathrectangle{\pgfqpoint{0.894063in}{3.540000in}}{\pgfqpoint{6.713438in}{2.060556in}} %
\pgfusepath{clip}%
\pgfsetbuttcap%
\pgfsetroundjoin%
\definecolor{currentfill}{rgb}{0.000000,0.000000,1.000000}%
\pgfsetfillcolor{currentfill}%
\pgfsetlinewidth{1.003750pt}%
\definecolor{currentstroke}{rgb}{0.000000,0.000000,0.000000}%
\pgfsetstrokecolor{currentstroke}%
\pgfsetdash{}{0pt}%
\pgfpathmoveto{\pgfqpoint{2.371019in}{5.127014in}}%
\pgfpathcurveto{\pgfqpoint{2.379255in}{5.127014in}}{\pgfqpoint{2.387155in}{5.130286in}}{\pgfqpoint{2.392979in}{5.136110in}}%
\pgfpathcurveto{\pgfqpoint{2.398803in}{5.141934in}}{\pgfqpoint{2.402075in}{5.149834in}}{\pgfqpoint{2.402075in}{5.158070in}}%
\pgfpathcurveto{\pgfqpoint{2.402075in}{5.166307in}}{\pgfqpoint{2.398803in}{5.174207in}}{\pgfqpoint{2.392979in}{5.180031in}}%
\pgfpathcurveto{\pgfqpoint{2.387155in}{5.185855in}}{\pgfqpoint{2.379255in}{5.189127in}}{\pgfqpoint{2.371019in}{5.189127in}}%
\pgfpathcurveto{\pgfqpoint{2.362782in}{5.189127in}}{\pgfqpoint{2.354882in}{5.185855in}}{\pgfqpoint{2.349058in}{5.180031in}}%
\pgfpathcurveto{\pgfqpoint{2.343235in}{5.174207in}}{\pgfqpoint{2.339962in}{5.166307in}}{\pgfqpoint{2.339962in}{5.158070in}}%
\pgfpathcurveto{\pgfqpoint{2.339962in}{5.149834in}}{\pgfqpoint{2.343235in}{5.141934in}}{\pgfqpoint{2.349058in}{5.136110in}}%
\pgfpathcurveto{\pgfqpoint{2.354882in}{5.130286in}}{\pgfqpoint{2.362782in}{5.127014in}}{\pgfqpoint{2.371019in}{5.127014in}}%
\pgfpathclose%
\pgfusepath{stroke,fill}%
\end{pgfscope}%
\begin{pgfscope}%
\pgfpathrectangle{\pgfqpoint{0.894063in}{3.540000in}}{\pgfqpoint{6.713438in}{2.060556in}} %
\pgfusepath{clip}%
\pgfsetbuttcap%
\pgfsetroundjoin%
\definecolor{currentfill}{rgb}{0.000000,0.000000,1.000000}%
\pgfsetfillcolor{currentfill}%
\pgfsetlinewidth{1.003750pt}%
\definecolor{currentstroke}{rgb}{0.000000,0.000000,0.000000}%
\pgfsetstrokecolor{currentstroke}%
\pgfsetdash{}{0pt}%
\pgfpathmoveto{\pgfqpoint{3.982244in}{5.126948in}}%
\pgfpathcurveto{\pgfqpoint{3.990480in}{5.126948in}}{\pgfqpoint{3.998380in}{5.130220in}}{\pgfqpoint{4.004204in}{5.136044in}}%
\pgfpathcurveto{\pgfqpoint{4.010028in}{5.141868in}}{\pgfqpoint{4.013300in}{5.149768in}}{\pgfqpoint{4.013300in}{5.158005in}}%
\pgfpathcurveto{\pgfqpoint{4.013300in}{5.166241in}}{\pgfqpoint{4.010028in}{5.174141in}}{\pgfqpoint{4.004204in}{5.179965in}}%
\pgfpathcurveto{\pgfqpoint{3.998380in}{5.185789in}}{\pgfqpoint{3.990480in}{5.189061in}}{\pgfqpoint{3.982244in}{5.189061in}}%
\pgfpathcurveto{\pgfqpoint{3.974007in}{5.189061in}}{\pgfqpoint{3.966107in}{5.185789in}}{\pgfqpoint{3.960283in}{5.179965in}}%
\pgfpathcurveto{\pgfqpoint{3.954460in}{5.174141in}}{\pgfqpoint{3.951187in}{5.166241in}}{\pgfqpoint{3.951187in}{5.158005in}}%
\pgfpathcurveto{\pgfqpoint{3.951187in}{5.149768in}}{\pgfqpoint{3.954460in}{5.141868in}}{\pgfqpoint{3.960283in}{5.136044in}}%
\pgfpathcurveto{\pgfqpoint{3.966107in}{5.130220in}}{\pgfqpoint{3.974007in}{5.126948in}}{\pgfqpoint{3.982244in}{5.126948in}}%
\pgfpathclose%
\pgfusepath{stroke,fill}%
\end{pgfscope}%
\begin{pgfscope}%
\pgfpathrectangle{\pgfqpoint{0.894063in}{3.540000in}}{\pgfqpoint{6.713438in}{2.060556in}} %
\pgfusepath{clip}%
\pgfsetbuttcap%
\pgfsetroundjoin%
\definecolor{currentfill}{rgb}{0.000000,0.000000,1.000000}%
\pgfsetfillcolor{currentfill}%
\pgfsetlinewidth{1.003750pt}%
\definecolor{currentstroke}{rgb}{0.000000,0.000000,0.000000}%
\pgfsetstrokecolor{currentstroke}%
\pgfsetdash{}{0pt}%
\pgfpathmoveto{\pgfqpoint{4.653588in}{5.126938in}}%
\pgfpathcurveto{\pgfqpoint{4.661824in}{5.126938in}}{\pgfqpoint{4.669724in}{5.130211in}}{\pgfqpoint{4.675548in}{5.136035in}}%
\pgfpathcurveto{\pgfqpoint{4.681372in}{5.141859in}}{\pgfqpoint{4.684644in}{5.149759in}}{\pgfqpoint{4.684644in}{5.157995in}}%
\pgfpathcurveto{\pgfqpoint{4.684644in}{5.166231in}}{\pgfqpoint{4.681372in}{5.174131in}}{\pgfqpoint{4.675548in}{5.179955in}}%
\pgfpathcurveto{\pgfqpoint{4.669724in}{5.185779in}}{\pgfqpoint{4.661824in}{5.189051in}}{\pgfqpoint{4.653588in}{5.189051in}}%
\pgfpathcurveto{\pgfqpoint{4.645351in}{5.189051in}}{\pgfqpoint{4.637451in}{5.185779in}}{\pgfqpoint{4.631627in}{5.179955in}}%
\pgfpathcurveto{\pgfqpoint{4.625803in}{5.174131in}}{\pgfqpoint{4.622531in}{5.166231in}}{\pgfqpoint{4.622531in}{5.157995in}}%
\pgfpathcurveto{\pgfqpoint{4.622531in}{5.149759in}}{\pgfqpoint{4.625803in}{5.141859in}}{\pgfqpoint{4.631627in}{5.136035in}}%
\pgfpathcurveto{\pgfqpoint{4.637451in}{5.130211in}}{\pgfqpoint{4.645351in}{5.126938in}}{\pgfqpoint{4.653588in}{5.126938in}}%
\pgfpathclose%
\pgfusepath{stroke,fill}%
\end{pgfscope}%
\begin{pgfscope}%
\pgfpathrectangle{\pgfqpoint{0.894063in}{3.540000in}}{\pgfqpoint{6.713438in}{2.060556in}} %
\pgfusepath{clip}%
\pgfsetbuttcap%
\pgfsetroundjoin%
\definecolor{currentfill}{rgb}{0.000000,0.000000,1.000000}%
\pgfsetfillcolor{currentfill}%
\pgfsetlinewidth{1.003750pt}%
\definecolor{currentstroke}{rgb}{0.000000,0.000000,0.000000}%
\pgfsetstrokecolor{currentstroke}%
\pgfsetdash{}{0pt}%
\pgfpathmoveto{\pgfqpoint{3.713706in}{5.126955in}}%
\pgfpathcurveto{\pgfqpoint{3.721943in}{5.126955in}}{\pgfqpoint{3.729843in}{5.130227in}}{\pgfqpoint{3.735667in}{5.136051in}}%
\pgfpathcurveto{\pgfqpoint{3.741490in}{5.141875in}}{\pgfqpoint{3.744763in}{5.149775in}}{\pgfqpoint{3.744763in}{5.158011in}}%
\pgfpathcurveto{\pgfqpoint{3.744763in}{5.166248in}}{\pgfqpoint{3.741490in}{5.174148in}}{\pgfqpoint{3.735667in}{5.179972in}}%
\pgfpathcurveto{\pgfqpoint{3.729843in}{5.185796in}}{\pgfqpoint{3.721943in}{5.189068in}}{\pgfqpoint{3.713706in}{5.189068in}}%
\pgfpathcurveto{\pgfqpoint{3.705470in}{5.189068in}}{\pgfqpoint{3.697570in}{5.185796in}}{\pgfqpoint{3.691746in}{5.179972in}}%
\pgfpathcurveto{\pgfqpoint{3.685922in}{5.174148in}}{\pgfqpoint{3.682650in}{5.166248in}}{\pgfqpoint{3.682650in}{5.158011in}}%
\pgfpathcurveto{\pgfqpoint{3.682650in}{5.149775in}}{\pgfqpoint{3.685922in}{5.141875in}}{\pgfqpoint{3.691746in}{5.136051in}}%
\pgfpathcurveto{\pgfqpoint{3.697570in}{5.130227in}}{\pgfqpoint{3.705470in}{5.126955in}}{\pgfqpoint{3.713706in}{5.126955in}}%
\pgfpathclose%
\pgfusepath{stroke,fill}%
\end{pgfscope}%
\begin{pgfscope}%
\pgfpathrectangle{\pgfqpoint{0.894063in}{3.540000in}}{\pgfqpoint{6.713438in}{2.060556in}} %
\pgfusepath{clip}%
\pgfsetbuttcap%
\pgfsetroundjoin%
\definecolor{currentfill}{rgb}{0.000000,0.000000,1.000000}%
\pgfsetfillcolor{currentfill}%
\pgfsetlinewidth{1.003750pt}%
\definecolor{currentstroke}{rgb}{0.000000,0.000000,0.000000}%
\pgfsetstrokecolor{currentstroke}%
\pgfsetdash{}{0pt}%
\pgfpathmoveto{\pgfqpoint{2.236750in}{5.127014in}}%
\pgfpathcurveto{\pgfqpoint{2.244986in}{5.127014in}}{\pgfqpoint{2.252886in}{5.130286in}}{\pgfqpoint{2.258710in}{5.136110in}}%
\pgfpathcurveto{\pgfqpoint{2.264534in}{5.141934in}}{\pgfqpoint{2.267806in}{5.149834in}}{\pgfqpoint{2.267806in}{5.158070in}}%
\pgfpathcurveto{\pgfqpoint{2.267806in}{5.166307in}}{\pgfqpoint{2.264534in}{5.174207in}}{\pgfqpoint{2.258710in}{5.180031in}}%
\pgfpathcurveto{\pgfqpoint{2.252886in}{5.185855in}}{\pgfqpoint{2.244986in}{5.189127in}}{\pgfqpoint{2.236750in}{5.189127in}}%
\pgfpathcurveto{\pgfqpoint{2.228514in}{5.189127in}}{\pgfqpoint{2.220614in}{5.185855in}}{\pgfqpoint{2.214790in}{5.180031in}}%
\pgfpathcurveto{\pgfqpoint{2.208966in}{5.174207in}}{\pgfqpoint{2.205694in}{5.166307in}}{\pgfqpoint{2.205694in}{5.158070in}}%
\pgfpathcurveto{\pgfqpoint{2.205694in}{5.149834in}}{\pgfqpoint{2.208966in}{5.141934in}}{\pgfqpoint{2.214790in}{5.136110in}}%
\pgfpathcurveto{\pgfqpoint{2.220614in}{5.130286in}}{\pgfqpoint{2.228514in}{5.127014in}}{\pgfqpoint{2.236750in}{5.127014in}}%
\pgfpathclose%
\pgfusepath{stroke,fill}%
\end{pgfscope}%
\begin{pgfscope}%
\pgfpathrectangle{\pgfqpoint{0.894063in}{3.540000in}}{\pgfqpoint{6.713438in}{2.060556in}} %
\pgfusepath{clip}%
\pgfsetbuttcap%
\pgfsetroundjoin%
\definecolor{currentfill}{rgb}{0.000000,0.000000,1.000000}%
\pgfsetfillcolor{currentfill}%
\pgfsetlinewidth{1.003750pt}%
\definecolor{currentstroke}{rgb}{0.000000,0.000000,0.000000}%
\pgfsetstrokecolor{currentstroke}%
\pgfsetdash{}{0pt}%
\pgfpathmoveto{\pgfqpoint{6.667619in}{4.862419in}}%
\pgfpathcurveto{\pgfqpoint{6.675855in}{4.862419in}}{\pgfqpoint{6.683755in}{4.865692in}}{\pgfqpoint{6.689579in}{4.871516in}}%
\pgfpathcurveto{\pgfqpoint{6.695403in}{4.877340in}}{\pgfqpoint{6.698675in}{4.885240in}}{\pgfqpoint{6.698675in}{4.893476in}}%
\pgfpathcurveto{\pgfqpoint{6.698675in}{4.901712in}}{\pgfqpoint{6.695403in}{4.909612in}}{\pgfqpoint{6.689579in}{4.915436in}}%
\pgfpathcurveto{\pgfqpoint{6.683755in}{4.921260in}}{\pgfqpoint{6.675855in}{4.924532in}}{\pgfqpoint{6.667619in}{4.924532in}}%
\pgfpathcurveto{\pgfqpoint{6.659382in}{4.924532in}}{\pgfqpoint{6.651482in}{4.921260in}}{\pgfqpoint{6.645658in}{4.915436in}}%
\pgfpathcurveto{\pgfqpoint{6.639835in}{4.909612in}}{\pgfqpoint{6.636562in}{4.901712in}}{\pgfqpoint{6.636562in}{4.893476in}}%
\pgfpathcurveto{\pgfqpoint{6.636562in}{4.885240in}}{\pgfqpoint{6.639835in}{4.877340in}}{\pgfqpoint{6.645658in}{4.871516in}}%
\pgfpathcurveto{\pgfqpoint{6.651482in}{4.865692in}}{\pgfqpoint{6.659382in}{4.862419in}}{\pgfqpoint{6.667619in}{4.862419in}}%
\pgfpathclose%
\pgfusepath{stroke,fill}%
\end{pgfscope}%
\begin{pgfscope}%
\pgfpathrectangle{\pgfqpoint{0.894063in}{3.540000in}}{\pgfqpoint{6.713438in}{2.060556in}} %
\pgfusepath{clip}%
\pgfsetbuttcap%
\pgfsetroundjoin%
\definecolor{currentfill}{rgb}{0.000000,0.000000,1.000000}%
\pgfsetfillcolor{currentfill}%
\pgfsetlinewidth{1.003750pt}%
\definecolor{currentstroke}{rgb}{0.000000,0.000000,0.000000}%
\pgfsetstrokecolor{currentstroke}%
\pgfsetdash{}{0pt}%
\pgfpathmoveto{\pgfqpoint{2.639556in}{5.126978in}}%
\pgfpathcurveto{\pgfqpoint{2.647793in}{5.126978in}}{\pgfqpoint{2.655693in}{5.130251in}}{\pgfqpoint{2.661517in}{5.136075in}}%
\pgfpathcurveto{\pgfqpoint{2.667340in}{5.141898in}}{\pgfqpoint{2.670613in}{5.149798in}}{\pgfqpoint{2.670613in}{5.158035in}}%
\pgfpathcurveto{\pgfqpoint{2.670613in}{5.166271in}}{\pgfqpoint{2.667340in}{5.174171in}}{\pgfqpoint{2.661517in}{5.179995in}}%
\pgfpathcurveto{\pgfqpoint{2.655693in}{5.185819in}}{\pgfqpoint{2.647793in}{5.189091in}}{\pgfqpoint{2.639556in}{5.189091in}}%
\pgfpathcurveto{\pgfqpoint{2.631320in}{5.189091in}}{\pgfqpoint{2.623420in}{5.185819in}}{\pgfqpoint{2.617596in}{5.179995in}}%
\pgfpathcurveto{\pgfqpoint{2.611772in}{5.174171in}}{\pgfqpoint{2.608500in}{5.166271in}}{\pgfqpoint{2.608500in}{5.158035in}}%
\pgfpathcurveto{\pgfqpoint{2.608500in}{5.149798in}}{\pgfqpoint{2.611772in}{5.141898in}}{\pgfqpoint{2.617596in}{5.136075in}}%
\pgfpathcurveto{\pgfqpoint{2.623420in}{5.130251in}}{\pgfqpoint{2.631320in}{5.126978in}}{\pgfqpoint{2.639556in}{5.126978in}}%
\pgfpathclose%
\pgfusepath{stroke,fill}%
\end{pgfscope}%
\begin{pgfscope}%
\pgfpathrectangle{\pgfqpoint{0.894063in}{3.540000in}}{\pgfqpoint{6.713438in}{2.060556in}} %
\pgfusepath{clip}%
\pgfsetbuttcap%
\pgfsetroundjoin%
\definecolor{currentfill}{rgb}{0.000000,0.000000,1.000000}%
\pgfsetfillcolor{currentfill}%
\pgfsetlinewidth{1.003750pt}%
\definecolor{currentstroke}{rgb}{0.000000,0.000000,0.000000}%
\pgfsetstrokecolor{currentstroke}%
\pgfsetdash{}{0pt}%
\pgfpathmoveto{\pgfqpoint{1.699675in}{5.130702in}}%
\pgfpathcurveto{\pgfqpoint{1.707911in}{5.130702in}}{\pgfqpoint{1.715811in}{5.133975in}}{\pgfqpoint{1.721635in}{5.139799in}}%
\pgfpathcurveto{\pgfqpoint{1.727459in}{5.145623in}}{\pgfqpoint{1.730731in}{5.153523in}}{\pgfqpoint{1.730731in}{5.161759in}}%
\pgfpathcurveto{\pgfqpoint{1.730731in}{5.169995in}}{\pgfqpoint{1.727459in}{5.177895in}}{\pgfqpoint{1.721635in}{5.183719in}}%
\pgfpathcurveto{\pgfqpoint{1.715811in}{5.189543in}}{\pgfqpoint{1.707911in}{5.192815in}}{\pgfqpoint{1.699675in}{5.192815in}}%
\pgfpathcurveto{\pgfqpoint{1.691439in}{5.192815in}}{\pgfqpoint{1.683539in}{5.189543in}}{\pgfqpoint{1.677715in}{5.183719in}}%
\pgfpathcurveto{\pgfqpoint{1.671891in}{5.177895in}}{\pgfqpoint{1.668619in}{5.169995in}}{\pgfqpoint{1.668619in}{5.161759in}}%
\pgfpathcurveto{\pgfqpoint{1.668619in}{5.153523in}}{\pgfqpoint{1.671891in}{5.145623in}}{\pgfqpoint{1.677715in}{5.139799in}}%
\pgfpathcurveto{\pgfqpoint{1.683539in}{5.133975in}}{\pgfqpoint{1.691439in}{5.130702in}}{\pgfqpoint{1.699675in}{5.130702in}}%
\pgfpathclose%
\pgfusepath{stroke,fill}%
\end{pgfscope}%
\begin{pgfscope}%
\pgfpathrectangle{\pgfqpoint{0.894063in}{3.540000in}}{\pgfqpoint{6.713438in}{2.060556in}} %
\pgfusepath{clip}%
\pgfsetbuttcap%
\pgfsetroundjoin%
\definecolor{currentfill}{rgb}{0.000000,0.000000,1.000000}%
\pgfsetfillcolor{currentfill}%
\pgfsetlinewidth{1.003750pt}%
\definecolor{currentstroke}{rgb}{0.000000,0.000000,0.000000}%
\pgfsetstrokecolor{currentstroke}%
\pgfsetdash{}{0pt}%
\pgfpathmoveto{\pgfqpoint{1.162600in}{5.264724in}}%
\pgfpathcurveto{\pgfqpoint{1.170836in}{5.264724in}}{\pgfqpoint{1.178736in}{5.267996in}}{\pgfqpoint{1.184560in}{5.273820in}}%
\pgfpathcurveto{\pgfqpoint{1.190384in}{5.279644in}}{\pgfqpoint{1.193656in}{5.287544in}}{\pgfqpoint{1.193656in}{5.295780in}}%
\pgfpathcurveto{\pgfqpoint{1.193656in}{5.304016in}}{\pgfqpoint{1.190384in}{5.311916in}}{\pgfqpoint{1.184560in}{5.317740in}}%
\pgfpathcurveto{\pgfqpoint{1.178736in}{5.323564in}}{\pgfqpoint{1.170836in}{5.326837in}}{\pgfqpoint{1.162600in}{5.326837in}}%
\pgfpathcurveto{\pgfqpoint{1.154364in}{5.326837in}}{\pgfqpoint{1.146464in}{5.323564in}}{\pgfqpoint{1.140640in}{5.317740in}}%
\pgfpathcurveto{\pgfqpoint{1.134816in}{5.311916in}}{\pgfqpoint{1.131544in}{5.304016in}}{\pgfqpoint{1.131544in}{5.295780in}}%
\pgfpathcurveto{\pgfqpoint{1.131544in}{5.287544in}}{\pgfqpoint{1.134816in}{5.279644in}}{\pgfqpoint{1.140640in}{5.273820in}}%
\pgfpathcurveto{\pgfqpoint{1.146464in}{5.267996in}}{\pgfqpoint{1.154364in}{5.264724in}}{\pgfqpoint{1.162600in}{5.264724in}}%
\pgfpathclose%
\pgfusepath{stroke,fill}%
\end{pgfscope}%
\begin{pgfscope}%
\pgfpathrectangle{\pgfqpoint{0.894063in}{3.540000in}}{\pgfqpoint{6.713438in}{2.060556in}} %
\pgfusepath{clip}%
\pgfsetbuttcap%
\pgfsetroundjoin%
\definecolor{currentfill}{rgb}{0.000000,0.000000,1.000000}%
\pgfsetfillcolor{currentfill}%
\pgfsetlinewidth{1.003750pt}%
\definecolor{currentstroke}{rgb}{0.000000,0.000000,0.000000}%
\pgfsetstrokecolor{currentstroke}%
\pgfsetdash{}{0pt}%
\pgfpathmoveto{\pgfqpoint{1.833944in}{5.127874in}}%
\pgfpathcurveto{\pgfqpoint{1.842180in}{5.127874in}}{\pgfqpoint{1.850080in}{5.131146in}}{\pgfqpoint{1.855904in}{5.136970in}}%
\pgfpathcurveto{\pgfqpoint{1.861728in}{5.142794in}}{\pgfqpoint{1.865000in}{5.150694in}}{\pgfqpoint{1.865000in}{5.158930in}}%
\pgfpathcurveto{\pgfqpoint{1.865000in}{5.167167in}}{\pgfqpoint{1.861728in}{5.175067in}}{\pgfqpoint{1.855904in}{5.180891in}}%
\pgfpathcurveto{\pgfqpoint{1.850080in}{5.186715in}}{\pgfqpoint{1.842180in}{5.189987in}}{\pgfqpoint{1.833944in}{5.189987in}}%
\pgfpathcurveto{\pgfqpoint{1.825707in}{5.189987in}}{\pgfqpoint{1.817807in}{5.186715in}}{\pgfqpoint{1.811983in}{5.180891in}}%
\pgfpathcurveto{\pgfqpoint{1.806160in}{5.175067in}}{\pgfqpoint{1.802887in}{5.167167in}}{\pgfqpoint{1.802887in}{5.158930in}}%
\pgfpathcurveto{\pgfqpoint{1.802887in}{5.150694in}}{\pgfqpoint{1.806160in}{5.142794in}}{\pgfqpoint{1.811983in}{5.136970in}}%
\pgfpathcurveto{\pgfqpoint{1.817807in}{5.131146in}}{\pgfqpoint{1.825707in}{5.127874in}}{\pgfqpoint{1.833944in}{5.127874in}}%
\pgfpathclose%
\pgfusepath{stroke,fill}%
\end{pgfscope}%
\begin{pgfscope}%
\pgfpathrectangle{\pgfqpoint{0.894063in}{3.540000in}}{\pgfqpoint{6.713438in}{2.060556in}} %
\pgfusepath{clip}%
\pgfsetbuttcap%
\pgfsetroundjoin%
\definecolor{currentfill}{rgb}{0.000000,0.000000,1.000000}%
\pgfsetfillcolor{currentfill}%
\pgfsetlinewidth{1.003750pt}%
\definecolor{currentstroke}{rgb}{0.000000,0.000000,0.000000}%
\pgfsetstrokecolor{currentstroke}%
\pgfsetdash{}{0pt}%
\pgfpathmoveto{\pgfqpoint{5.996275in}{4.978113in}}%
\pgfpathcurveto{\pgfqpoint{6.004511in}{4.978113in}}{\pgfqpoint{6.012411in}{4.981385in}}{\pgfqpoint{6.018235in}{4.987209in}}%
\pgfpathcurveto{\pgfqpoint{6.024059in}{4.993033in}}{\pgfqpoint{6.027331in}{5.000933in}}{\pgfqpoint{6.027331in}{5.009169in}}%
\pgfpathcurveto{\pgfqpoint{6.027331in}{5.017406in}}{\pgfqpoint{6.024059in}{5.025306in}}{\pgfqpoint{6.018235in}{5.031130in}}%
\pgfpathcurveto{\pgfqpoint{6.012411in}{5.036953in}}{\pgfqpoint{6.004511in}{5.040226in}}{\pgfqpoint{5.996275in}{5.040226in}}%
\pgfpathcurveto{\pgfqpoint{5.988039in}{5.040226in}}{\pgfqpoint{5.980139in}{5.036953in}}{\pgfqpoint{5.974315in}{5.031130in}}%
\pgfpathcurveto{\pgfqpoint{5.968491in}{5.025306in}}{\pgfqpoint{5.965219in}{5.017406in}}{\pgfqpoint{5.965219in}{5.009169in}}%
\pgfpathcurveto{\pgfqpoint{5.965219in}{5.000933in}}{\pgfqpoint{5.968491in}{4.993033in}}{\pgfqpoint{5.974315in}{4.987209in}}%
\pgfpathcurveto{\pgfqpoint{5.980139in}{4.981385in}}{\pgfqpoint{5.988039in}{4.978113in}}{\pgfqpoint{5.996275in}{4.978113in}}%
\pgfpathclose%
\pgfusepath{stroke,fill}%
\end{pgfscope}%
\begin{pgfscope}%
\pgfpathrectangle{\pgfqpoint{0.894063in}{3.540000in}}{\pgfqpoint{6.713438in}{2.060556in}} %
\pgfusepath{clip}%
\pgfsetbuttcap%
\pgfsetroundjoin%
\definecolor{currentfill}{rgb}{0.000000,0.000000,1.000000}%
\pgfsetfillcolor{currentfill}%
\pgfsetlinewidth{1.003750pt}%
\definecolor{currentstroke}{rgb}{0.000000,0.000000,0.000000}%
\pgfsetstrokecolor{currentstroke}%
\pgfsetdash{}{0pt}%
\pgfpathmoveto{\pgfqpoint{6.399081in}{4.901037in}}%
\pgfpathcurveto{\pgfqpoint{6.407318in}{4.901037in}}{\pgfqpoint{6.415218in}{4.904309in}}{\pgfqpoint{6.421042in}{4.910133in}}%
\pgfpathcurveto{\pgfqpoint{6.426865in}{4.915957in}}{\pgfqpoint{6.430138in}{4.923857in}}{\pgfqpoint{6.430138in}{4.932093in}}%
\pgfpathcurveto{\pgfqpoint{6.430138in}{4.940330in}}{\pgfqpoint{6.426865in}{4.948230in}}{\pgfqpoint{6.421042in}{4.954054in}}%
\pgfpathcurveto{\pgfqpoint{6.415218in}{4.959878in}}{\pgfqpoint{6.407318in}{4.963150in}}{\pgfqpoint{6.399081in}{4.963150in}}%
\pgfpathcurveto{\pgfqpoint{6.390845in}{4.963150in}}{\pgfqpoint{6.382945in}{4.959878in}}{\pgfqpoint{6.377121in}{4.954054in}}%
\pgfpathcurveto{\pgfqpoint{6.371297in}{4.948230in}}{\pgfqpoint{6.368025in}{4.940330in}}{\pgfqpoint{6.368025in}{4.932093in}}%
\pgfpathcurveto{\pgfqpoint{6.368025in}{4.923857in}}{\pgfqpoint{6.371297in}{4.915957in}}{\pgfqpoint{6.377121in}{4.910133in}}%
\pgfpathcurveto{\pgfqpoint{6.382945in}{4.904309in}}{\pgfqpoint{6.390845in}{4.901037in}}{\pgfqpoint{6.399081in}{4.901037in}}%
\pgfpathclose%
\pgfusepath{stroke,fill}%
\end{pgfscope}%
\begin{pgfscope}%
\pgfpathrectangle{\pgfqpoint{0.894063in}{3.540000in}}{\pgfqpoint{6.713438in}{2.060556in}} %
\pgfusepath{clip}%
\pgfsetbuttcap%
\pgfsetroundjoin%
\definecolor{currentfill}{rgb}{0.000000,0.000000,1.000000}%
\pgfsetfillcolor{currentfill}%
\pgfsetlinewidth{1.003750pt}%
\definecolor{currentstroke}{rgb}{0.000000,0.000000,0.000000}%
\pgfsetstrokecolor{currentstroke}%
\pgfsetdash{}{0pt}%
\pgfpathmoveto{\pgfqpoint{4.787856in}{5.126925in}}%
\pgfpathcurveto{\pgfqpoint{4.796093in}{5.126925in}}{\pgfqpoint{4.803993in}{5.130197in}}{\pgfqpoint{4.809817in}{5.136021in}}%
\pgfpathcurveto{\pgfqpoint{4.815640in}{5.141845in}}{\pgfqpoint{4.818913in}{5.149745in}}{\pgfqpoint{4.818913in}{5.157981in}}%
\pgfpathcurveto{\pgfqpoint{4.818913in}{5.166217in}}{\pgfqpoint{4.815640in}{5.174118in}}{\pgfqpoint{4.809817in}{5.179941in}}%
\pgfpathcurveto{\pgfqpoint{4.803993in}{5.185765in}}{\pgfqpoint{4.796093in}{5.189038in}}{\pgfqpoint{4.787856in}{5.189038in}}%
\pgfpathcurveto{\pgfqpoint{4.779620in}{5.189038in}}{\pgfqpoint{4.771720in}{5.185765in}}{\pgfqpoint{4.765896in}{5.179941in}}%
\pgfpathcurveto{\pgfqpoint{4.760072in}{5.174118in}}{\pgfqpoint{4.756800in}{5.166217in}}{\pgfqpoint{4.756800in}{5.157981in}}%
\pgfpathcurveto{\pgfqpoint{4.756800in}{5.149745in}}{\pgfqpoint{4.760072in}{5.141845in}}{\pgfqpoint{4.765896in}{5.136021in}}%
\pgfpathcurveto{\pgfqpoint{4.771720in}{5.130197in}}{\pgfqpoint{4.779620in}{5.126925in}}{\pgfqpoint{4.787856in}{5.126925in}}%
\pgfpathclose%
\pgfusepath{stroke,fill}%
\end{pgfscope}%
\begin{pgfscope}%
\pgfpathrectangle{\pgfqpoint{0.894063in}{3.540000in}}{\pgfqpoint{6.713438in}{2.060556in}} %
\pgfusepath{clip}%
\pgfsetbuttcap%
\pgfsetroundjoin%
\definecolor{currentfill}{rgb}{0.000000,0.000000,1.000000}%
\pgfsetfillcolor{currentfill}%
\pgfsetlinewidth{1.003750pt}%
\definecolor{currentstroke}{rgb}{0.000000,0.000000,0.000000}%
\pgfsetstrokecolor{currentstroke}%
\pgfsetdash{}{0pt}%
\pgfpathmoveto{\pgfqpoint{4.922125in}{5.123224in}}%
\pgfpathcurveto{\pgfqpoint{4.930361in}{5.123224in}}{\pgfqpoint{4.938261in}{5.126496in}}{\pgfqpoint{4.944085in}{5.132320in}}%
\pgfpathcurveto{\pgfqpoint{4.949909in}{5.138144in}}{\pgfqpoint{4.953181in}{5.146044in}}{\pgfqpoint{4.953181in}{5.154280in}}%
\pgfpathcurveto{\pgfqpoint{4.953181in}{5.162517in}}{\pgfqpoint{4.949909in}{5.170417in}}{\pgfqpoint{4.944085in}{5.176241in}}%
\pgfpathcurveto{\pgfqpoint{4.938261in}{5.182065in}}{\pgfqpoint{4.930361in}{5.185337in}}{\pgfqpoint{4.922125in}{5.185337in}}%
\pgfpathcurveto{\pgfqpoint{4.913889in}{5.185337in}}{\pgfqpoint{4.905989in}{5.182065in}}{\pgfqpoint{4.900165in}{5.176241in}}%
\pgfpathcurveto{\pgfqpoint{4.894341in}{5.170417in}}{\pgfqpoint{4.891069in}{5.162517in}}{\pgfqpoint{4.891069in}{5.154280in}}%
\pgfpathcurveto{\pgfqpoint{4.891069in}{5.146044in}}{\pgfqpoint{4.894341in}{5.138144in}}{\pgfqpoint{4.900165in}{5.132320in}}%
\pgfpathcurveto{\pgfqpoint{4.905989in}{5.126496in}}{\pgfqpoint{4.913889in}{5.123224in}}{\pgfqpoint{4.922125in}{5.123224in}}%
\pgfpathclose%
\pgfusepath{stroke,fill}%
\end{pgfscope}%
\begin{pgfscope}%
\pgfpathrectangle{\pgfqpoint{0.894063in}{3.540000in}}{\pgfqpoint{6.713438in}{2.060556in}} %
\pgfusepath{clip}%
\pgfsetbuttcap%
\pgfsetroundjoin%
\definecolor{currentfill}{rgb}{0.000000,0.000000,1.000000}%
\pgfsetfillcolor{currentfill}%
\pgfsetlinewidth{1.003750pt}%
\definecolor{currentstroke}{rgb}{0.000000,0.000000,0.000000}%
\pgfsetstrokecolor{currentstroke}%
\pgfsetdash{}{0pt}%
\pgfpathmoveto{\pgfqpoint{6.130544in}{4.950522in}}%
\pgfpathcurveto{\pgfqpoint{6.138780in}{4.950522in}}{\pgfqpoint{6.146680in}{4.953794in}}{\pgfqpoint{6.152504in}{4.959618in}}%
\pgfpathcurveto{\pgfqpoint{6.158328in}{4.965442in}}{\pgfqpoint{6.161600in}{4.973342in}}{\pgfqpoint{6.161600in}{4.981578in}}%
\pgfpathcurveto{\pgfqpoint{6.161600in}{4.989815in}}{\pgfqpoint{6.158328in}{4.997715in}}{\pgfqpoint{6.152504in}{5.003539in}}%
\pgfpathcurveto{\pgfqpoint{6.146680in}{5.009363in}}{\pgfqpoint{6.138780in}{5.012635in}}{\pgfqpoint{6.130544in}{5.012635in}}%
\pgfpathcurveto{\pgfqpoint{6.122307in}{5.012635in}}{\pgfqpoint{6.114407in}{5.009363in}}{\pgfqpoint{6.108583in}{5.003539in}}%
\pgfpathcurveto{\pgfqpoint{6.102760in}{4.997715in}}{\pgfqpoint{6.099487in}{4.989815in}}{\pgfqpoint{6.099487in}{4.981578in}}%
\pgfpathcurveto{\pgfqpoint{6.099487in}{4.973342in}}{\pgfqpoint{6.102760in}{4.965442in}}{\pgfqpoint{6.108583in}{4.959618in}}%
\pgfpathcurveto{\pgfqpoint{6.114407in}{4.953794in}}{\pgfqpoint{6.122307in}{4.950522in}}{\pgfqpoint{6.130544in}{4.950522in}}%
\pgfpathclose%
\pgfusepath{stroke,fill}%
\end{pgfscope}%
\begin{pgfscope}%
\pgfpathrectangle{\pgfqpoint{0.894063in}{3.540000in}}{\pgfqpoint{6.713438in}{2.060556in}} %
\pgfusepath{clip}%
\pgfsetbuttcap%
\pgfsetroundjoin%
\definecolor{currentfill}{rgb}{0.000000,0.000000,1.000000}%
\pgfsetfillcolor{currentfill}%
\pgfsetlinewidth{1.003750pt}%
\definecolor{currentstroke}{rgb}{0.000000,0.000000,0.000000}%
\pgfsetstrokecolor{currentstroke}%
\pgfsetdash{}{0pt}%
\pgfpathmoveto{\pgfqpoint{5.727738in}{5.013736in}}%
\pgfpathcurveto{\pgfqpoint{5.735974in}{5.013736in}}{\pgfqpoint{5.743874in}{5.017008in}}{\pgfqpoint{5.749698in}{5.022832in}}%
\pgfpathcurveto{\pgfqpoint{5.755522in}{5.028656in}}{\pgfqpoint{5.758794in}{5.036556in}}{\pgfqpoint{5.758794in}{5.044792in}}%
\pgfpathcurveto{\pgfqpoint{5.758794in}{5.053028in}}{\pgfqpoint{5.755522in}{5.060928in}}{\pgfqpoint{5.749698in}{5.066752in}}%
\pgfpathcurveto{\pgfqpoint{5.743874in}{5.072576in}}{\pgfqpoint{5.735974in}{5.075849in}}{\pgfqpoint{5.727738in}{5.075849in}}%
\pgfpathcurveto{\pgfqpoint{5.719501in}{5.075849in}}{\pgfqpoint{5.711601in}{5.072576in}}{\pgfqpoint{5.705777in}{5.066752in}}%
\pgfpathcurveto{\pgfqpoint{5.699953in}{5.060928in}}{\pgfqpoint{5.696681in}{5.053028in}}{\pgfqpoint{5.696681in}{5.044792in}}%
\pgfpathcurveto{\pgfqpoint{5.696681in}{5.036556in}}{\pgfqpoint{5.699953in}{5.028656in}}{\pgfqpoint{5.705777in}{5.022832in}}%
\pgfpathcurveto{\pgfqpoint{5.711601in}{5.017008in}}{\pgfqpoint{5.719501in}{5.013736in}}{\pgfqpoint{5.727738in}{5.013736in}}%
\pgfpathclose%
\pgfusepath{stroke,fill}%
\end{pgfscope}%
\begin{pgfscope}%
\pgfpathrectangle{\pgfqpoint{0.894063in}{3.540000in}}{\pgfqpoint{6.713438in}{2.060556in}} %
\pgfusepath{clip}%
\pgfsetbuttcap%
\pgfsetroundjoin%
\definecolor{currentfill}{rgb}{0.000000,0.000000,1.000000}%
\pgfsetfillcolor{currentfill}%
\pgfsetlinewidth{1.003750pt}%
\definecolor{currentstroke}{rgb}{0.000000,0.000000,0.000000}%
\pgfsetstrokecolor{currentstroke}%
\pgfsetdash{}{0pt}%
\pgfpathmoveto{\pgfqpoint{1.028331in}{5.264787in}}%
\pgfpathcurveto{\pgfqpoint{1.036568in}{5.264787in}}{\pgfqpoint{1.044468in}{5.268059in}}{\pgfqpoint{1.050292in}{5.273883in}}%
\pgfpathcurveto{\pgfqpoint{1.056115in}{5.279707in}}{\pgfqpoint{1.059388in}{5.287607in}}{\pgfqpoint{1.059388in}{5.295843in}}%
\pgfpathcurveto{\pgfqpoint{1.059388in}{5.304080in}}{\pgfqpoint{1.056115in}{5.311980in}}{\pgfqpoint{1.050292in}{5.317804in}}%
\pgfpathcurveto{\pgfqpoint{1.044468in}{5.323628in}}{\pgfqpoint{1.036568in}{5.326900in}}{\pgfqpoint{1.028331in}{5.326900in}}%
\pgfpathcurveto{\pgfqpoint{1.020095in}{5.326900in}}{\pgfqpoint{1.012195in}{5.323628in}}{\pgfqpoint{1.006371in}{5.317804in}}%
\pgfpathcurveto{\pgfqpoint{1.000547in}{5.311980in}}{\pgfqpoint{0.997275in}{5.304080in}}{\pgfqpoint{0.997275in}{5.295843in}}%
\pgfpathcurveto{\pgfqpoint{0.997275in}{5.287607in}}{\pgfqpoint{1.000547in}{5.279707in}}{\pgfqpoint{1.006371in}{5.273883in}}%
\pgfpathcurveto{\pgfqpoint{1.012195in}{5.268059in}}{\pgfqpoint{1.020095in}{5.264787in}}{\pgfqpoint{1.028331in}{5.264787in}}%
\pgfpathclose%
\pgfusepath{stroke,fill}%
\end{pgfscope}%
\begin{pgfscope}%
\pgfpathrectangle{\pgfqpoint{0.894063in}{3.540000in}}{\pgfqpoint{6.713438in}{2.060556in}} %
\pgfusepath{clip}%
\pgfsetbuttcap%
\pgfsetroundjoin%
\definecolor{currentfill}{rgb}{0.000000,0.000000,1.000000}%
\pgfsetfillcolor{currentfill}%
\pgfsetlinewidth{1.003750pt}%
\definecolor{currentstroke}{rgb}{0.000000,0.000000,0.000000}%
\pgfsetstrokecolor{currentstroke}%
\pgfsetdash{}{0pt}%
\pgfpathmoveto{\pgfqpoint{5.324931in}{5.075475in}}%
\pgfpathcurveto{\pgfqpoint{5.333168in}{5.075475in}}{\pgfqpoint{5.341068in}{5.078748in}}{\pgfqpoint{5.346892in}{5.084572in}}%
\pgfpathcurveto{\pgfqpoint{5.352715in}{5.090396in}}{\pgfqpoint{5.355988in}{5.098296in}}{\pgfqpoint{5.355988in}{5.106532in}}%
\pgfpathcurveto{\pgfqpoint{5.355988in}{5.114768in}}{\pgfqpoint{5.352715in}{5.122668in}}{\pgfqpoint{5.346892in}{5.128492in}}%
\pgfpathcurveto{\pgfqpoint{5.341068in}{5.134316in}}{\pgfqpoint{5.333168in}{5.137588in}}{\pgfqpoint{5.324931in}{5.137588in}}%
\pgfpathcurveto{\pgfqpoint{5.316695in}{5.137588in}}{\pgfqpoint{5.308795in}{5.134316in}}{\pgfqpoint{5.302971in}{5.128492in}}%
\pgfpathcurveto{\pgfqpoint{5.297147in}{5.122668in}}{\pgfqpoint{5.293875in}{5.114768in}}{\pgfqpoint{5.293875in}{5.106532in}}%
\pgfpathcurveto{\pgfqpoint{5.293875in}{5.098296in}}{\pgfqpoint{5.297147in}{5.090396in}}{\pgfqpoint{5.302971in}{5.084572in}}%
\pgfpathcurveto{\pgfqpoint{5.308795in}{5.078748in}}{\pgfqpoint{5.316695in}{5.075475in}}{\pgfqpoint{5.324931in}{5.075475in}}%
\pgfpathclose%
\pgfusepath{stroke,fill}%
\end{pgfscope}%
\begin{pgfscope}%
\pgfpathrectangle{\pgfqpoint{0.894063in}{3.540000in}}{\pgfqpoint{6.713438in}{2.060556in}} %
\pgfusepath{clip}%
\pgfsetbuttcap%
\pgfsetroundjoin%
\definecolor{currentfill}{rgb}{0.000000,0.000000,1.000000}%
\pgfsetfillcolor{currentfill}%
\pgfsetlinewidth{1.003750pt}%
\definecolor{currentstroke}{rgb}{0.000000,0.000000,0.000000}%
\pgfsetstrokecolor{currentstroke}%
\pgfsetdash{}{0pt}%
\pgfpathmoveto{\pgfqpoint{7.338963in}{4.758898in}}%
\pgfpathcurveto{\pgfqpoint{7.347199in}{4.758898in}}{\pgfqpoint{7.355099in}{4.762171in}}{\pgfqpoint{7.360923in}{4.767995in}}%
\pgfpathcurveto{\pgfqpoint{7.366747in}{4.773819in}}{\pgfqpoint{7.370019in}{4.781719in}}{\pgfqpoint{7.370019in}{4.789955in}}%
\pgfpathcurveto{\pgfqpoint{7.370019in}{4.798191in}}{\pgfqpoint{7.366747in}{4.806091in}}{\pgfqpoint{7.360923in}{4.811915in}}%
\pgfpathcurveto{\pgfqpoint{7.355099in}{4.817739in}}{\pgfqpoint{7.347199in}{4.821011in}}{\pgfqpoint{7.338963in}{4.821011in}}%
\pgfpathcurveto{\pgfqpoint{7.330726in}{4.821011in}}{\pgfqpoint{7.322826in}{4.817739in}}{\pgfqpoint{7.317002in}{4.811915in}}%
\pgfpathcurveto{\pgfqpoint{7.311178in}{4.806091in}}{\pgfqpoint{7.307906in}{4.798191in}}{\pgfqpoint{7.307906in}{4.789955in}}%
\pgfpathcurveto{\pgfqpoint{7.307906in}{4.781719in}}{\pgfqpoint{7.311178in}{4.773819in}}{\pgfqpoint{7.317002in}{4.767995in}}%
\pgfpathcurveto{\pgfqpoint{7.322826in}{4.762171in}}{\pgfqpoint{7.330726in}{4.758898in}}{\pgfqpoint{7.338963in}{4.758898in}}%
\pgfpathclose%
\pgfusepath{stroke,fill}%
\end{pgfscope}%
\begin{pgfscope}%
\pgfpathrectangle{\pgfqpoint{0.894063in}{3.540000in}}{\pgfqpoint{6.713438in}{2.060556in}} %
\pgfusepath{clip}%
\pgfsetbuttcap%
\pgfsetroundjoin%
\definecolor{currentfill}{rgb}{0.000000,0.000000,1.000000}%
\pgfsetfillcolor{currentfill}%
\pgfsetlinewidth{1.003750pt}%
\definecolor{currentstroke}{rgb}{0.000000,0.000000,0.000000}%
\pgfsetstrokecolor{currentstroke}%
\pgfsetdash{}{0pt}%
\pgfpathmoveto{\pgfqpoint{7.204694in}{4.804900in}}%
\pgfpathcurveto{\pgfqpoint{7.212930in}{4.804900in}}{\pgfqpoint{7.220830in}{4.808172in}}{\pgfqpoint{7.226654in}{4.813996in}}%
\pgfpathcurveto{\pgfqpoint{7.232478in}{4.819820in}}{\pgfqpoint{7.235750in}{4.827720in}}{\pgfqpoint{7.235750in}{4.835956in}}%
\pgfpathcurveto{\pgfqpoint{7.235750in}{4.844192in}}{\pgfqpoint{7.232478in}{4.852093in}}{\pgfqpoint{7.226654in}{4.857916in}}%
\pgfpathcurveto{\pgfqpoint{7.220830in}{4.863740in}}{\pgfqpoint{7.212930in}{4.867013in}}{\pgfqpoint{7.204694in}{4.867013in}}%
\pgfpathcurveto{\pgfqpoint{7.196457in}{4.867013in}}{\pgfqpoint{7.188557in}{4.863740in}}{\pgfqpoint{7.182733in}{4.857916in}}%
\pgfpathcurveto{\pgfqpoint{7.176910in}{4.852093in}}{\pgfqpoint{7.173637in}{4.844192in}}{\pgfqpoint{7.173637in}{4.835956in}}%
\pgfpathcurveto{\pgfqpoint{7.173637in}{4.827720in}}{\pgfqpoint{7.176910in}{4.819820in}}{\pgfqpoint{7.182733in}{4.813996in}}%
\pgfpathcurveto{\pgfqpoint{7.188557in}{4.808172in}}{\pgfqpoint{7.196457in}{4.804900in}}{\pgfqpoint{7.204694in}{4.804900in}}%
\pgfpathclose%
\pgfusepath{stroke,fill}%
\end{pgfscope}%
\begin{pgfscope}%
\pgfpathrectangle{\pgfqpoint{0.894063in}{3.540000in}}{\pgfqpoint{6.713438in}{2.060556in}} %
\pgfusepath{clip}%
\pgfsetbuttcap%
\pgfsetroundjoin%
\definecolor{currentfill}{rgb}{0.000000,0.000000,1.000000}%
\pgfsetfillcolor{currentfill}%
\pgfsetlinewidth{1.003750pt}%
\definecolor{currentstroke}{rgb}{0.000000,0.000000,0.000000}%
\pgfsetstrokecolor{currentstroke}%
\pgfsetdash{}{0pt}%
\pgfpathmoveto{\pgfqpoint{6.264813in}{4.925276in}}%
\pgfpathcurveto{\pgfqpoint{6.273049in}{4.925276in}}{\pgfqpoint{6.280949in}{4.928548in}}{\pgfqpoint{6.286773in}{4.934372in}}%
\pgfpathcurveto{\pgfqpoint{6.292597in}{4.940196in}}{\pgfqpoint{6.295869in}{4.948096in}}{\pgfqpoint{6.295869in}{4.956332in}}%
\pgfpathcurveto{\pgfqpoint{6.295869in}{4.964569in}}{\pgfqpoint{6.292597in}{4.972469in}}{\pgfqpoint{6.286773in}{4.978293in}}%
\pgfpathcurveto{\pgfqpoint{6.280949in}{4.984117in}}{\pgfqpoint{6.273049in}{4.987389in}}{\pgfqpoint{6.264813in}{4.987389in}}%
\pgfpathcurveto{\pgfqpoint{6.256576in}{4.987389in}}{\pgfqpoint{6.248676in}{4.984117in}}{\pgfqpoint{6.242852in}{4.978293in}}%
\pgfpathcurveto{\pgfqpoint{6.237028in}{4.972469in}}{\pgfqpoint{6.233756in}{4.964569in}}{\pgfqpoint{6.233756in}{4.956332in}}%
\pgfpathcurveto{\pgfqpoint{6.233756in}{4.948096in}}{\pgfqpoint{6.237028in}{4.940196in}}{\pgfqpoint{6.242852in}{4.934372in}}%
\pgfpathcurveto{\pgfqpoint{6.248676in}{4.928548in}}{\pgfqpoint{6.256576in}{4.925276in}}{\pgfqpoint{6.264813in}{4.925276in}}%
\pgfpathclose%
\pgfusepath{stroke,fill}%
\end{pgfscope}%
\begin{pgfscope}%
\pgfpathrectangle{\pgfqpoint{0.894063in}{3.540000in}}{\pgfqpoint{6.713438in}{2.060556in}} %
\pgfusepath{clip}%
\pgfsetbuttcap%
\pgfsetroundjoin%
\definecolor{currentfill}{rgb}{0.000000,0.000000,1.000000}%
\pgfsetfillcolor{currentfill}%
\pgfsetlinewidth{1.003750pt}%
\definecolor{currentstroke}{rgb}{0.000000,0.000000,0.000000}%
\pgfsetstrokecolor{currentstroke}%
\pgfsetdash{}{0pt}%
\pgfpathmoveto{\pgfqpoint{7.473231in}{4.740687in}}%
\pgfpathcurveto{\pgfqpoint{7.481468in}{4.740687in}}{\pgfqpoint{7.489368in}{4.743960in}}{\pgfqpoint{7.495192in}{4.749784in}}%
\pgfpathcurveto{\pgfqpoint{7.501015in}{4.755607in}}{\pgfqpoint{7.504288in}{4.763508in}}{\pgfqpoint{7.504288in}{4.771744in}}%
\pgfpathcurveto{\pgfqpoint{7.504288in}{4.779980in}}{\pgfqpoint{7.501015in}{4.787880in}}{\pgfqpoint{7.495192in}{4.793704in}}%
\pgfpathcurveto{\pgfqpoint{7.489368in}{4.799528in}}{\pgfqpoint{7.481468in}{4.802800in}}{\pgfqpoint{7.473231in}{4.802800in}}%
\pgfpathcurveto{\pgfqpoint{7.464995in}{4.802800in}}{\pgfqpoint{7.457095in}{4.799528in}}{\pgfqpoint{7.451271in}{4.793704in}}%
\pgfpathcurveto{\pgfqpoint{7.445447in}{4.787880in}}{\pgfqpoint{7.442175in}{4.779980in}}{\pgfqpoint{7.442175in}{4.771744in}}%
\pgfpathcurveto{\pgfqpoint{7.442175in}{4.763508in}}{\pgfqpoint{7.445447in}{4.755607in}}{\pgfqpoint{7.451271in}{4.749784in}}%
\pgfpathcurveto{\pgfqpoint{7.457095in}{4.743960in}}{\pgfqpoint{7.464995in}{4.740687in}}{\pgfqpoint{7.473231in}{4.740687in}}%
\pgfpathclose%
\pgfusepath{stroke,fill}%
\end{pgfscope}%
\begin{pgfscope}%
\pgfpathrectangle{\pgfqpoint{0.894063in}{3.540000in}}{\pgfqpoint{6.713438in}{2.060556in}} %
\pgfusepath{clip}%
\pgfsetbuttcap%
\pgfsetroundjoin%
\definecolor{currentfill}{rgb}{0.000000,0.000000,1.000000}%
\pgfsetfillcolor{currentfill}%
\pgfsetlinewidth{1.003750pt}%
\definecolor{currentstroke}{rgb}{0.000000,0.000000,0.000000}%
\pgfsetstrokecolor{currentstroke}%
\pgfsetdash{}{0pt}%
\pgfpathmoveto{\pgfqpoint{5.056394in}{5.101486in}}%
\pgfpathcurveto{\pgfqpoint{5.064630in}{5.101486in}}{\pgfqpoint{5.072530in}{5.104759in}}{\pgfqpoint{5.078354in}{5.110583in}}%
\pgfpathcurveto{\pgfqpoint{5.084178in}{5.116407in}}{\pgfqpoint{5.087450in}{5.124307in}}{\pgfqpoint{5.087450in}{5.132543in}}%
\pgfpathcurveto{\pgfqpoint{5.087450in}{5.140779in}}{\pgfqpoint{5.084178in}{5.148679in}}{\pgfqpoint{5.078354in}{5.154503in}}%
\pgfpathcurveto{\pgfqpoint{5.072530in}{5.160327in}}{\pgfqpoint{5.064630in}{5.163599in}}{\pgfqpoint{5.056394in}{5.163599in}}%
\pgfpathcurveto{\pgfqpoint{5.048157in}{5.163599in}}{\pgfqpoint{5.040257in}{5.160327in}}{\pgfqpoint{5.034433in}{5.154503in}}%
\pgfpathcurveto{\pgfqpoint{5.028610in}{5.148679in}}{\pgfqpoint{5.025337in}{5.140779in}}{\pgfqpoint{5.025337in}{5.132543in}}%
\pgfpathcurveto{\pgfqpoint{5.025337in}{5.124307in}}{\pgfqpoint{5.028610in}{5.116407in}}{\pgfqpoint{5.034433in}{5.110583in}}%
\pgfpathcurveto{\pgfqpoint{5.040257in}{5.104759in}}{\pgfqpoint{5.048157in}{5.101486in}}{\pgfqpoint{5.056394in}{5.101486in}}%
\pgfpathclose%
\pgfusepath{stroke,fill}%
\end{pgfscope}%
\begin{pgfscope}%
\pgfpathrectangle{\pgfqpoint{0.894063in}{3.540000in}}{\pgfqpoint{6.713438in}{2.060556in}} %
\pgfusepath{clip}%
\pgfsetbuttcap%
\pgfsetroundjoin%
\definecolor{currentfill}{rgb}{0.000000,0.000000,1.000000}%
\pgfsetfillcolor{currentfill}%
\pgfsetlinewidth{1.003750pt}%
\definecolor{currentstroke}{rgb}{0.000000,0.000000,0.000000}%
\pgfsetstrokecolor{currentstroke}%
\pgfsetdash{}{0pt}%
\pgfpathmoveto{\pgfqpoint{2.908094in}{5.126971in}}%
\pgfpathcurveto{\pgfqpoint{2.916330in}{5.126971in}}{\pgfqpoint{2.924230in}{5.130244in}}{\pgfqpoint{2.930054in}{5.136068in}}%
\pgfpathcurveto{\pgfqpoint{2.935878in}{5.141892in}}{\pgfqpoint{2.939150in}{5.149792in}}{\pgfqpoint{2.939150in}{5.158028in}}%
\pgfpathcurveto{\pgfqpoint{2.939150in}{5.166264in}}{\pgfqpoint{2.935878in}{5.174164in}}{\pgfqpoint{2.930054in}{5.179988in}}%
\pgfpathcurveto{\pgfqpoint{2.924230in}{5.185812in}}{\pgfqpoint{2.916330in}{5.189084in}}{\pgfqpoint{2.908094in}{5.189084in}}%
\pgfpathcurveto{\pgfqpoint{2.899857in}{5.189084in}}{\pgfqpoint{2.891957in}{5.185812in}}{\pgfqpoint{2.886133in}{5.179988in}}%
\pgfpathcurveto{\pgfqpoint{2.880310in}{5.174164in}}{\pgfqpoint{2.877037in}{5.166264in}}{\pgfqpoint{2.877037in}{5.158028in}}%
\pgfpathcurveto{\pgfqpoint{2.877037in}{5.149792in}}{\pgfqpoint{2.880310in}{5.141892in}}{\pgfqpoint{2.886133in}{5.136068in}}%
\pgfpathcurveto{\pgfqpoint{2.891957in}{5.130244in}}{\pgfqpoint{2.899857in}{5.126971in}}{\pgfqpoint{2.908094in}{5.126971in}}%
\pgfpathclose%
\pgfusepath{stroke,fill}%
\end{pgfscope}%
\begin{pgfscope}%
\pgfpathrectangle{\pgfqpoint{0.894063in}{3.540000in}}{\pgfqpoint{6.713438in}{2.060556in}} %
\pgfusepath{clip}%
\pgfsetbuttcap%
\pgfsetroundjoin%
\definecolor{currentfill}{rgb}{0.000000,0.000000,1.000000}%
\pgfsetfillcolor{currentfill}%
\pgfsetlinewidth{1.003750pt}%
\definecolor{currentstroke}{rgb}{0.000000,0.000000,0.000000}%
\pgfsetstrokecolor{currentstroke}%
\pgfsetdash{}{0pt}%
\pgfpathmoveto{\pgfqpoint{3.445169in}{5.126943in}}%
\pgfpathcurveto{\pgfqpoint{3.453405in}{5.126943in}}{\pgfqpoint{3.461305in}{5.130215in}}{\pgfqpoint{3.467129in}{5.136039in}}%
\pgfpathcurveto{\pgfqpoint{3.472953in}{5.141863in}}{\pgfqpoint{3.476225in}{5.149763in}}{\pgfqpoint{3.476225in}{5.157999in}}%
\pgfpathcurveto{\pgfqpoint{3.476225in}{5.166235in}}{\pgfqpoint{3.472953in}{5.174135in}}{\pgfqpoint{3.467129in}{5.179959in}}%
\pgfpathcurveto{\pgfqpoint{3.461305in}{5.185783in}}{\pgfqpoint{3.453405in}{5.189056in}}{\pgfqpoint{3.445169in}{5.189056in}}%
\pgfpathcurveto{\pgfqpoint{3.436932in}{5.189056in}}{\pgfqpoint{3.429032in}{5.185783in}}{\pgfqpoint{3.423208in}{5.179959in}}%
\pgfpathcurveto{\pgfqpoint{3.417385in}{5.174135in}}{\pgfqpoint{3.414112in}{5.166235in}}{\pgfqpoint{3.414112in}{5.157999in}}%
\pgfpathcurveto{\pgfqpoint{3.414112in}{5.149763in}}{\pgfqpoint{3.417385in}{5.141863in}}{\pgfqpoint{3.423208in}{5.136039in}}%
\pgfpathcurveto{\pgfqpoint{3.429032in}{5.130215in}}{\pgfqpoint{3.436932in}{5.126943in}}{\pgfqpoint{3.445169in}{5.126943in}}%
\pgfpathclose%
\pgfusepath{stroke,fill}%
\end{pgfscope}%
\begin{pgfscope}%
\pgfpathrectangle{\pgfqpoint{0.894063in}{3.540000in}}{\pgfqpoint{6.713438in}{2.060556in}} %
\pgfusepath{clip}%
\pgfsetbuttcap%
\pgfsetroundjoin%
\definecolor{currentfill}{rgb}{0.000000,0.000000,1.000000}%
\pgfsetfillcolor{currentfill}%
\pgfsetlinewidth{1.003750pt}%
\definecolor{currentstroke}{rgb}{0.000000,0.000000,0.000000}%
\pgfsetstrokecolor{currentstroke}%
\pgfsetdash{}{0pt}%
\pgfpathmoveto{\pgfqpoint{4.116513in}{5.126938in}}%
\pgfpathcurveto{\pgfqpoint{4.124749in}{5.126938in}}{\pgfqpoint{4.132649in}{5.130211in}}{\pgfqpoint{4.138473in}{5.136035in}}%
\pgfpathcurveto{\pgfqpoint{4.144297in}{5.141859in}}{\pgfqpoint{4.147569in}{5.149759in}}{\pgfqpoint{4.147569in}{5.157995in}}%
\pgfpathcurveto{\pgfqpoint{4.147569in}{5.166231in}}{\pgfqpoint{4.144297in}{5.174131in}}{\pgfqpoint{4.138473in}{5.179955in}}%
\pgfpathcurveto{\pgfqpoint{4.132649in}{5.185779in}}{\pgfqpoint{4.124749in}{5.189051in}}{\pgfqpoint{4.116513in}{5.189051in}}%
\pgfpathcurveto{\pgfqpoint{4.108276in}{5.189051in}}{\pgfqpoint{4.100376in}{5.185779in}}{\pgfqpoint{4.094552in}{5.179955in}}%
\pgfpathcurveto{\pgfqpoint{4.088728in}{5.174131in}}{\pgfqpoint{4.085456in}{5.166231in}}{\pgfqpoint{4.085456in}{5.157995in}}%
\pgfpathcurveto{\pgfqpoint{4.085456in}{5.149759in}}{\pgfqpoint{4.088728in}{5.141859in}}{\pgfqpoint{4.094552in}{5.136035in}}%
\pgfpathcurveto{\pgfqpoint{4.100376in}{5.130211in}}{\pgfqpoint{4.108276in}{5.126938in}}{\pgfqpoint{4.116513in}{5.126938in}}%
\pgfpathclose%
\pgfusepath{stroke,fill}%
\end{pgfscope}%
\begin{pgfscope}%
\pgfpathrectangle{\pgfqpoint{0.894063in}{3.540000in}}{\pgfqpoint{6.713438in}{2.060556in}} %
\pgfusepath{clip}%
\pgfsetbuttcap%
\pgfsetroundjoin%
\definecolor{currentfill}{rgb}{0.000000,0.000000,1.000000}%
\pgfsetfillcolor{currentfill}%
\pgfsetlinewidth{1.003750pt}%
\definecolor{currentstroke}{rgb}{0.000000,0.000000,0.000000}%
\pgfsetstrokecolor{currentstroke}%
\pgfsetdash{}{0pt}%
\pgfpathmoveto{\pgfqpoint{1.431138in}{5.137729in}}%
\pgfpathcurveto{\pgfqpoint{1.439374in}{5.137729in}}{\pgfqpoint{1.447274in}{5.141001in}}{\pgfqpoint{1.453098in}{5.146825in}}%
\pgfpathcurveto{\pgfqpoint{1.458922in}{5.152649in}}{\pgfqpoint{1.462194in}{5.160549in}}{\pgfqpoint{1.462194in}{5.168785in}}%
\pgfpathcurveto{\pgfqpoint{1.462194in}{5.177022in}}{\pgfqpoint{1.458922in}{5.184922in}}{\pgfqpoint{1.453098in}{5.190746in}}%
\pgfpathcurveto{\pgfqpoint{1.447274in}{5.196570in}}{\pgfqpoint{1.439374in}{5.199842in}}{\pgfqpoint{1.431138in}{5.199842in}}%
\pgfpathcurveto{\pgfqpoint{1.422901in}{5.199842in}}{\pgfqpoint{1.415001in}{5.196570in}}{\pgfqpoint{1.409177in}{5.190746in}}%
\pgfpathcurveto{\pgfqpoint{1.403353in}{5.184922in}}{\pgfqpoint{1.400081in}{5.177022in}}{\pgfqpoint{1.400081in}{5.168785in}}%
\pgfpathcurveto{\pgfqpoint{1.400081in}{5.160549in}}{\pgfqpoint{1.403353in}{5.152649in}}{\pgfqpoint{1.409177in}{5.146825in}}%
\pgfpathcurveto{\pgfqpoint{1.415001in}{5.141001in}}{\pgfqpoint{1.422901in}{5.137729in}}{\pgfqpoint{1.431138in}{5.137729in}}%
\pgfpathclose%
\pgfusepath{stroke,fill}%
\end{pgfscope}%
\begin{pgfscope}%
\pgfpathrectangle{\pgfqpoint{0.894063in}{3.540000in}}{\pgfqpoint{6.713438in}{2.060556in}} %
\pgfusepath{clip}%
\pgfsetbuttcap%
\pgfsetroundjoin%
\definecolor{currentfill}{rgb}{0.000000,0.000000,1.000000}%
\pgfsetfillcolor{currentfill}%
\pgfsetlinewidth{1.003750pt}%
\definecolor{currentstroke}{rgb}{0.000000,0.000000,0.000000}%
\pgfsetstrokecolor{currentstroke}%
\pgfsetdash{}{0pt}%
\pgfpathmoveto{\pgfqpoint{2.773825in}{5.126971in}}%
\pgfpathcurveto{\pgfqpoint{2.782061in}{5.126971in}}{\pgfqpoint{2.789961in}{5.130244in}}{\pgfqpoint{2.795785in}{5.136068in}}%
\pgfpathcurveto{\pgfqpoint{2.801609in}{5.141892in}}{\pgfqpoint{2.804881in}{5.149792in}}{\pgfqpoint{2.804881in}{5.158028in}}%
\pgfpathcurveto{\pgfqpoint{2.804881in}{5.166264in}}{\pgfqpoint{2.801609in}{5.174164in}}{\pgfqpoint{2.795785in}{5.179988in}}%
\pgfpathcurveto{\pgfqpoint{2.789961in}{5.185812in}}{\pgfqpoint{2.782061in}{5.189084in}}{\pgfqpoint{2.773825in}{5.189084in}}%
\pgfpathcurveto{\pgfqpoint{2.765589in}{5.189084in}}{\pgfqpoint{2.757689in}{5.185812in}}{\pgfqpoint{2.751865in}{5.179988in}}%
\pgfpathcurveto{\pgfqpoint{2.746041in}{5.174164in}}{\pgfqpoint{2.742769in}{5.166264in}}{\pgfqpoint{2.742769in}{5.158028in}}%
\pgfpathcurveto{\pgfqpoint{2.742769in}{5.149792in}}{\pgfqpoint{2.746041in}{5.141892in}}{\pgfqpoint{2.751865in}{5.136068in}}%
\pgfpathcurveto{\pgfqpoint{2.757689in}{5.130244in}}{\pgfqpoint{2.765589in}{5.126971in}}{\pgfqpoint{2.773825in}{5.126971in}}%
\pgfpathclose%
\pgfusepath{stroke,fill}%
\end{pgfscope}%
\begin{pgfscope}%
\pgfpathrectangle{\pgfqpoint{0.894063in}{3.540000in}}{\pgfqpoint{6.713438in}{2.060556in}} %
\pgfusepath{clip}%
\pgfsetbuttcap%
\pgfsetroundjoin%
\definecolor{currentfill}{rgb}{0.000000,0.000000,1.000000}%
\pgfsetfillcolor{currentfill}%
\pgfsetlinewidth{1.003750pt}%
\definecolor{currentstroke}{rgb}{0.000000,0.000000,0.000000}%
\pgfsetstrokecolor{currentstroke}%
\pgfsetdash{}{0pt}%
\pgfpathmoveto{\pgfqpoint{1.565406in}{5.135497in}}%
\pgfpathcurveto{\pgfqpoint{1.573643in}{5.135497in}}{\pgfqpoint{1.581543in}{5.138769in}}{\pgfqpoint{1.587367in}{5.144593in}}%
\pgfpathcurveto{\pgfqpoint{1.593190in}{5.150417in}}{\pgfqpoint{1.596463in}{5.158317in}}{\pgfqpoint{1.596463in}{5.166553in}}%
\pgfpathcurveto{\pgfqpoint{1.596463in}{5.174789in}}{\pgfqpoint{1.593190in}{5.182689in}}{\pgfqpoint{1.587367in}{5.188513in}}%
\pgfpathcurveto{\pgfqpoint{1.581543in}{5.194337in}}{\pgfqpoint{1.573643in}{5.197610in}}{\pgfqpoint{1.565406in}{5.197610in}}%
\pgfpathcurveto{\pgfqpoint{1.557170in}{5.197610in}}{\pgfqpoint{1.549270in}{5.194337in}}{\pgfqpoint{1.543446in}{5.188513in}}%
\pgfpathcurveto{\pgfqpoint{1.537622in}{5.182689in}}{\pgfqpoint{1.534350in}{5.174789in}}{\pgfqpoint{1.534350in}{5.166553in}}%
\pgfpathcurveto{\pgfqpoint{1.534350in}{5.158317in}}{\pgfqpoint{1.537622in}{5.150417in}}{\pgfqpoint{1.543446in}{5.144593in}}%
\pgfpathcurveto{\pgfqpoint{1.549270in}{5.138769in}}{\pgfqpoint{1.557170in}{5.135497in}}{\pgfqpoint{1.565406in}{5.135497in}}%
\pgfpathclose%
\pgfusepath{stroke,fill}%
\end{pgfscope}%
\begin{pgfscope}%
\pgfpathrectangle{\pgfqpoint{0.894063in}{3.540000in}}{\pgfqpoint{6.713438in}{2.060556in}} %
\pgfusepath{clip}%
\pgfsetbuttcap%
\pgfsetroundjoin%
\definecolor{currentfill}{rgb}{0.000000,0.000000,1.000000}%
\pgfsetfillcolor{currentfill}%
\pgfsetlinewidth{1.003750pt}%
\definecolor{currentstroke}{rgb}{0.000000,0.000000,0.000000}%
\pgfsetstrokecolor{currentstroke}%
\pgfsetdash{}{0pt}%
\pgfpathmoveto{\pgfqpoint{4.250781in}{5.126937in}}%
\pgfpathcurveto{\pgfqpoint{4.259018in}{5.126937in}}{\pgfqpoint{4.266918in}{5.130209in}}{\pgfqpoint{4.272742in}{5.136033in}}%
\pgfpathcurveto{\pgfqpoint{4.278565in}{5.141857in}}{\pgfqpoint{4.281838in}{5.149757in}}{\pgfqpoint{4.281838in}{5.157994in}}%
\pgfpathcurveto{\pgfqpoint{4.281838in}{5.166230in}}{\pgfqpoint{4.278565in}{5.174130in}}{\pgfqpoint{4.272742in}{5.179954in}}%
\pgfpathcurveto{\pgfqpoint{4.266918in}{5.185778in}}{\pgfqpoint{4.259018in}{5.189050in}}{\pgfqpoint{4.250781in}{5.189050in}}%
\pgfpathcurveto{\pgfqpoint{4.242545in}{5.189050in}}{\pgfqpoint{4.234645in}{5.185778in}}{\pgfqpoint{4.228821in}{5.179954in}}%
\pgfpathcurveto{\pgfqpoint{4.222997in}{5.174130in}}{\pgfqpoint{4.219725in}{5.166230in}}{\pgfqpoint{4.219725in}{5.157994in}}%
\pgfpathcurveto{\pgfqpoint{4.219725in}{5.149757in}}{\pgfqpoint{4.222997in}{5.141857in}}{\pgfqpoint{4.228821in}{5.136033in}}%
\pgfpathcurveto{\pgfqpoint{4.234645in}{5.130209in}}{\pgfqpoint{4.242545in}{5.126937in}}{\pgfqpoint{4.250781in}{5.126937in}}%
\pgfpathclose%
\pgfusepath{stroke,fill}%
\end{pgfscope}%
\begin{pgfscope}%
\pgfpathrectangle{\pgfqpoint{0.894063in}{3.540000in}}{\pgfqpoint{6.713438in}{2.060556in}} %
\pgfusepath{clip}%
\pgfsetbuttcap%
\pgfsetroundjoin%
\definecolor{currentfill}{rgb}{0.000000,0.000000,1.000000}%
\pgfsetfillcolor{currentfill}%
\pgfsetlinewidth{1.003750pt}%
\definecolor{currentstroke}{rgb}{0.000000,0.000000,0.000000}%
\pgfsetstrokecolor{currentstroke}%
\pgfsetdash{}{0pt}%
\pgfpathmoveto{\pgfqpoint{3.847975in}{5.126938in}}%
\pgfpathcurveto{\pgfqpoint{3.856211in}{5.126938in}}{\pgfqpoint{3.864111in}{5.130211in}}{\pgfqpoint{3.869935in}{5.136035in}}%
\pgfpathcurveto{\pgfqpoint{3.875759in}{5.141859in}}{\pgfqpoint{3.879031in}{5.149759in}}{\pgfqpoint{3.879031in}{5.157995in}}%
\pgfpathcurveto{\pgfqpoint{3.879031in}{5.166231in}}{\pgfqpoint{3.875759in}{5.174131in}}{\pgfqpoint{3.869935in}{5.179955in}}%
\pgfpathcurveto{\pgfqpoint{3.864111in}{5.185779in}}{\pgfqpoint{3.856211in}{5.189051in}}{\pgfqpoint{3.847975in}{5.189051in}}%
\pgfpathcurveto{\pgfqpoint{3.839739in}{5.189051in}}{\pgfqpoint{3.831839in}{5.185779in}}{\pgfqpoint{3.826015in}{5.179955in}}%
\pgfpathcurveto{\pgfqpoint{3.820191in}{5.174131in}}{\pgfqpoint{3.816919in}{5.166231in}}{\pgfqpoint{3.816919in}{5.157995in}}%
\pgfpathcurveto{\pgfqpoint{3.816919in}{5.149759in}}{\pgfqpoint{3.820191in}{5.141859in}}{\pgfqpoint{3.826015in}{5.136035in}}%
\pgfpathcurveto{\pgfqpoint{3.831839in}{5.130211in}}{\pgfqpoint{3.839739in}{5.126938in}}{\pgfqpoint{3.847975in}{5.126938in}}%
\pgfpathclose%
\pgfusepath{stroke,fill}%
\end{pgfscope}%
\begin{pgfscope}%
\pgfpathrectangle{\pgfqpoint{0.894063in}{3.540000in}}{\pgfqpoint{6.713438in}{2.060556in}} %
\pgfusepath{clip}%
\pgfsetbuttcap%
\pgfsetroundjoin%
\definecolor{currentfill}{rgb}{0.000000,0.000000,1.000000}%
\pgfsetfillcolor{currentfill}%
\pgfsetlinewidth{1.003750pt}%
\definecolor{currentstroke}{rgb}{0.000000,0.000000,0.000000}%
\pgfsetstrokecolor{currentstroke}%
\pgfsetdash{}{0pt}%
\pgfpathmoveto{\pgfqpoint{7.607500in}{4.725336in}}%
\pgfpathcurveto{\pgfqpoint{7.615736in}{4.725336in}}{\pgfqpoint{7.623636in}{4.728608in}}{\pgfqpoint{7.629460in}{4.734432in}}%
\pgfpathcurveto{\pgfqpoint{7.635284in}{4.740256in}}{\pgfqpoint{7.638556in}{4.748156in}}{\pgfqpoint{7.638556in}{4.756393in}}%
\pgfpathcurveto{\pgfqpoint{7.638556in}{4.764629in}}{\pgfqpoint{7.635284in}{4.772529in}}{\pgfqpoint{7.629460in}{4.778353in}}%
\pgfpathcurveto{\pgfqpoint{7.623636in}{4.784177in}}{\pgfqpoint{7.615736in}{4.787449in}}{\pgfqpoint{7.607500in}{4.787449in}}%
\pgfpathcurveto{\pgfqpoint{7.599264in}{4.787449in}}{\pgfqpoint{7.591364in}{4.784177in}}{\pgfqpoint{7.585540in}{4.778353in}}%
\pgfpathcurveto{\pgfqpoint{7.579716in}{4.772529in}}{\pgfqpoint{7.576444in}{4.764629in}}{\pgfqpoint{7.576444in}{4.756393in}}%
\pgfpathcurveto{\pgfqpoint{7.576444in}{4.748156in}}{\pgfqpoint{7.579716in}{4.740256in}}{\pgfqpoint{7.585540in}{4.734432in}}%
\pgfpathcurveto{\pgfqpoint{7.591364in}{4.728608in}}{\pgfqpoint{7.599264in}{4.725336in}}{\pgfqpoint{7.607500in}{4.725336in}}%
\pgfpathclose%
\pgfusepath{stroke,fill}%
\end{pgfscope}%
\begin{pgfscope}%
\pgfpathrectangle{\pgfqpoint{0.894063in}{3.540000in}}{\pgfqpoint{6.713438in}{2.060556in}} %
\pgfusepath{clip}%
\pgfsetbuttcap%
\pgfsetroundjoin%
\definecolor{currentfill}{rgb}{0.000000,0.000000,1.000000}%
\pgfsetfillcolor{currentfill}%
\pgfsetlinewidth{1.003750pt}%
\definecolor{currentstroke}{rgb}{0.000000,0.000000,0.000000}%
\pgfsetstrokecolor{currentstroke}%
\pgfsetdash{}{0pt}%
\pgfpathmoveto{\pgfqpoint{4.385050in}{5.126937in}}%
\pgfpathcurveto{\pgfqpoint{4.393286in}{5.126937in}}{\pgfqpoint{4.401186in}{5.130209in}}{\pgfqpoint{4.407010in}{5.136033in}}%
\pgfpathcurveto{\pgfqpoint{4.412834in}{5.141857in}}{\pgfqpoint{4.416106in}{5.149757in}}{\pgfqpoint{4.416106in}{5.157994in}}%
\pgfpathcurveto{\pgfqpoint{4.416106in}{5.166230in}}{\pgfqpoint{4.412834in}{5.174130in}}{\pgfqpoint{4.407010in}{5.179954in}}%
\pgfpathcurveto{\pgfqpoint{4.401186in}{5.185778in}}{\pgfqpoint{4.393286in}{5.189050in}}{\pgfqpoint{4.385050in}{5.189050in}}%
\pgfpathcurveto{\pgfqpoint{4.376814in}{5.189050in}}{\pgfqpoint{4.368914in}{5.185778in}}{\pgfqpoint{4.363090in}{5.179954in}}%
\pgfpathcurveto{\pgfqpoint{4.357266in}{5.174130in}}{\pgfqpoint{4.353994in}{5.166230in}}{\pgfqpoint{4.353994in}{5.157994in}}%
\pgfpathcurveto{\pgfqpoint{4.353994in}{5.149757in}}{\pgfqpoint{4.357266in}{5.141857in}}{\pgfqpoint{4.363090in}{5.136033in}}%
\pgfpathcurveto{\pgfqpoint{4.368914in}{5.130209in}}{\pgfqpoint{4.376814in}{5.126937in}}{\pgfqpoint{4.385050in}{5.126937in}}%
\pgfpathclose%
\pgfusepath{stroke,fill}%
\end{pgfscope}%
\begin{pgfscope}%
\pgfpathrectangle{\pgfqpoint{0.894063in}{3.540000in}}{\pgfqpoint{6.713438in}{2.060556in}} %
\pgfusepath{clip}%
\pgfsetbuttcap%
\pgfsetroundjoin%
\definecolor{currentfill}{rgb}{0.000000,0.000000,1.000000}%
\pgfsetfillcolor{currentfill}%
\pgfsetlinewidth{1.003750pt}%
\definecolor{currentstroke}{rgb}{0.000000,0.000000,0.000000}%
\pgfsetstrokecolor{currentstroke}%
\pgfsetdash{}{0pt}%
\pgfpathmoveto{\pgfqpoint{6.533350in}{4.882565in}}%
\pgfpathcurveto{\pgfqpoint{6.541586in}{4.882565in}}{\pgfqpoint{6.549486in}{4.885837in}}{\pgfqpoint{6.555310in}{4.891661in}}%
\pgfpathcurveto{\pgfqpoint{6.561134in}{4.897485in}}{\pgfqpoint{6.564406in}{4.905385in}}{\pgfqpoint{6.564406in}{4.913621in}}%
\pgfpathcurveto{\pgfqpoint{6.564406in}{4.921858in}}{\pgfqpoint{6.561134in}{4.929758in}}{\pgfqpoint{6.555310in}{4.935582in}}%
\pgfpathcurveto{\pgfqpoint{6.549486in}{4.941405in}}{\pgfqpoint{6.541586in}{4.944678in}}{\pgfqpoint{6.533350in}{4.944678in}}%
\pgfpathcurveto{\pgfqpoint{6.525114in}{4.944678in}}{\pgfqpoint{6.517214in}{4.941405in}}{\pgfqpoint{6.511390in}{4.935582in}}%
\pgfpathcurveto{\pgfqpoint{6.505566in}{4.929758in}}{\pgfqpoint{6.502294in}{4.921858in}}{\pgfqpoint{6.502294in}{4.913621in}}%
\pgfpathcurveto{\pgfqpoint{6.502294in}{4.905385in}}{\pgfqpoint{6.505566in}{4.897485in}}{\pgfqpoint{6.511390in}{4.891661in}}%
\pgfpathcurveto{\pgfqpoint{6.517214in}{4.885837in}}{\pgfqpoint{6.525114in}{4.882565in}}{\pgfqpoint{6.533350in}{4.882565in}}%
\pgfpathclose%
\pgfusepath{stroke,fill}%
\end{pgfscope}%
\begin{pgfscope}%
\pgfpathrectangle{\pgfqpoint{0.894063in}{3.540000in}}{\pgfqpoint{6.713438in}{2.060556in}} %
\pgfusepath{clip}%
\pgfsetbuttcap%
\pgfsetroundjoin%
\definecolor{currentfill}{rgb}{0.000000,0.000000,1.000000}%
\pgfsetfillcolor{currentfill}%
\pgfsetlinewidth{1.003750pt}%
\definecolor{currentstroke}{rgb}{0.000000,0.000000,0.000000}%
\pgfsetstrokecolor{currentstroke}%
\pgfsetdash{}{0pt}%
\pgfpathmoveto{\pgfqpoint{1.296869in}{5.140545in}}%
\pgfpathcurveto{\pgfqpoint{1.305105in}{5.140545in}}{\pgfqpoint{1.313005in}{5.143817in}}{\pgfqpoint{1.318829in}{5.149641in}}%
\pgfpathcurveto{\pgfqpoint{1.324653in}{5.155465in}}{\pgfqpoint{1.327925in}{5.163365in}}{\pgfqpoint{1.327925in}{5.171601in}}%
\pgfpathcurveto{\pgfqpoint{1.327925in}{5.179838in}}{\pgfqpoint{1.324653in}{5.187738in}}{\pgfqpoint{1.318829in}{5.193562in}}%
\pgfpathcurveto{\pgfqpoint{1.313005in}{5.199386in}}{\pgfqpoint{1.305105in}{5.202658in}}{\pgfqpoint{1.296869in}{5.202658in}}%
\pgfpathcurveto{\pgfqpoint{1.288632in}{5.202658in}}{\pgfqpoint{1.280732in}{5.199386in}}{\pgfqpoint{1.274908in}{5.193562in}}%
\pgfpathcurveto{\pgfqpoint{1.269085in}{5.187738in}}{\pgfqpoint{1.265812in}{5.179838in}}{\pgfqpoint{1.265812in}{5.171601in}}%
\pgfpathcurveto{\pgfqpoint{1.265812in}{5.163365in}}{\pgfqpoint{1.269085in}{5.155465in}}{\pgfqpoint{1.274908in}{5.149641in}}%
\pgfpathcurveto{\pgfqpoint{1.280732in}{5.143817in}}{\pgfqpoint{1.288632in}{5.140545in}}{\pgfqpoint{1.296869in}{5.140545in}}%
\pgfpathclose%
\pgfusepath{stroke,fill}%
\end{pgfscope}%
\begin{pgfscope}%
\pgfpathrectangle{\pgfqpoint{0.894063in}{3.540000in}}{\pgfqpoint{6.713438in}{2.060556in}} %
\pgfusepath{clip}%
\pgfsetbuttcap%
\pgfsetroundjoin%
\definecolor{currentfill}{rgb}{0.000000,0.000000,1.000000}%
\pgfsetfillcolor{currentfill}%
\pgfsetlinewidth{1.003750pt}%
\definecolor{currentstroke}{rgb}{0.000000,0.000000,0.000000}%
\pgfsetstrokecolor{currentstroke}%
\pgfsetdash{}{0pt}%
\pgfpathmoveto{\pgfqpoint{4.519319in}{5.126937in}}%
\pgfpathcurveto{\pgfqpoint{4.527555in}{5.126937in}}{\pgfqpoint{4.535455in}{5.130209in}}{\pgfqpoint{4.541279in}{5.136033in}}%
\pgfpathcurveto{\pgfqpoint{4.547103in}{5.141857in}}{\pgfqpoint{4.550375in}{5.149757in}}{\pgfqpoint{4.550375in}{5.157994in}}%
\pgfpathcurveto{\pgfqpoint{4.550375in}{5.166230in}}{\pgfqpoint{4.547103in}{5.174130in}}{\pgfqpoint{4.541279in}{5.179954in}}%
\pgfpathcurveto{\pgfqpoint{4.535455in}{5.185778in}}{\pgfqpoint{4.527555in}{5.189050in}}{\pgfqpoint{4.519319in}{5.189050in}}%
\pgfpathcurveto{\pgfqpoint{4.511082in}{5.189050in}}{\pgfqpoint{4.503182in}{5.185778in}}{\pgfqpoint{4.497358in}{5.179954in}}%
\pgfpathcurveto{\pgfqpoint{4.491535in}{5.174130in}}{\pgfqpoint{4.488262in}{5.166230in}}{\pgfqpoint{4.488262in}{5.157994in}}%
\pgfpathcurveto{\pgfqpoint{4.488262in}{5.149757in}}{\pgfqpoint{4.491535in}{5.141857in}}{\pgfqpoint{4.497358in}{5.136033in}}%
\pgfpathcurveto{\pgfqpoint{4.503182in}{5.130209in}}{\pgfqpoint{4.511082in}{5.126937in}}{\pgfqpoint{4.519319in}{5.126937in}}%
\pgfpathclose%
\pgfusepath{stroke,fill}%
\end{pgfscope}%
\begin{pgfscope}%
\pgfpathrectangle{\pgfqpoint{0.894063in}{3.540000in}}{\pgfqpoint{6.713438in}{2.060556in}} %
\pgfusepath{clip}%
\pgfsetbuttcap%
\pgfsetroundjoin%
\definecolor{currentfill}{rgb}{0.000000,0.000000,1.000000}%
\pgfsetfillcolor{currentfill}%
\pgfsetlinewidth{1.003750pt}%
\definecolor{currentstroke}{rgb}{0.000000,0.000000,0.000000}%
\pgfsetstrokecolor{currentstroke}%
\pgfsetdash{}{0pt}%
\pgfpathmoveto{\pgfqpoint{2.505288in}{5.126980in}}%
\pgfpathcurveto{\pgfqpoint{2.513524in}{5.126980in}}{\pgfqpoint{2.521424in}{5.130252in}}{\pgfqpoint{2.527248in}{5.136076in}}%
\pgfpathcurveto{\pgfqpoint{2.533072in}{5.141900in}}{\pgfqpoint{2.536344in}{5.149800in}}{\pgfqpoint{2.536344in}{5.158036in}}%
\pgfpathcurveto{\pgfqpoint{2.536344in}{5.166272in}}{\pgfqpoint{2.533072in}{5.174172in}}{\pgfqpoint{2.527248in}{5.179996in}}%
\pgfpathcurveto{\pgfqpoint{2.521424in}{5.185820in}}{\pgfqpoint{2.513524in}{5.189093in}}{\pgfqpoint{2.505288in}{5.189093in}}%
\pgfpathcurveto{\pgfqpoint{2.497051in}{5.189093in}}{\pgfqpoint{2.489151in}{5.185820in}}{\pgfqpoint{2.483327in}{5.179996in}}%
\pgfpathcurveto{\pgfqpoint{2.477503in}{5.174172in}}{\pgfqpoint{2.474231in}{5.166272in}}{\pgfqpoint{2.474231in}{5.158036in}}%
\pgfpathcurveto{\pgfqpoint{2.474231in}{5.149800in}}{\pgfqpoint{2.477503in}{5.141900in}}{\pgfqpoint{2.483327in}{5.136076in}}%
\pgfpathcurveto{\pgfqpoint{2.489151in}{5.130252in}}{\pgfqpoint{2.497051in}{5.126980in}}{\pgfqpoint{2.505288in}{5.126980in}}%
\pgfpathclose%
\pgfusepath{stroke,fill}%
\end{pgfscope}%
\begin{pgfscope}%
\pgfpathrectangle{\pgfqpoint{0.894063in}{3.540000in}}{\pgfqpoint{6.713438in}{2.060556in}} %
\pgfusepath{clip}%
\pgfsetbuttcap%
\pgfsetroundjoin%
\definecolor{currentfill}{rgb}{0.000000,0.000000,1.000000}%
\pgfsetfillcolor{currentfill}%
\pgfsetlinewidth{1.003750pt}%
\definecolor{currentstroke}{rgb}{0.000000,0.000000,0.000000}%
\pgfsetstrokecolor{currentstroke}%
\pgfsetdash{}{0pt}%
\pgfpathmoveto{\pgfqpoint{5.459200in}{5.048893in}}%
\pgfpathcurveto{\pgfqpoint{5.467436in}{5.048893in}}{\pgfqpoint{5.475336in}{5.052165in}}{\pgfqpoint{5.481160in}{5.057989in}}%
\pgfpathcurveto{\pgfqpoint{5.486984in}{5.063813in}}{\pgfqpoint{5.490256in}{5.071713in}}{\pgfqpoint{5.490256in}{5.079949in}}%
\pgfpathcurveto{\pgfqpoint{5.490256in}{5.088186in}}{\pgfqpoint{5.486984in}{5.096086in}}{\pgfqpoint{5.481160in}{5.101910in}}%
\pgfpathcurveto{\pgfqpoint{5.475336in}{5.107734in}}{\pgfqpoint{5.467436in}{5.111006in}}{\pgfqpoint{5.459200in}{5.111006in}}%
\pgfpathcurveto{\pgfqpoint{5.450964in}{5.111006in}}{\pgfqpoint{5.443064in}{5.107734in}}{\pgfqpoint{5.437240in}{5.101910in}}%
\pgfpathcurveto{\pgfqpoint{5.431416in}{5.096086in}}{\pgfqpoint{5.428144in}{5.088186in}}{\pgfqpoint{5.428144in}{5.079949in}}%
\pgfpathcurveto{\pgfqpoint{5.428144in}{5.071713in}}{\pgfqpoint{5.431416in}{5.063813in}}{\pgfqpoint{5.437240in}{5.057989in}}%
\pgfpathcurveto{\pgfqpoint{5.443064in}{5.052165in}}{\pgfqpoint{5.450964in}{5.048893in}}{\pgfqpoint{5.459200in}{5.048893in}}%
\pgfpathclose%
\pgfusepath{stroke,fill}%
\end{pgfscope}%
\begin{pgfscope}%
\pgfpathrectangle{\pgfqpoint{0.894063in}{3.540000in}}{\pgfqpoint{6.713438in}{2.060556in}} %
\pgfusepath{clip}%
\pgfsetbuttcap%
\pgfsetroundjoin%
\definecolor{currentfill}{rgb}{0.000000,0.000000,1.000000}%
\pgfsetfillcolor{currentfill}%
\pgfsetlinewidth{1.003750pt}%
\definecolor{currentstroke}{rgb}{0.000000,0.000000,0.000000}%
\pgfsetstrokecolor{currentstroke}%
\pgfsetdash{}{0pt}%
\pgfpathmoveto{\pgfqpoint{6.936156in}{4.824105in}}%
\pgfpathcurveto{\pgfqpoint{6.944393in}{4.824105in}}{\pgfqpoint{6.952293in}{4.827378in}}{\pgfqpoint{6.958117in}{4.833202in}}%
\pgfpathcurveto{\pgfqpoint{6.963940in}{4.839026in}}{\pgfqpoint{6.967213in}{4.846926in}}{\pgfqpoint{6.967213in}{4.855162in}}%
\pgfpathcurveto{\pgfqpoint{6.967213in}{4.863398in}}{\pgfqpoint{6.963940in}{4.871298in}}{\pgfqpoint{6.958117in}{4.877122in}}%
\pgfpathcurveto{\pgfqpoint{6.952293in}{4.882946in}}{\pgfqpoint{6.944393in}{4.886218in}}{\pgfqpoint{6.936156in}{4.886218in}}%
\pgfpathcurveto{\pgfqpoint{6.927920in}{4.886218in}}{\pgfqpoint{6.920020in}{4.882946in}}{\pgfqpoint{6.914196in}{4.877122in}}%
\pgfpathcurveto{\pgfqpoint{6.908372in}{4.871298in}}{\pgfqpoint{6.905100in}{4.863398in}}{\pgfqpoint{6.905100in}{4.855162in}}%
\pgfpathcurveto{\pgfqpoint{6.905100in}{4.846926in}}{\pgfqpoint{6.908372in}{4.839026in}}{\pgfqpoint{6.914196in}{4.833202in}}%
\pgfpathcurveto{\pgfqpoint{6.920020in}{4.827378in}}{\pgfqpoint{6.927920in}{4.824105in}}{\pgfqpoint{6.936156in}{4.824105in}}%
\pgfpathclose%
\pgfusepath{stroke,fill}%
\end{pgfscope}%
\begin{pgfscope}%
\pgfpathrectangle{\pgfqpoint{0.894063in}{3.540000in}}{\pgfqpoint{6.713438in}{2.060556in}} %
\pgfusepath{clip}%
\pgfsetbuttcap%
\pgfsetroundjoin%
\definecolor{currentfill}{rgb}{0.000000,0.000000,1.000000}%
\pgfsetfillcolor{currentfill}%
\pgfsetlinewidth{1.003750pt}%
\definecolor{currentstroke}{rgb}{0.000000,0.000000,0.000000}%
\pgfsetstrokecolor{currentstroke}%
\pgfsetdash{}{0pt}%
\pgfpathmoveto{\pgfqpoint{5.862006in}{4.978085in}}%
\pgfpathcurveto{\pgfqpoint{5.870243in}{4.978085in}}{\pgfqpoint{5.878143in}{4.981358in}}{\pgfqpoint{5.883967in}{4.987182in}}%
\pgfpathcurveto{\pgfqpoint{5.889790in}{4.993005in}}{\pgfqpoint{5.893063in}{5.000905in}}{\pgfqpoint{5.893063in}{5.009142in}}%
\pgfpathcurveto{\pgfqpoint{5.893063in}{5.017378in}}{\pgfqpoint{5.889790in}{5.025278in}}{\pgfqpoint{5.883967in}{5.031102in}}%
\pgfpathcurveto{\pgfqpoint{5.878143in}{5.036926in}}{\pgfqpoint{5.870243in}{5.040198in}}{\pgfqpoint{5.862006in}{5.040198in}}%
\pgfpathcurveto{\pgfqpoint{5.853770in}{5.040198in}}{\pgfqpoint{5.845870in}{5.036926in}}{\pgfqpoint{5.840046in}{5.031102in}}%
\pgfpathcurveto{\pgfqpoint{5.834222in}{5.025278in}}{\pgfqpoint{5.830950in}{5.017378in}}{\pgfqpoint{5.830950in}{5.009142in}}%
\pgfpathcurveto{\pgfqpoint{5.830950in}{5.000905in}}{\pgfqpoint{5.834222in}{4.993005in}}{\pgfqpoint{5.840046in}{4.987182in}}%
\pgfpathcurveto{\pgfqpoint{5.845870in}{4.981358in}}{\pgfqpoint{5.853770in}{4.978085in}}{\pgfqpoint{5.862006in}{4.978085in}}%
\pgfpathclose%
\pgfusepath{stroke,fill}%
\end{pgfscope}%
\begin{pgfscope}%
\pgfpathrectangle{\pgfqpoint{0.894063in}{3.540000in}}{\pgfqpoint{6.713438in}{2.060556in}} %
\pgfusepath{clip}%
\pgfsetbuttcap%
\pgfsetroundjoin%
\definecolor{currentfill}{rgb}{0.000000,0.000000,1.000000}%
\pgfsetfillcolor{currentfill}%
\pgfsetlinewidth{1.003750pt}%
\definecolor{currentstroke}{rgb}{0.000000,0.000000,0.000000}%
\pgfsetstrokecolor{currentstroke}%
\pgfsetdash{}{0pt}%
\pgfpathmoveto{\pgfqpoint{7.070425in}{4.804896in}}%
\pgfpathcurveto{\pgfqpoint{7.078661in}{4.804896in}}{\pgfqpoint{7.086561in}{4.808168in}}{\pgfqpoint{7.092385in}{4.813992in}}%
\pgfpathcurveto{\pgfqpoint{7.098209in}{4.819816in}}{\pgfqpoint{7.101481in}{4.827716in}}{\pgfqpoint{7.101481in}{4.835952in}}%
\pgfpathcurveto{\pgfqpoint{7.101481in}{4.844188in}}{\pgfqpoint{7.098209in}{4.852088in}}{\pgfqpoint{7.092385in}{4.857912in}}%
\pgfpathcurveto{\pgfqpoint{7.086561in}{4.863736in}}{\pgfqpoint{7.078661in}{4.867009in}}{\pgfqpoint{7.070425in}{4.867009in}}%
\pgfpathcurveto{\pgfqpoint{7.062189in}{4.867009in}}{\pgfqpoint{7.054289in}{4.863736in}}{\pgfqpoint{7.048465in}{4.857912in}}%
\pgfpathcurveto{\pgfqpoint{7.042641in}{4.852088in}}{\pgfqpoint{7.039369in}{4.844188in}}{\pgfqpoint{7.039369in}{4.835952in}}%
\pgfpathcurveto{\pgfqpoint{7.039369in}{4.827716in}}{\pgfqpoint{7.042641in}{4.819816in}}{\pgfqpoint{7.048465in}{4.813992in}}%
\pgfpathcurveto{\pgfqpoint{7.054289in}{4.808168in}}{\pgfqpoint{7.062189in}{4.804896in}}{\pgfqpoint{7.070425in}{4.804896in}}%
\pgfpathclose%
\pgfusepath{stroke,fill}%
\end{pgfscope}%
\begin{pgfscope}%
\pgfpathrectangle{\pgfqpoint{0.894063in}{3.540000in}}{\pgfqpoint{6.713438in}{2.060556in}} %
\pgfusepath{clip}%
\pgfsetbuttcap%
\pgfsetroundjoin%
\definecolor{currentfill}{rgb}{0.000000,0.000000,1.000000}%
\pgfsetfillcolor{currentfill}%
\pgfsetlinewidth{1.003750pt}%
\definecolor{currentstroke}{rgb}{0.000000,0.000000,0.000000}%
\pgfsetstrokecolor{currentstroke}%
\pgfsetdash{}{0pt}%
\pgfpathmoveto{\pgfqpoint{3.176631in}{5.126955in}}%
\pgfpathcurveto{\pgfqpoint{3.184868in}{5.126955in}}{\pgfqpoint{3.192768in}{5.130227in}}{\pgfqpoint{3.198592in}{5.136051in}}%
\pgfpathcurveto{\pgfqpoint{3.204415in}{5.141875in}}{\pgfqpoint{3.207688in}{5.149775in}}{\pgfqpoint{3.207688in}{5.158011in}}%
\pgfpathcurveto{\pgfqpoint{3.207688in}{5.166248in}}{\pgfqpoint{3.204415in}{5.174148in}}{\pgfqpoint{3.198592in}{5.179972in}}%
\pgfpathcurveto{\pgfqpoint{3.192768in}{5.185796in}}{\pgfqpoint{3.184868in}{5.189068in}}{\pgfqpoint{3.176631in}{5.189068in}}%
\pgfpathcurveto{\pgfqpoint{3.168395in}{5.189068in}}{\pgfqpoint{3.160495in}{5.185796in}}{\pgfqpoint{3.154671in}{5.179972in}}%
\pgfpathcurveto{\pgfqpoint{3.148847in}{5.174148in}}{\pgfqpoint{3.145575in}{5.166248in}}{\pgfqpoint{3.145575in}{5.158011in}}%
\pgfpathcurveto{\pgfqpoint{3.145575in}{5.149775in}}{\pgfqpoint{3.148847in}{5.141875in}}{\pgfqpoint{3.154671in}{5.136051in}}%
\pgfpathcurveto{\pgfqpoint{3.160495in}{5.130227in}}{\pgfqpoint{3.168395in}{5.126955in}}{\pgfqpoint{3.176631in}{5.126955in}}%
\pgfpathclose%
\pgfusepath{stroke,fill}%
\end{pgfscope}%
\begin{pgfscope}%
\pgfpathrectangle{\pgfqpoint{0.894063in}{3.540000in}}{\pgfqpoint{6.713438in}{2.060556in}} %
\pgfusepath{clip}%
\pgfsetbuttcap%
\pgfsetroundjoin%
\definecolor{currentfill}{rgb}{0.000000,0.000000,1.000000}%
\pgfsetfillcolor{currentfill}%
\pgfsetlinewidth{1.003750pt}%
\definecolor{currentstroke}{rgb}{0.000000,0.000000,0.000000}%
\pgfsetstrokecolor{currentstroke}%
\pgfsetdash{}{0pt}%
\pgfpathmoveto{\pgfqpoint{2.102481in}{5.127014in}}%
\pgfpathcurveto{\pgfqpoint{2.110718in}{5.127014in}}{\pgfqpoint{2.118618in}{5.130286in}}{\pgfqpoint{2.124442in}{5.136110in}}%
\pgfpathcurveto{\pgfqpoint{2.130265in}{5.141934in}}{\pgfqpoint{2.133538in}{5.149834in}}{\pgfqpoint{2.133538in}{5.158070in}}%
\pgfpathcurveto{\pgfqpoint{2.133538in}{5.166307in}}{\pgfqpoint{2.130265in}{5.174207in}}{\pgfqpoint{2.124442in}{5.180031in}}%
\pgfpathcurveto{\pgfqpoint{2.118618in}{5.185855in}}{\pgfqpoint{2.110718in}{5.189127in}}{\pgfqpoint{2.102481in}{5.189127in}}%
\pgfpathcurveto{\pgfqpoint{2.094245in}{5.189127in}}{\pgfqpoint{2.086345in}{5.185855in}}{\pgfqpoint{2.080521in}{5.180031in}}%
\pgfpathcurveto{\pgfqpoint{2.074697in}{5.174207in}}{\pgfqpoint{2.071425in}{5.166307in}}{\pgfqpoint{2.071425in}{5.158070in}}%
\pgfpathcurveto{\pgfqpoint{2.071425in}{5.149834in}}{\pgfqpoint{2.074697in}{5.141934in}}{\pgfqpoint{2.080521in}{5.136110in}}%
\pgfpathcurveto{\pgfqpoint{2.086345in}{5.130286in}}{\pgfqpoint{2.094245in}{5.127014in}}{\pgfqpoint{2.102481in}{5.127014in}}%
\pgfpathclose%
\pgfusepath{stroke,fill}%
\end{pgfscope}%
\begin{pgfscope}%
\pgfpathrectangle{\pgfqpoint{0.894063in}{3.540000in}}{\pgfqpoint{6.713438in}{2.060556in}} %
\pgfusepath{clip}%
\pgfsetbuttcap%
\pgfsetroundjoin%
\definecolor{currentfill}{rgb}{0.000000,0.000000,1.000000}%
\pgfsetfillcolor{currentfill}%
\pgfsetlinewidth{1.003750pt}%
\definecolor{currentstroke}{rgb}{0.000000,0.000000,0.000000}%
\pgfsetstrokecolor{currentstroke}%
\pgfsetdash{}{0pt}%
\pgfpathmoveto{\pgfqpoint{1.968213in}{5.127017in}}%
\pgfpathcurveto{\pgfqpoint{1.976449in}{5.127017in}}{\pgfqpoint{1.984349in}{5.130289in}}{\pgfqpoint{1.990173in}{5.136113in}}%
\pgfpathcurveto{\pgfqpoint{1.995997in}{5.141937in}}{\pgfqpoint{1.999269in}{5.149837in}}{\pgfqpoint{1.999269in}{5.158073in}}%
\pgfpathcurveto{\pgfqpoint{1.999269in}{5.166310in}}{\pgfqpoint{1.995997in}{5.174210in}}{\pgfqpoint{1.990173in}{5.180033in}}%
\pgfpathcurveto{\pgfqpoint{1.984349in}{5.185857in}}{\pgfqpoint{1.976449in}{5.189130in}}{\pgfqpoint{1.968213in}{5.189130in}}%
\pgfpathcurveto{\pgfqpoint{1.959976in}{5.189130in}}{\pgfqpoint{1.952076in}{5.185857in}}{\pgfqpoint{1.946252in}{5.180033in}}%
\pgfpathcurveto{\pgfqpoint{1.940428in}{5.174210in}}{\pgfqpoint{1.937156in}{5.166310in}}{\pgfqpoint{1.937156in}{5.158073in}}%
\pgfpathcurveto{\pgfqpoint{1.937156in}{5.149837in}}{\pgfqpoint{1.940428in}{5.141937in}}{\pgfqpoint{1.946252in}{5.136113in}}%
\pgfpathcurveto{\pgfqpoint{1.952076in}{5.130289in}}{\pgfqpoint{1.959976in}{5.127017in}}{\pgfqpoint{1.968213in}{5.127017in}}%
\pgfpathclose%
\pgfusepath{stroke,fill}%
\end{pgfscope}%
\begin{pgfscope}%
\pgfpathrectangle{\pgfqpoint{0.894063in}{3.540000in}}{\pgfqpoint{6.713438in}{2.060556in}} %
\pgfusepath{clip}%
\pgfsetbuttcap%
\pgfsetroundjoin%
\definecolor{currentfill}{rgb}{0.000000,0.000000,1.000000}%
\pgfsetfillcolor{currentfill}%
\pgfsetlinewidth{1.003750pt}%
\definecolor{currentstroke}{rgb}{0.000000,0.000000,0.000000}%
\pgfsetstrokecolor{currentstroke}%
\pgfsetdash{}{0pt}%
\pgfpathmoveto{\pgfqpoint{3.310900in}{5.126955in}}%
\pgfpathcurveto{\pgfqpoint{3.319136in}{5.126955in}}{\pgfqpoint{3.327036in}{5.130227in}}{\pgfqpoint{3.332860in}{5.136051in}}%
\pgfpathcurveto{\pgfqpoint{3.338684in}{5.141875in}}{\pgfqpoint{3.341956in}{5.149775in}}{\pgfqpoint{3.341956in}{5.158011in}}%
\pgfpathcurveto{\pgfqpoint{3.341956in}{5.166248in}}{\pgfqpoint{3.338684in}{5.174148in}}{\pgfqpoint{3.332860in}{5.179972in}}%
\pgfpathcurveto{\pgfqpoint{3.327036in}{5.185796in}}{\pgfqpoint{3.319136in}{5.189068in}}{\pgfqpoint{3.310900in}{5.189068in}}%
\pgfpathcurveto{\pgfqpoint{3.302664in}{5.189068in}}{\pgfqpoint{3.294764in}{5.185796in}}{\pgfqpoint{3.288940in}{5.179972in}}%
\pgfpathcurveto{\pgfqpoint{3.283116in}{5.174148in}}{\pgfqpoint{3.279844in}{5.166248in}}{\pgfqpoint{3.279844in}{5.158011in}}%
\pgfpathcurveto{\pgfqpoint{3.279844in}{5.149775in}}{\pgfqpoint{3.283116in}{5.141875in}}{\pgfqpoint{3.288940in}{5.136051in}}%
\pgfpathcurveto{\pgfqpoint{3.294764in}{5.130227in}}{\pgfqpoint{3.302664in}{5.126955in}}{\pgfqpoint{3.310900in}{5.126955in}}%
\pgfpathclose%
\pgfusepath{stroke,fill}%
\end{pgfscope}%
\begin{pgfscope}%
\pgfpathrectangle{\pgfqpoint{0.894063in}{3.540000in}}{\pgfqpoint{6.713438in}{2.060556in}} %
\pgfusepath{clip}%
\pgfsetbuttcap%
\pgfsetroundjoin%
\definecolor{currentfill}{rgb}{0.000000,0.000000,1.000000}%
\pgfsetfillcolor{currentfill}%
\pgfsetlinewidth{1.003750pt}%
\definecolor{currentstroke}{rgb}{0.000000,0.000000,0.000000}%
\pgfsetstrokecolor{currentstroke}%
\pgfsetdash{}{0pt}%
\pgfpathmoveto{\pgfqpoint{5.593469in}{5.016140in}}%
\pgfpathcurveto{\pgfqpoint{5.601705in}{5.016140in}}{\pgfqpoint{5.609605in}{5.019412in}}{\pgfqpoint{5.615429in}{5.025236in}}%
\pgfpathcurveto{\pgfqpoint{5.621253in}{5.031060in}}{\pgfqpoint{5.624525in}{5.038960in}}{\pgfqpoint{5.624525in}{5.047196in}}%
\pgfpathcurveto{\pgfqpoint{5.624525in}{5.055432in}}{\pgfqpoint{5.621253in}{5.063332in}}{\pgfqpoint{5.615429in}{5.069156in}}%
\pgfpathcurveto{\pgfqpoint{5.609605in}{5.074980in}}{\pgfqpoint{5.601705in}{5.078253in}}{\pgfqpoint{5.593469in}{5.078253in}}%
\pgfpathcurveto{\pgfqpoint{5.585232in}{5.078253in}}{\pgfqpoint{5.577332in}{5.074980in}}{\pgfqpoint{5.571508in}{5.069156in}}%
\pgfpathcurveto{\pgfqpoint{5.565685in}{5.063332in}}{\pgfqpoint{5.562412in}{5.055432in}}{\pgfqpoint{5.562412in}{5.047196in}}%
\pgfpathcurveto{\pgfqpoint{5.562412in}{5.038960in}}{\pgfqpoint{5.565685in}{5.031060in}}{\pgfqpoint{5.571508in}{5.025236in}}%
\pgfpathcurveto{\pgfqpoint{5.577332in}{5.019412in}}{\pgfqpoint{5.585232in}{5.016140in}}{\pgfqpoint{5.593469in}{5.016140in}}%
\pgfpathclose%
\pgfusepath{stroke,fill}%
\end{pgfscope}%
\begin{pgfscope}%
\pgfpathrectangle{\pgfqpoint{0.894063in}{3.540000in}}{\pgfqpoint{6.713438in}{2.060556in}} %
\pgfusepath{clip}%
\pgfsetbuttcap%
\pgfsetroundjoin%
\definecolor{currentfill}{rgb}{0.000000,0.000000,1.000000}%
\pgfsetfillcolor{currentfill}%
\pgfsetlinewidth{1.003750pt}%
\definecolor{currentstroke}{rgb}{0.000000,0.000000,0.000000}%
\pgfsetstrokecolor{currentstroke}%
\pgfsetdash{}{0pt}%
\pgfpathmoveto{\pgfqpoint{3.042363in}{5.126965in}}%
\pgfpathcurveto{\pgfqpoint{3.050599in}{5.126965in}}{\pgfqpoint{3.058499in}{5.130237in}}{\pgfqpoint{3.064323in}{5.136061in}}%
\pgfpathcurveto{\pgfqpoint{3.070147in}{5.141885in}}{\pgfqpoint{3.073419in}{5.149785in}}{\pgfqpoint{3.073419in}{5.158021in}}%
\pgfpathcurveto{\pgfqpoint{3.073419in}{5.166257in}}{\pgfqpoint{3.070147in}{5.174157in}}{\pgfqpoint{3.064323in}{5.179981in}}%
\pgfpathcurveto{\pgfqpoint{3.058499in}{5.185805in}}{\pgfqpoint{3.050599in}{5.189078in}}{\pgfqpoint{3.042363in}{5.189078in}}%
\pgfpathcurveto{\pgfqpoint{3.034126in}{5.189078in}}{\pgfqpoint{3.026226in}{5.185805in}}{\pgfqpoint{3.020402in}{5.179981in}}%
\pgfpathcurveto{\pgfqpoint{3.014578in}{5.174157in}}{\pgfqpoint{3.011306in}{5.166257in}}{\pgfqpoint{3.011306in}{5.158021in}}%
\pgfpathcurveto{\pgfqpoint{3.011306in}{5.149785in}}{\pgfqpoint{3.014578in}{5.141885in}}{\pgfqpoint{3.020402in}{5.136061in}}%
\pgfpathcurveto{\pgfqpoint{3.026226in}{5.130237in}}{\pgfqpoint{3.034126in}{5.126965in}}{\pgfqpoint{3.042363in}{5.126965in}}%
\pgfpathclose%
\pgfusepath{stroke,fill}%
\end{pgfscope}%
\begin{pgfscope}%
\pgfpathrectangle{\pgfqpoint{0.894063in}{3.540000in}}{\pgfqpoint{6.713438in}{2.060556in}} %
\pgfusepath{clip}%
\pgfsetbuttcap%
\pgfsetroundjoin%
\definecolor{currentfill}{rgb}{0.000000,0.000000,1.000000}%
\pgfsetfillcolor{currentfill}%
\pgfsetlinewidth{1.003750pt}%
\definecolor{currentstroke}{rgb}{0.000000,0.000000,0.000000}%
\pgfsetstrokecolor{currentstroke}%
\pgfsetdash{}{0pt}%
\pgfpathmoveto{\pgfqpoint{5.190663in}{5.076033in}}%
\pgfpathcurveto{\pgfqpoint{5.198899in}{5.076033in}}{\pgfqpoint{5.206799in}{5.079305in}}{\pgfqpoint{5.212623in}{5.085129in}}%
\pgfpathcurveto{\pgfqpoint{5.218447in}{5.090953in}}{\pgfqpoint{5.221719in}{5.098853in}}{\pgfqpoint{5.221719in}{5.107090in}}%
\pgfpathcurveto{\pgfqpoint{5.221719in}{5.115326in}}{\pgfqpoint{5.218447in}{5.123226in}}{\pgfqpoint{5.212623in}{5.129050in}}%
\pgfpathcurveto{\pgfqpoint{5.206799in}{5.134874in}}{\pgfqpoint{5.198899in}{5.138146in}}{\pgfqpoint{5.190663in}{5.138146in}}%
\pgfpathcurveto{\pgfqpoint{5.182426in}{5.138146in}}{\pgfqpoint{5.174526in}{5.134874in}}{\pgfqpoint{5.168702in}{5.129050in}}%
\pgfpathcurveto{\pgfqpoint{5.162878in}{5.123226in}}{\pgfqpoint{5.159606in}{5.115326in}}{\pgfqpoint{5.159606in}{5.107090in}}%
\pgfpathcurveto{\pgfqpoint{5.159606in}{5.098853in}}{\pgfqpoint{5.162878in}{5.090953in}}{\pgfqpoint{5.168702in}{5.085129in}}%
\pgfpathcurveto{\pgfqpoint{5.174526in}{5.079305in}}{\pgfqpoint{5.182426in}{5.076033in}}{\pgfqpoint{5.190663in}{5.076033in}}%
\pgfpathclose%
\pgfusepath{stroke,fill}%
\end{pgfscope}%
\begin{pgfscope}%
\pgfpathrectangle{\pgfqpoint{0.894063in}{3.540000in}}{\pgfqpoint{6.713438in}{2.060556in}} %
\pgfusepath{clip}%
\pgfsetbuttcap%
\pgfsetroundjoin%
\definecolor{currentfill}{rgb}{0.000000,0.000000,1.000000}%
\pgfsetfillcolor{currentfill}%
\pgfsetlinewidth{1.003750pt}%
\definecolor{currentstroke}{rgb}{0.000000,0.000000,0.000000}%
\pgfsetstrokecolor{currentstroke}%
\pgfsetdash{}{0pt}%
\pgfpathmoveto{\pgfqpoint{6.801888in}{4.842907in}}%
\pgfpathcurveto{\pgfqpoint{6.810124in}{4.842907in}}{\pgfqpoint{6.818024in}{4.846180in}}{\pgfqpoint{6.823848in}{4.852004in}}%
\pgfpathcurveto{\pgfqpoint{6.829672in}{4.857827in}}{\pgfqpoint{6.832944in}{4.865728in}}{\pgfqpoint{6.832944in}{4.873964in}}%
\pgfpathcurveto{\pgfqpoint{6.832944in}{4.882200in}}{\pgfqpoint{6.829672in}{4.890100in}}{\pgfqpoint{6.823848in}{4.895924in}}%
\pgfpathcurveto{\pgfqpoint{6.818024in}{4.901748in}}{\pgfqpoint{6.810124in}{4.905020in}}{\pgfqpoint{6.801888in}{4.905020in}}%
\pgfpathcurveto{\pgfqpoint{6.793651in}{4.905020in}}{\pgfqpoint{6.785751in}{4.901748in}}{\pgfqpoint{6.779927in}{4.895924in}}%
\pgfpathcurveto{\pgfqpoint{6.774103in}{4.890100in}}{\pgfqpoint{6.770831in}{4.882200in}}{\pgfqpoint{6.770831in}{4.873964in}}%
\pgfpathcurveto{\pgfqpoint{6.770831in}{4.865728in}}{\pgfqpoint{6.774103in}{4.857827in}}{\pgfqpoint{6.779927in}{4.852004in}}%
\pgfpathcurveto{\pgfqpoint{6.785751in}{4.846180in}}{\pgfqpoint{6.793651in}{4.842907in}}{\pgfqpoint{6.801888in}{4.842907in}}%
\pgfpathclose%
\pgfusepath{stroke,fill}%
\end{pgfscope}%
\begin{pgfscope}%
\pgfpathrectangle{\pgfqpoint{0.894063in}{3.540000in}}{\pgfqpoint{6.713438in}{2.060556in}} %
\pgfusepath{clip}%
\pgfsetbuttcap%
\pgfsetroundjoin%
\definecolor{currentfill}{rgb}{0.000000,0.000000,1.000000}%
\pgfsetfillcolor{currentfill}%
\pgfsetlinewidth{1.003750pt}%
\definecolor{currentstroke}{rgb}{0.000000,0.000000,0.000000}%
\pgfsetstrokecolor{currentstroke}%
\pgfsetdash{}{0pt}%
\pgfpathmoveto{\pgfqpoint{3.579438in}{5.126943in}}%
\pgfpathcurveto{\pgfqpoint{3.587674in}{5.126943in}}{\pgfqpoint{3.595574in}{5.130215in}}{\pgfqpoint{3.601398in}{5.136039in}}%
\pgfpathcurveto{\pgfqpoint{3.607222in}{5.141863in}}{\pgfqpoint{3.610494in}{5.149763in}}{\pgfqpoint{3.610494in}{5.157999in}}%
\pgfpathcurveto{\pgfqpoint{3.610494in}{5.166235in}}{\pgfqpoint{3.607222in}{5.174135in}}{\pgfqpoint{3.601398in}{5.179959in}}%
\pgfpathcurveto{\pgfqpoint{3.595574in}{5.185783in}}{\pgfqpoint{3.587674in}{5.189056in}}{\pgfqpoint{3.579438in}{5.189056in}}%
\pgfpathcurveto{\pgfqpoint{3.571201in}{5.189056in}}{\pgfqpoint{3.563301in}{5.185783in}}{\pgfqpoint{3.557477in}{5.179959in}}%
\pgfpathcurveto{\pgfqpoint{3.551653in}{5.174135in}}{\pgfqpoint{3.548381in}{5.166235in}}{\pgfqpoint{3.548381in}{5.157999in}}%
\pgfpathcurveto{\pgfqpoint{3.548381in}{5.149763in}}{\pgfqpoint{3.551653in}{5.141863in}}{\pgfqpoint{3.557477in}{5.136039in}}%
\pgfpathcurveto{\pgfqpoint{3.563301in}{5.130215in}}{\pgfqpoint{3.571201in}{5.126943in}}{\pgfqpoint{3.579438in}{5.126943in}}%
\pgfpathclose%
\pgfusepath{stroke,fill}%
\end{pgfscope}%
\begin{pgfscope}%
\pgfpathrectangle{\pgfqpoint{0.894063in}{3.540000in}}{\pgfqpoint{6.713438in}{2.060556in}} %
\pgfusepath{clip}%
\pgfsetbuttcap%
\pgfsetroundjoin%
\definecolor{currentfill}{rgb}{0.000000,0.000000,1.000000}%
\pgfsetfillcolor{currentfill}%
\pgfsetlinewidth{1.003750pt}%
\definecolor{currentstroke}{rgb}{0.000000,0.000000,0.000000}%
\pgfsetstrokecolor{currentstroke}%
\pgfsetdash{}{0pt}%
\pgfpathmoveto{\pgfqpoint{2.371019in}{5.126980in}}%
\pgfpathcurveto{\pgfqpoint{2.379255in}{5.126980in}}{\pgfqpoint{2.387155in}{5.130252in}}{\pgfqpoint{2.392979in}{5.136076in}}%
\pgfpathcurveto{\pgfqpoint{2.398803in}{5.141900in}}{\pgfqpoint{2.402075in}{5.149800in}}{\pgfqpoint{2.402075in}{5.158036in}}%
\pgfpathcurveto{\pgfqpoint{2.402075in}{5.166272in}}{\pgfqpoint{2.398803in}{5.174172in}}{\pgfqpoint{2.392979in}{5.179996in}}%
\pgfpathcurveto{\pgfqpoint{2.387155in}{5.185820in}}{\pgfqpoint{2.379255in}{5.189093in}}{\pgfqpoint{2.371019in}{5.189093in}}%
\pgfpathcurveto{\pgfqpoint{2.362782in}{5.189093in}}{\pgfqpoint{2.354882in}{5.185820in}}{\pgfqpoint{2.349058in}{5.179996in}}%
\pgfpathcurveto{\pgfqpoint{2.343235in}{5.174172in}}{\pgfqpoint{2.339962in}{5.166272in}}{\pgfqpoint{2.339962in}{5.158036in}}%
\pgfpathcurveto{\pgfqpoint{2.339962in}{5.149800in}}{\pgfqpoint{2.343235in}{5.141900in}}{\pgfqpoint{2.349058in}{5.136076in}}%
\pgfpathcurveto{\pgfqpoint{2.354882in}{5.130252in}}{\pgfqpoint{2.362782in}{5.126980in}}{\pgfqpoint{2.371019in}{5.126980in}}%
\pgfpathclose%
\pgfusepath{stroke,fill}%
\end{pgfscope}%
\begin{pgfscope}%
\pgfpathrectangle{\pgfqpoint{0.894063in}{3.540000in}}{\pgfqpoint{6.713438in}{2.060556in}} %
\pgfusepath{clip}%
\pgfsetbuttcap%
\pgfsetroundjoin%
\definecolor{currentfill}{rgb}{0.000000,0.000000,1.000000}%
\pgfsetfillcolor{currentfill}%
\pgfsetlinewidth{1.003750pt}%
\definecolor{currentstroke}{rgb}{0.000000,0.000000,0.000000}%
\pgfsetstrokecolor{currentstroke}%
\pgfsetdash{}{0pt}%
\pgfpathmoveto{\pgfqpoint{3.982244in}{5.126938in}}%
\pgfpathcurveto{\pgfqpoint{3.990480in}{5.126938in}}{\pgfqpoint{3.998380in}{5.130211in}}{\pgfqpoint{4.004204in}{5.136035in}}%
\pgfpathcurveto{\pgfqpoint{4.010028in}{5.141859in}}{\pgfqpoint{4.013300in}{5.149759in}}{\pgfqpoint{4.013300in}{5.157995in}}%
\pgfpathcurveto{\pgfqpoint{4.013300in}{5.166231in}}{\pgfqpoint{4.010028in}{5.174131in}}{\pgfqpoint{4.004204in}{5.179955in}}%
\pgfpathcurveto{\pgfqpoint{3.998380in}{5.185779in}}{\pgfqpoint{3.990480in}{5.189051in}}{\pgfqpoint{3.982244in}{5.189051in}}%
\pgfpathcurveto{\pgfqpoint{3.974007in}{5.189051in}}{\pgfqpoint{3.966107in}{5.185779in}}{\pgfqpoint{3.960283in}{5.179955in}}%
\pgfpathcurveto{\pgfqpoint{3.954460in}{5.174131in}}{\pgfqpoint{3.951187in}{5.166231in}}{\pgfqpoint{3.951187in}{5.157995in}}%
\pgfpathcurveto{\pgfqpoint{3.951187in}{5.149759in}}{\pgfqpoint{3.954460in}{5.141859in}}{\pgfqpoint{3.960283in}{5.136035in}}%
\pgfpathcurveto{\pgfqpoint{3.966107in}{5.130211in}}{\pgfqpoint{3.974007in}{5.126938in}}{\pgfqpoint{3.982244in}{5.126938in}}%
\pgfpathclose%
\pgfusepath{stroke,fill}%
\end{pgfscope}%
\begin{pgfscope}%
\pgfpathrectangle{\pgfqpoint{0.894063in}{3.540000in}}{\pgfqpoint{6.713438in}{2.060556in}} %
\pgfusepath{clip}%
\pgfsetbuttcap%
\pgfsetroundjoin%
\definecolor{currentfill}{rgb}{0.000000,0.000000,1.000000}%
\pgfsetfillcolor{currentfill}%
\pgfsetlinewidth{1.003750pt}%
\definecolor{currentstroke}{rgb}{0.000000,0.000000,0.000000}%
\pgfsetstrokecolor{currentstroke}%
\pgfsetdash{}{0pt}%
\pgfpathmoveto{\pgfqpoint{4.653588in}{5.126925in}}%
\pgfpathcurveto{\pgfqpoint{4.661824in}{5.126925in}}{\pgfqpoint{4.669724in}{5.130197in}}{\pgfqpoint{4.675548in}{5.136021in}}%
\pgfpathcurveto{\pgfqpoint{4.681372in}{5.141845in}}{\pgfqpoint{4.684644in}{5.149745in}}{\pgfqpoint{4.684644in}{5.157981in}}%
\pgfpathcurveto{\pgfqpoint{4.684644in}{5.166217in}}{\pgfqpoint{4.681372in}{5.174118in}}{\pgfqpoint{4.675548in}{5.179941in}}%
\pgfpathcurveto{\pgfqpoint{4.669724in}{5.185765in}}{\pgfqpoint{4.661824in}{5.189038in}}{\pgfqpoint{4.653588in}{5.189038in}}%
\pgfpathcurveto{\pgfqpoint{4.645351in}{5.189038in}}{\pgfqpoint{4.637451in}{5.185765in}}{\pgfqpoint{4.631627in}{5.179941in}}%
\pgfpathcurveto{\pgfqpoint{4.625803in}{5.174118in}}{\pgfqpoint{4.622531in}{5.166217in}}{\pgfqpoint{4.622531in}{5.157981in}}%
\pgfpathcurveto{\pgfqpoint{4.622531in}{5.149745in}}{\pgfqpoint{4.625803in}{5.141845in}}{\pgfqpoint{4.631627in}{5.136021in}}%
\pgfpathcurveto{\pgfqpoint{4.637451in}{5.130197in}}{\pgfqpoint{4.645351in}{5.126925in}}{\pgfqpoint{4.653588in}{5.126925in}}%
\pgfpathclose%
\pgfusepath{stroke,fill}%
\end{pgfscope}%
\begin{pgfscope}%
\pgfpathrectangle{\pgfqpoint{0.894063in}{3.540000in}}{\pgfqpoint{6.713438in}{2.060556in}} %
\pgfusepath{clip}%
\pgfsetbuttcap%
\pgfsetroundjoin%
\definecolor{currentfill}{rgb}{0.000000,0.000000,1.000000}%
\pgfsetfillcolor{currentfill}%
\pgfsetlinewidth{1.003750pt}%
\definecolor{currentstroke}{rgb}{0.000000,0.000000,0.000000}%
\pgfsetstrokecolor{currentstroke}%
\pgfsetdash{}{0pt}%
\pgfpathmoveto{\pgfqpoint{3.713706in}{5.126943in}}%
\pgfpathcurveto{\pgfqpoint{3.721943in}{5.126943in}}{\pgfqpoint{3.729843in}{5.130215in}}{\pgfqpoint{3.735667in}{5.136039in}}%
\pgfpathcurveto{\pgfqpoint{3.741490in}{5.141863in}}{\pgfqpoint{3.744763in}{5.149763in}}{\pgfqpoint{3.744763in}{5.157999in}}%
\pgfpathcurveto{\pgfqpoint{3.744763in}{5.166235in}}{\pgfqpoint{3.741490in}{5.174135in}}{\pgfqpoint{3.735667in}{5.179959in}}%
\pgfpathcurveto{\pgfqpoint{3.729843in}{5.185783in}}{\pgfqpoint{3.721943in}{5.189056in}}{\pgfqpoint{3.713706in}{5.189056in}}%
\pgfpathcurveto{\pgfqpoint{3.705470in}{5.189056in}}{\pgfqpoint{3.697570in}{5.185783in}}{\pgfqpoint{3.691746in}{5.179959in}}%
\pgfpathcurveto{\pgfqpoint{3.685922in}{5.174135in}}{\pgfqpoint{3.682650in}{5.166235in}}{\pgfqpoint{3.682650in}{5.157999in}}%
\pgfpathcurveto{\pgfqpoint{3.682650in}{5.149763in}}{\pgfqpoint{3.685922in}{5.141863in}}{\pgfqpoint{3.691746in}{5.136039in}}%
\pgfpathcurveto{\pgfqpoint{3.697570in}{5.130215in}}{\pgfqpoint{3.705470in}{5.126943in}}{\pgfqpoint{3.713706in}{5.126943in}}%
\pgfpathclose%
\pgfusepath{stroke,fill}%
\end{pgfscope}%
\begin{pgfscope}%
\pgfpathrectangle{\pgfqpoint{0.894063in}{3.540000in}}{\pgfqpoint{6.713438in}{2.060556in}} %
\pgfusepath{clip}%
\pgfsetbuttcap%
\pgfsetroundjoin%
\definecolor{currentfill}{rgb}{0.000000,0.000000,1.000000}%
\pgfsetfillcolor{currentfill}%
\pgfsetlinewidth{1.003750pt}%
\definecolor{currentstroke}{rgb}{0.000000,0.000000,0.000000}%
\pgfsetstrokecolor{currentstroke}%
\pgfsetdash{}{0pt}%
\pgfpathmoveto{\pgfqpoint{2.236750in}{5.127014in}}%
\pgfpathcurveto{\pgfqpoint{2.244986in}{5.127014in}}{\pgfqpoint{2.252886in}{5.130286in}}{\pgfqpoint{2.258710in}{5.136110in}}%
\pgfpathcurveto{\pgfqpoint{2.264534in}{5.141934in}}{\pgfqpoint{2.267806in}{5.149834in}}{\pgfqpoint{2.267806in}{5.158070in}}%
\pgfpathcurveto{\pgfqpoint{2.267806in}{5.166307in}}{\pgfqpoint{2.264534in}{5.174207in}}{\pgfqpoint{2.258710in}{5.180031in}}%
\pgfpathcurveto{\pgfqpoint{2.252886in}{5.185855in}}{\pgfqpoint{2.244986in}{5.189127in}}{\pgfqpoint{2.236750in}{5.189127in}}%
\pgfpathcurveto{\pgfqpoint{2.228514in}{5.189127in}}{\pgfqpoint{2.220614in}{5.185855in}}{\pgfqpoint{2.214790in}{5.180031in}}%
\pgfpathcurveto{\pgfqpoint{2.208966in}{5.174207in}}{\pgfqpoint{2.205694in}{5.166307in}}{\pgfqpoint{2.205694in}{5.158070in}}%
\pgfpathcurveto{\pgfqpoint{2.205694in}{5.149834in}}{\pgfqpoint{2.208966in}{5.141934in}}{\pgfqpoint{2.214790in}{5.136110in}}%
\pgfpathcurveto{\pgfqpoint{2.220614in}{5.130286in}}{\pgfqpoint{2.228514in}{5.127014in}}{\pgfqpoint{2.236750in}{5.127014in}}%
\pgfpathclose%
\pgfusepath{stroke,fill}%
\end{pgfscope}%
\begin{pgfscope}%
\pgfpathrectangle{\pgfqpoint{0.894063in}{3.540000in}}{\pgfqpoint{6.713438in}{2.060556in}} %
\pgfusepath{clip}%
\pgfsetbuttcap%
\pgfsetroundjoin%
\definecolor{currentfill}{rgb}{0.000000,0.000000,1.000000}%
\pgfsetfillcolor{currentfill}%
\pgfsetlinewidth{1.003750pt}%
\definecolor{currentstroke}{rgb}{0.000000,0.000000,0.000000}%
\pgfsetstrokecolor{currentstroke}%
\pgfsetdash{}{0pt}%
\pgfpathmoveto{\pgfqpoint{6.667619in}{4.862389in}}%
\pgfpathcurveto{\pgfqpoint{6.675855in}{4.862389in}}{\pgfqpoint{6.683755in}{4.865662in}}{\pgfqpoint{6.689579in}{4.871485in}}%
\pgfpathcurveto{\pgfqpoint{6.695403in}{4.877309in}}{\pgfqpoint{6.698675in}{4.885209in}}{\pgfqpoint{6.698675in}{4.893446in}}%
\pgfpathcurveto{\pgfqpoint{6.698675in}{4.901682in}}{\pgfqpoint{6.695403in}{4.909582in}}{\pgfqpoint{6.689579in}{4.915406in}}%
\pgfpathcurveto{\pgfqpoint{6.683755in}{4.921230in}}{\pgfqpoint{6.675855in}{4.924502in}}{\pgfqpoint{6.667619in}{4.924502in}}%
\pgfpathcurveto{\pgfqpoint{6.659382in}{4.924502in}}{\pgfqpoint{6.651482in}{4.921230in}}{\pgfqpoint{6.645658in}{4.915406in}}%
\pgfpathcurveto{\pgfqpoint{6.639835in}{4.909582in}}{\pgfqpoint{6.636562in}{4.901682in}}{\pgfqpoint{6.636562in}{4.893446in}}%
\pgfpathcurveto{\pgfqpoint{6.636562in}{4.885209in}}{\pgfqpoint{6.639835in}{4.877309in}}{\pgfqpoint{6.645658in}{4.871485in}}%
\pgfpathcurveto{\pgfqpoint{6.651482in}{4.865662in}}{\pgfqpoint{6.659382in}{4.862389in}}{\pgfqpoint{6.667619in}{4.862389in}}%
\pgfpathclose%
\pgfusepath{stroke,fill}%
\end{pgfscope}%
\begin{pgfscope}%
\pgfpathrectangle{\pgfqpoint{0.894063in}{3.540000in}}{\pgfqpoint{6.713438in}{2.060556in}} %
\pgfusepath{clip}%
\pgfsetbuttcap%
\pgfsetroundjoin%
\definecolor{currentfill}{rgb}{0.000000,0.000000,1.000000}%
\pgfsetfillcolor{currentfill}%
\pgfsetlinewidth{1.003750pt}%
\definecolor{currentstroke}{rgb}{0.000000,0.000000,0.000000}%
\pgfsetstrokecolor{currentstroke}%
\pgfsetdash{}{0pt}%
\pgfpathmoveto{\pgfqpoint{2.639556in}{5.126995in}}%
\pgfpathcurveto{\pgfqpoint{2.647793in}{5.126995in}}{\pgfqpoint{2.655693in}{5.130267in}}{\pgfqpoint{2.661517in}{5.136091in}}%
\pgfpathcurveto{\pgfqpoint{2.667340in}{5.141915in}}{\pgfqpoint{2.670613in}{5.149815in}}{\pgfqpoint{2.670613in}{5.158051in}}%
\pgfpathcurveto{\pgfqpoint{2.670613in}{5.166288in}}{\pgfqpoint{2.667340in}{5.174188in}}{\pgfqpoint{2.661517in}{5.180012in}}%
\pgfpathcurveto{\pgfqpoint{2.655693in}{5.185835in}}{\pgfqpoint{2.647793in}{5.189108in}}{\pgfqpoint{2.639556in}{5.189108in}}%
\pgfpathcurveto{\pgfqpoint{2.631320in}{5.189108in}}{\pgfqpoint{2.623420in}{5.185835in}}{\pgfqpoint{2.617596in}{5.180012in}}%
\pgfpathcurveto{\pgfqpoint{2.611772in}{5.174188in}}{\pgfqpoint{2.608500in}{5.166288in}}{\pgfqpoint{2.608500in}{5.158051in}}%
\pgfpathcurveto{\pgfqpoint{2.608500in}{5.149815in}}{\pgfqpoint{2.611772in}{5.141915in}}{\pgfqpoint{2.617596in}{5.136091in}}%
\pgfpathcurveto{\pgfqpoint{2.623420in}{5.130267in}}{\pgfqpoint{2.631320in}{5.126995in}}{\pgfqpoint{2.639556in}{5.126995in}}%
\pgfpathclose%
\pgfusepath{stroke,fill}%
\end{pgfscope}%
\begin{pgfscope}%
\pgfpathrectangle{\pgfqpoint{0.894063in}{3.540000in}}{\pgfqpoint{6.713438in}{2.060556in}} %
\pgfusepath{clip}%
\pgfsetbuttcap%
\pgfsetroundjoin%
\definecolor{currentfill}{rgb}{0.000000,0.000000,1.000000}%
\pgfsetfillcolor{currentfill}%
\pgfsetlinewidth{1.003750pt}%
\definecolor{currentstroke}{rgb}{0.000000,0.000000,0.000000}%
\pgfsetstrokecolor{currentstroke}%
\pgfsetdash{}{0pt}%
\pgfpathmoveto{\pgfqpoint{1.699675in}{5.137929in}}%
\pgfpathcurveto{\pgfqpoint{1.707911in}{5.137929in}}{\pgfqpoint{1.715811in}{5.141202in}}{\pgfqpoint{1.721635in}{5.147026in}}%
\pgfpathcurveto{\pgfqpoint{1.727459in}{5.152850in}}{\pgfqpoint{1.730731in}{5.160750in}}{\pgfqpoint{1.730731in}{5.168986in}}%
\pgfpathcurveto{\pgfqpoint{1.730731in}{5.177222in}}{\pgfqpoint{1.727459in}{5.185122in}}{\pgfqpoint{1.721635in}{5.190946in}}%
\pgfpathcurveto{\pgfqpoint{1.715811in}{5.196770in}}{\pgfqpoint{1.707911in}{5.200042in}}{\pgfqpoint{1.699675in}{5.200042in}}%
\pgfpathcurveto{\pgfqpoint{1.691439in}{5.200042in}}{\pgfqpoint{1.683539in}{5.196770in}}{\pgfqpoint{1.677715in}{5.190946in}}%
\pgfpathcurveto{\pgfqpoint{1.671891in}{5.185122in}}{\pgfqpoint{1.668619in}{5.177222in}}{\pgfqpoint{1.668619in}{5.168986in}}%
\pgfpathcurveto{\pgfqpoint{1.668619in}{5.160750in}}{\pgfqpoint{1.671891in}{5.152850in}}{\pgfqpoint{1.677715in}{5.147026in}}%
\pgfpathcurveto{\pgfqpoint{1.683539in}{5.141202in}}{\pgfqpoint{1.691439in}{5.137929in}}{\pgfqpoint{1.699675in}{5.137929in}}%
\pgfpathclose%
\pgfusepath{stroke,fill}%
\end{pgfscope}%
\begin{pgfscope}%
\pgfpathrectangle{\pgfqpoint{0.894063in}{3.540000in}}{\pgfqpoint{6.713438in}{2.060556in}} %
\pgfusepath{clip}%
\pgfsetbuttcap%
\pgfsetroundjoin%
\definecolor{currentfill}{rgb}{0.000000,0.000000,1.000000}%
\pgfsetfillcolor{currentfill}%
\pgfsetlinewidth{1.003750pt}%
\definecolor{currentstroke}{rgb}{0.000000,0.000000,0.000000}%
\pgfsetstrokecolor{currentstroke}%
\pgfsetdash{}{0pt}%
\pgfpathmoveto{\pgfqpoint{1.162600in}{5.264757in}}%
\pgfpathcurveto{\pgfqpoint{1.170836in}{5.264757in}}{\pgfqpoint{1.178736in}{5.268029in}}{\pgfqpoint{1.184560in}{5.273853in}}%
\pgfpathcurveto{\pgfqpoint{1.190384in}{5.279677in}}{\pgfqpoint{1.193656in}{5.287577in}}{\pgfqpoint{1.193656in}{5.295813in}}%
\pgfpathcurveto{\pgfqpoint{1.193656in}{5.304049in}}{\pgfqpoint{1.190384in}{5.311949in}}{\pgfqpoint{1.184560in}{5.317773in}}%
\pgfpathcurveto{\pgfqpoint{1.178736in}{5.323597in}}{\pgfqpoint{1.170836in}{5.326870in}}{\pgfqpoint{1.162600in}{5.326870in}}%
\pgfpathcurveto{\pgfqpoint{1.154364in}{5.326870in}}{\pgfqpoint{1.146464in}{5.323597in}}{\pgfqpoint{1.140640in}{5.317773in}}%
\pgfpathcurveto{\pgfqpoint{1.134816in}{5.311949in}}{\pgfqpoint{1.131544in}{5.304049in}}{\pgfqpoint{1.131544in}{5.295813in}}%
\pgfpathcurveto{\pgfqpoint{1.131544in}{5.287577in}}{\pgfqpoint{1.134816in}{5.279677in}}{\pgfqpoint{1.140640in}{5.273853in}}%
\pgfpathcurveto{\pgfqpoint{1.146464in}{5.268029in}}{\pgfqpoint{1.154364in}{5.264757in}}{\pgfqpoint{1.162600in}{5.264757in}}%
\pgfpathclose%
\pgfusepath{stroke,fill}%
\end{pgfscope}%
\begin{pgfscope}%
\pgfpathrectangle{\pgfqpoint{0.894063in}{3.540000in}}{\pgfqpoint{6.713438in}{2.060556in}} %
\pgfusepath{clip}%
\pgfsetbuttcap%
\pgfsetroundjoin%
\definecolor{currentfill}{rgb}{0.000000,0.000000,1.000000}%
\pgfsetfillcolor{currentfill}%
\pgfsetlinewidth{1.003750pt}%
\definecolor{currentstroke}{rgb}{0.000000,0.000000,0.000000}%
\pgfsetstrokecolor{currentstroke}%
\pgfsetdash{}{0pt}%
\pgfpathmoveto{\pgfqpoint{1.833944in}{5.134558in}}%
\pgfpathcurveto{\pgfqpoint{1.842180in}{5.134558in}}{\pgfqpoint{1.850080in}{5.137831in}}{\pgfqpoint{1.855904in}{5.143655in}}%
\pgfpathcurveto{\pgfqpoint{1.861728in}{5.149479in}}{\pgfqpoint{1.865000in}{5.157379in}}{\pgfqpoint{1.865000in}{5.165615in}}%
\pgfpathcurveto{\pgfqpoint{1.865000in}{5.173851in}}{\pgfqpoint{1.861728in}{5.181751in}}{\pgfqpoint{1.855904in}{5.187575in}}%
\pgfpathcurveto{\pgfqpoint{1.850080in}{5.193399in}}{\pgfqpoint{1.842180in}{5.196671in}}{\pgfqpoint{1.833944in}{5.196671in}}%
\pgfpathcurveto{\pgfqpoint{1.825707in}{5.196671in}}{\pgfqpoint{1.817807in}{5.193399in}}{\pgfqpoint{1.811983in}{5.187575in}}%
\pgfpathcurveto{\pgfqpoint{1.806160in}{5.181751in}}{\pgfqpoint{1.802887in}{5.173851in}}{\pgfqpoint{1.802887in}{5.165615in}}%
\pgfpathcurveto{\pgfqpoint{1.802887in}{5.157379in}}{\pgfqpoint{1.806160in}{5.149479in}}{\pgfqpoint{1.811983in}{5.143655in}}%
\pgfpathcurveto{\pgfqpoint{1.817807in}{5.137831in}}{\pgfqpoint{1.825707in}{5.134558in}}{\pgfqpoint{1.833944in}{5.134558in}}%
\pgfpathclose%
\pgfusepath{stroke,fill}%
\end{pgfscope}%
\begin{pgfscope}%
\pgfpathrectangle{\pgfqpoint{0.894063in}{3.540000in}}{\pgfqpoint{6.713438in}{2.060556in}} %
\pgfusepath{clip}%
\pgfsetbuttcap%
\pgfsetroundjoin%
\definecolor{currentfill}{rgb}{0.000000,0.000000,1.000000}%
\pgfsetfillcolor{currentfill}%
\pgfsetlinewidth{1.003750pt}%
\definecolor{currentstroke}{rgb}{0.000000,0.000000,0.000000}%
\pgfsetstrokecolor{currentstroke}%
\pgfsetdash{}{0pt}%
\pgfpathmoveto{\pgfqpoint{5.996275in}{4.978065in}}%
\pgfpathcurveto{\pgfqpoint{6.004511in}{4.978065in}}{\pgfqpoint{6.012411in}{4.981337in}}{\pgfqpoint{6.018235in}{4.987161in}}%
\pgfpathcurveto{\pgfqpoint{6.024059in}{4.992985in}}{\pgfqpoint{6.027331in}{5.000885in}}{\pgfqpoint{6.027331in}{5.009121in}}%
\pgfpathcurveto{\pgfqpoint{6.027331in}{5.017357in}}{\pgfqpoint{6.024059in}{5.025257in}}{\pgfqpoint{6.018235in}{5.031081in}}%
\pgfpathcurveto{\pgfqpoint{6.012411in}{5.036905in}}{\pgfqpoint{6.004511in}{5.040178in}}{\pgfqpoint{5.996275in}{5.040178in}}%
\pgfpathcurveto{\pgfqpoint{5.988039in}{5.040178in}}{\pgfqpoint{5.980139in}{5.036905in}}{\pgfqpoint{5.974315in}{5.031081in}}%
\pgfpathcurveto{\pgfqpoint{5.968491in}{5.025257in}}{\pgfqpoint{5.965219in}{5.017357in}}{\pgfqpoint{5.965219in}{5.009121in}}%
\pgfpathcurveto{\pgfqpoint{5.965219in}{5.000885in}}{\pgfqpoint{5.968491in}{4.992985in}}{\pgfqpoint{5.974315in}{4.987161in}}%
\pgfpathcurveto{\pgfqpoint{5.980139in}{4.981337in}}{\pgfqpoint{5.988039in}{4.978065in}}{\pgfqpoint{5.996275in}{4.978065in}}%
\pgfpathclose%
\pgfusepath{stroke,fill}%
\end{pgfscope}%
\begin{pgfscope}%
\pgfpathrectangle{\pgfqpoint{0.894063in}{3.540000in}}{\pgfqpoint{6.713438in}{2.060556in}} %
\pgfusepath{clip}%
\pgfsetbuttcap%
\pgfsetroundjoin%
\definecolor{currentfill}{rgb}{0.000000,0.000000,1.000000}%
\pgfsetfillcolor{currentfill}%
\pgfsetlinewidth{1.003750pt}%
\definecolor{currentstroke}{rgb}{0.000000,0.000000,0.000000}%
\pgfsetstrokecolor{currentstroke}%
\pgfsetdash{}{0pt}%
\pgfpathmoveto{\pgfqpoint{6.399081in}{4.901011in}}%
\pgfpathcurveto{\pgfqpoint{6.407318in}{4.901011in}}{\pgfqpoint{6.415218in}{4.904283in}}{\pgfqpoint{6.421042in}{4.910107in}}%
\pgfpathcurveto{\pgfqpoint{6.426865in}{4.915931in}}{\pgfqpoint{6.430138in}{4.923831in}}{\pgfqpoint{6.430138in}{4.932067in}}%
\pgfpathcurveto{\pgfqpoint{6.430138in}{4.940304in}}{\pgfqpoint{6.426865in}{4.948204in}}{\pgfqpoint{6.421042in}{4.954028in}}%
\pgfpathcurveto{\pgfqpoint{6.415218in}{4.959852in}}{\pgfqpoint{6.407318in}{4.963124in}}{\pgfqpoint{6.399081in}{4.963124in}}%
\pgfpathcurveto{\pgfqpoint{6.390845in}{4.963124in}}{\pgfqpoint{6.382945in}{4.959852in}}{\pgfqpoint{6.377121in}{4.954028in}}%
\pgfpathcurveto{\pgfqpoint{6.371297in}{4.948204in}}{\pgfqpoint{6.368025in}{4.940304in}}{\pgfqpoint{6.368025in}{4.932067in}}%
\pgfpathcurveto{\pgfqpoint{6.368025in}{4.923831in}}{\pgfqpoint{6.371297in}{4.915931in}}{\pgfqpoint{6.377121in}{4.910107in}}%
\pgfpathcurveto{\pgfqpoint{6.382945in}{4.904283in}}{\pgfqpoint{6.390845in}{4.901011in}}{\pgfqpoint{6.399081in}{4.901011in}}%
\pgfpathclose%
\pgfusepath{stroke,fill}%
\end{pgfscope}%
\begin{pgfscope}%
\pgfpathrectangle{\pgfqpoint{0.894063in}{3.540000in}}{\pgfqpoint{6.713438in}{2.060556in}} %
\pgfusepath{clip}%
\pgfsetbuttcap%
\pgfsetroundjoin%
\definecolor{currentfill}{rgb}{0.000000,0.000000,1.000000}%
\pgfsetfillcolor{currentfill}%
\pgfsetlinewidth{1.003750pt}%
\definecolor{currentstroke}{rgb}{0.000000,0.000000,0.000000}%
\pgfsetstrokecolor{currentstroke}%
\pgfsetdash{}{0pt}%
\pgfpathmoveto{\pgfqpoint{4.787856in}{5.126938in}}%
\pgfpathcurveto{\pgfqpoint{4.796093in}{5.126938in}}{\pgfqpoint{4.803993in}{5.130211in}}{\pgfqpoint{4.809817in}{5.136035in}}%
\pgfpathcurveto{\pgfqpoint{4.815640in}{5.141859in}}{\pgfqpoint{4.818913in}{5.149759in}}{\pgfqpoint{4.818913in}{5.157995in}}%
\pgfpathcurveto{\pgfqpoint{4.818913in}{5.166231in}}{\pgfqpoint{4.815640in}{5.174131in}}{\pgfqpoint{4.809817in}{5.179955in}}%
\pgfpathcurveto{\pgfqpoint{4.803993in}{5.185779in}}{\pgfqpoint{4.796093in}{5.189051in}}{\pgfqpoint{4.787856in}{5.189051in}}%
\pgfpathcurveto{\pgfqpoint{4.779620in}{5.189051in}}{\pgfqpoint{4.771720in}{5.185779in}}{\pgfqpoint{4.765896in}{5.179955in}}%
\pgfpathcurveto{\pgfqpoint{4.760072in}{5.174131in}}{\pgfqpoint{4.756800in}{5.166231in}}{\pgfqpoint{4.756800in}{5.157995in}}%
\pgfpathcurveto{\pgfqpoint{4.756800in}{5.149759in}}{\pgfqpoint{4.760072in}{5.141859in}}{\pgfqpoint{4.765896in}{5.136035in}}%
\pgfpathcurveto{\pgfqpoint{4.771720in}{5.130211in}}{\pgfqpoint{4.779620in}{5.126938in}}{\pgfqpoint{4.787856in}{5.126938in}}%
\pgfpathclose%
\pgfusepath{stroke,fill}%
\end{pgfscope}%
\begin{pgfscope}%
\pgfpathrectangle{\pgfqpoint{0.894063in}{3.540000in}}{\pgfqpoint{6.713438in}{2.060556in}} %
\pgfusepath{clip}%
\pgfsetbuttcap%
\pgfsetroundjoin%
\definecolor{currentfill}{rgb}{0.000000,0.000000,1.000000}%
\pgfsetfillcolor{currentfill}%
\pgfsetlinewidth{1.003750pt}%
\definecolor{currentstroke}{rgb}{0.000000,0.000000,0.000000}%
\pgfsetstrokecolor{currentstroke}%
\pgfsetdash{}{0pt}%
\pgfpathmoveto{\pgfqpoint{4.922125in}{5.123225in}}%
\pgfpathcurveto{\pgfqpoint{4.930361in}{5.123225in}}{\pgfqpoint{4.938261in}{5.126498in}}{\pgfqpoint{4.944085in}{5.132322in}}%
\pgfpathcurveto{\pgfqpoint{4.949909in}{5.138145in}}{\pgfqpoint{4.953181in}{5.146046in}}{\pgfqpoint{4.953181in}{5.154282in}}%
\pgfpathcurveto{\pgfqpoint{4.953181in}{5.162518in}}{\pgfqpoint{4.949909in}{5.170418in}}{\pgfqpoint{4.944085in}{5.176242in}}%
\pgfpathcurveto{\pgfqpoint{4.938261in}{5.182066in}}{\pgfqpoint{4.930361in}{5.185338in}}{\pgfqpoint{4.922125in}{5.185338in}}%
\pgfpathcurveto{\pgfqpoint{4.913889in}{5.185338in}}{\pgfqpoint{4.905989in}{5.182066in}}{\pgfqpoint{4.900165in}{5.176242in}}%
\pgfpathcurveto{\pgfqpoint{4.894341in}{5.170418in}}{\pgfqpoint{4.891069in}{5.162518in}}{\pgfqpoint{4.891069in}{5.154282in}}%
\pgfpathcurveto{\pgfqpoint{4.891069in}{5.146046in}}{\pgfqpoint{4.894341in}{5.138145in}}{\pgfqpoint{4.900165in}{5.132322in}}%
\pgfpathcurveto{\pgfqpoint{4.905989in}{5.126498in}}{\pgfqpoint{4.913889in}{5.123225in}}{\pgfqpoint{4.922125in}{5.123225in}}%
\pgfpathclose%
\pgfusepath{stroke,fill}%
\end{pgfscope}%
\begin{pgfscope}%
\pgfpathrectangle{\pgfqpoint{0.894063in}{3.540000in}}{\pgfqpoint{6.713438in}{2.060556in}} %
\pgfusepath{clip}%
\pgfsetbuttcap%
\pgfsetroundjoin%
\definecolor{currentfill}{rgb}{0.000000,0.000000,1.000000}%
\pgfsetfillcolor{currentfill}%
\pgfsetlinewidth{1.003750pt}%
\definecolor{currentstroke}{rgb}{0.000000,0.000000,0.000000}%
\pgfsetstrokecolor{currentstroke}%
\pgfsetdash{}{0pt}%
\pgfpathmoveto{\pgfqpoint{6.130544in}{4.950486in}}%
\pgfpathcurveto{\pgfqpoint{6.138780in}{4.950486in}}{\pgfqpoint{6.146680in}{4.953758in}}{\pgfqpoint{6.152504in}{4.959582in}}%
\pgfpathcurveto{\pgfqpoint{6.158328in}{4.965406in}}{\pgfqpoint{6.161600in}{4.973306in}}{\pgfqpoint{6.161600in}{4.981543in}}%
\pgfpathcurveto{\pgfqpoint{6.161600in}{4.989779in}}{\pgfqpoint{6.158328in}{4.997679in}}{\pgfqpoint{6.152504in}{5.003503in}}%
\pgfpathcurveto{\pgfqpoint{6.146680in}{5.009327in}}{\pgfqpoint{6.138780in}{5.012599in}}{\pgfqpoint{6.130544in}{5.012599in}}%
\pgfpathcurveto{\pgfqpoint{6.122307in}{5.012599in}}{\pgfqpoint{6.114407in}{5.009327in}}{\pgfqpoint{6.108583in}{5.003503in}}%
\pgfpathcurveto{\pgfqpoint{6.102760in}{4.997679in}}{\pgfqpoint{6.099487in}{4.989779in}}{\pgfqpoint{6.099487in}{4.981543in}}%
\pgfpathcurveto{\pgfqpoint{6.099487in}{4.973306in}}{\pgfqpoint{6.102760in}{4.965406in}}{\pgfqpoint{6.108583in}{4.959582in}}%
\pgfpathcurveto{\pgfqpoint{6.114407in}{4.953758in}}{\pgfqpoint{6.122307in}{4.950486in}}{\pgfqpoint{6.130544in}{4.950486in}}%
\pgfpathclose%
\pgfusepath{stroke,fill}%
\end{pgfscope}%
\begin{pgfscope}%
\pgfpathrectangle{\pgfqpoint{0.894063in}{3.540000in}}{\pgfqpoint{6.713438in}{2.060556in}} %
\pgfusepath{clip}%
\pgfsetbuttcap%
\pgfsetroundjoin%
\definecolor{currentfill}{rgb}{0.000000,0.000000,1.000000}%
\pgfsetfillcolor{currentfill}%
\pgfsetlinewidth{1.003750pt}%
\definecolor{currentstroke}{rgb}{0.000000,0.000000,0.000000}%
\pgfsetstrokecolor{currentstroke}%
\pgfsetdash{}{0pt}%
\pgfpathmoveto{\pgfqpoint{5.727738in}{5.013668in}}%
\pgfpathcurveto{\pgfqpoint{5.735974in}{5.013668in}}{\pgfqpoint{5.743874in}{5.016941in}}{\pgfqpoint{5.749698in}{5.022765in}}%
\pgfpathcurveto{\pgfqpoint{5.755522in}{5.028588in}}{\pgfqpoint{5.758794in}{5.036489in}}{\pgfqpoint{5.758794in}{5.044725in}}%
\pgfpathcurveto{\pgfqpoint{5.758794in}{5.052961in}}{\pgfqpoint{5.755522in}{5.060861in}}{\pgfqpoint{5.749698in}{5.066685in}}%
\pgfpathcurveto{\pgfqpoint{5.743874in}{5.072509in}}{\pgfqpoint{5.735974in}{5.075781in}}{\pgfqpoint{5.727738in}{5.075781in}}%
\pgfpathcurveto{\pgfqpoint{5.719501in}{5.075781in}}{\pgfqpoint{5.711601in}{5.072509in}}{\pgfqpoint{5.705777in}{5.066685in}}%
\pgfpathcurveto{\pgfqpoint{5.699953in}{5.060861in}}{\pgfqpoint{5.696681in}{5.052961in}}{\pgfqpoint{5.696681in}{5.044725in}}%
\pgfpathcurveto{\pgfqpoint{5.696681in}{5.036489in}}{\pgfqpoint{5.699953in}{5.028588in}}{\pgfqpoint{5.705777in}{5.022765in}}%
\pgfpathcurveto{\pgfqpoint{5.711601in}{5.016941in}}{\pgfqpoint{5.719501in}{5.013668in}}{\pgfqpoint{5.727738in}{5.013668in}}%
\pgfpathclose%
\pgfusepath{stroke,fill}%
\end{pgfscope}%
\begin{pgfscope}%
\pgfpathrectangle{\pgfqpoint{0.894063in}{3.540000in}}{\pgfqpoint{6.713438in}{2.060556in}} %
\pgfusepath{clip}%
\pgfsetbuttcap%
\pgfsetroundjoin%
\definecolor{currentfill}{rgb}{0.000000,0.000000,1.000000}%
\pgfsetfillcolor{currentfill}%
\pgfsetlinewidth{1.003750pt}%
\definecolor{currentstroke}{rgb}{0.000000,0.000000,0.000000}%
\pgfsetstrokecolor{currentstroke}%
\pgfsetdash{}{0pt}%
\pgfpathmoveto{\pgfqpoint{1.028331in}{5.264780in}}%
\pgfpathcurveto{\pgfqpoint{1.036568in}{5.264780in}}{\pgfqpoint{1.044468in}{5.268052in}}{\pgfqpoint{1.050292in}{5.273876in}}%
\pgfpathcurveto{\pgfqpoint{1.056115in}{5.279700in}}{\pgfqpoint{1.059388in}{5.287600in}}{\pgfqpoint{1.059388in}{5.295836in}}%
\pgfpathcurveto{\pgfqpoint{1.059388in}{5.304073in}}{\pgfqpoint{1.056115in}{5.311973in}}{\pgfqpoint{1.050292in}{5.317797in}}%
\pgfpathcurveto{\pgfqpoint{1.044468in}{5.323621in}}{\pgfqpoint{1.036568in}{5.326893in}}{\pgfqpoint{1.028331in}{5.326893in}}%
\pgfpathcurveto{\pgfqpoint{1.020095in}{5.326893in}}{\pgfqpoint{1.012195in}{5.323621in}}{\pgfqpoint{1.006371in}{5.317797in}}%
\pgfpathcurveto{\pgfqpoint{1.000547in}{5.311973in}}{\pgfqpoint{0.997275in}{5.304073in}}{\pgfqpoint{0.997275in}{5.295836in}}%
\pgfpathcurveto{\pgfqpoint{0.997275in}{5.287600in}}{\pgfqpoint{1.000547in}{5.279700in}}{\pgfqpoint{1.006371in}{5.273876in}}%
\pgfpathcurveto{\pgfqpoint{1.012195in}{5.268052in}}{\pgfqpoint{1.020095in}{5.264780in}}{\pgfqpoint{1.028331in}{5.264780in}}%
\pgfpathclose%
\pgfusepath{stroke,fill}%
\end{pgfscope}%
\begin{pgfscope}%
\pgfpathrectangle{\pgfqpoint{0.894063in}{3.540000in}}{\pgfqpoint{6.713438in}{2.060556in}} %
\pgfusepath{clip}%
\pgfsetbuttcap%
\pgfsetroundjoin%
\definecolor{currentfill}{rgb}{0.000000,0.000000,1.000000}%
\pgfsetfillcolor{currentfill}%
\pgfsetlinewidth{1.003750pt}%
\definecolor{currentstroke}{rgb}{0.000000,0.000000,0.000000}%
\pgfsetstrokecolor{currentstroke}%
\pgfsetdash{}{0pt}%
\pgfpathmoveto{\pgfqpoint{5.324931in}{5.075407in}}%
\pgfpathcurveto{\pgfqpoint{5.333168in}{5.075407in}}{\pgfqpoint{5.341068in}{5.078679in}}{\pgfqpoint{5.346892in}{5.084503in}}%
\pgfpathcurveto{\pgfqpoint{5.352715in}{5.090327in}}{\pgfqpoint{5.355988in}{5.098227in}}{\pgfqpoint{5.355988in}{5.106463in}}%
\pgfpathcurveto{\pgfqpoint{5.355988in}{5.114699in}}{\pgfqpoint{5.352715in}{5.122600in}}{\pgfqpoint{5.346892in}{5.128423in}}%
\pgfpathcurveto{\pgfqpoint{5.341068in}{5.134247in}}{\pgfqpoint{5.333168in}{5.137520in}}{\pgfqpoint{5.324931in}{5.137520in}}%
\pgfpathcurveto{\pgfqpoint{5.316695in}{5.137520in}}{\pgfqpoint{5.308795in}{5.134247in}}{\pgfqpoint{5.302971in}{5.128423in}}%
\pgfpathcurveto{\pgfqpoint{5.297147in}{5.122600in}}{\pgfqpoint{5.293875in}{5.114699in}}{\pgfqpoint{5.293875in}{5.106463in}}%
\pgfpathcurveto{\pgfqpoint{5.293875in}{5.098227in}}{\pgfqpoint{5.297147in}{5.090327in}}{\pgfqpoint{5.302971in}{5.084503in}}%
\pgfpathcurveto{\pgfqpoint{5.308795in}{5.078679in}}{\pgfqpoint{5.316695in}{5.075407in}}{\pgfqpoint{5.324931in}{5.075407in}}%
\pgfpathclose%
\pgfusepath{stroke,fill}%
\end{pgfscope}%
\begin{pgfscope}%
\pgfpathrectangle{\pgfqpoint{0.894063in}{3.540000in}}{\pgfqpoint{6.713438in}{2.060556in}} %
\pgfusepath{clip}%
\pgfsetbuttcap%
\pgfsetroundjoin%
\definecolor{currentfill}{rgb}{0.000000,0.000000,1.000000}%
\pgfsetfillcolor{currentfill}%
\pgfsetlinewidth{1.003750pt}%
\definecolor{currentstroke}{rgb}{0.000000,0.000000,0.000000}%
\pgfsetstrokecolor{currentstroke}%
\pgfsetdash{}{0pt}%
\pgfpathmoveto{\pgfqpoint{7.338963in}{4.758850in}}%
\pgfpathcurveto{\pgfqpoint{7.347199in}{4.758850in}}{\pgfqpoint{7.355099in}{4.762123in}}{\pgfqpoint{7.360923in}{4.767947in}}%
\pgfpathcurveto{\pgfqpoint{7.366747in}{4.773771in}}{\pgfqpoint{7.370019in}{4.781671in}}{\pgfqpoint{7.370019in}{4.789907in}}%
\pgfpathcurveto{\pgfqpoint{7.370019in}{4.798143in}}{\pgfqpoint{7.366747in}{4.806043in}}{\pgfqpoint{7.360923in}{4.811867in}}%
\pgfpathcurveto{\pgfqpoint{7.355099in}{4.817691in}}{\pgfqpoint{7.347199in}{4.820963in}}{\pgfqpoint{7.338963in}{4.820963in}}%
\pgfpathcurveto{\pgfqpoint{7.330726in}{4.820963in}}{\pgfqpoint{7.322826in}{4.817691in}}{\pgfqpoint{7.317002in}{4.811867in}}%
\pgfpathcurveto{\pgfqpoint{7.311178in}{4.806043in}}{\pgfqpoint{7.307906in}{4.798143in}}{\pgfqpoint{7.307906in}{4.789907in}}%
\pgfpathcurveto{\pgfqpoint{7.307906in}{4.781671in}}{\pgfqpoint{7.311178in}{4.773771in}}{\pgfqpoint{7.317002in}{4.767947in}}%
\pgfpathcurveto{\pgfqpoint{7.322826in}{4.762123in}}{\pgfqpoint{7.330726in}{4.758850in}}{\pgfqpoint{7.338963in}{4.758850in}}%
\pgfpathclose%
\pgfusepath{stroke,fill}%
\end{pgfscope}%
\begin{pgfscope}%
\pgfpathrectangle{\pgfqpoint{0.894063in}{3.540000in}}{\pgfqpoint{6.713438in}{2.060556in}} %
\pgfusepath{clip}%
\pgfsetbuttcap%
\pgfsetroundjoin%
\definecolor{currentfill}{rgb}{0.000000,0.000000,1.000000}%
\pgfsetfillcolor{currentfill}%
\pgfsetlinewidth{1.003750pt}%
\definecolor{currentstroke}{rgb}{0.000000,0.000000,0.000000}%
\pgfsetstrokecolor{currentstroke}%
\pgfsetdash{}{0pt}%
\pgfpathmoveto{\pgfqpoint{7.204694in}{4.804812in}}%
\pgfpathcurveto{\pgfqpoint{7.212930in}{4.804812in}}{\pgfqpoint{7.220830in}{4.808084in}}{\pgfqpoint{7.226654in}{4.813908in}}%
\pgfpathcurveto{\pgfqpoint{7.232478in}{4.819732in}}{\pgfqpoint{7.235750in}{4.827632in}}{\pgfqpoint{7.235750in}{4.835868in}}%
\pgfpathcurveto{\pgfqpoint{7.235750in}{4.844105in}}{\pgfqpoint{7.232478in}{4.852005in}}{\pgfqpoint{7.226654in}{4.857829in}}%
\pgfpathcurveto{\pgfqpoint{7.220830in}{4.863652in}}{\pgfqpoint{7.212930in}{4.866925in}}{\pgfqpoint{7.204694in}{4.866925in}}%
\pgfpathcurveto{\pgfqpoint{7.196457in}{4.866925in}}{\pgfqpoint{7.188557in}{4.863652in}}{\pgfqpoint{7.182733in}{4.857829in}}%
\pgfpathcurveto{\pgfqpoint{7.176910in}{4.852005in}}{\pgfqpoint{7.173637in}{4.844105in}}{\pgfqpoint{7.173637in}{4.835868in}}%
\pgfpathcurveto{\pgfqpoint{7.173637in}{4.827632in}}{\pgfqpoint{7.176910in}{4.819732in}}{\pgfqpoint{7.182733in}{4.813908in}}%
\pgfpathcurveto{\pgfqpoint{7.188557in}{4.808084in}}{\pgfqpoint{7.196457in}{4.804812in}}{\pgfqpoint{7.204694in}{4.804812in}}%
\pgfpathclose%
\pgfusepath{stroke,fill}%
\end{pgfscope}%
\begin{pgfscope}%
\pgfpathrectangle{\pgfqpoint{0.894063in}{3.540000in}}{\pgfqpoint{6.713438in}{2.060556in}} %
\pgfusepath{clip}%
\pgfsetbuttcap%
\pgfsetroundjoin%
\definecolor{currentfill}{rgb}{0.000000,0.000000,1.000000}%
\pgfsetfillcolor{currentfill}%
\pgfsetlinewidth{1.003750pt}%
\definecolor{currentstroke}{rgb}{0.000000,0.000000,0.000000}%
\pgfsetstrokecolor{currentstroke}%
\pgfsetdash{}{0pt}%
\pgfpathmoveto{\pgfqpoint{6.264813in}{4.925240in}}%
\pgfpathcurveto{\pgfqpoint{6.273049in}{4.925240in}}{\pgfqpoint{6.280949in}{4.928513in}}{\pgfqpoint{6.286773in}{4.934336in}}%
\pgfpathcurveto{\pgfqpoint{6.292597in}{4.940160in}}{\pgfqpoint{6.295869in}{4.948060in}}{\pgfqpoint{6.295869in}{4.956297in}}%
\pgfpathcurveto{\pgfqpoint{6.295869in}{4.964533in}}{\pgfqpoint{6.292597in}{4.972433in}}{\pgfqpoint{6.286773in}{4.978257in}}%
\pgfpathcurveto{\pgfqpoint{6.280949in}{4.984081in}}{\pgfqpoint{6.273049in}{4.987353in}}{\pgfqpoint{6.264813in}{4.987353in}}%
\pgfpathcurveto{\pgfqpoint{6.256576in}{4.987353in}}{\pgfqpoint{6.248676in}{4.984081in}}{\pgfqpoint{6.242852in}{4.978257in}}%
\pgfpathcurveto{\pgfqpoint{6.237028in}{4.972433in}}{\pgfqpoint{6.233756in}{4.964533in}}{\pgfqpoint{6.233756in}{4.956297in}}%
\pgfpathcurveto{\pgfqpoint{6.233756in}{4.948060in}}{\pgfqpoint{6.237028in}{4.940160in}}{\pgfqpoint{6.242852in}{4.934336in}}%
\pgfpathcurveto{\pgfqpoint{6.248676in}{4.928513in}}{\pgfqpoint{6.256576in}{4.925240in}}{\pgfqpoint{6.264813in}{4.925240in}}%
\pgfpathclose%
\pgfusepath{stroke,fill}%
\end{pgfscope}%
\begin{pgfscope}%
\pgfpathrectangle{\pgfqpoint{0.894063in}{3.540000in}}{\pgfqpoint{6.713438in}{2.060556in}} %
\pgfusepath{clip}%
\pgfsetbuttcap%
\pgfsetroundjoin%
\definecolor{currentfill}{rgb}{0.000000,0.000000,1.000000}%
\pgfsetfillcolor{currentfill}%
\pgfsetlinewidth{1.003750pt}%
\definecolor{currentstroke}{rgb}{0.000000,0.000000,0.000000}%
\pgfsetstrokecolor{currentstroke}%
\pgfsetdash{}{0pt}%
\pgfpathmoveto{\pgfqpoint{7.473231in}{4.740639in}}%
\pgfpathcurveto{\pgfqpoint{7.481468in}{4.740639in}}{\pgfqpoint{7.489368in}{4.743912in}}{\pgfqpoint{7.495192in}{4.749735in}}%
\pgfpathcurveto{\pgfqpoint{7.501015in}{4.755559in}}{\pgfqpoint{7.504288in}{4.763459in}}{\pgfqpoint{7.504288in}{4.771696in}}%
\pgfpathcurveto{\pgfqpoint{7.504288in}{4.779932in}}{\pgfqpoint{7.501015in}{4.787832in}}{\pgfqpoint{7.495192in}{4.793656in}}%
\pgfpathcurveto{\pgfqpoint{7.489368in}{4.799480in}}{\pgfqpoint{7.481468in}{4.802752in}}{\pgfqpoint{7.473231in}{4.802752in}}%
\pgfpathcurveto{\pgfqpoint{7.464995in}{4.802752in}}{\pgfqpoint{7.457095in}{4.799480in}}{\pgfqpoint{7.451271in}{4.793656in}}%
\pgfpathcurveto{\pgfqpoint{7.445447in}{4.787832in}}{\pgfqpoint{7.442175in}{4.779932in}}{\pgfqpoint{7.442175in}{4.771696in}}%
\pgfpathcurveto{\pgfqpoint{7.442175in}{4.763459in}}{\pgfqpoint{7.445447in}{4.755559in}}{\pgfqpoint{7.451271in}{4.749735in}}%
\pgfpathcurveto{\pgfqpoint{7.457095in}{4.743912in}}{\pgfqpoint{7.464995in}{4.740639in}}{\pgfqpoint{7.473231in}{4.740639in}}%
\pgfpathclose%
\pgfusepath{stroke,fill}%
\end{pgfscope}%
\begin{pgfscope}%
\pgfpathrectangle{\pgfqpoint{0.894063in}{3.540000in}}{\pgfqpoint{6.713438in}{2.060556in}} %
\pgfusepath{clip}%
\pgfsetbuttcap%
\pgfsetroundjoin%
\definecolor{currentfill}{rgb}{0.000000,0.000000,1.000000}%
\pgfsetfillcolor{currentfill}%
\pgfsetlinewidth{1.003750pt}%
\definecolor{currentstroke}{rgb}{0.000000,0.000000,0.000000}%
\pgfsetstrokecolor{currentstroke}%
\pgfsetdash{}{0pt}%
\pgfpathmoveto{\pgfqpoint{5.056394in}{5.101491in}}%
\pgfpathcurveto{\pgfqpoint{5.064630in}{5.101491in}}{\pgfqpoint{5.072530in}{5.104763in}}{\pgfqpoint{5.078354in}{5.110587in}}%
\pgfpathcurveto{\pgfqpoint{5.084178in}{5.116411in}}{\pgfqpoint{5.087450in}{5.124311in}}{\pgfqpoint{5.087450in}{5.132547in}}%
\pgfpathcurveto{\pgfqpoint{5.087450in}{5.140783in}}{\pgfqpoint{5.084178in}{5.148683in}}{\pgfqpoint{5.078354in}{5.154507in}}%
\pgfpathcurveto{\pgfqpoint{5.072530in}{5.160331in}}{\pgfqpoint{5.064630in}{5.163604in}}{\pgfqpoint{5.056394in}{5.163604in}}%
\pgfpathcurveto{\pgfqpoint{5.048157in}{5.163604in}}{\pgfqpoint{5.040257in}{5.160331in}}{\pgfqpoint{5.034433in}{5.154507in}}%
\pgfpathcurveto{\pgfqpoint{5.028610in}{5.148683in}}{\pgfqpoint{5.025337in}{5.140783in}}{\pgfqpoint{5.025337in}{5.132547in}}%
\pgfpathcurveto{\pgfqpoint{5.025337in}{5.124311in}}{\pgfqpoint{5.028610in}{5.116411in}}{\pgfqpoint{5.034433in}{5.110587in}}%
\pgfpathcurveto{\pgfqpoint{5.040257in}{5.104763in}}{\pgfqpoint{5.048157in}{5.101491in}}{\pgfqpoint{5.056394in}{5.101491in}}%
\pgfpathclose%
\pgfusepath{stroke,fill}%
\end{pgfscope}%
\begin{pgfscope}%
\pgfpathrectangle{\pgfqpoint{0.894063in}{3.540000in}}{\pgfqpoint{6.713438in}{2.060556in}} %
\pgfusepath{clip}%
\pgfsetbuttcap%
\pgfsetroundjoin%
\definecolor{currentfill}{rgb}{0.000000,0.000000,1.000000}%
\pgfsetfillcolor{currentfill}%
\pgfsetlinewidth{1.003750pt}%
\definecolor{currentstroke}{rgb}{0.000000,0.000000,0.000000}%
\pgfsetstrokecolor{currentstroke}%
\pgfsetdash{}{0pt}%
\pgfpathmoveto{\pgfqpoint{2.908094in}{5.126978in}}%
\pgfpathcurveto{\pgfqpoint{2.916330in}{5.126978in}}{\pgfqpoint{2.924230in}{5.130251in}}{\pgfqpoint{2.930054in}{5.136075in}}%
\pgfpathcurveto{\pgfqpoint{2.935878in}{5.141898in}}{\pgfqpoint{2.939150in}{5.149798in}}{\pgfqpoint{2.939150in}{5.158035in}}%
\pgfpathcurveto{\pgfqpoint{2.939150in}{5.166271in}}{\pgfqpoint{2.935878in}{5.174171in}}{\pgfqpoint{2.930054in}{5.179995in}}%
\pgfpathcurveto{\pgfqpoint{2.924230in}{5.185819in}}{\pgfqpoint{2.916330in}{5.189091in}}{\pgfqpoint{2.908094in}{5.189091in}}%
\pgfpathcurveto{\pgfqpoint{2.899857in}{5.189091in}}{\pgfqpoint{2.891957in}{5.185819in}}{\pgfqpoint{2.886133in}{5.179995in}}%
\pgfpathcurveto{\pgfqpoint{2.880310in}{5.174171in}}{\pgfqpoint{2.877037in}{5.166271in}}{\pgfqpoint{2.877037in}{5.158035in}}%
\pgfpathcurveto{\pgfqpoint{2.877037in}{5.149798in}}{\pgfqpoint{2.880310in}{5.141898in}}{\pgfqpoint{2.886133in}{5.136075in}}%
\pgfpathcurveto{\pgfqpoint{2.891957in}{5.130251in}}{\pgfqpoint{2.899857in}{5.126978in}}{\pgfqpoint{2.908094in}{5.126978in}}%
\pgfpathclose%
\pgfusepath{stroke,fill}%
\end{pgfscope}%
\begin{pgfscope}%
\pgfpathrectangle{\pgfqpoint{0.894063in}{3.540000in}}{\pgfqpoint{6.713438in}{2.060556in}} %
\pgfusepath{clip}%
\pgfsetbuttcap%
\pgfsetroundjoin%
\definecolor{currentfill}{rgb}{0.000000,0.000000,1.000000}%
\pgfsetfillcolor{currentfill}%
\pgfsetlinewidth{1.003750pt}%
\definecolor{currentstroke}{rgb}{0.000000,0.000000,0.000000}%
\pgfsetstrokecolor{currentstroke}%
\pgfsetdash{}{0pt}%
\pgfpathmoveto{\pgfqpoint{3.445169in}{5.126965in}}%
\pgfpathcurveto{\pgfqpoint{3.453405in}{5.126965in}}{\pgfqpoint{3.461305in}{5.130237in}}{\pgfqpoint{3.467129in}{5.136061in}}%
\pgfpathcurveto{\pgfqpoint{3.472953in}{5.141885in}}{\pgfqpoint{3.476225in}{5.149785in}}{\pgfqpoint{3.476225in}{5.158021in}}%
\pgfpathcurveto{\pgfqpoint{3.476225in}{5.166257in}}{\pgfqpoint{3.472953in}{5.174157in}}{\pgfqpoint{3.467129in}{5.179981in}}%
\pgfpathcurveto{\pgfqpoint{3.461305in}{5.185805in}}{\pgfqpoint{3.453405in}{5.189078in}}{\pgfqpoint{3.445169in}{5.189078in}}%
\pgfpathcurveto{\pgfqpoint{3.436932in}{5.189078in}}{\pgfqpoint{3.429032in}{5.185805in}}{\pgfqpoint{3.423208in}{5.179981in}}%
\pgfpathcurveto{\pgfqpoint{3.417385in}{5.174157in}}{\pgfqpoint{3.414112in}{5.166257in}}{\pgfqpoint{3.414112in}{5.158021in}}%
\pgfpathcurveto{\pgfqpoint{3.414112in}{5.149785in}}{\pgfqpoint{3.417385in}{5.141885in}}{\pgfqpoint{3.423208in}{5.136061in}}%
\pgfpathcurveto{\pgfqpoint{3.429032in}{5.130237in}}{\pgfqpoint{3.436932in}{5.126965in}}{\pgfqpoint{3.445169in}{5.126965in}}%
\pgfpathclose%
\pgfusepath{stroke,fill}%
\end{pgfscope}%
\begin{pgfscope}%
\pgfpathrectangle{\pgfqpoint{0.894063in}{3.540000in}}{\pgfqpoint{6.713438in}{2.060556in}} %
\pgfusepath{clip}%
\pgfsetbuttcap%
\pgfsetroundjoin%
\definecolor{currentfill}{rgb}{0.000000,0.000000,1.000000}%
\pgfsetfillcolor{currentfill}%
\pgfsetlinewidth{1.003750pt}%
\definecolor{currentstroke}{rgb}{0.000000,0.000000,0.000000}%
\pgfsetstrokecolor{currentstroke}%
\pgfsetdash{}{0pt}%
\pgfpathmoveto{\pgfqpoint{4.116513in}{5.126943in}}%
\pgfpathcurveto{\pgfqpoint{4.124749in}{5.126943in}}{\pgfqpoint{4.132649in}{5.130215in}}{\pgfqpoint{4.138473in}{5.136039in}}%
\pgfpathcurveto{\pgfqpoint{4.144297in}{5.141863in}}{\pgfqpoint{4.147569in}{5.149763in}}{\pgfqpoint{4.147569in}{5.157999in}}%
\pgfpathcurveto{\pgfqpoint{4.147569in}{5.166235in}}{\pgfqpoint{4.144297in}{5.174135in}}{\pgfqpoint{4.138473in}{5.179959in}}%
\pgfpathcurveto{\pgfqpoint{4.132649in}{5.185783in}}{\pgfqpoint{4.124749in}{5.189056in}}{\pgfqpoint{4.116513in}{5.189056in}}%
\pgfpathcurveto{\pgfqpoint{4.108276in}{5.189056in}}{\pgfqpoint{4.100376in}{5.185783in}}{\pgfqpoint{4.094552in}{5.179959in}}%
\pgfpathcurveto{\pgfqpoint{4.088728in}{5.174135in}}{\pgfqpoint{4.085456in}{5.166235in}}{\pgfqpoint{4.085456in}{5.157999in}}%
\pgfpathcurveto{\pgfqpoint{4.085456in}{5.149763in}}{\pgfqpoint{4.088728in}{5.141863in}}{\pgfqpoint{4.094552in}{5.136039in}}%
\pgfpathcurveto{\pgfqpoint{4.100376in}{5.130215in}}{\pgfqpoint{4.108276in}{5.126943in}}{\pgfqpoint{4.116513in}{5.126943in}}%
\pgfpathclose%
\pgfusepath{stroke,fill}%
\end{pgfscope}%
\begin{pgfscope}%
\pgfpathrectangle{\pgfqpoint{0.894063in}{3.540000in}}{\pgfqpoint{6.713438in}{2.060556in}} %
\pgfusepath{clip}%
\pgfsetbuttcap%
\pgfsetroundjoin%
\definecolor{currentfill}{rgb}{0.000000,0.000000,1.000000}%
\pgfsetfillcolor{currentfill}%
\pgfsetlinewidth{1.003750pt}%
\definecolor{currentstroke}{rgb}{0.000000,0.000000,0.000000}%
\pgfsetstrokecolor{currentstroke}%
\pgfsetdash{}{0pt}%
\pgfpathmoveto{\pgfqpoint{1.431138in}{5.141262in}}%
\pgfpathcurveto{\pgfqpoint{1.439374in}{5.141262in}}{\pgfqpoint{1.447274in}{5.144534in}}{\pgfqpoint{1.453098in}{5.150358in}}%
\pgfpathcurveto{\pgfqpoint{1.458922in}{5.156182in}}{\pgfqpoint{1.462194in}{5.164082in}}{\pgfqpoint{1.462194in}{5.172319in}}%
\pgfpathcurveto{\pgfqpoint{1.462194in}{5.180555in}}{\pgfqpoint{1.458922in}{5.188455in}}{\pgfqpoint{1.453098in}{5.194279in}}%
\pgfpathcurveto{\pgfqpoint{1.447274in}{5.200103in}}{\pgfqpoint{1.439374in}{5.203375in}}{\pgfqpoint{1.431138in}{5.203375in}}%
\pgfpathcurveto{\pgfqpoint{1.422901in}{5.203375in}}{\pgfqpoint{1.415001in}{5.200103in}}{\pgfqpoint{1.409177in}{5.194279in}}%
\pgfpathcurveto{\pgfqpoint{1.403353in}{5.188455in}}{\pgfqpoint{1.400081in}{5.180555in}}{\pgfqpoint{1.400081in}{5.172319in}}%
\pgfpathcurveto{\pgfqpoint{1.400081in}{5.164082in}}{\pgfqpoint{1.403353in}{5.156182in}}{\pgfqpoint{1.409177in}{5.150358in}}%
\pgfpathcurveto{\pgfqpoint{1.415001in}{5.144534in}}{\pgfqpoint{1.422901in}{5.141262in}}{\pgfqpoint{1.431138in}{5.141262in}}%
\pgfpathclose%
\pgfusepath{stroke,fill}%
\end{pgfscope}%
\begin{pgfscope}%
\pgfpathrectangle{\pgfqpoint{0.894063in}{3.540000in}}{\pgfqpoint{6.713438in}{2.060556in}} %
\pgfusepath{clip}%
\pgfsetbuttcap%
\pgfsetroundjoin%
\definecolor{currentfill}{rgb}{0.000000,0.000000,1.000000}%
\pgfsetfillcolor{currentfill}%
\pgfsetlinewidth{1.003750pt}%
\definecolor{currentstroke}{rgb}{0.000000,0.000000,0.000000}%
\pgfsetstrokecolor{currentstroke}%
\pgfsetdash{}{0pt}%
\pgfpathmoveto{\pgfqpoint{2.773825in}{5.126980in}}%
\pgfpathcurveto{\pgfqpoint{2.782061in}{5.126980in}}{\pgfqpoint{2.789961in}{5.130252in}}{\pgfqpoint{2.795785in}{5.136076in}}%
\pgfpathcurveto{\pgfqpoint{2.801609in}{5.141900in}}{\pgfqpoint{2.804881in}{5.149800in}}{\pgfqpoint{2.804881in}{5.158036in}}%
\pgfpathcurveto{\pgfqpoint{2.804881in}{5.166272in}}{\pgfqpoint{2.801609in}{5.174172in}}{\pgfqpoint{2.795785in}{5.179996in}}%
\pgfpathcurveto{\pgfqpoint{2.789961in}{5.185820in}}{\pgfqpoint{2.782061in}{5.189093in}}{\pgfqpoint{2.773825in}{5.189093in}}%
\pgfpathcurveto{\pgfqpoint{2.765589in}{5.189093in}}{\pgfqpoint{2.757689in}{5.185820in}}{\pgfqpoint{2.751865in}{5.179996in}}%
\pgfpathcurveto{\pgfqpoint{2.746041in}{5.174172in}}{\pgfqpoint{2.742769in}{5.166272in}}{\pgfqpoint{2.742769in}{5.158036in}}%
\pgfpathcurveto{\pgfqpoint{2.742769in}{5.149800in}}{\pgfqpoint{2.746041in}{5.141900in}}{\pgfqpoint{2.751865in}{5.136076in}}%
\pgfpathcurveto{\pgfqpoint{2.757689in}{5.130252in}}{\pgfqpoint{2.765589in}{5.126980in}}{\pgfqpoint{2.773825in}{5.126980in}}%
\pgfpathclose%
\pgfusepath{stroke,fill}%
\end{pgfscope}%
\begin{pgfscope}%
\pgfpathrectangle{\pgfqpoint{0.894063in}{3.540000in}}{\pgfqpoint{6.713438in}{2.060556in}} %
\pgfusepath{clip}%
\pgfsetbuttcap%
\pgfsetroundjoin%
\definecolor{currentfill}{rgb}{0.000000,0.000000,1.000000}%
\pgfsetfillcolor{currentfill}%
\pgfsetlinewidth{1.003750pt}%
\definecolor{currentstroke}{rgb}{0.000000,0.000000,0.000000}%
\pgfsetstrokecolor{currentstroke}%
\pgfsetdash{}{0pt}%
\pgfpathmoveto{\pgfqpoint{1.565406in}{5.140358in}}%
\pgfpathcurveto{\pgfqpoint{1.573643in}{5.140358in}}{\pgfqpoint{1.581543in}{5.143630in}}{\pgfqpoint{1.587367in}{5.149454in}}%
\pgfpathcurveto{\pgfqpoint{1.593190in}{5.155278in}}{\pgfqpoint{1.596463in}{5.163178in}}{\pgfqpoint{1.596463in}{5.171415in}}%
\pgfpathcurveto{\pgfqpoint{1.596463in}{5.179651in}}{\pgfqpoint{1.593190in}{5.187551in}}{\pgfqpoint{1.587367in}{5.193375in}}%
\pgfpathcurveto{\pgfqpoint{1.581543in}{5.199199in}}{\pgfqpoint{1.573643in}{5.202471in}}{\pgfqpoint{1.565406in}{5.202471in}}%
\pgfpathcurveto{\pgfqpoint{1.557170in}{5.202471in}}{\pgfqpoint{1.549270in}{5.199199in}}{\pgfqpoint{1.543446in}{5.193375in}}%
\pgfpathcurveto{\pgfqpoint{1.537622in}{5.187551in}}{\pgfqpoint{1.534350in}{5.179651in}}{\pgfqpoint{1.534350in}{5.171415in}}%
\pgfpathcurveto{\pgfqpoint{1.534350in}{5.163178in}}{\pgfqpoint{1.537622in}{5.155278in}}{\pgfqpoint{1.543446in}{5.149454in}}%
\pgfpathcurveto{\pgfqpoint{1.549270in}{5.143630in}}{\pgfqpoint{1.557170in}{5.140358in}}{\pgfqpoint{1.565406in}{5.140358in}}%
\pgfpathclose%
\pgfusepath{stroke,fill}%
\end{pgfscope}%
\begin{pgfscope}%
\pgfpathrectangle{\pgfqpoint{0.894063in}{3.540000in}}{\pgfqpoint{6.713438in}{2.060556in}} %
\pgfusepath{clip}%
\pgfsetbuttcap%
\pgfsetroundjoin%
\definecolor{currentfill}{rgb}{0.000000,0.000000,1.000000}%
\pgfsetfillcolor{currentfill}%
\pgfsetlinewidth{1.003750pt}%
\definecolor{currentstroke}{rgb}{0.000000,0.000000,0.000000}%
\pgfsetstrokecolor{currentstroke}%
\pgfsetdash{}{0pt}%
\pgfpathmoveto{\pgfqpoint{4.250781in}{5.126943in}}%
\pgfpathcurveto{\pgfqpoint{4.259018in}{5.126943in}}{\pgfqpoint{4.266918in}{5.130215in}}{\pgfqpoint{4.272742in}{5.136039in}}%
\pgfpathcurveto{\pgfqpoint{4.278565in}{5.141863in}}{\pgfqpoint{4.281838in}{5.149763in}}{\pgfqpoint{4.281838in}{5.157999in}}%
\pgfpathcurveto{\pgfqpoint{4.281838in}{5.166235in}}{\pgfqpoint{4.278565in}{5.174135in}}{\pgfqpoint{4.272742in}{5.179959in}}%
\pgfpathcurveto{\pgfqpoint{4.266918in}{5.185783in}}{\pgfqpoint{4.259018in}{5.189056in}}{\pgfqpoint{4.250781in}{5.189056in}}%
\pgfpathcurveto{\pgfqpoint{4.242545in}{5.189056in}}{\pgfqpoint{4.234645in}{5.185783in}}{\pgfqpoint{4.228821in}{5.179959in}}%
\pgfpathcurveto{\pgfqpoint{4.222997in}{5.174135in}}{\pgfqpoint{4.219725in}{5.166235in}}{\pgfqpoint{4.219725in}{5.157999in}}%
\pgfpathcurveto{\pgfqpoint{4.219725in}{5.149763in}}{\pgfqpoint{4.222997in}{5.141863in}}{\pgfqpoint{4.228821in}{5.136039in}}%
\pgfpathcurveto{\pgfqpoint{4.234645in}{5.130215in}}{\pgfqpoint{4.242545in}{5.126943in}}{\pgfqpoint{4.250781in}{5.126943in}}%
\pgfpathclose%
\pgfusepath{stroke,fill}%
\end{pgfscope}%
\begin{pgfscope}%
\pgfpathrectangle{\pgfqpoint{0.894063in}{3.540000in}}{\pgfqpoint{6.713438in}{2.060556in}} %
\pgfusepath{clip}%
\pgfsetbuttcap%
\pgfsetroundjoin%
\definecolor{currentfill}{rgb}{0.000000,0.000000,1.000000}%
\pgfsetfillcolor{currentfill}%
\pgfsetlinewidth{1.003750pt}%
\definecolor{currentstroke}{rgb}{0.000000,0.000000,0.000000}%
\pgfsetstrokecolor{currentstroke}%
\pgfsetdash{}{0pt}%
\pgfpathmoveto{\pgfqpoint{3.847975in}{5.126955in}}%
\pgfpathcurveto{\pgfqpoint{3.856211in}{5.126955in}}{\pgfqpoint{3.864111in}{5.130227in}}{\pgfqpoint{3.869935in}{5.136051in}}%
\pgfpathcurveto{\pgfqpoint{3.875759in}{5.141875in}}{\pgfqpoint{3.879031in}{5.149775in}}{\pgfqpoint{3.879031in}{5.158011in}}%
\pgfpathcurveto{\pgfqpoint{3.879031in}{5.166248in}}{\pgfqpoint{3.875759in}{5.174148in}}{\pgfqpoint{3.869935in}{5.179972in}}%
\pgfpathcurveto{\pgfqpoint{3.864111in}{5.185796in}}{\pgfqpoint{3.856211in}{5.189068in}}{\pgfqpoint{3.847975in}{5.189068in}}%
\pgfpathcurveto{\pgfqpoint{3.839739in}{5.189068in}}{\pgfqpoint{3.831839in}{5.185796in}}{\pgfqpoint{3.826015in}{5.179972in}}%
\pgfpathcurveto{\pgfqpoint{3.820191in}{5.174148in}}{\pgfqpoint{3.816919in}{5.166248in}}{\pgfqpoint{3.816919in}{5.158011in}}%
\pgfpathcurveto{\pgfqpoint{3.816919in}{5.149775in}}{\pgfqpoint{3.820191in}{5.141875in}}{\pgfqpoint{3.826015in}{5.136051in}}%
\pgfpathcurveto{\pgfqpoint{3.831839in}{5.130227in}}{\pgfqpoint{3.839739in}{5.126955in}}{\pgfqpoint{3.847975in}{5.126955in}}%
\pgfpathclose%
\pgfusepath{stroke,fill}%
\end{pgfscope}%
\begin{pgfscope}%
\pgfpathrectangle{\pgfqpoint{0.894063in}{3.540000in}}{\pgfqpoint{6.713438in}{2.060556in}} %
\pgfusepath{clip}%
\pgfsetbuttcap%
\pgfsetroundjoin%
\definecolor{currentfill}{rgb}{0.000000,0.000000,1.000000}%
\pgfsetfillcolor{currentfill}%
\pgfsetlinewidth{1.003750pt}%
\definecolor{currentstroke}{rgb}{0.000000,0.000000,0.000000}%
\pgfsetstrokecolor{currentstroke}%
\pgfsetdash{}{0pt}%
\pgfpathmoveto{\pgfqpoint{7.607500in}{4.725288in}}%
\pgfpathcurveto{\pgfqpoint{7.615736in}{4.725288in}}{\pgfqpoint{7.623636in}{4.728560in}}{\pgfqpoint{7.629460in}{4.734384in}}%
\pgfpathcurveto{\pgfqpoint{7.635284in}{4.740208in}}{\pgfqpoint{7.638556in}{4.748108in}}{\pgfqpoint{7.638556in}{4.756345in}}%
\pgfpathcurveto{\pgfqpoint{7.638556in}{4.764581in}}{\pgfqpoint{7.635284in}{4.772481in}}{\pgfqpoint{7.629460in}{4.778305in}}%
\pgfpathcurveto{\pgfqpoint{7.623636in}{4.784129in}}{\pgfqpoint{7.615736in}{4.787401in}}{\pgfqpoint{7.607500in}{4.787401in}}%
\pgfpathcurveto{\pgfqpoint{7.599264in}{4.787401in}}{\pgfqpoint{7.591364in}{4.784129in}}{\pgfqpoint{7.585540in}{4.778305in}}%
\pgfpathcurveto{\pgfqpoint{7.579716in}{4.772481in}}{\pgfqpoint{7.576444in}{4.764581in}}{\pgfqpoint{7.576444in}{4.756345in}}%
\pgfpathcurveto{\pgfqpoint{7.576444in}{4.748108in}}{\pgfqpoint{7.579716in}{4.740208in}}{\pgfqpoint{7.585540in}{4.734384in}}%
\pgfpathcurveto{\pgfqpoint{7.591364in}{4.728560in}}{\pgfqpoint{7.599264in}{4.725288in}}{\pgfqpoint{7.607500in}{4.725288in}}%
\pgfpathclose%
\pgfusepath{stroke,fill}%
\end{pgfscope}%
\begin{pgfscope}%
\pgfpathrectangle{\pgfqpoint{0.894063in}{3.540000in}}{\pgfqpoint{6.713438in}{2.060556in}} %
\pgfusepath{clip}%
\pgfsetbuttcap%
\pgfsetroundjoin%
\definecolor{currentfill}{rgb}{0.000000,0.000000,1.000000}%
\pgfsetfillcolor{currentfill}%
\pgfsetlinewidth{1.003750pt}%
\definecolor{currentstroke}{rgb}{0.000000,0.000000,0.000000}%
\pgfsetstrokecolor{currentstroke}%
\pgfsetdash{}{0pt}%
\pgfpathmoveto{\pgfqpoint{4.385050in}{5.126943in}}%
\pgfpathcurveto{\pgfqpoint{4.393286in}{5.126943in}}{\pgfqpoint{4.401186in}{5.130215in}}{\pgfqpoint{4.407010in}{5.136039in}}%
\pgfpathcurveto{\pgfqpoint{4.412834in}{5.141863in}}{\pgfqpoint{4.416106in}{5.149763in}}{\pgfqpoint{4.416106in}{5.157999in}}%
\pgfpathcurveto{\pgfqpoint{4.416106in}{5.166235in}}{\pgfqpoint{4.412834in}{5.174135in}}{\pgfqpoint{4.407010in}{5.179959in}}%
\pgfpathcurveto{\pgfqpoint{4.401186in}{5.185783in}}{\pgfqpoint{4.393286in}{5.189056in}}{\pgfqpoint{4.385050in}{5.189056in}}%
\pgfpathcurveto{\pgfqpoint{4.376814in}{5.189056in}}{\pgfqpoint{4.368914in}{5.185783in}}{\pgfqpoint{4.363090in}{5.179959in}}%
\pgfpathcurveto{\pgfqpoint{4.357266in}{5.174135in}}{\pgfqpoint{4.353994in}{5.166235in}}{\pgfqpoint{4.353994in}{5.157999in}}%
\pgfpathcurveto{\pgfqpoint{4.353994in}{5.149763in}}{\pgfqpoint{4.357266in}{5.141863in}}{\pgfqpoint{4.363090in}{5.136039in}}%
\pgfpathcurveto{\pgfqpoint{4.368914in}{5.130215in}}{\pgfqpoint{4.376814in}{5.126943in}}{\pgfqpoint{4.385050in}{5.126943in}}%
\pgfpathclose%
\pgfusepath{stroke,fill}%
\end{pgfscope}%
\begin{pgfscope}%
\pgfpathrectangle{\pgfqpoint{0.894063in}{3.540000in}}{\pgfqpoint{6.713438in}{2.060556in}} %
\pgfusepath{clip}%
\pgfsetbuttcap%
\pgfsetroundjoin%
\definecolor{currentfill}{rgb}{0.000000,0.000000,1.000000}%
\pgfsetfillcolor{currentfill}%
\pgfsetlinewidth{1.003750pt}%
\definecolor{currentstroke}{rgb}{0.000000,0.000000,0.000000}%
\pgfsetstrokecolor{currentstroke}%
\pgfsetdash{}{0pt}%
\pgfpathmoveto{\pgfqpoint{6.533350in}{4.882535in}}%
\pgfpathcurveto{\pgfqpoint{6.541586in}{4.882535in}}{\pgfqpoint{6.549486in}{4.885807in}}{\pgfqpoint{6.555310in}{4.891631in}}%
\pgfpathcurveto{\pgfqpoint{6.561134in}{4.897455in}}{\pgfqpoint{6.564406in}{4.905355in}}{\pgfqpoint{6.564406in}{4.913591in}}%
\pgfpathcurveto{\pgfqpoint{6.564406in}{4.921827in}}{\pgfqpoint{6.561134in}{4.929727in}}{\pgfqpoint{6.555310in}{4.935551in}}%
\pgfpathcurveto{\pgfqpoint{6.549486in}{4.941375in}}{\pgfqpoint{6.541586in}{4.944648in}}{\pgfqpoint{6.533350in}{4.944648in}}%
\pgfpathcurveto{\pgfqpoint{6.525114in}{4.944648in}}{\pgfqpoint{6.517214in}{4.941375in}}{\pgfqpoint{6.511390in}{4.935551in}}%
\pgfpathcurveto{\pgfqpoint{6.505566in}{4.929727in}}{\pgfqpoint{6.502294in}{4.921827in}}{\pgfqpoint{6.502294in}{4.913591in}}%
\pgfpathcurveto{\pgfqpoint{6.502294in}{4.905355in}}{\pgfqpoint{6.505566in}{4.897455in}}{\pgfqpoint{6.511390in}{4.891631in}}%
\pgfpathcurveto{\pgfqpoint{6.517214in}{4.885807in}}{\pgfqpoint{6.525114in}{4.882535in}}{\pgfqpoint{6.533350in}{4.882535in}}%
\pgfpathclose%
\pgfusepath{stroke,fill}%
\end{pgfscope}%
\begin{pgfscope}%
\pgfpathrectangle{\pgfqpoint{0.894063in}{3.540000in}}{\pgfqpoint{6.713438in}{2.060556in}} %
\pgfusepath{clip}%
\pgfsetbuttcap%
\pgfsetroundjoin%
\definecolor{currentfill}{rgb}{0.000000,0.000000,1.000000}%
\pgfsetfillcolor{currentfill}%
\pgfsetlinewidth{1.003750pt}%
\definecolor{currentstroke}{rgb}{0.000000,0.000000,0.000000}%
\pgfsetstrokecolor{currentstroke}%
\pgfsetdash{}{0pt}%
\pgfpathmoveto{\pgfqpoint{1.296869in}{5.264666in}}%
\pgfpathcurveto{\pgfqpoint{1.305105in}{5.264666in}}{\pgfqpoint{1.313005in}{5.267938in}}{\pgfqpoint{1.318829in}{5.273762in}}%
\pgfpathcurveto{\pgfqpoint{1.324653in}{5.279586in}}{\pgfqpoint{1.327925in}{5.287486in}}{\pgfqpoint{1.327925in}{5.295722in}}%
\pgfpathcurveto{\pgfqpoint{1.327925in}{5.303959in}}{\pgfqpoint{1.324653in}{5.311859in}}{\pgfqpoint{1.318829in}{5.317683in}}%
\pgfpathcurveto{\pgfqpoint{1.313005in}{5.323507in}}{\pgfqpoint{1.305105in}{5.326779in}}{\pgfqpoint{1.296869in}{5.326779in}}%
\pgfpathcurveto{\pgfqpoint{1.288632in}{5.326779in}}{\pgfqpoint{1.280732in}{5.323507in}}{\pgfqpoint{1.274908in}{5.317683in}}%
\pgfpathcurveto{\pgfqpoint{1.269085in}{5.311859in}}{\pgfqpoint{1.265812in}{5.303959in}}{\pgfqpoint{1.265812in}{5.295722in}}%
\pgfpathcurveto{\pgfqpoint{1.265812in}{5.287486in}}{\pgfqpoint{1.269085in}{5.279586in}}{\pgfqpoint{1.274908in}{5.273762in}}%
\pgfpathcurveto{\pgfqpoint{1.280732in}{5.267938in}}{\pgfqpoint{1.288632in}{5.264666in}}{\pgfqpoint{1.296869in}{5.264666in}}%
\pgfpathclose%
\pgfusepath{stroke,fill}%
\end{pgfscope}%
\begin{pgfscope}%
\pgfpathrectangle{\pgfqpoint{0.894063in}{3.540000in}}{\pgfqpoint{6.713438in}{2.060556in}} %
\pgfusepath{clip}%
\pgfsetbuttcap%
\pgfsetroundjoin%
\definecolor{currentfill}{rgb}{0.000000,0.000000,1.000000}%
\pgfsetfillcolor{currentfill}%
\pgfsetlinewidth{1.003750pt}%
\definecolor{currentstroke}{rgb}{0.000000,0.000000,0.000000}%
\pgfsetstrokecolor{currentstroke}%
\pgfsetdash{}{0pt}%
\pgfpathmoveto{\pgfqpoint{4.519319in}{5.126938in}}%
\pgfpathcurveto{\pgfqpoint{4.527555in}{5.126938in}}{\pgfqpoint{4.535455in}{5.130211in}}{\pgfqpoint{4.541279in}{5.136035in}}%
\pgfpathcurveto{\pgfqpoint{4.547103in}{5.141859in}}{\pgfqpoint{4.550375in}{5.149759in}}{\pgfqpoint{4.550375in}{5.157995in}}%
\pgfpathcurveto{\pgfqpoint{4.550375in}{5.166231in}}{\pgfqpoint{4.547103in}{5.174131in}}{\pgfqpoint{4.541279in}{5.179955in}}%
\pgfpathcurveto{\pgfqpoint{4.535455in}{5.185779in}}{\pgfqpoint{4.527555in}{5.189051in}}{\pgfqpoint{4.519319in}{5.189051in}}%
\pgfpathcurveto{\pgfqpoint{4.511082in}{5.189051in}}{\pgfqpoint{4.503182in}{5.185779in}}{\pgfqpoint{4.497358in}{5.179955in}}%
\pgfpathcurveto{\pgfqpoint{4.491535in}{5.174131in}}{\pgfqpoint{4.488262in}{5.166231in}}{\pgfqpoint{4.488262in}{5.157995in}}%
\pgfpathcurveto{\pgfqpoint{4.488262in}{5.149759in}}{\pgfqpoint{4.491535in}{5.141859in}}{\pgfqpoint{4.497358in}{5.136035in}}%
\pgfpathcurveto{\pgfqpoint{4.503182in}{5.130211in}}{\pgfqpoint{4.511082in}{5.126938in}}{\pgfqpoint{4.519319in}{5.126938in}}%
\pgfpathclose%
\pgfusepath{stroke,fill}%
\end{pgfscope}%
\begin{pgfscope}%
\pgfpathrectangle{\pgfqpoint{0.894063in}{3.540000in}}{\pgfqpoint{6.713438in}{2.060556in}} %
\pgfusepath{clip}%
\pgfsetbuttcap%
\pgfsetroundjoin%
\definecolor{currentfill}{rgb}{0.000000,0.000000,1.000000}%
\pgfsetfillcolor{currentfill}%
\pgfsetlinewidth{1.003750pt}%
\definecolor{currentstroke}{rgb}{0.000000,0.000000,0.000000}%
\pgfsetstrokecolor{currentstroke}%
\pgfsetdash{}{0pt}%
\pgfpathmoveto{\pgfqpoint{2.505288in}{5.127014in}}%
\pgfpathcurveto{\pgfqpoint{2.513524in}{5.127014in}}{\pgfqpoint{2.521424in}{5.130286in}}{\pgfqpoint{2.527248in}{5.136110in}}%
\pgfpathcurveto{\pgfqpoint{2.533072in}{5.141934in}}{\pgfqpoint{2.536344in}{5.149834in}}{\pgfqpoint{2.536344in}{5.158070in}}%
\pgfpathcurveto{\pgfqpoint{2.536344in}{5.166307in}}{\pgfqpoint{2.533072in}{5.174207in}}{\pgfqpoint{2.527248in}{5.180031in}}%
\pgfpathcurveto{\pgfqpoint{2.521424in}{5.185855in}}{\pgfqpoint{2.513524in}{5.189127in}}{\pgfqpoint{2.505288in}{5.189127in}}%
\pgfpathcurveto{\pgfqpoint{2.497051in}{5.189127in}}{\pgfqpoint{2.489151in}{5.185855in}}{\pgfqpoint{2.483327in}{5.180031in}}%
\pgfpathcurveto{\pgfqpoint{2.477503in}{5.174207in}}{\pgfqpoint{2.474231in}{5.166307in}}{\pgfqpoint{2.474231in}{5.158070in}}%
\pgfpathcurveto{\pgfqpoint{2.474231in}{5.149834in}}{\pgfqpoint{2.477503in}{5.141934in}}{\pgfqpoint{2.483327in}{5.136110in}}%
\pgfpathcurveto{\pgfqpoint{2.489151in}{5.130286in}}{\pgfqpoint{2.497051in}{5.127014in}}{\pgfqpoint{2.505288in}{5.127014in}}%
\pgfpathclose%
\pgfusepath{stroke,fill}%
\end{pgfscope}%
\begin{pgfscope}%
\pgfpathrectangle{\pgfqpoint{0.894063in}{3.540000in}}{\pgfqpoint{6.713438in}{2.060556in}} %
\pgfusepath{clip}%
\pgfsetbuttcap%
\pgfsetroundjoin%
\definecolor{currentfill}{rgb}{0.000000,0.000000,1.000000}%
\pgfsetfillcolor{currentfill}%
\pgfsetlinewidth{1.003750pt}%
\definecolor{currentstroke}{rgb}{0.000000,0.000000,0.000000}%
\pgfsetstrokecolor{currentstroke}%
\pgfsetdash{}{0pt}%
\pgfpathmoveto{\pgfqpoint{5.459200in}{5.048805in}}%
\pgfpathcurveto{\pgfqpoint{5.467436in}{5.048805in}}{\pgfqpoint{5.475336in}{5.052077in}}{\pgfqpoint{5.481160in}{5.057901in}}%
\pgfpathcurveto{\pgfqpoint{5.486984in}{5.063725in}}{\pgfqpoint{5.490256in}{5.071625in}}{\pgfqpoint{5.490256in}{5.079861in}}%
\pgfpathcurveto{\pgfqpoint{5.490256in}{5.088098in}}{\pgfqpoint{5.486984in}{5.095998in}}{\pgfqpoint{5.481160in}{5.101822in}}%
\pgfpathcurveto{\pgfqpoint{5.475336in}{5.107646in}}{\pgfqpoint{5.467436in}{5.110918in}}{\pgfqpoint{5.459200in}{5.110918in}}%
\pgfpathcurveto{\pgfqpoint{5.450964in}{5.110918in}}{\pgfqpoint{5.443064in}{5.107646in}}{\pgfqpoint{5.437240in}{5.101822in}}%
\pgfpathcurveto{\pgfqpoint{5.431416in}{5.095998in}}{\pgfqpoint{5.428144in}{5.088098in}}{\pgfqpoint{5.428144in}{5.079861in}}%
\pgfpathcurveto{\pgfqpoint{5.428144in}{5.071625in}}{\pgfqpoint{5.431416in}{5.063725in}}{\pgfqpoint{5.437240in}{5.057901in}}%
\pgfpathcurveto{\pgfqpoint{5.443064in}{5.052077in}}{\pgfqpoint{5.450964in}{5.048805in}}{\pgfqpoint{5.459200in}{5.048805in}}%
\pgfpathclose%
\pgfusepath{stroke,fill}%
\end{pgfscope}%
\begin{pgfscope}%
\pgfpathrectangle{\pgfqpoint{0.894063in}{3.540000in}}{\pgfqpoint{6.713438in}{2.060556in}} %
\pgfusepath{clip}%
\pgfsetbuttcap%
\pgfsetroundjoin%
\definecolor{currentfill}{rgb}{0.000000,0.000000,1.000000}%
\pgfsetfillcolor{currentfill}%
\pgfsetlinewidth{1.003750pt}%
\definecolor{currentstroke}{rgb}{0.000000,0.000000,0.000000}%
\pgfsetstrokecolor{currentstroke}%
\pgfsetdash{}{0pt}%
\pgfpathmoveto{\pgfqpoint{6.936156in}{4.824012in}}%
\pgfpathcurveto{\pgfqpoint{6.944393in}{4.824012in}}{\pgfqpoint{6.952293in}{4.827284in}}{\pgfqpoint{6.958117in}{4.833108in}}%
\pgfpathcurveto{\pgfqpoint{6.963940in}{4.838932in}}{\pgfqpoint{6.967213in}{4.846832in}}{\pgfqpoint{6.967213in}{4.855069in}}%
\pgfpathcurveto{\pgfqpoint{6.967213in}{4.863305in}}{\pgfqpoint{6.963940in}{4.871205in}}{\pgfqpoint{6.958117in}{4.877029in}}%
\pgfpathcurveto{\pgfqpoint{6.952293in}{4.882853in}}{\pgfqpoint{6.944393in}{4.886125in}}{\pgfqpoint{6.936156in}{4.886125in}}%
\pgfpathcurveto{\pgfqpoint{6.927920in}{4.886125in}}{\pgfqpoint{6.920020in}{4.882853in}}{\pgfqpoint{6.914196in}{4.877029in}}%
\pgfpathcurveto{\pgfqpoint{6.908372in}{4.871205in}}{\pgfqpoint{6.905100in}{4.863305in}}{\pgfqpoint{6.905100in}{4.855069in}}%
\pgfpathcurveto{\pgfqpoint{6.905100in}{4.846832in}}{\pgfqpoint{6.908372in}{4.838932in}}{\pgfqpoint{6.914196in}{4.833108in}}%
\pgfpathcurveto{\pgfqpoint{6.920020in}{4.827284in}}{\pgfqpoint{6.927920in}{4.824012in}}{\pgfqpoint{6.936156in}{4.824012in}}%
\pgfpathclose%
\pgfusepath{stroke,fill}%
\end{pgfscope}%
\begin{pgfscope}%
\pgfpathrectangle{\pgfqpoint{0.894063in}{3.540000in}}{\pgfqpoint{6.713438in}{2.060556in}} %
\pgfusepath{clip}%
\pgfsetbuttcap%
\pgfsetroundjoin%
\definecolor{currentfill}{rgb}{0.000000,0.000000,1.000000}%
\pgfsetfillcolor{currentfill}%
\pgfsetlinewidth{1.003750pt}%
\definecolor{currentstroke}{rgb}{0.000000,0.000000,0.000000}%
\pgfsetstrokecolor{currentstroke}%
\pgfsetdash{}{0pt}%
\pgfpathmoveto{\pgfqpoint{5.862006in}{4.978019in}}%
\pgfpathcurveto{\pgfqpoint{5.870243in}{4.978019in}}{\pgfqpoint{5.878143in}{4.981292in}}{\pgfqpoint{5.883967in}{4.987116in}}%
\pgfpathcurveto{\pgfqpoint{5.889790in}{4.992939in}}{\pgfqpoint{5.893063in}{5.000840in}}{\pgfqpoint{5.893063in}{5.009076in}}%
\pgfpathcurveto{\pgfqpoint{5.893063in}{5.017312in}}{\pgfqpoint{5.889790in}{5.025212in}}{\pgfqpoint{5.883967in}{5.031036in}}%
\pgfpathcurveto{\pgfqpoint{5.878143in}{5.036860in}}{\pgfqpoint{5.870243in}{5.040132in}}{\pgfqpoint{5.862006in}{5.040132in}}%
\pgfpathcurveto{\pgfqpoint{5.853770in}{5.040132in}}{\pgfqpoint{5.845870in}{5.036860in}}{\pgfqpoint{5.840046in}{5.031036in}}%
\pgfpathcurveto{\pgfqpoint{5.834222in}{5.025212in}}{\pgfqpoint{5.830950in}{5.017312in}}{\pgfqpoint{5.830950in}{5.009076in}}%
\pgfpathcurveto{\pgfqpoint{5.830950in}{5.000840in}}{\pgfqpoint{5.834222in}{4.992939in}}{\pgfqpoint{5.840046in}{4.987116in}}%
\pgfpathcurveto{\pgfqpoint{5.845870in}{4.981292in}}{\pgfqpoint{5.853770in}{4.978019in}}{\pgfqpoint{5.862006in}{4.978019in}}%
\pgfpathclose%
\pgfusepath{stroke,fill}%
\end{pgfscope}%
\begin{pgfscope}%
\pgfpathrectangle{\pgfqpoint{0.894063in}{3.540000in}}{\pgfqpoint{6.713438in}{2.060556in}} %
\pgfusepath{clip}%
\pgfsetbuttcap%
\pgfsetroundjoin%
\definecolor{currentfill}{rgb}{0.000000,0.000000,1.000000}%
\pgfsetfillcolor{currentfill}%
\pgfsetlinewidth{1.003750pt}%
\definecolor{currentstroke}{rgb}{0.000000,0.000000,0.000000}%
\pgfsetstrokecolor{currentstroke}%
\pgfsetdash{}{0pt}%
\pgfpathmoveto{\pgfqpoint{7.070425in}{4.804809in}}%
\pgfpathcurveto{\pgfqpoint{7.078661in}{4.804809in}}{\pgfqpoint{7.086561in}{4.808081in}}{\pgfqpoint{7.092385in}{4.813905in}}%
\pgfpathcurveto{\pgfqpoint{7.098209in}{4.819729in}}{\pgfqpoint{7.101481in}{4.827629in}}{\pgfqpoint{7.101481in}{4.835866in}}%
\pgfpathcurveto{\pgfqpoint{7.101481in}{4.844102in}}{\pgfqpoint{7.098209in}{4.852002in}}{\pgfqpoint{7.092385in}{4.857826in}}%
\pgfpathcurveto{\pgfqpoint{7.086561in}{4.863650in}}{\pgfqpoint{7.078661in}{4.866922in}}{\pgfqpoint{7.070425in}{4.866922in}}%
\pgfpathcurveto{\pgfqpoint{7.062189in}{4.866922in}}{\pgfqpoint{7.054289in}{4.863650in}}{\pgfqpoint{7.048465in}{4.857826in}}%
\pgfpathcurveto{\pgfqpoint{7.042641in}{4.852002in}}{\pgfqpoint{7.039369in}{4.844102in}}{\pgfqpoint{7.039369in}{4.835866in}}%
\pgfpathcurveto{\pgfqpoint{7.039369in}{4.827629in}}{\pgfqpoint{7.042641in}{4.819729in}}{\pgfqpoint{7.048465in}{4.813905in}}%
\pgfpathcurveto{\pgfqpoint{7.054289in}{4.808081in}}{\pgfqpoint{7.062189in}{4.804809in}}{\pgfqpoint{7.070425in}{4.804809in}}%
\pgfpathclose%
\pgfusepath{stroke,fill}%
\end{pgfscope}%
\begin{pgfscope}%
\pgfpathrectangle{\pgfqpoint{0.894063in}{3.540000in}}{\pgfqpoint{6.713438in}{2.060556in}} %
\pgfusepath{clip}%
\pgfsetbuttcap%
\pgfsetroundjoin%
\definecolor{currentfill}{rgb}{0.000000,0.000000,1.000000}%
\pgfsetfillcolor{currentfill}%
\pgfsetlinewidth{1.003750pt}%
\definecolor{currentstroke}{rgb}{0.000000,0.000000,0.000000}%
\pgfsetstrokecolor{currentstroke}%
\pgfsetdash{}{0pt}%
\pgfpathmoveto{\pgfqpoint{3.176631in}{5.126971in}}%
\pgfpathcurveto{\pgfqpoint{3.184868in}{5.126971in}}{\pgfqpoint{3.192768in}{5.130244in}}{\pgfqpoint{3.198592in}{5.136068in}}%
\pgfpathcurveto{\pgfqpoint{3.204415in}{5.141892in}}{\pgfqpoint{3.207688in}{5.149792in}}{\pgfqpoint{3.207688in}{5.158028in}}%
\pgfpathcurveto{\pgfqpoint{3.207688in}{5.166264in}}{\pgfqpoint{3.204415in}{5.174164in}}{\pgfqpoint{3.198592in}{5.179988in}}%
\pgfpathcurveto{\pgfqpoint{3.192768in}{5.185812in}}{\pgfqpoint{3.184868in}{5.189084in}}{\pgfqpoint{3.176631in}{5.189084in}}%
\pgfpathcurveto{\pgfqpoint{3.168395in}{5.189084in}}{\pgfqpoint{3.160495in}{5.185812in}}{\pgfqpoint{3.154671in}{5.179988in}}%
\pgfpathcurveto{\pgfqpoint{3.148847in}{5.174164in}}{\pgfqpoint{3.145575in}{5.166264in}}{\pgfqpoint{3.145575in}{5.158028in}}%
\pgfpathcurveto{\pgfqpoint{3.145575in}{5.149792in}}{\pgfqpoint{3.148847in}{5.141892in}}{\pgfqpoint{3.154671in}{5.136068in}}%
\pgfpathcurveto{\pgfqpoint{3.160495in}{5.130244in}}{\pgfqpoint{3.168395in}{5.126971in}}{\pgfqpoint{3.176631in}{5.126971in}}%
\pgfpathclose%
\pgfusepath{stroke,fill}%
\end{pgfscope}%
\begin{pgfscope}%
\pgfpathrectangle{\pgfqpoint{0.894063in}{3.540000in}}{\pgfqpoint{6.713438in}{2.060556in}} %
\pgfusepath{clip}%
\pgfsetbuttcap%
\pgfsetroundjoin%
\definecolor{currentfill}{rgb}{0.000000,0.000000,1.000000}%
\pgfsetfillcolor{currentfill}%
\pgfsetlinewidth{1.003750pt}%
\definecolor{currentstroke}{rgb}{0.000000,0.000000,0.000000}%
\pgfsetstrokecolor{currentstroke}%
\pgfsetdash{}{0pt}%
\pgfpathmoveto{\pgfqpoint{2.102481in}{5.129469in}}%
\pgfpathcurveto{\pgfqpoint{2.110718in}{5.129469in}}{\pgfqpoint{2.118618in}{5.132741in}}{\pgfqpoint{2.124442in}{5.138565in}}%
\pgfpathcurveto{\pgfqpoint{2.130265in}{5.144389in}}{\pgfqpoint{2.133538in}{5.152289in}}{\pgfqpoint{2.133538in}{5.160525in}}%
\pgfpathcurveto{\pgfqpoint{2.133538in}{5.168762in}}{\pgfqpoint{2.130265in}{5.176662in}}{\pgfqpoint{2.124442in}{5.182486in}}%
\pgfpathcurveto{\pgfqpoint{2.118618in}{5.188309in}}{\pgfqpoint{2.110718in}{5.191582in}}{\pgfqpoint{2.102481in}{5.191582in}}%
\pgfpathcurveto{\pgfqpoint{2.094245in}{5.191582in}}{\pgfqpoint{2.086345in}{5.188309in}}{\pgfqpoint{2.080521in}{5.182486in}}%
\pgfpathcurveto{\pgfqpoint{2.074697in}{5.176662in}}{\pgfqpoint{2.071425in}{5.168762in}}{\pgfqpoint{2.071425in}{5.160525in}}%
\pgfpathcurveto{\pgfqpoint{2.071425in}{5.152289in}}{\pgfqpoint{2.074697in}{5.144389in}}{\pgfqpoint{2.080521in}{5.138565in}}%
\pgfpathcurveto{\pgfqpoint{2.086345in}{5.132741in}}{\pgfqpoint{2.094245in}{5.129469in}}{\pgfqpoint{2.102481in}{5.129469in}}%
\pgfpathclose%
\pgfusepath{stroke,fill}%
\end{pgfscope}%
\begin{pgfscope}%
\pgfpathrectangle{\pgfqpoint{0.894063in}{3.540000in}}{\pgfqpoint{6.713438in}{2.060556in}} %
\pgfusepath{clip}%
\pgfsetbuttcap%
\pgfsetroundjoin%
\definecolor{currentfill}{rgb}{0.000000,0.000000,1.000000}%
\pgfsetfillcolor{currentfill}%
\pgfsetlinewidth{1.003750pt}%
\definecolor{currentstroke}{rgb}{0.000000,0.000000,0.000000}%
\pgfsetstrokecolor{currentstroke}%
\pgfsetdash{}{0pt}%
\pgfpathmoveto{\pgfqpoint{1.968213in}{5.131800in}}%
\pgfpathcurveto{\pgfqpoint{1.976449in}{5.131800in}}{\pgfqpoint{1.984349in}{5.135072in}}{\pgfqpoint{1.990173in}{5.140896in}}%
\pgfpathcurveto{\pgfqpoint{1.995997in}{5.146720in}}{\pgfqpoint{1.999269in}{5.154620in}}{\pgfqpoint{1.999269in}{5.162856in}}%
\pgfpathcurveto{\pgfqpoint{1.999269in}{5.171093in}}{\pgfqpoint{1.995997in}{5.178993in}}{\pgfqpoint{1.990173in}{5.184817in}}%
\pgfpathcurveto{\pgfqpoint{1.984349in}{5.190641in}}{\pgfqpoint{1.976449in}{5.193913in}}{\pgfqpoint{1.968213in}{5.193913in}}%
\pgfpathcurveto{\pgfqpoint{1.959976in}{5.193913in}}{\pgfqpoint{1.952076in}{5.190641in}}{\pgfqpoint{1.946252in}{5.184817in}}%
\pgfpathcurveto{\pgfqpoint{1.940428in}{5.178993in}}{\pgfqpoint{1.937156in}{5.171093in}}{\pgfqpoint{1.937156in}{5.162856in}}%
\pgfpathcurveto{\pgfqpoint{1.937156in}{5.154620in}}{\pgfqpoint{1.940428in}{5.146720in}}{\pgfqpoint{1.946252in}{5.140896in}}%
\pgfpathcurveto{\pgfqpoint{1.952076in}{5.135072in}}{\pgfqpoint{1.959976in}{5.131800in}}{\pgfqpoint{1.968213in}{5.131800in}}%
\pgfpathclose%
\pgfusepath{stroke,fill}%
\end{pgfscope}%
\begin{pgfscope}%
\pgfpathrectangle{\pgfqpoint{0.894063in}{3.540000in}}{\pgfqpoint{6.713438in}{2.060556in}} %
\pgfusepath{clip}%
\pgfsetbuttcap%
\pgfsetroundjoin%
\definecolor{currentfill}{rgb}{0.000000,0.000000,1.000000}%
\pgfsetfillcolor{currentfill}%
\pgfsetlinewidth{1.003750pt}%
\definecolor{currentstroke}{rgb}{0.000000,0.000000,0.000000}%
\pgfsetstrokecolor{currentstroke}%
\pgfsetdash{}{0pt}%
\pgfpathmoveto{\pgfqpoint{3.310900in}{5.126971in}}%
\pgfpathcurveto{\pgfqpoint{3.319136in}{5.126971in}}{\pgfqpoint{3.327036in}{5.130244in}}{\pgfqpoint{3.332860in}{5.136068in}}%
\pgfpathcurveto{\pgfqpoint{3.338684in}{5.141892in}}{\pgfqpoint{3.341956in}{5.149792in}}{\pgfqpoint{3.341956in}{5.158028in}}%
\pgfpathcurveto{\pgfqpoint{3.341956in}{5.166264in}}{\pgfqpoint{3.338684in}{5.174164in}}{\pgfqpoint{3.332860in}{5.179988in}}%
\pgfpathcurveto{\pgfqpoint{3.327036in}{5.185812in}}{\pgfqpoint{3.319136in}{5.189084in}}{\pgfqpoint{3.310900in}{5.189084in}}%
\pgfpathcurveto{\pgfqpoint{3.302664in}{5.189084in}}{\pgfqpoint{3.294764in}{5.185812in}}{\pgfqpoint{3.288940in}{5.179988in}}%
\pgfpathcurveto{\pgfqpoint{3.283116in}{5.174164in}}{\pgfqpoint{3.279844in}{5.166264in}}{\pgfqpoint{3.279844in}{5.158028in}}%
\pgfpathcurveto{\pgfqpoint{3.279844in}{5.149792in}}{\pgfqpoint{3.283116in}{5.141892in}}{\pgfqpoint{3.288940in}{5.136068in}}%
\pgfpathcurveto{\pgfqpoint{3.294764in}{5.130244in}}{\pgfqpoint{3.302664in}{5.126971in}}{\pgfqpoint{3.310900in}{5.126971in}}%
\pgfpathclose%
\pgfusepath{stroke,fill}%
\end{pgfscope}%
\begin{pgfscope}%
\pgfpathrectangle{\pgfqpoint{0.894063in}{3.540000in}}{\pgfqpoint{6.713438in}{2.060556in}} %
\pgfusepath{clip}%
\pgfsetbuttcap%
\pgfsetroundjoin%
\definecolor{currentfill}{rgb}{0.000000,0.000000,1.000000}%
\pgfsetfillcolor{currentfill}%
\pgfsetlinewidth{1.003750pt}%
\definecolor{currentstroke}{rgb}{0.000000,0.000000,0.000000}%
\pgfsetstrokecolor{currentstroke}%
\pgfsetdash{}{0pt}%
\pgfpathmoveto{\pgfqpoint{5.593469in}{5.016111in}}%
\pgfpathcurveto{\pgfqpoint{5.601705in}{5.016111in}}{\pgfqpoint{5.609605in}{5.019383in}}{\pgfqpoint{5.615429in}{5.025207in}}%
\pgfpathcurveto{\pgfqpoint{5.621253in}{5.031031in}}{\pgfqpoint{5.624525in}{5.038931in}}{\pgfqpoint{5.624525in}{5.047167in}}%
\pgfpathcurveto{\pgfqpoint{5.624525in}{5.055404in}}{\pgfqpoint{5.621253in}{5.063304in}}{\pgfqpoint{5.615429in}{5.069128in}}%
\pgfpathcurveto{\pgfqpoint{5.609605in}{5.074951in}}{\pgfqpoint{5.601705in}{5.078224in}}{\pgfqpoint{5.593469in}{5.078224in}}%
\pgfpathcurveto{\pgfqpoint{5.585232in}{5.078224in}}{\pgfqpoint{5.577332in}{5.074951in}}{\pgfqpoint{5.571508in}{5.069128in}}%
\pgfpathcurveto{\pgfqpoint{5.565685in}{5.063304in}}{\pgfqpoint{5.562412in}{5.055404in}}{\pgfqpoint{5.562412in}{5.047167in}}%
\pgfpathcurveto{\pgfqpoint{5.562412in}{5.038931in}}{\pgfqpoint{5.565685in}{5.031031in}}{\pgfqpoint{5.571508in}{5.025207in}}%
\pgfpathcurveto{\pgfqpoint{5.577332in}{5.019383in}}{\pgfqpoint{5.585232in}{5.016111in}}{\pgfqpoint{5.593469in}{5.016111in}}%
\pgfpathclose%
\pgfusepath{stroke,fill}%
\end{pgfscope}%
\begin{pgfscope}%
\pgfpathrectangle{\pgfqpoint{0.894063in}{3.540000in}}{\pgfqpoint{6.713438in}{2.060556in}} %
\pgfusepath{clip}%
\pgfsetbuttcap%
\pgfsetroundjoin%
\definecolor{currentfill}{rgb}{0.000000,0.000000,1.000000}%
\pgfsetfillcolor{currentfill}%
\pgfsetlinewidth{1.003750pt}%
\definecolor{currentstroke}{rgb}{0.000000,0.000000,0.000000}%
\pgfsetstrokecolor{currentstroke}%
\pgfsetdash{}{0pt}%
\pgfpathmoveto{\pgfqpoint{3.042363in}{5.126978in}}%
\pgfpathcurveto{\pgfqpoint{3.050599in}{5.126978in}}{\pgfqpoint{3.058499in}{5.130251in}}{\pgfqpoint{3.064323in}{5.136075in}}%
\pgfpathcurveto{\pgfqpoint{3.070147in}{5.141898in}}{\pgfqpoint{3.073419in}{5.149798in}}{\pgfqpoint{3.073419in}{5.158035in}}%
\pgfpathcurveto{\pgfqpoint{3.073419in}{5.166271in}}{\pgfqpoint{3.070147in}{5.174171in}}{\pgfqpoint{3.064323in}{5.179995in}}%
\pgfpathcurveto{\pgfqpoint{3.058499in}{5.185819in}}{\pgfqpoint{3.050599in}{5.189091in}}{\pgfqpoint{3.042363in}{5.189091in}}%
\pgfpathcurveto{\pgfqpoint{3.034126in}{5.189091in}}{\pgfqpoint{3.026226in}{5.185819in}}{\pgfqpoint{3.020402in}{5.179995in}}%
\pgfpathcurveto{\pgfqpoint{3.014578in}{5.174171in}}{\pgfqpoint{3.011306in}{5.166271in}}{\pgfqpoint{3.011306in}{5.158035in}}%
\pgfpathcurveto{\pgfqpoint{3.011306in}{5.149798in}}{\pgfqpoint{3.014578in}{5.141898in}}{\pgfqpoint{3.020402in}{5.136075in}}%
\pgfpathcurveto{\pgfqpoint{3.026226in}{5.130251in}}{\pgfqpoint{3.034126in}{5.126978in}}{\pgfqpoint{3.042363in}{5.126978in}}%
\pgfpathclose%
\pgfusepath{stroke,fill}%
\end{pgfscope}%
\begin{pgfscope}%
\pgfpathrectangle{\pgfqpoint{0.894063in}{3.540000in}}{\pgfqpoint{6.713438in}{2.060556in}} %
\pgfusepath{clip}%
\pgfsetbuttcap%
\pgfsetroundjoin%
\definecolor{currentfill}{rgb}{0.000000,0.000000,1.000000}%
\pgfsetfillcolor{currentfill}%
\pgfsetlinewidth{1.003750pt}%
\definecolor{currentstroke}{rgb}{0.000000,0.000000,0.000000}%
\pgfsetstrokecolor{currentstroke}%
\pgfsetdash{}{0pt}%
\pgfpathmoveto{\pgfqpoint{5.190663in}{5.075959in}}%
\pgfpathcurveto{\pgfqpoint{5.198899in}{5.075959in}}{\pgfqpoint{5.206799in}{5.079231in}}{\pgfqpoint{5.212623in}{5.085055in}}%
\pgfpathcurveto{\pgfqpoint{5.218447in}{5.090879in}}{\pgfqpoint{5.221719in}{5.098779in}}{\pgfqpoint{5.221719in}{5.107015in}}%
\pgfpathcurveto{\pgfqpoint{5.221719in}{5.115252in}}{\pgfqpoint{5.218447in}{5.123152in}}{\pgfqpoint{5.212623in}{5.128976in}}%
\pgfpathcurveto{\pgfqpoint{5.206799in}{5.134800in}}{\pgfqpoint{5.198899in}{5.138072in}}{\pgfqpoint{5.190663in}{5.138072in}}%
\pgfpathcurveto{\pgfqpoint{5.182426in}{5.138072in}}{\pgfqpoint{5.174526in}{5.134800in}}{\pgfqpoint{5.168702in}{5.128976in}}%
\pgfpathcurveto{\pgfqpoint{5.162878in}{5.123152in}}{\pgfqpoint{5.159606in}{5.115252in}}{\pgfqpoint{5.159606in}{5.107015in}}%
\pgfpathcurveto{\pgfqpoint{5.159606in}{5.098779in}}{\pgfqpoint{5.162878in}{5.090879in}}{\pgfqpoint{5.168702in}{5.085055in}}%
\pgfpathcurveto{\pgfqpoint{5.174526in}{5.079231in}}{\pgfqpoint{5.182426in}{5.075959in}}{\pgfqpoint{5.190663in}{5.075959in}}%
\pgfpathclose%
\pgfusepath{stroke,fill}%
\end{pgfscope}%
\begin{pgfscope}%
\pgfpathrectangle{\pgfqpoint{0.894063in}{3.540000in}}{\pgfqpoint{6.713438in}{2.060556in}} %
\pgfusepath{clip}%
\pgfsetbuttcap%
\pgfsetroundjoin%
\definecolor{currentfill}{rgb}{0.000000,0.000000,1.000000}%
\pgfsetfillcolor{currentfill}%
\pgfsetlinewidth{1.003750pt}%
\definecolor{currentstroke}{rgb}{0.000000,0.000000,0.000000}%
\pgfsetstrokecolor{currentstroke}%
\pgfsetdash{}{0pt}%
\pgfpathmoveto{\pgfqpoint{6.801888in}{4.842877in}}%
\pgfpathcurveto{\pgfqpoint{6.810124in}{4.842877in}}{\pgfqpoint{6.818024in}{4.846149in}}{\pgfqpoint{6.823848in}{4.851973in}}%
\pgfpathcurveto{\pgfqpoint{6.829672in}{4.857797in}}{\pgfqpoint{6.832944in}{4.865697in}}{\pgfqpoint{6.832944in}{4.873934in}}%
\pgfpathcurveto{\pgfqpoint{6.832944in}{4.882170in}}{\pgfqpoint{6.829672in}{4.890070in}}{\pgfqpoint{6.823848in}{4.895894in}}%
\pgfpathcurveto{\pgfqpoint{6.818024in}{4.901718in}}{\pgfqpoint{6.810124in}{4.904990in}}{\pgfqpoint{6.801888in}{4.904990in}}%
\pgfpathcurveto{\pgfqpoint{6.793651in}{4.904990in}}{\pgfqpoint{6.785751in}{4.901718in}}{\pgfqpoint{6.779927in}{4.895894in}}%
\pgfpathcurveto{\pgfqpoint{6.774103in}{4.890070in}}{\pgfqpoint{6.770831in}{4.882170in}}{\pgfqpoint{6.770831in}{4.873934in}}%
\pgfpathcurveto{\pgfqpoint{6.770831in}{4.865697in}}{\pgfqpoint{6.774103in}{4.857797in}}{\pgfqpoint{6.779927in}{4.851973in}}%
\pgfpathcurveto{\pgfqpoint{6.785751in}{4.846149in}}{\pgfqpoint{6.793651in}{4.842877in}}{\pgfqpoint{6.801888in}{4.842877in}}%
\pgfpathclose%
\pgfusepath{stroke,fill}%
\end{pgfscope}%
\begin{pgfscope}%
\pgfpathrectangle{\pgfqpoint{0.894063in}{3.540000in}}{\pgfqpoint{6.713438in}{2.060556in}} %
\pgfusepath{clip}%
\pgfsetbuttcap%
\pgfsetroundjoin%
\definecolor{currentfill}{rgb}{0.000000,0.000000,1.000000}%
\pgfsetfillcolor{currentfill}%
\pgfsetlinewidth{1.003750pt}%
\definecolor{currentstroke}{rgb}{0.000000,0.000000,0.000000}%
\pgfsetstrokecolor{currentstroke}%
\pgfsetdash{}{0pt}%
\pgfpathmoveto{\pgfqpoint{3.579438in}{5.126965in}}%
\pgfpathcurveto{\pgfqpoint{3.587674in}{5.126965in}}{\pgfqpoint{3.595574in}{5.130237in}}{\pgfqpoint{3.601398in}{5.136061in}}%
\pgfpathcurveto{\pgfqpoint{3.607222in}{5.141885in}}{\pgfqpoint{3.610494in}{5.149785in}}{\pgfqpoint{3.610494in}{5.158021in}}%
\pgfpathcurveto{\pgfqpoint{3.610494in}{5.166257in}}{\pgfqpoint{3.607222in}{5.174157in}}{\pgfqpoint{3.601398in}{5.179981in}}%
\pgfpathcurveto{\pgfqpoint{3.595574in}{5.185805in}}{\pgfqpoint{3.587674in}{5.189078in}}{\pgfqpoint{3.579438in}{5.189078in}}%
\pgfpathcurveto{\pgfqpoint{3.571201in}{5.189078in}}{\pgfqpoint{3.563301in}{5.185805in}}{\pgfqpoint{3.557477in}{5.179981in}}%
\pgfpathcurveto{\pgfqpoint{3.551653in}{5.174157in}}{\pgfqpoint{3.548381in}{5.166257in}}{\pgfqpoint{3.548381in}{5.158021in}}%
\pgfpathcurveto{\pgfqpoint{3.548381in}{5.149785in}}{\pgfqpoint{3.551653in}{5.141885in}}{\pgfqpoint{3.557477in}{5.136061in}}%
\pgfpathcurveto{\pgfqpoint{3.563301in}{5.130237in}}{\pgfqpoint{3.571201in}{5.126965in}}{\pgfqpoint{3.579438in}{5.126965in}}%
\pgfpathclose%
\pgfusepath{stroke,fill}%
\end{pgfscope}%
\begin{pgfscope}%
\pgfpathrectangle{\pgfqpoint{0.894063in}{3.540000in}}{\pgfqpoint{6.713438in}{2.060556in}} %
\pgfusepath{clip}%
\pgfsetbuttcap%
\pgfsetroundjoin%
\definecolor{currentfill}{rgb}{0.000000,0.000000,1.000000}%
\pgfsetfillcolor{currentfill}%
\pgfsetlinewidth{1.003750pt}%
\definecolor{currentstroke}{rgb}{0.000000,0.000000,0.000000}%
\pgfsetstrokecolor{currentstroke}%
\pgfsetdash{}{0pt}%
\pgfpathmoveto{\pgfqpoint{2.371019in}{5.127014in}}%
\pgfpathcurveto{\pgfqpoint{2.379255in}{5.127014in}}{\pgfqpoint{2.387155in}{5.130286in}}{\pgfqpoint{2.392979in}{5.136110in}}%
\pgfpathcurveto{\pgfqpoint{2.398803in}{5.141934in}}{\pgfqpoint{2.402075in}{5.149834in}}{\pgfqpoint{2.402075in}{5.158070in}}%
\pgfpathcurveto{\pgfqpoint{2.402075in}{5.166307in}}{\pgfqpoint{2.398803in}{5.174207in}}{\pgfqpoint{2.392979in}{5.180031in}}%
\pgfpathcurveto{\pgfqpoint{2.387155in}{5.185855in}}{\pgfqpoint{2.379255in}{5.189127in}}{\pgfqpoint{2.371019in}{5.189127in}}%
\pgfpathcurveto{\pgfqpoint{2.362782in}{5.189127in}}{\pgfqpoint{2.354882in}{5.185855in}}{\pgfqpoint{2.349058in}{5.180031in}}%
\pgfpathcurveto{\pgfqpoint{2.343235in}{5.174207in}}{\pgfqpoint{2.339962in}{5.166307in}}{\pgfqpoint{2.339962in}{5.158070in}}%
\pgfpathcurveto{\pgfqpoint{2.339962in}{5.149834in}}{\pgfqpoint{2.343235in}{5.141934in}}{\pgfqpoint{2.349058in}{5.136110in}}%
\pgfpathcurveto{\pgfqpoint{2.354882in}{5.130286in}}{\pgfqpoint{2.362782in}{5.127014in}}{\pgfqpoint{2.371019in}{5.127014in}}%
\pgfpathclose%
\pgfusepath{stroke,fill}%
\end{pgfscope}%
\begin{pgfscope}%
\pgfpathrectangle{\pgfqpoint{0.894063in}{3.540000in}}{\pgfqpoint{6.713438in}{2.060556in}} %
\pgfusepath{clip}%
\pgfsetbuttcap%
\pgfsetroundjoin%
\definecolor{currentfill}{rgb}{0.000000,0.000000,1.000000}%
\pgfsetfillcolor{currentfill}%
\pgfsetlinewidth{1.003750pt}%
\definecolor{currentstroke}{rgb}{0.000000,0.000000,0.000000}%
\pgfsetstrokecolor{currentstroke}%
\pgfsetdash{}{0pt}%
\pgfpathmoveto{\pgfqpoint{3.982244in}{5.126948in}}%
\pgfpathcurveto{\pgfqpoint{3.990480in}{5.126948in}}{\pgfqpoint{3.998380in}{5.130220in}}{\pgfqpoint{4.004204in}{5.136044in}}%
\pgfpathcurveto{\pgfqpoint{4.010028in}{5.141868in}}{\pgfqpoint{4.013300in}{5.149768in}}{\pgfqpoint{4.013300in}{5.158005in}}%
\pgfpathcurveto{\pgfqpoint{4.013300in}{5.166241in}}{\pgfqpoint{4.010028in}{5.174141in}}{\pgfqpoint{4.004204in}{5.179965in}}%
\pgfpathcurveto{\pgfqpoint{3.998380in}{5.185789in}}{\pgfqpoint{3.990480in}{5.189061in}}{\pgfqpoint{3.982244in}{5.189061in}}%
\pgfpathcurveto{\pgfqpoint{3.974007in}{5.189061in}}{\pgfqpoint{3.966107in}{5.185789in}}{\pgfqpoint{3.960283in}{5.179965in}}%
\pgfpathcurveto{\pgfqpoint{3.954460in}{5.174141in}}{\pgfqpoint{3.951187in}{5.166241in}}{\pgfqpoint{3.951187in}{5.158005in}}%
\pgfpathcurveto{\pgfqpoint{3.951187in}{5.149768in}}{\pgfqpoint{3.954460in}{5.141868in}}{\pgfqpoint{3.960283in}{5.136044in}}%
\pgfpathcurveto{\pgfqpoint{3.966107in}{5.130220in}}{\pgfqpoint{3.974007in}{5.126948in}}{\pgfqpoint{3.982244in}{5.126948in}}%
\pgfpathclose%
\pgfusepath{stroke,fill}%
\end{pgfscope}%
\begin{pgfscope}%
\pgfpathrectangle{\pgfqpoint{0.894063in}{3.540000in}}{\pgfqpoint{6.713438in}{2.060556in}} %
\pgfusepath{clip}%
\pgfsetbuttcap%
\pgfsetroundjoin%
\definecolor{currentfill}{rgb}{0.000000,0.000000,1.000000}%
\pgfsetfillcolor{currentfill}%
\pgfsetlinewidth{1.003750pt}%
\definecolor{currentstroke}{rgb}{0.000000,0.000000,0.000000}%
\pgfsetstrokecolor{currentstroke}%
\pgfsetdash{}{0pt}%
\pgfpathmoveto{\pgfqpoint{4.653588in}{5.126938in}}%
\pgfpathcurveto{\pgfqpoint{4.661824in}{5.126938in}}{\pgfqpoint{4.669724in}{5.130211in}}{\pgfqpoint{4.675548in}{5.136035in}}%
\pgfpathcurveto{\pgfqpoint{4.681372in}{5.141859in}}{\pgfqpoint{4.684644in}{5.149759in}}{\pgfqpoint{4.684644in}{5.157995in}}%
\pgfpathcurveto{\pgfqpoint{4.684644in}{5.166231in}}{\pgfqpoint{4.681372in}{5.174131in}}{\pgfqpoint{4.675548in}{5.179955in}}%
\pgfpathcurveto{\pgfqpoint{4.669724in}{5.185779in}}{\pgfqpoint{4.661824in}{5.189051in}}{\pgfqpoint{4.653588in}{5.189051in}}%
\pgfpathcurveto{\pgfqpoint{4.645351in}{5.189051in}}{\pgfqpoint{4.637451in}{5.185779in}}{\pgfqpoint{4.631627in}{5.179955in}}%
\pgfpathcurveto{\pgfqpoint{4.625803in}{5.174131in}}{\pgfqpoint{4.622531in}{5.166231in}}{\pgfqpoint{4.622531in}{5.157995in}}%
\pgfpathcurveto{\pgfqpoint{4.622531in}{5.149759in}}{\pgfqpoint{4.625803in}{5.141859in}}{\pgfqpoint{4.631627in}{5.136035in}}%
\pgfpathcurveto{\pgfqpoint{4.637451in}{5.130211in}}{\pgfqpoint{4.645351in}{5.126938in}}{\pgfqpoint{4.653588in}{5.126938in}}%
\pgfpathclose%
\pgfusepath{stroke,fill}%
\end{pgfscope}%
\begin{pgfscope}%
\pgfpathrectangle{\pgfqpoint{0.894063in}{3.540000in}}{\pgfqpoint{6.713438in}{2.060556in}} %
\pgfusepath{clip}%
\pgfsetbuttcap%
\pgfsetroundjoin%
\definecolor{currentfill}{rgb}{0.000000,0.000000,1.000000}%
\pgfsetfillcolor{currentfill}%
\pgfsetlinewidth{1.003750pt}%
\definecolor{currentstroke}{rgb}{0.000000,0.000000,0.000000}%
\pgfsetstrokecolor{currentstroke}%
\pgfsetdash{}{0pt}%
\pgfpathmoveto{\pgfqpoint{3.713706in}{5.126955in}}%
\pgfpathcurveto{\pgfqpoint{3.721943in}{5.126955in}}{\pgfqpoint{3.729843in}{5.130227in}}{\pgfqpoint{3.735667in}{5.136051in}}%
\pgfpathcurveto{\pgfqpoint{3.741490in}{5.141875in}}{\pgfqpoint{3.744763in}{5.149775in}}{\pgfqpoint{3.744763in}{5.158011in}}%
\pgfpathcurveto{\pgfqpoint{3.744763in}{5.166248in}}{\pgfqpoint{3.741490in}{5.174148in}}{\pgfqpoint{3.735667in}{5.179972in}}%
\pgfpathcurveto{\pgfqpoint{3.729843in}{5.185796in}}{\pgfqpoint{3.721943in}{5.189068in}}{\pgfqpoint{3.713706in}{5.189068in}}%
\pgfpathcurveto{\pgfqpoint{3.705470in}{5.189068in}}{\pgfqpoint{3.697570in}{5.185796in}}{\pgfqpoint{3.691746in}{5.179972in}}%
\pgfpathcurveto{\pgfqpoint{3.685922in}{5.174148in}}{\pgfqpoint{3.682650in}{5.166248in}}{\pgfqpoint{3.682650in}{5.158011in}}%
\pgfpathcurveto{\pgfqpoint{3.682650in}{5.149775in}}{\pgfqpoint{3.685922in}{5.141875in}}{\pgfqpoint{3.691746in}{5.136051in}}%
\pgfpathcurveto{\pgfqpoint{3.697570in}{5.130227in}}{\pgfqpoint{3.705470in}{5.126955in}}{\pgfqpoint{3.713706in}{5.126955in}}%
\pgfpathclose%
\pgfusepath{stroke,fill}%
\end{pgfscope}%
\begin{pgfscope}%
\pgfpathrectangle{\pgfqpoint{0.894063in}{3.540000in}}{\pgfqpoint{6.713438in}{2.060556in}} %
\pgfusepath{clip}%
\pgfsetbuttcap%
\pgfsetroundjoin%
\definecolor{currentfill}{rgb}{0.000000,0.000000,1.000000}%
\pgfsetfillcolor{currentfill}%
\pgfsetlinewidth{1.003750pt}%
\definecolor{currentstroke}{rgb}{0.000000,0.000000,0.000000}%
\pgfsetstrokecolor{currentstroke}%
\pgfsetdash{}{0pt}%
\pgfpathmoveto{\pgfqpoint{2.236750in}{5.127017in}}%
\pgfpathcurveto{\pgfqpoint{2.244986in}{5.127017in}}{\pgfqpoint{2.252886in}{5.130289in}}{\pgfqpoint{2.258710in}{5.136113in}}%
\pgfpathcurveto{\pgfqpoint{2.264534in}{5.141937in}}{\pgfqpoint{2.267806in}{5.149837in}}{\pgfqpoint{2.267806in}{5.158073in}}%
\pgfpathcurveto{\pgfqpoint{2.267806in}{5.166310in}}{\pgfqpoint{2.264534in}{5.174210in}}{\pgfqpoint{2.258710in}{5.180033in}}%
\pgfpathcurveto{\pgfqpoint{2.252886in}{5.185857in}}{\pgfqpoint{2.244986in}{5.189130in}}{\pgfqpoint{2.236750in}{5.189130in}}%
\pgfpathcurveto{\pgfqpoint{2.228514in}{5.189130in}}{\pgfqpoint{2.220614in}{5.185857in}}{\pgfqpoint{2.214790in}{5.180033in}}%
\pgfpathcurveto{\pgfqpoint{2.208966in}{5.174210in}}{\pgfqpoint{2.205694in}{5.166310in}}{\pgfqpoint{2.205694in}{5.158073in}}%
\pgfpathcurveto{\pgfqpoint{2.205694in}{5.149837in}}{\pgfqpoint{2.208966in}{5.141937in}}{\pgfqpoint{2.214790in}{5.136113in}}%
\pgfpathcurveto{\pgfqpoint{2.220614in}{5.130289in}}{\pgfqpoint{2.228514in}{5.127017in}}{\pgfqpoint{2.236750in}{5.127017in}}%
\pgfpathclose%
\pgfusepath{stroke,fill}%
\end{pgfscope}%
\begin{pgfscope}%
\pgfpathrectangle{\pgfqpoint{0.894063in}{3.540000in}}{\pgfqpoint{6.713438in}{2.060556in}} %
\pgfusepath{clip}%
\pgfsetbuttcap%
\pgfsetroundjoin%
\definecolor{currentfill}{rgb}{0.000000,0.000000,1.000000}%
\pgfsetfillcolor{currentfill}%
\pgfsetlinewidth{1.003750pt}%
\definecolor{currentstroke}{rgb}{0.000000,0.000000,0.000000}%
\pgfsetstrokecolor{currentstroke}%
\pgfsetdash{}{0pt}%
\pgfpathmoveto{\pgfqpoint{6.667619in}{4.862436in}}%
\pgfpathcurveto{\pgfqpoint{6.675855in}{4.862436in}}{\pgfqpoint{6.683755in}{4.865708in}}{\pgfqpoint{6.689579in}{4.871532in}}%
\pgfpathcurveto{\pgfqpoint{6.695403in}{4.877356in}}{\pgfqpoint{6.698675in}{4.885256in}}{\pgfqpoint{6.698675in}{4.893492in}}%
\pgfpathcurveto{\pgfqpoint{6.698675in}{4.901729in}}{\pgfqpoint{6.695403in}{4.909629in}}{\pgfqpoint{6.689579in}{4.915453in}}%
\pgfpathcurveto{\pgfqpoint{6.683755in}{4.921277in}}{\pgfqpoint{6.675855in}{4.924549in}}{\pgfqpoint{6.667619in}{4.924549in}}%
\pgfpathcurveto{\pgfqpoint{6.659382in}{4.924549in}}{\pgfqpoint{6.651482in}{4.921277in}}{\pgfqpoint{6.645658in}{4.915453in}}%
\pgfpathcurveto{\pgfqpoint{6.639835in}{4.909629in}}{\pgfqpoint{6.636562in}{4.901729in}}{\pgfqpoint{6.636562in}{4.893492in}}%
\pgfpathcurveto{\pgfqpoint{6.636562in}{4.885256in}}{\pgfqpoint{6.639835in}{4.877356in}}{\pgfqpoint{6.645658in}{4.871532in}}%
\pgfpathcurveto{\pgfqpoint{6.651482in}{4.865708in}}{\pgfqpoint{6.659382in}{4.862436in}}{\pgfqpoint{6.667619in}{4.862436in}}%
\pgfpathclose%
\pgfusepath{stroke,fill}%
\end{pgfscope}%
\begin{pgfscope}%
\pgfpathrectangle{\pgfqpoint{0.894063in}{3.540000in}}{\pgfqpoint{6.713438in}{2.060556in}} %
\pgfusepath{clip}%
\pgfsetbuttcap%
\pgfsetroundjoin%
\definecolor{currentfill}{rgb}{0.000000,0.000000,1.000000}%
\pgfsetfillcolor{currentfill}%
\pgfsetlinewidth{1.003750pt}%
\definecolor{currentstroke}{rgb}{0.000000,0.000000,0.000000}%
\pgfsetstrokecolor{currentstroke}%
\pgfsetdash{}{0pt}%
\pgfpathmoveto{\pgfqpoint{2.639556in}{5.126980in}}%
\pgfpathcurveto{\pgfqpoint{2.647793in}{5.126980in}}{\pgfqpoint{2.655693in}{5.130252in}}{\pgfqpoint{2.661517in}{5.136076in}}%
\pgfpathcurveto{\pgfqpoint{2.667340in}{5.141900in}}{\pgfqpoint{2.670613in}{5.149800in}}{\pgfqpoint{2.670613in}{5.158036in}}%
\pgfpathcurveto{\pgfqpoint{2.670613in}{5.166272in}}{\pgfqpoint{2.667340in}{5.174172in}}{\pgfqpoint{2.661517in}{5.179996in}}%
\pgfpathcurveto{\pgfqpoint{2.655693in}{5.185820in}}{\pgfqpoint{2.647793in}{5.189093in}}{\pgfqpoint{2.639556in}{5.189093in}}%
\pgfpathcurveto{\pgfqpoint{2.631320in}{5.189093in}}{\pgfqpoint{2.623420in}{5.185820in}}{\pgfqpoint{2.617596in}{5.179996in}}%
\pgfpathcurveto{\pgfqpoint{2.611772in}{5.174172in}}{\pgfqpoint{2.608500in}{5.166272in}}{\pgfqpoint{2.608500in}{5.158036in}}%
\pgfpathcurveto{\pgfqpoint{2.608500in}{5.149800in}}{\pgfqpoint{2.611772in}{5.141900in}}{\pgfqpoint{2.617596in}{5.136076in}}%
\pgfpathcurveto{\pgfqpoint{2.623420in}{5.130252in}}{\pgfqpoint{2.631320in}{5.126980in}}{\pgfqpoint{2.639556in}{5.126980in}}%
\pgfpathclose%
\pgfusepath{stroke,fill}%
\end{pgfscope}%
\begin{pgfscope}%
\pgfpathrectangle{\pgfqpoint{0.894063in}{3.540000in}}{\pgfqpoint{6.713438in}{2.060556in}} %
\pgfusepath{clip}%
\pgfsetbuttcap%
\pgfsetroundjoin%
\definecolor{currentfill}{rgb}{0.000000,0.000000,1.000000}%
\pgfsetfillcolor{currentfill}%
\pgfsetlinewidth{1.003750pt}%
\definecolor{currentstroke}{rgb}{0.000000,0.000000,0.000000}%
\pgfsetstrokecolor{currentstroke}%
\pgfsetdash{}{0pt}%
\pgfpathmoveto{\pgfqpoint{1.699675in}{5.136873in}}%
\pgfpathcurveto{\pgfqpoint{1.707911in}{5.136873in}}{\pgfqpoint{1.715811in}{5.140145in}}{\pgfqpoint{1.721635in}{5.145969in}}%
\pgfpathcurveto{\pgfqpoint{1.727459in}{5.151793in}}{\pgfqpoint{1.730731in}{5.159693in}}{\pgfqpoint{1.730731in}{5.167930in}}%
\pgfpathcurveto{\pgfqpoint{1.730731in}{5.176166in}}{\pgfqpoint{1.727459in}{5.184066in}}{\pgfqpoint{1.721635in}{5.189890in}}%
\pgfpathcurveto{\pgfqpoint{1.715811in}{5.195714in}}{\pgfqpoint{1.707911in}{5.198986in}}{\pgfqpoint{1.699675in}{5.198986in}}%
\pgfpathcurveto{\pgfqpoint{1.691439in}{5.198986in}}{\pgfqpoint{1.683539in}{5.195714in}}{\pgfqpoint{1.677715in}{5.189890in}}%
\pgfpathcurveto{\pgfqpoint{1.671891in}{5.184066in}}{\pgfqpoint{1.668619in}{5.176166in}}{\pgfqpoint{1.668619in}{5.167930in}}%
\pgfpathcurveto{\pgfqpoint{1.668619in}{5.159693in}}{\pgfqpoint{1.671891in}{5.151793in}}{\pgfqpoint{1.677715in}{5.145969in}}%
\pgfpathcurveto{\pgfqpoint{1.683539in}{5.140145in}}{\pgfqpoint{1.691439in}{5.136873in}}{\pgfqpoint{1.699675in}{5.136873in}}%
\pgfpathclose%
\pgfusepath{stroke,fill}%
\end{pgfscope}%
\begin{pgfscope}%
\pgfpathrectangle{\pgfqpoint{0.894063in}{3.540000in}}{\pgfqpoint{6.713438in}{2.060556in}} %
\pgfusepath{clip}%
\pgfsetbuttcap%
\pgfsetroundjoin%
\definecolor{currentfill}{rgb}{0.000000,0.000000,1.000000}%
\pgfsetfillcolor{currentfill}%
\pgfsetlinewidth{1.003750pt}%
\definecolor{currentstroke}{rgb}{0.000000,0.000000,0.000000}%
\pgfsetstrokecolor{currentstroke}%
\pgfsetdash{}{0pt}%
\pgfpathmoveto{\pgfqpoint{1.162600in}{5.264758in}}%
\pgfpathcurveto{\pgfqpoint{1.170836in}{5.264758in}}{\pgfqpoint{1.178736in}{5.268030in}}{\pgfqpoint{1.184560in}{5.273854in}}%
\pgfpathcurveto{\pgfqpoint{1.190384in}{5.279678in}}{\pgfqpoint{1.193656in}{5.287578in}}{\pgfqpoint{1.193656in}{5.295814in}}%
\pgfpathcurveto{\pgfqpoint{1.193656in}{5.304051in}}{\pgfqpoint{1.190384in}{5.311951in}}{\pgfqpoint{1.184560in}{5.317775in}}%
\pgfpathcurveto{\pgfqpoint{1.178736in}{5.323599in}}{\pgfqpoint{1.170836in}{5.326871in}}{\pgfqpoint{1.162600in}{5.326871in}}%
\pgfpathcurveto{\pgfqpoint{1.154364in}{5.326871in}}{\pgfqpoint{1.146464in}{5.323599in}}{\pgfqpoint{1.140640in}{5.317775in}}%
\pgfpathcurveto{\pgfqpoint{1.134816in}{5.311951in}}{\pgfqpoint{1.131544in}{5.304051in}}{\pgfqpoint{1.131544in}{5.295814in}}%
\pgfpathcurveto{\pgfqpoint{1.131544in}{5.287578in}}{\pgfqpoint{1.134816in}{5.279678in}}{\pgfqpoint{1.140640in}{5.273854in}}%
\pgfpathcurveto{\pgfqpoint{1.146464in}{5.268030in}}{\pgfqpoint{1.154364in}{5.264758in}}{\pgfqpoint{1.162600in}{5.264758in}}%
\pgfpathclose%
\pgfusepath{stroke,fill}%
\end{pgfscope}%
\begin{pgfscope}%
\pgfpathrectangle{\pgfqpoint{0.894063in}{3.540000in}}{\pgfqpoint{6.713438in}{2.060556in}} %
\pgfusepath{clip}%
\pgfsetbuttcap%
\pgfsetroundjoin%
\definecolor{currentfill}{rgb}{0.000000,0.000000,1.000000}%
\pgfsetfillcolor{currentfill}%
\pgfsetlinewidth{1.003750pt}%
\definecolor{currentstroke}{rgb}{0.000000,0.000000,0.000000}%
\pgfsetstrokecolor{currentstroke}%
\pgfsetdash{}{0pt}%
\pgfpathmoveto{\pgfqpoint{1.833944in}{5.134903in}}%
\pgfpathcurveto{\pgfqpoint{1.842180in}{5.134903in}}{\pgfqpoint{1.850080in}{5.138175in}}{\pgfqpoint{1.855904in}{5.143999in}}%
\pgfpathcurveto{\pgfqpoint{1.861728in}{5.149823in}}{\pgfqpoint{1.865000in}{5.157723in}}{\pgfqpoint{1.865000in}{5.165960in}}%
\pgfpathcurveto{\pgfqpoint{1.865000in}{5.174196in}}{\pgfqpoint{1.861728in}{5.182096in}}{\pgfqpoint{1.855904in}{5.187920in}}%
\pgfpathcurveto{\pgfqpoint{1.850080in}{5.193744in}}{\pgfqpoint{1.842180in}{5.197016in}}{\pgfqpoint{1.833944in}{5.197016in}}%
\pgfpathcurveto{\pgfqpoint{1.825707in}{5.197016in}}{\pgfqpoint{1.817807in}{5.193744in}}{\pgfqpoint{1.811983in}{5.187920in}}%
\pgfpathcurveto{\pgfqpoint{1.806160in}{5.182096in}}{\pgfqpoint{1.802887in}{5.174196in}}{\pgfqpoint{1.802887in}{5.165960in}}%
\pgfpathcurveto{\pgfqpoint{1.802887in}{5.157723in}}{\pgfqpoint{1.806160in}{5.149823in}}{\pgfqpoint{1.811983in}{5.143999in}}%
\pgfpathcurveto{\pgfqpoint{1.817807in}{5.138175in}}{\pgfqpoint{1.825707in}{5.134903in}}{\pgfqpoint{1.833944in}{5.134903in}}%
\pgfpathclose%
\pgfusepath{stroke,fill}%
\end{pgfscope}%
\begin{pgfscope}%
\pgfpathrectangle{\pgfqpoint{0.894063in}{3.540000in}}{\pgfqpoint{6.713438in}{2.060556in}} %
\pgfusepath{clip}%
\pgfsetbuttcap%
\pgfsetroundjoin%
\definecolor{currentfill}{rgb}{0.000000,0.000000,1.000000}%
\pgfsetfillcolor{currentfill}%
\pgfsetlinewidth{1.003750pt}%
\definecolor{currentstroke}{rgb}{0.000000,0.000000,0.000000}%
\pgfsetstrokecolor{currentstroke}%
\pgfsetdash{}{0pt}%
\pgfpathmoveto{\pgfqpoint{5.996275in}{4.978095in}}%
\pgfpathcurveto{\pgfqpoint{6.004511in}{4.978095in}}{\pgfqpoint{6.012411in}{4.981367in}}{\pgfqpoint{6.018235in}{4.987191in}}%
\pgfpathcurveto{\pgfqpoint{6.024059in}{4.993015in}}{\pgfqpoint{6.027331in}{5.000915in}}{\pgfqpoint{6.027331in}{5.009151in}}%
\pgfpathcurveto{\pgfqpoint{6.027331in}{5.017388in}}{\pgfqpoint{6.024059in}{5.025288in}}{\pgfqpoint{6.018235in}{5.031112in}}%
\pgfpathcurveto{\pgfqpoint{6.012411in}{5.036936in}}{\pgfqpoint{6.004511in}{5.040208in}}{\pgfqpoint{5.996275in}{5.040208in}}%
\pgfpathcurveto{\pgfqpoint{5.988039in}{5.040208in}}{\pgfqpoint{5.980139in}{5.036936in}}{\pgfqpoint{5.974315in}{5.031112in}}%
\pgfpathcurveto{\pgfqpoint{5.968491in}{5.025288in}}{\pgfqpoint{5.965219in}{5.017388in}}{\pgfqpoint{5.965219in}{5.009151in}}%
\pgfpathcurveto{\pgfqpoint{5.965219in}{5.000915in}}{\pgfqpoint{5.968491in}{4.993015in}}{\pgfqpoint{5.974315in}{4.987191in}}%
\pgfpathcurveto{\pgfqpoint{5.980139in}{4.981367in}}{\pgfqpoint{5.988039in}{4.978095in}}{\pgfqpoint{5.996275in}{4.978095in}}%
\pgfpathclose%
\pgfusepath{stroke,fill}%
\end{pgfscope}%
\begin{pgfscope}%
\pgfpathrectangle{\pgfqpoint{0.894063in}{3.540000in}}{\pgfqpoint{6.713438in}{2.060556in}} %
\pgfusepath{clip}%
\pgfsetbuttcap%
\pgfsetroundjoin%
\definecolor{currentfill}{rgb}{0.000000,0.000000,1.000000}%
\pgfsetfillcolor{currentfill}%
\pgfsetlinewidth{1.003750pt}%
\definecolor{currentstroke}{rgb}{0.000000,0.000000,0.000000}%
\pgfsetstrokecolor{currentstroke}%
\pgfsetdash{}{0pt}%
\pgfpathmoveto{\pgfqpoint{6.399081in}{4.901033in}}%
\pgfpathcurveto{\pgfqpoint{6.407318in}{4.901033in}}{\pgfqpoint{6.415218in}{4.904305in}}{\pgfqpoint{6.421042in}{4.910129in}}%
\pgfpathcurveto{\pgfqpoint{6.426865in}{4.915953in}}{\pgfqpoint{6.430138in}{4.923853in}}{\pgfqpoint{6.430138in}{4.932089in}}%
\pgfpathcurveto{\pgfqpoint{6.430138in}{4.940326in}}{\pgfqpoint{6.426865in}{4.948226in}}{\pgfqpoint{6.421042in}{4.954050in}}%
\pgfpathcurveto{\pgfqpoint{6.415218in}{4.959874in}}{\pgfqpoint{6.407318in}{4.963146in}}{\pgfqpoint{6.399081in}{4.963146in}}%
\pgfpathcurveto{\pgfqpoint{6.390845in}{4.963146in}}{\pgfqpoint{6.382945in}{4.959874in}}{\pgfqpoint{6.377121in}{4.954050in}}%
\pgfpathcurveto{\pgfqpoint{6.371297in}{4.948226in}}{\pgfqpoint{6.368025in}{4.940326in}}{\pgfqpoint{6.368025in}{4.932089in}}%
\pgfpathcurveto{\pgfqpoint{6.368025in}{4.923853in}}{\pgfqpoint{6.371297in}{4.915953in}}{\pgfqpoint{6.377121in}{4.910129in}}%
\pgfpathcurveto{\pgfqpoint{6.382945in}{4.904305in}}{\pgfqpoint{6.390845in}{4.901033in}}{\pgfqpoint{6.399081in}{4.901033in}}%
\pgfpathclose%
\pgfusepath{stroke,fill}%
\end{pgfscope}%
\begin{pgfscope}%
\pgfpathrectangle{\pgfqpoint{0.894063in}{3.540000in}}{\pgfqpoint{6.713438in}{2.060556in}} %
\pgfusepath{clip}%
\pgfsetbuttcap%
\pgfsetroundjoin%
\definecolor{currentfill}{rgb}{0.000000,0.000000,1.000000}%
\pgfsetfillcolor{currentfill}%
\pgfsetlinewidth{1.003750pt}%
\definecolor{currentstroke}{rgb}{0.000000,0.000000,0.000000}%
\pgfsetstrokecolor{currentstroke}%
\pgfsetdash{}{0pt}%
\pgfpathmoveto{\pgfqpoint{4.787856in}{5.126937in}}%
\pgfpathcurveto{\pgfqpoint{4.796093in}{5.126937in}}{\pgfqpoint{4.803993in}{5.130209in}}{\pgfqpoint{4.809817in}{5.136033in}}%
\pgfpathcurveto{\pgfqpoint{4.815640in}{5.141857in}}{\pgfqpoint{4.818913in}{5.149757in}}{\pgfqpoint{4.818913in}{5.157994in}}%
\pgfpathcurveto{\pgfqpoint{4.818913in}{5.166230in}}{\pgfqpoint{4.815640in}{5.174130in}}{\pgfqpoint{4.809817in}{5.179954in}}%
\pgfpathcurveto{\pgfqpoint{4.803993in}{5.185778in}}{\pgfqpoint{4.796093in}{5.189050in}}{\pgfqpoint{4.787856in}{5.189050in}}%
\pgfpathcurveto{\pgfqpoint{4.779620in}{5.189050in}}{\pgfqpoint{4.771720in}{5.185778in}}{\pgfqpoint{4.765896in}{5.179954in}}%
\pgfpathcurveto{\pgfqpoint{4.760072in}{5.174130in}}{\pgfqpoint{4.756800in}{5.166230in}}{\pgfqpoint{4.756800in}{5.157994in}}%
\pgfpathcurveto{\pgfqpoint{4.756800in}{5.149757in}}{\pgfqpoint{4.760072in}{5.141857in}}{\pgfqpoint{4.765896in}{5.136033in}}%
\pgfpathcurveto{\pgfqpoint{4.771720in}{5.130209in}}{\pgfqpoint{4.779620in}{5.126937in}}{\pgfqpoint{4.787856in}{5.126937in}}%
\pgfpathclose%
\pgfusepath{stroke,fill}%
\end{pgfscope}%
\begin{pgfscope}%
\pgfpathrectangle{\pgfqpoint{0.894063in}{3.540000in}}{\pgfqpoint{6.713438in}{2.060556in}} %
\pgfusepath{clip}%
\pgfsetbuttcap%
\pgfsetroundjoin%
\definecolor{currentfill}{rgb}{0.000000,0.000000,1.000000}%
\pgfsetfillcolor{currentfill}%
\pgfsetlinewidth{1.003750pt}%
\definecolor{currentstroke}{rgb}{0.000000,0.000000,0.000000}%
\pgfsetstrokecolor{currentstroke}%
\pgfsetdash{}{0pt}%
\pgfpathmoveto{\pgfqpoint{4.922125in}{5.123234in}}%
\pgfpathcurveto{\pgfqpoint{4.930361in}{5.123234in}}{\pgfqpoint{4.938261in}{5.126506in}}{\pgfqpoint{4.944085in}{5.132330in}}%
\pgfpathcurveto{\pgfqpoint{4.949909in}{5.138154in}}{\pgfqpoint{4.953181in}{5.146054in}}{\pgfqpoint{4.953181in}{5.154290in}}%
\pgfpathcurveto{\pgfqpoint{4.953181in}{5.162526in}}{\pgfqpoint{4.949909in}{5.170426in}}{\pgfqpoint{4.944085in}{5.176250in}}%
\pgfpathcurveto{\pgfqpoint{4.938261in}{5.182074in}}{\pgfqpoint{4.930361in}{5.185347in}}{\pgfqpoint{4.922125in}{5.185347in}}%
\pgfpathcurveto{\pgfqpoint{4.913889in}{5.185347in}}{\pgfqpoint{4.905989in}{5.182074in}}{\pgfqpoint{4.900165in}{5.176250in}}%
\pgfpathcurveto{\pgfqpoint{4.894341in}{5.170426in}}{\pgfqpoint{4.891069in}{5.162526in}}{\pgfqpoint{4.891069in}{5.154290in}}%
\pgfpathcurveto{\pgfqpoint{4.891069in}{5.146054in}}{\pgfqpoint{4.894341in}{5.138154in}}{\pgfqpoint{4.900165in}{5.132330in}}%
\pgfpathcurveto{\pgfqpoint{4.905989in}{5.126506in}}{\pgfqpoint{4.913889in}{5.123234in}}{\pgfqpoint{4.922125in}{5.123234in}}%
\pgfpathclose%
\pgfusepath{stroke,fill}%
\end{pgfscope}%
\begin{pgfscope}%
\pgfpathrectangle{\pgfqpoint{0.894063in}{3.540000in}}{\pgfqpoint{6.713438in}{2.060556in}} %
\pgfusepath{clip}%
\pgfsetbuttcap%
\pgfsetroundjoin%
\definecolor{currentfill}{rgb}{0.000000,0.000000,1.000000}%
\pgfsetfillcolor{currentfill}%
\pgfsetlinewidth{1.003750pt}%
\definecolor{currentstroke}{rgb}{0.000000,0.000000,0.000000}%
\pgfsetstrokecolor{currentstroke}%
\pgfsetdash{}{0pt}%
\pgfpathmoveto{\pgfqpoint{6.130544in}{4.950523in}}%
\pgfpathcurveto{\pgfqpoint{6.138780in}{4.950523in}}{\pgfqpoint{6.146680in}{4.953796in}}{\pgfqpoint{6.152504in}{4.959620in}}%
\pgfpathcurveto{\pgfqpoint{6.158328in}{4.965443in}}{\pgfqpoint{6.161600in}{4.973343in}}{\pgfqpoint{6.161600in}{4.981580in}}%
\pgfpathcurveto{\pgfqpoint{6.161600in}{4.989816in}}{\pgfqpoint{6.158328in}{4.997716in}}{\pgfqpoint{6.152504in}{5.003540in}}%
\pgfpathcurveto{\pgfqpoint{6.146680in}{5.009364in}}{\pgfqpoint{6.138780in}{5.012636in}}{\pgfqpoint{6.130544in}{5.012636in}}%
\pgfpathcurveto{\pgfqpoint{6.122307in}{5.012636in}}{\pgfqpoint{6.114407in}{5.009364in}}{\pgfqpoint{6.108583in}{5.003540in}}%
\pgfpathcurveto{\pgfqpoint{6.102760in}{4.997716in}}{\pgfqpoint{6.099487in}{4.989816in}}{\pgfqpoint{6.099487in}{4.981580in}}%
\pgfpathcurveto{\pgfqpoint{6.099487in}{4.973343in}}{\pgfqpoint{6.102760in}{4.965443in}}{\pgfqpoint{6.108583in}{4.959620in}}%
\pgfpathcurveto{\pgfqpoint{6.114407in}{4.953796in}}{\pgfqpoint{6.122307in}{4.950523in}}{\pgfqpoint{6.130544in}{4.950523in}}%
\pgfpathclose%
\pgfusepath{stroke,fill}%
\end{pgfscope}%
\begin{pgfscope}%
\pgfpathrectangle{\pgfqpoint{0.894063in}{3.540000in}}{\pgfqpoint{6.713438in}{2.060556in}} %
\pgfusepath{clip}%
\pgfsetbuttcap%
\pgfsetroundjoin%
\definecolor{currentfill}{rgb}{0.000000,0.000000,1.000000}%
\pgfsetfillcolor{currentfill}%
\pgfsetlinewidth{1.003750pt}%
\definecolor{currentstroke}{rgb}{0.000000,0.000000,0.000000}%
\pgfsetstrokecolor{currentstroke}%
\pgfsetdash{}{0pt}%
\pgfpathmoveto{\pgfqpoint{5.727738in}{5.013700in}}%
\pgfpathcurveto{\pgfqpoint{5.735974in}{5.013700in}}{\pgfqpoint{5.743874in}{5.016972in}}{\pgfqpoint{5.749698in}{5.022796in}}%
\pgfpathcurveto{\pgfqpoint{5.755522in}{5.028620in}}{\pgfqpoint{5.758794in}{5.036520in}}{\pgfqpoint{5.758794in}{5.044756in}}%
\pgfpathcurveto{\pgfqpoint{5.758794in}{5.052993in}}{\pgfqpoint{5.755522in}{5.060893in}}{\pgfqpoint{5.749698in}{5.066717in}}%
\pgfpathcurveto{\pgfqpoint{5.743874in}{5.072541in}}{\pgfqpoint{5.735974in}{5.075813in}}{\pgfqpoint{5.727738in}{5.075813in}}%
\pgfpathcurveto{\pgfqpoint{5.719501in}{5.075813in}}{\pgfqpoint{5.711601in}{5.072541in}}{\pgfqpoint{5.705777in}{5.066717in}}%
\pgfpathcurveto{\pgfqpoint{5.699953in}{5.060893in}}{\pgfqpoint{5.696681in}{5.052993in}}{\pgfqpoint{5.696681in}{5.044756in}}%
\pgfpathcurveto{\pgfqpoint{5.696681in}{5.036520in}}{\pgfqpoint{5.699953in}{5.028620in}}{\pgfqpoint{5.705777in}{5.022796in}}%
\pgfpathcurveto{\pgfqpoint{5.711601in}{5.016972in}}{\pgfqpoint{5.719501in}{5.013700in}}{\pgfqpoint{5.727738in}{5.013700in}}%
\pgfpathclose%
\pgfusepath{stroke,fill}%
\end{pgfscope}%
\begin{pgfscope}%
\pgfpathrectangle{\pgfqpoint{0.894063in}{3.540000in}}{\pgfqpoint{6.713438in}{2.060556in}} %
\pgfusepath{clip}%
\pgfsetbuttcap%
\pgfsetroundjoin%
\definecolor{currentfill}{rgb}{0.000000,0.000000,1.000000}%
\pgfsetfillcolor{currentfill}%
\pgfsetlinewidth{1.003750pt}%
\definecolor{currentstroke}{rgb}{0.000000,0.000000,0.000000}%
\pgfsetstrokecolor{currentstroke}%
\pgfsetdash{}{0pt}%
\pgfpathmoveto{\pgfqpoint{1.028331in}{5.264787in}}%
\pgfpathcurveto{\pgfqpoint{1.036568in}{5.264787in}}{\pgfqpoint{1.044468in}{5.268059in}}{\pgfqpoint{1.050292in}{5.273883in}}%
\pgfpathcurveto{\pgfqpoint{1.056115in}{5.279707in}}{\pgfqpoint{1.059388in}{5.287607in}}{\pgfqpoint{1.059388in}{5.295843in}}%
\pgfpathcurveto{\pgfqpoint{1.059388in}{5.304080in}}{\pgfqpoint{1.056115in}{5.311980in}}{\pgfqpoint{1.050292in}{5.317804in}}%
\pgfpathcurveto{\pgfqpoint{1.044468in}{5.323628in}}{\pgfqpoint{1.036568in}{5.326900in}}{\pgfqpoint{1.028331in}{5.326900in}}%
\pgfpathcurveto{\pgfqpoint{1.020095in}{5.326900in}}{\pgfqpoint{1.012195in}{5.323628in}}{\pgfqpoint{1.006371in}{5.317804in}}%
\pgfpathcurveto{\pgfqpoint{1.000547in}{5.311980in}}{\pgfqpoint{0.997275in}{5.304080in}}{\pgfqpoint{0.997275in}{5.295843in}}%
\pgfpathcurveto{\pgfqpoint{0.997275in}{5.287607in}}{\pgfqpoint{1.000547in}{5.279707in}}{\pgfqpoint{1.006371in}{5.273883in}}%
\pgfpathcurveto{\pgfqpoint{1.012195in}{5.268059in}}{\pgfqpoint{1.020095in}{5.264787in}}{\pgfqpoint{1.028331in}{5.264787in}}%
\pgfpathclose%
\pgfusepath{stroke,fill}%
\end{pgfscope}%
\begin{pgfscope}%
\pgfpathrectangle{\pgfqpoint{0.894063in}{3.540000in}}{\pgfqpoint{6.713438in}{2.060556in}} %
\pgfusepath{clip}%
\pgfsetbuttcap%
\pgfsetroundjoin%
\definecolor{currentfill}{rgb}{0.000000,0.000000,1.000000}%
\pgfsetfillcolor{currentfill}%
\pgfsetlinewidth{1.003750pt}%
\definecolor{currentstroke}{rgb}{0.000000,0.000000,0.000000}%
\pgfsetstrokecolor{currentstroke}%
\pgfsetdash{}{0pt}%
\pgfpathmoveto{\pgfqpoint{5.324931in}{5.075444in}}%
\pgfpathcurveto{\pgfqpoint{5.333168in}{5.075444in}}{\pgfqpoint{5.341068in}{5.078716in}}{\pgfqpoint{5.346892in}{5.084540in}}%
\pgfpathcurveto{\pgfqpoint{5.352715in}{5.090364in}}{\pgfqpoint{5.355988in}{5.098264in}}{\pgfqpoint{5.355988in}{5.106500in}}%
\pgfpathcurveto{\pgfqpoint{5.355988in}{5.114737in}}{\pgfqpoint{5.352715in}{5.122637in}}{\pgfqpoint{5.346892in}{5.128461in}}%
\pgfpathcurveto{\pgfqpoint{5.341068in}{5.134284in}}{\pgfqpoint{5.333168in}{5.137557in}}{\pgfqpoint{5.324931in}{5.137557in}}%
\pgfpathcurveto{\pgfqpoint{5.316695in}{5.137557in}}{\pgfqpoint{5.308795in}{5.134284in}}{\pgfqpoint{5.302971in}{5.128461in}}%
\pgfpathcurveto{\pgfqpoint{5.297147in}{5.122637in}}{\pgfqpoint{5.293875in}{5.114737in}}{\pgfqpoint{5.293875in}{5.106500in}}%
\pgfpathcurveto{\pgfqpoint{5.293875in}{5.098264in}}{\pgfqpoint{5.297147in}{5.090364in}}{\pgfqpoint{5.302971in}{5.084540in}}%
\pgfpathcurveto{\pgfqpoint{5.308795in}{5.078716in}}{\pgfqpoint{5.316695in}{5.075444in}}{\pgfqpoint{5.324931in}{5.075444in}}%
\pgfpathclose%
\pgfusepath{stroke,fill}%
\end{pgfscope}%
\begin{pgfscope}%
\pgfpathrectangle{\pgfqpoint{0.894063in}{3.540000in}}{\pgfqpoint{6.713438in}{2.060556in}} %
\pgfusepath{clip}%
\pgfsetbuttcap%
\pgfsetroundjoin%
\definecolor{currentfill}{rgb}{0.000000,0.000000,1.000000}%
\pgfsetfillcolor{currentfill}%
\pgfsetlinewidth{1.003750pt}%
\definecolor{currentstroke}{rgb}{0.000000,0.000000,0.000000}%
\pgfsetstrokecolor{currentstroke}%
\pgfsetdash{}{0pt}%
\pgfpathmoveto{\pgfqpoint{7.338963in}{4.758918in}}%
\pgfpathcurveto{\pgfqpoint{7.347199in}{4.758918in}}{\pgfqpoint{7.355099in}{4.762190in}}{\pgfqpoint{7.360923in}{4.768014in}}%
\pgfpathcurveto{\pgfqpoint{7.366747in}{4.773838in}}{\pgfqpoint{7.370019in}{4.781738in}}{\pgfqpoint{7.370019in}{4.789974in}}%
\pgfpathcurveto{\pgfqpoint{7.370019in}{4.798210in}}{\pgfqpoint{7.366747in}{4.806111in}}{\pgfqpoint{7.360923in}{4.811934in}}%
\pgfpathcurveto{\pgfqpoint{7.355099in}{4.817758in}}{\pgfqpoint{7.347199in}{4.821031in}}{\pgfqpoint{7.338963in}{4.821031in}}%
\pgfpathcurveto{\pgfqpoint{7.330726in}{4.821031in}}{\pgfqpoint{7.322826in}{4.817758in}}{\pgfqpoint{7.317002in}{4.811934in}}%
\pgfpathcurveto{\pgfqpoint{7.311178in}{4.806111in}}{\pgfqpoint{7.307906in}{4.798210in}}{\pgfqpoint{7.307906in}{4.789974in}}%
\pgfpathcurveto{\pgfqpoint{7.307906in}{4.781738in}}{\pgfqpoint{7.311178in}{4.773838in}}{\pgfqpoint{7.317002in}{4.768014in}}%
\pgfpathcurveto{\pgfqpoint{7.322826in}{4.762190in}}{\pgfqpoint{7.330726in}{4.758918in}}{\pgfqpoint{7.338963in}{4.758918in}}%
\pgfpathclose%
\pgfusepath{stroke,fill}%
\end{pgfscope}%
\begin{pgfscope}%
\pgfpathrectangle{\pgfqpoint{0.894063in}{3.540000in}}{\pgfqpoint{6.713438in}{2.060556in}} %
\pgfusepath{clip}%
\pgfsetbuttcap%
\pgfsetroundjoin%
\definecolor{currentfill}{rgb}{0.000000,0.000000,1.000000}%
\pgfsetfillcolor{currentfill}%
\pgfsetlinewidth{1.003750pt}%
\definecolor{currentstroke}{rgb}{0.000000,0.000000,0.000000}%
\pgfsetstrokecolor{currentstroke}%
\pgfsetdash{}{0pt}%
\pgfpathmoveto{\pgfqpoint{7.204694in}{4.804919in}}%
\pgfpathcurveto{\pgfqpoint{7.212930in}{4.804919in}}{\pgfqpoint{7.220830in}{4.808191in}}{\pgfqpoint{7.226654in}{4.814015in}}%
\pgfpathcurveto{\pgfqpoint{7.232478in}{4.819839in}}{\pgfqpoint{7.235750in}{4.827739in}}{\pgfqpoint{7.235750in}{4.835975in}}%
\pgfpathcurveto{\pgfqpoint{7.235750in}{4.844212in}}{\pgfqpoint{7.232478in}{4.852112in}}{\pgfqpoint{7.226654in}{4.857936in}}%
\pgfpathcurveto{\pgfqpoint{7.220830in}{4.863760in}}{\pgfqpoint{7.212930in}{4.867032in}}{\pgfqpoint{7.204694in}{4.867032in}}%
\pgfpathcurveto{\pgfqpoint{7.196457in}{4.867032in}}{\pgfqpoint{7.188557in}{4.863760in}}{\pgfqpoint{7.182733in}{4.857936in}}%
\pgfpathcurveto{\pgfqpoint{7.176910in}{4.852112in}}{\pgfqpoint{7.173637in}{4.844212in}}{\pgfqpoint{7.173637in}{4.835975in}}%
\pgfpathcurveto{\pgfqpoint{7.173637in}{4.827739in}}{\pgfqpoint{7.176910in}{4.819839in}}{\pgfqpoint{7.182733in}{4.814015in}}%
\pgfpathcurveto{\pgfqpoint{7.188557in}{4.808191in}}{\pgfqpoint{7.196457in}{4.804919in}}{\pgfqpoint{7.204694in}{4.804919in}}%
\pgfpathclose%
\pgfusepath{stroke,fill}%
\end{pgfscope}%
\begin{pgfscope}%
\pgfpathrectangle{\pgfqpoint{0.894063in}{3.540000in}}{\pgfqpoint{6.713438in}{2.060556in}} %
\pgfusepath{clip}%
\pgfsetbuttcap%
\pgfsetroundjoin%
\definecolor{currentfill}{rgb}{0.000000,0.000000,1.000000}%
\pgfsetfillcolor{currentfill}%
\pgfsetlinewidth{1.003750pt}%
\definecolor{currentstroke}{rgb}{0.000000,0.000000,0.000000}%
\pgfsetstrokecolor{currentstroke}%
\pgfsetdash{}{0pt}%
\pgfpathmoveto{\pgfqpoint{6.264813in}{4.925272in}}%
\pgfpathcurveto{\pgfqpoint{6.273049in}{4.925272in}}{\pgfqpoint{6.280949in}{4.928544in}}{\pgfqpoint{6.286773in}{4.934368in}}%
\pgfpathcurveto{\pgfqpoint{6.292597in}{4.940192in}}{\pgfqpoint{6.295869in}{4.948092in}}{\pgfqpoint{6.295869in}{4.956328in}}%
\pgfpathcurveto{\pgfqpoint{6.295869in}{4.964565in}}{\pgfqpoint{6.292597in}{4.972465in}}{\pgfqpoint{6.286773in}{4.978289in}}%
\pgfpathcurveto{\pgfqpoint{6.280949in}{4.984113in}}{\pgfqpoint{6.273049in}{4.987385in}}{\pgfqpoint{6.264813in}{4.987385in}}%
\pgfpathcurveto{\pgfqpoint{6.256576in}{4.987385in}}{\pgfqpoint{6.248676in}{4.984113in}}{\pgfqpoint{6.242852in}{4.978289in}}%
\pgfpathcurveto{\pgfqpoint{6.237028in}{4.972465in}}{\pgfqpoint{6.233756in}{4.964565in}}{\pgfqpoint{6.233756in}{4.956328in}}%
\pgfpathcurveto{\pgfqpoint{6.233756in}{4.948092in}}{\pgfqpoint{6.237028in}{4.940192in}}{\pgfqpoint{6.242852in}{4.934368in}}%
\pgfpathcurveto{\pgfqpoint{6.248676in}{4.928544in}}{\pgfqpoint{6.256576in}{4.925272in}}{\pgfqpoint{6.264813in}{4.925272in}}%
\pgfpathclose%
\pgfusepath{stroke,fill}%
\end{pgfscope}%
\begin{pgfscope}%
\pgfpathrectangle{\pgfqpoint{0.894063in}{3.540000in}}{\pgfqpoint{6.713438in}{2.060556in}} %
\pgfusepath{clip}%
\pgfsetbuttcap%
\pgfsetroundjoin%
\definecolor{currentfill}{rgb}{0.000000,0.000000,1.000000}%
\pgfsetfillcolor{currentfill}%
\pgfsetlinewidth{1.003750pt}%
\definecolor{currentstroke}{rgb}{0.000000,0.000000,0.000000}%
\pgfsetstrokecolor{currentstroke}%
\pgfsetdash{}{0pt}%
\pgfpathmoveto{\pgfqpoint{7.473231in}{4.740707in}}%
\pgfpathcurveto{\pgfqpoint{7.481468in}{4.740707in}}{\pgfqpoint{7.489368in}{4.743979in}}{\pgfqpoint{7.495192in}{4.749803in}}%
\pgfpathcurveto{\pgfqpoint{7.501015in}{4.755627in}}{\pgfqpoint{7.504288in}{4.763527in}}{\pgfqpoint{7.504288in}{4.771763in}}%
\pgfpathcurveto{\pgfqpoint{7.504288in}{4.779999in}}{\pgfqpoint{7.501015in}{4.787899in}}{\pgfqpoint{7.495192in}{4.793723in}}%
\pgfpathcurveto{\pgfqpoint{7.489368in}{4.799547in}}{\pgfqpoint{7.481468in}{4.802820in}}{\pgfqpoint{7.473231in}{4.802820in}}%
\pgfpathcurveto{\pgfqpoint{7.464995in}{4.802820in}}{\pgfqpoint{7.457095in}{4.799547in}}{\pgfqpoint{7.451271in}{4.793723in}}%
\pgfpathcurveto{\pgfqpoint{7.445447in}{4.787899in}}{\pgfqpoint{7.442175in}{4.779999in}}{\pgfqpoint{7.442175in}{4.771763in}}%
\pgfpathcurveto{\pgfqpoint{7.442175in}{4.763527in}}{\pgfqpoint{7.445447in}{4.755627in}}{\pgfqpoint{7.451271in}{4.749803in}}%
\pgfpathcurveto{\pgfqpoint{7.457095in}{4.743979in}}{\pgfqpoint{7.464995in}{4.740707in}}{\pgfqpoint{7.473231in}{4.740707in}}%
\pgfpathclose%
\pgfusepath{stroke,fill}%
\end{pgfscope}%
\begin{pgfscope}%
\pgfpathrectangle{\pgfqpoint{0.894063in}{3.540000in}}{\pgfqpoint{6.713438in}{2.060556in}} %
\pgfusepath{clip}%
\pgfsetbuttcap%
\pgfsetroundjoin%
\definecolor{currentfill}{rgb}{0.000000,0.000000,1.000000}%
\pgfsetfillcolor{currentfill}%
\pgfsetlinewidth{1.003750pt}%
\definecolor{currentstroke}{rgb}{0.000000,0.000000,0.000000}%
\pgfsetstrokecolor{currentstroke}%
\pgfsetdash{}{0pt}%
\pgfpathmoveto{\pgfqpoint{5.056394in}{5.101346in}}%
\pgfpathcurveto{\pgfqpoint{5.064630in}{5.101346in}}{\pgfqpoint{5.072530in}{5.104619in}}{\pgfqpoint{5.078354in}{5.110443in}}%
\pgfpathcurveto{\pgfqpoint{5.084178in}{5.116266in}}{\pgfqpoint{5.087450in}{5.124167in}}{\pgfqpoint{5.087450in}{5.132403in}}%
\pgfpathcurveto{\pgfqpoint{5.087450in}{5.140639in}}{\pgfqpoint{5.084178in}{5.148539in}}{\pgfqpoint{5.078354in}{5.154363in}}%
\pgfpathcurveto{\pgfqpoint{5.072530in}{5.160187in}}{\pgfqpoint{5.064630in}{5.163459in}}{\pgfqpoint{5.056394in}{5.163459in}}%
\pgfpathcurveto{\pgfqpoint{5.048157in}{5.163459in}}{\pgfqpoint{5.040257in}{5.160187in}}{\pgfqpoint{5.034433in}{5.154363in}}%
\pgfpathcurveto{\pgfqpoint{5.028610in}{5.148539in}}{\pgfqpoint{5.025337in}{5.140639in}}{\pgfqpoint{5.025337in}{5.132403in}}%
\pgfpathcurveto{\pgfqpoint{5.025337in}{5.124167in}}{\pgfqpoint{5.028610in}{5.116266in}}{\pgfqpoint{5.034433in}{5.110443in}}%
\pgfpathcurveto{\pgfqpoint{5.040257in}{5.104619in}}{\pgfqpoint{5.048157in}{5.101346in}}{\pgfqpoint{5.056394in}{5.101346in}}%
\pgfpathclose%
\pgfusepath{stroke,fill}%
\end{pgfscope}%
\begin{pgfscope}%
\pgfpathrectangle{\pgfqpoint{0.894063in}{3.540000in}}{\pgfqpoint{6.713438in}{2.060556in}} %
\pgfusepath{clip}%
\pgfsetbuttcap%
\pgfsetroundjoin%
\definecolor{currentfill}{rgb}{0.000000,0.000000,1.000000}%
\pgfsetfillcolor{currentfill}%
\pgfsetlinewidth{1.003750pt}%
\definecolor{currentstroke}{rgb}{0.000000,0.000000,0.000000}%
\pgfsetstrokecolor{currentstroke}%
\pgfsetdash{}{0pt}%
\pgfpathmoveto{\pgfqpoint{2.908094in}{5.126978in}}%
\pgfpathcurveto{\pgfqpoint{2.916330in}{5.126978in}}{\pgfqpoint{2.924230in}{5.130251in}}{\pgfqpoint{2.930054in}{5.136075in}}%
\pgfpathcurveto{\pgfqpoint{2.935878in}{5.141898in}}{\pgfqpoint{2.939150in}{5.149798in}}{\pgfqpoint{2.939150in}{5.158035in}}%
\pgfpathcurveto{\pgfqpoint{2.939150in}{5.166271in}}{\pgfqpoint{2.935878in}{5.174171in}}{\pgfqpoint{2.930054in}{5.179995in}}%
\pgfpathcurveto{\pgfqpoint{2.924230in}{5.185819in}}{\pgfqpoint{2.916330in}{5.189091in}}{\pgfqpoint{2.908094in}{5.189091in}}%
\pgfpathcurveto{\pgfqpoint{2.899857in}{5.189091in}}{\pgfqpoint{2.891957in}{5.185819in}}{\pgfqpoint{2.886133in}{5.179995in}}%
\pgfpathcurveto{\pgfqpoint{2.880310in}{5.174171in}}{\pgfqpoint{2.877037in}{5.166271in}}{\pgfqpoint{2.877037in}{5.158035in}}%
\pgfpathcurveto{\pgfqpoint{2.877037in}{5.149798in}}{\pgfqpoint{2.880310in}{5.141898in}}{\pgfqpoint{2.886133in}{5.136075in}}%
\pgfpathcurveto{\pgfqpoint{2.891957in}{5.130251in}}{\pgfqpoint{2.899857in}{5.126978in}}{\pgfqpoint{2.908094in}{5.126978in}}%
\pgfpathclose%
\pgfusepath{stroke,fill}%
\end{pgfscope}%
\begin{pgfscope}%
\pgfpathrectangle{\pgfqpoint{0.894063in}{3.540000in}}{\pgfqpoint{6.713438in}{2.060556in}} %
\pgfusepath{clip}%
\pgfsetbuttcap%
\pgfsetroundjoin%
\definecolor{currentfill}{rgb}{0.000000,0.000000,1.000000}%
\pgfsetfillcolor{currentfill}%
\pgfsetlinewidth{1.003750pt}%
\definecolor{currentstroke}{rgb}{0.000000,0.000000,0.000000}%
\pgfsetstrokecolor{currentstroke}%
\pgfsetdash{}{0pt}%
\pgfpathmoveto{\pgfqpoint{3.445169in}{5.126955in}}%
\pgfpathcurveto{\pgfqpoint{3.453405in}{5.126955in}}{\pgfqpoint{3.461305in}{5.130227in}}{\pgfqpoint{3.467129in}{5.136051in}}%
\pgfpathcurveto{\pgfqpoint{3.472953in}{5.141875in}}{\pgfqpoint{3.476225in}{5.149775in}}{\pgfqpoint{3.476225in}{5.158011in}}%
\pgfpathcurveto{\pgfqpoint{3.476225in}{5.166248in}}{\pgfqpoint{3.472953in}{5.174148in}}{\pgfqpoint{3.467129in}{5.179972in}}%
\pgfpathcurveto{\pgfqpoint{3.461305in}{5.185796in}}{\pgfqpoint{3.453405in}{5.189068in}}{\pgfqpoint{3.445169in}{5.189068in}}%
\pgfpathcurveto{\pgfqpoint{3.436932in}{5.189068in}}{\pgfqpoint{3.429032in}{5.185796in}}{\pgfqpoint{3.423208in}{5.179972in}}%
\pgfpathcurveto{\pgfqpoint{3.417385in}{5.174148in}}{\pgfqpoint{3.414112in}{5.166248in}}{\pgfqpoint{3.414112in}{5.158011in}}%
\pgfpathcurveto{\pgfqpoint{3.414112in}{5.149775in}}{\pgfqpoint{3.417385in}{5.141875in}}{\pgfqpoint{3.423208in}{5.136051in}}%
\pgfpathcurveto{\pgfqpoint{3.429032in}{5.130227in}}{\pgfqpoint{3.436932in}{5.126955in}}{\pgfqpoint{3.445169in}{5.126955in}}%
\pgfpathclose%
\pgfusepath{stroke,fill}%
\end{pgfscope}%
\begin{pgfscope}%
\pgfpathrectangle{\pgfqpoint{0.894063in}{3.540000in}}{\pgfqpoint{6.713438in}{2.060556in}} %
\pgfusepath{clip}%
\pgfsetbuttcap%
\pgfsetroundjoin%
\definecolor{currentfill}{rgb}{0.000000,0.000000,1.000000}%
\pgfsetfillcolor{currentfill}%
\pgfsetlinewidth{1.003750pt}%
\definecolor{currentstroke}{rgb}{0.000000,0.000000,0.000000}%
\pgfsetstrokecolor{currentstroke}%
\pgfsetdash{}{0pt}%
\pgfpathmoveto{\pgfqpoint{4.116513in}{5.126943in}}%
\pgfpathcurveto{\pgfqpoint{4.124749in}{5.126943in}}{\pgfqpoint{4.132649in}{5.130215in}}{\pgfqpoint{4.138473in}{5.136039in}}%
\pgfpathcurveto{\pgfqpoint{4.144297in}{5.141863in}}{\pgfqpoint{4.147569in}{5.149763in}}{\pgfqpoint{4.147569in}{5.157999in}}%
\pgfpathcurveto{\pgfqpoint{4.147569in}{5.166235in}}{\pgfqpoint{4.144297in}{5.174135in}}{\pgfqpoint{4.138473in}{5.179959in}}%
\pgfpathcurveto{\pgfqpoint{4.132649in}{5.185783in}}{\pgfqpoint{4.124749in}{5.189056in}}{\pgfqpoint{4.116513in}{5.189056in}}%
\pgfpathcurveto{\pgfqpoint{4.108276in}{5.189056in}}{\pgfqpoint{4.100376in}{5.185783in}}{\pgfqpoint{4.094552in}{5.179959in}}%
\pgfpathcurveto{\pgfqpoint{4.088728in}{5.174135in}}{\pgfqpoint{4.085456in}{5.166235in}}{\pgfqpoint{4.085456in}{5.157999in}}%
\pgfpathcurveto{\pgfqpoint{4.085456in}{5.149763in}}{\pgfqpoint{4.088728in}{5.141863in}}{\pgfqpoint{4.094552in}{5.136039in}}%
\pgfpathcurveto{\pgfqpoint{4.100376in}{5.130215in}}{\pgfqpoint{4.108276in}{5.126943in}}{\pgfqpoint{4.116513in}{5.126943in}}%
\pgfpathclose%
\pgfusepath{stroke,fill}%
\end{pgfscope}%
\begin{pgfscope}%
\pgfpathrectangle{\pgfqpoint{0.894063in}{3.540000in}}{\pgfqpoint{6.713438in}{2.060556in}} %
\pgfusepath{clip}%
\pgfsetbuttcap%
\pgfsetroundjoin%
\definecolor{currentfill}{rgb}{0.000000,0.000000,1.000000}%
\pgfsetfillcolor{currentfill}%
\pgfsetlinewidth{1.003750pt}%
\definecolor{currentstroke}{rgb}{0.000000,0.000000,0.000000}%
\pgfsetstrokecolor{currentstroke}%
\pgfsetdash{}{0pt}%
\pgfpathmoveto{\pgfqpoint{1.431138in}{5.140660in}}%
\pgfpathcurveto{\pgfqpoint{1.439374in}{5.140660in}}{\pgfqpoint{1.447274in}{5.143933in}}{\pgfqpoint{1.453098in}{5.149757in}}%
\pgfpathcurveto{\pgfqpoint{1.458922in}{5.155581in}}{\pgfqpoint{1.462194in}{5.163481in}}{\pgfqpoint{1.462194in}{5.171717in}}%
\pgfpathcurveto{\pgfqpoint{1.462194in}{5.179953in}}{\pgfqpoint{1.458922in}{5.187853in}}{\pgfqpoint{1.453098in}{5.193677in}}%
\pgfpathcurveto{\pgfqpoint{1.447274in}{5.199501in}}{\pgfqpoint{1.439374in}{5.202773in}}{\pgfqpoint{1.431138in}{5.202773in}}%
\pgfpathcurveto{\pgfqpoint{1.422901in}{5.202773in}}{\pgfqpoint{1.415001in}{5.199501in}}{\pgfqpoint{1.409177in}{5.193677in}}%
\pgfpathcurveto{\pgfqpoint{1.403353in}{5.187853in}}{\pgfqpoint{1.400081in}{5.179953in}}{\pgfqpoint{1.400081in}{5.171717in}}%
\pgfpathcurveto{\pgfqpoint{1.400081in}{5.163481in}}{\pgfqpoint{1.403353in}{5.155581in}}{\pgfqpoint{1.409177in}{5.149757in}}%
\pgfpathcurveto{\pgfqpoint{1.415001in}{5.143933in}}{\pgfqpoint{1.422901in}{5.140660in}}{\pgfqpoint{1.431138in}{5.140660in}}%
\pgfpathclose%
\pgfusepath{stroke,fill}%
\end{pgfscope}%
\begin{pgfscope}%
\pgfpathrectangle{\pgfqpoint{0.894063in}{3.540000in}}{\pgfqpoint{6.713438in}{2.060556in}} %
\pgfusepath{clip}%
\pgfsetbuttcap%
\pgfsetroundjoin%
\definecolor{currentfill}{rgb}{0.000000,0.000000,1.000000}%
\pgfsetfillcolor{currentfill}%
\pgfsetlinewidth{1.003750pt}%
\definecolor{currentstroke}{rgb}{0.000000,0.000000,0.000000}%
\pgfsetstrokecolor{currentstroke}%
\pgfsetdash{}{0pt}%
\pgfpathmoveto{\pgfqpoint{2.773825in}{5.126980in}}%
\pgfpathcurveto{\pgfqpoint{2.782061in}{5.126980in}}{\pgfqpoint{2.789961in}{5.130252in}}{\pgfqpoint{2.795785in}{5.136076in}}%
\pgfpathcurveto{\pgfqpoint{2.801609in}{5.141900in}}{\pgfqpoint{2.804881in}{5.149800in}}{\pgfqpoint{2.804881in}{5.158036in}}%
\pgfpathcurveto{\pgfqpoint{2.804881in}{5.166272in}}{\pgfqpoint{2.801609in}{5.174172in}}{\pgfqpoint{2.795785in}{5.179996in}}%
\pgfpathcurveto{\pgfqpoint{2.789961in}{5.185820in}}{\pgfqpoint{2.782061in}{5.189093in}}{\pgfqpoint{2.773825in}{5.189093in}}%
\pgfpathcurveto{\pgfqpoint{2.765589in}{5.189093in}}{\pgfqpoint{2.757689in}{5.185820in}}{\pgfqpoint{2.751865in}{5.179996in}}%
\pgfpathcurveto{\pgfqpoint{2.746041in}{5.174172in}}{\pgfqpoint{2.742769in}{5.166272in}}{\pgfqpoint{2.742769in}{5.158036in}}%
\pgfpathcurveto{\pgfqpoint{2.742769in}{5.149800in}}{\pgfqpoint{2.746041in}{5.141900in}}{\pgfqpoint{2.751865in}{5.136076in}}%
\pgfpathcurveto{\pgfqpoint{2.757689in}{5.130252in}}{\pgfqpoint{2.765589in}{5.126980in}}{\pgfqpoint{2.773825in}{5.126980in}}%
\pgfpathclose%
\pgfusepath{stroke,fill}%
\end{pgfscope}%
\begin{pgfscope}%
\pgfpathrectangle{\pgfqpoint{0.894063in}{3.540000in}}{\pgfqpoint{6.713438in}{2.060556in}} %
\pgfusepath{clip}%
\pgfsetbuttcap%
\pgfsetroundjoin%
\definecolor{currentfill}{rgb}{0.000000,0.000000,1.000000}%
\pgfsetfillcolor{currentfill}%
\pgfsetlinewidth{1.003750pt}%
\definecolor{currentstroke}{rgb}{0.000000,0.000000,0.000000}%
\pgfsetstrokecolor{currentstroke}%
\pgfsetdash{}{0pt}%
\pgfpathmoveto{\pgfqpoint{1.565406in}{5.138398in}}%
\pgfpathcurveto{\pgfqpoint{1.573643in}{5.138398in}}{\pgfqpoint{1.581543in}{5.141670in}}{\pgfqpoint{1.587367in}{5.147494in}}%
\pgfpathcurveto{\pgfqpoint{1.593190in}{5.153318in}}{\pgfqpoint{1.596463in}{5.161218in}}{\pgfqpoint{1.596463in}{5.169454in}}%
\pgfpathcurveto{\pgfqpoint{1.596463in}{5.177691in}}{\pgfqpoint{1.593190in}{5.185591in}}{\pgfqpoint{1.587367in}{5.191415in}}%
\pgfpathcurveto{\pgfqpoint{1.581543in}{5.197239in}}{\pgfqpoint{1.573643in}{5.200511in}}{\pgfqpoint{1.565406in}{5.200511in}}%
\pgfpathcurveto{\pgfqpoint{1.557170in}{5.200511in}}{\pgfqpoint{1.549270in}{5.197239in}}{\pgfqpoint{1.543446in}{5.191415in}}%
\pgfpathcurveto{\pgfqpoint{1.537622in}{5.185591in}}{\pgfqpoint{1.534350in}{5.177691in}}{\pgfqpoint{1.534350in}{5.169454in}}%
\pgfpathcurveto{\pgfqpoint{1.534350in}{5.161218in}}{\pgfqpoint{1.537622in}{5.153318in}}{\pgfqpoint{1.543446in}{5.147494in}}%
\pgfpathcurveto{\pgfqpoint{1.549270in}{5.141670in}}{\pgfqpoint{1.557170in}{5.138398in}}{\pgfqpoint{1.565406in}{5.138398in}}%
\pgfpathclose%
\pgfusepath{stroke,fill}%
\end{pgfscope}%
\begin{pgfscope}%
\pgfpathrectangle{\pgfqpoint{0.894063in}{3.540000in}}{\pgfqpoint{6.713438in}{2.060556in}} %
\pgfusepath{clip}%
\pgfsetbuttcap%
\pgfsetroundjoin%
\definecolor{currentfill}{rgb}{0.000000,0.000000,1.000000}%
\pgfsetfillcolor{currentfill}%
\pgfsetlinewidth{1.003750pt}%
\definecolor{currentstroke}{rgb}{0.000000,0.000000,0.000000}%
\pgfsetstrokecolor{currentstroke}%
\pgfsetdash{}{0pt}%
\pgfpathmoveto{\pgfqpoint{4.250781in}{5.126938in}}%
\pgfpathcurveto{\pgfqpoint{4.259018in}{5.126938in}}{\pgfqpoint{4.266918in}{5.130211in}}{\pgfqpoint{4.272742in}{5.136035in}}%
\pgfpathcurveto{\pgfqpoint{4.278565in}{5.141859in}}{\pgfqpoint{4.281838in}{5.149759in}}{\pgfqpoint{4.281838in}{5.157995in}}%
\pgfpathcurveto{\pgfqpoint{4.281838in}{5.166231in}}{\pgfqpoint{4.278565in}{5.174131in}}{\pgfqpoint{4.272742in}{5.179955in}}%
\pgfpathcurveto{\pgfqpoint{4.266918in}{5.185779in}}{\pgfqpoint{4.259018in}{5.189051in}}{\pgfqpoint{4.250781in}{5.189051in}}%
\pgfpathcurveto{\pgfqpoint{4.242545in}{5.189051in}}{\pgfqpoint{4.234645in}{5.185779in}}{\pgfqpoint{4.228821in}{5.179955in}}%
\pgfpathcurveto{\pgfqpoint{4.222997in}{5.174131in}}{\pgfqpoint{4.219725in}{5.166231in}}{\pgfqpoint{4.219725in}{5.157995in}}%
\pgfpathcurveto{\pgfqpoint{4.219725in}{5.149759in}}{\pgfqpoint{4.222997in}{5.141859in}}{\pgfqpoint{4.228821in}{5.136035in}}%
\pgfpathcurveto{\pgfqpoint{4.234645in}{5.130211in}}{\pgfqpoint{4.242545in}{5.126938in}}{\pgfqpoint{4.250781in}{5.126938in}}%
\pgfpathclose%
\pgfusepath{stroke,fill}%
\end{pgfscope}%
\begin{pgfscope}%
\pgfpathrectangle{\pgfqpoint{0.894063in}{3.540000in}}{\pgfqpoint{6.713438in}{2.060556in}} %
\pgfusepath{clip}%
\pgfsetbuttcap%
\pgfsetroundjoin%
\definecolor{currentfill}{rgb}{0.000000,0.000000,1.000000}%
\pgfsetfillcolor{currentfill}%
\pgfsetlinewidth{1.003750pt}%
\definecolor{currentstroke}{rgb}{0.000000,0.000000,0.000000}%
\pgfsetstrokecolor{currentstroke}%
\pgfsetdash{}{0pt}%
\pgfpathmoveto{\pgfqpoint{3.847975in}{5.126948in}}%
\pgfpathcurveto{\pgfqpoint{3.856211in}{5.126948in}}{\pgfqpoint{3.864111in}{5.130220in}}{\pgfqpoint{3.869935in}{5.136044in}}%
\pgfpathcurveto{\pgfqpoint{3.875759in}{5.141868in}}{\pgfqpoint{3.879031in}{5.149768in}}{\pgfqpoint{3.879031in}{5.158005in}}%
\pgfpathcurveto{\pgfqpoint{3.879031in}{5.166241in}}{\pgfqpoint{3.875759in}{5.174141in}}{\pgfqpoint{3.869935in}{5.179965in}}%
\pgfpathcurveto{\pgfqpoint{3.864111in}{5.185789in}}{\pgfqpoint{3.856211in}{5.189061in}}{\pgfqpoint{3.847975in}{5.189061in}}%
\pgfpathcurveto{\pgfqpoint{3.839739in}{5.189061in}}{\pgfqpoint{3.831839in}{5.185789in}}{\pgfqpoint{3.826015in}{5.179965in}}%
\pgfpathcurveto{\pgfqpoint{3.820191in}{5.174141in}}{\pgfqpoint{3.816919in}{5.166241in}}{\pgfqpoint{3.816919in}{5.158005in}}%
\pgfpathcurveto{\pgfqpoint{3.816919in}{5.149768in}}{\pgfqpoint{3.820191in}{5.141868in}}{\pgfqpoint{3.826015in}{5.136044in}}%
\pgfpathcurveto{\pgfqpoint{3.831839in}{5.130220in}}{\pgfqpoint{3.839739in}{5.126948in}}{\pgfqpoint{3.847975in}{5.126948in}}%
\pgfpathclose%
\pgfusepath{stroke,fill}%
\end{pgfscope}%
\begin{pgfscope}%
\pgfpathrectangle{\pgfqpoint{0.894063in}{3.540000in}}{\pgfqpoint{6.713438in}{2.060556in}} %
\pgfusepath{clip}%
\pgfsetbuttcap%
\pgfsetroundjoin%
\definecolor{currentfill}{rgb}{0.000000,0.000000,1.000000}%
\pgfsetfillcolor{currentfill}%
\pgfsetlinewidth{1.003750pt}%
\definecolor{currentstroke}{rgb}{0.000000,0.000000,0.000000}%
\pgfsetstrokecolor{currentstroke}%
\pgfsetdash{}{0pt}%
\pgfpathmoveto{\pgfqpoint{7.607500in}{4.725355in}}%
\pgfpathcurveto{\pgfqpoint{7.615736in}{4.725355in}}{\pgfqpoint{7.623636in}{4.728628in}}{\pgfqpoint{7.629460in}{4.734452in}}%
\pgfpathcurveto{\pgfqpoint{7.635284in}{4.740276in}}{\pgfqpoint{7.638556in}{4.748176in}}{\pgfqpoint{7.638556in}{4.756412in}}%
\pgfpathcurveto{\pgfqpoint{7.638556in}{4.764648in}}{\pgfqpoint{7.635284in}{4.772548in}}{\pgfqpoint{7.629460in}{4.778372in}}%
\pgfpathcurveto{\pgfqpoint{7.623636in}{4.784196in}}{\pgfqpoint{7.615736in}{4.787468in}}{\pgfqpoint{7.607500in}{4.787468in}}%
\pgfpathcurveto{\pgfqpoint{7.599264in}{4.787468in}}{\pgfqpoint{7.591364in}{4.784196in}}{\pgfqpoint{7.585540in}{4.778372in}}%
\pgfpathcurveto{\pgfqpoint{7.579716in}{4.772548in}}{\pgfqpoint{7.576444in}{4.764648in}}{\pgfqpoint{7.576444in}{4.756412in}}%
\pgfpathcurveto{\pgfqpoint{7.576444in}{4.748176in}}{\pgfqpoint{7.579716in}{4.740276in}}{\pgfqpoint{7.585540in}{4.734452in}}%
\pgfpathcurveto{\pgfqpoint{7.591364in}{4.728628in}}{\pgfqpoint{7.599264in}{4.725355in}}{\pgfqpoint{7.607500in}{4.725355in}}%
\pgfpathclose%
\pgfusepath{stroke,fill}%
\end{pgfscope}%
\begin{pgfscope}%
\pgfpathrectangle{\pgfqpoint{0.894063in}{3.540000in}}{\pgfqpoint{6.713438in}{2.060556in}} %
\pgfusepath{clip}%
\pgfsetbuttcap%
\pgfsetroundjoin%
\definecolor{currentfill}{rgb}{0.000000,0.000000,1.000000}%
\pgfsetfillcolor{currentfill}%
\pgfsetlinewidth{1.003750pt}%
\definecolor{currentstroke}{rgb}{0.000000,0.000000,0.000000}%
\pgfsetstrokecolor{currentstroke}%
\pgfsetdash{}{0pt}%
\pgfpathmoveto{\pgfqpoint{4.385050in}{5.126938in}}%
\pgfpathcurveto{\pgfqpoint{4.393286in}{5.126938in}}{\pgfqpoint{4.401186in}{5.130211in}}{\pgfqpoint{4.407010in}{5.136035in}}%
\pgfpathcurveto{\pgfqpoint{4.412834in}{5.141859in}}{\pgfqpoint{4.416106in}{5.149759in}}{\pgfqpoint{4.416106in}{5.157995in}}%
\pgfpathcurveto{\pgfqpoint{4.416106in}{5.166231in}}{\pgfqpoint{4.412834in}{5.174131in}}{\pgfqpoint{4.407010in}{5.179955in}}%
\pgfpathcurveto{\pgfqpoint{4.401186in}{5.185779in}}{\pgfqpoint{4.393286in}{5.189051in}}{\pgfqpoint{4.385050in}{5.189051in}}%
\pgfpathcurveto{\pgfqpoint{4.376814in}{5.189051in}}{\pgfqpoint{4.368914in}{5.185779in}}{\pgfqpoint{4.363090in}{5.179955in}}%
\pgfpathcurveto{\pgfqpoint{4.357266in}{5.174131in}}{\pgfqpoint{4.353994in}{5.166231in}}{\pgfqpoint{4.353994in}{5.157995in}}%
\pgfpathcurveto{\pgfqpoint{4.353994in}{5.149759in}}{\pgfqpoint{4.357266in}{5.141859in}}{\pgfqpoint{4.363090in}{5.136035in}}%
\pgfpathcurveto{\pgfqpoint{4.368914in}{5.130211in}}{\pgfqpoint{4.376814in}{5.126938in}}{\pgfqpoint{4.385050in}{5.126938in}}%
\pgfpathclose%
\pgfusepath{stroke,fill}%
\end{pgfscope}%
\begin{pgfscope}%
\pgfpathrectangle{\pgfqpoint{0.894063in}{3.540000in}}{\pgfqpoint{6.713438in}{2.060556in}} %
\pgfusepath{clip}%
\pgfsetbuttcap%
\pgfsetroundjoin%
\definecolor{currentfill}{rgb}{0.000000,0.000000,1.000000}%
\pgfsetfillcolor{currentfill}%
\pgfsetlinewidth{1.003750pt}%
\definecolor{currentstroke}{rgb}{0.000000,0.000000,0.000000}%
\pgfsetstrokecolor{currentstroke}%
\pgfsetdash{}{0pt}%
\pgfpathmoveto{\pgfqpoint{6.533350in}{4.882561in}}%
\pgfpathcurveto{\pgfqpoint{6.541586in}{4.882561in}}{\pgfqpoint{6.549486in}{4.885833in}}{\pgfqpoint{6.555310in}{4.891657in}}%
\pgfpathcurveto{\pgfqpoint{6.561134in}{4.897481in}}{\pgfqpoint{6.564406in}{4.905381in}}{\pgfqpoint{6.564406in}{4.913617in}}%
\pgfpathcurveto{\pgfqpoint{6.564406in}{4.921853in}}{\pgfqpoint{6.561134in}{4.929753in}}{\pgfqpoint{6.555310in}{4.935577in}}%
\pgfpathcurveto{\pgfqpoint{6.549486in}{4.941401in}}{\pgfqpoint{6.541586in}{4.944674in}}{\pgfqpoint{6.533350in}{4.944674in}}%
\pgfpathcurveto{\pgfqpoint{6.525114in}{4.944674in}}{\pgfqpoint{6.517214in}{4.941401in}}{\pgfqpoint{6.511390in}{4.935577in}}%
\pgfpathcurveto{\pgfqpoint{6.505566in}{4.929753in}}{\pgfqpoint{6.502294in}{4.921853in}}{\pgfqpoint{6.502294in}{4.913617in}}%
\pgfpathcurveto{\pgfqpoint{6.502294in}{4.905381in}}{\pgfqpoint{6.505566in}{4.897481in}}{\pgfqpoint{6.511390in}{4.891657in}}%
\pgfpathcurveto{\pgfqpoint{6.517214in}{4.885833in}}{\pgfqpoint{6.525114in}{4.882561in}}{\pgfqpoint{6.533350in}{4.882561in}}%
\pgfpathclose%
\pgfusepath{stroke,fill}%
\end{pgfscope}%
\begin{pgfscope}%
\pgfpathrectangle{\pgfqpoint{0.894063in}{3.540000in}}{\pgfqpoint{6.713438in}{2.060556in}} %
\pgfusepath{clip}%
\pgfsetbuttcap%
\pgfsetroundjoin%
\definecolor{currentfill}{rgb}{0.000000,0.000000,1.000000}%
\pgfsetfillcolor{currentfill}%
\pgfsetlinewidth{1.003750pt}%
\definecolor{currentstroke}{rgb}{0.000000,0.000000,0.000000}%
\pgfsetstrokecolor{currentstroke}%
\pgfsetdash{}{0pt}%
\pgfpathmoveto{\pgfqpoint{1.296869in}{5.264663in}}%
\pgfpathcurveto{\pgfqpoint{1.305105in}{5.264663in}}{\pgfqpoint{1.313005in}{5.267936in}}{\pgfqpoint{1.318829in}{5.273759in}}%
\pgfpathcurveto{\pgfqpoint{1.324653in}{5.279583in}}{\pgfqpoint{1.327925in}{5.287483in}}{\pgfqpoint{1.327925in}{5.295720in}}%
\pgfpathcurveto{\pgfqpoint{1.327925in}{5.303956in}}{\pgfqpoint{1.324653in}{5.311856in}}{\pgfqpoint{1.318829in}{5.317680in}}%
\pgfpathcurveto{\pgfqpoint{1.313005in}{5.323504in}}{\pgfqpoint{1.305105in}{5.326776in}}{\pgfqpoint{1.296869in}{5.326776in}}%
\pgfpathcurveto{\pgfqpoint{1.288632in}{5.326776in}}{\pgfqpoint{1.280732in}{5.323504in}}{\pgfqpoint{1.274908in}{5.317680in}}%
\pgfpathcurveto{\pgfqpoint{1.269085in}{5.311856in}}{\pgfqpoint{1.265812in}{5.303956in}}{\pgfqpoint{1.265812in}{5.295720in}}%
\pgfpathcurveto{\pgfqpoint{1.265812in}{5.287483in}}{\pgfqpoint{1.269085in}{5.279583in}}{\pgfqpoint{1.274908in}{5.273759in}}%
\pgfpathcurveto{\pgfqpoint{1.280732in}{5.267936in}}{\pgfqpoint{1.288632in}{5.264663in}}{\pgfqpoint{1.296869in}{5.264663in}}%
\pgfpathclose%
\pgfusepath{stroke,fill}%
\end{pgfscope}%
\begin{pgfscope}%
\pgfpathrectangle{\pgfqpoint{0.894063in}{3.540000in}}{\pgfqpoint{6.713438in}{2.060556in}} %
\pgfusepath{clip}%
\pgfsetbuttcap%
\pgfsetroundjoin%
\definecolor{currentfill}{rgb}{0.000000,0.000000,1.000000}%
\pgfsetfillcolor{currentfill}%
\pgfsetlinewidth{1.003750pt}%
\definecolor{currentstroke}{rgb}{0.000000,0.000000,0.000000}%
\pgfsetstrokecolor{currentstroke}%
\pgfsetdash{}{0pt}%
\pgfpathmoveto{\pgfqpoint{4.519319in}{5.126938in}}%
\pgfpathcurveto{\pgfqpoint{4.527555in}{5.126938in}}{\pgfqpoint{4.535455in}{5.130211in}}{\pgfqpoint{4.541279in}{5.136035in}}%
\pgfpathcurveto{\pgfqpoint{4.547103in}{5.141859in}}{\pgfqpoint{4.550375in}{5.149759in}}{\pgfqpoint{4.550375in}{5.157995in}}%
\pgfpathcurveto{\pgfqpoint{4.550375in}{5.166231in}}{\pgfqpoint{4.547103in}{5.174131in}}{\pgfqpoint{4.541279in}{5.179955in}}%
\pgfpathcurveto{\pgfqpoint{4.535455in}{5.185779in}}{\pgfqpoint{4.527555in}{5.189051in}}{\pgfqpoint{4.519319in}{5.189051in}}%
\pgfpathcurveto{\pgfqpoint{4.511082in}{5.189051in}}{\pgfqpoint{4.503182in}{5.185779in}}{\pgfqpoint{4.497358in}{5.179955in}}%
\pgfpathcurveto{\pgfqpoint{4.491535in}{5.174131in}}{\pgfqpoint{4.488262in}{5.166231in}}{\pgfqpoint{4.488262in}{5.157995in}}%
\pgfpathcurveto{\pgfqpoint{4.488262in}{5.149759in}}{\pgfqpoint{4.491535in}{5.141859in}}{\pgfqpoint{4.497358in}{5.136035in}}%
\pgfpathcurveto{\pgfqpoint{4.503182in}{5.130211in}}{\pgfqpoint{4.511082in}{5.126938in}}{\pgfqpoint{4.519319in}{5.126938in}}%
\pgfpathclose%
\pgfusepath{stroke,fill}%
\end{pgfscope}%
\begin{pgfscope}%
\pgfpathrectangle{\pgfqpoint{0.894063in}{3.540000in}}{\pgfqpoint{6.713438in}{2.060556in}} %
\pgfusepath{clip}%
\pgfsetbuttcap%
\pgfsetroundjoin%
\definecolor{currentfill}{rgb}{0.000000,0.000000,1.000000}%
\pgfsetfillcolor{currentfill}%
\pgfsetlinewidth{1.003750pt}%
\definecolor{currentstroke}{rgb}{0.000000,0.000000,0.000000}%
\pgfsetstrokecolor{currentstroke}%
\pgfsetdash{}{0pt}%
\pgfpathmoveto{\pgfqpoint{2.505288in}{5.127014in}}%
\pgfpathcurveto{\pgfqpoint{2.513524in}{5.127014in}}{\pgfqpoint{2.521424in}{5.130286in}}{\pgfqpoint{2.527248in}{5.136110in}}%
\pgfpathcurveto{\pgfqpoint{2.533072in}{5.141934in}}{\pgfqpoint{2.536344in}{5.149834in}}{\pgfqpoint{2.536344in}{5.158070in}}%
\pgfpathcurveto{\pgfqpoint{2.536344in}{5.166307in}}{\pgfqpoint{2.533072in}{5.174207in}}{\pgfqpoint{2.527248in}{5.180031in}}%
\pgfpathcurveto{\pgfqpoint{2.521424in}{5.185855in}}{\pgfqpoint{2.513524in}{5.189127in}}{\pgfqpoint{2.505288in}{5.189127in}}%
\pgfpathcurveto{\pgfqpoint{2.497051in}{5.189127in}}{\pgfqpoint{2.489151in}{5.185855in}}{\pgfqpoint{2.483327in}{5.180031in}}%
\pgfpathcurveto{\pgfqpoint{2.477503in}{5.174207in}}{\pgfqpoint{2.474231in}{5.166307in}}{\pgfqpoint{2.474231in}{5.158070in}}%
\pgfpathcurveto{\pgfqpoint{2.474231in}{5.149834in}}{\pgfqpoint{2.477503in}{5.141934in}}{\pgfqpoint{2.483327in}{5.136110in}}%
\pgfpathcurveto{\pgfqpoint{2.489151in}{5.130286in}}{\pgfqpoint{2.497051in}{5.127014in}}{\pgfqpoint{2.505288in}{5.127014in}}%
\pgfpathclose%
\pgfusepath{stroke,fill}%
\end{pgfscope}%
\begin{pgfscope}%
\pgfpathrectangle{\pgfqpoint{0.894063in}{3.540000in}}{\pgfqpoint{6.713438in}{2.060556in}} %
\pgfusepath{clip}%
\pgfsetbuttcap%
\pgfsetroundjoin%
\definecolor{currentfill}{rgb}{0.000000,0.000000,1.000000}%
\pgfsetfillcolor{currentfill}%
\pgfsetlinewidth{1.003750pt}%
\definecolor{currentstroke}{rgb}{0.000000,0.000000,0.000000}%
\pgfsetstrokecolor{currentstroke}%
\pgfsetdash{}{0pt}%
\pgfpathmoveto{\pgfqpoint{5.459200in}{5.048854in}}%
\pgfpathcurveto{\pgfqpoint{5.467436in}{5.048854in}}{\pgfqpoint{5.475336in}{5.052127in}}{\pgfqpoint{5.481160in}{5.057951in}}%
\pgfpathcurveto{\pgfqpoint{5.486984in}{5.063775in}}{\pgfqpoint{5.490256in}{5.071675in}}{\pgfqpoint{5.490256in}{5.079911in}}%
\pgfpathcurveto{\pgfqpoint{5.490256in}{5.088147in}}{\pgfqpoint{5.486984in}{5.096047in}}{\pgfqpoint{5.481160in}{5.101871in}}%
\pgfpathcurveto{\pgfqpoint{5.475336in}{5.107695in}}{\pgfqpoint{5.467436in}{5.110967in}}{\pgfqpoint{5.459200in}{5.110967in}}%
\pgfpathcurveto{\pgfqpoint{5.450964in}{5.110967in}}{\pgfqpoint{5.443064in}{5.107695in}}{\pgfqpoint{5.437240in}{5.101871in}}%
\pgfpathcurveto{\pgfqpoint{5.431416in}{5.096047in}}{\pgfqpoint{5.428144in}{5.088147in}}{\pgfqpoint{5.428144in}{5.079911in}}%
\pgfpathcurveto{\pgfqpoint{5.428144in}{5.071675in}}{\pgfqpoint{5.431416in}{5.063775in}}{\pgfqpoint{5.437240in}{5.057951in}}%
\pgfpathcurveto{\pgfqpoint{5.443064in}{5.052127in}}{\pgfqpoint{5.450964in}{5.048854in}}{\pgfqpoint{5.459200in}{5.048854in}}%
\pgfpathclose%
\pgfusepath{stroke,fill}%
\end{pgfscope}%
\begin{pgfscope}%
\pgfpathrectangle{\pgfqpoint{0.894063in}{3.540000in}}{\pgfqpoint{6.713438in}{2.060556in}} %
\pgfusepath{clip}%
\pgfsetbuttcap%
\pgfsetroundjoin%
\definecolor{currentfill}{rgb}{0.000000,0.000000,1.000000}%
\pgfsetfillcolor{currentfill}%
\pgfsetlinewidth{1.003750pt}%
\definecolor{currentstroke}{rgb}{0.000000,0.000000,0.000000}%
\pgfsetstrokecolor{currentstroke}%
\pgfsetdash{}{0pt}%
\pgfpathmoveto{\pgfqpoint{6.936156in}{4.824123in}}%
\pgfpathcurveto{\pgfqpoint{6.944393in}{4.824123in}}{\pgfqpoint{6.952293in}{4.827396in}}{\pgfqpoint{6.958117in}{4.833220in}}%
\pgfpathcurveto{\pgfqpoint{6.963940in}{4.839043in}}{\pgfqpoint{6.967213in}{4.846944in}}{\pgfqpoint{6.967213in}{4.855180in}}%
\pgfpathcurveto{\pgfqpoint{6.967213in}{4.863416in}}{\pgfqpoint{6.963940in}{4.871316in}}{\pgfqpoint{6.958117in}{4.877140in}}%
\pgfpathcurveto{\pgfqpoint{6.952293in}{4.882964in}}{\pgfqpoint{6.944393in}{4.886236in}}{\pgfqpoint{6.936156in}{4.886236in}}%
\pgfpathcurveto{\pgfqpoint{6.927920in}{4.886236in}}{\pgfqpoint{6.920020in}{4.882964in}}{\pgfqpoint{6.914196in}{4.877140in}}%
\pgfpathcurveto{\pgfqpoint{6.908372in}{4.871316in}}{\pgfqpoint{6.905100in}{4.863416in}}{\pgfqpoint{6.905100in}{4.855180in}}%
\pgfpathcurveto{\pgfqpoint{6.905100in}{4.846944in}}{\pgfqpoint{6.908372in}{4.839043in}}{\pgfqpoint{6.914196in}{4.833220in}}%
\pgfpathcurveto{\pgfqpoint{6.920020in}{4.827396in}}{\pgfqpoint{6.927920in}{4.824123in}}{\pgfqpoint{6.936156in}{4.824123in}}%
\pgfpathclose%
\pgfusepath{stroke,fill}%
\end{pgfscope}%
\begin{pgfscope}%
\pgfpathrectangle{\pgfqpoint{0.894063in}{3.540000in}}{\pgfqpoint{6.713438in}{2.060556in}} %
\pgfusepath{clip}%
\pgfsetbuttcap%
\pgfsetroundjoin%
\definecolor{currentfill}{rgb}{0.000000,0.000000,1.000000}%
\pgfsetfillcolor{currentfill}%
\pgfsetlinewidth{1.003750pt}%
\definecolor{currentstroke}{rgb}{0.000000,0.000000,0.000000}%
\pgfsetstrokecolor{currentstroke}%
\pgfsetdash{}{0pt}%
\pgfpathmoveto{\pgfqpoint{5.862006in}{4.978078in}}%
\pgfpathcurveto{\pgfqpoint{5.870243in}{4.978078in}}{\pgfqpoint{5.878143in}{4.981351in}}{\pgfqpoint{5.883967in}{4.987175in}}%
\pgfpathcurveto{\pgfqpoint{5.889790in}{4.992999in}}{\pgfqpoint{5.893063in}{5.000899in}}{\pgfqpoint{5.893063in}{5.009135in}}%
\pgfpathcurveto{\pgfqpoint{5.893063in}{5.017371in}}{\pgfqpoint{5.889790in}{5.025271in}}{\pgfqpoint{5.883967in}{5.031095in}}%
\pgfpathcurveto{\pgfqpoint{5.878143in}{5.036919in}}{\pgfqpoint{5.870243in}{5.040191in}}{\pgfqpoint{5.862006in}{5.040191in}}%
\pgfpathcurveto{\pgfqpoint{5.853770in}{5.040191in}}{\pgfqpoint{5.845870in}{5.036919in}}{\pgfqpoint{5.840046in}{5.031095in}}%
\pgfpathcurveto{\pgfqpoint{5.834222in}{5.025271in}}{\pgfqpoint{5.830950in}{5.017371in}}{\pgfqpoint{5.830950in}{5.009135in}}%
\pgfpathcurveto{\pgfqpoint{5.830950in}{5.000899in}}{\pgfqpoint{5.834222in}{4.992999in}}{\pgfqpoint{5.840046in}{4.987175in}}%
\pgfpathcurveto{\pgfqpoint{5.845870in}{4.981351in}}{\pgfqpoint{5.853770in}{4.978078in}}{\pgfqpoint{5.862006in}{4.978078in}}%
\pgfpathclose%
\pgfusepath{stroke,fill}%
\end{pgfscope}%
\begin{pgfscope}%
\pgfpathrectangle{\pgfqpoint{0.894063in}{3.540000in}}{\pgfqpoint{6.713438in}{2.060556in}} %
\pgfusepath{clip}%
\pgfsetbuttcap%
\pgfsetroundjoin%
\definecolor{currentfill}{rgb}{0.000000,0.000000,1.000000}%
\pgfsetfillcolor{currentfill}%
\pgfsetlinewidth{1.003750pt}%
\definecolor{currentstroke}{rgb}{0.000000,0.000000,0.000000}%
\pgfsetstrokecolor{currentstroke}%
\pgfsetdash{}{0pt}%
\pgfpathmoveto{\pgfqpoint{7.070425in}{4.804919in}}%
\pgfpathcurveto{\pgfqpoint{7.078661in}{4.804919in}}{\pgfqpoint{7.086561in}{4.808191in}}{\pgfqpoint{7.092385in}{4.814015in}}%
\pgfpathcurveto{\pgfqpoint{7.098209in}{4.819839in}}{\pgfqpoint{7.101481in}{4.827739in}}{\pgfqpoint{7.101481in}{4.835975in}}%
\pgfpathcurveto{\pgfqpoint{7.101481in}{4.844212in}}{\pgfqpoint{7.098209in}{4.852112in}}{\pgfqpoint{7.092385in}{4.857936in}}%
\pgfpathcurveto{\pgfqpoint{7.086561in}{4.863760in}}{\pgfqpoint{7.078661in}{4.867032in}}{\pgfqpoint{7.070425in}{4.867032in}}%
\pgfpathcurveto{\pgfqpoint{7.062189in}{4.867032in}}{\pgfqpoint{7.054289in}{4.863760in}}{\pgfqpoint{7.048465in}{4.857936in}}%
\pgfpathcurveto{\pgfqpoint{7.042641in}{4.852112in}}{\pgfqpoint{7.039369in}{4.844212in}}{\pgfqpoint{7.039369in}{4.835975in}}%
\pgfpathcurveto{\pgfqpoint{7.039369in}{4.827739in}}{\pgfqpoint{7.042641in}{4.819839in}}{\pgfqpoint{7.048465in}{4.814015in}}%
\pgfpathcurveto{\pgfqpoint{7.054289in}{4.808191in}}{\pgfqpoint{7.062189in}{4.804919in}}{\pgfqpoint{7.070425in}{4.804919in}}%
\pgfpathclose%
\pgfusepath{stroke,fill}%
\end{pgfscope}%
\begin{pgfscope}%
\pgfpathrectangle{\pgfqpoint{0.894063in}{3.540000in}}{\pgfqpoint{6.713438in}{2.060556in}} %
\pgfusepath{clip}%
\pgfsetbuttcap%
\pgfsetroundjoin%
\definecolor{currentfill}{rgb}{0.000000,0.000000,1.000000}%
\pgfsetfillcolor{currentfill}%
\pgfsetlinewidth{1.003750pt}%
\definecolor{currentstroke}{rgb}{0.000000,0.000000,0.000000}%
\pgfsetstrokecolor{currentstroke}%
\pgfsetdash{}{0pt}%
\pgfpathmoveto{\pgfqpoint{3.176631in}{5.126967in}}%
\pgfpathcurveto{\pgfqpoint{3.184868in}{5.126967in}}{\pgfqpoint{3.192768in}{5.130240in}}{\pgfqpoint{3.198592in}{5.136064in}}%
\pgfpathcurveto{\pgfqpoint{3.204415in}{5.141887in}}{\pgfqpoint{3.207688in}{5.149787in}}{\pgfqpoint{3.207688in}{5.158024in}}%
\pgfpathcurveto{\pgfqpoint{3.207688in}{5.166260in}}{\pgfqpoint{3.204415in}{5.174160in}}{\pgfqpoint{3.198592in}{5.179984in}}%
\pgfpathcurveto{\pgfqpoint{3.192768in}{5.185808in}}{\pgfqpoint{3.184868in}{5.189080in}}{\pgfqpoint{3.176631in}{5.189080in}}%
\pgfpathcurveto{\pgfqpoint{3.168395in}{5.189080in}}{\pgfqpoint{3.160495in}{5.185808in}}{\pgfqpoint{3.154671in}{5.179984in}}%
\pgfpathcurveto{\pgfqpoint{3.148847in}{5.174160in}}{\pgfqpoint{3.145575in}{5.166260in}}{\pgfqpoint{3.145575in}{5.158024in}}%
\pgfpathcurveto{\pgfqpoint{3.145575in}{5.149787in}}{\pgfqpoint{3.148847in}{5.141887in}}{\pgfqpoint{3.154671in}{5.136064in}}%
\pgfpathcurveto{\pgfqpoint{3.160495in}{5.130240in}}{\pgfqpoint{3.168395in}{5.126967in}}{\pgfqpoint{3.176631in}{5.126967in}}%
\pgfpathclose%
\pgfusepath{stroke,fill}%
\end{pgfscope}%
\begin{pgfscope}%
\pgfpathrectangle{\pgfqpoint{0.894063in}{3.540000in}}{\pgfqpoint{6.713438in}{2.060556in}} %
\pgfusepath{clip}%
\pgfsetbuttcap%
\pgfsetroundjoin%
\definecolor{currentfill}{rgb}{0.000000,0.000000,1.000000}%
\pgfsetfillcolor{currentfill}%
\pgfsetlinewidth{1.003750pt}%
\definecolor{currentstroke}{rgb}{0.000000,0.000000,0.000000}%
\pgfsetstrokecolor{currentstroke}%
\pgfsetdash{}{0pt}%
\pgfpathmoveto{\pgfqpoint{2.102481in}{5.129417in}}%
\pgfpathcurveto{\pgfqpoint{2.110718in}{5.129417in}}{\pgfqpoint{2.118618in}{5.132689in}}{\pgfqpoint{2.124442in}{5.138513in}}%
\pgfpathcurveto{\pgfqpoint{2.130265in}{5.144337in}}{\pgfqpoint{2.133538in}{5.152237in}}{\pgfqpoint{2.133538in}{5.160473in}}%
\pgfpathcurveto{\pgfqpoint{2.133538in}{5.168709in}}{\pgfqpoint{2.130265in}{5.176609in}}{\pgfqpoint{2.124442in}{5.182433in}}%
\pgfpathcurveto{\pgfqpoint{2.118618in}{5.188257in}}{\pgfqpoint{2.110718in}{5.191530in}}{\pgfqpoint{2.102481in}{5.191530in}}%
\pgfpathcurveto{\pgfqpoint{2.094245in}{5.191530in}}{\pgfqpoint{2.086345in}{5.188257in}}{\pgfqpoint{2.080521in}{5.182433in}}%
\pgfpathcurveto{\pgfqpoint{2.074697in}{5.176609in}}{\pgfqpoint{2.071425in}{5.168709in}}{\pgfqpoint{2.071425in}{5.160473in}}%
\pgfpathcurveto{\pgfqpoint{2.071425in}{5.152237in}}{\pgfqpoint{2.074697in}{5.144337in}}{\pgfqpoint{2.080521in}{5.138513in}}%
\pgfpathcurveto{\pgfqpoint{2.086345in}{5.132689in}}{\pgfqpoint{2.094245in}{5.129417in}}{\pgfqpoint{2.102481in}{5.129417in}}%
\pgfpathclose%
\pgfusepath{stroke,fill}%
\end{pgfscope}%
\begin{pgfscope}%
\pgfpathrectangle{\pgfqpoint{0.894063in}{3.540000in}}{\pgfqpoint{6.713438in}{2.060556in}} %
\pgfusepath{clip}%
\pgfsetbuttcap%
\pgfsetroundjoin%
\definecolor{currentfill}{rgb}{0.000000,0.000000,1.000000}%
\pgfsetfillcolor{currentfill}%
\pgfsetlinewidth{1.003750pt}%
\definecolor{currentstroke}{rgb}{0.000000,0.000000,0.000000}%
\pgfsetstrokecolor{currentstroke}%
\pgfsetdash{}{0pt}%
\pgfpathmoveto{\pgfqpoint{1.968213in}{5.130797in}}%
\pgfpathcurveto{\pgfqpoint{1.976449in}{5.130797in}}{\pgfqpoint{1.984349in}{5.134069in}}{\pgfqpoint{1.990173in}{5.139893in}}%
\pgfpathcurveto{\pgfqpoint{1.995997in}{5.145717in}}{\pgfqpoint{1.999269in}{5.153617in}}{\pgfqpoint{1.999269in}{5.161854in}}%
\pgfpathcurveto{\pgfqpoint{1.999269in}{5.170090in}}{\pgfqpoint{1.995997in}{5.177990in}}{\pgfqpoint{1.990173in}{5.183814in}}%
\pgfpathcurveto{\pgfqpoint{1.984349in}{5.189638in}}{\pgfqpoint{1.976449in}{5.192910in}}{\pgfqpoint{1.968213in}{5.192910in}}%
\pgfpathcurveto{\pgfqpoint{1.959976in}{5.192910in}}{\pgfqpoint{1.952076in}{5.189638in}}{\pgfqpoint{1.946252in}{5.183814in}}%
\pgfpathcurveto{\pgfqpoint{1.940428in}{5.177990in}}{\pgfqpoint{1.937156in}{5.170090in}}{\pgfqpoint{1.937156in}{5.161854in}}%
\pgfpathcurveto{\pgfqpoint{1.937156in}{5.153617in}}{\pgfqpoint{1.940428in}{5.145717in}}{\pgfqpoint{1.946252in}{5.139893in}}%
\pgfpathcurveto{\pgfqpoint{1.952076in}{5.134069in}}{\pgfqpoint{1.959976in}{5.130797in}}{\pgfqpoint{1.968213in}{5.130797in}}%
\pgfpathclose%
\pgfusepath{stroke,fill}%
\end{pgfscope}%
\begin{pgfscope}%
\pgfpathrectangle{\pgfqpoint{0.894063in}{3.540000in}}{\pgfqpoint{6.713438in}{2.060556in}} %
\pgfusepath{clip}%
\pgfsetbuttcap%
\pgfsetroundjoin%
\definecolor{currentfill}{rgb}{0.000000,0.000000,1.000000}%
\pgfsetfillcolor{currentfill}%
\pgfsetlinewidth{1.003750pt}%
\definecolor{currentstroke}{rgb}{0.000000,0.000000,0.000000}%
\pgfsetstrokecolor{currentstroke}%
\pgfsetdash{}{0pt}%
\pgfpathmoveto{\pgfqpoint{3.310900in}{5.126965in}}%
\pgfpathcurveto{\pgfqpoint{3.319136in}{5.126965in}}{\pgfqpoint{3.327036in}{5.130237in}}{\pgfqpoint{3.332860in}{5.136061in}}%
\pgfpathcurveto{\pgfqpoint{3.338684in}{5.141885in}}{\pgfqpoint{3.341956in}{5.149785in}}{\pgfqpoint{3.341956in}{5.158021in}}%
\pgfpathcurveto{\pgfqpoint{3.341956in}{5.166257in}}{\pgfqpoint{3.338684in}{5.174157in}}{\pgfqpoint{3.332860in}{5.179981in}}%
\pgfpathcurveto{\pgfqpoint{3.327036in}{5.185805in}}{\pgfqpoint{3.319136in}{5.189078in}}{\pgfqpoint{3.310900in}{5.189078in}}%
\pgfpathcurveto{\pgfqpoint{3.302664in}{5.189078in}}{\pgfqpoint{3.294764in}{5.185805in}}{\pgfqpoint{3.288940in}{5.179981in}}%
\pgfpathcurveto{\pgfqpoint{3.283116in}{5.174157in}}{\pgfqpoint{3.279844in}{5.166257in}}{\pgfqpoint{3.279844in}{5.158021in}}%
\pgfpathcurveto{\pgfqpoint{3.279844in}{5.149785in}}{\pgfqpoint{3.283116in}{5.141885in}}{\pgfqpoint{3.288940in}{5.136061in}}%
\pgfpathcurveto{\pgfqpoint{3.294764in}{5.130237in}}{\pgfqpoint{3.302664in}{5.126965in}}{\pgfqpoint{3.310900in}{5.126965in}}%
\pgfpathclose%
\pgfusepath{stroke,fill}%
\end{pgfscope}%
\begin{pgfscope}%
\pgfpathrectangle{\pgfqpoint{0.894063in}{3.540000in}}{\pgfqpoint{6.713438in}{2.060556in}} %
\pgfusepath{clip}%
\pgfsetbuttcap%
\pgfsetroundjoin%
\definecolor{currentfill}{rgb}{0.000000,0.000000,1.000000}%
\pgfsetfillcolor{currentfill}%
\pgfsetlinewidth{1.003750pt}%
\definecolor{currentstroke}{rgb}{0.000000,0.000000,0.000000}%
\pgfsetstrokecolor{currentstroke}%
\pgfsetdash{}{0pt}%
\pgfpathmoveto{\pgfqpoint{5.593469in}{5.016108in}}%
\pgfpathcurveto{\pgfqpoint{5.601705in}{5.016108in}}{\pgfqpoint{5.609605in}{5.019380in}}{\pgfqpoint{5.615429in}{5.025204in}}%
\pgfpathcurveto{\pgfqpoint{5.621253in}{5.031028in}}{\pgfqpoint{5.624525in}{5.038928in}}{\pgfqpoint{5.624525in}{5.047165in}}%
\pgfpathcurveto{\pgfqpoint{5.624525in}{5.055401in}}{\pgfqpoint{5.621253in}{5.063301in}}{\pgfqpoint{5.615429in}{5.069125in}}%
\pgfpathcurveto{\pgfqpoint{5.609605in}{5.074949in}}{\pgfqpoint{5.601705in}{5.078221in}}{\pgfqpoint{5.593469in}{5.078221in}}%
\pgfpathcurveto{\pgfqpoint{5.585232in}{5.078221in}}{\pgfqpoint{5.577332in}{5.074949in}}{\pgfqpoint{5.571508in}{5.069125in}}%
\pgfpathcurveto{\pgfqpoint{5.565685in}{5.063301in}}{\pgfqpoint{5.562412in}{5.055401in}}{\pgfqpoint{5.562412in}{5.047165in}}%
\pgfpathcurveto{\pgfqpoint{5.562412in}{5.038928in}}{\pgfqpoint{5.565685in}{5.031028in}}{\pgfqpoint{5.571508in}{5.025204in}}%
\pgfpathcurveto{\pgfqpoint{5.577332in}{5.019380in}}{\pgfqpoint{5.585232in}{5.016108in}}{\pgfqpoint{5.593469in}{5.016108in}}%
\pgfpathclose%
\pgfusepath{stroke,fill}%
\end{pgfscope}%
\begin{pgfscope}%
\pgfpathrectangle{\pgfqpoint{0.894063in}{3.540000in}}{\pgfqpoint{6.713438in}{2.060556in}} %
\pgfusepath{clip}%
\pgfsetbuttcap%
\pgfsetroundjoin%
\definecolor{currentfill}{rgb}{0.000000,0.000000,1.000000}%
\pgfsetfillcolor{currentfill}%
\pgfsetlinewidth{1.003750pt}%
\definecolor{currentstroke}{rgb}{0.000000,0.000000,0.000000}%
\pgfsetstrokecolor{currentstroke}%
\pgfsetdash{}{0pt}%
\pgfpathmoveto{\pgfqpoint{3.042363in}{5.126971in}}%
\pgfpathcurveto{\pgfqpoint{3.050599in}{5.126971in}}{\pgfqpoint{3.058499in}{5.130244in}}{\pgfqpoint{3.064323in}{5.136068in}}%
\pgfpathcurveto{\pgfqpoint{3.070147in}{5.141892in}}{\pgfqpoint{3.073419in}{5.149792in}}{\pgfqpoint{3.073419in}{5.158028in}}%
\pgfpathcurveto{\pgfqpoint{3.073419in}{5.166264in}}{\pgfqpoint{3.070147in}{5.174164in}}{\pgfqpoint{3.064323in}{5.179988in}}%
\pgfpathcurveto{\pgfqpoint{3.058499in}{5.185812in}}{\pgfqpoint{3.050599in}{5.189084in}}{\pgfqpoint{3.042363in}{5.189084in}}%
\pgfpathcurveto{\pgfqpoint{3.034126in}{5.189084in}}{\pgfqpoint{3.026226in}{5.185812in}}{\pgfqpoint{3.020402in}{5.179988in}}%
\pgfpathcurveto{\pgfqpoint{3.014578in}{5.174164in}}{\pgfqpoint{3.011306in}{5.166264in}}{\pgfqpoint{3.011306in}{5.158028in}}%
\pgfpathcurveto{\pgfqpoint{3.011306in}{5.149792in}}{\pgfqpoint{3.014578in}{5.141892in}}{\pgfqpoint{3.020402in}{5.136068in}}%
\pgfpathcurveto{\pgfqpoint{3.026226in}{5.130244in}}{\pgfqpoint{3.034126in}{5.126971in}}{\pgfqpoint{3.042363in}{5.126971in}}%
\pgfpathclose%
\pgfusepath{stroke,fill}%
\end{pgfscope}%
\begin{pgfscope}%
\pgfpathrectangle{\pgfqpoint{0.894063in}{3.540000in}}{\pgfqpoint{6.713438in}{2.060556in}} %
\pgfusepath{clip}%
\pgfsetbuttcap%
\pgfsetroundjoin%
\definecolor{currentfill}{rgb}{0.000000,0.000000,1.000000}%
\pgfsetfillcolor{currentfill}%
\pgfsetlinewidth{1.003750pt}%
\definecolor{currentstroke}{rgb}{0.000000,0.000000,0.000000}%
\pgfsetstrokecolor{currentstroke}%
\pgfsetdash{}{0pt}%
\pgfpathmoveto{\pgfqpoint{5.190663in}{5.075996in}}%
\pgfpathcurveto{\pgfqpoint{5.198899in}{5.075996in}}{\pgfqpoint{5.206799in}{5.079268in}}{\pgfqpoint{5.212623in}{5.085092in}}%
\pgfpathcurveto{\pgfqpoint{5.218447in}{5.090916in}}{\pgfqpoint{5.221719in}{5.098816in}}{\pgfqpoint{5.221719in}{5.107052in}}%
\pgfpathcurveto{\pgfqpoint{5.221719in}{5.115289in}}{\pgfqpoint{5.218447in}{5.123189in}}{\pgfqpoint{5.212623in}{5.129013in}}%
\pgfpathcurveto{\pgfqpoint{5.206799in}{5.134837in}}{\pgfqpoint{5.198899in}{5.138109in}}{\pgfqpoint{5.190663in}{5.138109in}}%
\pgfpathcurveto{\pgfqpoint{5.182426in}{5.138109in}}{\pgfqpoint{5.174526in}{5.134837in}}{\pgfqpoint{5.168702in}{5.129013in}}%
\pgfpathcurveto{\pgfqpoint{5.162878in}{5.123189in}}{\pgfqpoint{5.159606in}{5.115289in}}{\pgfqpoint{5.159606in}{5.107052in}}%
\pgfpathcurveto{\pgfqpoint{5.159606in}{5.098816in}}{\pgfqpoint{5.162878in}{5.090916in}}{\pgfqpoint{5.168702in}{5.085092in}}%
\pgfpathcurveto{\pgfqpoint{5.174526in}{5.079268in}}{\pgfqpoint{5.182426in}{5.075996in}}{\pgfqpoint{5.190663in}{5.075996in}}%
\pgfpathclose%
\pgfusepath{stroke,fill}%
\end{pgfscope}%
\begin{pgfscope}%
\pgfpathrectangle{\pgfqpoint{0.894063in}{3.540000in}}{\pgfqpoint{6.713438in}{2.060556in}} %
\pgfusepath{clip}%
\pgfsetbuttcap%
\pgfsetroundjoin%
\definecolor{currentfill}{rgb}{0.000000,0.000000,1.000000}%
\pgfsetfillcolor{currentfill}%
\pgfsetlinewidth{1.003750pt}%
\definecolor{currentstroke}{rgb}{0.000000,0.000000,0.000000}%
\pgfsetstrokecolor{currentstroke}%
\pgfsetdash{}{0pt}%
\pgfpathmoveto{\pgfqpoint{6.801888in}{4.842924in}}%
\pgfpathcurveto{\pgfqpoint{6.810124in}{4.842924in}}{\pgfqpoint{6.818024in}{4.846196in}}{\pgfqpoint{6.823848in}{4.852020in}}%
\pgfpathcurveto{\pgfqpoint{6.829672in}{4.857844in}}{\pgfqpoint{6.832944in}{4.865744in}}{\pgfqpoint{6.832944in}{4.873980in}}%
\pgfpathcurveto{\pgfqpoint{6.832944in}{4.882217in}}{\pgfqpoint{6.829672in}{4.890117in}}{\pgfqpoint{6.823848in}{4.895941in}}%
\pgfpathcurveto{\pgfqpoint{6.818024in}{4.901765in}}{\pgfqpoint{6.810124in}{4.905037in}}{\pgfqpoint{6.801888in}{4.905037in}}%
\pgfpathcurveto{\pgfqpoint{6.793651in}{4.905037in}}{\pgfqpoint{6.785751in}{4.901765in}}{\pgfqpoint{6.779927in}{4.895941in}}%
\pgfpathcurveto{\pgfqpoint{6.774103in}{4.890117in}}{\pgfqpoint{6.770831in}{4.882217in}}{\pgfqpoint{6.770831in}{4.873980in}}%
\pgfpathcurveto{\pgfqpoint{6.770831in}{4.865744in}}{\pgfqpoint{6.774103in}{4.857844in}}{\pgfqpoint{6.779927in}{4.852020in}}%
\pgfpathcurveto{\pgfqpoint{6.785751in}{4.846196in}}{\pgfqpoint{6.793651in}{4.842924in}}{\pgfqpoint{6.801888in}{4.842924in}}%
\pgfpathclose%
\pgfusepath{stroke,fill}%
\end{pgfscope}%
\begin{pgfscope}%
\pgfpathrectangle{\pgfqpoint{0.894063in}{3.540000in}}{\pgfqpoint{6.713438in}{2.060556in}} %
\pgfusepath{clip}%
\pgfsetbuttcap%
\pgfsetroundjoin%
\definecolor{currentfill}{rgb}{0.000000,0.000000,1.000000}%
\pgfsetfillcolor{currentfill}%
\pgfsetlinewidth{1.003750pt}%
\definecolor{currentstroke}{rgb}{0.000000,0.000000,0.000000}%
\pgfsetstrokecolor{currentstroke}%
\pgfsetdash{}{0pt}%
\pgfpathmoveto{\pgfqpoint{3.579438in}{5.126955in}}%
\pgfpathcurveto{\pgfqpoint{3.587674in}{5.126955in}}{\pgfqpoint{3.595574in}{5.130227in}}{\pgfqpoint{3.601398in}{5.136051in}}%
\pgfpathcurveto{\pgfqpoint{3.607222in}{5.141875in}}{\pgfqpoint{3.610494in}{5.149775in}}{\pgfqpoint{3.610494in}{5.158011in}}%
\pgfpathcurveto{\pgfqpoint{3.610494in}{5.166248in}}{\pgfqpoint{3.607222in}{5.174148in}}{\pgfqpoint{3.601398in}{5.179972in}}%
\pgfpathcurveto{\pgfqpoint{3.595574in}{5.185796in}}{\pgfqpoint{3.587674in}{5.189068in}}{\pgfqpoint{3.579438in}{5.189068in}}%
\pgfpathcurveto{\pgfqpoint{3.571201in}{5.189068in}}{\pgfqpoint{3.563301in}{5.185796in}}{\pgfqpoint{3.557477in}{5.179972in}}%
\pgfpathcurveto{\pgfqpoint{3.551653in}{5.174148in}}{\pgfqpoint{3.548381in}{5.166248in}}{\pgfqpoint{3.548381in}{5.158011in}}%
\pgfpathcurveto{\pgfqpoint{3.548381in}{5.149775in}}{\pgfqpoint{3.551653in}{5.141875in}}{\pgfqpoint{3.557477in}{5.136051in}}%
\pgfpathcurveto{\pgfqpoint{3.563301in}{5.130227in}}{\pgfqpoint{3.571201in}{5.126955in}}{\pgfqpoint{3.579438in}{5.126955in}}%
\pgfpathclose%
\pgfusepath{stroke,fill}%
\end{pgfscope}%
\begin{pgfscope}%
\pgfpathrectangle{\pgfqpoint{0.894063in}{3.540000in}}{\pgfqpoint{6.713438in}{2.060556in}} %
\pgfusepath{clip}%
\pgfsetbuttcap%
\pgfsetroundjoin%
\definecolor{currentfill}{rgb}{0.000000,0.000000,1.000000}%
\pgfsetfillcolor{currentfill}%
\pgfsetlinewidth{1.003750pt}%
\definecolor{currentstroke}{rgb}{0.000000,0.000000,0.000000}%
\pgfsetstrokecolor{currentstroke}%
\pgfsetdash{}{0pt}%
\pgfpathmoveto{\pgfqpoint{2.371019in}{5.127014in}}%
\pgfpathcurveto{\pgfqpoint{2.379255in}{5.127014in}}{\pgfqpoint{2.387155in}{5.130286in}}{\pgfqpoint{2.392979in}{5.136110in}}%
\pgfpathcurveto{\pgfqpoint{2.398803in}{5.141934in}}{\pgfqpoint{2.402075in}{5.149834in}}{\pgfqpoint{2.402075in}{5.158070in}}%
\pgfpathcurveto{\pgfqpoint{2.402075in}{5.166307in}}{\pgfqpoint{2.398803in}{5.174207in}}{\pgfqpoint{2.392979in}{5.180031in}}%
\pgfpathcurveto{\pgfqpoint{2.387155in}{5.185855in}}{\pgfqpoint{2.379255in}{5.189127in}}{\pgfqpoint{2.371019in}{5.189127in}}%
\pgfpathcurveto{\pgfqpoint{2.362782in}{5.189127in}}{\pgfqpoint{2.354882in}{5.185855in}}{\pgfqpoint{2.349058in}{5.180031in}}%
\pgfpathcurveto{\pgfqpoint{2.343235in}{5.174207in}}{\pgfqpoint{2.339962in}{5.166307in}}{\pgfqpoint{2.339962in}{5.158070in}}%
\pgfpathcurveto{\pgfqpoint{2.339962in}{5.149834in}}{\pgfqpoint{2.343235in}{5.141934in}}{\pgfqpoint{2.349058in}{5.136110in}}%
\pgfpathcurveto{\pgfqpoint{2.354882in}{5.130286in}}{\pgfqpoint{2.362782in}{5.127014in}}{\pgfqpoint{2.371019in}{5.127014in}}%
\pgfpathclose%
\pgfusepath{stroke,fill}%
\end{pgfscope}%
\begin{pgfscope}%
\pgfpathrectangle{\pgfqpoint{0.894063in}{3.540000in}}{\pgfqpoint{6.713438in}{2.060556in}} %
\pgfusepath{clip}%
\pgfsetbuttcap%
\pgfsetroundjoin%
\definecolor{currentfill}{rgb}{0.000000,0.000000,1.000000}%
\pgfsetfillcolor{currentfill}%
\pgfsetlinewidth{1.003750pt}%
\definecolor{currentstroke}{rgb}{0.000000,0.000000,0.000000}%
\pgfsetstrokecolor{currentstroke}%
\pgfsetdash{}{0pt}%
\pgfpathmoveto{\pgfqpoint{3.982244in}{5.126943in}}%
\pgfpathcurveto{\pgfqpoint{3.990480in}{5.126943in}}{\pgfqpoint{3.998380in}{5.130215in}}{\pgfqpoint{4.004204in}{5.136039in}}%
\pgfpathcurveto{\pgfqpoint{4.010028in}{5.141863in}}{\pgfqpoint{4.013300in}{5.149763in}}{\pgfqpoint{4.013300in}{5.157999in}}%
\pgfpathcurveto{\pgfqpoint{4.013300in}{5.166235in}}{\pgfqpoint{4.010028in}{5.174135in}}{\pgfqpoint{4.004204in}{5.179959in}}%
\pgfpathcurveto{\pgfqpoint{3.998380in}{5.185783in}}{\pgfqpoint{3.990480in}{5.189056in}}{\pgfqpoint{3.982244in}{5.189056in}}%
\pgfpathcurveto{\pgfqpoint{3.974007in}{5.189056in}}{\pgfqpoint{3.966107in}{5.185783in}}{\pgfqpoint{3.960283in}{5.179959in}}%
\pgfpathcurveto{\pgfqpoint{3.954460in}{5.174135in}}{\pgfqpoint{3.951187in}{5.166235in}}{\pgfqpoint{3.951187in}{5.157999in}}%
\pgfpathcurveto{\pgfqpoint{3.951187in}{5.149763in}}{\pgfqpoint{3.954460in}{5.141863in}}{\pgfqpoint{3.960283in}{5.136039in}}%
\pgfpathcurveto{\pgfqpoint{3.966107in}{5.130215in}}{\pgfqpoint{3.974007in}{5.126943in}}{\pgfqpoint{3.982244in}{5.126943in}}%
\pgfpathclose%
\pgfusepath{stroke,fill}%
\end{pgfscope}%
\begin{pgfscope}%
\pgfpathrectangle{\pgfqpoint{0.894063in}{3.540000in}}{\pgfqpoint{6.713438in}{2.060556in}} %
\pgfusepath{clip}%
\pgfsetbuttcap%
\pgfsetroundjoin%
\definecolor{currentfill}{rgb}{0.000000,0.000000,1.000000}%
\pgfsetfillcolor{currentfill}%
\pgfsetlinewidth{1.003750pt}%
\definecolor{currentstroke}{rgb}{0.000000,0.000000,0.000000}%
\pgfsetstrokecolor{currentstroke}%
\pgfsetdash{}{0pt}%
\pgfpathmoveto{\pgfqpoint{4.653588in}{5.126937in}}%
\pgfpathcurveto{\pgfqpoint{4.661824in}{5.126937in}}{\pgfqpoint{4.669724in}{5.130209in}}{\pgfqpoint{4.675548in}{5.136033in}}%
\pgfpathcurveto{\pgfqpoint{4.681372in}{5.141857in}}{\pgfqpoint{4.684644in}{5.149757in}}{\pgfqpoint{4.684644in}{5.157994in}}%
\pgfpathcurveto{\pgfqpoint{4.684644in}{5.166230in}}{\pgfqpoint{4.681372in}{5.174130in}}{\pgfqpoint{4.675548in}{5.179954in}}%
\pgfpathcurveto{\pgfqpoint{4.669724in}{5.185778in}}{\pgfqpoint{4.661824in}{5.189050in}}{\pgfqpoint{4.653588in}{5.189050in}}%
\pgfpathcurveto{\pgfqpoint{4.645351in}{5.189050in}}{\pgfqpoint{4.637451in}{5.185778in}}{\pgfqpoint{4.631627in}{5.179954in}}%
\pgfpathcurveto{\pgfqpoint{4.625803in}{5.174130in}}{\pgfqpoint{4.622531in}{5.166230in}}{\pgfqpoint{4.622531in}{5.157994in}}%
\pgfpathcurveto{\pgfqpoint{4.622531in}{5.149757in}}{\pgfqpoint{4.625803in}{5.141857in}}{\pgfqpoint{4.631627in}{5.136033in}}%
\pgfpathcurveto{\pgfqpoint{4.637451in}{5.130209in}}{\pgfqpoint{4.645351in}{5.126937in}}{\pgfqpoint{4.653588in}{5.126937in}}%
\pgfpathclose%
\pgfusepath{stroke,fill}%
\end{pgfscope}%
\begin{pgfscope}%
\pgfpathrectangle{\pgfqpoint{0.894063in}{3.540000in}}{\pgfqpoint{6.713438in}{2.060556in}} %
\pgfusepath{clip}%
\pgfsetbuttcap%
\pgfsetroundjoin%
\definecolor{currentfill}{rgb}{0.000000,0.000000,1.000000}%
\pgfsetfillcolor{currentfill}%
\pgfsetlinewidth{1.003750pt}%
\definecolor{currentstroke}{rgb}{0.000000,0.000000,0.000000}%
\pgfsetstrokecolor{currentstroke}%
\pgfsetdash{}{0pt}%
\pgfpathmoveto{\pgfqpoint{3.713706in}{5.126948in}}%
\pgfpathcurveto{\pgfqpoint{3.721943in}{5.126948in}}{\pgfqpoint{3.729843in}{5.130220in}}{\pgfqpoint{3.735667in}{5.136044in}}%
\pgfpathcurveto{\pgfqpoint{3.741490in}{5.141868in}}{\pgfqpoint{3.744763in}{5.149768in}}{\pgfqpoint{3.744763in}{5.158005in}}%
\pgfpathcurveto{\pgfqpoint{3.744763in}{5.166241in}}{\pgfqpoint{3.741490in}{5.174141in}}{\pgfqpoint{3.735667in}{5.179965in}}%
\pgfpathcurveto{\pgfqpoint{3.729843in}{5.185789in}}{\pgfqpoint{3.721943in}{5.189061in}}{\pgfqpoint{3.713706in}{5.189061in}}%
\pgfpathcurveto{\pgfqpoint{3.705470in}{5.189061in}}{\pgfqpoint{3.697570in}{5.185789in}}{\pgfqpoint{3.691746in}{5.179965in}}%
\pgfpathcurveto{\pgfqpoint{3.685922in}{5.174141in}}{\pgfqpoint{3.682650in}{5.166241in}}{\pgfqpoint{3.682650in}{5.158005in}}%
\pgfpathcurveto{\pgfqpoint{3.682650in}{5.149768in}}{\pgfqpoint{3.685922in}{5.141868in}}{\pgfqpoint{3.691746in}{5.136044in}}%
\pgfpathcurveto{\pgfqpoint{3.697570in}{5.130220in}}{\pgfqpoint{3.705470in}{5.126948in}}{\pgfqpoint{3.713706in}{5.126948in}}%
\pgfpathclose%
\pgfusepath{stroke,fill}%
\end{pgfscope}%
\begin{pgfscope}%
\pgfpathrectangle{\pgfqpoint{0.894063in}{3.540000in}}{\pgfqpoint{6.713438in}{2.060556in}} %
\pgfusepath{clip}%
\pgfsetbuttcap%
\pgfsetroundjoin%
\definecolor{currentfill}{rgb}{0.000000,0.000000,1.000000}%
\pgfsetfillcolor{currentfill}%
\pgfsetlinewidth{1.003750pt}%
\definecolor{currentstroke}{rgb}{0.000000,0.000000,0.000000}%
\pgfsetstrokecolor{currentstroke}%
\pgfsetdash{}{0pt}%
\pgfpathmoveto{\pgfqpoint{2.236750in}{5.127135in}}%
\pgfpathcurveto{\pgfqpoint{2.244986in}{5.127135in}}{\pgfqpoint{2.252886in}{5.130407in}}{\pgfqpoint{2.258710in}{5.136231in}}%
\pgfpathcurveto{\pgfqpoint{2.264534in}{5.142055in}}{\pgfqpoint{2.267806in}{5.149955in}}{\pgfqpoint{2.267806in}{5.158191in}}%
\pgfpathcurveto{\pgfqpoint{2.267806in}{5.166428in}}{\pgfqpoint{2.264534in}{5.174328in}}{\pgfqpoint{2.258710in}{5.180152in}}%
\pgfpathcurveto{\pgfqpoint{2.252886in}{5.185976in}}{\pgfqpoint{2.244986in}{5.189248in}}{\pgfqpoint{2.236750in}{5.189248in}}%
\pgfpathcurveto{\pgfqpoint{2.228514in}{5.189248in}}{\pgfqpoint{2.220614in}{5.185976in}}{\pgfqpoint{2.214790in}{5.180152in}}%
\pgfpathcurveto{\pgfqpoint{2.208966in}{5.174328in}}{\pgfqpoint{2.205694in}{5.166428in}}{\pgfqpoint{2.205694in}{5.158191in}}%
\pgfpathcurveto{\pgfqpoint{2.205694in}{5.149955in}}{\pgfqpoint{2.208966in}{5.142055in}}{\pgfqpoint{2.214790in}{5.136231in}}%
\pgfpathcurveto{\pgfqpoint{2.220614in}{5.130407in}}{\pgfqpoint{2.228514in}{5.127135in}}{\pgfqpoint{2.236750in}{5.127135in}}%
\pgfpathclose%
\pgfusepath{stroke,fill}%
\end{pgfscope}%
\begin{pgfscope}%
\pgfsetrectcap%
\pgfsetmiterjoin%
\pgfsetlinewidth{1.003750pt}%
\definecolor{currentstroke}{rgb}{0.000000,0.000000,0.000000}%
\pgfsetstrokecolor{currentstroke}%
\pgfsetdash{}{0pt}%
\pgfpathmoveto{\pgfqpoint{0.894063in}{5.600556in}}%
\pgfpathlineto{\pgfqpoint{7.607500in}{5.600556in}}%
\pgfusepath{stroke}%
\end{pgfscope}%
\begin{pgfscope}%
\pgfsetrectcap%
\pgfsetmiterjoin%
\pgfsetlinewidth{1.003750pt}%
\definecolor{currentstroke}{rgb}{0.000000,0.000000,0.000000}%
\pgfsetstrokecolor{currentstroke}%
\pgfsetdash{}{0pt}%
\pgfpathmoveto{\pgfqpoint{7.607500in}{3.540000in}}%
\pgfpathlineto{\pgfqpoint{7.607500in}{5.600556in}}%
\pgfusepath{stroke}%
\end{pgfscope}%
\begin{pgfscope}%
\pgfsetrectcap%
\pgfsetmiterjoin%
\pgfsetlinewidth{1.003750pt}%
\definecolor{currentstroke}{rgb}{0.000000,0.000000,0.000000}%
\pgfsetstrokecolor{currentstroke}%
\pgfsetdash{}{0pt}%
\pgfpathmoveto{\pgfqpoint{0.894063in}{3.540000in}}%
\pgfpathlineto{\pgfqpoint{7.607500in}{3.540000in}}%
\pgfusepath{stroke}%
\end{pgfscope}%
\begin{pgfscope}%
\pgfsetrectcap%
\pgfsetmiterjoin%
\pgfsetlinewidth{1.003750pt}%
\definecolor{currentstroke}{rgb}{0.000000,0.000000,0.000000}%
\pgfsetstrokecolor{currentstroke}%
\pgfsetdash{}{0pt}%
\pgfpathmoveto{\pgfqpoint{0.894063in}{3.540000in}}%
\pgfpathlineto{\pgfqpoint{0.894063in}{5.600556in}}%
\pgfusepath{stroke}%
\end{pgfscope}%
\begin{pgfscope}%
\pgfsetbuttcap%
\pgfsetroundjoin%
\definecolor{currentfill}{rgb}{0.000000,0.000000,0.000000}%
\pgfsetfillcolor{currentfill}%
\pgfsetlinewidth{0.501875pt}%
\definecolor{currentstroke}{rgb}{0.000000,0.000000,0.000000}%
\pgfsetstrokecolor{currentstroke}%
\pgfsetdash{}{0pt}%
\pgfsys@defobject{currentmarker}{\pgfqpoint{0.000000in}{0.000000in}}{\pgfqpoint{0.000000in}{0.055556in}}{%
\pgfpathmoveto{\pgfqpoint{0.000000in}{0.000000in}}%
\pgfpathlineto{\pgfqpoint{0.000000in}{0.055556in}}%
\pgfusepath{stroke,fill}%
}%
\begin{pgfscope}%
\pgfsys@transformshift{0.894063in}{3.540000in}%
\pgfsys@useobject{currentmarker}{}%
\end{pgfscope}%
\end{pgfscope}%
\begin{pgfscope}%
\pgfsetbuttcap%
\pgfsetroundjoin%
\definecolor{currentfill}{rgb}{0.000000,0.000000,0.000000}%
\pgfsetfillcolor{currentfill}%
\pgfsetlinewidth{0.501875pt}%
\definecolor{currentstroke}{rgb}{0.000000,0.000000,0.000000}%
\pgfsetstrokecolor{currentstroke}%
\pgfsetdash{}{0pt}%
\pgfsys@defobject{currentmarker}{\pgfqpoint{0.000000in}{-0.055556in}}{\pgfqpoint{0.000000in}{0.000000in}}{%
\pgfpathmoveto{\pgfqpoint{0.000000in}{0.000000in}}%
\pgfpathlineto{\pgfqpoint{0.000000in}{-0.055556in}}%
\pgfusepath{stroke,fill}%
}%
\begin{pgfscope}%
\pgfsys@transformshift{0.894063in}{5.600556in}%
\pgfsys@useobject{currentmarker}{}%
\end{pgfscope}%
\end{pgfscope}%
\begin{pgfscope}%
\pgftext[x=0.894063in,y=3.484444in,,top]{\sffamily\fontsize{12.000000}{14.400000}\selectfont 0}%
\end{pgfscope}%
\begin{pgfscope}%
\pgfsetbuttcap%
\pgfsetroundjoin%
\definecolor{currentfill}{rgb}{0.000000,0.000000,0.000000}%
\pgfsetfillcolor{currentfill}%
\pgfsetlinewidth{0.501875pt}%
\definecolor{currentstroke}{rgb}{0.000000,0.000000,0.000000}%
\pgfsetstrokecolor{currentstroke}%
\pgfsetdash{}{0pt}%
\pgfsys@defobject{currentmarker}{\pgfqpoint{0.000000in}{0.000000in}}{\pgfqpoint{0.000000in}{0.055556in}}{%
\pgfpathmoveto{\pgfqpoint{0.000000in}{0.000000in}}%
\pgfpathlineto{\pgfqpoint{0.000000in}{0.055556in}}%
\pgfusepath{stroke,fill}%
}%
\begin{pgfscope}%
\pgfsys@transformshift{2.236750in}{3.540000in}%
\pgfsys@useobject{currentmarker}{}%
\end{pgfscope}%
\end{pgfscope}%
\begin{pgfscope}%
\pgfsetbuttcap%
\pgfsetroundjoin%
\definecolor{currentfill}{rgb}{0.000000,0.000000,0.000000}%
\pgfsetfillcolor{currentfill}%
\pgfsetlinewidth{0.501875pt}%
\definecolor{currentstroke}{rgb}{0.000000,0.000000,0.000000}%
\pgfsetstrokecolor{currentstroke}%
\pgfsetdash{}{0pt}%
\pgfsys@defobject{currentmarker}{\pgfqpoint{0.000000in}{-0.055556in}}{\pgfqpoint{0.000000in}{0.000000in}}{%
\pgfpathmoveto{\pgfqpoint{0.000000in}{0.000000in}}%
\pgfpathlineto{\pgfqpoint{0.000000in}{-0.055556in}}%
\pgfusepath{stroke,fill}%
}%
\begin{pgfscope}%
\pgfsys@transformshift{2.236750in}{5.600556in}%
\pgfsys@useobject{currentmarker}{}%
\end{pgfscope}%
\end{pgfscope}%
\begin{pgfscope}%
\pgftext[x=2.236750in,y=3.484444in,,top]{\sffamily\fontsize{12.000000}{14.400000}\selectfont 1000}%
\end{pgfscope}%
\begin{pgfscope}%
\pgfsetbuttcap%
\pgfsetroundjoin%
\definecolor{currentfill}{rgb}{0.000000,0.000000,0.000000}%
\pgfsetfillcolor{currentfill}%
\pgfsetlinewidth{0.501875pt}%
\definecolor{currentstroke}{rgb}{0.000000,0.000000,0.000000}%
\pgfsetstrokecolor{currentstroke}%
\pgfsetdash{}{0pt}%
\pgfsys@defobject{currentmarker}{\pgfqpoint{0.000000in}{0.000000in}}{\pgfqpoint{0.000000in}{0.055556in}}{%
\pgfpathmoveto{\pgfqpoint{0.000000in}{0.000000in}}%
\pgfpathlineto{\pgfqpoint{0.000000in}{0.055556in}}%
\pgfusepath{stroke,fill}%
}%
\begin{pgfscope}%
\pgfsys@transformshift{3.579438in}{3.540000in}%
\pgfsys@useobject{currentmarker}{}%
\end{pgfscope}%
\end{pgfscope}%
\begin{pgfscope}%
\pgfsetbuttcap%
\pgfsetroundjoin%
\definecolor{currentfill}{rgb}{0.000000,0.000000,0.000000}%
\pgfsetfillcolor{currentfill}%
\pgfsetlinewidth{0.501875pt}%
\definecolor{currentstroke}{rgb}{0.000000,0.000000,0.000000}%
\pgfsetstrokecolor{currentstroke}%
\pgfsetdash{}{0pt}%
\pgfsys@defobject{currentmarker}{\pgfqpoint{0.000000in}{-0.055556in}}{\pgfqpoint{0.000000in}{0.000000in}}{%
\pgfpathmoveto{\pgfqpoint{0.000000in}{0.000000in}}%
\pgfpathlineto{\pgfqpoint{0.000000in}{-0.055556in}}%
\pgfusepath{stroke,fill}%
}%
\begin{pgfscope}%
\pgfsys@transformshift{3.579438in}{5.600556in}%
\pgfsys@useobject{currentmarker}{}%
\end{pgfscope}%
\end{pgfscope}%
\begin{pgfscope}%
\pgftext[x=3.579438in,y=3.484444in,,top]{\sffamily\fontsize{12.000000}{14.400000}\selectfont 2000}%
\end{pgfscope}%
\begin{pgfscope}%
\pgfsetbuttcap%
\pgfsetroundjoin%
\definecolor{currentfill}{rgb}{0.000000,0.000000,0.000000}%
\pgfsetfillcolor{currentfill}%
\pgfsetlinewidth{0.501875pt}%
\definecolor{currentstroke}{rgb}{0.000000,0.000000,0.000000}%
\pgfsetstrokecolor{currentstroke}%
\pgfsetdash{}{0pt}%
\pgfsys@defobject{currentmarker}{\pgfqpoint{0.000000in}{0.000000in}}{\pgfqpoint{0.000000in}{0.055556in}}{%
\pgfpathmoveto{\pgfqpoint{0.000000in}{0.000000in}}%
\pgfpathlineto{\pgfqpoint{0.000000in}{0.055556in}}%
\pgfusepath{stroke,fill}%
}%
\begin{pgfscope}%
\pgfsys@transformshift{4.922125in}{3.540000in}%
\pgfsys@useobject{currentmarker}{}%
\end{pgfscope}%
\end{pgfscope}%
\begin{pgfscope}%
\pgfsetbuttcap%
\pgfsetroundjoin%
\definecolor{currentfill}{rgb}{0.000000,0.000000,0.000000}%
\pgfsetfillcolor{currentfill}%
\pgfsetlinewidth{0.501875pt}%
\definecolor{currentstroke}{rgb}{0.000000,0.000000,0.000000}%
\pgfsetstrokecolor{currentstroke}%
\pgfsetdash{}{0pt}%
\pgfsys@defobject{currentmarker}{\pgfqpoint{0.000000in}{-0.055556in}}{\pgfqpoint{0.000000in}{0.000000in}}{%
\pgfpathmoveto{\pgfqpoint{0.000000in}{0.000000in}}%
\pgfpathlineto{\pgfqpoint{0.000000in}{-0.055556in}}%
\pgfusepath{stroke,fill}%
}%
\begin{pgfscope}%
\pgfsys@transformshift{4.922125in}{5.600556in}%
\pgfsys@useobject{currentmarker}{}%
\end{pgfscope}%
\end{pgfscope}%
\begin{pgfscope}%
\pgftext[x=4.922125in,y=3.484444in,,top]{\sffamily\fontsize{12.000000}{14.400000}\selectfont 3000}%
\end{pgfscope}%
\begin{pgfscope}%
\pgfsetbuttcap%
\pgfsetroundjoin%
\definecolor{currentfill}{rgb}{0.000000,0.000000,0.000000}%
\pgfsetfillcolor{currentfill}%
\pgfsetlinewidth{0.501875pt}%
\definecolor{currentstroke}{rgb}{0.000000,0.000000,0.000000}%
\pgfsetstrokecolor{currentstroke}%
\pgfsetdash{}{0pt}%
\pgfsys@defobject{currentmarker}{\pgfqpoint{0.000000in}{0.000000in}}{\pgfqpoint{0.000000in}{0.055556in}}{%
\pgfpathmoveto{\pgfqpoint{0.000000in}{0.000000in}}%
\pgfpathlineto{\pgfqpoint{0.000000in}{0.055556in}}%
\pgfusepath{stroke,fill}%
}%
\begin{pgfscope}%
\pgfsys@transformshift{6.264813in}{3.540000in}%
\pgfsys@useobject{currentmarker}{}%
\end{pgfscope}%
\end{pgfscope}%
\begin{pgfscope}%
\pgfsetbuttcap%
\pgfsetroundjoin%
\definecolor{currentfill}{rgb}{0.000000,0.000000,0.000000}%
\pgfsetfillcolor{currentfill}%
\pgfsetlinewidth{0.501875pt}%
\definecolor{currentstroke}{rgb}{0.000000,0.000000,0.000000}%
\pgfsetstrokecolor{currentstroke}%
\pgfsetdash{}{0pt}%
\pgfsys@defobject{currentmarker}{\pgfqpoint{0.000000in}{-0.055556in}}{\pgfqpoint{0.000000in}{0.000000in}}{%
\pgfpathmoveto{\pgfqpoint{0.000000in}{0.000000in}}%
\pgfpathlineto{\pgfqpoint{0.000000in}{-0.055556in}}%
\pgfusepath{stroke,fill}%
}%
\begin{pgfscope}%
\pgfsys@transformshift{6.264813in}{5.600556in}%
\pgfsys@useobject{currentmarker}{}%
\end{pgfscope}%
\end{pgfscope}%
\begin{pgfscope}%
\pgftext[x=6.264813in,y=3.484444in,,top]{\sffamily\fontsize{12.000000}{14.400000}\selectfont 4000}%
\end{pgfscope}%
\begin{pgfscope}%
\pgfsetbuttcap%
\pgfsetroundjoin%
\definecolor{currentfill}{rgb}{0.000000,0.000000,0.000000}%
\pgfsetfillcolor{currentfill}%
\pgfsetlinewidth{0.501875pt}%
\definecolor{currentstroke}{rgb}{0.000000,0.000000,0.000000}%
\pgfsetstrokecolor{currentstroke}%
\pgfsetdash{}{0pt}%
\pgfsys@defobject{currentmarker}{\pgfqpoint{0.000000in}{0.000000in}}{\pgfqpoint{0.000000in}{0.055556in}}{%
\pgfpathmoveto{\pgfqpoint{0.000000in}{0.000000in}}%
\pgfpathlineto{\pgfqpoint{0.000000in}{0.055556in}}%
\pgfusepath{stroke,fill}%
}%
\begin{pgfscope}%
\pgfsys@transformshift{7.607500in}{3.540000in}%
\pgfsys@useobject{currentmarker}{}%
\end{pgfscope}%
\end{pgfscope}%
\begin{pgfscope}%
\pgfsetbuttcap%
\pgfsetroundjoin%
\definecolor{currentfill}{rgb}{0.000000,0.000000,0.000000}%
\pgfsetfillcolor{currentfill}%
\pgfsetlinewidth{0.501875pt}%
\definecolor{currentstroke}{rgb}{0.000000,0.000000,0.000000}%
\pgfsetstrokecolor{currentstroke}%
\pgfsetdash{}{0pt}%
\pgfsys@defobject{currentmarker}{\pgfqpoint{0.000000in}{-0.055556in}}{\pgfqpoint{0.000000in}{0.000000in}}{%
\pgfpathmoveto{\pgfqpoint{0.000000in}{0.000000in}}%
\pgfpathlineto{\pgfqpoint{0.000000in}{-0.055556in}}%
\pgfusepath{stroke,fill}%
}%
\begin{pgfscope}%
\pgfsys@transformshift{7.607500in}{5.600556in}%
\pgfsys@useobject{currentmarker}{}%
\end{pgfscope}%
\end{pgfscope}%
\begin{pgfscope}%
\pgftext[x=7.607500in,y=3.484444in,,top]{\sffamily\fontsize{12.000000}{14.400000}\selectfont 5000}%
\end{pgfscope}%
\begin{pgfscope}%
\pgftext[x=4.250781in,y=3.253705in,,top]{\sffamily\fontsize{12.000000}{14.400000}\selectfont Requests Served}%
\end{pgfscope}%
\begin{pgfscope}%
\pgfsetbuttcap%
\pgfsetroundjoin%
\definecolor{currentfill}{rgb}{0.000000,0.000000,0.000000}%
\pgfsetfillcolor{currentfill}%
\pgfsetlinewidth{0.501875pt}%
\definecolor{currentstroke}{rgb}{0.000000,0.000000,0.000000}%
\pgfsetstrokecolor{currentstroke}%
\pgfsetdash{}{0pt}%
\pgfsys@defobject{currentmarker}{\pgfqpoint{0.000000in}{0.000000in}}{\pgfqpoint{0.055556in}{0.000000in}}{%
\pgfpathmoveto{\pgfqpoint{0.000000in}{0.000000in}}%
\pgfpathlineto{\pgfqpoint{0.055556in}{0.000000in}}%
\pgfusepath{stroke,fill}%
}%
\begin{pgfscope}%
\pgfsys@transformshift{0.894063in}{3.540000in}%
\pgfsys@useobject{currentmarker}{}%
\end{pgfscope}%
\end{pgfscope}%
\begin{pgfscope}%
\pgfsetbuttcap%
\pgfsetroundjoin%
\definecolor{currentfill}{rgb}{0.000000,0.000000,0.000000}%
\pgfsetfillcolor{currentfill}%
\pgfsetlinewidth{0.501875pt}%
\definecolor{currentstroke}{rgb}{0.000000,0.000000,0.000000}%
\pgfsetstrokecolor{currentstroke}%
\pgfsetdash{}{0pt}%
\pgfsys@defobject{currentmarker}{\pgfqpoint{-0.055556in}{0.000000in}}{\pgfqpoint{0.000000in}{0.000000in}}{%
\pgfpathmoveto{\pgfqpoint{0.000000in}{0.000000in}}%
\pgfpathlineto{\pgfqpoint{-0.055556in}{0.000000in}}%
\pgfusepath{stroke,fill}%
}%
\begin{pgfscope}%
\pgfsys@transformshift{7.607500in}{3.540000in}%
\pgfsys@useobject{currentmarker}{}%
\end{pgfscope}%
\end{pgfscope}%
\begin{pgfscope}%
\pgftext[x=0.838507in,y=3.540000in,right,]{\sffamily\fontsize{12.000000}{14.400000}\selectfont 0}%
\end{pgfscope}%
\begin{pgfscope}%
\pgfsetbuttcap%
\pgfsetroundjoin%
\definecolor{currentfill}{rgb}{0.000000,0.000000,0.000000}%
\pgfsetfillcolor{currentfill}%
\pgfsetlinewidth{0.501875pt}%
\definecolor{currentstroke}{rgb}{0.000000,0.000000,0.000000}%
\pgfsetstrokecolor{currentstroke}%
\pgfsetdash{}{0pt}%
\pgfsys@defobject{currentmarker}{\pgfqpoint{0.000000in}{0.000000in}}{\pgfqpoint{0.055556in}{0.000000in}}{%
\pgfpathmoveto{\pgfqpoint{0.000000in}{0.000000in}}%
\pgfpathlineto{\pgfqpoint{0.055556in}{0.000000in}}%
\pgfusepath{stroke,fill}%
}%
\begin{pgfscope}%
\pgfsys@transformshift{0.894063in}{3.883426in}%
\pgfsys@useobject{currentmarker}{}%
\end{pgfscope}%
\end{pgfscope}%
\begin{pgfscope}%
\pgfsetbuttcap%
\pgfsetroundjoin%
\definecolor{currentfill}{rgb}{0.000000,0.000000,0.000000}%
\pgfsetfillcolor{currentfill}%
\pgfsetlinewidth{0.501875pt}%
\definecolor{currentstroke}{rgb}{0.000000,0.000000,0.000000}%
\pgfsetstrokecolor{currentstroke}%
\pgfsetdash{}{0pt}%
\pgfsys@defobject{currentmarker}{\pgfqpoint{-0.055556in}{0.000000in}}{\pgfqpoint{0.000000in}{0.000000in}}{%
\pgfpathmoveto{\pgfqpoint{0.000000in}{0.000000in}}%
\pgfpathlineto{\pgfqpoint{-0.055556in}{0.000000in}}%
\pgfusepath{stroke,fill}%
}%
\begin{pgfscope}%
\pgfsys@transformshift{7.607500in}{3.883426in}%
\pgfsys@useobject{currentmarker}{}%
\end{pgfscope}%
\end{pgfscope}%
\begin{pgfscope}%
\pgftext[x=0.838507in,y=3.883426in,right,]{\sffamily\fontsize{12.000000}{14.400000}\selectfont 1000}%
\end{pgfscope}%
\begin{pgfscope}%
\pgfsetbuttcap%
\pgfsetroundjoin%
\definecolor{currentfill}{rgb}{0.000000,0.000000,0.000000}%
\pgfsetfillcolor{currentfill}%
\pgfsetlinewidth{0.501875pt}%
\definecolor{currentstroke}{rgb}{0.000000,0.000000,0.000000}%
\pgfsetstrokecolor{currentstroke}%
\pgfsetdash{}{0pt}%
\pgfsys@defobject{currentmarker}{\pgfqpoint{0.000000in}{0.000000in}}{\pgfqpoint{0.055556in}{0.000000in}}{%
\pgfpathmoveto{\pgfqpoint{0.000000in}{0.000000in}}%
\pgfpathlineto{\pgfqpoint{0.055556in}{0.000000in}}%
\pgfusepath{stroke,fill}%
}%
\begin{pgfscope}%
\pgfsys@transformshift{0.894063in}{4.226852in}%
\pgfsys@useobject{currentmarker}{}%
\end{pgfscope}%
\end{pgfscope}%
\begin{pgfscope}%
\pgfsetbuttcap%
\pgfsetroundjoin%
\definecolor{currentfill}{rgb}{0.000000,0.000000,0.000000}%
\pgfsetfillcolor{currentfill}%
\pgfsetlinewidth{0.501875pt}%
\definecolor{currentstroke}{rgb}{0.000000,0.000000,0.000000}%
\pgfsetstrokecolor{currentstroke}%
\pgfsetdash{}{0pt}%
\pgfsys@defobject{currentmarker}{\pgfqpoint{-0.055556in}{0.000000in}}{\pgfqpoint{0.000000in}{0.000000in}}{%
\pgfpathmoveto{\pgfqpoint{0.000000in}{0.000000in}}%
\pgfpathlineto{\pgfqpoint{-0.055556in}{0.000000in}}%
\pgfusepath{stroke,fill}%
}%
\begin{pgfscope}%
\pgfsys@transformshift{7.607500in}{4.226852in}%
\pgfsys@useobject{currentmarker}{}%
\end{pgfscope}%
\end{pgfscope}%
\begin{pgfscope}%
\pgftext[x=0.838507in,y=4.226852in,right,]{\sffamily\fontsize{12.000000}{14.400000}\selectfont 2000}%
\end{pgfscope}%
\begin{pgfscope}%
\pgfsetbuttcap%
\pgfsetroundjoin%
\definecolor{currentfill}{rgb}{0.000000,0.000000,0.000000}%
\pgfsetfillcolor{currentfill}%
\pgfsetlinewidth{0.501875pt}%
\definecolor{currentstroke}{rgb}{0.000000,0.000000,0.000000}%
\pgfsetstrokecolor{currentstroke}%
\pgfsetdash{}{0pt}%
\pgfsys@defobject{currentmarker}{\pgfqpoint{0.000000in}{0.000000in}}{\pgfqpoint{0.055556in}{0.000000in}}{%
\pgfpathmoveto{\pgfqpoint{0.000000in}{0.000000in}}%
\pgfpathlineto{\pgfqpoint{0.055556in}{0.000000in}}%
\pgfusepath{stroke,fill}%
}%
\begin{pgfscope}%
\pgfsys@transformshift{0.894063in}{4.570278in}%
\pgfsys@useobject{currentmarker}{}%
\end{pgfscope}%
\end{pgfscope}%
\begin{pgfscope}%
\pgfsetbuttcap%
\pgfsetroundjoin%
\definecolor{currentfill}{rgb}{0.000000,0.000000,0.000000}%
\pgfsetfillcolor{currentfill}%
\pgfsetlinewidth{0.501875pt}%
\definecolor{currentstroke}{rgb}{0.000000,0.000000,0.000000}%
\pgfsetstrokecolor{currentstroke}%
\pgfsetdash{}{0pt}%
\pgfsys@defobject{currentmarker}{\pgfqpoint{-0.055556in}{0.000000in}}{\pgfqpoint{0.000000in}{0.000000in}}{%
\pgfpathmoveto{\pgfqpoint{0.000000in}{0.000000in}}%
\pgfpathlineto{\pgfqpoint{-0.055556in}{0.000000in}}%
\pgfusepath{stroke,fill}%
}%
\begin{pgfscope}%
\pgfsys@transformshift{7.607500in}{4.570278in}%
\pgfsys@useobject{currentmarker}{}%
\end{pgfscope}%
\end{pgfscope}%
\begin{pgfscope}%
\pgftext[x=0.838507in,y=4.570278in,right,]{\sffamily\fontsize{12.000000}{14.400000}\selectfont 3000}%
\end{pgfscope}%
\begin{pgfscope}%
\pgfsetbuttcap%
\pgfsetroundjoin%
\definecolor{currentfill}{rgb}{0.000000,0.000000,0.000000}%
\pgfsetfillcolor{currentfill}%
\pgfsetlinewidth{0.501875pt}%
\definecolor{currentstroke}{rgb}{0.000000,0.000000,0.000000}%
\pgfsetstrokecolor{currentstroke}%
\pgfsetdash{}{0pt}%
\pgfsys@defobject{currentmarker}{\pgfqpoint{0.000000in}{0.000000in}}{\pgfqpoint{0.055556in}{0.000000in}}{%
\pgfpathmoveto{\pgfqpoint{0.000000in}{0.000000in}}%
\pgfpathlineto{\pgfqpoint{0.055556in}{0.000000in}}%
\pgfusepath{stroke,fill}%
}%
\begin{pgfscope}%
\pgfsys@transformshift{0.894063in}{4.913704in}%
\pgfsys@useobject{currentmarker}{}%
\end{pgfscope}%
\end{pgfscope}%
\begin{pgfscope}%
\pgfsetbuttcap%
\pgfsetroundjoin%
\definecolor{currentfill}{rgb}{0.000000,0.000000,0.000000}%
\pgfsetfillcolor{currentfill}%
\pgfsetlinewidth{0.501875pt}%
\definecolor{currentstroke}{rgb}{0.000000,0.000000,0.000000}%
\pgfsetstrokecolor{currentstroke}%
\pgfsetdash{}{0pt}%
\pgfsys@defobject{currentmarker}{\pgfqpoint{-0.055556in}{0.000000in}}{\pgfqpoint{0.000000in}{0.000000in}}{%
\pgfpathmoveto{\pgfqpoint{0.000000in}{0.000000in}}%
\pgfpathlineto{\pgfqpoint{-0.055556in}{0.000000in}}%
\pgfusepath{stroke,fill}%
}%
\begin{pgfscope}%
\pgfsys@transformshift{7.607500in}{4.913704in}%
\pgfsys@useobject{currentmarker}{}%
\end{pgfscope}%
\end{pgfscope}%
\begin{pgfscope}%
\pgftext[x=0.838507in,y=4.913704in,right,]{\sffamily\fontsize{12.000000}{14.400000}\selectfont 4000}%
\end{pgfscope}%
\begin{pgfscope}%
\pgfsetbuttcap%
\pgfsetroundjoin%
\definecolor{currentfill}{rgb}{0.000000,0.000000,0.000000}%
\pgfsetfillcolor{currentfill}%
\pgfsetlinewidth{0.501875pt}%
\definecolor{currentstroke}{rgb}{0.000000,0.000000,0.000000}%
\pgfsetstrokecolor{currentstroke}%
\pgfsetdash{}{0pt}%
\pgfsys@defobject{currentmarker}{\pgfqpoint{0.000000in}{0.000000in}}{\pgfqpoint{0.055556in}{0.000000in}}{%
\pgfpathmoveto{\pgfqpoint{0.000000in}{0.000000in}}%
\pgfpathlineto{\pgfqpoint{0.055556in}{0.000000in}}%
\pgfusepath{stroke,fill}%
}%
\begin{pgfscope}%
\pgfsys@transformshift{0.894063in}{5.257130in}%
\pgfsys@useobject{currentmarker}{}%
\end{pgfscope}%
\end{pgfscope}%
\begin{pgfscope}%
\pgfsetbuttcap%
\pgfsetroundjoin%
\definecolor{currentfill}{rgb}{0.000000,0.000000,0.000000}%
\pgfsetfillcolor{currentfill}%
\pgfsetlinewidth{0.501875pt}%
\definecolor{currentstroke}{rgb}{0.000000,0.000000,0.000000}%
\pgfsetstrokecolor{currentstroke}%
\pgfsetdash{}{0pt}%
\pgfsys@defobject{currentmarker}{\pgfqpoint{-0.055556in}{0.000000in}}{\pgfqpoint{0.000000in}{0.000000in}}{%
\pgfpathmoveto{\pgfqpoint{0.000000in}{0.000000in}}%
\pgfpathlineto{\pgfqpoint{-0.055556in}{0.000000in}}%
\pgfusepath{stroke,fill}%
}%
\begin{pgfscope}%
\pgfsys@transformshift{7.607500in}{5.257130in}%
\pgfsys@useobject{currentmarker}{}%
\end{pgfscope}%
\end{pgfscope}%
\begin{pgfscope}%
\pgftext[x=0.838507in,y=5.257130in,right,]{\sffamily\fontsize{12.000000}{14.400000}\selectfont 5000}%
\end{pgfscope}%
\begin{pgfscope}%
\pgfsetbuttcap%
\pgfsetroundjoin%
\definecolor{currentfill}{rgb}{0.000000,0.000000,0.000000}%
\pgfsetfillcolor{currentfill}%
\pgfsetlinewidth{0.501875pt}%
\definecolor{currentstroke}{rgb}{0.000000,0.000000,0.000000}%
\pgfsetstrokecolor{currentstroke}%
\pgfsetdash{}{0pt}%
\pgfsys@defobject{currentmarker}{\pgfqpoint{0.000000in}{0.000000in}}{\pgfqpoint{0.055556in}{0.000000in}}{%
\pgfpathmoveto{\pgfqpoint{0.000000in}{0.000000in}}%
\pgfpathlineto{\pgfqpoint{0.055556in}{0.000000in}}%
\pgfusepath{stroke,fill}%
}%
\begin{pgfscope}%
\pgfsys@transformshift{0.894063in}{5.600556in}%
\pgfsys@useobject{currentmarker}{}%
\end{pgfscope}%
\end{pgfscope}%
\begin{pgfscope}%
\pgfsetbuttcap%
\pgfsetroundjoin%
\definecolor{currentfill}{rgb}{0.000000,0.000000,0.000000}%
\pgfsetfillcolor{currentfill}%
\pgfsetlinewidth{0.501875pt}%
\definecolor{currentstroke}{rgb}{0.000000,0.000000,0.000000}%
\pgfsetstrokecolor{currentstroke}%
\pgfsetdash{}{0pt}%
\pgfsys@defobject{currentmarker}{\pgfqpoint{-0.055556in}{0.000000in}}{\pgfqpoint{0.000000in}{0.000000in}}{%
\pgfpathmoveto{\pgfqpoint{0.000000in}{0.000000in}}%
\pgfpathlineto{\pgfqpoint{-0.055556in}{0.000000in}}%
\pgfusepath{stroke,fill}%
}%
\begin{pgfscope}%
\pgfsys@transformshift{7.607500in}{5.600556in}%
\pgfsys@useobject{currentmarker}{}%
\end{pgfscope}%
\end{pgfscope}%
\begin{pgfscope}%
\pgftext[x=0.838507in,y=5.600556in,right,]{\sffamily\fontsize{12.000000}{14.400000}\selectfont 6000}%
\end{pgfscope}%
\begin{pgfscope}%
\pgftext[x=0.344909in,y=4.570278in,,bottom,rotate=90.000000]{\sffamily\fontsize{12.000000}{14.400000}\selectfont Memory (Mb)}%
\end{pgfscope}%
\begin{pgfscope}%
\pgftext[x=4.250781in,y=5.670000in,,base]{\sffamily\fontsize{14.400000}{17.280000}\selectfont Shared memory}%
\end{pgfscope}%
\begin{pgfscope}%
\pgfsetbuttcap%
\pgfsetmiterjoin%
\definecolor{currentfill}{rgb}{1.000000,1.000000,1.000000}%
\pgfsetfillcolor{currentfill}%
\pgfsetlinewidth{0.000000pt}%
\definecolor{currentstroke}{rgb}{0.000000,0.000000,0.000000}%
\pgfsetstrokecolor{currentstroke}%
\pgfsetstrokeopacity{0.000000}%
\pgfsetdash{}{0pt}%
\pgfpathmoveto{\pgfqpoint{0.894063in}{0.630000in}}%
\pgfpathlineto{\pgfqpoint{7.607500in}{0.630000in}}%
\pgfpathlineto{\pgfqpoint{7.607500in}{2.690556in}}%
\pgfpathlineto{\pgfqpoint{0.894063in}{2.690556in}}%
\pgfpathclose%
\pgfusepath{fill}%
\end{pgfscope}%
\begin{pgfscope}%
\pgfpathrectangle{\pgfqpoint{0.894063in}{0.630000in}}{\pgfqpoint{6.713438in}{2.060556in}} %
\pgfusepath{clip}%
\pgfsetbuttcap%
\pgfsetroundjoin%
\definecolor{currentfill}{rgb}{0.000000,0.000000,1.000000}%
\pgfsetfillcolor{currentfill}%
\pgfsetlinewidth{1.003750pt}%
\definecolor{currentstroke}{rgb}{0.000000,0.000000,0.000000}%
\pgfsetstrokecolor{currentstroke}%
\pgfsetdash{}{0pt}%
\pgfpathmoveto{\pgfqpoint{6.667619in}{1.983825in}}%
\pgfpathcurveto{\pgfqpoint{6.675855in}{1.983825in}}{\pgfqpoint{6.683755in}{1.987098in}}{\pgfqpoint{6.689579in}{1.992921in}}%
\pgfpathcurveto{\pgfqpoint{6.695403in}{1.998745in}}{\pgfqpoint{6.698675in}{2.006645in}}{\pgfqpoint{6.698675in}{2.014882in}}%
\pgfpathcurveto{\pgfqpoint{6.698675in}{2.023118in}}{\pgfqpoint{6.695403in}{2.031018in}}{\pgfqpoint{6.689579in}{2.036842in}}%
\pgfpathcurveto{\pgfqpoint{6.683755in}{2.042666in}}{\pgfqpoint{6.675855in}{2.045938in}}{\pgfqpoint{6.667619in}{2.045938in}}%
\pgfpathcurveto{\pgfqpoint{6.659382in}{2.045938in}}{\pgfqpoint{6.651482in}{2.042666in}}{\pgfqpoint{6.645658in}{2.036842in}}%
\pgfpathcurveto{\pgfqpoint{6.639835in}{2.031018in}}{\pgfqpoint{6.636562in}{2.023118in}}{\pgfqpoint{6.636562in}{2.014882in}}%
\pgfpathcurveto{\pgfqpoint{6.636562in}{2.006645in}}{\pgfqpoint{6.639835in}{1.998745in}}{\pgfqpoint{6.645658in}{1.992921in}}%
\pgfpathcurveto{\pgfqpoint{6.651482in}{1.987098in}}{\pgfqpoint{6.659382in}{1.983825in}}{\pgfqpoint{6.667619in}{1.983825in}}%
\pgfpathclose%
\pgfusepath{stroke,fill}%
\end{pgfscope}%
\begin{pgfscope}%
\pgfpathrectangle{\pgfqpoint{0.894063in}{0.630000in}}{\pgfqpoint{6.713438in}{2.060556in}} %
\pgfusepath{clip}%
\pgfsetbuttcap%
\pgfsetroundjoin%
\definecolor{currentfill}{rgb}{0.000000,0.000000,1.000000}%
\pgfsetfillcolor{currentfill}%
\pgfsetlinewidth{1.003750pt}%
\definecolor{currentstroke}{rgb}{0.000000,0.000000,0.000000}%
\pgfsetstrokecolor{currentstroke}%
\pgfsetdash{}{0pt}%
\pgfpathmoveto{\pgfqpoint{2.639556in}{1.414871in}}%
\pgfpathcurveto{\pgfqpoint{2.647793in}{1.414871in}}{\pgfqpoint{2.655693in}{1.418143in}}{\pgfqpoint{2.661517in}{1.423967in}}%
\pgfpathcurveto{\pgfqpoint{2.667340in}{1.429791in}}{\pgfqpoint{2.670613in}{1.437691in}}{\pgfqpoint{2.670613in}{1.445927in}}%
\pgfpathcurveto{\pgfqpoint{2.670613in}{1.454163in}}{\pgfqpoint{2.667340in}{1.462063in}}{\pgfqpoint{2.661517in}{1.467887in}}%
\pgfpathcurveto{\pgfqpoint{2.655693in}{1.473711in}}{\pgfqpoint{2.647793in}{1.476984in}}{\pgfqpoint{2.639556in}{1.476984in}}%
\pgfpathcurveto{\pgfqpoint{2.631320in}{1.476984in}}{\pgfqpoint{2.623420in}{1.473711in}}{\pgfqpoint{2.617596in}{1.467887in}}%
\pgfpathcurveto{\pgfqpoint{2.611772in}{1.462063in}}{\pgfqpoint{2.608500in}{1.454163in}}{\pgfqpoint{2.608500in}{1.445927in}}%
\pgfpathcurveto{\pgfqpoint{2.608500in}{1.437691in}}{\pgfqpoint{2.611772in}{1.429791in}}{\pgfqpoint{2.617596in}{1.423967in}}%
\pgfpathcurveto{\pgfqpoint{2.623420in}{1.418143in}}{\pgfqpoint{2.631320in}{1.414871in}}{\pgfqpoint{2.639556in}{1.414871in}}%
\pgfpathclose%
\pgfusepath{stroke,fill}%
\end{pgfscope}%
\begin{pgfscope}%
\pgfpathrectangle{\pgfqpoint{0.894063in}{0.630000in}}{\pgfqpoint{6.713438in}{2.060556in}} %
\pgfusepath{clip}%
\pgfsetbuttcap%
\pgfsetroundjoin%
\definecolor{currentfill}{rgb}{0.000000,0.000000,1.000000}%
\pgfsetfillcolor{currentfill}%
\pgfsetlinewidth{1.003750pt}%
\definecolor{currentstroke}{rgb}{0.000000,0.000000,0.000000}%
\pgfsetstrokecolor{currentstroke}%
\pgfsetdash{}{0pt}%
\pgfpathmoveto{\pgfqpoint{1.699675in}{1.286162in}}%
\pgfpathcurveto{\pgfqpoint{1.707911in}{1.286162in}}{\pgfqpoint{1.715811in}{1.289435in}}{\pgfqpoint{1.721635in}{1.295259in}}%
\pgfpathcurveto{\pgfqpoint{1.727459in}{1.301082in}}{\pgfqpoint{1.730731in}{1.308983in}}{\pgfqpoint{1.730731in}{1.317219in}}%
\pgfpathcurveto{\pgfqpoint{1.730731in}{1.325455in}}{\pgfqpoint{1.727459in}{1.333355in}}{\pgfqpoint{1.721635in}{1.339179in}}%
\pgfpathcurveto{\pgfqpoint{1.715811in}{1.345003in}}{\pgfqpoint{1.707911in}{1.348275in}}{\pgfqpoint{1.699675in}{1.348275in}}%
\pgfpathcurveto{\pgfqpoint{1.691439in}{1.348275in}}{\pgfqpoint{1.683539in}{1.345003in}}{\pgfqpoint{1.677715in}{1.339179in}}%
\pgfpathcurveto{\pgfqpoint{1.671891in}{1.333355in}}{\pgfqpoint{1.668619in}{1.325455in}}{\pgfqpoint{1.668619in}{1.317219in}}%
\pgfpathcurveto{\pgfqpoint{1.668619in}{1.308983in}}{\pgfqpoint{1.671891in}{1.301082in}}{\pgfqpoint{1.677715in}{1.295259in}}%
\pgfpathcurveto{\pgfqpoint{1.683539in}{1.289435in}}{\pgfqpoint{1.691439in}{1.286162in}}{\pgfqpoint{1.699675in}{1.286162in}}%
\pgfpathclose%
\pgfusepath{stroke,fill}%
\end{pgfscope}%
\begin{pgfscope}%
\pgfpathrectangle{\pgfqpoint{0.894063in}{0.630000in}}{\pgfqpoint{6.713438in}{2.060556in}} %
\pgfusepath{clip}%
\pgfsetbuttcap%
\pgfsetroundjoin%
\definecolor{currentfill}{rgb}{0.000000,0.000000,1.000000}%
\pgfsetfillcolor{currentfill}%
\pgfsetlinewidth{1.003750pt}%
\definecolor{currentstroke}{rgb}{0.000000,0.000000,0.000000}%
\pgfsetstrokecolor{currentstroke}%
\pgfsetdash{}{0pt}%
\pgfpathmoveto{\pgfqpoint{1.162600in}{0.719203in}}%
\pgfpathcurveto{\pgfqpoint{1.170836in}{0.719203in}}{\pgfqpoint{1.178736in}{0.722476in}}{\pgfqpoint{1.184560in}{0.728300in}}%
\pgfpathcurveto{\pgfqpoint{1.190384in}{0.734124in}}{\pgfqpoint{1.193656in}{0.742024in}}{\pgfqpoint{1.193656in}{0.750260in}}%
\pgfpathcurveto{\pgfqpoint{1.193656in}{0.758496in}}{\pgfqpoint{1.190384in}{0.766396in}}{\pgfqpoint{1.184560in}{0.772220in}}%
\pgfpathcurveto{\pgfqpoint{1.178736in}{0.778044in}}{\pgfqpoint{1.170836in}{0.781316in}}{\pgfqpoint{1.162600in}{0.781316in}}%
\pgfpathcurveto{\pgfqpoint{1.154364in}{0.781316in}}{\pgfqpoint{1.146464in}{0.778044in}}{\pgfqpoint{1.140640in}{0.772220in}}%
\pgfpathcurveto{\pgfqpoint{1.134816in}{0.766396in}}{\pgfqpoint{1.131544in}{0.758496in}}{\pgfqpoint{1.131544in}{0.750260in}}%
\pgfpathcurveto{\pgfqpoint{1.131544in}{0.742024in}}{\pgfqpoint{1.134816in}{0.734124in}}{\pgfqpoint{1.140640in}{0.728300in}}%
\pgfpathcurveto{\pgfqpoint{1.146464in}{0.722476in}}{\pgfqpoint{1.154364in}{0.719203in}}{\pgfqpoint{1.162600in}{0.719203in}}%
\pgfpathclose%
\pgfusepath{stroke,fill}%
\end{pgfscope}%
\begin{pgfscope}%
\pgfpathrectangle{\pgfqpoint{0.894063in}{0.630000in}}{\pgfqpoint{6.713438in}{2.060556in}} %
\pgfusepath{clip}%
\pgfsetbuttcap%
\pgfsetroundjoin%
\definecolor{currentfill}{rgb}{0.000000,0.000000,1.000000}%
\pgfsetfillcolor{currentfill}%
\pgfsetlinewidth{1.003750pt}%
\definecolor{currentstroke}{rgb}{0.000000,0.000000,0.000000}%
\pgfsetstrokecolor{currentstroke}%
\pgfsetdash{}{0pt}%
\pgfpathmoveto{\pgfqpoint{1.833944in}{1.295146in}}%
\pgfpathcurveto{\pgfqpoint{1.842180in}{1.295146in}}{\pgfqpoint{1.850080in}{1.298419in}}{\pgfqpoint{1.855904in}{1.304243in}}%
\pgfpathcurveto{\pgfqpoint{1.861728in}{1.310067in}}{\pgfqpoint{1.865000in}{1.317967in}}{\pgfqpoint{1.865000in}{1.326203in}}%
\pgfpathcurveto{\pgfqpoint{1.865000in}{1.334439in}}{\pgfqpoint{1.861728in}{1.342339in}}{\pgfqpoint{1.855904in}{1.348163in}}%
\pgfpathcurveto{\pgfqpoint{1.850080in}{1.353987in}}{\pgfqpoint{1.842180in}{1.357259in}}{\pgfqpoint{1.833944in}{1.357259in}}%
\pgfpathcurveto{\pgfqpoint{1.825707in}{1.357259in}}{\pgfqpoint{1.817807in}{1.353987in}}{\pgfqpoint{1.811983in}{1.348163in}}%
\pgfpathcurveto{\pgfqpoint{1.806160in}{1.342339in}}{\pgfqpoint{1.802887in}{1.334439in}}{\pgfqpoint{1.802887in}{1.326203in}}%
\pgfpathcurveto{\pgfqpoint{1.802887in}{1.317967in}}{\pgfqpoint{1.806160in}{1.310067in}}{\pgfqpoint{1.811983in}{1.304243in}}%
\pgfpathcurveto{\pgfqpoint{1.817807in}{1.298419in}}{\pgfqpoint{1.825707in}{1.295146in}}{\pgfqpoint{1.833944in}{1.295146in}}%
\pgfpathclose%
\pgfusepath{stroke,fill}%
\end{pgfscope}%
\begin{pgfscope}%
\pgfpathrectangle{\pgfqpoint{0.894063in}{0.630000in}}{\pgfqpoint{6.713438in}{2.060556in}} %
\pgfusepath{clip}%
\pgfsetbuttcap%
\pgfsetroundjoin%
\definecolor{currentfill}{rgb}{0.000000,0.000000,1.000000}%
\pgfsetfillcolor{currentfill}%
\pgfsetlinewidth{1.003750pt}%
\definecolor{currentstroke}{rgb}{0.000000,0.000000,0.000000}%
\pgfsetstrokecolor{currentstroke}%
\pgfsetdash{}{0pt}%
\pgfpathmoveto{\pgfqpoint{5.996275in}{1.883199in}}%
\pgfpathcurveto{\pgfqpoint{6.004511in}{1.883199in}}{\pgfqpoint{6.012411in}{1.886472in}}{\pgfqpoint{6.018235in}{1.892296in}}%
\pgfpathcurveto{\pgfqpoint{6.024059in}{1.898120in}}{\pgfqpoint{6.027331in}{1.906020in}}{\pgfqpoint{6.027331in}{1.914256in}}%
\pgfpathcurveto{\pgfqpoint{6.027331in}{1.922492in}}{\pgfqpoint{6.024059in}{1.930392in}}{\pgfqpoint{6.018235in}{1.936216in}}%
\pgfpathcurveto{\pgfqpoint{6.012411in}{1.942040in}}{\pgfqpoint{6.004511in}{1.945312in}}{\pgfqpoint{5.996275in}{1.945312in}}%
\pgfpathcurveto{\pgfqpoint{5.988039in}{1.945312in}}{\pgfqpoint{5.980139in}{1.942040in}}{\pgfqpoint{5.974315in}{1.936216in}}%
\pgfpathcurveto{\pgfqpoint{5.968491in}{1.930392in}}{\pgfqpoint{5.965219in}{1.922492in}}{\pgfqpoint{5.965219in}{1.914256in}}%
\pgfpathcurveto{\pgfqpoint{5.965219in}{1.906020in}}{\pgfqpoint{5.968491in}{1.898120in}}{\pgfqpoint{5.974315in}{1.892296in}}%
\pgfpathcurveto{\pgfqpoint{5.980139in}{1.886472in}}{\pgfqpoint{5.988039in}{1.883199in}}{\pgfqpoint{5.996275in}{1.883199in}}%
\pgfpathclose%
\pgfusepath{stroke,fill}%
\end{pgfscope}%
\begin{pgfscope}%
\pgfpathrectangle{\pgfqpoint{0.894063in}{0.630000in}}{\pgfqpoint{6.713438in}{2.060556in}} %
\pgfusepath{clip}%
\pgfsetbuttcap%
\pgfsetroundjoin%
\definecolor{currentfill}{rgb}{0.000000,0.000000,1.000000}%
\pgfsetfillcolor{currentfill}%
\pgfsetlinewidth{1.003750pt}%
\definecolor{currentstroke}{rgb}{0.000000,0.000000,0.000000}%
\pgfsetstrokecolor{currentstroke}%
\pgfsetdash{}{0pt}%
\pgfpathmoveto{\pgfqpoint{6.399081in}{1.951922in}}%
\pgfpathcurveto{\pgfqpoint{6.407318in}{1.951922in}}{\pgfqpoint{6.415218in}{1.955194in}}{\pgfqpoint{6.421042in}{1.961018in}}%
\pgfpathcurveto{\pgfqpoint{6.426865in}{1.966842in}}{\pgfqpoint{6.430138in}{1.974742in}}{\pgfqpoint{6.430138in}{1.982978in}}%
\pgfpathcurveto{\pgfqpoint{6.430138in}{1.991215in}}{\pgfqpoint{6.426865in}{1.999115in}}{\pgfqpoint{6.421042in}{2.004939in}}%
\pgfpathcurveto{\pgfqpoint{6.415218in}{2.010763in}}{\pgfqpoint{6.407318in}{2.014035in}}{\pgfqpoint{6.399081in}{2.014035in}}%
\pgfpathcurveto{\pgfqpoint{6.390845in}{2.014035in}}{\pgfqpoint{6.382945in}{2.010763in}}{\pgfqpoint{6.377121in}{2.004939in}}%
\pgfpathcurveto{\pgfqpoint{6.371297in}{1.999115in}}{\pgfqpoint{6.368025in}{1.991215in}}{\pgfqpoint{6.368025in}{1.982978in}}%
\pgfpathcurveto{\pgfqpoint{6.368025in}{1.974742in}}{\pgfqpoint{6.371297in}{1.966842in}}{\pgfqpoint{6.377121in}{1.961018in}}%
\pgfpathcurveto{\pgfqpoint{6.382945in}{1.955194in}}{\pgfqpoint{6.390845in}{1.951922in}}{\pgfqpoint{6.399081in}{1.951922in}}%
\pgfpathclose%
\pgfusepath{stroke,fill}%
\end{pgfscope}%
\begin{pgfscope}%
\pgfpathrectangle{\pgfqpoint{0.894063in}{0.630000in}}{\pgfqpoint{6.713438in}{2.060556in}} %
\pgfusepath{clip}%
\pgfsetbuttcap%
\pgfsetroundjoin%
\definecolor{currentfill}{rgb}{0.000000,0.000000,1.000000}%
\pgfsetfillcolor{currentfill}%
\pgfsetlinewidth{1.003750pt}%
\definecolor{currentstroke}{rgb}{0.000000,0.000000,0.000000}%
\pgfsetstrokecolor{currentstroke}%
\pgfsetdash{}{0pt}%
\pgfpathmoveto{\pgfqpoint{4.787856in}{1.726656in}}%
\pgfpathcurveto{\pgfqpoint{4.796093in}{1.726656in}}{\pgfqpoint{4.803993in}{1.729928in}}{\pgfqpoint{4.809817in}{1.735752in}}%
\pgfpathcurveto{\pgfqpoint{4.815640in}{1.741576in}}{\pgfqpoint{4.818913in}{1.749476in}}{\pgfqpoint{4.818913in}{1.757713in}}%
\pgfpathcurveto{\pgfqpoint{4.818913in}{1.765949in}}{\pgfqpoint{4.815640in}{1.773849in}}{\pgfqpoint{4.809817in}{1.779673in}}%
\pgfpathcurveto{\pgfqpoint{4.803993in}{1.785497in}}{\pgfqpoint{4.796093in}{1.788769in}}{\pgfqpoint{4.787856in}{1.788769in}}%
\pgfpathcurveto{\pgfqpoint{4.779620in}{1.788769in}}{\pgfqpoint{4.771720in}{1.785497in}}{\pgfqpoint{4.765896in}{1.779673in}}%
\pgfpathcurveto{\pgfqpoint{4.760072in}{1.773849in}}{\pgfqpoint{4.756800in}{1.765949in}}{\pgfqpoint{4.756800in}{1.757713in}}%
\pgfpathcurveto{\pgfqpoint{4.756800in}{1.749476in}}{\pgfqpoint{4.760072in}{1.741576in}}{\pgfqpoint{4.765896in}{1.735752in}}%
\pgfpathcurveto{\pgfqpoint{4.771720in}{1.729928in}}{\pgfqpoint{4.779620in}{1.726656in}}{\pgfqpoint{4.787856in}{1.726656in}}%
\pgfpathclose%
\pgfusepath{stroke,fill}%
\end{pgfscope}%
\begin{pgfscope}%
\pgfpathrectangle{\pgfqpoint{0.894063in}{0.630000in}}{\pgfqpoint{6.713438in}{2.060556in}} %
\pgfusepath{clip}%
\pgfsetbuttcap%
\pgfsetroundjoin%
\definecolor{currentfill}{rgb}{0.000000,0.000000,1.000000}%
\pgfsetfillcolor{currentfill}%
\pgfsetlinewidth{1.003750pt}%
\definecolor{currentstroke}{rgb}{0.000000,0.000000,0.000000}%
\pgfsetstrokecolor{currentstroke}%
\pgfsetdash{}{0pt}%
\pgfpathmoveto{\pgfqpoint{4.922125in}{1.748098in}}%
\pgfpathcurveto{\pgfqpoint{4.930361in}{1.748098in}}{\pgfqpoint{4.938261in}{1.751370in}}{\pgfqpoint{4.944085in}{1.757194in}}%
\pgfpathcurveto{\pgfqpoint{4.949909in}{1.763018in}}{\pgfqpoint{4.953181in}{1.770918in}}{\pgfqpoint{4.953181in}{1.779154in}}%
\pgfpathcurveto{\pgfqpoint{4.953181in}{1.787390in}}{\pgfqpoint{4.949909in}{1.795291in}}{\pgfqpoint{4.944085in}{1.801114in}}%
\pgfpathcurveto{\pgfqpoint{4.938261in}{1.806938in}}{\pgfqpoint{4.930361in}{1.810211in}}{\pgfqpoint{4.922125in}{1.810211in}}%
\pgfpathcurveto{\pgfqpoint{4.913889in}{1.810211in}}{\pgfqpoint{4.905989in}{1.806938in}}{\pgfqpoint{4.900165in}{1.801114in}}%
\pgfpathcurveto{\pgfqpoint{4.894341in}{1.795291in}}{\pgfqpoint{4.891069in}{1.787390in}}{\pgfqpoint{4.891069in}{1.779154in}}%
\pgfpathcurveto{\pgfqpoint{4.891069in}{1.770918in}}{\pgfqpoint{4.894341in}{1.763018in}}{\pgfqpoint{4.900165in}{1.757194in}}%
\pgfpathcurveto{\pgfqpoint{4.905989in}{1.751370in}}{\pgfqpoint{4.913889in}{1.748098in}}{\pgfqpoint{4.922125in}{1.748098in}}%
\pgfpathclose%
\pgfusepath{stroke,fill}%
\end{pgfscope}%
\begin{pgfscope}%
\pgfpathrectangle{\pgfqpoint{0.894063in}{0.630000in}}{\pgfqpoint{6.713438in}{2.060556in}} %
\pgfusepath{clip}%
\pgfsetbuttcap%
\pgfsetroundjoin%
\definecolor{currentfill}{rgb}{0.000000,0.000000,1.000000}%
\pgfsetfillcolor{currentfill}%
\pgfsetlinewidth{1.003750pt}%
\definecolor{currentstroke}{rgb}{0.000000,0.000000,0.000000}%
\pgfsetstrokecolor{currentstroke}%
\pgfsetdash{}{0pt}%
\pgfpathmoveto{\pgfqpoint{6.130544in}{1.907761in}}%
\pgfpathcurveto{\pgfqpoint{6.138780in}{1.907761in}}{\pgfqpoint{6.146680in}{1.911034in}}{\pgfqpoint{6.152504in}{1.916858in}}%
\pgfpathcurveto{\pgfqpoint{6.158328in}{1.922681in}}{\pgfqpoint{6.161600in}{1.930582in}}{\pgfqpoint{6.161600in}{1.938818in}}%
\pgfpathcurveto{\pgfqpoint{6.161600in}{1.947054in}}{\pgfqpoint{6.158328in}{1.954954in}}{\pgfqpoint{6.152504in}{1.960778in}}%
\pgfpathcurveto{\pgfqpoint{6.146680in}{1.966602in}}{\pgfqpoint{6.138780in}{1.969874in}}{\pgfqpoint{6.130544in}{1.969874in}}%
\pgfpathcurveto{\pgfqpoint{6.122307in}{1.969874in}}{\pgfqpoint{6.114407in}{1.966602in}}{\pgfqpoint{6.108583in}{1.960778in}}%
\pgfpathcurveto{\pgfqpoint{6.102760in}{1.954954in}}{\pgfqpoint{6.099487in}{1.947054in}}{\pgfqpoint{6.099487in}{1.938818in}}%
\pgfpathcurveto{\pgfqpoint{6.099487in}{1.930582in}}{\pgfqpoint{6.102760in}{1.922681in}}{\pgfqpoint{6.108583in}{1.916858in}}%
\pgfpathcurveto{\pgfqpoint{6.114407in}{1.911034in}}{\pgfqpoint{6.122307in}{1.907761in}}{\pgfqpoint{6.130544in}{1.907761in}}%
\pgfpathclose%
\pgfusepath{stroke,fill}%
\end{pgfscope}%
\begin{pgfscope}%
\pgfpathrectangle{\pgfqpoint{0.894063in}{0.630000in}}{\pgfqpoint{6.713438in}{2.060556in}} %
\pgfusepath{clip}%
\pgfsetbuttcap%
\pgfsetroundjoin%
\definecolor{currentfill}{rgb}{0.000000,0.000000,1.000000}%
\pgfsetfillcolor{currentfill}%
\pgfsetlinewidth{1.003750pt}%
\definecolor{currentstroke}{rgb}{0.000000,0.000000,0.000000}%
\pgfsetstrokecolor{currentstroke}%
\pgfsetdash{}{0pt}%
\pgfpathmoveto{\pgfqpoint{5.727738in}{1.858384in}}%
\pgfpathcurveto{\pgfqpoint{5.735974in}{1.858384in}}{\pgfqpoint{5.743874in}{1.861657in}}{\pgfqpoint{5.749698in}{1.867481in}}%
\pgfpathcurveto{\pgfqpoint{5.755522in}{1.873305in}}{\pgfqpoint{5.758794in}{1.881205in}}{\pgfqpoint{5.758794in}{1.889441in}}%
\pgfpathcurveto{\pgfqpoint{5.758794in}{1.897677in}}{\pgfqpoint{5.755522in}{1.905577in}}{\pgfqpoint{5.749698in}{1.911401in}}%
\pgfpathcurveto{\pgfqpoint{5.743874in}{1.917225in}}{\pgfqpoint{5.735974in}{1.920497in}}{\pgfqpoint{5.727738in}{1.920497in}}%
\pgfpathcurveto{\pgfqpoint{5.719501in}{1.920497in}}{\pgfqpoint{5.711601in}{1.917225in}}{\pgfqpoint{5.705777in}{1.911401in}}%
\pgfpathcurveto{\pgfqpoint{5.699953in}{1.905577in}}{\pgfqpoint{5.696681in}{1.897677in}}{\pgfqpoint{5.696681in}{1.889441in}}%
\pgfpathcurveto{\pgfqpoint{5.696681in}{1.881205in}}{\pgfqpoint{5.699953in}{1.873305in}}{\pgfqpoint{5.705777in}{1.867481in}}%
\pgfpathcurveto{\pgfqpoint{5.711601in}{1.861657in}}{\pgfqpoint{5.719501in}{1.858384in}}{\pgfqpoint{5.727738in}{1.858384in}}%
\pgfpathclose%
\pgfusepath{stroke,fill}%
\end{pgfscope}%
\begin{pgfscope}%
\pgfpathrectangle{\pgfqpoint{0.894063in}{0.630000in}}{\pgfqpoint{6.713438in}{2.060556in}} %
\pgfusepath{clip}%
\pgfsetbuttcap%
\pgfsetroundjoin%
\definecolor{currentfill}{rgb}{0.000000,0.000000,1.000000}%
\pgfsetfillcolor{currentfill}%
\pgfsetlinewidth{1.003750pt}%
\definecolor{currentstroke}{rgb}{0.000000,0.000000,0.000000}%
\pgfsetstrokecolor{currentstroke}%
\pgfsetdash{}{0pt}%
\pgfpathmoveto{\pgfqpoint{1.028331in}{0.702389in}}%
\pgfpathcurveto{\pgfqpoint{1.036568in}{0.702389in}}{\pgfqpoint{1.044468in}{0.705662in}}{\pgfqpoint{1.050292in}{0.711486in}}%
\pgfpathcurveto{\pgfqpoint{1.056115in}{0.717309in}}{\pgfqpoint{1.059388in}{0.725209in}}{\pgfqpoint{1.059388in}{0.733446in}}%
\pgfpathcurveto{\pgfqpoint{1.059388in}{0.741682in}}{\pgfqpoint{1.056115in}{0.749582in}}{\pgfqpoint{1.050292in}{0.755406in}}%
\pgfpathcurveto{\pgfqpoint{1.044468in}{0.761230in}}{\pgfqpoint{1.036568in}{0.764502in}}{\pgfqpoint{1.028331in}{0.764502in}}%
\pgfpathcurveto{\pgfqpoint{1.020095in}{0.764502in}}{\pgfqpoint{1.012195in}{0.761230in}}{\pgfqpoint{1.006371in}{0.755406in}}%
\pgfpathcurveto{\pgfqpoint{1.000547in}{0.749582in}}{\pgfqpoint{0.997275in}{0.741682in}}{\pgfqpoint{0.997275in}{0.733446in}}%
\pgfpathcurveto{\pgfqpoint{0.997275in}{0.725209in}}{\pgfqpoint{1.000547in}{0.717309in}}{\pgfqpoint{1.006371in}{0.711486in}}%
\pgfpathcurveto{\pgfqpoint{1.012195in}{0.705662in}}{\pgfqpoint{1.020095in}{0.702389in}}{\pgfqpoint{1.028331in}{0.702389in}}%
\pgfpathclose%
\pgfusepath{stroke,fill}%
\end{pgfscope}%
\begin{pgfscope}%
\pgfpathrectangle{\pgfqpoint{0.894063in}{0.630000in}}{\pgfqpoint{6.713438in}{2.060556in}} %
\pgfusepath{clip}%
\pgfsetbuttcap%
\pgfsetroundjoin%
\definecolor{currentfill}{rgb}{0.000000,0.000000,1.000000}%
\pgfsetfillcolor{currentfill}%
\pgfsetlinewidth{1.003750pt}%
\definecolor{currentstroke}{rgb}{0.000000,0.000000,0.000000}%
\pgfsetstrokecolor{currentstroke}%
\pgfsetdash{}{0pt}%
\pgfpathmoveto{\pgfqpoint{5.324931in}{1.808272in}}%
\pgfpathcurveto{\pgfqpoint{5.333168in}{1.808272in}}{\pgfqpoint{5.341068in}{1.811544in}}{\pgfqpoint{5.346892in}{1.817368in}}%
\pgfpathcurveto{\pgfqpoint{5.352715in}{1.823192in}}{\pgfqpoint{5.355988in}{1.831092in}}{\pgfqpoint{5.355988in}{1.839328in}}%
\pgfpathcurveto{\pgfqpoint{5.355988in}{1.847565in}}{\pgfqpoint{5.352715in}{1.855465in}}{\pgfqpoint{5.346892in}{1.861289in}}%
\pgfpathcurveto{\pgfqpoint{5.341068in}{1.867112in}}{\pgfqpoint{5.333168in}{1.870385in}}{\pgfqpoint{5.324931in}{1.870385in}}%
\pgfpathcurveto{\pgfqpoint{5.316695in}{1.870385in}}{\pgfqpoint{5.308795in}{1.867112in}}{\pgfqpoint{5.302971in}{1.861289in}}%
\pgfpathcurveto{\pgfqpoint{5.297147in}{1.855465in}}{\pgfqpoint{5.293875in}{1.847565in}}{\pgfqpoint{5.293875in}{1.839328in}}%
\pgfpathcurveto{\pgfqpoint{5.293875in}{1.831092in}}{\pgfqpoint{5.297147in}{1.823192in}}{\pgfqpoint{5.302971in}{1.817368in}}%
\pgfpathcurveto{\pgfqpoint{5.308795in}{1.811544in}}{\pgfqpoint{5.316695in}{1.808272in}}{\pgfqpoint{5.324931in}{1.808272in}}%
\pgfpathclose%
\pgfusepath{stroke,fill}%
\end{pgfscope}%
\begin{pgfscope}%
\pgfpathrectangle{\pgfqpoint{0.894063in}{0.630000in}}{\pgfqpoint{6.713438in}{2.060556in}} %
\pgfusepath{clip}%
\pgfsetbuttcap%
\pgfsetroundjoin%
\definecolor{currentfill}{rgb}{0.000000,0.000000,1.000000}%
\pgfsetfillcolor{currentfill}%
\pgfsetlinewidth{1.003750pt}%
\definecolor{currentstroke}{rgb}{0.000000,0.000000,0.000000}%
\pgfsetstrokecolor{currentstroke}%
\pgfsetdash{}{0pt}%
\pgfpathmoveto{\pgfqpoint{7.338963in}{2.068231in}}%
\pgfpathcurveto{\pgfqpoint{7.347199in}{2.068231in}}{\pgfqpoint{7.355099in}{2.071504in}}{\pgfqpoint{7.360923in}{2.077328in}}%
\pgfpathcurveto{\pgfqpoint{7.366747in}{2.083152in}}{\pgfqpoint{7.370019in}{2.091052in}}{\pgfqpoint{7.370019in}{2.099288in}}%
\pgfpathcurveto{\pgfqpoint{7.370019in}{2.107524in}}{\pgfqpoint{7.366747in}{2.115424in}}{\pgfqpoint{7.360923in}{2.121248in}}%
\pgfpathcurveto{\pgfqpoint{7.355099in}{2.127072in}}{\pgfqpoint{7.347199in}{2.130344in}}{\pgfqpoint{7.338963in}{2.130344in}}%
\pgfpathcurveto{\pgfqpoint{7.330726in}{2.130344in}}{\pgfqpoint{7.322826in}{2.127072in}}{\pgfqpoint{7.317002in}{2.121248in}}%
\pgfpathcurveto{\pgfqpoint{7.311178in}{2.115424in}}{\pgfqpoint{7.307906in}{2.107524in}}{\pgfqpoint{7.307906in}{2.099288in}}%
\pgfpathcurveto{\pgfqpoint{7.307906in}{2.091052in}}{\pgfqpoint{7.311178in}{2.083152in}}{\pgfqpoint{7.317002in}{2.077328in}}%
\pgfpathcurveto{\pgfqpoint{7.322826in}{2.071504in}}{\pgfqpoint{7.330726in}{2.068231in}}{\pgfqpoint{7.338963in}{2.068231in}}%
\pgfpathclose%
\pgfusepath{stroke,fill}%
\end{pgfscope}%
\begin{pgfscope}%
\pgfpathrectangle{\pgfqpoint{0.894063in}{0.630000in}}{\pgfqpoint{6.713438in}{2.060556in}} %
\pgfusepath{clip}%
\pgfsetbuttcap%
\pgfsetroundjoin%
\definecolor{currentfill}{rgb}{0.000000,0.000000,1.000000}%
\pgfsetfillcolor{currentfill}%
\pgfsetlinewidth{1.003750pt}%
\definecolor{currentstroke}{rgb}{0.000000,0.000000,0.000000}%
\pgfsetstrokecolor{currentstroke}%
\pgfsetdash{}{0pt}%
\pgfpathmoveto{\pgfqpoint{7.204694in}{2.043134in}}%
\pgfpathcurveto{\pgfqpoint{7.212930in}{2.043134in}}{\pgfqpoint{7.220830in}{2.046406in}}{\pgfqpoint{7.226654in}{2.052230in}}%
\pgfpathcurveto{\pgfqpoint{7.232478in}{2.058054in}}{\pgfqpoint{7.235750in}{2.065954in}}{\pgfqpoint{7.235750in}{2.074190in}}%
\pgfpathcurveto{\pgfqpoint{7.235750in}{2.082427in}}{\pgfqpoint{7.232478in}{2.090327in}}{\pgfqpoint{7.226654in}{2.096151in}}%
\pgfpathcurveto{\pgfqpoint{7.220830in}{2.101975in}}{\pgfqpoint{7.212930in}{2.105247in}}{\pgfqpoint{7.204694in}{2.105247in}}%
\pgfpathcurveto{\pgfqpoint{7.196457in}{2.105247in}}{\pgfqpoint{7.188557in}{2.101975in}}{\pgfqpoint{7.182733in}{2.096151in}}%
\pgfpathcurveto{\pgfqpoint{7.176910in}{2.090327in}}{\pgfqpoint{7.173637in}{2.082427in}}{\pgfqpoint{7.173637in}{2.074190in}}%
\pgfpathcurveto{\pgfqpoint{7.173637in}{2.065954in}}{\pgfqpoint{7.176910in}{2.058054in}}{\pgfqpoint{7.182733in}{2.052230in}}%
\pgfpathcurveto{\pgfqpoint{7.188557in}{2.046406in}}{\pgfqpoint{7.196457in}{2.043134in}}{\pgfqpoint{7.204694in}{2.043134in}}%
\pgfpathclose%
\pgfusepath{stroke,fill}%
\end{pgfscope}%
\begin{pgfscope}%
\pgfpathrectangle{\pgfqpoint{0.894063in}{0.630000in}}{\pgfqpoint{6.713438in}{2.060556in}} %
\pgfusepath{clip}%
\pgfsetbuttcap%
\pgfsetroundjoin%
\definecolor{currentfill}{rgb}{0.000000,0.000000,1.000000}%
\pgfsetfillcolor{currentfill}%
\pgfsetlinewidth{1.003750pt}%
\definecolor{currentstroke}{rgb}{0.000000,0.000000,0.000000}%
\pgfsetstrokecolor{currentstroke}%
\pgfsetdash{}{0pt}%
\pgfpathmoveto{\pgfqpoint{6.264813in}{1.928549in}}%
\pgfpathcurveto{\pgfqpoint{6.273049in}{1.928549in}}{\pgfqpoint{6.280949in}{1.931822in}}{\pgfqpoint{6.286773in}{1.937646in}}%
\pgfpathcurveto{\pgfqpoint{6.292597in}{1.943470in}}{\pgfqpoint{6.295869in}{1.951370in}}{\pgfqpoint{6.295869in}{1.959606in}}%
\pgfpathcurveto{\pgfqpoint{6.295869in}{1.967842in}}{\pgfqpoint{6.292597in}{1.975742in}}{\pgfqpoint{6.286773in}{1.981566in}}%
\pgfpathcurveto{\pgfqpoint{6.280949in}{1.987390in}}{\pgfqpoint{6.273049in}{1.990662in}}{\pgfqpoint{6.264813in}{1.990662in}}%
\pgfpathcurveto{\pgfqpoint{6.256576in}{1.990662in}}{\pgfqpoint{6.248676in}{1.987390in}}{\pgfqpoint{6.242852in}{1.981566in}}%
\pgfpathcurveto{\pgfqpoint{6.237028in}{1.975742in}}{\pgfqpoint{6.233756in}{1.967842in}}{\pgfqpoint{6.233756in}{1.959606in}}%
\pgfpathcurveto{\pgfqpoint{6.233756in}{1.951370in}}{\pgfqpoint{6.237028in}{1.943470in}}{\pgfqpoint{6.242852in}{1.937646in}}%
\pgfpathcurveto{\pgfqpoint{6.248676in}{1.931822in}}{\pgfqpoint{6.256576in}{1.928549in}}{\pgfqpoint{6.264813in}{1.928549in}}%
\pgfpathclose%
\pgfusepath{stroke,fill}%
\end{pgfscope}%
\begin{pgfscope}%
\pgfpathrectangle{\pgfqpoint{0.894063in}{0.630000in}}{\pgfqpoint{6.713438in}{2.060556in}} %
\pgfusepath{clip}%
\pgfsetbuttcap%
\pgfsetroundjoin%
\definecolor{currentfill}{rgb}{0.000000,0.000000,1.000000}%
\pgfsetfillcolor{currentfill}%
\pgfsetlinewidth{1.003750pt}%
\definecolor{currentstroke}{rgb}{0.000000,0.000000,0.000000}%
\pgfsetstrokecolor{currentstroke}%
\pgfsetdash{}{0pt}%
\pgfpathmoveto{\pgfqpoint{7.473231in}{2.088042in}}%
\pgfpathcurveto{\pgfqpoint{7.481468in}{2.088042in}}{\pgfqpoint{7.489368in}{2.091315in}}{\pgfqpoint{7.495192in}{2.097138in}}%
\pgfpathcurveto{\pgfqpoint{7.501015in}{2.102962in}}{\pgfqpoint{7.504288in}{2.110862in}}{\pgfqpoint{7.504288in}{2.119099in}}%
\pgfpathcurveto{\pgfqpoint{7.504288in}{2.127335in}}{\pgfqpoint{7.501015in}{2.135235in}}{\pgfqpoint{7.495192in}{2.141059in}}%
\pgfpathcurveto{\pgfqpoint{7.489368in}{2.146883in}}{\pgfqpoint{7.481468in}{2.150155in}}{\pgfqpoint{7.473231in}{2.150155in}}%
\pgfpathcurveto{\pgfqpoint{7.464995in}{2.150155in}}{\pgfqpoint{7.457095in}{2.146883in}}{\pgfqpoint{7.451271in}{2.141059in}}%
\pgfpathcurveto{\pgfqpoint{7.445447in}{2.135235in}}{\pgfqpoint{7.442175in}{2.127335in}}{\pgfqpoint{7.442175in}{2.119099in}}%
\pgfpathcurveto{\pgfqpoint{7.442175in}{2.110862in}}{\pgfqpoint{7.445447in}{2.102962in}}{\pgfqpoint{7.451271in}{2.097138in}}%
\pgfpathcurveto{\pgfqpoint{7.457095in}{2.091315in}}{\pgfqpoint{7.464995in}{2.088042in}}{\pgfqpoint{7.473231in}{2.088042in}}%
\pgfpathclose%
\pgfusepath{stroke,fill}%
\end{pgfscope}%
\begin{pgfscope}%
\pgfpathrectangle{\pgfqpoint{0.894063in}{0.630000in}}{\pgfqpoint{6.713438in}{2.060556in}} %
\pgfusepath{clip}%
\pgfsetbuttcap%
\pgfsetroundjoin%
\definecolor{currentfill}{rgb}{0.000000,0.000000,1.000000}%
\pgfsetfillcolor{currentfill}%
\pgfsetlinewidth{1.003750pt}%
\definecolor{currentstroke}{rgb}{0.000000,0.000000,0.000000}%
\pgfsetstrokecolor{currentstroke}%
\pgfsetdash{}{0pt}%
\pgfpathmoveto{\pgfqpoint{5.056394in}{1.768003in}}%
\pgfpathcurveto{\pgfqpoint{5.064630in}{1.768003in}}{\pgfqpoint{5.072530in}{1.771275in}}{\pgfqpoint{5.078354in}{1.777099in}}%
\pgfpathcurveto{\pgfqpoint{5.084178in}{1.782923in}}{\pgfqpoint{5.087450in}{1.790823in}}{\pgfqpoint{5.087450in}{1.799059in}}%
\pgfpathcurveto{\pgfqpoint{5.087450in}{1.807295in}}{\pgfqpoint{5.084178in}{1.815195in}}{\pgfqpoint{5.078354in}{1.821019in}}%
\pgfpathcurveto{\pgfqpoint{5.072530in}{1.826843in}}{\pgfqpoint{5.064630in}{1.830116in}}{\pgfqpoint{5.056394in}{1.830116in}}%
\pgfpathcurveto{\pgfqpoint{5.048157in}{1.830116in}}{\pgfqpoint{5.040257in}{1.826843in}}{\pgfqpoint{5.034433in}{1.821019in}}%
\pgfpathcurveto{\pgfqpoint{5.028610in}{1.815195in}}{\pgfqpoint{5.025337in}{1.807295in}}{\pgfqpoint{5.025337in}{1.799059in}}%
\pgfpathcurveto{\pgfqpoint{5.025337in}{1.790823in}}{\pgfqpoint{5.028610in}{1.782923in}}{\pgfqpoint{5.034433in}{1.777099in}}%
\pgfpathcurveto{\pgfqpoint{5.040257in}{1.771275in}}{\pgfqpoint{5.048157in}{1.768003in}}{\pgfqpoint{5.056394in}{1.768003in}}%
\pgfpathclose%
\pgfusepath{stroke,fill}%
\end{pgfscope}%
\begin{pgfscope}%
\pgfpathrectangle{\pgfqpoint{0.894063in}{0.630000in}}{\pgfqpoint{6.713438in}{2.060556in}} %
\pgfusepath{clip}%
\pgfsetbuttcap%
\pgfsetroundjoin%
\definecolor{currentfill}{rgb}{0.000000,0.000000,1.000000}%
\pgfsetfillcolor{currentfill}%
\pgfsetlinewidth{1.003750pt}%
\definecolor{currentstroke}{rgb}{0.000000,0.000000,0.000000}%
\pgfsetstrokecolor{currentstroke}%
\pgfsetdash{}{0pt}%
\pgfpathmoveto{\pgfqpoint{2.908094in}{1.452125in}}%
\pgfpathcurveto{\pgfqpoint{2.916330in}{1.452125in}}{\pgfqpoint{2.924230in}{1.455398in}}{\pgfqpoint{2.930054in}{1.461222in}}%
\pgfpathcurveto{\pgfqpoint{2.935878in}{1.467046in}}{\pgfqpoint{2.939150in}{1.474946in}}{\pgfqpoint{2.939150in}{1.483182in}}%
\pgfpathcurveto{\pgfqpoint{2.939150in}{1.491418in}}{\pgfqpoint{2.935878in}{1.499318in}}{\pgfqpoint{2.930054in}{1.505142in}}%
\pgfpathcurveto{\pgfqpoint{2.924230in}{1.510966in}}{\pgfqpoint{2.916330in}{1.514238in}}{\pgfqpoint{2.908094in}{1.514238in}}%
\pgfpathcurveto{\pgfqpoint{2.899857in}{1.514238in}}{\pgfqpoint{2.891957in}{1.510966in}}{\pgfqpoint{2.886133in}{1.505142in}}%
\pgfpathcurveto{\pgfqpoint{2.880310in}{1.499318in}}{\pgfqpoint{2.877037in}{1.491418in}}{\pgfqpoint{2.877037in}{1.483182in}}%
\pgfpathcurveto{\pgfqpoint{2.877037in}{1.474946in}}{\pgfqpoint{2.880310in}{1.467046in}}{\pgfqpoint{2.886133in}{1.461222in}}%
\pgfpathcurveto{\pgfqpoint{2.891957in}{1.455398in}}{\pgfqpoint{2.899857in}{1.452125in}}{\pgfqpoint{2.908094in}{1.452125in}}%
\pgfpathclose%
\pgfusepath{stroke,fill}%
\end{pgfscope}%
\begin{pgfscope}%
\pgfpathrectangle{\pgfqpoint{0.894063in}{0.630000in}}{\pgfqpoint{6.713438in}{2.060556in}} %
\pgfusepath{clip}%
\pgfsetbuttcap%
\pgfsetroundjoin%
\definecolor{currentfill}{rgb}{0.000000,0.000000,1.000000}%
\pgfsetfillcolor{currentfill}%
\pgfsetlinewidth{1.003750pt}%
\definecolor{currentstroke}{rgb}{0.000000,0.000000,0.000000}%
\pgfsetstrokecolor{currentstroke}%
\pgfsetdash{}{0pt}%
\pgfpathmoveto{\pgfqpoint{3.445169in}{1.513659in}}%
\pgfpathcurveto{\pgfqpoint{3.453405in}{1.513659in}}{\pgfqpoint{3.461305in}{1.516932in}}{\pgfqpoint{3.467129in}{1.522756in}}%
\pgfpathcurveto{\pgfqpoint{3.472953in}{1.528580in}}{\pgfqpoint{3.476225in}{1.536480in}}{\pgfqpoint{3.476225in}{1.544716in}}%
\pgfpathcurveto{\pgfqpoint{3.476225in}{1.552952in}}{\pgfqpoint{3.472953in}{1.560852in}}{\pgfqpoint{3.467129in}{1.566676in}}%
\pgfpathcurveto{\pgfqpoint{3.461305in}{1.572500in}}{\pgfqpoint{3.453405in}{1.575772in}}{\pgfqpoint{3.445169in}{1.575772in}}%
\pgfpathcurveto{\pgfqpoint{3.436932in}{1.575772in}}{\pgfqpoint{3.429032in}{1.572500in}}{\pgfqpoint{3.423208in}{1.566676in}}%
\pgfpathcurveto{\pgfqpoint{3.417385in}{1.560852in}}{\pgfqpoint{3.414112in}{1.552952in}}{\pgfqpoint{3.414112in}{1.544716in}}%
\pgfpathcurveto{\pgfqpoint{3.414112in}{1.536480in}}{\pgfqpoint{3.417385in}{1.528580in}}{\pgfqpoint{3.423208in}{1.522756in}}%
\pgfpathcurveto{\pgfqpoint{3.429032in}{1.516932in}}{\pgfqpoint{3.436932in}{1.513659in}}{\pgfqpoint{3.445169in}{1.513659in}}%
\pgfpathclose%
\pgfusepath{stroke,fill}%
\end{pgfscope}%
\begin{pgfscope}%
\pgfpathrectangle{\pgfqpoint{0.894063in}{0.630000in}}{\pgfqpoint{6.713438in}{2.060556in}} %
\pgfusepath{clip}%
\pgfsetbuttcap%
\pgfsetroundjoin%
\definecolor{currentfill}{rgb}{0.000000,0.000000,1.000000}%
\pgfsetfillcolor{currentfill}%
\pgfsetlinewidth{1.003750pt}%
\definecolor{currentstroke}{rgb}{0.000000,0.000000,0.000000}%
\pgfsetstrokecolor{currentstroke}%
\pgfsetdash{}{0pt}%
\pgfpathmoveto{\pgfqpoint{4.116513in}{1.630717in}}%
\pgfpathcurveto{\pgfqpoint{4.124749in}{1.630717in}}{\pgfqpoint{4.132649in}{1.633989in}}{\pgfqpoint{4.138473in}{1.639813in}}%
\pgfpathcurveto{\pgfqpoint{4.144297in}{1.645637in}}{\pgfqpoint{4.147569in}{1.653537in}}{\pgfqpoint{4.147569in}{1.661773in}}%
\pgfpathcurveto{\pgfqpoint{4.147569in}{1.670009in}}{\pgfqpoint{4.144297in}{1.677909in}}{\pgfqpoint{4.138473in}{1.683733in}}%
\pgfpathcurveto{\pgfqpoint{4.132649in}{1.689557in}}{\pgfqpoint{4.124749in}{1.692830in}}{\pgfqpoint{4.116513in}{1.692830in}}%
\pgfpathcurveto{\pgfqpoint{4.108276in}{1.692830in}}{\pgfqpoint{4.100376in}{1.689557in}}{\pgfqpoint{4.094552in}{1.683733in}}%
\pgfpathcurveto{\pgfqpoint{4.088728in}{1.677909in}}{\pgfqpoint{4.085456in}{1.670009in}}{\pgfqpoint{4.085456in}{1.661773in}}%
\pgfpathcurveto{\pgfqpoint{4.085456in}{1.653537in}}{\pgfqpoint{4.088728in}{1.645637in}}{\pgfqpoint{4.094552in}{1.639813in}}%
\pgfpathcurveto{\pgfqpoint{4.100376in}{1.633989in}}{\pgfqpoint{4.108276in}{1.630717in}}{\pgfqpoint{4.116513in}{1.630717in}}%
\pgfpathclose%
\pgfusepath{stroke,fill}%
\end{pgfscope}%
\begin{pgfscope}%
\pgfpathrectangle{\pgfqpoint{0.894063in}{0.630000in}}{\pgfqpoint{6.713438in}{2.060556in}} %
\pgfusepath{clip}%
\pgfsetbuttcap%
\pgfsetroundjoin%
\definecolor{currentfill}{rgb}{0.000000,0.000000,1.000000}%
\pgfsetfillcolor{currentfill}%
\pgfsetlinewidth{1.003750pt}%
\definecolor{currentstroke}{rgb}{0.000000,0.000000,0.000000}%
\pgfsetstrokecolor{currentstroke}%
\pgfsetdash{}{0pt}%
\pgfpathmoveto{\pgfqpoint{1.431138in}{1.256690in}}%
\pgfpathcurveto{\pgfqpoint{1.439374in}{1.256690in}}{\pgfqpoint{1.447274in}{1.259963in}}{\pgfqpoint{1.453098in}{1.265787in}}%
\pgfpathcurveto{\pgfqpoint{1.458922in}{1.271611in}}{\pgfqpoint{1.462194in}{1.279511in}}{\pgfqpoint{1.462194in}{1.287747in}}%
\pgfpathcurveto{\pgfqpoint{1.462194in}{1.295983in}}{\pgfqpoint{1.458922in}{1.303883in}}{\pgfqpoint{1.453098in}{1.309707in}}%
\pgfpathcurveto{\pgfqpoint{1.447274in}{1.315531in}}{\pgfqpoint{1.439374in}{1.318803in}}{\pgfqpoint{1.431138in}{1.318803in}}%
\pgfpathcurveto{\pgfqpoint{1.422901in}{1.318803in}}{\pgfqpoint{1.415001in}{1.315531in}}{\pgfqpoint{1.409177in}{1.309707in}}%
\pgfpathcurveto{\pgfqpoint{1.403353in}{1.303883in}}{\pgfqpoint{1.400081in}{1.295983in}}{\pgfqpoint{1.400081in}{1.287747in}}%
\pgfpathcurveto{\pgfqpoint{1.400081in}{1.279511in}}{\pgfqpoint{1.403353in}{1.271611in}}{\pgfqpoint{1.409177in}{1.265787in}}%
\pgfpathcurveto{\pgfqpoint{1.415001in}{1.259963in}}{\pgfqpoint{1.422901in}{1.256690in}}{\pgfqpoint{1.431138in}{1.256690in}}%
\pgfpathclose%
\pgfusepath{stroke,fill}%
\end{pgfscope}%
\begin{pgfscope}%
\pgfpathrectangle{\pgfqpoint{0.894063in}{0.630000in}}{\pgfqpoint{6.713438in}{2.060556in}} %
\pgfusepath{clip}%
\pgfsetbuttcap%
\pgfsetroundjoin%
\definecolor{currentfill}{rgb}{0.000000,0.000000,1.000000}%
\pgfsetfillcolor{currentfill}%
\pgfsetlinewidth{1.003750pt}%
\definecolor{currentstroke}{rgb}{0.000000,0.000000,0.000000}%
\pgfsetstrokecolor{currentstroke}%
\pgfsetdash{}{0pt}%
\pgfpathmoveto{\pgfqpoint{2.773825in}{1.437054in}}%
\pgfpathcurveto{\pgfqpoint{2.782061in}{1.437054in}}{\pgfqpoint{2.789961in}{1.440326in}}{\pgfqpoint{2.795785in}{1.446150in}}%
\pgfpathcurveto{\pgfqpoint{2.801609in}{1.451974in}}{\pgfqpoint{2.804881in}{1.459874in}}{\pgfqpoint{2.804881in}{1.468110in}}%
\pgfpathcurveto{\pgfqpoint{2.804881in}{1.476347in}}{\pgfqpoint{2.801609in}{1.484247in}}{\pgfqpoint{2.795785in}{1.490071in}}%
\pgfpathcurveto{\pgfqpoint{2.789961in}{1.495895in}}{\pgfqpoint{2.782061in}{1.499167in}}{\pgfqpoint{2.773825in}{1.499167in}}%
\pgfpathcurveto{\pgfqpoint{2.765589in}{1.499167in}}{\pgfqpoint{2.757689in}{1.495895in}}{\pgfqpoint{2.751865in}{1.490071in}}%
\pgfpathcurveto{\pgfqpoint{2.746041in}{1.484247in}}{\pgfqpoint{2.742769in}{1.476347in}}{\pgfqpoint{2.742769in}{1.468110in}}%
\pgfpathcurveto{\pgfqpoint{2.742769in}{1.459874in}}{\pgfqpoint{2.746041in}{1.451974in}}{\pgfqpoint{2.751865in}{1.446150in}}%
\pgfpathcurveto{\pgfqpoint{2.757689in}{1.440326in}}{\pgfqpoint{2.765589in}{1.437054in}}{\pgfqpoint{2.773825in}{1.437054in}}%
\pgfpathclose%
\pgfusepath{stroke,fill}%
\end{pgfscope}%
\begin{pgfscope}%
\pgfpathrectangle{\pgfqpoint{0.894063in}{0.630000in}}{\pgfqpoint{6.713438in}{2.060556in}} %
\pgfusepath{clip}%
\pgfsetbuttcap%
\pgfsetroundjoin%
\definecolor{currentfill}{rgb}{0.000000,0.000000,1.000000}%
\pgfsetfillcolor{currentfill}%
\pgfsetlinewidth{1.003750pt}%
\definecolor{currentstroke}{rgb}{0.000000,0.000000,0.000000}%
\pgfsetstrokecolor{currentstroke}%
\pgfsetdash{}{0pt}%
\pgfpathmoveto{\pgfqpoint{1.565406in}{1.267547in}}%
\pgfpathcurveto{\pgfqpoint{1.573643in}{1.267547in}}{\pgfqpoint{1.581543in}{1.270819in}}{\pgfqpoint{1.587367in}{1.276643in}}%
\pgfpathcurveto{\pgfqpoint{1.593190in}{1.282467in}}{\pgfqpoint{1.596463in}{1.290367in}}{\pgfqpoint{1.596463in}{1.298603in}}%
\pgfpathcurveto{\pgfqpoint{1.596463in}{1.306839in}}{\pgfqpoint{1.593190in}{1.314740in}}{\pgfqpoint{1.587367in}{1.320563in}}%
\pgfpathcurveto{\pgfqpoint{1.581543in}{1.326387in}}{\pgfqpoint{1.573643in}{1.329660in}}{\pgfqpoint{1.565406in}{1.329660in}}%
\pgfpathcurveto{\pgfqpoint{1.557170in}{1.329660in}}{\pgfqpoint{1.549270in}{1.326387in}}{\pgfqpoint{1.543446in}{1.320563in}}%
\pgfpathcurveto{\pgfqpoint{1.537622in}{1.314740in}}{\pgfqpoint{1.534350in}{1.306839in}}{\pgfqpoint{1.534350in}{1.298603in}}%
\pgfpathcurveto{\pgfqpoint{1.534350in}{1.290367in}}{\pgfqpoint{1.537622in}{1.282467in}}{\pgfqpoint{1.543446in}{1.276643in}}%
\pgfpathcurveto{\pgfqpoint{1.549270in}{1.270819in}}{\pgfqpoint{1.557170in}{1.267547in}}{\pgfqpoint{1.565406in}{1.267547in}}%
\pgfpathclose%
\pgfusepath{stroke,fill}%
\end{pgfscope}%
\begin{pgfscope}%
\pgfpathrectangle{\pgfqpoint{0.894063in}{0.630000in}}{\pgfqpoint{6.713438in}{2.060556in}} %
\pgfusepath{clip}%
\pgfsetbuttcap%
\pgfsetroundjoin%
\definecolor{currentfill}{rgb}{0.000000,0.000000,1.000000}%
\pgfsetfillcolor{currentfill}%
\pgfsetlinewidth{1.003750pt}%
\definecolor{currentstroke}{rgb}{0.000000,0.000000,0.000000}%
\pgfsetstrokecolor{currentstroke}%
\pgfsetdash{}{0pt}%
\pgfpathmoveto{\pgfqpoint{4.250781in}{1.651711in}}%
\pgfpathcurveto{\pgfqpoint{4.259018in}{1.651711in}}{\pgfqpoint{4.266918in}{1.654983in}}{\pgfqpoint{4.272742in}{1.660807in}}%
\pgfpathcurveto{\pgfqpoint{4.278565in}{1.666631in}}{\pgfqpoint{4.281838in}{1.674531in}}{\pgfqpoint{4.281838in}{1.682767in}}%
\pgfpathcurveto{\pgfqpoint{4.281838in}{1.691004in}}{\pgfqpoint{4.278565in}{1.698904in}}{\pgfqpoint{4.272742in}{1.704728in}}%
\pgfpathcurveto{\pgfqpoint{4.266918in}{1.710551in}}{\pgfqpoint{4.259018in}{1.713824in}}{\pgfqpoint{4.250781in}{1.713824in}}%
\pgfpathcurveto{\pgfqpoint{4.242545in}{1.713824in}}{\pgfqpoint{4.234645in}{1.710551in}}{\pgfqpoint{4.228821in}{1.704728in}}%
\pgfpathcurveto{\pgfqpoint{4.222997in}{1.698904in}}{\pgfqpoint{4.219725in}{1.691004in}}{\pgfqpoint{4.219725in}{1.682767in}}%
\pgfpathcurveto{\pgfqpoint{4.219725in}{1.674531in}}{\pgfqpoint{4.222997in}{1.666631in}}{\pgfqpoint{4.228821in}{1.660807in}}%
\pgfpathcurveto{\pgfqpoint{4.234645in}{1.654983in}}{\pgfqpoint{4.242545in}{1.651711in}}{\pgfqpoint{4.250781in}{1.651711in}}%
\pgfpathclose%
\pgfusepath{stroke,fill}%
\end{pgfscope}%
\begin{pgfscope}%
\pgfpathrectangle{\pgfqpoint{0.894063in}{0.630000in}}{\pgfqpoint{6.713438in}{2.060556in}} %
\pgfusepath{clip}%
\pgfsetbuttcap%
\pgfsetroundjoin%
\definecolor{currentfill}{rgb}{0.000000,0.000000,1.000000}%
\pgfsetfillcolor{currentfill}%
\pgfsetlinewidth{1.003750pt}%
\definecolor{currentstroke}{rgb}{0.000000,0.000000,0.000000}%
\pgfsetstrokecolor{currentstroke}%
\pgfsetdash{}{0pt}%
\pgfpathmoveto{\pgfqpoint{3.847975in}{1.582335in}}%
\pgfpathcurveto{\pgfqpoint{3.856211in}{1.582335in}}{\pgfqpoint{3.864111in}{1.585607in}}{\pgfqpoint{3.869935in}{1.591431in}}%
\pgfpathcurveto{\pgfqpoint{3.875759in}{1.597255in}}{\pgfqpoint{3.879031in}{1.605155in}}{\pgfqpoint{3.879031in}{1.613391in}}%
\pgfpathcurveto{\pgfqpoint{3.879031in}{1.621628in}}{\pgfqpoint{3.875759in}{1.629528in}}{\pgfqpoint{3.869935in}{1.635352in}}%
\pgfpathcurveto{\pgfqpoint{3.864111in}{1.641175in}}{\pgfqpoint{3.856211in}{1.644448in}}{\pgfqpoint{3.847975in}{1.644448in}}%
\pgfpathcurveto{\pgfqpoint{3.839739in}{1.644448in}}{\pgfqpoint{3.831839in}{1.641175in}}{\pgfqpoint{3.826015in}{1.635352in}}%
\pgfpathcurveto{\pgfqpoint{3.820191in}{1.629528in}}{\pgfqpoint{3.816919in}{1.621628in}}{\pgfqpoint{3.816919in}{1.613391in}}%
\pgfpathcurveto{\pgfqpoint{3.816919in}{1.605155in}}{\pgfqpoint{3.820191in}{1.597255in}}{\pgfqpoint{3.826015in}{1.591431in}}%
\pgfpathcurveto{\pgfqpoint{3.831839in}{1.585607in}}{\pgfqpoint{3.839739in}{1.582335in}}{\pgfqpoint{3.847975in}{1.582335in}}%
\pgfpathclose%
\pgfusepath{stroke,fill}%
\end{pgfscope}%
\begin{pgfscope}%
\pgfpathrectangle{\pgfqpoint{0.894063in}{0.630000in}}{\pgfqpoint{6.713438in}{2.060556in}} %
\pgfusepath{clip}%
\pgfsetbuttcap%
\pgfsetroundjoin%
\definecolor{currentfill}{rgb}{0.000000,0.000000,1.000000}%
\pgfsetfillcolor{currentfill}%
\pgfsetlinewidth{1.003750pt}%
\definecolor{currentstroke}{rgb}{0.000000,0.000000,0.000000}%
\pgfsetstrokecolor{currentstroke}%
\pgfsetdash{}{0pt}%
\pgfpathmoveto{\pgfqpoint{7.607500in}{2.102007in}}%
\pgfpathcurveto{\pgfqpoint{7.615736in}{2.102007in}}{\pgfqpoint{7.623636in}{2.105279in}}{\pgfqpoint{7.629460in}{2.111103in}}%
\pgfpathcurveto{\pgfqpoint{7.635284in}{2.116927in}}{\pgfqpoint{7.638556in}{2.124827in}}{\pgfqpoint{7.638556in}{2.133063in}}%
\pgfpathcurveto{\pgfqpoint{7.638556in}{2.141300in}}{\pgfqpoint{7.635284in}{2.149200in}}{\pgfqpoint{7.629460in}{2.155024in}}%
\pgfpathcurveto{\pgfqpoint{7.623636in}{2.160848in}}{\pgfqpoint{7.615736in}{2.164120in}}{\pgfqpoint{7.607500in}{2.164120in}}%
\pgfpathcurveto{\pgfqpoint{7.599264in}{2.164120in}}{\pgfqpoint{7.591364in}{2.160848in}}{\pgfqpoint{7.585540in}{2.155024in}}%
\pgfpathcurveto{\pgfqpoint{7.579716in}{2.149200in}}{\pgfqpoint{7.576444in}{2.141300in}}{\pgfqpoint{7.576444in}{2.133063in}}%
\pgfpathcurveto{\pgfqpoint{7.576444in}{2.124827in}}{\pgfqpoint{7.579716in}{2.116927in}}{\pgfqpoint{7.585540in}{2.111103in}}%
\pgfpathcurveto{\pgfqpoint{7.591364in}{2.105279in}}{\pgfqpoint{7.599264in}{2.102007in}}{\pgfqpoint{7.607500in}{2.102007in}}%
\pgfpathclose%
\pgfusepath{stroke,fill}%
\end{pgfscope}%
\begin{pgfscope}%
\pgfpathrectangle{\pgfqpoint{0.894063in}{0.630000in}}{\pgfqpoint{6.713438in}{2.060556in}} %
\pgfusepath{clip}%
\pgfsetbuttcap%
\pgfsetroundjoin%
\definecolor{currentfill}{rgb}{0.000000,0.000000,1.000000}%
\pgfsetfillcolor{currentfill}%
\pgfsetlinewidth{1.003750pt}%
\definecolor{currentstroke}{rgb}{0.000000,0.000000,0.000000}%
\pgfsetstrokecolor{currentstroke}%
\pgfsetdash{}{0pt}%
\pgfpathmoveto{\pgfqpoint{4.385050in}{1.671274in}}%
\pgfpathcurveto{\pgfqpoint{4.393286in}{1.671274in}}{\pgfqpoint{4.401186in}{1.674547in}}{\pgfqpoint{4.407010in}{1.680371in}}%
\pgfpathcurveto{\pgfqpoint{4.412834in}{1.686194in}}{\pgfqpoint{4.416106in}{1.694094in}}{\pgfqpoint{4.416106in}{1.702331in}}%
\pgfpathcurveto{\pgfqpoint{4.416106in}{1.710567in}}{\pgfqpoint{4.412834in}{1.718467in}}{\pgfqpoint{4.407010in}{1.724291in}}%
\pgfpathcurveto{\pgfqpoint{4.401186in}{1.730115in}}{\pgfqpoint{4.393286in}{1.733387in}}{\pgfqpoint{4.385050in}{1.733387in}}%
\pgfpathcurveto{\pgfqpoint{4.376814in}{1.733387in}}{\pgfqpoint{4.368914in}{1.730115in}}{\pgfqpoint{4.363090in}{1.724291in}}%
\pgfpathcurveto{\pgfqpoint{4.357266in}{1.718467in}}{\pgfqpoint{4.353994in}{1.710567in}}{\pgfqpoint{4.353994in}{1.702331in}}%
\pgfpathcurveto{\pgfqpoint{4.353994in}{1.694094in}}{\pgfqpoint{4.357266in}{1.686194in}}{\pgfqpoint{4.363090in}{1.680371in}}%
\pgfpathcurveto{\pgfqpoint{4.368914in}{1.674547in}}{\pgfqpoint{4.376814in}{1.671274in}}{\pgfqpoint{4.385050in}{1.671274in}}%
\pgfpathclose%
\pgfusepath{stroke,fill}%
\end{pgfscope}%
\begin{pgfscope}%
\pgfpathrectangle{\pgfqpoint{0.894063in}{0.630000in}}{\pgfqpoint{6.713438in}{2.060556in}} %
\pgfusepath{clip}%
\pgfsetbuttcap%
\pgfsetroundjoin%
\definecolor{currentfill}{rgb}{0.000000,0.000000,1.000000}%
\pgfsetfillcolor{currentfill}%
\pgfsetlinewidth{1.003750pt}%
\definecolor{currentstroke}{rgb}{0.000000,0.000000,0.000000}%
\pgfsetstrokecolor{currentstroke}%
\pgfsetdash{}{0pt}%
\pgfpathmoveto{\pgfqpoint{6.533350in}{1.962543in}}%
\pgfpathcurveto{\pgfqpoint{6.541586in}{1.962543in}}{\pgfqpoint{6.549486in}{1.965815in}}{\pgfqpoint{6.555310in}{1.971639in}}%
\pgfpathcurveto{\pgfqpoint{6.561134in}{1.977463in}}{\pgfqpoint{6.564406in}{1.985363in}}{\pgfqpoint{6.564406in}{1.993599in}}%
\pgfpathcurveto{\pgfqpoint{6.564406in}{2.001835in}}{\pgfqpoint{6.561134in}{2.009735in}}{\pgfqpoint{6.555310in}{2.015559in}}%
\pgfpathcurveto{\pgfqpoint{6.549486in}{2.021383in}}{\pgfqpoint{6.541586in}{2.024656in}}{\pgfqpoint{6.533350in}{2.024656in}}%
\pgfpathcurveto{\pgfqpoint{6.525114in}{2.024656in}}{\pgfqpoint{6.517214in}{2.021383in}}{\pgfqpoint{6.511390in}{2.015559in}}%
\pgfpathcurveto{\pgfqpoint{6.505566in}{2.009735in}}{\pgfqpoint{6.502294in}{2.001835in}}{\pgfqpoint{6.502294in}{1.993599in}}%
\pgfpathcurveto{\pgfqpoint{6.502294in}{1.985363in}}{\pgfqpoint{6.505566in}{1.977463in}}{\pgfqpoint{6.511390in}{1.971639in}}%
\pgfpathcurveto{\pgfqpoint{6.517214in}{1.965815in}}{\pgfqpoint{6.525114in}{1.962543in}}{\pgfqpoint{6.533350in}{1.962543in}}%
\pgfpathclose%
\pgfusepath{stroke,fill}%
\end{pgfscope}%
\begin{pgfscope}%
\pgfpathrectangle{\pgfqpoint{0.894063in}{0.630000in}}{\pgfqpoint{6.713438in}{2.060556in}} %
\pgfusepath{clip}%
\pgfsetbuttcap%
\pgfsetroundjoin%
\definecolor{currentfill}{rgb}{0.000000,0.000000,1.000000}%
\pgfsetfillcolor{currentfill}%
\pgfsetlinewidth{1.003750pt}%
\definecolor{currentstroke}{rgb}{0.000000,0.000000,0.000000}%
\pgfsetstrokecolor{currentstroke}%
\pgfsetdash{}{0pt}%
\pgfpathmoveto{\pgfqpoint{1.296869in}{1.243579in}}%
\pgfpathcurveto{\pgfqpoint{1.305105in}{1.243579in}}{\pgfqpoint{1.313005in}{1.246852in}}{\pgfqpoint{1.318829in}{1.252676in}}%
\pgfpathcurveto{\pgfqpoint{1.324653in}{1.258500in}}{\pgfqpoint{1.327925in}{1.266400in}}{\pgfqpoint{1.327925in}{1.274636in}}%
\pgfpathcurveto{\pgfqpoint{1.327925in}{1.282872in}}{\pgfqpoint{1.324653in}{1.290772in}}{\pgfqpoint{1.318829in}{1.296596in}}%
\pgfpathcurveto{\pgfqpoint{1.313005in}{1.302420in}}{\pgfqpoint{1.305105in}{1.305692in}}{\pgfqpoint{1.296869in}{1.305692in}}%
\pgfpathcurveto{\pgfqpoint{1.288632in}{1.305692in}}{\pgfqpoint{1.280732in}{1.302420in}}{\pgfqpoint{1.274908in}{1.296596in}}%
\pgfpathcurveto{\pgfqpoint{1.269085in}{1.290772in}}{\pgfqpoint{1.265812in}{1.282872in}}{\pgfqpoint{1.265812in}{1.274636in}}%
\pgfpathcurveto{\pgfqpoint{1.265812in}{1.266400in}}{\pgfqpoint{1.269085in}{1.258500in}}{\pgfqpoint{1.274908in}{1.252676in}}%
\pgfpathcurveto{\pgfqpoint{1.280732in}{1.246852in}}{\pgfqpoint{1.288632in}{1.243579in}}{\pgfqpoint{1.296869in}{1.243579in}}%
\pgfpathclose%
\pgfusepath{stroke,fill}%
\end{pgfscope}%
\begin{pgfscope}%
\pgfpathrectangle{\pgfqpoint{0.894063in}{0.630000in}}{\pgfqpoint{6.713438in}{2.060556in}} %
\pgfusepath{clip}%
\pgfsetbuttcap%
\pgfsetroundjoin%
\definecolor{currentfill}{rgb}{0.000000,0.000000,1.000000}%
\pgfsetfillcolor{currentfill}%
\pgfsetlinewidth{1.003750pt}%
\definecolor{currentstroke}{rgb}{0.000000,0.000000,0.000000}%
\pgfsetstrokecolor{currentstroke}%
\pgfsetdash{}{0pt}%
\pgfpathmoveto{\pgfqpoint{4.519319in}{1.694005in}}%
\pgfpathcurveto{\pgfqpoint{4.527555in}{1.694005in}}{\pgfqpoint{4.535455in}{1.697277in}}{\pgfqpoint{4.541279in}{1.703101in}}%
\pgfpathcurveto{\pgfqpoint{4.547103in}{1.708925in}}{\pgfqpoint{4.550375in}{1.716825in}}{\pgfqpoint{4.550375in}{1.725062in}}%
\pgfpathcurveto{\pgfqpoint{4.550375in}{1.733298in}}{\pgfqpoint{4.547103in}{1.741198in}}{\pgfqpoint{4.541279in}{1.747022in}}%
\pgfpathcurveto{\pgfqpoint{4.535455in}{1.752846in}}{\pgfqpoint{4.527555in}{1.756118in}}{\pgfqpoint{4.519319in}{1.756118in}}%
\pgfpathcurveto{\pgfqpoint{4.511082in}{1.756118in}}{\pgfqpoint{4.503182in}{1.752846in}}{\pgfqpoint{4.497358in}{1.747022in}}%
\pgfpathcurveto{\pgfqpoint{4.491535in}{1.741198in}}{\pgfqpoint{4.488262in}{1.733298in}}{\pgfqpoint{4.488262in}{1.725062in}}%
\pgfpathcurveto{\pgfqpoint{4.488262in}{1.716825in}}{\pgfqpoint{4.491535in}{1.708925in}}{\pgfqpoint{4.497358in}{1.703101in}}%
\pgfpathcurveto{\pgfqpoint{4.503182in}{1.697277in}}{\pgfqpoint{4.511082in}{1.694005in}}{\pgfqpoint{4.519319in}{1.694005in}}%
\pgfpathclose%
\pgfusepath{stroke,fill}%
\end{pgfscope}%
\begin{pgfscope}%
\pgfpathrectangle{\pgfqpoint{0.894063in}{0.630000in}}{\pgfqpoint{6.713438in}{2.060556in}} %
\pgfusepath{clip}%
\pgfsetbuttcap%
\pgfsetroundjoin%
\definecolor{currentfill}{rgb}{0.000000,0.000000,1.000000}%
\pgfsetfillcolor{currentfill}%
\pgfsetlinewidth{1.003750pt}%
\definecolor{currentstroke}{rgb}{0.000000,0.000000,0.000000}%
\pgfsetstrokecolor{currentstroke}%
\pgfsetdash{}{0pt}%
\pgfpathmoveto{\pgfqpoint{2.505288in}{1.385399in}}%
\pgfpathcurveto{\pgfqpoint{2.513524in}{1.385399in}}{\pgfqpoint{2.521424in}{1.388671in}}{\pgfqpoint{2.527248in}{1.394495in}}%
\pgfpathcurveto{\pgfqpoint{2.533072in}{1.400319in}}{\pgfqpoint{2.536344in}{1.408219in}}{\pgfqpoint{2.536344in}{1.416455in}}%
\pgfpathcurveto{\pgfqpoint{2.536344in}{1.424691in}}{\pgfqpoint{2.533072in}{1.432592in}}{\pgfqpoint{2.527248in}{1.438415in}}%
\pgfpathcurveto{\pgfqpoint{2.521424in}{1.444239in}}{\pgfqpoint{2.513524in}{1.447512in}}{\pgfqpoint{2.505288in}{1.447512in}}%
\pgfpathcurveto{\pgfqpoint{2.497051in}{1.447512in}}{\pgfqpoint{2.489151in}{1.444239in}}{\pgfqpoint{2.483327in}{1.438415in}}%
\pgfpathcurveto{\pgfqpoint{2.477503in}{1.432592in}}{\pgfqpoint{2.474231in}{1.424691in}}{\pgfqpoint{2.474231in}{1.416455in}}%
\pgfpathcurveto{\pgfqpoint{2.474231in}{1.408219in}}{\pgfqpoint{2.477503in}{1.400319in}}{\pgfqpoint{2.483327in}{1.394495in}}%
\pgfpathcurveto{\pgfqpoint{2.489151in}{1.388671in}}{\pgfqpoint{2.497051in}{1.385399in}}{\pgfqpoint{2.505288in}{1.385399in}}%
\pgfpathclose%
\pgfusepath{stroke,fill}%
\end{pgfscope}%
\begin{pgfscope}%
\pgfpathrectangle{\pgfqpoint{0.894063in}{0.630000in}}{\pgfqpoint{6.713438in}{2.060556in}} %
\pgfusepath{clip}%
\pgfsetbuttcap%
\pgfsetroundjoin%
\definecolor{currentfill}{rgb}{0.000000,0.000000,1.000000}%
\pgfsetfillcolor{currentfill}%
\pgfsetlinewidth{1.003750pt}%
\definecolor{currentstroke}{rgb}{0.000000,0.000000,0.000000}%
\pgfsetstrokecolor{currentstroke}%
\pgfsetdash{}{0pt}%
\pgfpathmoveto{\pgfqpoint{5.459200in}{1.817379in}}%
\pgfpathcurveto{\pgfqpoint{5.467436in}{1.817379in}}{\pgfqpoint{5.475336in}{1.820652in}}{\pgfqpoint{5.481160in}{1.826476in}}%
\pgfpathcurveto{\pgfqpoint{5.486984in}{1.832300in}}{\pgfqpoint{5.490256in}{1.840200in}}{\pgfqpoint{5.490256in}{1.848436in}}%
\pgfpathcurveto{\pgfqpoint{5.490256in}{1.856672in}}{\pgfqpoint{5.486984in}{1.864572in}}{\pgfqpoint{5.481160in}{1.870396in}}%
\pgfpathcurveto{\pgfqpoint{5.475336in}{1.876220in}}{\pgfqpoint{5.467436in}{1.879492in}}{\pgfqpoint{5.459200in}{1.879492in}}%
\pgfpathcurveto{\pgfqpoint{5.450964in}{1.879492in}}{\pgfqpoint{5.443064in}{1.876220in}}{\pgfqpoint{5.437240in}{1.870396in}}%
\pgfpathcurveto{\pgfqpoint{5.431416in}{1.864572in}}{\pgfqpoint{5.428144in}{1.856672in}}{\pgfqpoint{5.428144in}{1.848436in}}%
\pgfpathcurveto{\pgfqpoint{5.428144in}{1.840200in}}{\pgfqpoint{5.431416in}{1.832300in}}{\pgfqpoint{5.437240in}{1.826476in}}%
\pgfpathcurveto{\pgfqpoint{5.443064in}{1.820652in}}{\pgfqpoint{5.450964in}{1.817379in}}{\pgfqpoint{5.459200in}{1.817379in}}%
\pgfpathclose%
\pgfusepath{stroke,fill}%
\end{pgfscope}%
\begin{pgfscope}%
\pgfpathrectangle{\pgfqpoint{0.894063in}{0.630000in}}{\pgfqpoint{6.713438in}{2.060556in}} %
\pgfusepath{clip}%
\pgfsetbuttcap%
\pgfsetroundjoin%
\definecolor{currentfill}{rgb}{0.000000,0.000000,1.000000}%
\pgfsetfillcolor{currentfill}%
\pgfsetlinewidth{1.003750pt}%
\definecolor{currentstroke}{rgb}{0.000000,0.000000,0.000000}%
\pgfsetstrokecolor{currentstroke}%
\pgfsetdash{}{0pt}%
\pgfpathmoveto{\pgfqpoint{6.936156in}{2.017665in}}%
\pgfpathcurveto{\pgfqpoint{6.944393in}{2.017665in}}{\pgfqpoint{6.952293in}{2.020938in}}{\pgfqpoint{6.958117in}{2.026762in}}%
\pgfpathcurveto{\pgfqpoint{6.963940in}{2.032586in}}{\pgfqpoint{6.967213in}{2.040486in}}{\pgfqpoint{6.967213in}{2.048722in}}%
\pgfpathcurveto{\pgfqpoint{6.967213in}{2.056958in}}{\pgfqpoint{6.963940in}{2.064858in}}{\pgfqpoint{6.958117in}{2.070682in}}%
\pgfpathcurveto{\pgfqpoint{6.952293in}{2.076506in}}{\pgfqpoint{6.944393in}{2.079778in}}{\pgfqpoint{6.936156in}{2.079778in}}%
\pgfpathcurveto{\pgfqpoint{6.927920in}{2.079778in}}{\pgfqpoint{6.920020in}{2.076506in}}{\pgfqpoint{6.914196in}{2.070682in}}%
\pgfpathcurveto{\pgfqpoint{6.908372in}{2.064858in}}{\pgfqpoint{6.905100in}{2.056958in}}{\pgfqpoint{6.905100in}{2.048722in}}%
\pgfpathcurveto{\pgfqpoint{6.905100in}{2.040486in}}{\pgfqpoint{6.908372in}{2.032586in}}{\pgfqpoint{6.914196in}{2.026762in}}%
\pgfpathcurveto{\pgfqpoint{6.920020in}{2.020938in}}{\pgfqpoint{6.927920in}{2.017665in}}{\pgfqpoint{6.936156in}{2.017665in}}%
\pgfpathclose%
\pgfusepath{stroke,fill}%
\end{pgfscope}%
\begin{pgfscope}%
\pgfpathrectangle{\pgfqpoint{0.894063in}{0.630000in}}{\pgfqpoint{6.713438in}{2.060556in}} %
\pgfusepath{clip}%
\pgfsetbuttcap%
\pgfsetroundjoin%
\definecolor{currentfill}{rgb}{0.000000,0.000000,1.000000}%
\pgfsetfillcolor{currentfill}%
\pgfsetlinewidth{1.003750pt}%
\definecolor{currentstroke}{rgb}{0.000000,0.000000,0.000000}%
\pgfsetstrokecolor{currentstroke}%
\pgfsetdash{}{0pt}%
\pgfpathmoveto{\pgfqpoint{5.862006in}{1.875269in}}%
\pgfpathcurveto{\pgfqpoint{5.870243in}{1.875269in}}{\pgfqpoint{5.878143in}{1.878542in}}{\pgfqpoint{5.883967in}{1.884366in}}%
\pgfpathcurveto{\pgfqpoint{5.889790in}{1.890189in}}{\pgfqpoint{5.893063in}{1.898089in}}{\pgfqpoint{5.893063in}{1.906326in}}%
\pgfpathcurveto{\pgfqpoint{5.893063in}{1.914562in}}{\pgfqpoint{5.889790in}{1.922462in}}{\pgfqpoint{5.883967in}{1.928286in}}%
\pgfpathcurveto{\pgfqpoint{5.878143in}{1.934110in}}{\pgfqpoint{5.870243in}{1.937382in}}{\pgfqpoint{5.862006in}{1.937382in}}%
\pgfpathcurveto{\pgfqpoint{5.853770in}{1.937382in}}{\pgfqpoint{5.845870in}{1.934110in}}{\pgfqpoint{5.840046in}{1.928286in}}%
\pgfpathcurveto{\pgfqpoint{5.834222in}{1.922462in}}{\pgfqpoint{5.830950in}{1.914562in}}{\pgfqpoint{5.830950in}{1.906326in}}%
\pgfpathcurveto{\pgfqpoint{5.830950in}{1.898089in}}{\pgfqpoint{5.834222in}{1.890189in}}{\pgfqpoint{5.840046in}{1.884366in}}%
\pgfpathcurveto{\pgfqpoint{5.845870in}{1.878542in}}{\pgfqpoint{5.853770in}{1.875269in}}{\pgfqpoint{5.862006in}{1.875269in}}%
\pgfpathclose%
\pgfusepath{stroke,fill}%
\end{pgfscope}%
\begin{pgfscope}%
\pgfpathrectangle{\pgfqpoint{0.894063in}{0.630000in}}{\pgfqpoint{6.713438in}{2.060556in}} %
\pgfusepath{clip}%
\pgfsetbuttcap%
\pgfsetroundjoin%
\definecolor{currentfill}{rgb}{0.000000,0.000000,1.000000}%
\pgfsetfillcolor{currentfill}%
\pgfsetlinewidth{1.003750pt}%
\definecolor{currentstroke}{rgb}{0.000000,0.000000,0.000000}%
\pgfsetstrokecolor{currentstroke}%
\pgfsetdash{}{0pt}%
\pgfpathmoveto{\pgfqpoint{7.070425in}{2.040302in}}%
\pgfpathcurveto{\pgfqpoint{7.078661in}{2.040302in}}{\pgfqpoint{7.086561in}{2.043574in}}{\pgfqpoint{7.092385in}{2.049398in}}%
\pgfpathcurveto{\pgfqpoint{7.098209in}{2.055222in}}{\pgfqpoint{7.101481in}{2.063122in}}{\pgfqpoint{7.101481in}{2.071359in}}%
\pgfpathcurveto{\pgfqpoint{7.101481in}{2.079595in}}{\pgfqpoint{7.098209in}{2.087495in}}{\pgfqpoint{7.092385in}{2.093319in}}%
\pgfpathcurveto{\pgfqpoint{7.086561in}{2.099143in}}{\pgfqpoint{7.078661in}{2.102415in}}{\pgfqpoint{7.070425in}{2.102415in}}%
\pgfpathcurveto{\pgfqpoint{7.062189in}{2.102415in}}{\pgfqpoint{7.054289in}{2.099143in}}{\pgfqpoint{7.048465in}{2.093319in}}%
\pgfpathcurveto{\pgfqpoint{7.042641in}{2.087495in}}{\pgfqpoint{7.039369in}{2.079595in}}{\pgfqpoint{7.039369in}{2.071359in}}%
\pgfpathcurveto{\pgfqpoint{7.039369in}{2.063122in}}{\pgfqpoint{7.042641in}{2.055222in}}{\pgfqpoint{7.048465in}{2.049398in}}%
\pgfpathcurveto{\pgfqpoint{7.054289in}{2.043574in}}{\pgfqpoint{7.062189in}{2.040302in}}{\pgfqpoint{7.070425in}{2.040302in}}%
\pgfpathclose%
\pgfusepath{stroke,fill}%
\end{pgfscope}%
\begin{pgfscope}%
\pgfpathrectangle{\pgfqpoint{0.894063in}{0.630000in}}{\pgfqpoint{6.713438in}{2.060556in}} %
\pgfusepath{clip}%
\pgfsetbuttcap%
\pgfsetroundjoin%
\definecolor{currentfill}{rgb}{0.000000,0.000000,1.000000}%
\pgfsetfillcolor{currentfill}%
\pgfsetlinewidth{1.003750pt}%
\definecolor{currentstroke}{rgb}{0.000000,0.000000,0.000000}%
\pgfsetstrokecolor{currentstroke}%
\pgfsetdash{}{0pt}%
\pgfpathmoveto{\pgfqpoint{3.176631in}{1.476958in}}%
\pgfpathcurveto{\pgfqpoint{3.184868in}{1.476958in}}{\pgfqpoint{3.192768in}{1.480230in}}{\pgfqpoint{3.198592in}{1.486054in}}%
\pgfpathcurveto{\pgfqpoint{3.204415in}{1.491878in}}{\pgfqpoint{3.207688in}{1.499778in}}{\pgfqpoint{3.207688in}{1.508014in}}%
\pgfpathcurveto{\pgfqpoint{3.207688in}{1.516251in}}{\pgfqpoint{3.204415in}{1.524151in}}{\pgfqpoint{3.198592in}{1.529975in}}%
\pgfpathcurveto{\pgfqpoint{3.192768in}{1.535799in}}{\pgfqpoint{3.184868in}{1.539071in}}{\pgfqpoint{3.176631in}{1.539071in}}%
\pgfpathcurveto{\pgfqpoint{3.168395in}{1.539071in}}{\pgfqpoint{3.160495in}{1.535799in}}{\pgfqpoint{3.154671in}{1.529975in}}%
\pgfpathcurveto{\pgfqpoint{3.148847in}{1.524151in}}{\pgfqpoint{3.145575in}{1.516251in}}{\pgfqpoint{3.145575in}{1.508014in}}%
\pgfpathcurveto{\pgfqpoint{3.145575in}{1.499778in}}{\pgfqpoint{3.148847in}{1.491878in}}{\pgfqpoint{3.154671in}{1.486054in}}%
\pgfpathcurveto{\pgfqpoint{3.160495in}{1.480230in}}{\pgfqpoint{3.168395in}{1.476958in}}{\pgfqpoint{3.176631in}{1.476958in}}%
\pgfpathclose%
\pgfusepath{stroke,fill}%
\end{pgfscope}%
\begin{pgfscope}%
\pgfpathrectangle{\pgfqpoint{0.894063in}{0.630000in}}{\pgfqpoint{6.713438in}{2.060556in}} %
\pgfusepath{clip}%
\pgfsetbuttcap%
\pgfsetroundjoin%
\definecolor{currentfill}{rgb}{0.000000,0.000000,1.000000}%
\pgfsetfillcolor{currentfill}%
\pgfsetlinewidth{1.003750pt}%
\definecolor{currentstroke}{rgb}{0.000000,0.000000,0.000000}%
\pgfsetstrokecolor{currentstroke}%
\pgfsetdash{}{0pt}%
\pgfpathmoveto{\pgfqpoint{2.102481in}{1.334538in}}%
\pgfpathcurveto{\pgfqpoint{2.110718in}{1.334538in}}{\pgfqpoint{2.118618in}{1.337811in}}{\pgfqpoint{2.124442in}{1.343635in}}%
\pgfpathcurveto{\pgfqpoint{2.130265in}{1.349458in}}{\pgfqpoint{2.133538in}{1.357359in}}{\pgfqpoint{2.133538in}{1.365595in}}%
\pgfpathcurveto{\pgfqpoint{2.133538in}{1.373831in}}{\pgfqpoint{2.130265in}{1.381731in}}{\pgfqpoint{2.124442in}{1.387555in}}%
\pgfpathcurveto{\pgfqpoint{2.118618in}{1.393379in}}{\pgfqpoint{2.110718in}{1.396651in}}{\pgfqpoint{2.102481in}{1.396651in}}%
\pgfpathcurveto{\pgfqpoint{2.094245in}{1.396651in}}{\pgfqpoint{2.086345in}{1.393379in}}{\pgfqpoint{2.080521in}{1.387555in}}%
\pgfpathcurveto{\pgfqpoint{2.074697in}{1.381731in}}{\pgfqpoint{2.071425in}{1.373831in}}{\pgfqpoint{2.071425in}{1.365595in}}%
\pgfpathcurveto{\pgfqpoint{2.071425in}{1.357359in}}{\pgfqpoint{2.074697in}{1.349458in}}{\pgfqpoint{2.080521in}{1.343635in}}%
\pgfpathcurveto{\pgfqpoint{2.086345in}{1.337811in}}{\pgfqpoint{2.094245in}{1.334538in}}{\pgfqpoint{2.102481in}{1.334538in}}%
\pgfpathclose%
\pgfusepath{stroke,fill}%
\end{pgfscope}%
\begin{pgfscope}%
\pgfpathrectangle{\pgfqpoint{0.894063in}{0.630000in}}{\pgfqpoint{6.713438in}{2.060556in}} %
\pgfusepath{clip}%
\pgfsetbuttcap%
\pgfsetroundjoin%
\definecolor{currentfill}{rgb}{0.000000,0.000000,1.000000}%
\pgfsetfillcolor{currentfill}%
\pgfsetlinewidth{1.003750pt}%
\definecolor{currentstroke}{rgb}{0.000000,0.000000,0.000000}%
\pgfsetstrokecolor{currentstroke}%
\pgfsetdash{}{0pt}%
\pgfpathmoveto{\pgfqpoint{1.968213in}{1.314233in}}%
\pgfpathcurveto{\pgfqpoint{1.976449in}{1.314233in}}{\pgfqpoint{1.984349in}{1.317505in}}{\pgfqpoint{1.990173in}{1.323329in}}%
\pgfpathcurveto{\pgfqpoint{1.995997in}{1.329153in}}{\pgfqpoint{1.999269in}{1.337053in}}{\pgfqpoint{1.999269in}{1.345289in}}%
\pgfpathcurveto{\pgfqpoint{1.999269in}{1.353526in}}{\pgfqpoint{1.995997in}{1.361426in}}{\pgfqpoint{1.990173in}{1.367250in}}%
\pgfpathcurveto{\pgfqpoint{1.984349in}{1.373074in}}{\pgfqpoint{1.976449in}{1.376346in}}{\pgfqpoint{1.968213in}{1.376346in}}%
\pgfpathcurveto{\pgfqpoint{1.959976in}{1.376346in}}{\pgfqpoint{1.952076in}{1.373074in}}{\pgfqpoint{1.946252in}{1.367250in}}%
\pgfpathcurveto{\pgfqpoint{1.940428in}{1.361426in}}{\pgfqpoint{1.937156in}{1.353526in}}{\pgfqpoint{1.937156in}{1.345289in}}%
\pgfpathcurveto{\pgfqpoint{1.937156in}{1.337053in}}{\pgfqpoint{1.940428in}{1.329153in}}{\pgfqpoint{1.946252in}{1.323329in}}%
\pgfpathcurveto{\pgfqpoint{1.952076in}{1.317505in}}{\pgfqpoint{1.959976in}{1.314233in}}{\pgfqpoint{1.968213in}{1.314233in}}%
\pgfpathclose%
\pgfusepath{stroke,fill}%
\end{pgfscope}%
\begin{pgfscope}%
\pgfpathrectangle{\pgfqpoint{0.894063in}{0.630000in}}{\pgfqpoint{6.713438in}{2.060556in}} %
\pgfusepath{clip}%
\pgfsetbuttcap%
\pgfsetroundjoin%
\definecolor{currentfill}{rgb}{0.000000,0.000000,1.000000}%
\pgfsetfillcolor{currentfill}%
\pgfsetlinewidth{1.003750pt}%
\definecolor{currentstroke}{rgb}{0.000000,0.000000,0.000000}%
\pgfsetstrokecolor{currentstroke}%
\pgfsetdash{}{0pt}%
\pgfpathmoveto{\pgfqpoint{3.310900in}{1.494108in}}%
\pgfpathcurveto{\pgfqpoint{3.319136in}{1.494108in}}{\pgfqpoint{3.327036in}{1.497380in}}{\pgfqpoint{3.332860in}{1.503204in}}%
\pgfpathcurveto{\pgfqpoint{3.338684in}{1.509028in}}{\pgfqpoint{3.341956in}{1.516928in}}{\pgfqpoint{3.341956in}{1.525164in}}%
\pgfpathcurveto{\pgfqpoint{3.341956in}{1.533400in}}{\pgfqpoint{3.338684in}{1.541301in}}{\pgfqpoint{3.332860in}{1.547124in}}%
\pgfpathcurveto{\pgfqpoint{3.327036in}{1.552948in}}{\pgfqpoint{3.319136in}{1.556221in}}{\pgfqpoint{3.310900in}{1.556221in}}%
\pgfpathcurveto{\pgfqpoint{3.302664in}{1.556221in}}{\pgfqpoint{3.294764in}{1.552948in}}{\pgfqpoint{3.288940in}{1.547124in}}%
\pgfpathcurveto{\pgfqpoint{3.283116in}{1.541301in}}{\pgfqpoint{3.279844in}{1.533400in}}{\pgfqpoint{3.279844in}{1.525164in}}%
\pgfpathcurveto{\pgfqpoint{3.279844in}{1.516928in}}{\pgfqpoint{3.283116in}{1.509028in}}{\pgfqpoint{3.288940in}{1.503204in}}%
\pgfpathcurveto{\pgfqpoint{3.294764in}{1.497380in}}{\pgfqpoint{3.302664in}{1.494108in}}{\pgfqpoint{3.310900in}{1.494108in}}%
\pgfpathclose%
\pgfusepath{stroke,fill}%
\end{pgfscope}%
\begin{pgfscope}%
\pgfpathrectangle{\pgfqpoint{0.894063in}{0.630000in}}{\pgfqpoint{6.713438in}{2.060556in}} %
\pgfusepath{clip}%
\pgfsetbuttcap%
\pgfsetroundjoin%
\definecolor{currentfill}{rgb}{0.000000,0.000000,1.000000}%
\pgfsetfillcolor{currentfill}%
\pgfsetlinewidth{1.003750pt}%
\definecolor{currentstroke}{rgb}{0.000000,0.000000,0.000000}%
\pgfsetstrokecolor{currentstroke}%
\pgfsetdash{}{0pt}%
\pgfpathmoveto{\pgfqpoint{5.593469in}{1.840670in}}%
\pgfpathcurveto{\pgfqpoint{5.601705in}{1.840670in}}{\pgfqpoint{5.609605in}{1.843942in}}{\pgfqpoint{5.615429in}{1.849766in}}%
\pgfpathcurveto{\pgfqpoint{5.621253in}{1.855590in}}{\pgfqpoint{5.624525in}{1.863490in}}{\pgfqpoint{5.624525in}{1.871726in}}%
\pgfpathcurveto{\pgfqpoint{5.624525in}{1.879962in}}{\pgfqpoint{5.621253in}{1.887862in}}{\pgfqpoint{5.615429in}{1.893686in}}%
\pgfpathcurveto{\pgfqpoint{5.609605in}{1.899510in}}{\pgfqpoint{5.601705in}{1.902783in}}{\pgfqpoint{5.593469in}{1.902783in}}%
\pgfpathcurveto{\pgfqpoint{5.585232in}{1.902783in}}{\pgfqpoint{5.577332in}{1.899510in}}{\pgfqpoint{5.571508in}{1.893686in}}%
\pgfpathcurveto{\pgfqpoint{5.565685in}{1.887862in}}{\pgfqpoint{5.562412in}{1.879962in}}{\pgfqpoint{5.562412in}{1.871726in}}%
\pgfpathcurveto{\pgfqpoint{5.562412in}{1.863490in}}{\pgfqpoint{5.565685in}{1.855590in}}{\pgfqpoint{5.571508in}{1.849766in}}%
\pgfpathcurveto{\pgfqpoint{5.577332in}{1.843942in}}{\pgfqpoint{5.585232in}{1.840670in}}{\pgfqpoint{5.593469in}{1.840670in}}%
\pgfpathclose%
\pgfusepath{stroke,fill}%
\end{pgfscope}%
\begin{pgfscope}%
\pgfpathrectangle{\pgfqpoint{0.894063in}{0.630000in}}{\pgfqpoint{6.713438in}{2.060556in}} %
\pgfusepath{clip}%
\pgfsetbuttcap%
\pgfsetroundjoin%
\definecolor{currentfill}{rgb}{0.000000,0.000000,1.000000}%
\pgfsetfillcolor{currentfill}%
\pgfsetlinewidth{1.003750pt}%
\definecolor{currentstroke}{rgb}{0.000000,0.000000,0.000000}%
\pgfsetstrokecolor{currentstroke}%
\pgfsetdash{}{0pt}%
\pgfpathmoveto{\pgfqpoint{3.042363in}{1.466355in}}%
\pgfpathcurveto{\pgfqpoint{3.050599in}{1.466355in}}{\pgfqpoint{3.058499in}{1.469627in}}{\pgfqpoint{3.064323in}{1.475451in}}%
\pgfpathcurveto{\pgfqpoint{3.070147in}{1.481275in}}{\pgfqpoint{3.073419in}{1.489175in}}{\pgfqpoint{3.073419in}{1.497411in}}%
\pgfpathcurveto{\pgfqpoint{3.073419in}{1.505648in}}{\pgfqpoint{3.070147in}{1.513548in}}{\pgfqpoint{3.064323in}{1.519372in}}%
\pgfpathcurveto{\pgfqpoint{3.058499in}{1.525196in}}{\pgfqpoint{3.050599in}{1.528468in}}{\pgfqpoint{3.042363in}{1.528468in}}%
\pgfpathcurveto{\pgfqpoint{3.034126in}{1.528468in}}{\pgfqpoint{3.026226in}{1.525196in}}{\pgfqpoint{3.020402in}{1.519372in}}%
\pgfpathcurveto{\pgfqpoint{3.014578in}{1.513548in}}{\pgfqpoint{3.011306in}{1.505648in}}{\pgfqpoint{3.011306in}{1.497411in}}%
\pgfpathcurveto{\pgfqpoint{3.011306in}{1.489175in}}{\pgfqpoint{3.014578in}{1.481275in}}{\pgfqpoint{3.020402in}{1.475451in}}%
\pgfpathcurveto{\pgfqpoint{3.026226in}{1.469627in}}{\pgfqpoint{3.034126in}{1.466355in}}{\pgfqpoint{3.042363in}{1.466355in}}%
\pgfpathclose%
\pgfusepath{stroke,fill}%
\end{pgfscope}%
\begin{pgfscope}%
\pgfpathrectangle{\pgfqpoint{0.894063in}{0.630000in}}{\pgfqpoint{6.713438in}{2.060556in}} %
\pgfusepath{clip}%
\pgfsetbuttcap%
\pgfsetroundjoin%
\definecolor{currentfill}{rgb}{0.000000,0.000000,1.000000}%
\pgfsetfillcolor{currentfill}%
\pgfsetlinewidth{1.003750pt}%
\definecolor{currentstroke}{rgb}{0.000000,0.000000,0.000000}%
\pgfsetstrokecolor{currentstroke}%
\pgfsetdash{}{0pt}%
\pgfpathmoveto{\pgfqpoint{5.190663in}{1.785341in}}%
\pgfpathcurveto{\pgfqpoint{5.198899in}{1.785341in}}{\pgfqpoint{5.206799in}{1.788613in}}{\pgfqpoint{5.212623in}{1.794437in}}%
\pgfpathcurveto{\pgfqpoint{5.218447in}{1.800261in}}{\pgfqpoint{5.221719in}{1.808161in}}{\pgfqpoint{5.221719in}{1.816397in}}%
\pgfpathcurveto{\pgfqpoint{5.221719in}{1.824634in}}{\pgfqpoint{5.218447in}{1.832534in}}{\pgfqpoint{5.212623in}{1.838358in}}%
\pgfpathcurveto{\pgfqpoint{5.206799in}{1.844181in}}{\pgfqpoint{5.198899in}{1.847454in}}{\pgfqpoint{5.190663in}{1.847454in}}%
\pgfpathcurveto{\pgfqpoint{5.182426in}{1.847454in}}{\pgfqpoint{5.174526in}{1.844181in}}{\pgfqpoint{5.168702in}{1.838358in}}%
\pgfpathcurveto{\pgfqpoint{5.162878in}{1.832534in}}{\pgfqpoint{5.159606in}{1.824634in}}{\pgfqpoint{5.159606in}{1.816397in}}%
\pgfpathcurveto{\pgfqpoint{5.159606in}{1.808161in}}{\pgfqpoint{5.162878in}{1.800261in}}{\pgfqpoint{5.168702in}{1.794437in}}%
\pgfpathcurveto{\pgfqpoint{5.174526in}{1.788613in}}{\pgfqpoint{5.182426in}{1.785341in}}{\pgfqpoint{5.190663in}{1.785341in}}%
\pgfpathclose%
\pgfusepath{stroke,fill}%
\end{pgfscope}%
\begin{pgfscope}%
\pgfpathrectangle{\pgfqpoint{0.894063in}{0.630000in}}{\pgfqpoint{6.713438in}{2.060556in}} %
\pgfusepath{clip}%
\pgfsetbuttcap%
\pgfsetroundjoin%
\definecolor{currentfill}{rgb}{0.000000,0.000000,1.000000}%
\pgfsetfillcolor{currentfill}%
\pgfsetlinewidth{1.003750pt}%
\definecolor{currentstroke}{rgb}{0.000000,0.000000,0.000000}%
\pgfsetstrokecolor{currentstroke}%
\pgfsetdash{}{0pt}%
\pgfpathmoveto{\pgfqpoint{6.801888in}{2.001399in}}%
\pgfpathcurveto{\pgfqpoint{6.810124in}{2.001399in}}{\pgfqpoint{6.818024in}{2.004671in}}{\pgfqpoint{6.823848in}{2.010495in}}%
\pgfpathcurveto{\pgfqpoint{6.829672in}{2.016319in}}{\pgfqpoint{6.832944in}{2.024219in}}{\pgfqpoint{6.832944in}{2.032455in}}%
\pgfpathcurveto{\pgfqpoint{6.832944in}{2.040692in}}{\pgfqpoint{6.829672in}{2.048592in}}{\pgfqpoint{6.823848in}{2.054416in}}%
\pgfpathcurveto{\pgfqpoint{6.818024in}{2.060240in}}{\pgfqpoint{6.810124in}{2.063512in}}{\pgfqpoint{6.801888in}{2.063512in}}%
\pgfpathcurveto{\pgfqpoint{6.793651in}{2.063512in}}{\pgfqpoint{6.785751in}{2.060240in}}{\pgfqpoint{6.779927in}{2.054416in}}%
\pgfpathcurveto{\pgfqpoint{6.774103in}{2.048592in}}{\pgfqpoint{6.770831in}{2.040692in}}{\pgfqpoint{6.770831in}{2.032455in}}%
\pgfpathcurveto{\pgfqpoint{6.770831in}{2.024219in}}{\pgfqpoint{6.774103in}{2.016319in}}{\pgfqpoint{6.779927in}{2.010495in}}%
\pgfpathcurveto{\pgfqpoint{6.785751in}{2.004671in}}{\pgfqpoint{6.793651in}{2.001399in}}{\pgfqpoint{6.801888in}{2.001399in}}%
\pgfpathclose%
\pgfusepath{stroke,fill}%
\end{pgfscope}%
\begin{pgfscope}%
\pgfpathrectangle{\pgfqpoint{0.894063in}{0.630000in}}{\pgfqpoint{6.713438in}{2.060556in}} %
\pgfusepath{clip}%
\pgfsetbuttcap%
\pgfsetroundjoin%
\definecolor{currentfill}{rgb}{0.000000,0.000000,1.000000}%
\pgfsetfillcolor{currentfill}%
\pgfsetlinewidth{1.003750pt}%
\definecolor{currentstroke}{rgb}{0.000000,0.000000,0.000000}%
\pgfsetstrokecolor{currentstroke}%
\pgfsetdash{}{0pt}%
\pgfpathmoveto{\pgfqpoint{3.579438in}{1.540152in}}%
\pgfpathcurveto{\pgfqpoint{3.587674in}{1.540152in}}{\pgfqpoint{3.595574in}{1.543425in}}{\pgfqpoint{3.601398in}{1.549249in}}%
\pgfpathcurveto{\pgfqpoint{3.607222in}{1.555072in}}{\pgfqpoint{3.610494in}{1.562973in}}{\pgfqpoint{3.610494in}{1.571209in}}%
\pgfpathcurveto{\pgfqpoint{3.610494in}{1.579445in}}{\pgfqpoint{3.607222in}{1.587345in}}{\pgfqpoint{3.601398in}{1.593169in}}%
\pgfpathcurveto{\pgfqpoint{3.595574in}{1.598993in}}{\pgfqpoint{3.587674in}{1.602265in}}{\pgfqpoint{3.579438in}{1.602265in}}%
\pgfpathcurveto{\pgfqpoint{3.571201in}{1.602265in}}{\pgfqpoint{3.563301in}{1.598993in}}{\pgfqpoint{3.557477in}{1.593169in}}%
\pgfpathcurveto{\pgfqpoint{3.551653in}{1.587345in}}{\pgfqpoint{3.548381in}{1.579445in}}{\pgfqpoint{3.548381in}{1.571209in}}%
\pgfpathcurveto{\pgfqpoint{3.548381in}{1.562973in}}{\pgfqpoint{3.551653in}{1.555072in}}{\pgfqpoint{3.557477in}{1.549249in}}%
\pgfpathcurveto{\pgfqpoint{3.563301in}{1.543425in}}{\pgfqpoint{3.571201in}{1.540152in}}{\pgfqpoint{3.579438in}{1.540152in}}%
\pgfpathclose%
\pgfusepath{stroke,fill}%
\end{pgfscope}%
\begin{pgfscope}%
\pgfpathrectangle{\pgfqpoint{0.894063in}{0.630000in}}{\pgfqpoint{6.713438in}{2.060556in}} %
\pgfusepath{clip}%
\pgfsetbuttcap%
\pgfsetroundjoin%
\definecolor{currentfill}{rgb}{0.000000,0.000000,1.000000}%
\pgfsetfillcolor{currentfill}%
\pgfsetlinewidth{1.003750pt}%
\definecolor{currentstroke}{rgb}{0.000000,0.000000,0.000000}%
\pgfsetstrokecolor{currentstroke}%
\pgfsetdash{}{0pt}%
\pgfpathmoveto{\pgfqpoint{2.371019in}{1.372765in}}%
\pgfpathcurveto{\pgfqpoint{2.379255in}{1.372765in}}{\pgfqpoint{2.387155in}{1.376037in}}{\pgfqpoint{2.392979in}{1.381861in}}%
\pgfpathcurveto{\pgfqpoint{2.398803in}{1.387685in}}{\pgfqpoint{2.402075in}{1.395585in}}{\pgfqpoint{2.402075in}{1.403821in}}%
\pgfpathcurveto{\pgfqpoint{2.402075in}{1.412057in}}{\pgfqpoint{2.398803in}{1.419957in}}{\pgfqpoint{2.392979in}{1.425781in}}%
\pgfpathcurveto{\pgfqpoint{2.387155in}{1.431605in}}{\pgfqpoint{2.379255in}{1.434878in}}{\pgfqpoint{2.371019in}{1.434878in}}%
\pgfpathcurveto{\pgfqpoint{2.362782in}{1.434878in}}{\pgfqpoint{2.354882in}{1.431605in}}{\pgfqpoint{2.349058in}{1.425781in}}%
\pgfpathcurveto{\pgfqpoint{2.343235in}{1.419957in}}{\pgfqpoint{2.339962in}{1.412057in}}{\pgfqpoint{2.339962in}{1.403821in}}%
\pgfpathcurveto{\pgfqpoint{2.339962in}{1.395585in}}{\pgfqpoint{2.343235in}{1.387685in}}{\pgfqpoint{2.349058in}{1.381861in}}%
\pgfpathcurveto{\pgfqpoint{2.354882in}{1.376037in}}{\pgfqpoint{2.362782in}{1.372765in}}{\pgfqpoint{2.371019in}{1.372765in}}%
\pgfpathclose%
\pgfusepath{stroke,fill}%
\end{pgfscope}%
\begin{pgfscope}%
\pgfpathrectangle{\pgfqpoint{0.894063in}{0.630000in}}{\pgfqpoint{6.713438in}{2.060556in}} %
\pgfusepath{clip}%
\pgfsetbuttcap%
\pgfsetroundjoin%
\definecolor{currentfill}{rgb}{0.000000,0.000000,1.000000}%
\pgfsetfillcolor{currentfill}%
\pgfsetlinewidth{1.003750pt}%
\definecolor{currentstroke}{rgb}{0.000000,0.000000,0.000000}%
\pgfsetstrokecolor{currentstroke}%
\pgfsetdash{}{0pt}%
\pgfpathmoveto{\pgfqpoint{3.982244in}{1.608115in}}%
\pgfpathcurveto{\pgfqpoint{3.990480in}{1.608115in}}{\pgfqpoint{3.998380in}{1.611388in}}{\pgfqpoint{4.004204in}{1.617212in}}%
\pgfpathcurveto{\pgfqpoint{4.010028in}{1.623035in}}{\pgfqpoint{4.013300in}{1.630936in}}{\pgfqpoint{4.013300in}{1.639172in}}%
\pgfpathcurveto{\pgfqpoint{4.013300in}{1.647408in}}{\pgfqpoint{4.010028in}{1.655308in}}{\pgfqpoint{4.004204in}{1.661132in}}%
\pgfpathcurveto{\pgfqpoint{3.998380in}{1.666956in}}{\pgfqpoint{3.990480in}{1.670228in}}{\pgfqpoint{3.982244in}{1.670228in}}%
\pgfpathcurveto{\pgfqpoint{3.974007in}{1.670228in}}{\pgfqpoint{3.966107in}{1.666956in}}{\pgfqpoint{3.960283in}{1.661132in}}%
\pgfpathcurveto{\pgfqpoint{3.954460in}{1.655308in}}{\pgfqpoint{3.951187in}{1.647408in}}{\pgfqpoint{3.951187in}{1.639172in}}%
\pgfpathcurveto{\pgfqpoint{3.951187in}{1.630936in}}{\pgfqpoint{3.954460in}{1.623035in}}{\pgfqpoint{3.960283in}{1.617212in}}%
\pgfpathcurveto{\pgfqpoint{3.966107in}{1.611388in}}{\pgfqpoint{3.974007in}{1.608115in}}{\pgfqpoint{3.982244in}{1.608115in}}%
\pgfpathclose%
\pgfusepath{stroke,fill}%
\end{pgfscope}%
\begin{pgfscope}%
\pgfpathrectangle{\pgfqpoint{0.894063in}{0.630000in}}{\pgfqpoint{6.713438in}{2.060556in}} %
\pgfusepath{clip}%
\pgfsetbuttcap%
\pgfsetroundjoin%
\definecolor{currentfill}{rgb}{0.000000,0.000000,1.000000}%
\pgfsetfillcolor{currentfill}%
\pgfsetlinewidth{1.003750pt}%
\definecolor{currentstroke}{rgb}{0.000000,0.000000,0.000000}%
\pgfsetstrokecolor{currentstroke}%
\pgfsetdash{}{0pt}%
\pgfpathmoveto{\pgfqpoint{4.653588in}{1.708682in}}%
\pgfpathcurveto{\pgfqpoint{4.661824in}{1.708682in}}{\pgfqpoint{4.669724in}{1.711954in}}{\pgfqpoint{4.675548in}{1.717778in}}%
\pgfpathcurveto{\pgfqpoint{4.681372in}{1.723602in}}{\pgfqpoint{4.684644in}{1.731502in}}{\pgfqpoint{4.684644in}{1.739739in}}%
\pgfpathcurveto{\pgfqpoint{4.684644in}{1.747975in}}{\pgfqpoint{4.681372in}{1.755875in}}{\pgfqpoint{4.675548in}{1.761699in}}%
\pgfpathcurveto{\pgfqpoint{4.669724in}{1.767523in}}{\pgfqpoint{4.661824in}{1.770795in}}{\pgfqpoint{4.653588in}{1.770795in}}%
\pgfpathcurveto{\pgfqpoint{4.645351in}{1.770795in}}{\pgfqpoint{4.637451in}{1.767523in}}{\pgfqpoint{4.631627in}{1.761699in}}%
\pgfpathcurveto{\pgfqpoint{4.625803in}{1.755875in}}{\pgfqpoint{4.622531in}{1.747975in}}{\pgfqpoint{4.622531in}{1.739739in}}%
\pgfpathcurveto{\pgfqpoint{4.622531in}{1.731502in}}{\pgfqpoint{4.625803in}{1.723602in}}{\pgfqpoint{4.631627in}{1.717778in}}%
\pgfpathcurveto{\pgfqpoint{4.637451in}{1.711954in}}{\pgfqpoint{4.645351in}{1.708682in}}{\pgfqpoint{4.653588in}{1.708682in}}%
\pgfpathclose%
\pgfusepath{stroke,fill}%
\end{pgfscope}%
\begin{pgfscope}%
\pgfpathrectangle{\pgfqpoint{0.894063in}{0.630000in}}{\pgfqpoint{6.713438in}{2.060556in}} %
\pgfusepath{clip}%
\pgfsetbuttcap%
\pgfsetroundjoin%
\definecolor{currentfill}{rgb}{0.000000,0.000000,1.000000}%
\pgfsetfillcolor{currentfill}%
\pgfsetlinewidth{1.003750pt}%
\definecolor{currentstroke}{rgb}{0.000000,0.000000,0.000000}%
\pgfsetstrokecolor{currentstroke}%
\pgfsetdash{}{0pt}%
\pgfpathmoveto{\pgfqpoint{3.713706in}{1.564249in}}%
\pgfpathcurveto{\pgfqpoint{3.721943in}{1.564249in}}{\pgfqpoint{3.729843in}{1.567521in}}{\pgfqpoint{3.735667in}{1.573345in}}%
\pgfpathcurveto{\pgfqpoint{3.741490in}{1.579169in}}{\pgfqpoint{3.744763in}{1.587069in}}{\pgfqpoint{3.744763in}{1.595306in}}%
\pgfpathcurveto{\pgfqpoint{3.744763in}{1.603542in}}{\pgfqpoint{3.741490in}{1.611442in}}{\pgfqpoint{3.735667in}{1.617266in}}%
\pgfpathcurveto{\pgfqpoint{3.729843in}{1.623090in}}{\pgfqpoint{3.721943in}{1.626362in}}{\pgfqpoint{3.713706in}{1.626362in}}%
\pgfpathcurveto{\pgfqpoint{3.705470in}{1.626362in}}{\pgfqpoint{3.697570in}{1.623090in}}{\pgfqpoint{3.691746in}{1.617266in}}%
\pgfpathcurveto{\pgfqpoint{3.685922in}{1.611442in}}{\pgfqpoint{3.682650in}{1.603542in}}{\pgfqpoint{3.682650in}{1.595306in}}%
\pgfpathcurveto{\pgfqpoint{3.682650in}{1.587069in}}{\pgfqpoint{3.685922in}{1.579169in}}{\pgfqpoint{3.691746in}{1.573345in}}%
\pgfpathcurveto{\pgfqpoint{3.697570in}{1.567521in}}{\pgfqpoint{3.705470in}{1.564249in}}{\pgfqpoint{3.713706in}{1.564249in}}%
\pgfpathclose%
\pgfusepath{stroke,fill}%
\end{pgfscope}%
\begin{pgfscope}%
\pgfpathrectangle{\pgfqpoint{0.894063in}{0.630000in}}{\pgfqpoint{6.713438in}{2.060556in}} %
\pgfusepath{clip}%
\pgfsetbuttcap%
\pgfsetroundjoin%
\definecolor{currentfill}{rgb}{0.000000,0.000000,1.000000}%
\pgfsetfillcolor{currentfill}%
\pgfsetlinewidth{1.003750pt}%
\definecolor{currentstroke}{rgb}{0.000000,0.000000,0.000000}%
\pgfsetstrokecolor{currentstroke}%
\pgfsetdash{}{0pt}%
\pgfpathmoveto{\pgfqpoint{2.236750in}{1.351935in}}%
\pgfpathcurveto{\pgfqpoint{2.244986in}{1.351935in}}{\pgfqpoint{2.252886in}{1.355208in}}{\pgfqpoint{2.258710in}{1.361031in}}%
\pgfpathcurveto{\pgfqpoint{2.264534in}{1.366855in}}{\pgfqpoint{2.267806in}{1.374755in}}{\pgfqpoint{2.267806in}{1.382992in}}%
\pgfpathcurveto{\pgfqpoint{2.267806in}{1.391228in}}{\pgfqpoint{2.264534in}{1.399128in}}{\pgfqpoint{2.258710in}{1.404952in}}%
\pgfpathcurveto{\pgfqpoint{2.252886in}{1.410776in}}{\pgfqpoint{2.244986in}{1.414048in}}{\pgfqpoint{2.236750in}{1.414048in}}%
\pgfpathcurveto{\pgfqpoint{2.228514in}{1.414048in}}{\pgfqpoint{2.220614in}{1.410776in}}{\pgfqpoint{2.214790in}{1.404952in}}%
\pgfpathcurveto{\pgfqpoint{2.208966in}{1.399128in}}{\pgfqpoint{2.205694in}{1.391228in}}{\pgfqpoint{2.205694in}{1.382992in}}%
\pgfpathcurveto{\pgfqpoint{2.205694in}{1.374755in}}{\pgfqpoint{2.208966in}{1.366855in}}{\pgfqpoint{2.214790in}{1.361031in}}%
\pgfpathcurveto{\pgfqpoint{2.220614in}{1.355208in}}{\pgfqpoint{2.228514in}{1.351935in}}{\pgfqpoint{2.236750in}{1.351935in}}%
\pgfpathclose%
\pgfusepath{stroke,fill}%
\end{pgfscope}%
\begin{pgfscope}%
\pgfpathrectangle{\pgfqpoint{0.894063in}{0.630000in}}{\pgfqpoint{6.713438in}{2.060556in}} %
\pgfusepath{clip}%
\pgfsetbuttcap%
\pgfsetroundjoin%
\definecolor{currentfill}{rgb}{0.000000,0.000000,1.000000}%
\pgfsetfillcolor{currentfill}%
\pgfsetlinewidth{1.003750pt}%
\definecolor{currentstroke}{rgb}{0.000000,0.000000,0.000000}%
\pgfsetstrokecolor{currentstroke}%
\pgfsetdash{}{0pt}%
\pgfpathmoveto{\pgfqpoint{6.667619in}{2.253175in}}%
\pgfpathcurveto{\pgfqpoint{6.675855in}{2.253175in}}{\pgfqpoint{6.683755in}{2.256447in}}{\pgfqpoint{6.689579in}{2.262271in}}%
\pgfpathcurveto{\pgfqpoint{6.695403in}{2.268095in}}{\pgfqpoint{6.698675in}{2.275995in}}{\pgfqpoint{6.698675in}{2.284232in}}%
\pgfpathcurveto{\pgfqpoint{6.698675in}{2.292468in}}{\pgfqpoint{6.695403in}{2.300368in}}{\pgfqpoint{6.689579in}{2.306192in}}%
\pgfpathcurveto{\pgfqpoint{6.683755in}{2.312016in}}{\pgfqpoint{6.675855in}{2.315288in}}{\pgfqpoint{6.667619in}{2.315288in}}%
\pgfpathcurveto{\pgfqpoint{6.659382in}{2.315288in}}{\pgfqpoint{6.651482in}{2.312016in}}{\pgfqpoint{6.645658in}{2.306192in}}%
\pgfpathcurveto{\pgfqpoint{6.639835in}{2.300368in}}{\pgfqpoint{6.636562in}{2.292468in}}{\pgfqpoint{6.636562in}{2.284232in}}%
\pgfpathcurveto{\pgfqpoint{6.636562in}{2.275995in}}{\pgfqpoint{6.639835in}{2.268095in}}{\pgfqpoint{6.645658in}{2.262271in}}%
\pgfpathcurveto{\pgfqpoint{6.651482in}{2.256447in}}{\pgfqpoint{6.659382in}{2.253175in}}{\pgfqpoint{6.667619in}{2.253175in}}%
\pgfpathclose%
\pgfusepath{stroke,fill}%
\end{pgfscope}%
\begin{pgfscope}%
\pgfpathrectangle{\pgfqpoint{0.894063in}{0.630000in}}{\pgfqpoint{6.713438in}{2.060556in}} %
\pgfusepath{clip}%
\pgfsetbuttcap%
\pgfsetroundjoin%
\definecolor{currentfill}{rgb}{0.000000,0.000000,1.000000}%
\pgfsetfillcolor{currentfill}%
\pgfsetlinewidth{1.003750pt}%
\definecolor{currentstroke}{rgb}{0.000000,0.000000,0.000000}%
\pgfsetstrokecolor{currentstroke}%
\pgfsetdash{}{0pt}%
\pgfpathmoveto{\pgfqpoint{2.639556in}{1.443960in}}%
\pgfpathcurveto{\pgfqpoint{2.647793in}{1.443960in}}{\pgfqpoint{2.655693in}{1.447232in}}{\pgfqpoint{2.661517in}{1.453056in}}%
\pgfpathcurveto{\pgfqpoint{2.667340in}{1.458880in}}{\pgfqpoint{2.670613in}{1.466780in}}{\pgfqpoint{2.670613in}{1.475016in}}%
\pgfpathcurveto{\pgfqpoint{2.670613in}{1.483252in}}{\pgfqpoint{2.667340in}{1.491153in}}{\pgfqpoint{2.661517in}{1.496976in}}%
\pgfpathcurveto{\pgfqpoint{2.655693in}{1.502800in}}{\pgfqpoint{2.647793in}{1.506073in}}{\pgfqpoint{2.639556in}{1.506073in}}%
\pgfpathcurveto{\pgfqpoint{2.631320in}{1.506073in}}{\pgfqpoint{2.623420in}{1.502800in}}{\pgfqpoint{2.617596in}{1.496976in}}%
\pgfpathcurveto{\pgfqpoint{2.611772in}{1.491153in}}{\pgfqpoint{2.608500in}{1.483252in}}{\pgfqpoint{2.608500in}{1.475016in}}%
\pgfpathcurveto{\pgfqpoint{2.608500in}{1.466780in}}{\pgfqpoint{2.611772in}{1.458880in}}{\pgfqpoint{2.617596in}{1.453056in}}%
\pgfpathcurveto{\pgfqpoint{2.623420in}{1.447232in}}{\pgfqpoint{2.631320in}{1.443960in}}{\pgfqpoint{2.639556in}{1.443960in}}%
\pgfpathclose%
\pgfusepath{stroke,fill}%
\end{pgfscope}%
\begin{pgfscope}%
\pgfpathrectangle{\pgfqpoint{0.894063in}{0.630000in}}{\pgfqpoint{6.713438in}{2.060556in}} %
\pgfusepath{clip}%
\pgfsetbuttcap%
\pgfsetroundjoin%
\definecolor{currentfill}{rgb}{0.000000,0.000000,1.000000}%
\pgfsetfillcolor{currentfill}%
\pgfsetlinewidth{1.003750pt}%
\definecolor{currentstroke}{rgb}{0.000000,0.000000,0.000000}%
\pgfsetstrokecolor{currentstroke}%
\pgfsetdash{}{0pt}%
\pgfpathmoveto{\pgfqpoint{1.699675in}{1.293828in}}%
\pgfpathcurveto{\pgfqpoint{1.707911in}{1.293828in}}{\pgfqpoint{1.715811in}{1.297100in}}{\pgfqpoint{1.721635in}{1.302924in}}%
\pgfpathcurveto{\pgfqpoint{1.727459in}{1.308748in}}{\pgfqpoint{1.730731in}{1.316648in}}{\pgfqpoint{1.730731in}{1.324884in}}%
\pgfpathcurveto{\pgfqpoint{1.730731in}{1.333120in}}{\pgfqpoint{1.727459in}{1.341020in}}{\pgfqpoint{1.721635in}{1.346844in}}%
\pgfpathcurveto{\pgfqpoint{1.715811in}{1.352668in}}{\pgfqpoint{1.707911in}{1.355941in}}{\pgfqpoint{1.699675in}{1.355941in}}%
\pgfpathcurveto{\pgfqpoint{1.691439in}{1.355941in}}{\pgfqpoint{1.683539in}{1.352668in}}{\pgfqpoint{1.677715in}{1.346844in}}%
\pgfpathcurveto{\pgfqpoint{1.671891in}{1.341020in}}{\pgfqpoint{1.668619in}{1.333120in}}{\pgfqpoint{1.668619in}{1.324884in}}%
\pgfpathcurveto{\pgfqpoint{1.668619in}{1.316648in}}{\pgfqpoint{1.671891in}{1.308748in}}{\pgfqpoint{1.677715in}{1.302924in}}%
\pgfpathcurveto{\pgfqpoint{1.683539in}{1.297100in}}{\pgfqpoint{1.691439in}{1.293828in}}{\pgfqpoint{1.699675in}{1.293828in}}%
\pgfpathclose%
\pgfusepath{stroke,fill}%
\end{pgfscope}%
\begin{pgfscope}%
\pgfpathrectangle{\pgfqpoint{0.894063in}{0.630000in}}{\pgfqpoint{6.713438in}{2.060556in}} %
\pgfusepath{clip}%
\pgfsetbuttcap%
\pgfsetroundjoin%
\definecolor{currentfill}{rgb}{0.000000,0.000000,1.000000}%
\pgfsetfillcolor{currentfill}%
\pgfsetlinewidth{1.003750pt}%
\definecolor{currentstroke}{rgb}{0.000000,0.000000,0.000000}%
\pgfsetstrokecolor{currentstroke}%
\pgfsetdash{}{0pt}%
\pgfpathmoveto{\pgfqpoint{1.162600in}{0.702625in}}%
\pgfpathcurveto{\pgfqpoint{1.170836in}{0.702625in}}{\pgfqpoint{1.178736in}{0.705897in}}{\pgfqpoint{1.184560in}{0.711721in}}%
\pgfpathcurveto{\pgfqpoint{1.190384in}{0.717545in}}{\pgfqpoint{1.193656in}{0.725445in}}{\pgfqpoint{1.193656in}{0.733681in}}%
\pgfpathcurveto{\pgfqpoint{1.193656in}{0.741918in}}{\pgfqpoint{1.190384in}{0.749818in}}{\pgfqpoint{1.184560in}{0.755642in}}%
\pgfpathcurveto{\pgfqpoint{1.178736in}{0.761465in}}{\pgfqpoint{1.170836in}{0.764738in}}{\pgfqpoint{1.162600in}{0.764738in}}%
\pgfpathcurveto{\pgfqpoint{1.154364in}{0.764738in}}{\pgfqpoint{1.146464in}{0.761465in}}{\pgfqpoint{1.140640in}{0.755642in}}%
\pgfpathcurveto{\pgfqpoint{1.134816in}{0.749818in}}{\pgfqpoint{1.131544in}{0.741918in}}{\pgfqpoint{1.131544in}{0.733681in}}%
\pgfpathcurveto{\pgfqpoint{1.131544in}{0.725445in}}{\pgfqpoint{1.134816in}{0.717545in}}{\pgfqpoint{1.140640in}{0.711721in}}%
\pgfpathcurveto{\pgfqpoint{1.146464in}{0.705897in}}{\pgfqpoint{1.154364in}{0.702625in}}{\pgfqpoint{1.162600in}{0.702625in}}%
\pgfpathclose%
\pgfusepath{stroke,fill}%
\end{pgfscope}%
\begin{pgfscope}%
\pgfpathrectangle{\pgfqpoint{0.894063in}{0.630000in}}{\pgfqpoint{6.713438in}{2.060556in}} %
\pgfusepath{clip}%
\pgfsetbuttcap%
\pgfsetroundjoin%
\definecolor{currentfill}{rgb}{0.000000,0.000000,1.000000}%
\pgfsetfillcolor{currentfill}%
\pgfsetlinewidth{1.003750pt}%
\definecolor{currentstroke}{rgb}{0.000000,0.000000,0.000000}%
\pgfsetstrokecolor{currentstroke}%
\pgfsetdash{}{0pt}%
\pgfpathmoveto{\pgfqpoint{1.833944in}{1.305950in}}%
\pgfpathcurveto{\pgfqpoint{1.842180in}{1.305950in}}{\pgfqpoint{1.850080in}{1.309222in}}{\pgfqpoint{1.855904in}{1.315046in}}%
\pgfpathcurveto{\pgfqpoint{1.861728in}{1.320870in}}{\pgfqpoint{1.865000in}{1.328770in}}{\pgfqpoint{1.865000in}{1.337006in}}%
\pgfpathcurveto{\pgfqpoint{1.865000in}{1.345242in}}{\pgfqpoint{1.861728in}{1.353142in}}{\pgfqpoint{1.855904in}{1.358966in}}%
\pgfpathcurveto{\pgfqpoint{1.850080in}{1.364790in}}{\pgfqpoint{1.842180in}{1.368063in}}{\pgfqpoint{1.833944in}{1.368063in}}%
\pgfpathcurveto{\pgfqpoint{1.825707in}{1.368063in}}{\pgfqpoint{1.817807in}{1.364790in}}{\pgfqpoint{1.811983in}{1.358966in}}%
\pgfpathcurveto{\pgfqpoint{1.806160in}{1.353142in}}{\pgfqpoint{1.802887in}{1.345242in}}{\pgfqpoint{1.802887in}{1.337006in}}%
\pgfpathcurveto{\pgfqpoint{1.802887in}{1.328770in}}{\pgfqpoint{1.806160in}{1.320870in}}{\pgfqpoint{1.811983in}{1.315046in}}%
\pgfpathcurveto{\pgfqpoint{1.817807in}{1.309222in}}{\pgfqpoint{1.825707in}{1.305950in}}{\pgfqpoint{1.833944in}{1.305950in}}%
\pgfpathclose%
\pgfusepath{stroke,fill}%
\end{pgfscope}%
\begin{pgfscope}%
\pgfpathrectangle{\pgfqpoint{0.894063in}{0.630000in}}{\pgfqpoint{6.713438in}{2.060556in}} %
\pgfusepath{clip}%
\pgfsetbuttcap%
\pgfsetroundjoin%
\definecolor{currentfill}{rgb}{0.000000,0.000000,1.000000}%
\pgfsetfillcolor{currentfill}%
\pgfsetlinewidth{1.003750pt}%
\definecolor{currentstroke}{rgb}{0.000000,0.000000,0.000000}%
\pgfsetstrokecolor{currentstroke}%
\pgfsetdash{}{0pt}%
\pgfpathmoveto{\pgfqpoint{5.996275in}{2.124402in}}%
\pgfpathcurveto{\pgfqpoint{6.004511in}{2.124402in}}{\pgfqpoint{6.012411in}{2.127675in}}{\pgfqpoint{6.018235in}{2.133498in}}%
\pgfpathcurveto{\pgfqpoint{6.024059in}{2.139322in}}{\pgfqpoint{6.027331in}{2.147222in}}{\pgfqpoint{6.027331in}{2.155459in}}%
\pgfpathcurveto{\pgfqpoint{6.027331in}{2.163695in}}{\pgfqpoint{6.024059in}{2.171595in}}{\pgfqpoint{6.018235in}{2.177419in}}%
\pgfpathcurveto{\pgfqpoint{6.012411in}{2.183243in}}{\pgfqpoint{6.004511in}{2.186515in}}{\pgfqpoint{5.996275in}{2.186515in}}%
\pgfpathcurveto{\pgfqpoint{5.988039in}{2.186515in}}{\pgfqpoint{5.980139in}{2.183243in}}{\pgfqpoint{5.974315in}{2.177419in}}%
\pgfpathcurveto{\pgfqpoint{5.968491in}{2.171595in}}{\pgfqpoint{5.965219in}{2.163695in}}{\pgfqpoint{5.965219in}{2.155459in}}%
\pgfpathcurveto{\pgfqpoint{5.965219in}{2.147222in}}{\pgfqpoint{5.968491in}{2.139322in}}{\pgfqpoint{5.974315in}{2.133498in}}%
\pgfpathcurveto{\pgfqpoint{5.980139in}{2.127675in}}{\pgfqpoint{5.988039in}{2.124402in}}{\pgfqpoint{5.996275in}{2.124402in}}%
\pgfpathclose%
\pgfusepath{stroke,fill}%
\end{pgfscope}%
\begin{pgfscope}%
\pgfpathrectangle{\pgfqpoint{0.894063in}{0.630000in}}{\pgfqpoint{6.713438in}{2.060556in}} %
\pgfusepath{clip}%
\pgfsetbuttcap%
\pgfsetroundjoin%
\definecolor{currentfill}{rgb}{0.000000,0.000000,1.000000}%
\pgfsetfillcolor{currentfill}%
\pgfsetlinewidth{1.003750pt}%
\definecolor{currentstroke}{rgb}{0.000000,0.000000,0.000000}%
\pgfsetstrokecolor{currentstroke}%
\pgfsetdash{}{0pt}%
\pgfpathmoveto{\pgfqpoint{6.399081in}{2.200278in}}%
\pgfpathcurveto{\pgfqpoint{6.407318in}{2.200278in}}{\pgfqpoint{6.415218in}{2.203550in}}{\pgfqpoint{6.421042in}{2.209374in}}%
\pgfpathcurveto{\pgfqpoint{6.426865in}{2.215198in}}{\pgfqpoint{6.430138in}{2.223098in}}{\pgfqpoint{6.430138in}{2.231334in}}%
\pgfpathcurveto{\pgfqpoint{6.430138in}{2.239571in}}{\pgfqpoint{6.426865in}{2.247471in}}{\pgfqpoint{6.421042in}{2.253295in}}%
\pgfpathcurveto{\pgfqpoint{6.415218in}{2.259118in}}{\pgfqpoint{6.407318in}{2.262391in}}{\pgfqpoint{6.399081in}{2.262391in}}%
\pgfpathcurveto{\pgfqpoint{6.390845in}{2.262391in}}{\pgfqpoint{6.382945in}{2.259118in}}{\pgfqpoint{6.377121in}{2.253295in}}%
\pgfpathcurveto{\pgfqpoint{6.371297in}{2.247471in}}{\pgfqpoint{6.368025in}{2.239571in}}{\pgfqpoint{6.368025in}{2.231334in}}%
\pgfpathcurveto{\pgfqpoint{6.368025in}{2.223098in}}{\pgfqpoint{6.371297in}{2.215198in}}{\pgfqpoint{6.377121in}{2.209374in}}%
\pgfpathcurveto{\pgfqpoint{6.382945in}{2.203550in}}{\pgfqpoint{6.390845in}{2.200278in}}{\pgfqpoint{6.399081in}{2.200278in}}%
\pgfpathclose%
\pgfusepath{stroke,fill}%
\end{pgfscope}%
\begin{pgfscope}%
\pgfpathrectangle{\pgfqpoint{0.894063in}{0.630000in}}{\pgfqpoint{6.713438in}{2.060556in}} %
\pgfusepath{clip}%
\pgfsetbuttcap%
\pgfsetroundjoin%
\definecolor{currentfill}{rgb}{0.000000,0.000000,1.000000}%
\pgfsetfillcolor{currentfill}%
\pgfsetlinewidth{1.003750pt}%
\definecolor{currentstroke}{rgb}{0.000000,0.000000,0.000000}%
\pgfsetstrokecolor{currentstroke}%
\pgfsetdash{}{0pt}%
\pgfpathmoveto{\pgfqpoint{4.787856in}{1.879814in}}%
\pgfpathcurveto{\pgfqpoint{4.796093in}{1.879814in}}{\pgfqpoint{4.803993in}{1.883087in}}{\pgfqpoint{4.809817in}{1.888911in}}%
\pgfpathcurveto{\pgfqpoint{4.815640in}{1.894734in}}{\pgfqpoint{4.818913in}{1.902634in}}{\pgfqpoint{4.818913in}{1.910871in}}%
\pgfpathcurveto{\pgfqpoint{4.818913in}{1.919107in}}{\pgfqpoint{4.815640in}{1.927007in}}{\pgfqpoint{4.809817in}{1.932831in}}%
\pgfpathcurveto{\pgfqpoint{4.803993in}{1.938655in}}{\pgfqpoint{4.796093in}{1.941927in}}{\pgfqpoint{4.787856in}{1.941927in}}%
\pgfpathcurveto{\pgfqpoint{4.779620in}{1.941927in}}{\pgfqpoint{4.771720in}{1.938655in}}{\pgfqpoint{4.765896in}{1.932831in}}%
\pgfpathcurveto{\pgfqpoint{4.760072in}{1.927007in}}{\pgfqpoint{4.756800in}{1.919107in}}{\pgfqpoint{4.756800in}{1.910871in}}%
\pgfpathcurveto{\pgfqpoint{4.756800in}{1.902634in}}{\pgfqpoint{4.760072in}{1.894734in}}{\pgfqpoint{4.765896in}{1.888911in}}%
\pgfpathcurveto{\pgfqpoint{4.771720in}{1.883087in}}{\pgfqpoint{4.779620in}{1.879814in}}{\pgfqpoint{4.787856in}{1.879814in}}%
\pgfpathclose%
\pgfusepath{stroke,fill}%
\end{pgfscope}%
\begin{pgfscope}%
\pgfpathrectangle{\pgfqpoint{0.894063in}{0.630000in}}{\pgfqpoint{6.713438in}{2.060556in}} %
\pgfusepath{clip}%
\pgfsetbuttcap%
\pgfsetroundjoin%
\definecolor{currentfill}{rgb}{0.000000,0.000000,1.000000}%
\pgfsetfillcolor{currentfill}%
\pgfsetlinewidth{1.003750pt}%
\definecolor{currentstroke}{rgb}{0.000000,0.000000,0.000000}%
\pgfsetstrokecolor{currentstroke}%
\pgfsetdash{}{0pt}%
\pgfpathmoveto{\pgfqpoint{4.922125in}{1.901638in}}%
\pgfpathcurveto{\pgfqpoint{4.930361in}{1.901638in}}{\pgfqpoint{4.938261in}{1.904911in}}{\pgfqpoint{4.944085in}{1.910735in}}%
\pgfpathcurveto{\pgfqpoint{4.949909in}{1.916559in}}{\pgfqpoint{4.953181in}{1.924459in}}{\pgfqpoint{4.953181in}{1.932695in}}%
\pgfpathcurveto{\pgfqpoint{4.953181in}{1.940931in}}{\pgfqpoint{4.949909in}{1.948831in}}{\pgfqpoint{4.944085in}{1.954655in}}%
\pgfpathcurveto{\pgfqpoint{4.938261in}{1.960479in}}{\pgfqpoint{4.930361in}{1.963751in}}{\pgfqpoint{4.922125in}{1.963751in}}%
\pgfpathcurveto{\pgfqpoint{4.913889in}{1.963751in}}{\pgfqpoint{4.905989in}{1.960479in}}{\pgfqpoint{4.900165in}{1.954655in}}%
\pgfpathcurveto{\pgfqpoint{4.894341in}{1.948831in}}{\pgfqpoint{4.891069in}{1.940931in}}{\pgfqpoint{4.891069in}{1.932695in}}%
\pgfpathcurveto{\pgfqpoint{4.891069in}{1.924459in}}{\pgfqpoint{4.894341in}{1.916559in}}{\pgfqpoint{4.900165in}{1.910735in}}%
\pgfpathcurveto{\pgfqpoint{4.905989in}{1.904911in}}{\pgfqpoint{4.913889in}{1.901638in}}{\pgfqpoint{4.922125in}{1.901638in}}%
\pgfpathclose%
\pgfusepath{stroke,fill}%
\end{pgfscope}%
\begin{pgfscope}%
\pgfpathrectangle{\pgfqpoint{0.894063in}{0.630000in}}{\pgfqpoint{6.713438in}{2.060556in}} %
\pgfusepath{clip}%
\pgfsetbuttcap%
\pgfsetroundjoin%
\definecolor{currentfill}{rgb}{0.000000,0.000000,1.000000}%
\pgfsetfillcolor{currentfill}%
\pgfsetlinewidth{1.003750pt}%
\definecolor{currentstroke}{rgb}{0.000000,0.000000,0.000000}%
\pgfsetstrokecolor{currentstroke}%
\pgfsetdash{}{0pt}%
\pgfpathmoveto{\pgfqpoint{6.130544in}{2.151213in}}%
\pgfpathcurveto{\pgfqpoint{6.138780in}{2.151213in}}{\pgfqpoint{6.146680in}{2.154485in}}{\pgfqpoint{6.152504in}{2.160309in}}%
\pgfpathcurveto{\pgfqpoint{6.158328in}{2.166133in}}{\pgfqpoint{6.161600in}{2.174033in}}{\pgfqpoint{6.161600in}{2.182269in}}%
\pgfpathcurveto{\pgfqpoint{6.161600in}{2.190506in}}{\pgfqpoint{6.158328in}{2.198406in}}{\pgfqpoint{6.152504in}{2.204230in}}%
\pgfpathcurveto{\pgfqpoint{6.146680in}{2.210054in}}{\pgfqpoint{6.138780in}{2.213326in}}{\pgfqpoint{6.130544in}{2.213326in}}%
\pgfpathcurveto{\pgfqpoint{6.122307in}{2.213326in}}{\pgfqpoint{6.114407in}{2.210054in}}{\pgfqpoint{6.108583in}{2.204230in}}%
\pgfpathcurveto{\pgfqpoint{6.102760in}{2.198406in}}{\pgfqpoint{6.099487in}{2.190506in}}{\pgfqpoint{6.099487in}{2.182269in}}%
\pgfpathcurveto{\pgfqpoint{6.099487in}{2.174033in}}{\pgfqpoint{6.102760in}{2.166133in}}{\pgfqpoint{6.108583in}{2.160309in}}%
\pgfpathcurveto{\pgfqpoint{6.114407in}{2.154485in}}{\pgfqpoint{6.122307in}{2.151213in}}{\pgfqpoint{6.130544in}{2.151213in}}%
\pgfpathclose%
\pgfusepath{stroke,fill}%
\end{pgfscope}%
\begin{pgfscope}%
\pgfpathrectangle{\pgfqpoint{0.894063in}{0.630000in}}{\pgfqpoint{6.713438in}{2.060556in}} %
\pgfusepath{clip}%
\pgfsetbuttcap%
\pgfsetroundjoin%
\definecolor{currentfill}{rgb}{0.000000,0.000000,1.000000}%
\pgfsetfillcolor{currentfill}%
\pgfsetlinewidth{1.003750pt}%
\definecolor{currentstroke}{rgb}{0.000000,0.000000,0.000000}%
\pgfsetstrokecolor{currentstroke}%
\pgfsetdash{}{0pt}%
\pgfpathmoveto{\pgfqpoint{5.727738in}{2.057305in}}%
\pgfpathcurveto{\pgfqpoint{5.735974in}{2.057305in}}{\pgfqpoint{5.743874in}{2.060577in}}{\pgfqpoint{5.749698in}{2.066401in}}%
\pgfpathcurveto{\pgfqpoint{5.755522in}{2.072225in}}{\pgfqpoint{5.758794in}{2.080125in}}{\pgfqpoint{5.758794in}{2.088361in}}%
\pgfpathcurveto{\pgfqpoint{5.758794in}{2.096597in}}{\pgfqpoint{5.755522in}{2.104497in}}{\pgfqpoint{5.749698in}{2.110321in}}%
\pgfpathcurveto{\pgfqpoint{5.743874in}{2.116145in}}{\pgfqpoint{5.735974in}{2.119418in}}{\pgfqpoint{5.727738in}{2.119418in}}%
\pgfpathcurveto{\pgfqpoint{5.719501in}{2.119418in}}{\pgfqpoint{5.711601in}{2.116145in}}{\pgfqpoint{5.705777in}{2.110321in}}%
\pgfpathcurveto{\pgfqpoint{5.699953in}{2.104497in}}{\pgfqpoint{5.696681in}{2.096597in}}{\pgfqpoint{5.696681in}{2.088361in}}%
\pgfpathcurveto{\pgfqpoint{5.696681in}{2.080125in}}{\pgfqpoint{5.699953in}{2.072225in}}{\pgfqpoint{5.705777in}{2.066401in}}%
\pgfpathcurveto{\pgfqpoint{5.711601in}{2.060577in}}{\pgfqpoint{5.719501in}{2.057305in}}{\pgfqpoint{5.727738in}{2.057305in}}%
\pgfpathclose%
\pgfusepath{stroke,fill}%
\end{pgfscope}%
\begin{pgfscope}%
\pgfpathrectangle{\pgfqpoint{0.894063in}{0.630000in}}{\pgfqpoint{6.713438in}{2.060556in}} %
\pgfusepath{clip}%
\pgfsetbuttcap%
\pgfsetroundjoin%
\definecolor{currentfill}{rgb}{0.000000,0.000000,1.000000}%
\pgfsetfillcolor{currentfill}%
\pgfsetlinewidth{1.003750pt}%
\definecolor{currentstroke}{rgb}{0.000000,0.000000,0.000000}%
\pgfsetstrokecolor{currentstroke}%
\pgfsetdash{}{0pt}%
\pgfpathmoveto{\pgfqpoint{1.028331in}{0.674166in}}%
\pgfpathcurveto{\pgfqpoint{1.036568in}{0.674166in}}{\pgfqpoint{1.044468in}{0.677438in}}{\pgfqpoint{1.050292in}{0.683262in}}%
\pgfpathcurveto{\pgfqpoint{1.056115in}{0.689086in}}{\pgfqpoint{1.059388in}{0.696986in}}{\pgfqpoint{1.059388in}{0.705222in}}%
\pgfpathcurveto{\pgfqpoint{1.059388in}{0.713458in}}{\pgfqpoint{1.056115in}{0.721358in}}{\pgfqpoint{1.050292in}{0.727182in}}%
\pgfpathcurveto{\pgfqpoint{1.044468in}{0.733006in}}{\pgfqpoint{1.036568in}{0.736279in}}{\pgfqpoint{1.028331in}{0.736279in}}%
\pgfpathcurveto{\pgfqpoint{1.020095in}{0.736279in}}{\pgfqpoint{1.012195in}{0.733006in}}{\pgfqpoint{1.006371in}{0.727182in}}%
\pgfpathcurveto{\pgfqpoint{1.000547in}{0.721358in}}{\pgfqpoint{0.997275in}{0.713458in}}{\pgfqpoint{0.997275in}{0.705222in}}%
\pgfpathcurveto{\pgfqpoint{0.997275in}{0.696986in}}{\pgfqpoint{1.000547in}{0.689086in}}{\pgfqpoint{1.006371in}{0.683262in}}%
\pgfpathcurveto{\pgfqpoint{1.012195in}{0.677438in}}{\pgfqpoint{1.020095in}{0.674166in}}{\pgfqpoint{1.028331in}{0.674166in}}%
\pgfpathclose%
\pgfusepath{stroke,fill}%
\end{pgfscope}%
\begin{pgfscope}%
\pgfpathrectangle{\pgfqpoint{0.894063in}{0.630000in}}{\pgfqpoint{6.713438in}{2.060556in}} %
\pgfusepath{clip}%
\pgfsetbuttcap%
\pgfsetroundjoin%
\definecolor{currentfill}{rgb}{0.000000,0.000000,1.000000}%
\pgfsetfillcolor{currentfill}%
\pgfsetlinewidth{1.003750pt}%
\definecolor{currentstroke}{rgb}{0.000000,0.000000,0.000000}%
\pgfsetstrokecolor{currentstroke}%
\pgfsetdash{}{0pt}%
\pgfpathmoveto{\pgfqpoint{5.324931in}{1.966982in}}%
\pgfpathcurveto{\pgfqpoint{5.333168in}{1.966982in}}{\pgfqpoint{5.341068in}{1.970254in}}{\pgfqpoint{5.346892in}{1.976078in}}%
\pgfpathcurveto{\pgfqpoint{5.352715in}{1.981902in}}{\pgfqpoint{5.355988in}{1.989802in}}{\pgfqpoint{5.355988in}{1.998038in}}%
\pgfpathcurveto{\pgfqpoint{5.355988in}{2.006274in}}{\pgfqpoint{5.352715in}{2.014174in}}{\pgfqpoint{5.346892in}{2.019998in}}%
\pgfpathcurveto{\pgfqpoint{5.341068in}{2.025822in}}{\pgfqpoint{5.333168in}{2.029095in}}{\pgfqpoint{5.324931in}{2.029095in}}%
\pgfpathcurveto{\pgfqpoint{5.316695in}{2.029095in}}{\pgfqpoint{5.308795in}{2.025822in}}{\pgfqpoint{5.302971in}{2.019998in}}%
\pgfpathcurveto{\pgfqpoint{5.297147in}{2.014174in}}{\pgfqpoint{5.293875in}{2.006274in}}{\pgfqpoint{5.293875in}{1.998038in}}%
\pgfpathcurveto{\pgfqpoint{5.293875in}{1.989802in}}{\pgfqpoint{5.297147in}{1.981902in}}{\pgfqpoint{5.302971in}{1.976078in}}%
\pgfpathcurveto{\pgfqpoint{5.308795in}{1.970254in}}{\pgfqpoint{5.316695in}{1.966982in}}{\pgfqpoint{5.324931in}{1.966982in}}%
\pgfpathclose%
\pgfusepath{stroke,fill}%
\end{pgfscope}%
\begin{pgfscope}%
\pgfpathrectangle{\pgfqpoint{0.894063in}{0.630000in}}{\pgfqpoint{6.713438in}{2.060556in}} %
\pgfusepath{clip}%
\pgfsetbuttcap%
\pgfsetroundjoin%
\definecolor{currentfill}{rgb}{0.000000,0.000000,1.000000}%
\pgfsetfillcolor{currentfill}%
\pgfsetlinewidth{1.003750pt}%
\definecolor{currentstroke}{rgb}{0.000000,0.000000,0.000000}%
\pgfsetstrokecolor{currentstroke}%
\pgfsetdash{}{0pt}%
\pgfpathmoveto{\pgfqpoint{7.338963in}{2.403602in}}%
\pgfpathcurveto{\pgfqpoint{7.347199in}{2.403602in}}{\pgfqpoint{7.355099in}{2.406874in}}{\pgfqpoint{7.360923in}{2.412698in}}%
\pgfpathcurveto{\pgfqpoint{7.366747in}{2.418522in}}{\pgfqpoint{7.370019in}{2.426422in}}{\pgfqpoint{7.370019in}{2.434658in}}%
\pgfpathcurveto{\pgfqpoint{7.370019in}{2.442894in}}{\pgfqpoint{7.366747in}{2.450794in}}{\pgfqpoint{7.360923in}{2.456618in}}%
\pgfpathcurveto{\pgfqpoint{7.355099in}{2.462442in}}{\pgfqpoint{7.347199in}{2.465715in}}{\pgfqpoint{7.338963in}{2.465715in}}%
\pgfpathcurveto{\pgfqpoint{7.330726in}{2.465715in}}{\pgfqpoint{7.322826in}{2.462442in}}{\pgfqpoint{7.317002in}{2.456618in}}%
\pgfpathcurveto{\pgfqpoint{7.311178in}{2.450794in}}{\pgfqpoint{7.307906in}{2.442894in}}{\pgfqpoint{7.307906in}{2.434658in}}%
\pgfpathcurveto{\pgfqpoint{7.307906in}{2.426422in}}{\pgfqpoint{7.311178in}{2.418522in}}{\pgfqpoint{7.317002in}{2.412698in}}%
\pgfpathcurveto{\pgfqpoint{7.322826in}{2.406874in}}{\pgfqpoint{7.330726in}{2.403602in}}{\pgfqpoint{7.338963in}{2.403602in}}%
\pgfpathclose%
\pgfusepath{stroke,fill}%
\end{pgfscope}%
\begin{pgfscope}%
\pgfpathrectangle{\pgfqpoint{0.894063in}{0.630000in}}{\pgfqpoint{6.713438in}{2.060556in}} %
\pgfusepath{clip}%
\pgfsetbuttcap%
\pgfsetroundjoin%
\definecolor{currentfill}{rgb}{0.000000,0.000000,1.000000}%
\pgfsetfillcolor{currentfill}%
\pgfsetlinewidth{1.003750pt}%
\definecolor{currentstroke}{rgb}{0.000000,0.000000,0.000000}%
\pgfsetstrokecolor{currentstroke}%
\pgfsetdash{}{0pt}%
\pgfpathmoveto{\pgfqpoint{7.204694in}{2.373288in}}%
\pgfpathcurveto{\pgfqpoint{7.212930in}{2.373288in}}{\pgfqpoint{7.220830in}{2.376560in}}{\pgfqpoint{7.226654in}{2.382384in}}%
\pgfpathcurveto{\pgfqpoint{7.232478in}{2.388208in}}{\pgfqpoint{7.235750in}{2.396108in}}{\pgfqpoint{7.235750in}{2.404344in}}%
\pgfpathcurveto{\pgfqpoint{7.235750in}{2.412581in}}{\pgfqpoint{7.232478in}{2.420481in}}{\pgfqpoint{7.226654in}{2.426305in}}%
\pgfpathcurveto{\pgfqpoint{7.220830in}{2.432129in}}{\pgfqpoint{7.212930in}{2.435401in}}{\pgfqpoint{7.204694in}{2.435401in}}%
\pgfpathcurveto{\pgfqpoint{7.196457in}{2.435401in}}{\pgfqpoint{7.188557in}{2.432129in}}{\pgfqpoint{7.182733in}{2.426305in}}%
\pgfpathcurveto{\pgfqpoint{7.176910in}{2.420481in}}{\pgfqpoint{7.173637in}{2.412581in}}{\pgfqpoint{7.173637in}{2.404344in}}%
\pgfpathcurveto{\pgfqpoint{7.173637in}{2.396108in}}{\pgfqpoint{7.176910in}{2.388208in}}{\pgfqpoint{7.182733in}{2.382384in}}%
\pgfpathcurveto{\pgfqpoint{7.188557in}{2.376560in}}{\pgfqpoint{7.196457in}{2.373288in}}{\pgfqpoint{7.204694in}{2.373288in}}%
\pgfpathclose%
\pgfusepath{stroke,fill}%
\end{pgfscope}%
\begin{pgfscope}%
\pgfpathrectangle{\pgfqpoint{0.894063in}{0.630000in}}{\pgfqpoint{6.713438in}{2.060556in}} %
\pgfusepath{clip}%
\pgfsetbuttcap%
\pgfsetroundjoin%
\definecolor{currentfill}{rgb}{0.000000,0.000000,1.000000}%
\pgfsetfillcolor{currentfill}%
\pgfsetlinewidth{1.003750pt}%
\definecolor{currentstroke}{rgb}{0.000000,0.000000,0.000000}%
\pgfsetstrokecolor{currentstroke}%
\pgfsetdash{}{0pt}%
\pgfpathmoveto{\pgfqpoint{6.264813in}{2.174150in}}%
\pgfpathcurveto{\pgfqpoint{6.273049in}{2.174150in}}{\pgfqpoint{6.280949in}{2.177422in}}{\pgfqpoint{6.286773in}{2.183246in}}%
\pgfpathcurveto{\pgfqpoint{6.292597in}{2.189070in}}{\pgfqpoint{6.295869in}{2.196970in}}{\pgfqpoint{6.295869in}{2.205206in}}%
\pgfpathcurveto{\pgfqpoint{6.295869in}{2.213443in}}{\pgfqpoint{6.292597in}{2.221343in}}{\pgfqpoint{6.286773in}{2.227167in}}%
\pgfpathcurveto{\pgfqpoint{6.280949in}{2.232991in}}{\pgfqpoint{6.273049in}{2.236263in}}{\pgfqpoint{6.264813in}{2.236263in}}%
\pgfpathcurveto{\pgfqpoint{6.256576in}{2.236263in}}{\pgfqpoint{6.248676in}{2.232991in}}{\pgfqpoint{6.242852in}{2.227167in}}%
\pgfpathcurveto{\pgfqpoint{6.237028in}{2.221343in}}{\pgfqpoint{6.233756in}{2.213443in}}{\pgfqpoint{6.233756in}{2.205206in}}%
\pgfpathcurveto{\pgfqpoint{6.233756in}{2.196970in}}{\pgfqpoint{6.237028in}{2.189070in}}{\pgfqpoint{6.242852in}{2.183246in}}%
\pgfpathcurveto{\pgfqpoint{6.248676in}{2.177422in}}{\pgfqpoint{6.256576in}{2.174150in}}{\pgfqpoint{6.264813in}{2.174150in}}%
\pgfpathclose%
\pgfusepath{stroke,fill}%
\end{pgfscope}%
\begin{pgfscope}%
\pgfpathrectangle{\pgfqpoint{0.894063in}{0.630000in}}{\pgfqpoint{6.713438in}{2.060556in}} %
\pgfusepath{clip}%
\pgfsetbuttcap%
\pgfsetroundjoin%
\definecolor{currentfill}{rgb}{0.000000,0.000000,1.000000}%
\pgfsetfillcolor{currentfill}%
\pgfsetlinewidth{1.003750pt}%
\definecolor{currentstroke}{rgb}{0.000000,0.000000,0.000000}%
\pgfsetstrokecolor{currentstroke}%
\pgfsetdash{}{0pt}%
\pgfpathmoveto{\pgfqpoint{7.473231in}{2.429512in}}%
\pgfpathcurveto{\pgfqpoint{7.481468in}{2.429512in}}{\pgfqpoint{7.489368in}{2.432784in}}{\pgfqpoint{7.495192in}{2.438608in}}%
\pgfpathcurveto{\pgfqpoint{7.501015in}{2.444432in}}{\pgfqpoint{7.504288in}{2.452332in}}{\pgfqpoint{7.504288in}{2.460568in}}%
\pgfpathcurveto{\pgfqpoint{7.504288in}{2.468804in}}{\pgfqpoint{7.501015in}{2.476704in}}{\pgfqpoint{7.495192in}{2.482528in}}%
\pgfpathcurveto{\pgfqpoint{7.489368in}{2.488352in}}{\pgfqpoint{7.481468in}{2.491625in}}{\pgfqpoint{7.473231in}{2.491625in}}%
\pgfpathcurveto{\pgfqpoint{7.464995in}{2.491625in}}{\pgfqpoint{7.457095in}{2.488352in}}{\pgfqpoint{7.451271in}{2.482528in}}%
\pgfpathcurveto{\pgfqpoint{7.445447in}{2.476704in}}{\pgfqpoint{7.442175in}{2.468804in}}{\pgfqpoint{7.442175in}{2.460568in}}%
\pgfpathcurveto{\pgfqpoint{7.442175in}{2.452332in}}{\pgfqpoint{7.445447in}{2.444432in}}{\pgfqpoint{7.451271in}{2.438608in}}%
\pgfpathcurveto{\pgfqpoint{7.457095in}{2.432784in}}{\pgfqpoint{7.464995in}{2.429512in}}{\pgfqpoint{7.473231in}{2.429512in}}%
\pgfpathclose%
\pgfusepath{stroke,fill}%
\end{pgfscope}%
\begin{pgfscope}%
\pgfpathrectangle{\pgfqpoint{0.894063in}{0.630000in}}{\pgfqpoint{6.713438in}{2.060556in}} %
\pgfusepath{clip}%
\pgfsetbuttcap%
\pgfsetroundjoin%
\definecolor{currentfill}{rgb}{0.000000,0.000000,1.000000}%
\pgfsetfillcolor{currentfill}%
\pgfsetlinewidth{1.003750pt}%
\definecolor{currentstroke}{rgb}{0.000000,0.000000,0.000000}%
\pgfsetstrokecolor{currentstroke}%
\pgfsetdash{}{0pt}%
\pgfpathmoveto{\pgfqpoint{5.056394in}{1.935820in}}%
\pgfpathcurveto{\pgfqpoint{5.064630in}{1.935820in}}{\pgfqpoint{5.072530in}{1.939092in}}{\pgfqpoint{5.078354in}{1.944916in}}%
\pgfpathcurveto{\pgfqpoint{5.084178in}{1.950740in}}{\pgfqpoint{5.087450in}{1.958640in}}{\pgfqpoint{5.087450in}{1.966877in}}%
\pgfpathcurveto{\pgfqpoint{5.087450in}{1.975113in}}{\pgfqpoint{5.084178in}{1.983013in}}{\pgfqpoint{5.078354in}{1.988837in}}%
\pgfpathcurveto{\pgfqpoint{5.072530in}{1.994661in}}{\pgfqpoint{5.064630in}{1.997933in}}{\pgfqpoint{5.056394in}{1.997933in}}%
\pgfpathcurveto{\pgfqpoint{5.048157in}{1.997933in}}{\pgfqpoint{5.040257in}{1.994661in}}{\pgfqpoint{5.034433in}{1.988837in}}%
\pgfpathcurveto{\pgfqpoint{5.028610in}{1.983013in}}{\pgfqpoint{5.025337in}{1.975113in}}{\pgfqpoint{5.025337in}{1.966877in}}%
\pgfpathcurveto{\pgfqpoint{5.025337in}{1.958640in}}{\pgfqpoint{5.028610in}{1.950740in}}{\pgfqpoint{5.034433in}{1.944916in}}%
\pgfpathcurveto{\pgfqpoint{5.040257in}{1.939092in}}{\pgfqpoint{5.048157in}{1.935820in}}{\pgfqpoint{5.056394in}{1.935820in}}%
\pgfpathclose%
\pgfusepath{stroke,fill}%
\end{pgfscope}%
\begin{pgfscope}%
\pgfpathrectangle{\pgfqpoint{0.894063in}{0.630000in}}{\pgfqpoint{6.713438in}{2.060556in}} %
\pgfusepath{clip}%
\pgfsetbuttcap%
\pgfsetroundjoin%
\definecolor{currentfill}{rgb}{0.000000,0.000000,1.000000}%
\pgfsetfillcolor{currentfill}%
\pgfsetlinewidth{1.003750pt}%
\definecolor{currentstroke}{rgb}{0.000000,0.000000,0.000000}%
\pgfsetstrokecolor{currentstroke}%
\pgfsetdash{}{0pt}%
\pgfpathmoveto{\pgfqpoint{2.908094in}{1.497846in}}%
\pgfpathcurveto{\pgfqpoint{2.916330in}{1.497846in}}{\pgfqpoint{2.924230in}{1.501118in}}{\pgfqpoint{2.930054in}{1.506942in}}%
\pgfpathcurveto{\pgfqpoint{2.935878in}{1.512766in}}{\pgfqpoint{2.939150in}{1.520666in}}{\pgfqpoint{2.939150in}{1.528903in}}%
\pgfpathcurveto{\pgfqpoint{2.939150in}{1.537139in}}{\pgfqpoint{2.935878in}{1.545039in}}{\pgfqpoint{2.930054in}{1.550863in}}%
\pgfpathcurveto{\pgfqpoint{2.924230in}{1.556687in}}{\pgfqpoint{2.916330in}{1.559959in}}{\pgfqpoint{2.908094in}{1.559959in}}%
\pgfpathcurveto{\pgfqpoint{2.899857in}{1.559959in}}{\pgfqpoint{2.891957in}{1.556687in}}{\pgfqpoint{2.886133in}{1.550863in}}%
\pgfpathcurveto{\pgfqpoint{2.880310in}{1.545039in}}{\pgfqpoint{2.877037in}{1.537139in}}{\pgfqpoint{2.877037in}{1.528903in}}%
\pgfpathcurveto{\pgfqpoint{2.877037in}{1.520666in}}{\pgfqpoint{2.880310in}{1.512766in}}{\pgfqpoint{2.886133in}{1.506942in}}%
\pgfpathcurveto{\pgfqpoint{2.891957in}{1.501118in}}{\pgfqpoint{2.899857in}{1.497846in}}{\pgfqpoint{2.908094in}{1.497846in}}%
\pgfpathclose%
\pgfusepath{stroke,fill}%
\end{pgfscope}%
\begin{pgfscope}%
\pgfpathrectangle{\pgfqpoint{0.894063in}{0.630000in}}{\pgfqpoint{6.713438in}{2.060556in}} %
\pgfusepath{clip}%
\pgfsetbuttcap%
\pgfsetroundjoin%
\definecolor{currentfill}{rgb}{0.000000,0.000000,1.000000}%
\pgfsetfillcolor{currentfill}%
\pgfsetlinewidth{1.003750pt}%
\definecolor{currentstroke}{rgb}{0.000000,0.000000,0.000000}%
\pgfsetstrokecolor{currentstroke}%
\pgfsetdash{}{0pt}%
\pgfpathmoveto{\pgfqpoint{3.445169in}{1.613225in}}%
\pgfpathcurveto{\pgfqpoint{3.453405in}{1.613225in}}{\pgfqpoint{3.461305in}{1.616498in}}{\pgfqpoint{3.467129in}{1.622322in}}%
\pgfpathcurveto{\pgfqpoint{3.472953in}{1.628146in}}{\pgfqpoint{3.476225in}{1.636046in}}{\pgfqpoint{3.476225in}{1.644282in}}%
\pgfpathcurveto{\pgfqpoint{3.476225in}{1.652518in}}{\pgfqpoint{3.472953in}{1.660418in}}{\pgfqpoint{3.467129in}{1.666242in}}%
\pgfpathcurveto{\pgfqpoint{3.461305in}{1.672066in}}{\pgfqpoint{3.453405in}{1.675338in}}{\pgfqpoint{3.445169in}{1.675338in}}%
\pgfpathcurveto{\pgfqpoint{3.436932in}{1.675338in}}{\pgfqpoint{3.429032in}{1.672066in}}{\pgfqpoint{3.423208in}{1.666242in}}%
\pgfpathcurveto{\pgfqpoint{3.417385in}{1.660418in}}{\pgfqpoint{3.414112in}{1.652518in}}{\pgfqpoint{3.414112in}{1.644282in}}%
\pgfpathcurveto{\pgfqpoint{3.414112in}{1.636046in}}{\pgfqpoint{3.417385in}{1.628146in}}{\pgfqpoint{3.423208in}{1.622322in}}%
\pgfpathcurveto{\pgfqpoint{3.429032in}{1.616498in}}{\pgfqpoint{3.436932in}{1.613225in}}{\pgfqpoint{3.445169in}{1.613225in}}%
\pgfpathclose%
\pgfusepath{stroke,fill}%
\end{pgfscope}%
\begin{pgfscope}%
\pgfpathrectangle{\pgfqpoint{0.894063in}{0.630000in}}{\pgfqpoint{6.713438in}{2.060556in}} %
\pgfusepath{clip}%
\pgfsetbuttcap%
\pgfsetroundjoin%
\definecolor{currentfill}{rgb}{0.000000,0.000000,1.000000}%
\pgfsetfillcolor{currentfill}%
\pgfsetlinewidth{1.003750pt}%
\definecolor{currentstroke}{rgb}{0.000000,0.000000,0.000000}%
\pgfsetstrokecolor{currentstroke}%
\pgfsetdash{}{0pt}%
\pgfpathmoveto{\pgfqpoint{4.116513in}{1.737995in}}%
\pgfpathcurveto{\pgfqpoint{4.124749in}{1.737995in}}{\pgfqpoint{4.132649in}{1.741267in}}{\pgfqpoint{4.138473in}{1.747091in}}%
\pgfpathcurveto{\pgfqpoint{4.144297in}{1.752915in}}{\pgfqpoint{4.147569in}{1.760815in}}{\pgfqpoint{4.147569in}{1.769052in}}%
\pgfpathcurveto{\pgfqpoint{4.147569in}{1.777288in}}{\pgfqpoint{4.144297in}{1.785188in}}{\pgfqpoint{4.138473in}{1.791012in}}%
\pgfpathcurveto{\pgfqpoint{4.132649in}{1.796836in}}{\pgfqpoint{4.124749in}{1.800108in}}{\pgfqpoint{4.116513in}{1.800108in}}%
\pgfpathcurveto{\pgfqpoint{4.108276in}{1.800108in}}{\pgfqpoint{4.100376in}{1.796836in}}{\pgfqpoint{4.094552in}{1.791012in}}%
\pgfpathcurveto{\pgfqpoint{4.088728in}{1.785188in}}{\pgfqpoint{4.085456in}{1.777288in}}{\pgfqpoint{4.085456in}{1.769052in}}%
\pgfpathcurveto{\pgfqpoint{4.085456in}{1.760815in}}{\pgfqpoint{4.088728in}{1.752915in}}{\pgfqpoint{4.094552in}{1.747091in}}%
\pgfpathcurveto{\pgfqpoint{4.100376in}{1.741267in}}{\pgfqpoint{4.108276in}{1.737995in}}{\pgfqpoint{4.116513in}{1.737995in}}%
\pgfpathclose%
\pgfusepath{stroke,fill}%
\end{pgfscope}%
\begin{pgfscope}%
\pgfpathrectangle{\pgfqpoint{0.894063in}{0.630000in}}{\pgfqpoint{6.713438in}{2.060556in}} %
\pgfusepath{clip}%
\pgfsetbuttcap%
\pgfsetroundjoin%
\definecolor{currentfill}{rgb}{0.000000,0.000000,1.000000}%
\pgfsetfillcolor{currentfill}%
\pgfsetlinewidth{1.003750pt}%
\definecolor{currentstroke}{rgb}{0.000000,0.000000,0.000000}%
\pgfsetstrokecolor{currentstroke}%
\pgfsetdash{}{0pt}%
\pgfpathmoveto{\pgfqpoint{1.431138in}{1.263714in}}%
\pgfpathcurveto{\pgfqpoint{1.439374in}{1.263714in}}{\pgfqpoint{1.447274in}{1.266986in}}{\pgfqpoint{1.453098in}{1.272810in}}%
\pgfpathcurveto{\pgfqpoint{1.458922in}{1.278634in}}{\pgfqpoint{1.462194in}{1.286534in}}{\pgfqpoint{1.462194in}{1.294771in}}%
\pgfpathcurveto{\pgfqpoint{1.462194in}{1.303007in}}{\pgfqpoint{1.458922in}{1.310907in}}{\pgfqpoint{1.453098in}{1.316731in}}%
\pgfpathcurveto{\pgfqpoint{1.447274in}{1.322555in}}{\pgfqpoint{1.439374in}{1.325827in}}{\pgfqpoint{1.431138in}{1.325827in}}%
\pgfpathcurveto{\pgfqpoint{1.422901in}{1.325827in}}{\pgfqpoint{1.415001in}{1.322555in}}{\pgfqpoint{1.409177in}{1.316731in}}%
\pgfpathcurveto{\pgfqpoint{1.403353in}{1.310907in}}{\pgfqpoint{1.400081in}{1.303007in}}{\pgfqpoint{1.400081in}{1.294771in}}%
\pgfpathcurveto{\pgfqpoint{1.400081in}{1.286534in}}{\pgfqpoint{1.403353in}{1.278634in}}{\pgfqpoint{1.409177in}{1.272810in}}%
\pgfpathcurveto{\pgfqpoint{1.415001in}{1.266986in}}{\pgfqpoint{1.422901in}{1.263714in}}{\pgfqpoint{1.431138in}{1.263714in}}%
\pgfpathclose%
\pgfusepath{stroke,fill}%
\end{pgfscope}%
\begin{pgfscope}%
\pgfpathrectangle{\pgfqpoint{0.894063in}{0.630000in}}{\pgfqpoint{6.713438in}{2.060556in}} %
\pgfusepath{clip}%
\pgfsetbuttcap%
\pgfsetroundjoin%
\definecolor{currentfill}{rgb}{0.000000,0.000000,1.000000}%
\pgfsetfillcolor{currentfill}%
\pgfsetlinewidth{1.003750pt}%
\definecolor{currentstroke}{rgb}{0.000000,0.000000,0.000000}%
\pgfsetstrokecolor{currentstroke}%
\pgfsetdash{}{0pt}%
\pgfpathmoveto{\pgfqpoint{2.773825in}{1.470400in}}%
\pgfpathcurveto{\pgfqpoint{2.782061in}{1.470400in}}{\pgfqpoint{2.789961in}{1.473672in}}{\pgfqpoint{2.795785in}{1.479496in}}%
\pgfpathcurveto{\pgfqpoint{2.801609in}{1.485320in}}{\pgfqpoint{2.804881in}{1.493220in}}{\pgfqpoint{2.804881in}{1.501456in}}%
\pgfpathcurveto{\pgfqpoint{2.804881in}{1.509692in}}{\pgfqpoint{2.801609in}{1.517592in}}{\pgfqpoint{2.795785in}{1.523416in}}%
\pgfpathcurveto{\pgfqpoint{2.789961in}{1.529240in}}{\pgfqpoint{2.782061in}{1.532513in}}{\pgfqpoint{2.773825in}{1.532513in}}%
\pgfpathcurveto{\pgfqpoint{2.765589in}{1.532513in}}{\pgfqpoint{2.757689in}{1.529240in}}{\pgfqpoint{2.751865in}{1.523416in}}%
\pgfpathcurveto{\pgfqpoint{2.746041in}{1.517592in}}{\pgfqpoint{2.742769in}{1.509692in}}{\pgfqpoint{2.742769in}{1.501456in}}%
\pgfpathcurveto{\pgfqpoint{2.742769in}{1.493220in}}{\pgfqpoint{2.746041in}{1.485320in}}{\pgfqpoint{2.751865in}{1.479496in}}%
\pgfpathcurveto{\pgfqpoint{2.757689in}{1.473672in}}{\pgfqpoint{2.765589in}{1.470400in}}{\pgfqpoint{2.773825in}{1.470400in}}%
\pgfpathclose%
\pgfusepath{stroke,fill}%
\end{pgfscope}%
\begin{pgfscope}%
\pgfpathrectangle{\pgfqpoint{0.894063in}{0.630000in}}{\pgfqpoint{6.713438in}{2.060556in}} %
\pgfusepath{clip}%
\pgfsetbuttcap%
\pgfsetroundjoin%
\definecolor{currentfill}{rgb}{0.000000,0.000000,1.000000}%
\pgfsetfillcolor{currentfill}%
\pgfsetlinewidth{1.003750pt}%
\definecolor{currentstroke}{rgb}{0.000000,0.000000,0.000000}%
\pgfsetstrokecolor{currentstroke}%
\pgfsetdash{}{0pt}%
\pgfpathmoveto{\pgfqpoint{1.565406in}{1.273281in}}%
\pgfpathcurveto{\pgfqpoint{1.573643in}{1.273281in}}{\pgfqpoint{1.581543in}{1.276553in}}{\pgfqpoint{1.587367in}{1.282377in}}%
\pgfpathcurveto{\pgfqpoint{1.593190in}{1.288201in}}{\pgfqpoint{1.596463in}{1.296101in}}{\pgfqpoint{1.596463in}{1.304337in}}%
\pgfpathcurveto{\pgfqpoint{1.596463in}{1.312574in}}{\pgfqpoint{1.593190in}{1.320474in}}{\pgfqpoint{1.587367in}{1.326298in}}%
\pgfpathcurveto{\pgfqpoint{1.581543in}{1.332122in}}{\pgfqpoint{1.573643in}{1.335394in}}{\pgfqpoint{1.565406in}{1.335394in}}%
\pgfpathcurveto{\pgfqpoint{1.557170in}{1.335394in}}{\pgfqpoint{1.549270in}{1.332122in}}{\pgfqpoint{1.543446in}{1.326298in}}%
\pgfpathcurveto{\pgfqpoint{1.537622in}{1.320474in}}{\pgfqpoint{1.534350in}{1.312574in}}{\pgfqpoint{1.534350in}{1.304337in}}%
\pgfpathcurveto{\pgfqpoint{1.534350in}{1.296101in}}{\pgfqpoint{1.537622in}{1.288201in}}{\pgfqpoint{1.543446in}{1.282377in}}%
\pgfpathcurveto{\pgfqpoint{1.549270in}{1.276553in}}{\pgfqpoint{1.557170in}{1.273281in}}{\pgfqpoint{1.565406in}{1.273281in}}%
\pgfpathclose%
\pgfusepath{stroke,fill}%
\end{pgfscope}%
\begin{pgfscope}%
\pgfpathrectangle{\pgfqpoint{0.894063in}{0.630000in}}{\pgfqpoint{6.713438in}{2.060556in}} %
\pgfusepath{clip}%
\pgfsetbuttcap%
\pgfsetroundjoin%
\definecolor{currentfill}{rgb}{0.000000,0.000000,1.000000}%
\pgfsetfillcolor{currentfill}%
\pgfsetlinewidth{1.003750pt}%
\definecolor{currentstroke}{rgb}{0.000000,0.000000,0.000000}%
\pgfsetstrokecolor{currentstroke}%
\pgfsetdash{}{0pt}%
\pgfpathmoveto{\pgfqpoint{4.250781in}{1.773072in}}%
\pgfpathcurveto{\pgfqpoint{4.259018in}{1.773072in}}{\pgfqpoint{4.266918in}{1.776344in}}{\pgfqpoint{4.272742in}{1.782168in}}%
\pgfpathcurveto{\pgfqpoint{4.278565in}{1.787992in}}{\pgfqpoint{4.281838in}{1.795892in}}{\pgfqpoint{4.281838in}{1.804128in}}%
\pgfpathcurveto{\pgfqpoint{4.281838in}{1.812364in}}{\pgfqpoint{4.278565in}{1.820264in}}{\pgfqpoint{4.272742in}{1.826088in}}%
\pgfpathcurveto{\pgfqpoint{4.266918in}{1.831912in}}{\pgfqpoint{4.259018in}{1.835185in}}{\pgfqpoint{4.250781in}{1.835185in}}%
\pgfpathcurveto{\pgfqpoint{4.242545in}{1.835185in}}{\pgfqpoint{4.234645in}{1.831912in}}{\pgfqpoint{4.228821in}{1.826088in}}%
\pgfpathcurveto{\pgfqpoint{4.222997in}{1.820264in}}{\pgfqpoint{4.219725in}{1.812364in}}{\pgfqpoint{4.219725in}{1.804128in}}%
\pgfpathcurveto{\pgfqpoint{4.219725in}{1.795892in}}{\pgfqpoint{4.222997in}{1.787992in}}{\pgfqpoint{4.228821in}{1.782168in}}%
\pgfpathcurveto{\pgfqpoint{4.234645in}{1.776344in}}{\pgfqpoint{4.242545in}{1.773072in}}{\pgfqpoint{4.250781in}{1.773072in}}%
\pgfpathclose%
\pgfusepath{stroke,fill}%
\end{pgfscope}%
\begin{pgfscope}%
\pgfpathrectangle{\pgfqpoint{0.894063in}{0.630000in}}{\pgfqpoint{6.713438in}{2.060556in}} %
\pgfusepath{clip}%
\pgfsetbuttcap%
\pgfsetroundjoin%
\definecolor{currentfill}{rgb}{0.000000,0.000000,1.000000}%
\pgfsetfillcolor{currentfill}%
\pgfsetlinewidth{1.003750pt}%
\definecolor{currentstroke}{rgb}{0.000000,0.000000,0.000000}%
\pgfsetstrokecolor{currentstroke}%
\pgfsetdash{}{0pt}%
\pgfpathmoveto{\pgfqpoint{3.847975in}{1.694211in}}%
\pgfpathcurveto{\pgfqpoint{3.856211in}{1.694211in}}{\pgfqpoint{3.864111in}{1.697484in}}{\pgfqpoint{3.869935in}{1.703307in}}%
\pgfpathcurveto{\pgfqpoint{3.875759in}{1.709131in}}{\pgfqpoint{3.879031in}{1.717031in}}{\pgfqpoint{3.879031in}{1.725268in}}%
\pgfpathcurveto{\pgfqpoint{3.879031in}{1.733504in}}{\pgfqpoint{3.875759in}{1.741404in}}{\pgfqpoint{3.869935in}{1.747228in}}%
\pgfpathcurveto{\pgfqpoint{3.864111in}{1.753052in}}{\pgfqpoint{3.856211in}{1.756324in}}{\pgfqpoint{3.847975in}{1.756324in}}%
\pgfpathcurveto{\pgfqpoint{3.839739in}{1.756324in}}{\pgfqpoint{3.831839in}{1.753052in}}{\pgfqpoint{3.826015in}{1.747228in}}%
\pgfpathcurveto{\pgfqpoint{3.820191in}{1.741404in}}{\pgfqpoint{3.816919in}{1.733504in}}{\pgfqpoint{3.816919in}{1.725268in}}%
\pgfpathcurveto{\pgfqpoint{3.816919in}{1.717031in}}{\pgfqpoint{3.820191in}{1.709131in}}{\pgfqpoint{3.826015in}{1.703307in}}%
\pgfpathcurveto{\pgfqpoint{3.831839in}{1.697484in}}{\pgfqpoint{3.839739in}{1.694211in}}{\pgfqpoint{3.847975in}{1.694211in}}%
\pgfpathclose%
\pgfusepath{stroke,fill}%
\end{pgfscope}%
\begin{pgfscope}%
\pgfpathrectangle{\pgfqpoint{0.894063in}{0.630000in}}{\pgfqpoint{6.713438in}{2.060556in}} %
\pgfusepath{clip}%
\pgfsetbuttcap%
\pgfsetroundjoin%
\definecolor{currentfill}{rgb}{0.000000,0.000000,1.000000}%
\pgfsetfillcolor{currentfill}%
\pgfsetlinewidth{1.003750pt}%
\definecolor{currentstroke}{rgb}{0.000000,0.000000,0.000000}%
\pgfsetstrokecolor{currentstroke}%
\pgfsetdash{}{0pt}%
\pgfpathmoveto{\pgfqpoint{7.607500in}{2.457364in}}%
\pgfpathcurveto{\pgfqpoint{7.615736in}{2.457364in}}{\pgfqpoint{7.623636in}{2.460637in}}{\pgfqpoint{7.629460in}{2.466461in}}%
\pgfpathcurveto{\pgfqpoint{7.635284in}{2.472285in}}{\pgfqpoint{7.638556in}{2.480185in}}{\pgfqpoint{7.638556in}{2.488421in}}%
\pgfpathcurveto{\pgfqpoint{7.638556in}{2.496657in}}{\pgfqpoint{7.635284in}{2.504557in}}{\pgfqpoint{7.629460in}{2.510381in}}%
\pgfpathcurveto{\pgfqpoint{7.623636in}{2.516205in}}{\pgfqpoint{7.615736in}{2.519477in}}{\pgfqpoint{7.607500in}{2.519477in}}%
\pgfpathcurveto{\pgfqpoint{7.599264in}{2.519477in}}{\pgfqpoint{7.591364in}{2.516205in}}{\pgfqpoint{7.585540in}{2.510381in}}%
\pgfpathcurveto{\pgfqpoint{7.579716in}{2.504557in}}{\pgfqpoint{7.576444in}{2.496657in}}{\pgfqpoint{7.576444in}{2.488421in}}%
\pgfpathcurveto{\pgfqpoint{7.576444in}{2.480185in}}{\pgfqpoint{7.579716in}{2.472285in}}{\pgfqpoint{7.585540in}{2.466461in}}%
\pgfpathcurveto{\pgfqpoint{7.591364in}{2.460637in}}{\pgfqpoint{7.599264in}{2.457364in}}{\pgfqpoint{7.607500in}{2.457364in}}%
\pgfpathclose%
\pgfusepath{stroke,fill}%
\end{pgfscope}%
\begin{pgfscope}%
\pgfpathrectangle{\pgfqpoint{0.894063in}{0.630000in}}{\pgfqpoint{6.713438in}{2.060556in}} %
\pgfusepath{clip}%
\pgfsetbuttcap%
\pgfsetroundjoin%
\definecolor{currentfill}{rgb}{0.000000,0.000000,1.000000}%
\pgfsetfillcolor{currentfill}%
\pgfsetlinewidth{1.003750pt}%
\definecolor{currentstroke}{rgb}{0.000000,0.000000,0.000000}%
\pgfsetstrokecolor{currentstroke}%
\pgfsetdash{}{0pt}%
\pgfpathmoveto{\pgfqpoint{4.385050in}{1.796756in}}%
\pgfpathcurveto{\pgfqpoint{4.393286in}{1.796756in}}{\pgfqpoint{4.401186in}{1.800029in}}{\pgfqpoint{4.407010in}{1.805852in}}%
\pgfpathcurveto{\pgfqpoint{4.412834in}{1.811676in}}{\pgfqpoint{4.416106in}{1.819576in}}{\pgfqpoint{4.416106in}{1.827813in}}%
\pgfpathcurveto{\pgfqpoint{4.416106in}{1.836049in}}{\pgfqpoint{4.412834in}{1.843949in}}{\pgfqpoint{4.407010in}{1.849773in}}%
\pgfpathcurveto{\pgfqpoint{4.401186in}{1.855597in}}{\pgfqpoint{4.393286in}{1.858869in}}{\pgfqpoint{4.385050in}{1.858869in}}%
\pgfpathcurveto{\pgfqpoint{4.376814in}{1.858869in}}{\pgfqpoint{4.368914in}{1.855597in}}{\pgfqpoint{4.363090in}{1.849773in}}%
\pgfpathcurveto{\pgfqpoint{4.357266in}{1.843949in}}{\pgfqpoint{4.353994in}{1.836049in}}{\pgfqpoint{4.353994in}{1.827813in}}%
\pgfpathcurveto{\pgfqpoint{4.353994in}{1.819576in}}{\pgfqpoint{4.357266in}{1.811676in}}{\pgfqpoint{4.363090in}{1.805852in}}%
\pgfpathcurveto{\pgfqpoint{4.368914in}{1.800029in}}{\pgfqpoint{4.376814in}{1.796756in}}{\pgfqpoint{4.385050in}{1.796756in}}%
\pgfpathclose%
\pgfusepath{stroke,fill}%
\end{pgfscope}%
\begin{pgfscope}%
\pgfpathrectangle{\pgfqpoint{0.894063in}{0.630000in}}{\pgfqpoint{6.713438in}{2.060556in}} %
\pgfusepath{clip}%
\pgfsetbuttcap%
\pgfsetroundjoin%
\definecolor{currentfill}{rgb}{0.000000,0.000000,1.000000}%
\pgfsetfillcolor{currentfill}%
\pgfsetlinewidth{1.003750pt}%
\definecolor{currentstroke}{rgb}{0.000000,0.000000,0.000000}%
\pgfsetstrokecolor{currentstroke}%
\pgfsetdash{}{0pt}%
\pgfpathmoveto{\pgfqpoint{6.533350in}{2.230079in}}%
\pgfpathcurveto{\pgfqpoint{6.541586in}{2.230079in}}{\pgfqpoint{6.549486in}{2.233352in}}{\pgfqpoint{6.555310in}{2.239176in}}%
\pgfpathcurveto{\pgfqpoint{6.561134in}{2.244999in}}{\pgfqpoint{6.564406in}{2.252899in}}{\pgfqpoint{6.564406in}{2.261136in}}%
\pgfpathcurveto{\pgfqpoint{6.564406in}{2.269372in}}{\pgfqpoint{6.561134in}{2.277272in}}{\pgfqpoint{6.555310in}{2.283096in}}%
\pgfpathcurveto{\pgfqpoint{6.549486in}{2.288920in}}{\pgfqpoint{6.541586in}{2.292192in}}{\pgfqpoint{6.533350in}{2.292192in}}%
\pgfpathcurveto{\pgfqpoint{6.525114in}{2.292192in}}{\pgfqpoint{6.517214in}{2.288920in}}{\pgfqpoint{6.511390in}{2.283096in}}%
\pgfpathcurveto{\pgfqpoint{6.505566in}{2.277272in}}{\pgfqpoint{6.502294in}{2.269372in}}{\pgfqpoint{6.502294in}{2.261136in}}%
\pgfpathcurveto{\pgfqpoint{6.502294in}{2.252899in}}{\pgfqpoint{6.505566in}{2.244999in}}{\pgfqpoint{6.511390in}{2.239176in}}%
\pgfpathcurveto{\pgfqpoint{6.517214in}{2.233352in}}{\pgfqpoint{6.525114in}{2.230079in}}{\pgfqpoint{6.533350in}{2.230079in}}%
\pgfpathclose%
\pgfusepath{stroke,fill}%
\end{pgfscope}%
\begin{pgfscope}%
\pgfpathrectangle{\pgfqpoint{0.894063in}{0.630000in}}{\pgfqpoint{6.713438in}{2.060556in}} %
\pgfusepath{clip}%
\pgfsetbuttcap%
\pgfsetroundjoin%
\definecolor{currentfill}{rgb}{0.000000,0.000000,1.000000}%
\pgfsetfillcolor{currentfill}%
\pgfsetlinewidth{1.003750pt}%
\definecolor{currentstroke}{rgb}{0.000000,0.000000,0.000000}%
\pgfsetstrokecolor{currentstroke}%
\pgfsetdash{}{0pt}%
\pgfpathmoveto{\pgfqpoint{1.296869in}{1.251633in}}%
\pgfpathcurveto{\pgfqpoint{1.305105in}{1.251633in}}{\pgfqpoint{1.313005in}{1.254906in}}{\pgfqpoint{1.318829in}{1.260730in}}%
\pgfpathcurveto{\pgfqpoint{1.324653in}{1.266553in}}{\pgfqpoint{1.327925in}{1.274454in}}{\pgfqpoint{1.327925in}{1.282690in}}%
\pgfpathcurveto{\pgfqpoint{1.327925in}{1.290926in}}{\pgfqpoint{1.324653in}{1.298826in}}{\pgfqpoint{1.318829in}{1.304650in}}%
\pgfpathcurveto{\pgfqpoint{1.313005in}{1.310474in}}{\pgfqpoint{1.305105in}{1.313746in}}{\pgfqpoint{1.296869in}{1.313746in}}%
\pgfpathcurveto{\pgfqpoint{1.288632in}{1.313746in}}{\pgfqpoint{1.280732in}{1.310474in}}{\pgfqpoint{1.274908in}{1.304650in}}%
\pgfpathcurveto{\pgfqpoint{1.269085in}{1.298826in}}{\pgfqpoint{1.265812in}{1.290926in}}{\pgfqpoint{1.265812in}{1.282690in}}%
\pgfpathcurveto{\pgfqpoint{1.265812in}{1.274454in}}{\pgfqpoint{1.269085in}{1.266553in}}{\pgfqpoint{1.274908in}{1.260730in}}%
\pgfpathcurveto{\pgfqpoint{1.280732in}{1.254906in}}{\pgfqpoint{1.288632in}{1.251633in}}{\pgfqpoint{1.296869in}{1.251633in}}%
\pgfpathclose%
\pgfusepath{stroke,fill}%
\end{pgfscope}%
\begin{pgfscope}%
\pgfpathrectangle{\pgfqpoint{0.894063in}{0.630000in}}{\pgfqpoint{6.713438in}{2.060556in}} %
\pgfusepath{clip}%
\pgfsetbuttcap%
\pgfsetroundjoin%
\definecolor{currentfill}{rgb}{0.000000,0.000000,1.000000}%
\pgfsetfillcolor{currentfill}%
\pgfsetlinewidth{1.003750pt}%
\definecolor{currentstroke}{rgb}{0.000000,0.000000,0.000000}%
\pgfsetstrokecolor{currentstroke}%
\pgfsetdash{}{0pt}%
\pgfpathmoveto{\pgfqpoint{4.519319in}{1.820935in}}%
\pgfpathcurveto{\pgfqpoint{4.527555in}{1.820935in}}{\pgfqpoint{4.535455in}{1.824208in}}{\pgfqpoint{4.541279in}{1.830032in}}%
\pgfpathcurveto{\pgfqpoint{4.547103in}{1.835856in}}{\pgfqpoint{4.550375in}{1.843756in}}{\pgfqpoint{4.550375in}{1.851992in}}%
\pgfpathcurveto{\pgfqpoint{4.550375in}{1.860228in}}{\pgfqpoint{4.547103in}{1.868128in}}{\pgfqpoint{4.541279in}{1.873952in}}%
\pgfpathcurveto{\pgfqpoint{4.535455in}{1.879776in}}{\pgfqpoint{4.527555in}{1.883048in}}{\pgfqpoint{4.519319in}{1.883048in}}%
\pgfpathcurveto{\pgfqpoint{4.511082in}{1.883048in}}{\pgfqpoint{4.503182in}{1.879776in}}{\pgfqpoint{4.497358in}{1.873952in}}%
\pgfpathcurveto{\pgfqpoint{4.491535in}{1.868128in}}{\pgfqpoint{4.488262in}{1.860228in}}{\pgfqpoint{4.488262in}{1.851992in}}%
\pgfpathcurveto{\pgfqpoint{4.488262in}{1.843756in}}{\pgfqpoint{4.491535in}{1.835856in}}{\pgfqpoint{4.497358in}{1.830032in}}%
\pgfpathcurveto{\pgfqpoint{4.503182in}{1.824208in}}{\pgfqpoint{4.511082in}{1.820935in}}{\pgfqpoint{4.519319in}{1.820935in}}%
\pgfpathclose%
\pgfusepath{stroke,fill}%
\end{pgfscope}%
\begin{pgfscope}%
\pgfpathrectangle{\pgfqpoint{0.894063in}{0.630000in}}{\pgfqpoint{6.713438in}{2.060556in}} %
\pgfusepath{clip}%
\pgfsetbuttcap%
\pgfsetroundjoin%
\definecolor{currentfill}{rgb}{0.000000,0.000000,1.000000}%
\pgfsetfillcolor{currentfill}%
\pgfsetlinewidth{1.003750pt}%
\definecolor{currentstroke}{rgb}{0.000000,0.000000,0.000000}%
\pgfsetstrokecolor{currentstroke}%
\pgfsetdash{}{0pt}%
\pgfpathmoveto{\pgfqpoint{2.505288in}{1.419486in}}%
\pgfpathcurveto{\pgfqpoint{2.513524in}{1.419486in}}{\pgfqpoint{2.521424in}{1.422758in}}{\pgfqpoint{2.527248in}{1.428582in}}%
\pgfpathcurveto{\pgfqpoint{2.533072in}{1.434406in}}{\pgfqpoint{2.536344in}{1.442306in}}{\pgfqpoint{2.536344in}{1.450543in}}%
\pgfpathcurveto{\pgfqpoint{2.536344in}{1.458779in}}{\pgfqpoint{2.533072in}{1.466679in}}{\pgfqpoint{2.527248in}{1.472503in}}%
\pgfpathcurveto{\pgfqpoint{2.521424in}{1.478327in}}{\pgfqpoint{2.513524in}{1.481599in}}{\pgfqpoint{2.505288in}{1.481599in}}%
\pgfpathcurveto{\pgfqpoint{2.497051in}{1.481599in}}{\pgfqpoint{2.489151in}{1.478327in}}{\pgfqpoint{2.483327in}{1.472503in}}%
\pgfpathcurveto{\pgfqpoint{2.477503in}{1.466679in}}{\pgfqpoint{2.474231in}{1.458779in}}{\pgfqpoint{2.474231in}{1.450543in}}%
\pgfpathcurveto{\pgfqpoint{2.474231in}{1.442306in}}{\pgfqpoint{2.477503in}{1.434406in}}{\pgfqpoint{2.483327in}{1.428582in}}%
\pgfpathcurveto{\pgfqpoint{2.489151in}{1.422758in}}{\pgfqpoint{2.497051in}{1.419486in}}{\pgfqpoint{2.505288in}{1.419486in}}%
\pgfpathclose%
\pgfusepath{stroke,fill}%
\end{pgfscope}%
\begin{pgfscope}%
\pgfpathrectangle{\pgfqpoint{0.894063in}{0.630000in}}{\pgfqpoint{6.713438in}{2.060556in}} %
\pgfusepath{clip}%
\pgfsetbuttcap%
\pgfsetroundjoin%
\definecolor{currentfill}{rgb}{0.000000,0.000000,1.000000}%
\pgfsetfillcolor{currentfill}%
\pgfsetlinewidth{1.003750pt}%
\definecolor{currentstroke}{rgb}{0.000000,0.000000,0.000000}%
\pgfsetstrokecolor{currentstroke}%
\pgfsetdash{}{0pt}%
\pgfpathmoveto{\pgfqpoint{5.459200in}{2.001363in}}%
\pgfpathcurveto{\pgfqpoint{5.467436in}{2.001363in}}{\pgfqpoint{5.475336in}{2.004636in}}{\pgfqpoint{5.481160in}{2.010460in}}%
\pgfpathcurveto{\pgfqpoint{5.486984in}{2.016284in}}{\pgfqpoint{5.490256in}{2.024184in}}{\pgfqpoint{5.490256in}{2.032420in}}%
\pgfpathcurveto{\pgfqpoint{5.490256in}{2.040656in}}{\pgfqpoint{5.486984in}{2.048556in}}{\pgfqpoint{5.481160in}{2.054380in}}%
\pgfpathcurveto{\pgfqpoint{5.475336in}{2.060204in}}{\pgfqpoint{5.467436in}{2.063476in}}{\pgfqpoint{5.459200in}{2.063476in}}%
\pgfpathcurveto{\pgfqpoint{5.450964in}{2.063476in}}{\pgfqpoint{5.443064in}{2.060204in}}{\pgfqpoint{5.437240in}{2.054380in}}%
\pgfpathcurveto{\pgfqpoint{5.431416in}{2.048556in}}{\pgfqpoint{5.428144in}{2.040656in}}{\pgfqpoint{5.428144in}{2.032420in}}%
\pgfpathcurveto{\pgfqpoint{5.428144in}{2.024184in}}{\pgfqpoint{5.431416in}{2.016284in}}{\pgfqpoint{5.437240in}{2.010460in}}%
\pgfpathcurveto{\pgfqpoint{5.443064in}{2.004636in}}{\pgfqpoint{5.450964in}{2.001363in}}{\pgfqpoint{5.459200in}{2.001363in}}%
\pgfpathclose%
\pgfusepath{stroke,fill}%
\end{pgfscope}%
\begin{pgfscope}%
\pgfpathrectangle{\pgfqpoint{0.894063in}{0.630000in}}{\pgfqpoint{6.713438in}{2.060556in}} %
\pgfusepath{clip}%
\pgfsetbuttcap%
\pgfsetroundjoin%
\definecolor{currentfill}{rgb}{0.000000,0.000000,1.000000}%
\pgfsetfillcolor{currentfill}%
\pgfsetlinewidth{1.003750pt}%
\definecolor{currentstroke}{rgb}{0.000000,0.000000,0.000000}%
\pgfsetstrokecolor{currentstroke}%
\pgfsetdash{}{0pt}%
\pgfpathmoveto{\pgfqpoint{6.936156in}{2.305925in}}%
\pgfpathcurveto{\pgfqpoint{6.944393in}{2.305925in}}{\pgfqpoint{6.952293in}{2.309198in}}{\pgfqpoint{6.958117in}{2.315022in}}%
\pgfpathcurveto{\pgfqpoint{6.963940in}{2.320846in}}{\pgfqpoint{6.967213in}{2.328746in}}{\pgfqpoint{6.967213in}{2.336982in}}%
\pgfpathcurveto{\pgfqpoint{6.967213in}{2.345218in}}{\pgfqpoint{6.963940in}{2.353118in}}{\pgfqpoint{6.958117in}{2.358942in}}%
\pgfpathcurveto{\pgfqpoint{6.952293in}{2.364766in}}{\pgfqpoint{6.944393in}{2.368038in}}{\pgfqpoint{6.936156in}{2.368038in}}%
\pgfpathcurveto{\pgfqpoint{6.927920in}{2.368038in}}{\pgfqpoint{6.920020in}{2.364766in}}{\pgfqpoint{6.914196in}{2.358942in}}%
\pgfpathcurveto{\pgfqpoint{6.908372in}{2.353118in}}{\pgfqpoint{6.905100in}{2.345218in}}{\pgfqpoint{6.905100in}{2.336982in}}%
\pgfpathcurveto{\pgfqpoint{6.905100in}{2.328746in}}{\pgfqpoint{6.908372in}{2.320846in}}{\pgfqpoint{6.914196in}{2.315022in}}%
\pgfpathcurveto{\pgfqpoint{6.920020in}{2.309198in}}{\pgfqpoint{6.927920in}{2.305925in}}{\pgfqpoint{6.936156in}{2.305925in}}%
\pgfpathclose%
\pgfusepath{stroke,fill}%
\end{pgfscope}%
\begin{pgfscope}%
\pgfpathrectangle{\pgfqpoint{0.894063in}{0.630000in}}{\pgfqpoint{6.713438in}{2.060556in}} %
\pgfusepath{clip}%
\pgfsetbuttcap%
\pgfsetroundjoin%
\definecolor{currentfill}{rgb}{0.000000,0.000000,1.000000}%
\pgfsetfillcolor{currentfill}%
\pgfsetlinewidth{1.003750pt}%
\definecolor{currentstroke}{rgb}{0.000000,0.000000,0.000000}%
\pgfsetstrokecolor{currentstroke}%
\pgfsetdash{}{0pt}%
\pgfpathmoveto{\pgfqpoint{5.862006in}{2.092417in}}%
\pgfpathcurveto{\pgfqpoint{5.870243in}{2.092417in}}{\pgfqpoint{5.878143in}{2.095689in}}{\pgfqpoint{5.883967in}{2.101513in}}%
\pgfpathcurveto{\pgfqpoint{5.889790in}{2.107337in}}{\pgfqpoint{5.893063in}{2.115237in}}{\pgfqpoint{5.893063in}{2.123473in}}%
\pgfpathcurveto{\pgfqpoint{5.893063in}{2.131709in}}{\pgfqpoint{5.889790in}{2.139609in}}{\pgfqpoint{5.883967in}{2.145433in}}%
\pgfpathcurveto{\pgfqpoint{5.878143in}{2.151257in}}{\pgfqpoint{5.870243in}{2.154530in}}{\pgfqpoint{5.862006in}{2.154530in}}%
\pgfpathcurveto{\pgfqpoint{5.853770in}{2.154530in}}{\pgfqpoint{5.845870in}{2.151257in}}{\pgfqpoint{5.840046in}{2.145433in}}%
\pgfpathcurveto{\pgfqpoint{5.834222in}{2.139609in}}{\pgfqpoint{5.830950in}{2.131709in}}{\pgfqpoint{5.830950in}{2.123473in}}%
\pgfpathcurveto{\pgfqpoint{5.830950in}{2.115237in}}{\pgfqpoint{5.834222in}{2.107337in}}{\pgfqpoint{5.840046in}{2.101513in}}%
\pgfpathcurveto{\pgfqpoint{5.845870in}{2.095689in}}{\pgfqpoint{5.853770in}{2.092417in}}{\pgfqpoint{5.862006in}{2.092417in}}%
\pgfpathclose%
\pgfusepath{stroke,fill}%
\end{pgfscope}%
\begin{pgfscope}%
\pgfpathrectangle{\pgfqpoint{0.894063in}{0.630000in}}{\pgfqpoint{6.713438in}{2.060556in}} %
\pgfusepath{clip}%
\pgfsetbuttcap%
\pgfsetroundjoin%
\definecolor{currentfill}{rgb}{0.000000,0.000000,1.000000}%
\pgfsetfillcolor{currentfill}%
\pgfsetlinewidth{1.003750pt}%
\definecolor{currentstroke}{rgb}{0.000000,0.000000,0.000000}%
\pgfsetstrokecolor{currentstroke}%
\pgfsetdash{}{0pt}%
\pgfpathmoveto{\pgfqpoint{7.070425in}{2.336475in}}%
\pgfpathcurveto{\pgfqpoint{7.078661in}{2.336475in}}{\pgfqpoint{7.086561in}{2.339747in}}{\pgfqpoint{7.092385in}{2.345571in}}%
\pgfpathcurveto{\pgfqpoint{7.098209in}{2.351395in}}{\pgfqpoint{7.101481in}{2.359295in}}{\pgfqpoint{7.101481in}{2.367531in}}%
\pgfpathcurveto{\pgfqpoint{7.101481in}{2.375767in}}{\pgfqpoint{7.098209in}{2.383667in}}{\pgfqpoint{7.092385in}{2.389491in}}%
\pgfpathcurveto{\pgfqpoint{7.086561in}{2.395315in}}{\pgfqpoint{7.078661in}{2.398588in}}{\pgfqpoint{7.070425in}{2.398588in}}%
\pgfpathcurveto{\pgfqpoint{7.062189in}{2.398588in}}{\pgfqpoint{7.054289in}{2.395315in}}{\pgfqpoint{7.048465in}{2.389491in}}%
\pgfpathcurveto{\pgfqpoint{7.042641in}{2.383667in}}{\pgfqpoint{7.039369in}{2.375767in}}{\pgfqpoint{7.039369in}{2.367531in}}%
\pgfpathcurveto{\pgfqpoint{7.039369in}{2.359295in}}{\pgfqpoint{7.042641in}{2.351395in}}{\pgfqpoint{7.048465in}{2.345571in}}%
\pgfpathcurveto{\pgfqpoint{7.054289in}{2.339747in}}{\pgfqpoint{7.062189in}{2.336475in}}{\pgfqpoint{7.070425in}{2.336475in}}%
\pgfpathclose%
\pgfusepath{stroke,fill}%
\end{pgfscope}%
\begin{pgfscope}%
\pgfpathrectangle{\pgfqpoint{0.894063in}{0.630000in}}{\pgfqpoint{6.713438in}{2.060556in}} %
\pgfusepath{clip}%
\pgfsetbuttcap%
\pgfsetroundjoin%
\definecolor{currentfill}{rgb}{0.000000,0.000000,1.000000}%
\pgfsetfillcolor{currentfill}%
\pgfsetlinewidth{1.003750pt}%
\definecolor{currentstroke}{rgb}{0.000000,0.000000,0.000000}%
\pgfsetstrokecolor{currentstroke}%
\pgfsetdash{}{0pt}%
\pgfpathmoveto{\pgfqpoint{3.176631in}{1.548000in}}%
\pgfpathcurveto{\pgfqpoint{3.184868in}{1.548000in}}{\pgfqpoint{3.192768in}{1.551272in}}{\pgfqpoint{3.198592in}{1.557096in}}%
\pgfpathcurveto{\pgfqpoint{3.204415in}{1.562920in}}{\pgfqpoint{3.207688in}{1.570820in}}{\pgfqpoint{3.207688in}{1.579057in}}%
\pgfpathcurveto{\pgfqpoint{3.207688in}{1.587293in}}{\pgfqpoint{3.204415in}{1.595193in}}{\pgfqpoint{3.198592in}{1.601017in}}%
\pgfpathcurveto{\pgfqpoint{3.192768in}{1.606841in}}{\pgfqpoint{3.184868in}{1.610113in}}{\pgfqpoint{3.176631in}{1.610113in}}%
\pgfpathcurveto{\pgfqpoint{3.168395in}{1.610113in}}{\pgfqpoint{3.160495in}{1.606841in}}{\pgfqpoint{3.154671in}{1.601017in}}%
\pgfpathcurveto{\pgfqpoint{3.148847in}{1.595193in}}{\pgfqpoint{3.145575in}{1.587293in}}{\pgfqpoint{3.145575in}{1.579057in}}%
\pgfpathcurveto{\pgfqpoint{3.145575in}{1.570820in}}{\pgfqpoint{3.148847in}{1.562920in}}{\pgfqpoint{3.154671in}{1.557096in}}%
\pgfpathcurveto{\pgfqpoint{3.160495in}{1.551272in}}{\pgfqpoint{3.168395in}{1.548000in}}{\pgfqpoint{3.176631in}{1.548000in}}%
\pgfpathclose%
\pgfusepath{stroke,fill}%
\end{pgfscope}%
\begin{pgfscope}%
\pgfpathrectangle{\pgfqpoint{0.894063in}{0.630000in}}{\pgfqpoint{6.713438in}{2.060556in}} %
\pgfusepath{clip}%
\pgfsetbuttcap%
\pgfsetroundjoin%
\definecolor{currentfill}{rgb}{0.000000,0.000000,1.000000}%
\pgfsetfillcolor{currentfill}%
\pgfsetlinewidth{1.003750pt}%
\definecolor{currentstroke}{rgb}{0.000000,0.000000,0.000000}%
\pgfsetstrokecolor{currentstroke}%
\pgfsetdash{}{0pt}%
\pgfpathmoveto{\pgfqpoint{2.102481in}{1.355786in}}%
\pgfpathcurveto{\pgfqpoint{2.110718in}{1.355786in}}{\pgfqpoint{2.118618in}{1.359058in}}{\pgfqpoint{2.124442in}{1.364882in}}%
\pgfpathcurveto{\pgfqpoint{2.130265in}{1.370706in}}{\pgfqpoint{2.133538in}{1.378606in}}{\pgfqpoint{2.133538in}{1.386842in}}%
\pgfpathcurveto{\pgfqpoint{2.133538in}{1.395078in}}{\pgfqpoint{2.130265in}{1.402978in}}{\pgfqpoint{2.124442in}{1.408802in}}%
\pgfpathcurveto{\pgfqpoint{2.118618in}{1.414626in}}{\pgfqpoint{2.110718in}{1.417899in}}{\pgfqpoint{2.102481in}{1.417899in}}%
\pgfpathcurveto{\pgfqpoint{2.094245in}{1.417899in}}{\pgfqpoint{2.086345in}{1.414626in}}{\pgfqpoint{2.080521in}{1.408802in}}%
\pgfpathcurveto{\pgfqpoint{2.074697in}{1.402978in}}{\pgfqpoint{2.071425in}{1.395078in}}{\pgfqpoint{2.071425in}{1.386842in}}%
\pgfpathcurveto{\pgfqpoint{2.071425in}{1.378606in}}{\pgfqpoint{2.074697in}{1.370706in}}{\pgfqpoint{2.080521in}{1.364882in}}%
\pgfpathcurveto{\pgfqpoint{2.086345in}{1.359058in}}{\pgfqpoint{2.094245in}{1.355786in}}{\pgfqpoint{2.102481in}{1.355786in}}%
\pgfpathclose%
\pgfusepath{stroke,fill}%
\end{pgfscope}%
\begin{pgfscope}%
\pgfpathrectangle{\pgfqpoint{0.894063in}{0.630000in}}{\pgfqpoint{6.713438in}{2.060556in}} %
\pgfusepath{clip}%
\pgfsetbuttcap%
\pgfsetroundjoin%
\definecolor{currentfill}{rgb}{0.000000,0.000000,1.000000}%
\pgfsetfillcolor{currentfill}%
\pgfsetlinewidth{1.003750pt}%
\definecolor{currentstroke}{rgb}{0.000000,0.000000,0.000000}%
\pgfsetstrokecolor{currentstroke}%
\pgfsetdash{}{0pt}%
\pgfpathmoveto{\pgfqpoint{1.968213in}{1.331689in}}%
\pgfpathcurveto{\pgfqpoint{1.976449in}{1.331689in}}{\pgfqpoint{1.984349in}{1.334961in}}{\pgfqpoint{1.990173in}{1.340785in}}%
\pgfpathcurveto{\pgfqpoint{1.995997in}{1.346609in}}{\pgfqpoint{1.999269in}{1.354509in}}{\pgfqpoint{1.999269in}{1.362745in}}%
\pgfpathcurveto{\pgfqpoint{1.999269in}{1.370982in}}{\pgfqpoint{1.995997in}{1.378882in}}{\pgfqpoint{1.990173in}{1.384706in}}%
\pgfpathcurveto{\pgfqpoint{1.984349in}{1.390530in}}{\pgfqpoint{1.976449in}{1.393802in}}{\pgfqpoint{1.968213in}{1.393802in}}%
\pgfpathcurveto{\pgfqpoint{1.959976in}{1.393802in}}{\pgfqpoint{1.952076in}{1.390530in}}{\pgfqpoint{1.946252in}{1.384706in}}%
\pgfpathcurveto{\pgfqpoint{1.940428in}{1.378882in}}{\pgfqpoint{1.937156in}{1.370982in}}{\pgfqpoint{1.937156in}{1.362745in}}%
\pgfpathcurveto{\pgfqpoint{1.937156in}{1.354509in}}{\pgfqpoint{1.940428in}{1.346609in}}{\pgfqpoint{1.946252in}{1.340785in}}%
\pgfpathcurveto{\pgfqpoint{1.952076in}{1.334961in}}{\pgfqpoint{1.959976in}{1.331689in}}{\pgfqpoint{1.968213in}{1.331689in}}%
\pgfpathclose%
\pgfusepath{stroke,fill}%
\end{pgfscope}%
\begin{pgfscope}%
\pgfpathrectangle{\pgfqpoint{0.894063in}{0.630000in}}{\pgfqpoint{6.713438in}{2.060556in}} %
\pgfusepath{clip}%
\pgfsetbuttcap%
\pgfsetroundjoin%
\definecolor{currentfill}{rgb}{0.000000,0.000000,1.000000}%
\pgfsetfillcolor{currentfill}%
\pgfsetlinewidth{1.003750pt}%
\definecolor{currentstroke}{rgb}{0.000000,0.000000,0.000000}%
\pgfsetstrokecolor{currentstroke}%
\pgfsetdash{}{0pt}%
\pgfpathmoveto{\pgfqpoint{3.310900in}{1.575753in}}%
\pgfpathcurveto{\pgfqpoint{3.319136in}{1.575753in}}{\pgfqpoint{3.327036in}{1.579025in}}{\pgfqpoint{3.332860in}{1.584849in}}%
\pgfpathcurveto{\pgfqpoint{3.338684in}{1.590673in}}{\pgfqpoint{3.341956in}{1.598573in}}{\pgfqpoint{3.341956in}{1.606809in}}%
\pgfpathcurveto{\pgfqpoint{3.341956in}{1.615046in}}{\pgfqpoint{3.338684in}{1.622946in}}{\pgfqpoint{3.332860in}{1.628770in}}%
\pgfpathcurveto{\pgfqpoint{3.327036in}{1.634593in}}{\pgfqpoint{3.319136in}{1.637866in}}{\pgfqpoint{3.310900in}{1.637866in}}%
\pgfpathcurveto{\pgfqpoint{3.302664in}{1.637866in}}{\pgfqpoint{3.294764in}{1.634593in}}{\pgfqpoint{3.288940in}{1.628770in}}%
\pgfpathcurveto{\pgfqpoint{3.283116in}{1.622946in}}{\pgfqpoint{3.279844in}{1.615046in}}{\pgfqpoint{3.279844in}{1.606809in}}%
\pgfpathcurveto{\pgfqpoint{3.279844in}{1.598573in}}{\pgfqpoint{3.283116in}{1.590673in}}{\pgfqpoint{3.288940in}{1.584849in}}%
\pgfpathcurveto{\pgfqpoint{3.294764in}{1.579025in}}{\pgfqpoint{3.302664in}{1.575753in}}{\pgfqpoint{3.310900in}{1.575753in}}%
\pgfpathclose%
\pgfusepath{stroke,fill}%
\end{pgfscope}%
\begin{pgfscope}%
\pgfpathrectangle{\pgfqpoint{0.894063in}{0.630000in}}{\pgfqpoint{6.713438in}{2.060556in}} %
\pgfusepath{clip}%
\pgfsetbuttcap%
\pgfsetroundjoin%
\definecolor{currentfill}{rgb}{0.000000,0.000000,1.000000}%
\pgfsetfillcolor{currentfill}%
\pgfsetlinewidth{1.003750pt}%
\definecolor{currentstroke}{rgb}{0.000000,0.000000,0.000000}%
\pgfsetstrokecolor{currentstroke}%
\pgfsetdash{}{0pt}%
\pgfpathmoveto{\pgfqpoint{5.593469in}{2.027191in}}%
\pgfpathcurveto{\pgfqpoint{5.601705in}{2.027191in}}{\pgfqpoint{5.609605in}{2.030463in}}{\pgfqpoint{5.615429in}{2.036287in}}%
\pgfpathcurveto{\pgfqpoint{5.621253in}{2.042111in}}{\pgfqpoint{5.624525in}{2.050011in}}{\pgfqpoint{5.624525in}{2.058248in}}%
\pgfpathcurveto{\pgfqpoint{5.624525in}{2.066484in}}{\pgfqpoint{5.621253in}{2.074384in}}{\pgfqpoint{5.615429in}{2.080208in}}%
\pgfpathcurveto{\pgfqpoint{5.609605in}{2.086032in}}{\pgfqpoint{5.601705in}{2.089304in}}{\pgfqpoint{5.593469in}{2.089304in}}%
\pgfpathcurveto{\pgfqpoint{5.585232in}{2.089304in}}{\pgfqpoint{5.577332in}{2.086032in}}{\pgfqpoint{5.571508in}{2.080208in}}%
\pgfpathcurveto{\pgfqpoint{5.565685in}{2.074384in}}{\pgfqpoint{5.562412in}{2.066484in}}{\pgfqpoint{5.562412in}{2.058248in}}%
\pgfpathcurveto{\pgfqpoint{5.562412in}{2.050011in}}{\pgfqpoint{5.565685in}{2.042111in}}{\pgfqpoint{5.571508in}{2.036287in}}%
\pgfpathcurveto{\pgfqpoint{5.577332in}{2.030463in}}{\pgfqpoint{5.585232in}{2.027191in}}{\pgfqpoint{5.593469in}{2.027191in}}%
\pgfpathclose%
\pgfusepath{stroke,fill}%
\end{pgfscope}%
\begin{pgfscope}%
\pgfpathrectangle{\pgfqpoint{0.894063in}{0.630000in}}{\pgfqpoint{6.713438in}{2.060556in}} %
\pgfusepath{clip}%
\pgfsetbuttcap%
\pgfsetroundjoin%
\definecolor{currentfill}{rgb}{0.000000,0.000000,1.000000}%
\pgfsetfillcolor{currentfill}%
\pgfsetlinewidth{1.003750pt}%
\definecolor{currentstroke}{rgb}{0.000000,0.000000,0.000000}%
\pgfsetstrokecolor{currentstroke}%
\pgfsetdash{}{0pt}%
\pgfpathmoveto{\pgfqpoint{3.042363in}{1.522108in}}%
\pgfpathcurveto{\pgfqpoint{3.050599in}{1.522108in}}{\pgfqpoint{3.058499in}{1.525380in}}{\pgfqpoint{3.064323in}{1.531204in}}%
\pgfpathcurveto{\pgfqpoint{3.070147in}{1.537028in}}{\pgfqpoint{3.073419in}{1.544928in}}{\pgfqpoint{3.073419in}{1.553164in}}%
\pgfpathcurveto{\pgfqpoint{3.073419in}{1.561400in}}{\pgfqpoint{3.070147in}{1.569301in}}{\pgfqpoint{3.064323in}{1.575124in}}%
\pgfpathcurveto{\pgfqpoint{3.058499in}{1.580948in}}{\pgfqpoint{3.050599in}{1.584221in}}{\pgfqpoint{3.042363in}{1.584221in}}%
\pgfpathcurveto{\pgfqpoint{3.034126in}{1.584221in}}{\pgfqpoint{3.026226in}{1.580948in}}{\pgfqpoint{3.020402in}{1.575124in}}%
\pgfpathcurveto{\pgfqpoint{3.014578in}{1.569301in}}{\pgfqpoint{3.011306in}{1.561400in}}{\pgfqpoint{3.011306in}{1.553164in}}%
\pgfpathcurveto{\pgfqpoint{3.011306in}{1.544928in}}{\pgfqpoint{3.014578in}{1.537028in}}{\pgfqpoint{3.020402in}{1.531204in}}%
\pgfpathcurveto{\pgfqpoint{3.026226in}{1.525380in}}{\pgfqpoint{3.034126in}{1.522108in}}{\pgfqpoint{3.042363in}{1.522108in}}%
\pgfpathclose%
\pgfusepath{stroke,fill}%
\end{pgfscope}%
\begin{pgfscope}%
\pgfpathrectangle{\pgfqpoint{0.894063in}{0.630000in}}{\pgfqpoint{6.713438in}{2.060556in}} %
\pgfusepath{clip}%
\pgfsetbuttcap%
\pgfsetroundjoin%
\definecolor{currentfill}{rgb}{0.000000,0.000000,1.000000}%
\pgfsetfillcolor{currentfill}%
\pgfsetlinewidth{1.003750pt}%
\definecolor{currentstroke}{rgb}{0.000000,0.000000,0.000000}%
\pgfsetstrokecolor{currentstroke}%
\pgfsetdash{}{0pt}%
\pgfpathmoveto{\pgfqpoint{5.190663in}{1.955625in}}%
\pgfpathcurveto{\pgfqpoint{5.198899in}{1.955625in}}{\pgfqpoint{5.206799in}{1.958897in}}{\pgfqpoint{5.212623in}{1.964721in}}%
\pgfpathcurveto{\pgfqpoint{5.218447in}{1.970545in}}{\pgfqpoint{5.221719in}{1.978445in}}{\pgfqpoint{5.221719in}{1.986682in}}%
\pgfpathcurveto{\pgfqpoint{5.221719in}{1.994918in}}{\pgfqpoint{5.218447in}{2.002818in}}{\pgfqpoint{5.212623in}{2.008642in}}%
\pgfpathcurveto{\pgfqpoint{5.206799in}{2.014466in}}{\pgfqpoint{5.198899in}{2.017738in}}{\pgfqpoint{5.190663in}{2.017738in}}%
\pgfpathcurveto{\pgfqpoint{5.182426in}{2.017738in}}{\pgfqpoint{5.174526in}{2.014466in}}{\pgfqpoint{5.168702in}{2.008642in}}%
\pgfpathcurveto{\pgfqpoint{5.162878in}{2.002818in}}{\pgfqpoint{5.159606in}{1.994918in}}{\pgfqpoint{5.159606in}{1.986682in}}%
\pgfpathcurveto{\pgfqpoint{5.159606in}{1.978445in}}{\pgfqpoint{5.162878in}{1.970545in}}{\pgfqpoint{5.168702in}{1.964721in}}%
\pgfpathcurveto{\pgfqpoint{5.174526in}{1.958897in}}{\pgfqpoint{5.182426in}{1.955625in}}{\pgfqpoint{5.190663in}{1.955625in}}%
\pgfpathclose%
\pgfusepath{stroke,fill}%
\end{pgfscope}%
\begin{pgfscope}%
\pgfpathrectangle{\pgfqpoint{0.894063in}{0.630000in}}{\pgfqpoint{6.713438in}{2.060556in}} %
\pgfusepath{clip}%
\pgfsetbuttcap%
\pgfsetroundjoin%
\definecolor{currentfill}{rgb}{0.000000,0.000000,1.000000}%
\pgfsetfillcolor{currentfill}%
\pgfsetlinewidth{1.003750pt}%
\definecolor{currentstroke}{rgb}{0.000000,0.000000,0.000000}%
\pgfsetstrokecolor{currentstroke}%
\pgfsetdash{}{0pt}%
\pgfpathmoveto{\pgfqpoint{6.801888in}{2.278196in}}%
\pgfpathcurveto{\pgfqpoint{6.810124in}{2.278196in}}{\pgfqpoint{6.818024in}{2.281469in}}{\pgfqpoint{6.823848in}{2.287292in}}%
\pgfpathcurveto{\pgfqpoint{6.829672in}{2.293116in}}{\pgfqpoint{6.832944in}{2.301016in}}{\pgfqpoint{6.832944in}{2.309253in}}%
\pgfpathcurveto{\pgfqpoint{6.832944in}{2.317489in}}{\pgfqpoint{6.829672in}{2.325389in}}{\pgfqpoint{6.823848in}{2.331213in}}%
\pgfpathcurveto{\pgfqpoint{6.818024in}{2.337037in}}{\pgfqpoint{6.810124in}{2.340309in}}{\pgfqpoint{6.801888in}{2.340309in}}%
\pgfpathcurveto{\pgfqpoint{6.793651in}{2.340309in}}{\pgfqpoint{6.785751in}{2.337037in}}{\pgfqpoint{6.779927in}{2.331213in}}%
\pgfpathcurveto{\pgfqpoint{6.774103in}{2.325389in}}{\pgfqpoint{6.770831in}{2.317489in}}{\pgfqpoint{6.770831in}{2.309253in}}%
\pgfpathcurveto{\pgfqpoint{6.770831in}{2.301016in}}{\pgfqpoint{6.774103in}{2.293116in}}{\pgfqpoint{6.779927in}{2.287292in}}%
\pgfpathcurveto{\pgfqpoint{6.785751in}{2.281469in}}{\pgfqpoint{6.793651in}{2.278196in}}{\pgfqpoint{6.801888in}{2.278196in}}%
\pgfpathclose%
\pgfusepath{stroke,fill}%
\end{pgfscope}%
\begin{pgfscope}%
\pgfpathrectangle{\pgfqpoint{0.894063in}{0.630000in}}{\pgfqpoint{6.713438in}{2.060556in}} %
\pgfusepath{clip}%
\pgfsetbuttcap%
\pgfsetroundjoin%
\definecolor{currentfill}{rgb}{0.000000,0.000000,1.000000}%
\pgfsetfillcolor{currentfill}%
\pgfsetlinewidth{1.003750pt}%
\definecolor{currentstroke}{rgb}{0.000000,0.000000,0.000000}%
\pgfsetstrokecolor{currentstroke}%
\pgfsetdash{}{0pt}%
\pgfpathmoveto{\pgfqpoint{3.579438in}{1.639830in}}%
\pgfpathcurveto{\pgfqpoint{3.587674in}{1.639830in}}{\pgfqpoint{3.595574in}{1.643103in}}{\pgfqpoint{3.601398in}{1.648926in}}%
\pgfpathcurveto{\pgfqpoint{3.607222in}{1.654750in}}{\pgfqpoint{3.610494in}{1.662650in}}{\pgfqpoint{3.610494in}{1.670887in}}%
\pgfpathcurveto{\pgfqpoint{3.610494in}{1.679123in}}{\pgfqpoint{3.607222in}{1.687023in}}{\pgfqpoint{3.601398in}{1.692847in}}%
\pgfpathcurveto{\pgfqpoint{3.595574in}{1.698671in}}{\pgfqpoint{3.587674in}{1.701943in}}{\pgfqpoint{3.579438in}{1.701943in}}%
\pgfpathcurveto{\pgfqpoint{3.571201in}{1.701943in}}{\pgfqpoint{3.563301in}{1.698671in}}{\pgfqpoint{3.557477in}{1.692847in}}%
\pgfpathcurveto{\pgfqpoint{3.551653in}{1.687023in}}{\pgfqpoint{3.548381in}{1.679123in}}{\pgfqpoint{3.548381in}{1.670887in}}%
\pgfpathcurveto{\pgfqpoint{3.548381in}{1.662650in}}{\pgfqpoint{3.551653in}{1.654750in}}{\pgfqpoint{3.557477in}{1.648926in}}%
\pgfpathcurveto{\pgfqpoint{3.563301in}{1.643103in}}{\pgfqpoint{3.571201in}{1.639830in}}{\pgfqpoint{3.579438in}{1.639830in}}%
\pgfpathclose%
\pgfusepath{stroke,fill}%
\end{pgfscope}%
\begin{pgfscope}%
\pgfpathrectangle{\pgfqpoint{0.894063in}{0.630000in}}{\pgfqpoint{6.713438in}{2.060556in}} %
\pgfusepath{clip}%
\pgfsetbuttcap%
\pgfsetroundjoin%
\definecolor{currentfill}{rgb}{0.000000,0.000000,1.000000}%
\pgfsetfillcolor{currentfill}%
\pgfsetlinewidth{1.003750pt}%
\definecolor{currentstroke}{rgb}{0.000000,0.000000,0.000000}%
\pgfsetstrokecolor{currentstroke}%
\pgfsetdash{}{0pt}%
\pgfpathmoveto{\pgfqpoint{2.371019in}{1.386429in}}%
\pgfpathcurveto{\pgfqpoint{2.379255in}{1.386429in}}{\pgfqpoint{2.387155in}{1.389701in}}{\pgfqpoint{2.392979in}{1.395525in}}%
\pgfpathcurveto{\pgfqpoint{2.398803in}{1.401349in}}{\pgfqpoint{2.402075in}{1.409249in}}{\pgfqpoint{2.402075in}{1.417485in}}%
\pgfpathcurveto{\pgfqpoint{2.402075in}{1.425722in}}{\pgfqpoint{2.398803in}{1.433622in}}{\pgfqpoint{2.392979in}{1.439446in}}%
\pgfpathcurveto{\pgfqpoint{2.387155in}{1.445270in}}{\pgfqpoint{2.379255in}{1.448542in}}{\pgfqpoint{2.371019in}{1.448542in}}%
\pgfpathcurveto{\pgfqpoint{2.362782in}{1.448542in}}{\pgfqpoint{2.354882in}{1.445270in}}{\pgfqpoint{2.349058in}{1.439446in}}%
\pgfpathcurveto{\pgfqpoint{2.343235in}{1.433622in}}{\pgfqpoint{2.339962in}{1.425722in}}{\pgfqpoint{2.339962in}{1.417485in}}%
\pgfpathcurveto{\pgfqpoint{2.339962in}{1.409249in}}{\pgfqpoint{2.343235in}{1.401349in}}{\pgfqpoint{2.349058in}{1.395525in}}%
\pgfpathcurveto{\pgfqpoint{2.354882in}{1.389701in}}{\pgfqpoint{2.362782in}{1.386429in}}{\pgfqpoint{2.371019in}{1.386429in}}%
\pgfpathclose%
\pgfusepath{stroke,fill}%
\end{pgfscope}%
\begin{pgfscope}%
\pgfpathrectangle{\pgfqpoint{0.894063in}{0.630000in}}{\pgfqpoint{6.713438in}{2.060556in}} %
\pgfusepath{clip}%
\pgfsetbuttcap%
\pgfsetroundjoin%
\definecolor{currentfill}{rgb}{0.000000,0.000000,1.000000}%
\pgfsetfillcolor{currentfill}%
\pgfsetlinewidth{1.003750pt}%
\definecolor{currentstroke}{rgb}{0.000000,0.000000,0.000000}%
\pgfsetstrokecolor{currentstroke}%
\pgfsetdash{}{0pt}%
\pgfpathmoveto{\pgfqpoint{3.982244in}{1.712609in}}%
\pgfpathcurveto{\pgfqpoint{3.990480in}{1.712609in}}{\pgfqpoint{3.998380in}{1.715881in}}{\pgfqpoint{4.004204in}{1.721705in}}%
\pgfpathcurveto{\pgfqpoint{4.010028in}{1.727529in}}{\pgfqpoint{4.013300in}{1.735429in}}{\pgfqpoint{4.013300in}{1.743666in}}%
\pgfpathcurveto{\pgfqpoint{4.013300in}{1.751902in}}{\pgfqpoint{4.010028in}{1.759802in}}{\pgfqpoint{4.004204in}{1.765626in}}%
\pgfpathcurveto{\pgfqpoint{3.998380in}{1.771450in}}{\pgfqpoint{3.990480in}{1.774722in}}{\pgfqpoint{3.982244in}{1.774722in}}%
\pgfpathcurveto{\pgfqpoint{3.974007in}{1.774722in}}{\pgfqpoint{3.966107in}{1.771450in}}{\pgfqpoint{3.960283in}{1.765626in}}%
\pgfpathcurveto{\pgfqpoint{3.954460in}{1.759802in}}{\pgfqpoint{3.951187in}{1.751902in}}{\pgfqpoint{3.951187in}{1.743666in}}%
\pgfpathcurveto{\pgfqpoint{3.951187in}{1.735429in}}{\pgfqpoint{3.954460in}{1.727529in}}{\pgfqpoint{3.960283in}{1.721705in}}%
\pgfpathcurveto{\pgfqpoint{3.966107in}{1.715881in}}{\pgfqpoint{3.974007in}{1.712609in}}{\pgfqpoint{3.982244in}{1.712609in}}%
\pgfpathclose%
\pgfusepath{stroke,fill}%
\end{pgfscope}%
\begin{pgfscope}%
\pgfpathrectangle{\pgfqpoint{0.894063in}{0.630000in}}{\pgfqpoint{6.713438in}{2.060556in}} %
\pgfusepath{clip}%
\pgfsetbuttcap%
\pgfsetroundjoin%
\definecolor{currentfill}{rgb}{0.000000,0.000000,1.000000}%
\pgfsetfillcolor{currentfill}%
\pgfsetlinewidth{1.003750pt}%
\definecolor{currentstroke}{rgb}{0.000000,0.000000,0.000000}%
\pgfsetstrokecolor{currentstroke}%
\pgfsetdash{}{0pt}%
\pgfpathmoveto{\pgfqpoint{4.653588in}{1.851196in}}%
\pgfpathcurveto{\pgfqpoint{4.661824in}{1.851196in}}{\pgfqpoint{4.669724in}{1.854468in}}{\pgfqpoint{4.675548in}{1.860292in}}%
\pgfpathcurveto{\pgfqpoint{4.681372in}{1.866116in}}{\pgfqpoint{4.684644in}{1.874016in}}{\pgfqpoint{4.684644in}{1.882253in}}%
\pgfpathcurveto{\pgfqpoint{4.684644in}{1.890489in}}{\pgfqpoint{4.681372in}{1.898389in}}{\pgfqpoint{4.675548in}{1.904213in}}%
\pgfpathcurveto{\pgfqpoint{4.669724in}{1.910037in}}{\pgfqpoint{4.661824in}{1.913309in}}{\pgfqpoint{4.653588in}{1.913309in}}%
\pgfpathcurveto{\pgfqpoint{4.645351in}{1.913309in}}{\pgfqpoint{4.637451in}{1.910037in}}{\pgfqpoint{4.631627in}{1.904213in}}%
\pgfpathcurveto{\pgfqpoint{4.625803in}{1.898389in}}{\pgfqpoint{4.622531in}{1.890489in}}{\pgfqpoint{4.622531in}{1.882253in}}%
\pgfpathcurveto{\pgfqpoint{4.622531in}{1.874016in}}{\pgfqpoint{4.625803in}{1.866116in}}{\pgfqpoint{4.631627in}{1.860292in}}%
\pgfpathcurveto{\pgfqpoint{4.637451in}{1.854468in}}{\pgfqpoint{4.645351in}{1.851196in}}{\pgfqpoint{4.653588in}{1.851196in}}%
\pgfpathclose%
\pgfusepath{stroke,fill}%
\end{pgfscope}%
\begin{pgfscope}%
\pgfpathrectangle{\pgfqpoint{0.894063in}{0.630000in}}{\pgfqpoint{6.713438in}{2.060556in}} %
\pgfusepath{clip}%
\pgfsetbuttcap%
\pgfsetroundjoin%
\definecolor{currentfill}{rgb}{0.000000,0.000000,1.000000}%
\pgfsetfillcolor{currentfill}%
\pgfsetlinewidth{1.003750pt}%
\definecolor{currentstroke}{rgb}{0.000000,0.000000,0.000000}%
\pgfsetstrokecolor{currentstroke}%
\pgfsetdash{}{0pt}%
\pgfpathmoveto{\pgfqpoint{3.713706in}{1.662426in}}%
\pgfpathcurveto{\pgfqpoint{3.721943in}{1.662426in}}{\pgfqpoint{3.729843in}{1.665698in}}{\pgfqpoint{3.735667in}{1.671522in}}%
\pgfpathcurveto{\pgfqpoint{3.741490in}{1.677346in}}{\pgfqpoint{3.744763in}{1.685246in}}{\pgfqpoint{3.744763in}{1.693482in}}%
\pgfpathcurveto{\pgfqpoint{3.744763in}{1.701718in}}{\pgfqpoint{3.741490in}{1.709618in}}{\pgfqpoint{3.735667in}{1.715442in}}%
\pgfpathcurveto{\pgfqpoint{3.729843in}{1.721266in}}{\pgfqpoint{3.721943in}{1.724539in}}{\pgfqpoint{3.713706in}{1.724539in}}%
\pgfpathcurveto{\pgfqpoint{3.705470in}{1.724539in}}{\pgfqpoint{3.697570in}{1.721266in}}{\pgfqpoint{3.691746in}{1.715442in}}%
\pgfpathcurveto{\pgfqpoint{3.685922in}{1.709618in}}{\pgfqpoint{3.682650in}{1.701718in}}{\pgfqpoint{3.682650in}{1.693482in}}%
\pgfpathcurveto{\pgfqpoint{3.682650in}{1.685246in}}{\pgfqpoint{3.685922in}{1.677346in}}{\pgfqpoint{3.691746in}{1.671522in}}%
\pgfpathcurveto{\pgfqpoint{3.697570in}{1.665698in}}{\pgfqpoint{3.705470in}{1.662426in}}{\pgfqpoint{3.713706in}{1.662426in}}%
\pgfpathclose%
\pgfusepath{stroke,fill}%
\end{pgfscope}%
\begin{pgfscope}%
\pgfpathrectangle{\pgfqpoint{0.894063in}{0.630000in}}{\pgfqpoint{6.713438in}{2.060556in}} %
\pgfusepath{clip}%
\pgfsetbuttcap%
\pgfsetroundjoin%
\definecolor{currentfill}{rgb}{0.000000,0.000000,1.000000}%
\pgfsetfillcolor{currentfill}%
\pgfsetlinewidth{1.003750pt}%
\definecolor{currentstroke}{rgb}{0.000000,0.000000,0.000000}%
\pgfsetstrokecolor{currentstroke}%
\pgfsetdash{}{0pt}%
\pgfpathmoveto{\pgfqpoint{2.236750in}{1.371817in}}%
\pgfpathcurveto{\pgfqpoint{2.244986in}{1.371817in}}{\pgfqpoint{2.252886in}{1.375089in}}{\pgfqpoint{2.258710in}{1.380913in}}%
\pgfpathcurveto{\pgfqpoint{2.264534in}{1.386737in}}{\pgfqpoint{2.267806in}{1.394637in}}{\pgfqpoint{2.267806in}{1.402873in}}%
\pgfpathcurveto{\pgfqpoint{2.267806in}{1.411109in}}{\pgfqpoint{2.264534in}{1.419010in}}{\pgfqpoint{2.258710in}{1.424833in}}%
\pgfpathcurveto{\pgfqpoint{2.252886in}{1.430657in}}{\pgfqpoint{2.244986in}{1.433930in}}{\pgfqpoint{2.236750in}{1.433930in}}%
\pgfpathcurveto{\pgfqpoint{2.228514in}{1.433930in}}{\pgfqpoint{2.220614in}{1.430657in}}{\pgfqpoint{2.214790in}{1.424833in}}%
\pgfpathcurveto{\pgfqpoint{2.208966in}{1.419010in}}{\pgfqpoint{2.205694in}{1.411109in}}{\pgfqpoint{2.205694in}{1.402873in}}%
\pgfpathcurveto{\pgfqpoint{2.205694in}{1.394637in}}{\pgfqpoint{2.208966in}{1.386737in}}{\pgfqpoint{2.214790in}{1.380913in}}%
\pgfpathcurveto{\pgfqpoint{2.220614in}{1.375089in}}{\pgfqpoint{2.228514in}{1.371817in}}{\pgfqpoint{2.236750in}{1.371817in}}%
\pgfpathclose%
\pgfusepath{stroke,fill}%
\end{pgfscope}%
\begin{pgfscope}%
\pgfpathrectangle{\pgfqpoint{0.894063in}{0.630000in}}{\pgfqpoint{6.713438in}{2.060556in}} %
\pgfusepath{clip}%
\pgfsetbuttcap%
\pgfsetroundjoin%
\definecolor{currentfill}{rgb}{0.000000,0.000000,1.000000}%
\pgfsetfillcolor{currentfill}%
\pgfsetlinewidth{1.003750pt}%
\definecolor{currentstroke}{rgb}{0.000000,0.000000,0.000000}%
\pgfsetstrokecolor{currentstroke}%
\pgfsetdash{}{0pt}%
\pgfpathmoveto{\pgfqpoint{6.667619in}{2.001505in}}%
\pgfpathcurveto{\pgfqpoint{6.675855in}{2.001505in}}{\pgfqpoint{6.683755in}{2.004777in}}{\pgfqpoint{6.689579in}{2.010601in}}%
\pgfpathcurveto{\pgfqpoint{6.695403in}{2.016425in}}{\pgfqpoint{6.698675in}{2.024325in}}{\pgfqpoint{6.698675in}{2.032561in}}%
\pgfpathcurveto{\pgfqpoint{6.698675in}{2.040798in}}{\pgfqpoint{6.695403in}{2.048698in}}{\pgfqpoint{6.689579in}{2.054522in}}%
\pgfpathcurveto{\pgfqpoint{6.683755in}{2.060345in}}{\pgfqpoint{6.675855in}{2.063618in}}{\pgfqpoint{6.667619in}{2.063618in}}%
\pgfpathcurveto{\pgfqpoint{6.659382in}{2.063618in}}{\pgfqpoint{6.651482in}{2.060345in}}{\pgfqpoint{6.645658in}{2.054522in}}%
\pgfpathcurveto{\pgfqpoint{6.639835in}{2.048698in}}{\pgfqpoint{6.636562in}{2.040798in}}{\pgfqpoint{6.636562in}{2.032561in}}%
\pgfpathcurveto{\pgfqpoint{6.636562in}{2.024325in}}{\pgfqpoint{6.639835in}{2.016425in}}{\pgfqpoint{6.645658in}{2.010601in}}%
\pgfpathcurveto{\pgfqpoint{6.651482in}{2.004777in}}{\pgfqpoint{6.659382in}{2.001505in}}{\pgfqpoint{6.667619in}{2.001505in}}%
\pgfpathclose%
\pgfusepath{stroke,fill}%
\end{pgfscope}%
\begin{pgfscope}%
\pgfpathrectangle{\pgfqpoint{0.894063in}{0.630000in}}{\pgfqpoint{6.713438in}{2.060556in}} %
\pgfusepath{clip}%
\pgfsetbuttcap%
\pgfsetroundjoin%
\definecolor{currentfill}{rgb}{0.000000,0.000000,1.000000}%
\pgfsetfillcolor{currentfill}%
\pgfsetlinewidth{1.003750pt}%
\definecolor{currentstroke}{rgb}{0.000000,0.000000,0.000000}%
\pgfsetstrokecolor{currentstroke}%
\pgfsetdash{}{0pt}%
\pgfpathmoveto{\pgfqpoint{2.639556in}{1.384257in}}%
\pgfpathcurveto{\pgfqpoint{2.647793in}{1.384257in}}{\pgfqpoint{2.655693in}{1.387529in}}{\pgfqpoint{2.661517in}{1.393353in}}%
\pgfpathcurveto{\pgfqpoint{2.667340in}{1.399177in}}{\pgfqpoint{2.670613in}{1.407077in}}{\pgfqpoint{2.670613in}{1.415313in}}%
\pgfpathcurveto{\pgfqpoint{2.670613in}{1.423549in}}{\pgfqpoint{2.667340in}{1.431449in}}{\pgfqpoint{2.661517in}{1.437273in}}%
\pgfpathcurveto{\pgfqpoint{2.655693in}{1.443097in}}{\pgfqpoint{2.647793in}{1.446370in}}{\pgfqpoint{2.639556in}{1.446370in}}%
\pgfpathcurveto{\pgfqpoint{2.631320in}{1.446370in}}{\pgfqpoint{2.623420in}{1.443097in}}{\pgfqpoint{2.617596in}{1.437273in}}%
\pgfpathcurveto{\pgfqpoint{2.611772in}{1.431449in}}{\pgfqpoint{2.608500in}{1.423549in}}{\pgfqpoint{2.608500in}{1.415313in}}%
\pgfpathcurveto{\pgfqpoint{2.608500in}{1.407077in}}{\pgfqpoint{2.611772in}{1.399177in}}{\pgfqpoint{2.617596in}{1.393353in}}%
\pgfpathcurveto{\pgfqpoint{2.623420in}{1.387529in}}{\pgfqpoint{2.631320in}{1.384257in}}{\pgfqpoint{2.639556in}{1.384257in}}%
\pgfpathclose%
\pgfusepath{stroke,fill}%
\end{pgfscope}%
\begin{pgfscope}%
\pgfpathrectangle{\pgfqpoint{0.894063in}{0.630000in}}{\pgfqpoint{6.713438in}{2.060556in}} %
\pgfusepath{clip}%
\pgfsetbuttcap%
\pgfsetroundjoin%
\definecolor{currentfill}{rgb}{0.000000,0.000000,1.000000}%
\pgfsetfillcolor{currentfill}%
\pgfsetlinewidth{1.003750pt}%
\definecolor{currentstroke}{rgb}{0.000000,0.000000,0.000000}%
\pgfsetstrokecolor{currentstroke}%
\pgfsetdash{}{0pt}%
\pgfpathmoveto{\pgfqpoint{1.699675in}{1.262849in}}%
\pgfpathcurveto{\pgfqpoint{1.707911in}{1.262849in}}{\pgfqpoint{1.715811in}{1.266121in}}{\pgfqpoint{1.721635in}{1.271945in}}%
\pgfpathcurveto{\pgfqpoint{1.727459in}{1.277769in}}{\pgfqpoint{1.730731in}{1.285669in}}{\pgfqpoint{1.730731in}{1.293905in}}%
\pgfpathcurveto{\pgfqpoint{1.730731in}{1.302141in}}{\pgfqpoint{1.727459in}{1.310041in}}{\pgfqpoint{1.721635in}{1.315865in}}%
\pgfpathcurveto{\pgfqpoint{1.715811in}{1.321689in}}{\pgfqpoint{1.707911in}{1.324962in}}{\pgfqpoint{1.699675in}{1.324962in}}%
\pgfpathcurveto{\pgfqpoint{1.691439in}{1.324962in}}{\pgfqpoint{1.683539in}{1.321689in}}{\pgfqpoint{1.677715in}{1.315865in}}%
\pgfpathcurveto{\pgfqpoint{1.671891in}{1.310041in}}{\pgfqpoint{1.668619in}{1.302141in}}{\pgfqpoint{1.668619in}{1.293905in}}%
\pgfpathcurveto{\pgfqpoint{1.668619in}{1.285669in}}{\pgfqpoint{1.671891in}{1.277769in}}{\pgfqpoint{1.677715in}{1.271945in}}%
\pgfpathcurveto{\pgfqpoint{1.683539in}{1.266121in}}{\pgfqpoint{1.691439in}{1.262849in}}{\pgfqpoint{1.699675in}{1.262849in}}%
\pgfpathclose%
\pgfusepath{stroke,fill}%
\end{pgfscope}%
\begin{pgfscope}%
\pgfpathrectangle{\pgfqpoint{0.894063in}{0.630000in}}{\pgfqpoint{6.713438in}{2.060556in}} %
\pgfusepath{clip}%
\pgfsetbuttcap%
\pgfsetroundjoin%
\definecolor{currentfill}{rgb}{0.000000,0.000000,1.000000}%
\pgfsetfillcolor{currentfill}%
\pgfsetlinewidth{1.003750pt}%
\definecolor{currentstroke}{rgb}{0.000000,0.000000,0.000000}%
\pgfsetstrokecolor{currentstroke}%
\pgfsetdash{}{0pt}%
\pgfpathmoveto{\pgfqpoint{1.162600in}{0.702177in}}%
\pgfpathcurveto{\pgfqpoint{1.170836in}{0.702177in}}{\pgfqpoint{1.178736in}{0.705450in}}{\pgfqpoint{1.184560in}{0.711274in}}%
\pgfpathcurveto{\pgfqpoint{1.190384in}{0.717098in}}{\pgfqpoint{1.193656in}{0.724998in}}{\pgfqpoint{1.193656in}{0.733234in}}%
\pgfpathcurveto{\pgfqpoint{1.193656in}{0.741470in}}{\pgfqpoint{1.190384in}{0.749370in}}{\pgfqpoint{1.184560in}{0.755194in}}%
\pgfpathcurveto{\pgfqpoint{1.178736in}{0.761018in}}{\pgfqpoint{1.170836in}{0.764290in}}{\pgfqpoint{1.162600in}{0.764290in}}%
\pgfpathcurveto{\pgfqpoint{1.154364in}{0.764290in}}{\pgfqpoint{1.146464in}{0.761018in}}{\pgfqpoint{1.140640in}{0.755194in}}%
\pgfpathcurveto{\pgfqpoint{1.134816in}{0.749370in}}{\pgfqpoint{1.131544in}{0.741470in}}{\pgfqpoint{1.131544in}{0.733234in}}%
\pgfpathcurveto{\pgfqpoint{1.131544in}{0.724998in}}{\pgfqpoint{1.134816in}{0.717098in}}{\pgfqpoint{1.140640in}{0.711274in}}%
\pgfpathcurveto{\pgfqpoint{1.146464in}{0.705450in}}{\pgfqpoint{1.154364in}{0.702177in}}{\pgfqpoint{1.162600in}{0.702177in}}%
\pgfpathclose%
\pgfusepath{stroke,fill}%
\end{pgfscope}%
\begin{pgfscope}%
\pgfpathrectangle{\pgfqpoint{0.894063in}{0.630000in}}{\pgfqpoint{6.713438in}{2.060556in}} %
\pgfusepath{clip}%
\pgfsetbuttcap%
\pgfsetroundjoin%
\definecolor{currentfill}{rgb}{0.000000,0.000000,1.000000}%
\pgfsetfillcolor{currentfill}%
\pgfsetlinewidth{1.003750pt}%
\definecolor{currentstroke}{rgb}{0.000000,0.000000,0.000000}%
\pgfsetstrokecolor{currentstroke}%
\pgfsetdash{}{0pt}%
\pgfpathmoveto{\pgfqpoint{1.833944in}{1.277302in}}%
\pgfpathcurveto{\pgfqpoint{1.842180in}{1.277302in}}{\pgfqpoint{1.850080in}{1.280574in}}{\pgfqpoint{1.855904in}{1.286398in}}%
\pgfpathcurveto{\pgfqpoint{1.861728in}{1.292222in}}{\pgfqpoint{1.865000in}{1.300122in}}{\pgfqpoint{1.865000in}{1.308358in}}%
\pgfpathcurveto{\pgfqpoint{1.865000in}{1.316595in}}{\pgfqpoint{1.861728in}{1.324495in}}{\pgfqpoint{1.855904in}{1.330319in}}%
\pgfpathcurveto{\pgfqpoint{1.850080in}{1.336143in}}{\pgfqpoint{1.842180in}{1.339415in}}{\pgfqpoint{1.833944in}{1.339415in}}%
\pgfpathcurveto{\pgfqpoint{1.825707in}{1.339415in}}{\pgfqpoint{1.817807in}{1.336143in}}{\pgfqpoint{1.811983in}{1.330319in}}%
\pgfpathcurveto{\pgfqpoint{1.806160in}{1.324495in}}{\pgfqpoint{1.802887in}{1.316595in}}{\pgfqpoint{1.802887in}{1.308358in}}%
\pgfpathcurveto{\pgfqpoint{1.802887in}{1.300122in}}{\pgfqpoint{1.806160in}{1.292222in}}{\pgfqpoint{1.811983in}{1.286398in}}%
\pgfpathcurveto{\pgfqpoint{1.817807in}{1.280574in}}{\pgfqpoint{1.825707in}{1.277302in}}{\pgfqpoint{1.833944in}{1.277302in}}%
\pgfpathclose%
\pgfusepath{stroke,fill}%
\end{pgfscope}%
\begin{pgfscope}%
\pgfpathrectangle{\pgfqpoint{0.894063in}{0.630000in}}{\pgfqpoint{6.713438in}{2.060556in}} %
\pgfusepath{clip}%
\pgfsetbuttcap%
\pgfsetroundjoin%
\definecolor{currentfill}{rgb}{0.000000,0.000000,1.000000}%
\pgfsetfillcolor{currentfill}%
\pgfsetlinewidth{1.003750pt}%
\definecolor{currentstroke}{rgb}{0.000000,0.000000,0.000000}%
\pgfsetstrokecolor{currentstroke}%
\pgfsetdash{}{0pt}%
\pgfpathmoveto{\pgfqpoint{5.996275in}{1.907897in}}%
\pgfpathcurveto{\pgfqpoint{6.004511in}{1.907897in}}{\pgfqpoint{6.012411in}{1.911169in}}{\pgfqpoint{6.018235in}{1.916993in}}%
\pgfpathcurveto{\pgfqpoint{6.024059in}{1.922817in}}{\pgfqpoint{6.027331in}{1.930717in}}{\pgfqpoint{6.027331in}{1.938953in}}%
\pgfpathcurveto{\pgfqpoint{6.027331in}{1.947189in}}{\pgfqpoint{6.024059in}{1.955090in}}{\pgfqpoint{6.018235in}{1.960913in}}%
\pgfpathcurveto{\pgfqpoint{6.012411in}{1.966737in}}{\pgfqpoint{6.004511in}{1.970010in}}{\pgfqpoint{5.996275in}{1.970010in}}%
\pgfpathcurveto{\pgfqpoint{5.988039in}{1.970010in}}{\pgfqpoint{5.980139in}{1.966737in}}{\pgfqpoint{5.974315in}{1.960913in}}%
\pgfpathcurveto{\pgfqpoint{5.968491in}{1.955090in}}{\pgfqpoint{5.965219in}{1.947189in}}{\pgfqpoint{5.965219in}{1.938953in}}%
\pgfpathcurveto{\pgfqpoint{5.965219in}{1.930717in}}{\pgfqpoint{5.968491in}{1.922817in}}{\pgfqpoint{5.974315in}{1.916993in}}%
\pgfpathcurveto{\pgfqpoint{5.980139in}{1.911169in}}{\pgfqpoint{5.988039in}{1.907897in}}{\pgfqpoint{5.996275in}{1.907897in}}%
\pgfpathclose%
\pgfusepath{stroke,fill}%
\end{pgfscope}%
\begin{pgfscope}%
\pgfpathrectangle{\pgfqpoint{0.894063in}{0.630000in}}{\pgfqpoint{6.713438in}{2.060556in}} %
\pgfusepath{clip}%
\pgfsetbuttcap%
\pgfsetroundjoin%
\definecolor{currentfill}{rgb}{0.000000,0.000000,1.000000}%
\pgfsetfillcolor{currentfill}%
\pgfsetlinewidth{1.003750pt}%
\definecolor{currentstroke}{rgb}{0.000000,0.000000,0.000000}%
\pgfsetstrokecolor{currentstroke}%
\pgfsetdash{}{0pt}%
\pgfpathmoveto{\pgfqpoint{6.399081in}{1.969460in}}%
\pgfpathcurveto{\pgfqpoint{6.407318in}{1.969460in}}{\pgfqpoint{6.415218in}{1.972733in}}{\pgfqpoint{6.421042in}{1.978556in}}%
\pgfpathcurveto{\pgfqpoint{6.426865in}{1.984380in}}{\pgfqpoint{6.430138in}{1.992280in}}{\pgfqpoint{6.430138in}{2.000517in}}%
\pgfpathcurveto{\pgfqpoint{6.430138in}{2.008753in}}{\pgfqpoint{6.426865in}{2.016653in}}{\pgfqpoint{6.421042in}{2.022477in}}%
\pgfpathcurveto{\pgfqpoint{6.415218in}{2.028301in}}{\pgfqpoint{6.407318in}{2.031573in}}{\pgfqpoint{6.399081in}{2.031573in}}%
\pgfpathcurveto{\pgfqpoint{6.390845in}{2.031573in}}{\pgfqpoint{6.382945in}{2.028301in}}{\pgfqpoint{6.377121in}{2.022477in}}%
\pgfpathcurveto{\pgfqpoint{6.371297in}{2.016653in}}{\pgfqpoint{6.368025in}{2.008753in}}{\pgfqpoint{6.368025in}{2.000517in}}%
\pgfpathcurveto{\pgfqpoint{6.368025in}{1.992280in}}{\pgfqpoint{6.371297in}{1.984380in}}{\pgfqpoint{6.377121in}{1.978556in}}%
\pgfpathcurveto{\pgfqpoint{6.382945in}{1.972733in}}{\pgfqpoint{6.390845in}{1.969460in}}{\pgfqpoint{6.399081in}{1.969460in}}%
\pgfpathclose%
\pgfusepath{stroke,fill}%
\end{pgfscope}%
\begin{pgfscope}%
\pgfpathrectangle{\pgfqpoint{0.894063in}{0.630000in}}{\pgfqpoint{6.713438in}{2.060556in}} %
\pgfusepath{clip}%
\pgfsetbuttcap%
\pgfsetroundjoin%
\definecolor{currentfill}{rgb}{0.000000,0.000000,1.000000}%
\pgfsetfillcolor{currentfill}%
\pgfsetlinewidth{1.003750pt}%
\definecolor{currentstroke}{rgb}{0.000000,0.000000,0.000000}%
\pgfsetstrokecolor{currentstroke}%
\pgfsetdash{}{0pt}%
\pgfpathmoveto{\pgfqpoint{4.787856in}{1.710077in}}%
\pgfpathcurveto{\pgfqpoint{4.796093in}{1.710077in}}{\pgfqpoint{4.803993in}{1.713350in}}{\pgfqpoint{4.809817in}{1.719174in}}%
\pgfpathcurveto{\pgfqpoint{4.815640in}{1.724998in}}{\pgfqpoint{4.818913in}{1.732898in}}{\pgfqpoint{4.818913in}{1.741134in}}%
\pgfpathcurveto{\pgfqpoint{4.818913in}{1.749370in}}{\pgfqpoint{4.815640in}{1.757270in}}{\pgfqpoint{4.809817in}{1.763094in}}%
\pgfpathcurveto{\pgfqpoint{4.803993in}{1.768918in}}{\pgfqpoint{4.796093in}{1.772190in}}{\pgfqpoint{4.787856in}{1.772190in}}%
\pgfpathcurveto{\pgfqpoint{4.779620in}{1.772190in}}{\pgfqpoint{4.771720in}{1.768918in}}{\pgfqpoint{4.765896in}{1.763094in}}%
\pgfpathcurveto{\pgfqpoint{4.760072in}{1.757270in}}{\pgfqpoint{4.756800in}{1.749370in}}{\pgfqpoint{4.756800in}{1.741134in}}%
\pgfpathcurveto{\pgfqpoint{4.756800in}{1.732898in}}{\pgfqpoint{4.760072in}{1.724998in}}{\pgfqpoint{4.765896in}{1.719174in}}%
\pgfpathcurveto{\pgfqpoint{4.771720in}{1.713350in}}{\pgfqpoint{4.779620in}{1.710077in}}{\pgfqpoint{4.787856in}{1.710077in}}%
\pgfpathclose%
\pgfusepath{stroke,fill}%
\end{pgfscope}%
\begin{pgfscope}%
\pgfpathrectangle{\pgfqpoint{0.894063in}{0.630000in}}{\pgfqpoint{6.713438in}{2.060556in}} %
\pgfusepath{clip}%
\pgfsetbuttcap%
\pgfsetroundjoin%
\definecolor{currentfill}{rgb}{0.000000,0.000000,1.000000}%
\pgfsetfillcolor{currentfill}%
\pgfsetlinewidth{1.003750pt}%
\definecolor{currentstroke}{rgb}{0.000000,0.000000,0.000000}%
\pgfsetstrokecolor{currentstroke}%
\pgfsetdash{}{0pt}%
\pgfpathmoveto{\pgfqpoint{4.922125in}{1.728499in}}%
\pgfpathcurveto{\pgfqpoint{4.930361in}{1.728499in}}{\pgfqpoint{4.938261in}{1.731771in}}{\pgfqpoint{4.944085in}{1.737595in}}%
\pgfpathcurveto{\pgfqpoint{4.949909in}{1.743419in}}{\pgfqpoint{4.953181in}{1.751319in}}{\pgfqpoint{4.953181in}{1.759555in}}%
\pgfpathcurveto{\pgfqpoint{4.953181in}{1.767792in}}{\pgfqpoint{4.949909in}{1.775692in}}{\pgfqpoint{4.944085in}{1.781516in}}%
\pgfpathcurveto{\pgfqpoint{4.938261in}{1.787340in}}{\pgfqpoint{4.930361in}{1.790612in}}{\pgfqpoint{4.922125in}{1.790612in}}%
\pgfpathcurveto{\pgfqpoint{4.913889in}{1.790612in}}{\pgfqpoint{4.905989in}{1.787340in}}{\pgfqpoint{4.900165in}{1.781516in}}%
\pgfpathcurveto{\pgfqpoint{4.894341in}{1.775692in}}{\pgfqpoint{4.891069in}{1.767792in}}{\pgfqpoint{4.891069in}{1.759555in}}%
\pgfpathcurveto{\pgfqpoint{4.891069in}{1.751319in}}{\pgfqpoint{4.894341in}{1.743419in}}{\pgfqpoint{4.900165in}{1.737595in}}%
\pgfpathcurveto{\pgfqpoint{4.905989in}{1.731771in}}{\pgfqpoint{4.913889in}{1.728499in}}{\pgfqpoint{4.922125in}{1.728499in}}%
\pgfpathclose%
\pgfusepath{stroke,fill}%
\end{pgfscope}%
\begin{pgfscope}%
\pgfpathrectangle{\pgfqpoint{0.894063in}{0.630000in}}{\pgfqpoint{6.713438in}{2.060556in}} %
\pgfusepath{clip}%
\pgfsetbuttcap%
\pgfsetroundjoin%
\definecolor{currentfill}{rgb}{0.000000,0.000000,1.000000}%
\pgfsetfillcolor{currentfill}%
\pgfsetlinewidth{1.003750pt}%
\definecolor{currentstroke}{rgb}{0.000000,0.000000,0.000000}%
\pgfsetstrokecolor{currentstroke}%
\pgfsetdash{}{0pt}%
\pgfpathmoveto{\pgfqpoint{6.130544in}{1.928496in}}%
\pgfpathcurveto{\pgfqpoint{6.138780in}{1.928496in}}{\pgfqpoint{6.146680in}{1.931769in}}{\pgfqpoint{6.152504in}{1.937593in}}%
\pgfpathcurveto{\pgfqpoint{6.158328in}{1.943417in}}{\pgfqpoint{6.161600in}{1.951317in}}{\pgfqpoint{6.161600in}{1.959553in}}%
\pgfpathcurveto{\pgfqpoint{6.161600in}{1.967789in}}{\pgfqpoint{6.158328in}{1.975689in}}{\pgfqpoint{6.152504in}{1.981513in}}%
\pgfpathcurveto{\pgfqpoint{6.146680in}{1.987337in}}{\pgfqpoint{6.138780in}{1.990609in}}{\pgfqpoint{6.130544in}{1.990609in}}%
\pgfpathcurveto{\pgfqpoint{6.122307in}{1.990609in}}{\pgfqpoint{6.114407in}{1.987337in}}{\pgfqpoint{6.108583in}{1.981513in}}%
\pgfpathcurveto{\pgfqpoint{6.102760in}{1.975689in}}{\pgfqpoint{6.099487in}{1.967789in}}{\pgfqpoint{6.099487in}{1.959553in}}%
\pgfpathcurveto{\pgfqpoint{6.099487in}{1.951317in}}{\pgfqpoint{6.102760in}{1.943417in}}{\pgfqpoint{6.108583in}{1.937593in}}%
\pgfpathcurveto{\pgfqpoint{6.114407in}{1.931769in}}{\pgfqpoint{6.122307in}{1.928496in}}{\pgfqpoint{6.130544in}{1.928496in}}%
\pgfpathclose%
\pgfusepath{stroke,fill}%
\end{pgfscope}%
\begin{pgfscope}%
\pgfpathrectangle{\pgfqpoint{0.894063in}{0.630000in}}{\pgfqpoint{6.713438in}{2.060556in}} %
\pgfusepath{clip}%
\pgfsetbuttcap%
\pgfsetroundjoin%
\definecolor{currentfill}{rgb}{0.000000,0.000000,1.000000}%
\pgfsetfillcolor{currentfill}%
\pgfsetlinewidth{1.003750pt}%
\definecolor{currentstroke}{rgb}{0.000000,0.000000,0.000000}%
\pgfsetstrokecolor{currentstroke}%
\pgfsetdash{}{0pt}%
\pgfpathmoveto{\pgfqpoint{5.727738in}{1.861499in}}%
\pgfpathcurveto{\pgfqpoint{5.735974in}{1.861499in}}{\pgfqpoint{5.743874in}{1.864771in}}{\pgfqpoint{5.749698in}{1.870595in}}%
\pgfpathcurveto{\pgfqpoint{5.755522in}{1.876419in}}{\pgfqpoint{5.758794in}{1.884319in}}{\pgfqpoint{5.758794in}{1.892555in}}%
\pgfpathcurveto{\pgfqpoint{5.758794in}{1.900792in}}{\pgfqpoint{5.755522in}{1.908692in}}{\pgfqpoint{5.749698in}{1.914516in}}%
\pgfpathcurveto{\pgfqpoint{5.743874in}{1.920340in}}{\pgfqpoint{5.735974in}{1.923612in}}{\pgfqpoint{5.727738in}{1.923612in}}%
\pgfpathcurveto{\pgfqpoint{5.719501in}{1.923612in}}{\pgfqpoint{5.711601in}{1.920340in}}{\pgfqpoint{5.705777in}{1.914516in}}%
\pgfpathcurveto{\pgfqpoint{5.699953in}{1.908692in}}{\pgfqpoint{5.696681in}{1.900792in}}{\pgfqpoint{5.696681in}{1.892555in}}%
\pgfpathcurveto{\pgfqpoint{5.696681in}{1.884319in}}{\pgfqpoint{5.699953in}{1.876419in}}{\pgfqpoint{5.705777in}{1.870595in}}%
\pgfpathcurveto{\pgfqpoint{5.711601in}{1.864771in}}{\pgfqpoint{5.719501in}{1.861499in}}{\pgfqpoint{5.727738in}{1.861499in}}%
\pgfpathclose%
\pgfusepath{stroke,fill}%
\end{pgfscope}%
\begin{pgfscope}%
\pgfpathrectangle{\pgfqpoint{0.894063in}{0.630000in}}{\pgfqpoint{6.713438in}{2.060556in}} %
\pgfusepath{clip}%
\pgfsetbuttcap%
\pgfsetroundjoin%
\definecolor{currentfill}{rgb}{0.000000,0.000000,1.000000}%
\pgfsetfillcolor{currentfill}%
\pgfsetlinewidth{1.003750pt}%
\definecolor{currentstroke}{rgb}{0.000000,0.000000,0.000000}%
\pgfsetstrokecolor{currentstroke}%
\pgfsetdash{}{0pt}%
\pgfpathmoveto{\pgfqpoint{1.028331in}{0.680506in}}%
\pgfpathcurveto{\pgfqpoint{1.036568in}{0.680506in}}{\pgfqpoint{1.044468in}{0.683778in}}{\pgfqpoint{1.050292in}{0.689602in}}%
\pgfpathcurveto{\pgfqpoint{1.056115in}{0.695426in}}{\pgfqpoint{1.059388in}{0.703326in}}{\pgfqpoint{1.059388in}{0.711563in}}%
\pgfpathcurveto{\pgfqpoint{1.059388in}{0.719799in}}{\pgfqpoint{1.056115in}{0.727699in}}{\pgfqpoint{1.050292in}{0.733523in}}%
\pgfpathcurveto{\pgfqpoint{1.044468in}{0.739347in}}{\pgfqpoint{1.036568in}{0.742619in}}{\pgfqpoint{1.028331in}{0.742619in}}%
\pgfpathcurveto{\pgfqpoint{1.020095in}{0.742619in}}{\pgfqpoint{1.012195in}{0.739347in}}{\pgfqpoint{1.006371in}{0.733523in}}%
\pgfpathcurveto{\pgfqpoint{1.000547in}{0.727699in}}{\pgfqpoint{0.997275in}{0.719799in}}{\pgfqpoint{0.997275in}{0.711563in}}%
\pgfpathcurveto{\pgfqpoint{0.997275in}{0.703326in}}{\pgfqpoint{1.000547in}{0.695426in}}{\pgfqpoint{1.006371in}{0.689602in}}%
\pgfpathcurveto{\pgfqpoint{1.012195in}{0.683778in}}{\pgfqpoint{1.020095in}{0.680506in}}{\pgfqpoint{1.028331in}{0.680506in}}%
\pgfpathclose%
\pgfusepath{stroke,fill}%
\end{pgfscope}%
\begin{pgfscope}%
\pgfpathrectangle{\pgfqpoint{0.894063in}{0.630000in}}{\pgfqpoint{6.713438in}{2.060556in}} %
\pgfusepath{clip}%
\pgfsetbuttcap%
\pgfsetroundjoin%
\definecolor{currentfill}{rgb}{0.000000,0.000000,1.000000}%
\pgfsetfillcolor{currentfill}%
\pgfsetlinewidth{1.003750pt}%
\definecolor{currentstroke}{rgb}{0.000000,0.000000,0.000000}%
\pgfsetstrokecolor{currentstroke}%
\pgfsetdash{}{0pt}%
\pgfpathmoveto{\pgfqpoint{5.324931in}{1.806971in}}%
\pgfpathcurveto{\pgfqpoint{5.333168in}{1.806971in}}{\pgfqpoint{5.341068in}{1.810243in}}{\pgfqpoint{5.346892in}{1.816067in}}%
\pgfpathcurveto{\pgfqpoint{5.352715in}{1.821891in}}{\pgfqpoint{5.355988in}{1.829791in}}{\pgfqpoint{5.355988in}{1.838027in}}%
\pgfpathcurveto{\pgfqpoint{5.355988in}{1.846263in}}{\pgfqpoint{5.352715in}{1.854164in}}{\pgfqpoint{5.346892in}{1.859987in}}%
\pgfpathcurveto{\pgfqpoint{5.341068in}{1.865811in}}{\pgfqpoint{5.333168in}{1.869084in}}{\pgfqpoint{5.324931in}{1.869084in}}%
\pgfpathcurveto{\pgfqpoint{5.316695in}{1.869084in}}{\pgfqpoint{5.308795in}{1.865811in}}{\pgfqpoint{5.302971in}{1.859987in}}%
\pgfpathcurveto{\pgfqpoint{5.297147in}{1.854164in}}{\pgfqpoint{5.293875in}{1.846263in}}{\pgfqpoint{5.293875in}{1.838027in}}%
\pgfpathcurveto{\pgfqpoint{5.293875in}{1.829791in}}{\pgfqpoint{5.297147in}{1.821891in}}{\pgfqpoint{5.302971in}{1.816067in}}%
\pgfpathcurveto{\pgfqpoint{5.308795in}{1.810243in}}{\pgfqpoint{5.316695in}{1.806971in}}{\pgfqpoint{5.324931in}{1.806971in}}%
\pgfpathclose%
\pgfusepath{stroke,fill}%
\end{pgfscope}%
\begin{pgfscope}%
\pgfpathrectangle{\pgfqpoint{0.894063in}{0.630000in}}{\pgfqpoint{6.713438in}{2.060556in}} %
\pgfusepath{clip}%
\pgfsetbuttcap%
\pgfsetroundjoin%
\definecolor{currentfill}{rgb}{0.000000,0.000000,1.000000}%
\pgfsetfillcolor{currentfill}%
\pgfsetlinewidth{1.003750pt}%
\definecolor{currentstroke}{rgb}{0.000000,0.000000,0.000000}%
\pgfsetstrokecolor{currentstroke}%
\pgfsetdash{}{0pt}%
\pgfpathmoveto{\pgfqpoint{7.338963in}{2.099234in}}%
\pgfpathcurveto{\pgfqpoint{7.347199in}{2.099234in}}{\pgfqpoint{7.355099in}{2.102506in}}{\pgfqpoint{7.360923in}{2.108330in}}%
\pgfpathcurveto{\pgfqpoint{7.366747in}{2.114154in}}{\pgfqpoint{7.370019in}{2.122054in}}{\pgfqpoint{7.370019in}{2.130290in}}%
\pgfpathcurveto{\pgfqpoint{7.370019in}{2.138527in}}{\pgfqpoint{7.366747in}{2.146427in}}{\pgfqpoint{7.360923in}{2.152251in}}%
\pgfpathcurveto{\pgfqpoint{7.355099in}{2.158075in}}{\pgfqpoint{7.347199in}{2.161347in}}{\pgfqpoint{7.338963in}{2.161347in}}%
\pgfpathcurveto{\pgfqpoint{7.330726in}{2.161347in}}{\pgfqpoint{7.322826in}{2.158075in}}{\pgfqpoint{7.317002in}{2.152251in}}%
\pgfpathcurveto{\pgfqpoint{7.311178in}{2.146427in}}{\pgfqpoint{7.307906in}{2.138527in}}{\pgfqpoint{7.307906in}{2.130290in}}%
\pgfpathcurveto{\pgfqpoint{7.307906in}{2.122054in}}{\pgfqpoint{7.311178in}{2.114154in}}{\pgfqpoint{7.317002in}{2.108330in}}%
\pgfpathcurveto{\pgfqpoint{7.322826in}{2.102506in}}{\pgfqpoint{7.330726in}{2.099234in}}{\pgfqpoint{7.338963in}{2.099234in}}%
\pgfpathclose%
\pgfusepath{stroke,fill}%
\end{pgfscope}%
\begin{pgfscope}%
\pgfpathrectangle{\pgfqpoint{0.894063in}{0.630000in}}{\pgfqpoint{6.713438in}{2.060556in}} %
\pgfusepath{clip}%
\pgfsetbuttcap%
\pgfsetroundjoin%
\definecolor{currentfill}{rgb}{0.000000,0.000000,1.000000}%
\pgfsetfillcolor{currentfill}%
\pgfsetlinewidth{1.003750pt}%
\definecolor{currentstroke}{rgb}{0.000000,0.000000,0.000000}%
\pgfsetstrokecolor{currentstroke}%
\pgfsetdash{}{0pt}%
\pgfpathmoveto{\pgfqpoint{7.204694in}{2.078263in}}%
\pgfpathcurveto{\pgfqpoint{7.212930in}{2.078263in}}{\pgfqpoint{7.220830in}{2.081536in}}{\pgfqpoint{7.226654in}{2.087360in}}%
\pgfpathcurveto{\pgfqpoint{7.232478in}{2.093184in}}{\pgfqpoint{7.235750in}{2.101084in}}{\pgfqpoint{7.235750in}{2.109320in}}%
\pgfpathcurveto{\pgfqpoint{7.235750in}{2.117556in}}{\pgfqpoint{7.232478in}{2.125456in}}{\pgfqpoint{7.226654in}{2.131280in}}%
\pgfpathcurveto{\pgfqpoint{7.220830in}{2.137104in}}{\pgfqpoint{7.212930in}{2.140376in}}{\pgfqpoint{7.204694in}{2.140376in}}%
\pgfpathcurveto{\pgfqpoint{7.196457in}{2.140376in}}{\pgfqpoint{7.188557in}{2.137104in}}{\pgfqpoint{7.182733in}{2.131280in}}%
\pgfpathcurveto{\pgfqpoint{7.176910in}{2.125456in}}{\pgfqpoint{7.173637in}{2.117556in}}{\pgfqpoint{7.173637in}{2.109320in}}%
\pgfpathcurveto{\pgfqpoint{7.173637in}{2.101084in}}{\pgfqpoint{7.176910in}{2.093184in}}{\pgfqpoint{7.182733in}{2.087360in}}%
\pgfpathcurveto{\pgfqpoint{7.188557in}{2.081536in}}{\pgfqpoint{7.196457in}{2.078263in}}{\pgfqpoint{7.204694in}{2.078263in}}%
\pgfpathclose%
\pgfusepath{stroke,fill}%
\end{pgfscope}%
\begin{pgfscope}%
\pgfpathrectangle{\pgfqpoint{0.894063in}{0.630000in}}{\pgfqpoint{6.713438in}{2.060556in}} %
\pgfusepath{clip}%
\pgfsetbuttcap%
\pgfsetroundjoin%
\definecolor{currentfill}{rgb}{0.000000,0.000000,1.000000}%
\pgfsetfillcolor{currentfill}%
\pgfsetlinewidth{1.003750pt}%
\definecolor{currentstroke}{rgb}{0.000000,0.000000,0.000000}%
\pgfsetstrokecolor{currentstroke}%
\pgfsetdash{}{0pt}%
\pgfpathmoveto{\pgfqpoint{6.264813in}{1.949332in}}%
\pgfpathcurveto{\pgfqpoint{6.273049in}{1.949332in}}{\pgfqpoint{6.280949in}{1.952604in}}{\pgfqpoint{6.286773in}{1.958428in}}%
\pgfpathcurveto{\pgfqpoint{6.292597in}{1.964252in}}{\pgfqpoint{6.295869in}{1.972152in}}{\pgfqpoint{6.295869in}{1.980388in}}%
\pgfpathcurveto{\pgfqpoint{6.295869in}{1.988624in}}{\pgfqpoint{6.292597in}{1.996524in}}{\pgfqpoint{6.286773in}{2.002348in}}%
\pgfpathcurveto{\pgfqpoint{6.280949in}{2.008172in}}{\pgfqpoint{6.273049in}{2.011445in}}{\pgfqpoint{6.264813in}{2.011445in}}%
\pgfpathcurveto{\pgfqpoint{6.256576in}{2.011445in}}{\pgfqpoint{6.248676in}{2.008172in}}{\pgfqpoint{6.242852in}{2.002348in}}%
\pgfpathcurveto{\pgfqpoint{6.237028in}{1.996524in}}{\pgfqpoint{6.233756in}{1.988624in}}{\pgfqpoint{6.233756in}{1.980388in}}%
\pgfpathcurveto{\pgfqpoint{6.233756in}{1.972152in}}{\pgfqpoint{6.237028in}{1.964252in}}{\pgfqpoint{6.242852in}{1.958428in}}%
\pgfpathcurveto{\pgfqpoint{6.248676in}{1.952604in}}{\pgfqpoint{6.256576in}{1.949332in}}{\pgfqpoint{6.264813in}{1.949332in}}%
\pgfpathclose%
\pgfusepath{stroke,fill}%
\end{pgfscope}%
\begin{pgfscope}%
\pgfpathrectangle{\pgfqpoint{0.894063in}{0.630000in}}{\pgfqpoint{6.713438in}{2.060556in}} %
\pgfusepath{clip}%
\pgfsetbuttcap%
\pgfsetroundjoin%
\definecolor{currentfill}{rgb}{0.000000,0.000000,1.000000}%
\pgfsetfillcolor{currentfill}%
\pgfsetlinewidth{1.003750pt}%
\definecolor{currentstroke}{rgb}{0.000000,0.000000,0.000000}%
\pgfsetstrokecolor{currentstroke}%
\pgfsetdash{}{0pt}%
\pgfpathmoveto{\pgfqpoint{7.473231in}{2.116808in}}%
\pgfpathcurveto{\pgfqpoint{7.481468in}{2.116808in}}{\pgfqpoint{7.489368in}{2.120080in}}{\pgfqpoint{7.495192in}{2.125904in}}%
\pgfpathcurveto{\pgfqpoint{7.501015in}{2.131728in}}{\pgfqpoint{7.504288in}{2.139628in}}{\pgfqpoint{7.504288in}{2.147864in}}%
\pgfpathcurveto{\pgfqpoint{7.504288in}{2.156100in}}{\pgfqpoint{7.501015in}{2.164000in}}{\pgfqpoint{7.495192in}{2.169824in}}%
\pgfpathcurveto{\pgfqpoint{7.489368in}{2.175648in}}{\pgfqpoint{7.481468in}{2.178921in}}{\pgfqpoint{7.473231in}{2.178921in}}%
\pgfpathcurveto{\pgfqpoint{7.464995in}{2.178921in}}{\pgfqpoint{7.457095in}{2.175648in}}{\pgfqpoint{7.451271in}{2.169824in}}%
\pgfpathcurveto{\pgfqpoint{7.445447in}{2.164000in}}{\pgfqpoint{7.442175in}{2.156100in}}{\pgfqpoint{7.442175in}{2.147864in}}%
\pgfpathcurveto{\pgfqpoint{7.442175in}{2.139628in}}{\pgfqpoint{7.445447in}{2.131728in}}{\pgfqpoint{7.451271in}{2.125904in}}%
\pgfpathcurveto{\pgfqpoint{7.457095in}{2.120080in}}{\pgfqpoint{7.464995in}{2.116808in}}{\pgfqpoint{7.473231in}{2.116808in}}%
\pgfpathclose%
\pgfusepath{stroke,fill}%
\end{pgfscope}%
\begin{pgfscope}%
\pgfpathrectangle{\pgfqpoint{0.894063in}{0.630000in}}{\pgfqpoint{6.713438in}{2.060556in}} %
\pgfusepath{clip}%
\pgfsetbuttcap%
\pgfsetroundjoin%
\definecolor{currentfill}{rgb}{0.000000,0.000000,1.000000}%
\pgfsetfillcolor{currentfill}%
\pgfsetlinewidth{1.003750pt}%
\definecolor{currentstroke}{rgb}{0.000000,0.000000,0.000000}%
\pgfsetstrokecolor{currentstroke}%
\pgfsetdash{}{0pt}%
\pgfpathmoveto{\pgfqpoint{5.056394in}{1.748710in}}%
\pgfpathcurveto{\pgfqpoint{5.064630in}{1.748710in}}{\pgfqpoint{5.072530in}{1.751982in}}{\pgfqpoint{5.078354in}{1.757806in}}%
\pgfpathcurveto{\pgfqpoint{5.084178in}{1.763630in}}{\pgfqpoint{5.087450in}{1.771530in}}{\pgfqpoint{5.087450in}{1.779766in}}%
\pgfpathcurveto{\pgfqpoint{5.087450in}{1.788003in}}{\pgfqpoint{5.084178in}{1.795903in}}{\pgfqpoint{5.078354in}{1.801727in}}%
\pgfpathcurveto{\pgfqpoint{5.072530in}{1.807551in}}{\pgfqpoint{5.064630in}{1.810823in}}{\pgfqpoint{5.056394in}{1.810823in}}%
\pgfpathcurveto{\pgfqpoint{5.048157in}{1.810823in}}{\pgfqpoint{5.040257in}{1.807551in}}{\pgfqpoint{5.034433in}{1.801727in}}%
\pgfpathcurveto{\pgfqpoint{5.028610in}{1.795903in}}{\pgfqpoint{5.025337in}{1.788003in}}{\pgfqpoint{5.025337in}{1.779766in}}%
\pgfpathcurveto{\pgfqpoint{5.025337in}{1.771530in}}{\pgfqpoint{5.028610in}{1.763630in}}{\pgfqpoint{5.034433in}{1.757806in}}%
\pgfpathcurveto{\pgfqpoint{5.040257in}{1.751982in}}{\pgfqpoint{5.048157in}{1.748710in}}{\pgfqpoint{5.056394in}{1.748710in}}%
\pgfpathclose%
\pgfusepath{stroke,fill}%
\end{pgfscope}%
\begin{pgfscope}%
\pgfpathrectangle{\pgfqpoint{0.894063in}{0.630000in}}{\pgfqpoint{6.713438in}{2.060556in}} %
\pgfusepath{clip}%
\pgfsetbuttcap%
\pgfsetroundjoin%
\definecolor{currentfill}{rgb}{0.000000,0.000000,1.000000}%
\pgfsetfillcolor{currentfill}%
\pgfsetlinewidth{1.003750pt}%
\definecolor{currentstroke}{rgb}{0.000000,0.000000,0.000000}%
\pgfsetstrokecolor{currentstroke}%
\pgfsetdash{}{0pt}%
\pgfpathmoveto{\pgfqpoint{2.908094in}{1.429889in}}%
\pgfpathcurveto{\pgfqpoint{2.916330in}{1.429889in}}{\pgfqpoint{2.924230in}{1.433161in}}{\pgfqpoint{2.930054in}{1.438985in}}%
\pgfpathcurveto{\pgfqpoint{2.935878in}{1.444809in}}{\pgfqpoint{2.939150in}{1.452709in}}{\pgfqpoint{2.939150in}{1.460946in}}%
\pgfpathcurveto{\pgfqpoint{2.939150in}{1.469182in}}{\pgfqpoint{2.935878in}{1.477082in}}{\pgfqpoint{2.930054in}{1.482906in}}%
\pgfpathcurveto{\pgfqpoint{2.924230in}{1.488730in}}{\pgfqpoint{2.916330in}{1.492002in}}{\pgfqpoint{2.908094in}{1.492002in}}%
\pgfpathcurveto{\pgfqpoint{2.899857in}{1.492002in}}{\pgfqpoint{2.891957in}{1.488730in}}{\pgfqpoint{2.886133in}{1.482906in}}%
\pgfpathcurveto{\pgfqpoint{2.880310in}{1.477082in}}{\pgfqpoint{2.877037in}{1.469182in}}{\pgfqpoint{2.877037in}{1.460946in}}%
\pgfpathcurveto{\pgfqpoint{2.877037in}{1.452709in}}{\pgfqpoint{2.880310in}{1.444809in}}{\pgfqpoint{2.886133in}{1.438985in}}%
\pgfpathcurveto{\pgfqpoint{2.891957in}{1.433161in}}{\pgfqpoint{2.899857in}{1.429889in}}{\pgfqpoint{2.908094in}{1.429889in}}%
\pgfpathclose%
\pgfusepath{stroke,fill}%
\end{pgfscope}%
\begin{pgfscope}%
\pgfpathrectangle{\pgfqpoint{0.894063in}{0.630000in}}{\pgfqpoint{6.713438in}{2.060556in}} %
\pgfusepath{clip}%
\pgfsetbuttcap%
\pgfsetroundjoin%
\definecolor{currentfill}{rgb}{0.000000,0.000000,1.000000}%
\pgfsetfillcolor{currentfill}%
\pgfsetlinewidth{1.003750pt}%
\definecolor{currentstroke}{rgb}{0.000000,0.000000,0.000000}%
\pgfsetstrokecolor{currentstroke}%
\pgfsetdash{}{0pt}%
\pgfpathmoveto{\pgfqpoint{3.445169in}{1.509568in}}%
\pgfpathcurveto{\pgfqpoint{3.453405in}{1.509568in}}{\pgfqpoint{3.461305in}{1.512840in}}{\pgfqpoint{3.467129in}{1.518664in}}%
\pgfpathcurveto{\pgfqpoint{3.472953in}{1.524488in}}{\pgfqpoint{3.476225in}{1.532388in}}{\pgfqpoint{3.476225in}{1.540624in}}%
\pgfpathcurveto{\pgfqpoint{3.476225in}{1.548861in}}{\pgfqpoint{3.472953in}{1.556761in}}{\pgfqpoint{3.467129in}{1.562585in}}%
\pgfpathcurveto{\pgfqpoint{3.461305in}{1.568408in}}{\pgfqpoint{3.453405in}{1.571681in}}{\pgfqpoint{3.445169in}{1.571681in}}%
\pgfpathcurveto{\pgfqpoint{3.436932in}{1.571681in}}{\pgfqpoint{3.429032in}{1.568408in}}{\pgfqpoint{3.423208in}{1.562585in}}%
\pgfpathcurveto{\pgfqpoint{3.417385in}{1.556761in}}{\pgfqpoint{3.414112in}{1.548861in}}{\pgfqpoint{3.414112in}{1.540624in}}%
\pgfpathcurveto{\pgfqpoint{3.414112in}{1.532388in}}{\pgfqpoint{3.417385in}{1.524488in}}{\pgfqpoint{3.423208in}{1.518664in}}%
\pgfpathcurveto{\pgfqpoint{3.429032in}{1.512840in}}{\pgfqpoint{3.436932in}{1.509568in}}{\pgfqpoint{3.445169in}{1.509568in}}%
\pgfpathclose%
\pgfusepath{stroke,fill}%
\end{pgfscope}%
\begin{pgfscope}%
\pgfpathrectangle{\pgfqpoint{0.894063in}{0.630000in}}{\pgfqpoint{6.713438in}{2.060556in}} %
\pgfusepath{clip}%
\pgfsetbuttcap%
\pgfsetroundjoin%
\definecolor{currentfill}{rgb}{0.000000,0.000000,1.000000}%
\pgfsetfillcolor{currentfill}%
\pgfsetlinewidth{1.003750pt}%
\definecolor{currentstroke}{rgb}{0.000000,0.000000,0.000000}%
\pgfsetstrokecolor{currentstroke}%
\pgfsetdash{}{0pt}%
\pgfpathmoveto{\pgfqpoint{4.116513in}{1.612531in}}%
\pgfpathcurveto{\pgfqpoint{4.124749in}{1.612531in}}{\pgfqpoint{4.132649in}{1.615803in}}{\pgfqpoint{4.138473in}{1.621627in}}%
\pgfpathcurveto{\pgfqpoint{4.144297in}{1.627451in}}{\pgfqpoint{4.147569in}{1.635351in}}{\pgfqpoint{4.147569in}{1.643587in}}%
\pgfpathcurveto{\pgfqpoint{4.147569in}{1.651824in}}{\pgfqpoint{4.144297in}{1.659724in}}{\pgfqpoint{4.138473in}{1.665548in}}%
\pgfpathcurveto{\pgfqpoint{4.132649in}{1.671371in}}{\pgfqpoint{4.124749in}{1.674644in}}{\pgfqpoint{4.116513in}{1.674644in}}%
\pgfpathcurveto{\pgfqpoint{4.108276in}{1.674644in}}{\pgfqpoint{4.100376in}{1.671371in}}{\pgfqpoint{4.094552in}{1.665548in}}%
\pgfpathcurveto{\pgfqpoint{4.088728in}{1.659724in}}{\pgfqpoint{4.085456in}{1.651824in}}{\pgfqpoint{4.085456in}{1.643587in}}%
\pgfpathcurveto{\pgfqpoint{4.085456in}{1.635351in}}{\pgfqpoint{4.088728in}{1.627451in}}{\pgfqpoint{4.094552in}{1.621627in}}%
\pgfpathcurveto{\pgfqpoint{4.100376in}{1.615803in}}{\pgfqpoint{4.108276in}{1.612531in}}{\pgfqpoint{4.116513in}{1.612531in}}%
\pgfpathclose%
\pgfusepath{stroke,fill}%
\end{pgfscope}%
\begin{pgfscope}%
\pgfpathrectangle{\pgfqpoint{0.894063in}{0.630000in}}{\pgfqpoint{6.713438in}{2.060556in}} %
\pgfusepath{clip}%
\pgfsetbuttcap%
\pgfsetroundjoin%
\definecolor{currentfill}{rgb}{0.000000,0.000000,1.000000}%
\pgfsetfillcolor{currentfill}%
\pgfsetlinewidth{1.003750pt}%
\definecolor{currentstroke}{rgb}{0.000000,0.000000,0.000000}%
\pgfsetstrokecolor{currentstroke}%
\pgfsetdash{}{0pt}%
\pgfpathmoveto{\pgfqpoint{1.431138in}{1.248554in}}%
\pgfpathcurveto{\pgfqpoint{1.439374in}{1.248554in}}{\pgfqpoint{1.447274in}{1.251827in}}{\pgfqpoint{1.453098in}{1.257650in}}%
\pgfpathcurveto{\pgfqpoint{1.458922in}{1.263474in}}{\pgfqpoint{1.462194in}{1.271374in}}{\pgfqpoint{1.462194in}{1.279611in}}%
\pgfpathcurveto{\pgfqpoint{1.462194in}{1.287847in}}{\pgfqpoint{1.458922in}{1.295747in}}{\pgfqpoint{1.453098in}{1.301571in}}%
\pgfpathcurveto{\pgfqpoint{1.447274in}{1.307395in}}{\pgfqpoint{1.439374in}{1.310667in}}{\pgfqpoint{1.431138in}{1.310667in}}%
\pgfpathcurveto{\pgfqpoint{1.422901in}{1.310667in}}{\pgfqpoint{1.415001in}{1.307395in}}{\pgfqpoint{1.409177in}{1.301571in}}%
\pgfpathcurveto{\pgfqpoint{1.403353in}{1.295747in}}{\pgfqpoint{1.400081in}{1.287847in}}{\pgfqpoint{1.400081in}{1.279611in}}%
\pgfpathcurveto{\pgfqpoint{1.400081in}{1.271374in}}{\pgfqpoint{1.403353in}{1.263474in}}{\pgfqpoint{1.409177in}{1.257650in}}%
\pgfpathcurveto{\pgfqpoint{1.415001in}{1.251827in}}{\pgfqpoint{1.422901in}{1.248554in}}{\pgfqpoint{1.431138in}{1.248554in}}%
\pgfpathclose%
\pgfusepath{stroke,fill}%
\end{pgfscope}%
\begin{pgfscope}%
\pgfpathrectangle{\pgfqpoint{0.894063in}{0.630000in}}{\pgfqpoint{6.713438in}{2.060556in}} %
\pgfusepath{clip}%
\pgfsetbuttcap%
\pgfsetroundjoin%
\definecolor{currentfill}{rgb}{0.000000,0.000000,1.000000}%
\pgfsetfillcolor{currentfill}%
\pgfsetlinewidth{1.003750pt}%
\definecolor{currentstroke}{rgb}{0.000000,0.000000,0.000000}%
\pgfsetstrokecolor{currentstroke}%
\pgfsetdash{}{0pt}%
\pgfpathmoveto{\pgfqpoint{2.773825in}{1.407923in}}%
\pgfpathcurveto{\pgfqpoint{2.782061in}{1.407923in}}{\pgfqpoint{2.789961in}{1.411196in}}{\pgfqpoint{2.795785in}{1.417020in}}%
\pgfpathcurveto{\pgfqpoint{2.801609in}{1.422844in}}{\pgfqpoint{2.804881in}{1.430744in}}{\pgfqpoint{2.804881in}{1.438980in}}%
\pgfpathcurveto{\pgfqpoint{2.804881in}{1.447216in}}{\pgfqpoint{2.801609in}{1.455116in}}{\pgfqpoint{2.795785in}{1.460940in}}%
\pgfpathcurveto{\pgfqpoint{2.789961in}{1.466764in}}{\pgfqpoint{2.782061in}{1.470036in}}{\pgfqpoint{2.773825in}{1.470036in}}%
\pgfpathcurveto{\pgfqpoint{2.765589in}{1.470036in}}{\pgfqpoint{2.757689in}{1.466764in}}{\pgfqpoint{2.751865in}{1.460940in}}%
\pgfpathcurveto{\pgfqpoint{2.746041in}{1.455116in}}{\pgfqpoint{2.742769in}{1.447216in}}{\pgfqpoint{2.742769in}{1.438980in}}%
\pgfpathcurveto{\pgfqpoint{2.742769in}{1.430744in}}{\pgfqpoint{2.746041in}{1.422844in}}{\pgfqpoint{2.751865in}{1.417020in}}%
\pgfpathcurveto{\pgfqpoint{2.757689in}{1.411196in}}{\pgfqpoint{2.765589in}{1.407923in}}{\pgfqpoint{2.773825in}{1.407923in}}%
\pgfpathclose%
\pgfusepath{stroke,fill}%
\end{pgfscope}%
\begin{pgfscope}%
\pgfpathrectangle{\pgfqpoint{0.894063in}{0.630000in}}{\pgfqpoint{6.713438in}{2.060556in}} %
\pgfusepath{clip}%
\pgfsetbuttcap%
\pgfsetroundjoin%
\definecolor{currentfill}{rgb}{0.000000,0.000000,1.000000}%
\pgfsetfillcolor{currentfill}%
\pgfsetlinewidth{1.003750pt}%
\definecolor{currentstroke}{rgb}{0.000000,0.000000,0.000000}%
\pgfsetstrokecolor{currentstroke}%
\pgfsetdash{}{0pt}%
\pgfpathmoveto{\pgfqpoint{1.565406in}{1.252434in}}%
\pgfpathcurveto{\pgfqpoint{1.573643in}{1.252434in}}{\pgfqpoint{1.581543in}{1.255706in}}{\pgfqpoint{1.587367in}{1.261530in}}%
\pgfpathcurveto{\pgfqpoint{1.593190in}{1.267354in}}{\pgfqpoint{1.596463in}{1.275254in}}{\pgfqpoint{1.596463in}{1.283490in}}%
\pgfpathcurveto{\pgfqpoint{1.596463in}{1.291727in}}{\pgfqpoint{1.593190in}{1.299627in}}{\pgfqpoint{1.587367in}{1.305451in}}%
\pgfpathcurveto{\pgfqpoint{1.581543in}{1.311275in}}{\pgfqpoint{1.573643in}{1.314547in}}{\pgfqpoint{1.565406in}{1.314547in}}%
\pgfpathcurveto{\pgfqpoint{1.557170in}{1.314547in}}{\pgfqpoint{1.549270in}{1.311275in}}{\pgfqpoint{1.543446in}{1.305451in}}%
\pgfpathcurveto{\pgfqpoint{1.537622in}{1.299627in}}{\pgfqpoint{1.534350in}{1.291727in}}{\pgfqpoint{1.534350in}{1.283490in}}%
\pgfpathcurveto{\pgfqpoint{1.534350in}{1.275254in}}{\pgfqpoint{1.537622in}{1.267354in}}{\pgfqpoint{1.543446in}{1.261530in}}%
\pgfpathcurveto{\pgfqpoint{1.549270in}{1.255706in}}{\pgfqpoint{1.557170in}{1.252434in}}{\pgfqpoint{1.565406in}{1.252434in}}%
\pgfpathclose%
\pgfusepath{stroke,fill}%
\end{pgfscope}%
\begin{pgfscope}%
\pgfpathrectangle{\pgfqpoint{0.894063in}{0.630000in}}{\pgfqpoint{6.713438in}{2.060556in}} %
\pgfusepath{clip}%
\pgfsetbuttcap%
\pgfsetroundjoin%
\definecolor{currentfill}{rgb}{0.000000,0.000000,1.000000}%
\pgfsetfillcolor{currentfill}%
\pgfsetlinewidth{1.003750pt}%
\definecolor{currentstroke}{rgb}{0.000000,0.000000,0.000000}%
\pgfsetstrokecolor{currentstroke}%
\pgfsetdash{}{0pt}%
\pgfpathmoveto{\pgfqpoint{4.250781in}{1.630717in}}%
\pgfpathcurveto{\pgfqpoint{4.259018in}{1.630717in}}{\pgfqpoint{4.266918in}{1.633989in}}{\pgfqpoint{4.272742in}{1.639813in}}%
\pgfpathcurveto{\pgfqpoint{4.278565in}{1.645637in}}{\pgfqpoint{4.281838in}{1.653537in}}{\pgfqpoint{4.281838in}{1.661773in}}%
\pgfpathcurveto{\pgfqpoint{4.281838in}{1.670009in}}{\pgfqpoint{4.278565in}{1.677909in}}{\pgfqpoint{4.272742in}{1.683733in}}%
\pgfpathcurveto{\pgfqpoint{4.266918in}{1.689557in}}{\pgfqpoint{4.259018in}{1.692830in}}{\pgfqpoint{4.250781in}{1.692830in}}%
\pgfpathcurveto{\pgfqpoint{4.242545in}{1.692830in}}{\pgfqpoint{4.234645in}{1.689557in}}{\pgfqpoint{4.228821in}{1.683733in}}%
\pgfpathcurveto{\pgfqpoint{4.222997in}{1.677909in}}{\pgfqpoint{4.219725in}{1.670009in}}{\pgfqpoint{4.219725in}{1.661773in}}%
\pgfpathcurveto{\pgfqpoint{4.219725in}{1.653537in}}{\pgfqpoint{4.222997in}{1.645637in}}{\pgfqpoint{4.228821in}{1.639813in}}%
\pgfpathcurveto{\pgfqpoint{4.234645in}{1.633989in}}{\pgfqpoint{4.242545in}{1.630717in}}{\pgfqpoint{4.250781in}{1.630717in}}%
\pgfpathclose%
\pgfusepath{stroke,fill}%
\end{pgfscope}%
\begin{pgfscope}%
\pgfpathrectangle{\pgfqpoint{0.894063in}{0.630000in}}{\pgfqpoint{6.713438in}{2.060556in}} %
\pgfusepath{clip}%
\pgfsetbuttcap%
\pgfsetroundjoin%
\definecolor{currentfill}{rgb}{0.000000,0.000000,1.000000}%
\pgfsetfillcolor{currentfill}%
\pgfsetlinewidth{1.003750pt}%
\definecolor{currentstroke}{rgb}{0.000000,0.000000,0.000000}%
\pgfsetstrokecolor{currentstroke}%
\pgfsetdash{}{0pt}%
\pgfpathmoveto{\pgfqpoint{3.847975in}{1.572215in}}%
\pgfpathcurveto{\pgfqpoint{3.856211in}{1.572215in}}{\pgfqpoint{3.864111in}{1.575487in}}{\pgfqpoint{3.869935in}{1.581311in}}%
\pgfpathcurveto{\pgfqpoint{3.875759in}{1.587135in}}{\pgfqpoint{3.879031in}{1.595035in}}{\pgfqpoint{3.879031in}{1.603271in}}%
\pgfpathcurveto{\pgfqpoint{3.879031in}{1.611507in}}{\pgfqpoint{3.875759in}{1.619407in}}{\pgfqpoint{3.869935in}{1.625231in}}%
\pgfpathcurveto{\pgfqpoint{3.864111in}{1.631055in}}{\pgfqpoint{3.856211in}{1.634328in}}{\pgfqpoint{3.847975in}{1.634328in}}%
\pgfpathcurveto{\pgfqpoint{3.839739in}{1.634328in}}{\pgfqpoint{3.831839in}{1.631055in}}{\pgfqpoint{3.826015in}{1.625231in}}%
\pgfpathcurveto{\pgfqpoint{3.820191in}{1.619407in}}{\pgfqpoint{3.816919in}{1.611507in}}{\pgfqpoint{3.816919in}{1.603271in}}%
\pgfpathcurveto{\pgfqpoint{3.816919in}{1.595035in}}{\pgfqpoint{3.820191in}{1.587135in}}{\pgfqpoint{3.826015in}{1.581311in}}%
\pgfpathcurveto{\pgfqpoint{3.831839in}{1.575487in}}{\pgfqpoint{3.839739in}{1.572215in}}{\pgfqpoint{3.847975in}{1.572215in}}%
\pgfpathclose%
\pgfusepath{stroke,fill}%
\end{pgfscope}%
\begin{pgfscope}%
\pgfpathrectangle{\pgfqpoint{0.894063in}{0.630000in}}{\pgfqpoint{6.713438in}{2.060556in}} %
\pgfusepath{clip}%
\pgfsetbuttcap%
\pgfsetroundjoin%
\definecolor{currentfill}{rgb}{0.000000,0.000000,1.000000}%
\pgfsetfillcolor{currentfill}%
\pgfsetlinewidth{1.003750pt}%
\definecolor{currentstroke}{rgb}{0.000000,0.000000,0.000000}%
\pgfsetstrokecolor{currentstroke}%
\pgfsetdash{}{0pt}%
\pgfpathmoveto{\pgfqpoint{7.607500in}{2.138632in}}%
\pgfpathcurveto{\pgfqpoint{7.615736in}{2.138632in}}{\pgfqpoint{7.623636in}{2.141904in}}{\pgfqpoint{7.629460in}{2.147728in}}%
\pgfpathcurveto{\pgfqpoint{7.635284in}{2.153552in}}{\pgfqpoint{7.638556in}{2.161452in}}{\pgfqpoint{7.638556in}{2.169688in}}%
\pgfpathcurveto{\pgfqpoint{7.638556in}{2.177925in}}{\pgfqpoint{7.635284in}{2.185825in}}{\pgfqpoint{7.629460in}{2.191649in}}%
\pgfpathcurveto{\pgfqpoint{7.623636in}{2.197473in}}{\pgfqpoint{7.615736in}{2.200745in}}{\pgfqpoint{7.607500in}{2.200745in}}%
\pgfpathcurveto{\pgfqpoint{7.599264in}{2.200745in}}{\pgfqpoint{7.591364in}{2.197473in}}{\pgfqpoint{7.585540in}{2.191649in}}%
\pgfpathcurveto{\pgfqpoint{7.579716in}{2.185825in}}{\pgfqpoint{7.576444in}{2.177925in}}{\pgfqpoint{7.576444in}{2.169688in}}%
\pgfpathcurveto{\pgfqpoint{7.576444in}{2.161452in}}{\pgfqpoint{7.579716in}{2.153552in}}{\pgfqpoint{7.585540in}{2.147728in}}%
\pgfpathcurveto{\pgfqpoint{7.591364in}{2.141904in}}{\pgfqpoint{7.599264in}{2.138632in}}{\pgfqpoint{7.607500in}{2.138632in}}%
\pgfpathclose%
\pgfusepath{stroke,fill}%
\end{pgfscope}%
\begin{pgfscope}%
\pgfpathrectangle{\pgfqpoint{0.894063in}{0.630000in}}{\pgfqpoint{6.713438in}{2.060556in}} %
\pgfusepath{clip}%
\pgfsetbuttcap%
\pgfsetroundjoin%
\definecolor{currentfill}{rgb}{0.000000,0.000000,1.000000}%
\pgfsetfillcolor{currentfill}%
\pgfsetlinewidth{1.003750pt}%
\definecolor{currentstroke}{rgb}{0.000000,0.000000,0.000000}%
\pgfsetstrokecolor{currentstroke}%
\pgfsetdash{}{0pt}%
\pgfpathmoveto{\pgfqpoint{4.385050in}{1.652511in}}%
\pgfpathcurveto{\pgfqpoint{4.393286in}{1.652511in}}{\pgfqpoint{4.401186in}{1.655784in}}{\pgfqpoint{4.407010in}{1.661608in}}%
\pgfpathcurveto{\pgfqpoint{4.412834in}{1.667432in}}{\pgfqpoint{4.416106in}{1.675332in}}{\pgfqpoint{4.416106in}{1.683568in}}%
\pgfpathcurveto{\pgfqpoint{4.416106in}{1.691804in}}{\pgfqpoint{4.412834in}{1.699704in}}{\pgfqpoint{4.407010in}{1.705528in}}%
\pgfpathcurveto{\pgfqpoint{4.401186in}{1.711352in}}{\pgfqpoint{4.393286in}{1.714624in}}{\pgfqpoint{4.385050in}{1.714624in}}%
\pgfpathcurveto{\pgfqpoint{4.376814in}{1.714624in}}{\pgfqpoint{4.368914in}{1.711352in}}{\pgfqpoint{4.363090in}{1.705528in}}%
\pgfpathcurveto{\pgfqpoint{4.357266in}{1.699704in}}{\pgfqpoint{4.353994in}{1.691804in}}{\pgfqpoint{4.353994in}{1.683568in}}%
\pgfpathcurveto{\pgfqpoint{4.353994in}{1.675332in}}{\pgfqpoint{4.357266in}{1.667432in}}{\pgfqpoint{4.363090in}{1.661608in}}%
\pgfpathcurveto{\pgfqpoint{4.368914in}{1.655784in}}{\pgfqpoint{4.376814in}{1.652511in}}{\pgfqpoint{4.385050in}{1.652511in}}%
\pgfpathclose%
\pgfusepath{stroke,fill}%
\end{pgfscope}%
\begin{pgfscope}%
\pgfpathrectangle{\pgfqpoint{0.894063in}{0.630000in}}{\pgfqpoint{6.713438in}{2.060556in}} %
\pgfusepath{clip}%
\pgfsetbuttcap%
\pgfsetroundjoin%
\definecolor{currentfill}{rgb}{0.000000,0.000000,1.000000}%
\pgfsetfillcolor{currentfill}%
\pgfsetlinewidth{1.003750pt}%
\definecolor{currentstroke}{rgb}{0.000000,0.000000,0.000000}%
\pgfsetstrokecolor{currentstroke}%
\pgfsetdash{}{0pt}%
\pgfpathmoveto{\pgfqpoint{6.533350in}{1.980358in}}%
\pgfpathcurveto{\pgfqpoint{6.541586in}{1.980358in}}{\pgfqpoint{6.549486in}{1.983630in}}{\pgfqpoint{6.555310in}{1.989454in}}%
\pgfpathcurveto{\pgfqpoint{6.561134in}{1.995278in}}{\pgfqpoint{6.564406in}{2.003178in}}{\pgfqpoint{6.564406in}{2.011414in}}%
\pgfpathcurveto{\pgfqpoint{6.564406in}{2.019650in}}{\pgfqpoint{6.561134in}{2.027550in}}{\pgfqpoint{6.555310in}{2.033374in}}%
\pgfpathcurveto{\pgfqpoint{6.549486in}{2.039198in}}{\pgfqpoint{6.541586in}{2.042471in}}{\pgfqpoint{6.533350in}{2.042471in}}%
\pgfpathcurveto{\pgfqpoint{6.525114in}{2.042471in}}{\pgfqpoint{6.517214in}{2.039198in}}{\pgfqpoint{6.511390in}{2.033374in}}%
\pgfpathcurveto{\pgfqpoint{6.505566in}{2.027550in}}{\pgfqpoint{6.502294in}{2.019650in}}{\pgfqpoint{6.502294in}{2.011414in}}%
\pgfpathcurveto{\pgfqpoint{6.502294in}{2.003178in}}{\pgfqpoint{6.505566in}{1.995278in}}{\pgfqpoint{6.511390in}{1.989454in}}%
\pgfpathcurveto{\pgfqpoint{6.517214in}{1.983630in}}{\pgfqpoint{6.525114in}{1.980358in}}{\pgfqpoint{6.533350in}{1.980358in}}%
\pgfpathclose%
\pgfusepath{stroke,fill}%
\end{pgfscope}%
\begin{pgfscope}%
\pgfpathrectangle{\pgfqpoint{0.894063in}{0.630000in}}{\pgfqpoint{6.713438in}{2.060556in}} %
\pgfusepath{clip}%
\pgfsetbuttcap%
\pgfsetroundjoin%
\definecolor{currentfill}{rgb}{0.000000,0.000000,1.000000}%
\pgfsetfillcolor{currentfill}%
\pgfsetlinewidth{1.003750pt}%
\definecolor{currentstroke}{rgb}{0.000000,0.000000,0.000000}%
\pgfsetstrokecolor{currentstroke}%
\pgfsetdash{}{0pt}%
\pgfpathmoveto{\pgfqpoint{1.296869in}{0.714859in}}%
\pgfpathcurveto{\pgfqpoint{1.305105in}{0.714859in}}{\pgfqpoint{1.313005in}{0.718131in}}{\pgfqpoint{1.318829in}{0.723955in}}%
\pgfpathcurveto{\pgfqpoint{1.324653in}{0.729779in}}{\pgfqpoint{1.327925in}{0.737679in}}{\pgfqpoint{1.327925in}{0.745915in}}%
\pgfpathcurveto{\pgfqpoint{1.327925in}{0.754151in}}{\pgfqpoint{1.324653in}{0.762051in}}{\pgfqpoint{1.318829in}{0.767875in}}%
\pgfpathcurveto{\pgfqpoint{1.313005in}{0.773699in}}{\pgfqpoint{1.305105in}{0.776972in}}{\pgfqpoint{1.296869in}{0.776972in}}%
\pgfpathcurveto{\pgfqpoint{1.288632in}{0.776972in}}{\pgfqpoint{1.280732in}{0.773699in}}{\pgfqpoint{1.274908in}{0.767875in}}%
\pgfpathcurveto{\pgfqpoint{1.269085in}{0.762051in}}{\pgfqpoint{1.265812in}{0.754151in}}{\pgfqpoint{1.265812in}{0.745915in}}%
\pgfpathcurveto{\pgfqpoint{1.265812in}{0.737679in}}{\pgfqpoint{1.269085in}{0.729779in}}{\pgfqpoint{1.274908in}{0.723955in}}%
\pgfpathcurveto{\pgfqpoint{1.280732in}{0.718131in}}{\pgfqpoint{1.288632in}{0.714859in}}{\pgfqpoint{1.296869in}{0.714859in}}%
\pgfpathclose%
\pgfusepath{stroke,fill}%
\end{pgfscope}%
\begin{pgfscope}%
\pgfpathrectangle{\pgfqpoint{0.894063in}{0.630000in}}{\pgfqpoint{6.713438in}{2.060556in}} %
\pgfusepath{clip}%
\pgfsetbuttcap%
\pgfsetroundjoin%
\definecolor{currentfill}{rgb}{0.000000,0.000000,1.000000}%
\pgfsetfillcolor{currentfill}%
\pgfsetlinewidth{1.003750pt}%
\definecolor{currentstroke}{rgb}{0.000000,0.000000,0.000000}%
\pgfsetstrokecolor{currentstroke}%
\pgfsetdash{}{0pt}%
\pgfpathmoveto{\pgfqpoint{4.519319in}{1.671274in}}%
\pgfpathcurveto{\pgfqpoint{4.527555in}{1.671274in}}{\pgfqpoint{4.535455in}{1.674547in}}{\pgfqpoint{4.541279in}{1.680371in}}%
\pgfpathcurveto{\pgfqpoint{4.547103in}{1.686194in}}{\pgfqpoint{4.550375in}{1.694094in}}{\pgfqpoint{4.550375in}{1.702331in}}%
\pgfpathcurveto{\pgfqpoint{4.550375in}{1.710567in}}{\pgfqpoint{4.547103in}{1.718467in}}{\pgfqpoint{4.541279in}{1.724291in}}%
\pgfpathcurveto{\pgfqpoint{4.535455in}{1.730115in}}{\pgfqpoint{4.527555in}{1.733387in}}{\pgfqpoint{4.519319in}{1.733387in}}%
\pgfpathcurveto{\pgfqpoint{4.511082in}{1.733387in}}{\pgfqpoint{4.503182in}{1.730115in}}{\pgfqpoint{4.497358in}{1.724291in}}%
\pgfpathcurveto{\pgfqpoint{4.491535in}{1.718467in}}{\pgfqpoint{4.488262in}{1.710567in}}{\pgfqpoint{4.488262in}{1.702331in}}%
\pgfpathcurveto{\pgfqpoint{4.488262in}{1.694094in}}{\pgfqpoint{4.491535in}{1.686194in}}{\pgfqpoint{4.497358in}{1.680371in}}%
\pgfpathcurveto{\pgfqpoint{4.503182in}{1.674547in}}{\pgfqpoint{4.511082in}{1.671274in}}{\pgfqpoint{4.519319in}{1.671274in}}%
\pgfpathclose%
\pgfusepath{stroke,fill}%
\end{pgfscope}%
\begin{pgfscope}%
\pgfpathrectangle{\pgfqpoint{0.894063in}{0.630000in}}{\pgfqpoint{6.713438in}{2.060556in}} %
\pgfusepath{clip}%
\pgfsetbuttcap%
\pgfsetroundjoin%
\definecolor{currentfill}{rgb}{0.000000,0.000000,1.000000}%
\pgfsetfillcolor{currentfill}%
\pgfsetlinewidth{1.003750pt}%
\definecolor{currentstroke}{rgb}{0.000000,0.000000,0.000000}%
\pgfsetstrokecolor{currentstroke}%
\pgfsetdash{}{0pt}%
\pgfpathmoveto{\pgfqpoint{2.505288in}{1.365429in}}%
\pgfpathcurveto{\pgfqpoint{2.513524in}{1.365429in}}{\pgfqpoint{2.521424in}{1.368701in}}{\pgfqpoint{2.527248in}{1.374525in}}%
\pgfpathcurveto{\pgfqpoint{2.533072in}{1.380349in}}{\pgfqpoint{2.536344in}{1.388249in}}{\pgfqpoint{2.536344in}{1.396485in}}%
\pgfpathcurveto{\pgfqpoint{2.536344in}{1.404722in}}{\pgfqpoint{2.533072in}{1.412622in}}{\pgfqpoint{2.527248in}{1.418446in}}%
\pgfpathcurveto{\pgfqpoint{2.521424in}{1.424270in}}{\pgfqpoint{2.513524in}{1.427542in}}{\pgfqpoint{2.505288in}{1.427542in}}%
\pgfpathcurveto{\pgfqpoint{2.497051in}{1.427542in}}{\pgfqpoint{2.489151in}{1.424270in}}{\pgfqpoint{2.483327in}{1.418446in}}%
\pgfpathcurveto{\pgfqpoint{2.477503in}{1.412622in}}{\pgfqpoint{2.474231in}{1.404722in}}{\pgfqpoint{2.474231in}{1.396485in}}%
\pgfpathcurveto{\pgfqpoint{2.474231in}{1.388249in}}{\pgfqpoint{2.477503in}{1.380349in}}{\pgfqpoint{2.483327in}{1.374525in}}%
\pgfpathcurveto{\pgfqpoint{2.489151in}{1.368701in}}{\pgfqpoint{2.497051in}{1.365429in}}{\pgfqpoint{2.505288in}{1.365429in}}%
\pgfpathclose%
\pgfusepath{stroke,fill}%
\end{pgfscope}%
\begin{pgfscope}%
\pgfpathrectangle{\pgfqpoint{0.894063in}{0.630000in}}{\pgfqpoint{6.713438in}{2.060556in}} %
\pgfusepath{clip}%
\pgfsetbuttcap%
\pgfsetroundjoin%
\definecolor{currentfill}{rgb}{0.000000,0.000000,1.000000}%
\pgfsetfillcolor{currentfill}%
\pgfsetlinewidth{1.003750pt}%
\definecolor{currentstroke}{rgb}{0.000000,0.000000,0.000000}%
\pgfsetstrokecolor{currentstroke}%
\pgfsetdash{}{0pt}%
\pgfpathmoveto{\pgfqpoint{5.459200in}{1.821065in}}%
\pgfpathcurveto{\pgfqpoint{5.467436in}{1.821065in}}{\pgfqpoint{5.475336in}{1.824337in}}{\pgfqpoint{5.481160in}{1.830161in}}%
\pgfpathcurveto{\pgfqpoint{5.486984in}{1.835985in}}{\pgfqpoint{5.490256in}{1.843885in}}{\pgfqpoint{5.490256in}{1.852121in}}%
\pgfpathcurveto{\pgfqpoint{5.490256in}{1.860358in}}{\pgfqpoint{5.486984in}{1.868258in}}{\pgfqpoint{5.481160in}{1.874082in}}%
\pgfpathcurveto{\pgfqpoint{5.475336in}{1.879906in}}{\pgfqpoint{5.467436in}{1.883178in}}{\pgfqpoint{5.459200in}{1.883178in}}%
\pgfpathcurveto{\pgfqpoint{5.450964in}{1.883178in}}{\pgfqpoint{5.443064in}{1.879906in}}{\pgfqpoint{5.437240in}{1.874082in}}%
\pgfpathcurveto{\pgfqpoint{5.431416in}{1.868258in}}{\pgfqpoint{5.428144in}{1.860358in}}{\pgfqpoint{5.428144in}{1.852121in}}%
\pgfpathcurveto{\pgfqpoint{5.428144in}{1.843885in}}{\pgfqpoint{5.431416in}{1.835985in}}{\pgfqpoint{5.437240in}{1.830161in}}%
\pgfpathcurveto{\pgfqpoint{5.443064in}{1.824337in}}{\pgfqpoint{5.450964in}{1.821065in}}{\pgfqpoint{5.459200in}{1.821065in}}%
\pgfpathclose%
\pgfusepath{stroke,fill}%
\end{pgfscope}%
\begin{pgfscope}%
\pgfpathrectangle{\pgfqpoint{0.894063in}{0.630000in}}{\pgfqpoint{6.713438in}{2.060556in}} %
\pgfusepath{clip}%
\pgfsetbuttcap%
\pgfsetroundjoin%
\definecolor{currentfill}{rgb}{0.000000,0.000000,1.000000}%
\pgfsetfillcolor{currentfill}%
\pgfsetlinewidth{1.003750pt}%
\definecolor{currentstroke}{rgb}{0.000000,0.000000,0.000000}%
\pgfsetstrokecolor{currentstroke}%
\pgfsetdash{}{0pt}%
\pgfpathmoveto{\pgfqpoint{6.936156in}{2.043104in}}%
\pgfpathcurveto{\pgfqpoint{6.944393in}{2.043104in}}{\pgfqpoint{6.952293in}{2.046377in}}{\pgfqpoint{6.958117in}{2.052201in}}%
\pgfpathcurveto{\pgfqpoint{6.963940in}{2.058025in}}{\pgfqpoint{6.967213in}{2.065925in}}{\pgfqpoint{6.967213in}{2.074161in}}%
\pgfpathcurveto{\pgfqpoint{6.967213in}{2.082397in}}{\pgfqpoint{6.963940in}{2.090297in}}{\pgfqpoint{6.958117in}{2.096121in}}%
\pgfpathcurveto{\pgfqpoint{6.952293in}{2.101945in}}{\pgfqpoint{6.944393in}{2.105217in}}{\pgfqpoint{6.936156in}{2.105217in}}%
\pgfpathcurveto{\pgfqpoint{6.927920in}{2.105217in}}{\pgfqpoint{6.920020in}{2.101945in}}{\pgfqpoint{6.914196in}{2.096121in}}%
\pgfpathcurveto{\pgfqpoint{6.908372in}{2.090297in}}{\pgfqpoint{6.905100in}{2.082397in}}{\pgfqpoint{6.905100in}{2.074161in}}%
\pgfpathcurveto{\pgfqpoint{6.905100in}{2.065925in}}{\pgfqpoint{6.908372in}{2.058025in}}{\pgfqpoint{6.914196in}{2.052201in}}%
\pgfpathcurveto{\pgfqpoint{6.920020in}{2.046377in}}{\pgfqpoint{6.927920in}{2.043104in}}{\pgfqpoint{6.936156in}{2.043104in}}%
\pgfpathclose%
\pgfusepath{stroke,fill}%
\end{pgfscope}%
\begin{pgfscope}%
\pgfpathrectangle{\pgfqpoint{0.894063in}{0.630000in}}{\pgfqpoint{6.713438in}{2.060556in}} %
\pgfusepath{clip}%
\pgfsetbuttcap%
\pgfsetroundjoin%
\definecolor{currentfill}{rgb}{0.000000,0.000000,1.000000}%
\pgfsetfillcolor{currentfill}%
\pgfsetlinewidth{1.003750pt}%
\definecolor{currentstroke}{rgb}{0.000000,0.000000,0.000000}%
\pgfsetstrokecolor{currentstroke}%
\pgfsetdash{}{0pt}%
\pgfpathmoveto{\pgfqpoint{5.862006in}{1.879944in}}%
\pgfpathcurveto{\pgfqpoint{5.870243in}{1.879944in}}{\pgfqpoint{5.878143in}{1.883216in}}{\pgfqpoint{5.883967in}{1.889040in}}%
\pgfpathcurveto{\pgfqpoint{5.889790in}{1.894864in}}{\pgfqpoint{5.893063in}{1.902764in}}{\pgfqpoint{5.893063in}{1.911000in}}%
\pgfpathcurveto{\pgfqpoint{5.893063in}{1.919237in}}{\pgfqpoint{5.889790in}{1.927137in}}{\pgfqpoint{5.883967in}{1.932961in}}%
\pgfpathcurveto{\pgfqpoint{5.878143in}{1.938784in}}{\pgfqpoint{5.870243in}{1.942057in}}{\pgfqpoint{5.862006in}{1.942057in}}%
\pgfpathcurveto{\pgfqpoint{5.853770in}{1.942057in}}{\pgfqpoint{5.845870in}{1.938784in}}{\pgfqpoint{5.840046in}{1.932961in}}%
\pgfpathcurveto{\pgfqpoint{5.834222in}{1.927137in}}{\pgfqpoint{5.830950in}{1.919237in}}{\pgfqpoint{5.830950in}{1.911000in}}%
\pgfpathcurveto{\pgfqpoint{5.830950in}{1.902764in}}{\pgfqpoint{5.834222in}{1.894864in}}{\pgfqpoint{5.840046in}{1.889040in}}%
\pgfpathcurveto{\pgfqpoint{5.845870in}{1.883216in}}{\pgfqpoint{5.853770in}{1.879944in}}{\pgfqpoint{5.862006in}{1.879944in}}%
\pgfpathclose%
\pgfusepath{stroke,fill}%
\end{pgfscope}%
\begin{pgfscope}%
\pgfpathrectangle{\pgfqpoint{0.894063in}{0.630000in}}{\pgfqpoint{6.713438in}{2.060556in}} %
\pgfusepath{clip}%
\pgfsetbuttcap%
\pgfsetroundjoin%
\definecolor{currentfill}{rgb}{0.000000,0.000000,1.000000}%
\pgfsetfillcolor{currentfill}%
\pgfsetlinewidth{1.003750pt}%
\definecolor{currentstroke}{rgb}{0.000000,0.000000,0.000000}%
\pgfsetstrokecolor{currentstroke}%
\pgfsetdash{}{0pt}%
\pgfpathmoveto{\pgfqpoint{7.070425in}{2.058759in}}%
\pgfpathcurveto{\pgfqpoint{7.078661in}{2.058759in}}{\pgfqpoint{7.086561in}{2.062031in}}{\pgfqpoint{7.092385in}{2.067855in}}%
\pgfpathcurveto{\pgfqpoint{7.098209in}{2.073679in}}{\pgfqpoint{7.101481in}{2.081579in}}{\pgfqpoint{7.101481in}{2.089815in}}%
\pgfpathcurveto{\pgfqpoint{7.101481in}{2.098052in}}{\pgfqpoint{7.098209in}{2.105952in}}{\pgfqpoint{7.092385in}{2.111776in}}%
\pgfpathcurveto{\pgfqpoint{7.086561in}{2.117599in}}{\pgfqpoint{7.078661in}{2.120872in}}{\pgfqpoint{7.070425in}{2.120872in}}%
\pgfpathcurveto{\pgfqpoint{7.062189in}{2.120872in}}{\pgfqpoint{7.054289in}{2.117599in}}{\pgfqpoint{7.048465in}{2.111776in}}%
\pgfpathcurveto{\pgfqpoint{7.042641in}{2.105952in}}{\pgfqpoint{7.039369in}{2.098052in}}{\pgfqpoint{7.039369in}{2.089815in}}%
\pgfpathcurveto{\pgfqpoint{7.039369in}{2.081579in}}{\pgfqpoint{7.042641in}{2.073679in}}{\pgfqpoint{7.048465in}{2.067855in}}%
\pgfpathcurveto{\pgfqpoint{7.054289in}{2.062031in}}{\pgfqpoint{7.062189in}{2.058759in}}{\pgfqpoint{7.070425in}{2.058759in}}%
\pgfpathclose%
\pgfusepath{stroke,fill}%
\end{pgfscope}%
\begin{pgfscope}%
\pgfpathrectangle{\pgfqpoint{0.894063in}{0.630000in}}{\pgfqpoint{6.713438in}{2.060556in}} %
\pgfusepath{clip}%
\pgfsetbuttcap%
\pgfsetroundjoin%
\definecolor{currentfill}{rgb}{0.000000,0.000000,1.000000}%
\pgfsetfillcolor{currentfill}%
\pgfsetlinewidth{1.003750pt}%
\definecolor{currentstroke}{rgb}{0.000000,0.000000,0.000000}%
\pgfsetstrokecolor{currentstroke}%
\pgfsetdash{}{0pt}%
\pgfpathmoveto{\pgfqpoint{3.176631in}{1.472902in}}%
\pgfpathcurveto{\pgfqpoint{3.184868in}{1.472902in}}{\pgfqpoint{3.192768in}{1.476174in}}{\pgfqpoint{3.198592in}{1.481998in}}%
\pgfpathcurveto{\pgfqpoint{3.204415in}{1.487822in}}{\pgfqpoint{3.207688in}{1.495722in}}{\pgfqpoint{3.207688in}{1.503958in}}%
\pgfpathcurveto{\pgfqpoint{3.207688in}{1.512194in}}{\pgfqpoint{3.204415in}{1.520094in}}{\pgfqpoint{3.198592in}{1.525918in}}%
\pgfpathcurveto{\pgfqpoint{3.192768in}{1.531742in}}{\pgfqpoint{3.184868in}{1.535015in}}{\pgfqpoint{3.176631in}{1.535015in}}%
\pgfpathcurveto{\pgfqpoint{3.168395in}{1.535015in}}{\pgfqpoint{3.160495in}{1.531742in}}{\pgfqpoint{3.154671in}{1.525918in}}%
\pgfpathcurveto{\pgfqpoint{3.148847in}{1.520094in}}{\pgfqpoint{3.145575in}{1.512194in}}{\pgfqpoint{3.145575in}{1.503958in}}%
\pgfpathcurveto{\pgfqpoint{3.145575in}{1.495722in}}{\pgfqpoint{3.148847in}{1.487822in}}{\pgfqpoint{3.154671in}{1.481998in}}%
\pgfpathcurveto{\pgfqpoint{3.160495in}{1.476174in}}{\pgfqpoint{3.168395in}{1.472902in}}{\pgfqpoint{3.176631in}{1.472902in}}%
\pgfpathclose%
\pgfusepath{stroke,fill}%
\end{pgfscope}%
\begin{pgfscope}%
\pgfpathrectangle{\pgfqpoint{0.894063in}{0.630000in}}{\pgfqpoint{6.713438in}{2.060556in}} %
\pgfusepath{clip}%
\pgfsetbuttcap%
\pgfsetroundjoin%
\definecolor{currentfill}{rgb}{0.000000,0.000000,1.000000}%
\pgfsetfillcolor{currentfill}%
\pgfsetlinewidth{1.003750pt}%
\definecolor{currentstroke}{rgb}{0.000000,0.000000,0.000000}%
\pgfsetstrokecolor{currentstroke}%
\pgfsetdash{}{0pt}%
\pgfpathmoveto{\pgfqpoint{2.102481in}{1.299114in}}%
\pgfpathcurveto{\pgfqpoint{2.110718in}{1.299114in}}{\pgfqpoint{2.118618in}{1.302387in}}{\pgfqpoint{2.124442in}{1.308211in}}%
\pgfpathcurveto{\pgfqpoint{2.130265in}{1.314035in}}{\pgfqpoint{2.133538in}{1.321935in}}{\pgfqpoint{2.133538in}{1.330171in}}%
\pgfpathcurveto{\pgfqpoint{2.133538in}{1.338407in}}{\pgfqpoint{2.130265in}{1.346307in}}{\pgfqpoint{2.124442in}{1.352131in}}%
\pgfpathcurveto{\pgfqpoint{2.118618in}{1.357955in}}{\pgfqpoint{2.110718in}{1.361227in}}{\pgfqpoint{2.102481in}{1.361227in}}%
\pgfpathcurveto{\pgfqpoint{2.094245in}{1.361227in}}{\pgfqpoint{2.086345in}{1.357955in}}{\pgfqpoint{2.080521in}{1.352131in}}%
\pgfpathcurveto{\pgfqpoint{2.074697in}{1.346307in}}{\pgfqpoint{2.071425in}{1.338407in}}{\pgfqpoint{2.071425in}{1.330171in}}%
\pgfpathcurveto{\pgfqpoint{2.071425in}{1.321935in}}{\pgfqpoint{2.074697in}{1.314035in}}{\pgfqpoint{2.080521in}{1.308211in}}%
\pgfpathcurveto{\pgfqpoint{2.086345in}{1.302387in}}{\pgfqpoint{2.094245in}{1.299114in}}{\pgfqpoint{2.102481in}{1.299114in}}%
\pgfpathclose%
\pgfusepath{stroke,fill}%
\end{pgfscope}%
\begin{pgfscope}%
\pgfpathrectangle{\pgfqpoint{0.894063in}{0.630000in}}{\pgfqpoint{6.713438in}{2.060556in}} %
\pgfusepath{clip}%
\pgfsetbuttcap%
\pgfsetroundjoin%
\definecolor{currentfill}{rgb}{0.000000,0.000000,1.000000}%
\pgfsetfillcolor{currentfill}%
\pgfsetlinewidth{1.003750pt}%
\definecolor{currentstroke}{rgb}{0.000000,0.000000,0.000000}%
\pgfsetstrokecolor{currentstroke}%
\pgfsetdash{}{0pt}%
\pgfpathmoveto{\pgfqpoint{1.968213in}{1.289124in}}%
\pgfpathcurveto{\pgfqpoint{1.976449in}{1.289124in}}{\pgfqpoint{1.984349in}{1.292396in}}{\pgfqpoint{1.990173in}{1.298220in}}%
\pgfpathcurveto{\pgfqpoint{1.995997in}{1.304044in}}{\pgfqpoint{1.999269in}{1.311944in}}{\pgfqpoint{1.999269in}{1.320180in}}%
\pgfpathcurveto{\pgfqpoint{1.999269in}{1.328416in}}{\pgfqpoint{1.995997in}{1.336316in}}{\pgfqpoint{1.990173in}{1.342140in}}%
\pgfpathcurveto{\pgfqpoint{1.984349in}{1.347964in}}{\pgfqpoint{1.976449in}{1.351237in}}{\pgfqpoint{1.968213in}{1.351237in}}%
\pgfpathcurveto{\pgfqpoint{1.959976in}{1.351237in}}{\pgfqpoint{1.952076in}{1.347964in}}{\pgfqpoint{1.946252in}{1.342140in}}%
\pgfpathcurveto{\pgfqpoint{1.940428in}{1.336316in}}{\pgfqpoint{1.937156in}{1.328416in}}{\pgfqpoint{1.937156in}{1.320180in}}%
\pgfpathcurveto{\pgfqpoint{1.937156in}{1.311944in}}{\pgfqpoint{1.940428in}{1.304044in}}{\pgfqpoint{1.946252in}{1.298220in}}%
\pgfpathcurveto{\pgfqpoint{1.952076in}{1.292396in}}{\pgfqpoint{1.959976in}{1.289124in}}{\pgfqpoint{1.968213in}{1.289124in}}%
\pgfpathclose%
\pgfusepath{stroke,fill}%
\end{pgfscope}%
\begin{pgfscope}%
\pgfpathrectangle{\pgfqpoint{0.894063in}{0.630000in}}{\pgfqpoint{6.713438in}{2.060556in}} %
\pgfusepath{clip}%
\pgfsetbuttcap%
\pgfsetroundjoin%
\definecolor{currentfill}{rgb}{0.000000,0.000000,1.000000}%
\pgfsetfillcolor{currentfill}%
\pgfsetlinewidth{1.003750pt}%
\definecolor{currentstroke}{rgb}{0.000000,0.000000,0.000000}%
\pgfsetstrokecolor{currentstroke}%
\pgfsetdash{}{0pt}%
\pgfpathmoveto{\pgfqpoint{3.310900in}{1.494108in}}%
\pgfpathcurveto{\pgfqpoint{3.319136in}{1.494108in}}{\pgfqpoint{3.327036in}{1.497380in}}{\pgfqpoint{3.332860in}{1.503204in}}%
\pgfpathcurveto{\pgfqpoint{3.338684in}{1.509028in}}{\pgfqpoint{3.341956in}{1.516928in}}{\pgfqpoint{3.341956in}{1.525164in}}%
\pgfpathcurveto{\pgfqpoint{3.341956in}{1.533400in}}{\pgfqpoint{3.338684in}{1.541301in}}{\pgfqpoint{3.332860in}{1.547124in}}%
\pgfpathcurveto{\pgfqpoint{3.327036in}{1.552948in}}{\pgfqpoint{3.319136in}{1.556221in}}{\pgfqpoint{3.310900in}{1.556221in}}%
\pgfpathcurveto{\pgfqpoint{3.302664in}{1.556221in}}{\pgfqpoint{3.294764in}{1.552948in}}{\pgfqpoint{3.288940in}{1.547124in}}%
\pgfpathcurveto{\pgfqpoint{3.283116in}{1.541301in}}{\pgfqpoint{3.279844in}{1.533400in}}{\pgfqpoint{3.279844in}{1.525164in}}%
\pgfpathcurveto{\pgfqpoint{3.279844in}{1.516928in}}{\pgfqpoint{3.283116in}{1.509028in}}{\pgfqpoint{3.288940in}{1.503204in}}%
\pgfpathcurveto{\pgfqpoint{3.294764in}{1.497380in}}{\pgfqpoint{3.302664in}{1.494108in}}{\pgfqpoint{3.310900in}{1.494108in}}%
\pgfpathclose%
\pgfusepath{stroke,fill}%
\end{pgfscope}%
\begin{pgfscope}%
\pgfpathrectangle{\pgfqpoint{0.894063in}{0.630000in}}{\pgfqpoint{6.713438in}{2.060556in}} %
\pgfusepath{clip}%
\pgfsetbuttcap%
\pgfsetroundjoin%
\definecolor{currentfill}{rgb}{0.000000,0.000000,1.000000}%
\pgfsetfillcolor{currentfill}%
\pgfsetlinewidth{1.003750pt}%
\definecolor{currentstroke}{rgb}{0.000000,0.000000,0.000000}%
\pgfsetstrokecolor{currentstroke}%
\pgfsetdash{}{0pt}%
\pgfpathmoveto{\pgfqpoint{5.593469in}{1.840805in}}%
\pgfpathcurveto{\pgfqpoint{5.601705in}{1.840805in}}{\pgfqpoint{5.609605in}{1.844077in}}{\pgfqpoint{5.615429in}{1.849901in}}%
\pgfpathcurveto{\pgfqpoint{5.621253in}{1.855725in}}{\pgfqpoint{5.624525in}{1.863625in}}{\pgfqpoint{5.624525in}{1.871862in}}%
\pgfpathcurveto{\pgfqpoint{5.624525in}{1.880098in}}{\pgfqpoint{5.621253in}{1.887998in}}{\pgfqpoint{5.615429in}{1.893822in}}%
\pgfpathcurveto{\pgfqpoint{5.609605in}{1.899646in}}{\pgfqpoint{5.601705in}{1.902918in}}{\pgfqpoint{5.593469in}{1.902918in}}%
\pgfpathcurveto{\pgfqpoint{5.585232in}{1.902918in}}{\pgfqpoint{5.577332in}{1.899646in}}{\pgfqpoint{5.571508in}{1.893822in}}%
\pgfpathcurveto{\pgfqpoint{5.565685in}{1.887998in}}{\pgfqpoint{5.562412in}{1.880098in}}{\pgfqpoint{5.562412in}{1.871862in}}%
\pgfpathcurveto{\pgfqpoint{5.562412in}{1.863625in}}{\pgfqpoint{5.565685in}{1.855725in}}{\pgfqpoint{5.571508in}{1.849901in}}%
\pgfpathcurveto{\pgfqpoint{5.577332in}{1.844077in}}{\pgfqpoint{5.585232in}{1.840805in}}{\pgfqpoint{5.593469in}{1.840805in}}%
\pgfpathclose%
\pgfusepath{stroke,fill}%
\end{pgfscope}%
\begin{pgfscope}%
\pgfpathrectangle{\pgfqpoint{0.894063in}{0.630000in}}{\pgfqpoint{6.713438in}{2.060556in}} %
\pgfusepath{clip}%
\pgfsetbuttcap%
\pgfsetroundjoin%
\definecolor{currentfill}{rgb}{0.000000,0.000000,1.000000}%
\pgfsetfillcolor{currentfill}%
\pgfsetlinewidth{1.003750pt}%
\definecolor{currentstroke}{rgb}{0.000000,0.000000,0.000000}%
\pgfsetstrokecolor{currentstroke}%
\pgfsetdash{}{0pt}%
\pgfpathmoveto{\pgfqpoint{3.042363in}{1.450318in}}%
\pgfpathcurveto{\pgfqpoint{3.050599in}{1.450318in}}{\pgfqpoint{3.058499in}{1.453590in}}{\pgfqpoint{3.064323in}{1.459414in}}%
\pgfpathcurveto{\pgfqpoint{3.070147in}{1.465238in}}{\pgfqpoint{3.073419in}{1.473138in}}{\pgfqpoint{3.073419in}{1.481374in}}%
\pgfpathcurveto{\pgfqpoint{3.073419in}{1.489611in}}{\pgfqpoint{3.070147in}{1.497511in}}{\pgfqpoint{3.064323in}{1.503335in}}%
\pgfpathcurveto{\pgfqpoint{3.058499in}{1.509159in}}{\pgfqpoint{3.050599in}{1.512431in}}{\pgfqpoint{3.042363in}{1.512431in}}%
\pgfpathcurveto{\pgfqpoint{3.034126in}{1.512431in}}{\pgfqpoint{3.026226in}{1.509159in}}{\pgfqpoint{3.020402in}{1.503335in}}%
\pgfpathcurveto{\pgfqpoint{3.014578in}{1.497511in}}{\pgfqpoint{3.011306in}{1.489611in}}{\pgfqpoint{3.011306in}{1.481374in}}%
\pgfpathcurveto{\pgfqpoint{3.011306in}{1.473138in}}{\pgfqpoint{3.014578in}{1.465238in}}{\pgfqpoint{3.020402in}{1.459414in}}%
\pgfpathcurveto{\pgfqpoint{3.026226in}{1.453590in}}{\pgfqpoint{3.034126in}{1.450318in}}{\pgfqpoint{3.042363in}{1.450318in}}%
\pgfpathclose%
\pgfusepath{stroke,fill}%
\end{pgfscope}%
\begin{pgfscope}%
\pgfpathrectangle{\pgfqpoint{0.894063in}{0.630000in}}{\pgfqpoint{6.713438in}{2.060556in}} %
\pgfusepath{clip}%
\pgfsetbuttcap%
\pgfsetroundjoin%
\definecolor{currentfill}{rgb}{0.000000,0.000000,1.000000}%
\pgfsetfillcolor{currentfill}%
\pgfsetlinewidth{1.003750pt}%
\definecolor{currentstroke}{rgb}{0.000000,0.000000,0.000000}%
\pgfsetstrokecolor{currentstroke}%
\pgfsetdash{}{0pt}%
\pgfpathmoveto{\pgfqpoint{5.190663in}{1.771423in}}%
\pgfpathcurveto{\pgfqpoint{5.198899in}{1.771423in}}{\pgfqpoint{5.206799in}{1.774695in}}{\pgfqpoint{5.212623in}{1.780519in}}%
\pgfpathcurveto{\pgfqpoint{5.218447in}{1.786343in}}{\pgfqpoint{5.221719in}{1.794243in}}{\pgfqpoint{5.221719in}{1.802480in}}%
\pgfpathcurveto{\pgfqpoint{5.221719in}{1.810716in}}{\pgfqpoint{5.218447in}{1.818616in}}{\pgfqpoint{5.212623in}{1.824440in}}%
\pgfpathcurveto{\pgfqpoint{5.206799in}{1.830264in}}{\pgfqpoint{5.198899in}{1.833536in}}{\pgfqpoint{5.190663in}{1.833536in}}%
\pgfpathcurveto{\pgfqpoint{5.182426in}{1.833536in}}{\pgfqpoint{5.174526in}{1.830264in}}{\pgfqpoint{5.168702in}{1.824440in}}%
\pgfpathcurveto{\pgfqpoint{5.162878in}{1.818616in}}{\pgfqpoint{5.159606in}{1.810716in}}{\pgfqpoint{5.159606in}{1.802480in}}%
\pgfpathcurveto{\pgfqpoint{5.159606in}{1.794243in}}{\pgfqpoint{5.162878in}{1.786343in}}{\pgfqpoint{5.168702in}{1.780519in}}%
\pgfpathcurveto{\pgfqpoint{5.174526in}{1.774695in}}{\pgfqpoint{5.182426in}{1.771423in}}{\pgfqpoint{5.190663in}{1.771423in}}%
\pgfpathclose%
\pgfusepath{stroke,fill}%
\end{pgfscope}%
\begin{pgfscope}%
\pgfpathrectangle{\pgfqpoint{0.894063in}{0.630000in}}{\pgfqpoint{6.713438in}{2.060556in}} %
\pgfusepath{clip}%
\pgfsetbuttcap%
\pgfsetroundjoin%
\definecolor{currentfill}{rgb}{0.000000,0.000000,1.000000}%
\pgfsetfillcolor{currentfill}%
\pgfsetlinewidth{1.003750pt}%
\definecolor{currentstroke}{rgb}{0.000000,0.000000,0.000000}%
\pgfsetstrokecolor{currentstroke}%
\pgfsetdash{}{0pt}%
\pgfpathmoveto{\pgfqpoint{6.801888in}{2.022258in}}%
\pgfpathcurveto{\pgfqpoint{6.810124in}{2.022258in}}{\pgfqpoint{6.818024in}{2.025530in}}{\pgfqpoint{6.823848in}{2.031354in}}%
\pgfpathcurveto{\pgfqpoint{6.829672in}{2.037178in}}{\pgfqpoint{6.832944in}{2.045078in}}{\pgfqpoint{6.832944in}{2.053314in}}%
\pgfpathcurveto{\pgfqpoint{6.832944in}{2.061550in}}{\pgfqpoint{6.829672in}{2.069450in}}{\pgfqpoint{6.823848in}{2.075274in}}%
\pgfpathcurveto{\pgfqpoint{6.818024in}{2.081098in}}{\pgfqpoint{6.810124in}{2.084371in}}{\pgfqpoint{6.801888in}{2.084371in}}%
\pgfpathcurveto{\pgfqpoint{6.793651in}{2.084371in}}{\pgfqpoint{6.785751in}{2.081098in}}{\pgfqpoint{6.779927in}{2.075274in}}%
\pgfpathcurveto{\pgfqpoint{6.774103in}{2.069450in}}{\pgfqpoint{6.770831in}{2.061550in}}{\pgfqpoint{6.770831in}{2.053314in}}%
\pgfpathcurveto{\pgfqpoint{6.770831in}{2.045078in}}{\pgfqpoint{6.774103in}{2.037178in}}{\pgfqpoint{6.779927in}{2.031354in}}%
\pgfpathcurveto{\pgfqpoint{6.785751in}{2.025530in}}{\pgfqpoint{6.793651in}{2.022258in}}{\pgfqpoint{6.801888in}{2.022258in}}%
\pgfpathclose%
\pgfusepath{stroke,fill}%
\end{pgfscope}%
\begin{pgfscope}%
\pgfpathrectangle{\pgfqpoint{0.894063in}{0.630000in}}{\pgfqpoint{6.713438in}{2.060556in}} %
\pgfusepath{clip}%
\pgfsetbuttcap%
\pgfsetroundjoin%
\definecolor{currentfill}{rgb}{0.000000,0.000000,1.000000}%
\pgfsetfillcolor{currentfill}%
\pgfsetlinewidth{1.003750pt}%
\definecolor{currentstroke}{rgb}{0.000000,0.000000,0.000000}%
\pgfsetstrokecolor{currentstroke}%
\pgfsetdash{}{0pt}%
\pgfpathmoveto{\pgfqpoint{3.579438in}{1.523556in}}%
\pgfpathcurveto{\pgfqpoint{3.587674in}{1.523556in}}{\pgfqpoint{3.595574in}{1.526828in}}{\pgfqpoint{3.601398in}{1.532652in}}%
\pgfpathcurveto{\pgfqpoint{3.607222in}{1.538476in}}{\pgfqpoint{3.610494in}{1.546376in}}{\pgfqpoint{3.610494in}{1.554612in}}%
\pgfpathcurveto{\pgfqpoint{3.610494in}{1.562849in}}{\pgfqpoint{3.607222in}{1.570749in}}{\pgfqpoint{3.601398in}{1.576573in}}%
\pgfpathcurveto{\pgfqpoint{3.595574in}{1.582397in}}{\pgfqpoint{3.587674in}{1.585669in}}{\pgfqpoint{3.579438in}{1.585669in}}%
\pgfpathcurveto{\pgfqpoint{3.571201in}{1.585669in}}{\pgfqpoint{3.563301in}{1.582397in}}{\pgfqpoint{3.557477in}{1.576573in}}%
\pgfpathcurveto{\pgfqpoint{3.551653in}{1.570749in}}{\pgfqpoint{3.548381in}{1.562849in}}{\pgfqpoint{3.548381in}{1.554612in}}%
\pgfpathcurveto{\pgfqpoint{3.548381in}{1.546376in}}{\pgfqpoint{3.551653in}{1.538476in}}{\pgfqpoint{3.557477in}{1.532652in}}%
\pgfpathcurveto{\pgfqpoint{3.563301in}{1.526828in}}{\pgfqpoint{3.571201in}{1.523556in}}{\pgfqpoint{3.579438in}{1.523556in}}%
\pgfpathclose%
\pgfusepath{stroke,fill}%
\end{pgfscope}%
\begin{pgfscope}%
\pgfpathrectangle{\pgfqpoint{0.894063in}{0.630000in}}{\pgfqpoint{6.713438in}{2.060556in}} %
\pgfusepath{clip}%
\pgfsetbuttcap%
\pgfsetroundjoin%
\definecolor{currentfill}{rgb}{0.000000,0.000000,1.000000}%
\pgfsetfillcolor{currentfill}%
\pgfsetlinewidth{1.003750pt}%
\definecolor{currentstroke}{rgb}{0.000000,0.000000,0.000000}%
\pgfsetstrokecolor{currentstroke}%
\pgfsetdash{}{0pt}%
\pgfpathmoveto{\pgfqpoint{2.371019in}{1.349086in}}%
\pgfpathcurveto{\pgfqpoint{2.379255in}{1.349086in}}{\pgfqpoint{2.387155in}{1.352358in}}{\pgfqpoint{2.392979in}{1.358182in}}%
\pgfpathcurveto{\pgfqpoint{2.398803in}{1.364006in}}{\pgfqpoint{2.402075in}{1.371906in}}{\pgfqpoint{2.402075in}{1.380142in}}%
\pgfpathcurveto{\pgfqpoint{2.402075in}{1.388379in}}{\pgfqpoint{2.398803in}{1.396279in}}{\pgfqpoint{2.392979in}{1.402103in}}%
\pgfpathcurveto{\pgfqpoint{2.387155in}{1.407926in}}{\pgfqpoint{2.379255in}{1.411199in}}{\pgfqpoint{2.371019in}{1.411199in}}%
\pgfpathcurveto{\pgfqpoint{2.362782in}{1.411199in}}{\pgfqpoint{2.354882in}{1.407926in}}{\pgfqpoint{2.349058in}{1.402103in}}%
\pgfpathcurveto{\pgfqpoint{2.343235in}{1.396279in}}{\pgfqpoint{2.339962in}{1.388379in}}{\pgfqpoint{2.339962in}{1.380142in}}%
\pgfpathcurveto{\pgfqpoint{2.339962in}{1.371906in}}{\pgfqpoint{2.343235in}{1.364006in}}{\pgfqpoint{2.349058in}{1.358182in}}%
\pgfpathcurveto{\pgfqpoint{2.354882in}{1.352358in}}{\pgfqpoint{2.362782in}{1.349086in}}{\pgfqpoint{2.371019in}{1.349086in}}%
\pgfpathclose%
\pgfusepath{stroke,fill}%
\end{pgfscope}%
\begin{pgfscope}%
\pgfpathrectangle{\pgfqpoint{0.894063in}{0.630000in}}{\pgfqpoint{6.713438in}{2.060556in}} %
\pgfusepath{clip}%
\pgfsetbuttcap%
\pgfsetroundjoin%
\definecolor{currentfill}{rgb}{0.000000,0.000000,1.000000}%
\pgfsetfillcolor{currentfill}%
\pgfsetlinewidth{1.003750pt}%
\definecolor{currentstroke}{rgb}{0.000000,0.000000,0.000000}%
\pgfsetstrokecolor{currentstroke}%
\pgfsetdash{}{0pt}%
\pgfpathmoveto{\pgfqpoint{3.982244in}{1.592608in}}%
\pgfpathcurveto{\pgfqpoint{3.990480in}{1.592608in}}{\pgfqpoint{3.998380in}{1.595880in}}{\pgfqpoint{4.004204in}{1.601704in}}%
\pgfpathcurveto{\pgfqpoint{4.010028in}{1.607528in}}{\pgfqpoint{4.013300in}{1.615428in}}{\pgfqpoint{4.013300in}{1.623665in}}%
\pgfpathcurveto{\pgfqpoint{4.013300in}{1.631901in}}{\pgfqpoint{4.010028in}{1.639801in}}{\pgfqpoint{4.004204in}{1.645625in}}%
\pgfpathcurveto{\pgfqpoint{3.998380in}{1.651449in}}{\pgfqpoint{3.990480in}{1.654721in}}{\pgfqpoint{3.982244in}{1.654721in}}%
\pgfpathcurveto{\pgfqpoint{3.974007in}{1.654721in}}{\pgfqpoint{3.966107in}{1.651449in}}{\pgfqpoint{3.960283in}{1.645625in}}%
\pgfpathcurveto{\pgfqpoint{3.954460in}{1.639801in}}{\pgfqpoint{3.951187in}{1.631901in}}{\pgfqpoint{3.951187in}{1.623665in}}%
\pgfpathcurveto{\pgfqpoint{3.951187in}{1.615428in}}{\pgfqpoint{3.954460in}{1.607528in}}{\pgfqpoint{3.960283in}{1.601704in}}%
\pgfpathcurveto{\pgfqpoint{3.966107in}{1.595880in}}{\pgfqpoint{3.974007in}{1.592608in}}{\pgfqpoint{3.982244in}{1.592608in}}%
\pgfpathclose%
\pgfusepath{stroke,fill}%
\end{pgfscope}%
\begin{pgfscope}%
\pgfpathrectangle{\pgfqpoint{0.894063in}{0.630000in}}{\pgfqpoint{6.713438in}{2.060556in}} %
\pgfusepath{clip}%
\pgfsetbuttcap%
\pgfsetroundjoin%
\definecolor{currentfill}{rgb}{0.000000,0.000000,1.000000}%
\pgfsetfillcolor{currentfill}%
\pgfsetlinewidth{1.003750pt}%
\definecolor{currentstroke}{rgb}{0.000000,0.000000,0.000000}%
\pgfsetstrokecolor{currentstroke}%
\pgfsetdash{}{0pt}%
\pgfpathmoveto{\pgfqpoint{4.653588in}{1.694211in}}%
\pgfpathcurveto{\pgfqpoint{4.661824in}{1.694211in}}{\pgfqpoint{4.669724in}{1.697484in}}{\pgfqpoint{4.675548in}{1.703307in}}%
\pgfpathcurveto{\pgfqpoint{4.681372in}{1.709131in}}{\pgfqpoint{4.684644in}{1.717031in}}{\pgfqpoint{4.684644in}{1.725268in}}%
\pgfpathcurveto{\pgfqpoint{4.684644in}{1.733504in}}{\pgfqpoint{4.681372in}{1.741404in}}{\pgfqpoint{4.675548in}{1.747228in}}%
\pgfpathcurveto{\pgfqpoint{4.669724in}{1.753052in}}{\pgfqpoint{4.661824in}{1.756324in}}{\pgfqpoint{4.653588in}{1.756324in}}%
\pgfpathcurveto{\pgfqpoint{4.645351in}{1.756324in}}{\pgfqpoint{4.637451in}{1.753052in}}{\pgfqpoint{4.631627in}{1.747228in}}%
\pgfpathcurveto{\pgfqpoint{4.625803in}{1.741404in}}{\pgfqpoint{4.622531in}{1.733504in}}{\pgfqpoint{4.622531in}{1.725268in}}%
\pgfpathcurveto{\pgfqpoint{4.622531in}{1.717031in}}{\pgfqpoint{4.625803in}{1.709131in}}{\pgfqpoint{4.631627in}{1.703307in}}%
\pgfpathcurveto{\pgfqpoint{4.637451in}{1.697484in}}{\pgfqpoint{4.645351in}{1.694211in}}{\pgfqpoint{4.653588in}{1.694211in}}%
\pgfpathclose%
\pgfusepath{stroke,fill}%
\end{pgfscope}%
\begin{pgfscope}%
\pgfpathrectangle{\pgfqpoint{0.894063in}{0.630000in}}{\pgfqpoint{6.713438in}{2.060556in}} %
\pgfusepath{clip}%
\pgfsetbuttcap%
\pgfsetroundjoin%
\definecolor{currentfill}{rgb}{0.000000,0.000000,1.000000}%
\pgfsetfillcolor{currentfill}%
\pgfsetlinewidth{1.003750pt}%
\definecolor{currentstroke}{rgb}{0.000000,0.000000,0.000000}%
\pgfsetstrokecolor{currentstroke}%
\pgfsetdash{}{0pt}%
\pgfpathmoveto{\pgfqpoint{3.713706in}{1.548000in}}%
\pgfpathcurveto{\pgfqpoint{3.721943in}{1.548000in}}{\pgfqpoint{3.729843in}{1.551272in}}{\pgfqpoint{3.735667in}{1.557096in}}%
\pgfpathcurveto{\pgfqpoint{3.741490in}{1.562920in}}{\pgfqpoint{3.744763in}{1.570820in}}{\pgfqpoint{3.744763in}{1.579057in}}%
\pgfpathcurveto{\pgfqpoint{3.744763in}{1.587293in}}{\pgfqpoint{3.741490in}{1.595193in}}{\pgfqpoint{3.735667in}{1.601017in}}%
\pgfpathcurveto{\pgfqpoint{3.729843in}{1.606841in}}{\pgfqpoint{3.721943in}{1.610113in}}{\pgfqpoint{3.713706in}{1.610113in}}%
\pgfpathcurveto{\pgfqpoint{3.705470in}{1.610113in}}{\pgfqpoint{3.697570in}{1.606841in}}{\pgfqpoint{3.691746in}{1.601017in}}%
\pgfpathcurveto{\pgfqpoint{3.685922in}{1.595193in}}{\pgfqpoint{3.682650in}{1.587293in}}{\pgfqpoint{3.682650in}{1.579057in}}%
\pgfpathcurveto{\pgfqpoint{3.682650in}{1.570820in}}{\pgfqpoint{3.685922in}{1.562920in}}{\pgfqpoint{3.691746in}{1.557096in}}%
\pgfpathcurveto{\pgfqpoint{3.697570in}{1.551272in}}{\pgfqpoint{3.705470in}{1.548000in}}{\pgfqpoint{3.713706in}{1.548000in}}%
\pgfpathclose%
\pgfusepath{stroke,fill}%
\end{pgfscope}%
\begin{pgfscope}%
\pgfpathrectangle{\pgfqpoint{0.894063in}{0.630000in}}{\pgfqpoint{6.713438in}{2.060556in}} %
\pgfusepath{clip}%
\pgfsetbuttcap%
\pgfsetroundjoin%
\definecolor{currentfill}{rgb}{0.000000,0.000000,1.000000}%
\pgfsetfillcolor{currentfill}%
\pgfsetlinewidth{1.003750pt}%
\definecolor{currentstroke}{rgb}{0.000000,0.000000,0.000000}%
\pgfsetstrokecolor{currentstroke}%
\pgfsetdash{}{0pt}%
\pgfpathmoveto{\pgfqpoint{2.236750in}{1.326367in}}%
\pgfpathcurveto{\pgfqpoint{2.244986in}{1.326367in}}{\pgfqpoint{2.252886in}{1.329639in}}{\pgfqpoint{2.258710in}{1.335463in}}%
\pgfpathcurveto{\pgfqpoint{2.264534in}{1.341287in}}{\pgfqpoint{2.267806in}{1.349187in}}{\pgfqpoint{2.267806in}{1.357423in}}%
\pgfpathcurveto{\pgfqpoint{2.267806in}{1.365659in}}{\pgfqpoint{2.264534in}{1.373560in}}{\pgfqpoint{2.258710in}{1.379383in}}%
\pgfpathcurveto{\pgfqpoint{2.252886in}{1.385207in}}{\pgfqpoint{2.244986in}{1.388480in}}{\pgfqpoint{2.236750in}{1.388480in}}%
\pgfpathcurveto{\pgfqpoint{2.228514in}{1.388480in}}{\pgfqpoint{2.220614in}{1.385207in}}{\pgfqpoint{2.214790in}{1.379383in}}%
\pgfpathcurveto{\pgfqpoint{2.208966in}{1.373560in}}{\pgfqpoint{2.205694in}{1.365659in}}{\pgfqpoint{2.205694in}{1.357423in}}%
\pgfpathcurveto{\pgfqpoint{2.205694in}{1.349187in}}{\pgfqpoint{2.208966in}{1.341287in}}{\pgfqpoint{2.214790in}{1.335463in}}%
\pgfpathcurveto{\pgfqpoint{2.220614in}{1.329639in}}{\pgfqpoint{2.228514in}{1.326367in}}{\pgfqpoint{2.236750in}{1.326367in}}%
\pgfpathclose%
\pgfusepath{stroke,fill}%
\end{pgfscope}%
\begin{pgfscope}%
\pgfpathrectangle{\pgfqpoint{0.894063in}{0.630000in}}{\pgfqpoint{6.713438in}{2.060556in}} %
\pgfusepath{clip}%
\pgfsetbuttcap%
\pgfsetroundjoin%
\definecolor{currentfill}{rgb}{0.000000,0.000000,1.000000}%
\pgfsetfillcolor{currentfill}%
\pgfsetlinewidth{1.003750pt}%
\definecolor{currentstroke}{rgb}{0.000000,0.000000,0.000000}%
\pgfsetstrokecolor{currentstroke}%
\pgfsetdash{}{0pt}%
\pgfpathmoveto{\pgfqpoint{6.667619in}{2.102013in}}%
\pgfpathcurveto{\pgfqpoint{6.675855in}{2.102013in}}{\pgfqpoint{6.683755in}{2.105285in}}{\pgfqpoint{6.689579in}{2.111109in}}%
\pgfpathcurveto{\pgfqpoint{6.695403in}{2.116933in}}{\pgfqpoint{6.698675in}{2.124833in}}{\pgfqpoint{6.698675in}{2.133069in}}%
\pgfpathcurveto{\pgfqpoint{6.698675in}{2.141306in}}{\pgfqpoint{6.695403in}{2.149206in}}{\pgfqpoint{6.689579in}{2.155030in}}%
\pgfpathcurveto{\pgfqpoint{6.683755in}{2.160853in}}{\pgfqpoint{6.675855in}{2.164126in}}{\pgfqpoint{6.667619in}{2.164126in}}%
\pgfpathcurveto{\pgfqpoint{6.659382in}{2.164126in}}{\pgfqpoint{6.651482in}{2.160853in}}{\pgfqpoint{6.645658in}{2.155030in}}%
\pgfpathcurveto{\pgfqpoint{6.639835in}{2.149206in}}{\pgfqpoint{6.636562in}{2.141306in}}{\pgfqpoint{6.636562in}{2.133069in}}%
\pgfpathcurveto{\pgfqpoint{6.636562in}{2.124833in}}{\pgfqpoint{6.639835in}{2.116933in}}{\pgfqpoint{6.645658in}{2.111109in}}%
\pgfpathcurveto{\pgfqpoint{6.651482in}{2.105285in}}{\pgfqpoint{6.659382in}{2.102013in}}{\pgfqpoint{6.667619in}{2.102013in}}%
\pgfpathclose%
\pgfusepath{stroke,fill}%
\end{pgfscope}%
\begin{pgfscope}%
\pgfpathrectangle{\pgfqpoint{0.894063in}{0.630000in}}{\pgfqpoint{6.713438in}{2.060556in}} %
\pgfusepath{clip}%
\pgfsetbuttcap%
\pgfsetroundjoin%
\definecolor{currentfill}{rgb}{0.000000,0.000000,1.000000}%
\pgfsetfillcolor{currentfill}%
\pgfsetlinewidth{1.003750pt}%
\definecolor{currentstroke}{rgb}{0.000000,0.000000,0.000000}%
\pgfsetstrokecolor{currentstroke}%
\pgfsetdash{}{0pt}%
\pgfpathmoveto{\pgfqpoint{2.639556in}{1.400458in}}%
\pgfpathcurveto{\pgfqpoint{2.647793in}{1.400458in}}{\pgfqpoint{2.655693in}{1.403731in}}{\pgfqpoint{2.661517in}{1.409555in}}%
\pgfpathcurveto{\pgfqpoint{2.667340in}{1.415379in}}{\pgfqpoint{2.670613in}{1.423279in}}{\pgfqpoint{2.670613in}{1.431515in}}%
\pgfpathcurveto{\pgfqpoint{2.670613in}{1.439751in}}{\pgfqpoint{2.667340in}{1.447651in}}{\pgfqpoint{2.661517in}{1.453475in}}%
\pgfpathcurveto{\pgfqpoint{2.655693in}{1.459299in}}{\pgfqpoint{2.647793in}{1.462571in}}{\pgfqpoint{2.639556in}{1.462571in}}%
\pgfpathcurveto{\pgfqpoint{2.631320in}{1.462571in}}{\pgfqpoint{2.623420in}{1.459299in}}{\pgfqpoint{2.617596in}{1.453475in}}%
\pgfpathcurveto{\pgfqpoint{2.611772in}{1.447651in}}{\pgfqpoint{2.608500in}{1.439751in}}{\pgfqpoint{2.608500in}{1.431515in}}%
\pgfpathcurveto{\pgfqpoint{2.608500in}{1.423279in}}{\pgfqpoint{2.611772in}{1.415379in}}{\pgfqpoint{2.617596in}{1.409555in}}%
\pgfpathcurveto{\pgfqpoint{2.623420in}{1.403731in}}{\pgfqpoint{2.631320in}{1.400458in}}{\pgfqpoint{2.639556in}{1.400458in}}%
\pgfpathclose%
\pgfusepath{stroke,fill}%
\end{pgfscope}%
\begin{pgfscope}%
\pgfpathrectangle{\pgfqpoint{0.894063in}{0.630000in}}{\pgfqpoint{6.713438in}{2.060556in}} %
\pgfusepath{clip}%
\pgfsetbuttcap%
\pgfsetroundjoin%
\definecolor{currentfill}{rgb}{0.000000,0.000000,1.000000}%
\pgfsetfillcolor{currentfill}%
\pgfsetlinewidth{1.003750pt}%
\definecolor{currentstroke}{rgb}{0.000000,0.000000,0.000000}%
\pgfsetstrokecolor{currentstroke}%
\pgfsetdash{}{0pt}%
\pgfpathmoveto{\pgfqpoint{1.699675in}{1.267382in}}%
\pgfpathcurveto{\pgfqpoint{1.707911in}{1.267382in}}{\pgfqpoint{1.715811in}{1.270654in}}{\pgfqpoint{1.721635in}{1.276478in}}%
\pgfpathcurveto{\pgfqpoint{1.727459in}{1.282302in}}{\pgfqpoint{1.730731in}{1.290202in}}{\pgfqpoint{1.730731in}{1.298438in}}%
\pgfpathcurveto{\pgfqpoint{1.730731in}{1.306675in}}{\pgfqpoint{1.727459in}{1.314575in}}{\pgfqpoint{1.721635in}{1.320399in}}%
\pgfpathcurveto{\pgfqpoint{1.715811in}{1.326223in}}{\pgfqpoint{1.707911in}{1.329495in}}{\pgfqpoint{1.699675in}{1.329495in}}%
\pgfpathcurveto{\pgfqpoint{1.691439in}{1.329495in}}{\pgfqpoint{1.683539in}{1.326223in}}{\pgfqpoint{1.677715in}{1.320399in}}%
\pgfpathcurveto{\pgfqpoint{1.671891in}{1.314575in}}{\pgfqpoint{1.668619in}{1.306675in}}{\pgfqpoint{1.668619in}{1.298438in}}%
\pgfpathcurveto{\pgfqpoint{1.668619in}{1.290202in}}{\pgfqpoint{1.671891in}{1.282302in}}{\pgfqpoint{1.677715in}{1.276478in}}%
\pgfpathcurveto{\pgfqpoint{1.683539in}{1.270654in}}{\pgfqpoint{1.691439in}{1.267382in}}{\pgfqpoint{1.699675in}{1.267382in}}%
\pgfpathclose%
\pgfusepath{stroke,fill}%
\end{pgfscope}%
\begin{pgfscope}%
\pgfpathrectangle{\pgfqpoint{0.894063in}{0.630000in}}{\pgfqpoint{6.713438in}{2.060556in}} %
\pgfusepath{clip}%
\pgfsetbuttcap%
\pgfsetroundjoin%
\definecolor{currentfill}{rgb}{0.000000,0.000000,1.000000}%
\pgfsetfillcolor{currentfill}%
\pgfsetlinewidth{1.003750pt}%
\definecolor{currentstroke}{rgb}{0.000000,0.000000,0.000000}%
\pgfsetstrokecolor{currentstroke}%
\pgfsetdash{}{0pt}%
\pgfpathmoveto{\pgfqpoint{1.162600in}{0.699504in}}%
\pgfpathcurveto{\pgfqpoint{1.170836in}{0.699504in}}{\pgfqpoint{1.178736in}{0.702777in}}{\pgfqpoint{1.184560in}{0.708601in}}%
\pgfpathcurveto{\pgfqpoint{1.190384in}{0.714425in}}{\pgfqpoint{1.193656in}{0.722325in}}{\pgfqpoint{1.193656in}{0.730561in}}%
\pgfpathcurveto{\pgfqpoint{1.193656in}{0.738797in}}{\pgfqpoint{1.190384in}{0.746697in}}{\pgfqpoint{1.184560in}{0.752521in}}%
\pgfpathcurveto{\pgfqpoint{1.178736in}{0.758345in}}{\pgfqpoint{1.170836in}{0.761617in}}{\pgfqpoint{1.162600in}{0.761617in}}%
\pgfpathcurveto{\pgfqpoint{1.154364in}{0.761617in}}{\pgfqpoint{1.146464in}{0.758345in}}{\pgfqpoint{1.140640in}{0.752521in}}%
\pgfpathcurveto{\pgfqpoint{1.134816in}{0.746697in}}{\pgfqpoint{1.131544in}{0.738797in}}{\pgfqpoint{1.131544in}{0.730561in}}%
\pgfpathcurveto{\pgfqpoint{1.131544in}{0.722325in}}{\pgfqpoint{1.134816in}{0.714425in}}{\pgfqpoint{1.140640in}{0.708601in}}%
\pgfpathcurveto{\pgfqpoint{1.146464in}{0.702777in}}{\pgfqpoint{1.154364in}{0.699504in}}{\pgfqpoint{1.162600in}{0.699504in}}%
\pgfpathclose%
\pgfusepath{stroke,fill}%
\end{pgfscope}%
\begin{pgfscope}%
\pgfpathrectangle{\pgfqpoint{0.894063in}{0.630000in}}{\pgfqpoint{6.713438in}{2.060556in}} %
\pgfusepath{clip}%
\pgfsetbuttcap%
\pgfsetroundjoin%
\definecolor{currentfill}{rgb}{0.000000,0.000000,1.000000}%
\pgfsetfillcolor{currentfill}%
\pgfsetlinewidth{1.003750pt}%
\definecolor{currentstroke}{rgb}{0.000000,0.000000,0.000000}%
\pgfsetstrokecolor{currentstroke}%
\pgfsetdash{}{0pt}%
\pgfpathmoveto{\pgfqpoint{1.833944in}{1.275824in}}%
\pgfpathcurveto{\pgfqpoint{1.842180in}{1.275824in}}{\pgfqpoint{1.850080in}{1.279097in}}{\pgfqpoint{1.855904in}{1.284920in}}%
\pgfpathcurveto{\pgfqpoint{1.861728in}{1.290744in}}{\pgfqpoint{1.865000in}{1.298644in}}{\pgfqpoint{1.865000in}{1.306881in}}%
\pgfpathcurveto{\pgfqpoint{1.865000in}{1.315117in}}{\pgfqpoint{1.861728in}{1.323017in}}{\pgfqpoint{1.855904in}{1.328841in}}%
\pgfpathcurveto{\pgfqpoint{1.850080in}{1.334665in}}{\pgfqpoint{1.842180in}{1.337937in}}{\pgfqpoint{1.833944in}{1.337937in}}%
\pgfpathcurveto{\pgfqpoint{1.825707in}{1.337937in}}{\pgfqpoint{1.817807in}{1.334665in}}{\pgfqpoint{1.811983in}{1.328841in}}%
\pgfpathcurveto{\pgfqpoint{1.806160in}{1.323017in}}{\pgfqpoint{1.802887in}{1.315117in}}{\pgfqpoint{1.802887in}{1.306881in}}%
\pgfpathcurveto{\pgfqpoint{1.802887in}{1.298644in}}{\pgfqpoint{1.806160in}{1.290744in}}{\pgfqpoint{1.811983in}{1.284920in}}%
\pgfpathcurveto{\pgfqpoint{1.817807in}{1.279097in}}{\pgfqpoint{1.825707in}{1.275824in}}{\pgfqpoint{1.833944in}{1.275824in}}%
\pgfpathclose%
\pgfusepath{stroke,fill}%
\end{pgfscope}%
\begin{pgfscope}%
\pgfpathrectangle{\pgfqpoint{0.894063in}{0.630000in}}{\pgfqpoint{6.713438in}{2.060556in}} %
\pgfusepath{clip}%
\pgfsetbuttcap%
\pgfsetroundjoin%
\definecolor{currentfill}{rgb}{0.000000,0.000000,1.000000}%
\pgfsetfillcolor{currentfill}%
\pgfsetlinewidth{1.003750pt}%
\definecolor{currentstroke}{rgb}{0.000000,0.000000,0.000000}%
\pgfsetstrokecolor{currentstroke}%
\pgfsetdash{}{0pt}%
\pgfpathmoveto{\pgfqpoint{5.996275in}{1.980258in}}%
\pgfpathcurveto{\pgfqpoint{6.004511in}{1.980258in}}{\pgfqpoint{6.012411in}{1.983530in}}{\pgfqpoint{6.018235in}{1.989354in}}%
\pgfpathcurveto{\pgfqpoint{6.024059in}{1.995178in}}{\pgfqpoint{6.027331in}{2.003078in}}{\pgfqpoint{6.027331in}{2.011314in}}%
\pgfpathcurveto{\pgfqpoint{6.027331in}{2.019550in}}{\pgfqpoint{6.024059in}{2.027450in}}{\pgfqpoint{6.018235in}{2.033274in}}%
\pgfpathcurveto{\pgfqpoint{6.012411in}{2.039098in}}{\pgfqpoint{6.004511in}{2.042371in}}{\pgfqpoint{5.996275in}{2.042371in}}%
\pgfpathcurveto{\pgfqpoint{5.988039in}{2.042371in}}{\pgfqpoint{5.980139in}{2.039098in}}{\pgfqpoint{5.974315in}{2.033274in}}%
\pgfpathcurveto{\pgfqpoint{5.968491in}{2.027450in}}{\pgfqpoint{5.965219in}{2.019550in}}{\pgfqpoint{5.965219in}{2.011314in}}%
\pgfpathcurveto{\pgfqpoint{5.965219in}{2.003078in}}{\pgfqpoint{5.968491in}{1.995178in}}{\pgfqpoint{5.974315in}{1.989354in}}%
\pgfpathcurveto{\pgfqpoint{5.980139in}{1.983530in}}{\pgfqpoint{5.988039in}{1.980258in}}{\pgfqpoint{5.996275in}{1.980258in}}%
\pgfpathclose%
\pgfusepath{stroke,fill}%
\end{pgfscope}%
\begin{pgfscope}%
\pgfpathrectangle{\pgfqpoint{0.894063in}{0.630000in}}{\pgfqpoint{6.713438in}{2.060556in}} %
\pgfusepath{clip}%
\pgfsetbuttcap%
\pgfsetroundjoin%
\definecolor{currentfill}{rgb}{0.000000,0.000000,1.000000}%
\pgfsetfillcolor{currentfill}%
\pgfsetlinewidth{1.003750pt}%
\definecolor{currentstroke}{rgb}{0.000000,0.000000,0.000000}%
\pgfsetstrokecolor{currentstroke}%
\pgfsetdash{}{0pt}%
\pgfpathmoveto{\pgfqpoint{6.399081in}{2.043004in}}%
\pgfpathcurveto{\pgfqpoint{6.407318in}{2.043004in}}{\pgfqpoint{6.415218in}{2.046277in}}{\pgfqpoint{6.421042in}{2.052101in}}%
\pgfpathcurveto{\pgfqpoint{6.426865in}{2.057925in}}{\pgfqpoint{6.430138in}{2.065825in}}{\pgfqpoint{6.430138in}{2.074061in}}%
\pgfpathcurveto{\pgfqpoint{6.430138in}{2.082297in}}{\pgfqpoint{6.426865in}{2.090197in}}{\pgfqpoint{6.421042in}{2.096021in}}%
\pgfpathcurveto{\pgfqpoint{6.415218in}{2.101845in}}{\pgfqpoint{6.407318in}{2.105117in}}{\pgfqpoint{6.399081in}{2.105117in}}%
\pgfpathcurveto{\pgfqpoint{6.390845in}{2.105117in}}{\pgfqpoint{6.382945in}{2.101845in}}{\pgfqpoint{6.377121in}{2.096021in}}%
\pgfpathcurveto{\pgfqpoint{6.371297in}{2.090197in}}{\pgfqpoint{6.368025in}{2.082297in}}{\pgfqpoint{6.368025in}{2.074061in}}%
\pgfpathcurveto{\pgfqpoint{6.368025in}{2.065825in}}{\pgfqpoint{6.371297in}{2.057925in}}{\pgfqpoint{6.377121in}{2.052101in}}%
\pgfpathcurveto{\pgfqpoint{6.382945in}{2.046277in}}{\pgfqpoint{6.390845in}{2.043004in}}{\pgfqpoint{6.399081in}{2.043004in}}%
\pgfpathclose%
\pgfusepath{stroke,fill}%
\end{pgfscope}%
\begin{pgfscope}%
\pgfpathrectangle{\pgfqpoint{0.894063in}{0.630000in}}{\pgfqpoint{6.713438in}{2.060556in}} %
\pgfusepath{clip}%
\pgfsetbuttcap%
\pgfsetroundjoin%
\definecolor{currentfill}{rgb}{0.000000,0.000000,1.000000}%
\pgfsetfillcolor{currentfill}%
\pgfsetlinewidth{1.003750pt}%
\definecolor{currentstroke}{rgb}{0.000000,0.000000,0.000000}%
\pgfsetstrokecolor{currentstroke}%
\pgfsetdash{}{0pt}%
\pgfpathmoveto{\pgfqpoint{4.787856in}{1.771141in}}%
\pgfpathcurveto{\pgfqpoint{4.796093in}{1.771141in}}{\pgfqpoint{4.803993in}{1.774413in}}{\pgfqpoint{4.809817in}{1.780237in}}%
\pgfpathcurveto{\pgfqpoint{4.815640in}{1.786061in}}{\pgfqpoint{4.818913in}{1.793961in}}{\pgfqpoint{4.818913in}{1.802197in}}%
\pgfpathcurveto{\pgfqpoint{4.818913in}{1.810433in}}{\pgfqpoint{4.815640in}{1.818333in}}{\pgfqpoint{4.809817in}{1.824157in}}%
\pgfpathcurveto{\pgfqpoint{4.803993in}{1.829981in}}{\pgfqpoint{4.796093in}{1.833254in}}{\pgfqpoint{4.787856in}{1.833254in}}%
\pgfpathcurveto{\pgfqpoint{4.779620in}{1.833254in}}{\pgfqpoint{4.771720in}{1.829981in}}{\pgfqpoint{4.765896in}{1.824157in}}%
\pgfpathcurveto{\pgfqpoint{4.760072in}{1.818333in}}{\pgfqpoint{4.756800in}{1.810433in}}{\pgfqpoint{4.756800in}{1.802197in}}%
\pgfpathcurveto{\pgfqpoint{4.756800in}{1.793961in}}{\pgfqpoint{4.760072in}{1.786061in}}{\pgfqpoint{4.765896in}{1.780237in}}%
\pgfpathcurveto{\pgfqpoint{4.771720in}{1.774413in}}{\pgfqpoint{4.779620in}{1.771141in}}{\pgfqpoint{4.787856in}{1.771141in}}%
\pgfpathclose%
\pgfusepath{stroke,fill}%
\end{pgfscope}%
\begin{pgfscope}%
\pgfpathrectangle{\pgfqpoint{0.894063in}{0.630000in}}{\pgfqpoint{6.713438in}{2.060556in}} %
\pgfusepath{clip}%
\pgfsetbuttcap%
\pgfsetroundjoin%
\definecolor{currentfill}{rgb}{0.000000,0.000000,1.000000}%
\pgfsetfillcolor{currentfill}%
\pgfsetlinewidth{1.003750pt}%
\definecolor{currentstroke}{rgb}{0.000000,0.000000,0.000000}%
\pgfsetstrokecolor{currentstroke}%
\pgfsetdash{}{0pt}%
\pgfpathmoveto{\pgfqpoint{4.922125in}{1.803733in}}%
\pgfpathcurveto{\pgfqpoint{4.930361in}{1.803733in}}{\pgfqpoint{4.938261in}{1.807005in}}{\pgfqpoint{4.944085in}{1.812829in}}%
\pgfpathcurveto{\pgfqpoint{4.949909in}{1.818653in}}{\pgfqpoint{4.953181in}{1.826553in}}{\pgfqpoint{4.953181in}{1.834789in}}%
\pgfpathcurveto{\pgfqpoint{4.953181in}{1.843025in}}{\pgfqpoint{4.949909in}{1.850926in}}{\pgfqpoint{4.944085in}{1.856749in}}%
\pgfpathcurveto{\pgfqpoint{4.938261in}{1.862573in}}{\pgfqpoint{4.930361in}{1.865846in}}{\pgfqpoint{4.922125in}{1.865846in}}%
\pgfpathcurveto{\pgfqpoint{4.913889in}{1.865846in}}{\pgfqpoint{4.905989in}{1.862573in}}{\pgfqpoint{4.900165in}{1.856749in}}%
\pgfpathcurveto{\pgfqpoint{4.894341in}{1.850926in}}{\pgfqpoint{4.891069in}{1.843025in}}{\pgfqpoint{4.891069in}{1.834789in}}%
\pgfpathcurveto{\pgfqpoint{4.891069in}{1.826553in}}{\pgfqpoint{4.894341in}{1.818653in}}{\pgfqpoint{4.900165in}{1.812829in}}%
\pgfpathcurveto{\pgfqpoint{4.905989in}{1.807005in}}{\pgfqpoint{4.913889in}{1.803733in}}{\pgfqpoint{4.922125in}{1.803733in}}%
\pgfpathclose%
\pgfusepath{stroke,fill}%
\end{pgfscope}%
\begin{pgfscope}%
\pgfpathrectangle{\pgfqpoint{0.894063in}{0.630000in}}{\pgfqpoint{6.713438in}{2.060556in}} %
\pgfusepath{clip}%
\pgfsetbuttcap%
\pgfsetroundjoin%
\definecolor{currentfill}{rgb}{0.000000,0.000000,1.000000}%
\pgfsetfillcolor{currentfill}%
\pgfsetlinewidth{1.003750pt}%
\definecolor{currentstroke}{rgb}{0.000000,0.000000,0.000000}%
\pgfsetstrokecolor{currentstroke}%
\pgfsetdash{}{0pt}%
\pgfpathmoveto{\pgfqpoint{6.130544in}{1.998261in}}%
\pgfpathcurveto{\pgfqpoint{6.138780in}{1.998261in}}{\pgfqpoint{6.146680in}{2.001533in}}{\pgfqpoint{6.152504in}{2.007357in}}%
\pgfpathcurveto{\pgfqpoint{6.158328in}{2.013181in}}{\pgfqpoint{6.161600in}{2.021081in}}{\pgfqpoint{6.161600in}{2.029317in}}%
\pgfpathcurveto{\pgfqpoint{6.161600in}{2.037554in}}{\pgfqpoint{6.158328in}{2.045454in}}{\pgfqpoint{6.152504in}{2.051278in}}%
\pgfpathcurveto{\pgfqpoint{6.146680in}{2.057102in}}{\pgfqpoint{6.138780in}{2.060374in}}{\pgfqpoint{6.130544in}{2.060374in}}%
\pgfpathcurveto{\pgfqpoint{6.122307in}{2.060374in}}{\pgfqpoint{6.114407in}{2.057102in}}{\pgfqpoint{6.108583in}{2.051278in}}%
\pgfpathcurveto{\pgfqpoint{6.102760in}{2.045454in}}{\pgfqpoint{6.099487in}{2.037554in}}{\pgfqpoint{6.099487in}{2.029317in}}%
\pgfpathcurveto{\pgfqpoint{6.099487in}{2.021081in}}{\pgfqpoint{6.102760in}{2.013181in}}{\pgfqpoint{6.108583in}{2.007357in}}%
\pgfpathcurveto{\pgfqpoint{6.114407in}{2.001533in}}{\pgfqpoint{6.122307in}{1.998261in}}{\pgfqpoint{6.130544in}{1.998261in}}%
\pgfpathclose%
\pgfusepath{stroke,fill}%
\end{pgfscope}%
\begin{pgfscope}%
\pgfpathrectangle{\pgfqpoint{0.894063in}{0.630000in}}{\pgfqpoint{6.713438in}{2.060556in}} %
\pgfusepath{clip}%
\pgfsetbuttcap%
\pgfsetroundjoin%
\definecolor{currentfill}{rgb}{0.000000,0.000000,1.000000}%
\pgfsetfillcolor{currentfill}%
\pgfsetlinewidth{1.003750pt}%
\definecolor{currentstroke}{rgb}{0.000000,0.000000,0.000000}%
\pgfsetstrokecolor{currentstroke}%
\pgfsetdash{}{0pt}%
\pgfpathmoveto{\pgfqpoint{5.727738in}{1.938905in}}%
\pgfpathcurveto{\pgfqpoint{5.735974in}{1.938905in}}{\pgfqpoint{5.743874in}{1.942177in}}{\pgfqpoint{5.749698in}{1.948001in}}%
\pgfpathcurveto{\pgfqpoint{5.755522in}{1.953825in}}{\pgfqpoint{5.758794in}{1.961725in}}{\pgfqpoint{5.758794in}{1.969962in}}%
\pgfpathcurveto{\pgfqpoint{5.758794in}{1.978198in}}{\pgfqpoint{5.755522in}{1.986098in}}{\pgfqpoint{5.749698in}{1.991922in}}%
\pgfpathcurveto{\pgfqpoint{5.743874in}{1.997746in}}{\pgfqpoint{5.735974in}{2.001018in}}{\pgfqpoint{5.727738in}{2.001018in}}%
\pgfpathcurveto{\pgfqpoint{5.719501in}{2.001018in}}{\pgfqpoint{5.711601in}{1.997746in}}{\pgfqpoint{5.705777in}{1.991922in}}%
\pgfpathcurveto{\pgfqpoint{5.699953in}{1.986098in}}{\pgfqpoint{5.696681in}{1.978198in}}{\pgfqpoint{5.696681in}{1.969962in}}%
\pgfpathcurveto{\pgfqpoint{5.696681in}{1.961725in}}{\pgfqpoint{5.699953in}{1.953825in}}{\pgfqpoint{5.705777in}{1.948001in}}%
\pgfpathcurveto{\pgfqpoint{5.711601in}{1.942177in}}{\pgfqpoint{5.719501in}{1.938905in}}{\pgfqpoint{5.727738in}{1.938905in}}%
\pgfpathclose%
\pgfusepath{stroke,fill}%
\end{pgfscope}%
\begin{pgfscope}%
\pgfpathrectangle{\pgfqpoint{0.894063in}{0.630000in}}{\pgfqpoint{6.713438in}{2.060556in}} %
\pgfusepath{clip}%
\pgfsetbuttcap%
\pgfsetroundjoin%
\definecolor{currentfill}{rgb}{0.000000,0.000000,1.000000}%
\pgfsetfillcolor{currentfill}%
\pgfsetlinewidth{1.003750pt}%
\definecolor{currentstroke}{rgb}{0.000000,0.000000,0.000000}%
\pgfsetstrokecolor{currentstroke}%
\pgfsetdash{}{0pt}%
\pgfpathmoveto{\pgfqpoint{1.028331in}{0.672240in}}%
\pgfpathcurveto{\pgfqpoint{1.036568in}{0.672240in}}{\pgfqpoint{1.044468in}{0.675513in}}{\pgfqpoint{1.050292in}{0.681337in}}%
\pgfpathcurveto{\pgfqpoint{1.056115in}{0.687161in}}{\pgfqpoint{1.059388in}{0.695061in}}{\pgfqpoint{1.059388in}{0.703297in}}%
\pgfpathcurveto{\pgfqpoint{1.059388in}{0.711533in}}{\pgfqpoint{1.056115in}{0.719433in}}{\pgfqpoint{1.050292in}{0.725257in}}%
\pgfpathcurveto{\pgfqpoint{1.044468in}{0.731081in}}{\pgfqpoint{1.036568in}{0.734353in}}{\pgfqpoint{1.028331in}{0.734353in}}%
\pgfpathcurveto{\pgfqpoint{1.020095in}{0.734353in}}{\pgfqpoint{1.012195in}{0.731081in}}{\pgfqpoint{1.006371in}{0.725257in}}%
\pgfpathcurveto{\pgfqpoint{1.000547in}{0.719433in}}{\pgfqpoint{0.997275in}{0.711533in}}{\pgfqpoint{0.997275in}{0.703297in}}%
\pgfpathcurveto{\pgfqpoint{0.997275in}{0.695061in}}{\pgfqpoint{1.000547in}{0.687161in}}{\pgfqpoint{1.006371in}{0.681337in}}%
\pgfpathcurveto{\pgfqpoint{1.012195in}{0.675513in}}{\pgfqpoint{1.020095in}{0.672240in}}{\pgfqpoint{1.028331in}{0.672240in}}%
\pgfpathclose%
\pgfusepath{stroke,fill}%
\end{pgfscope}%
\begin{pgfscope}%
\pgfpathrectangle{\pgfqpoint{0.894063in}{0.630000in}}{\pgfqpoint{6.713438in}{2.060556in}} %
\pgfusepath{clip}%
\pgfsetbuttcap%
\pgfsetroundjoin%
\definecolor{currentfill}{rgb}{0.000000,0.000000,1.000000}%
\pgfsetfillcolor{currentfill}%
\pgfsetlinewidth{1.003750pt}%
\definecolor{currentstroke}{rgb}{0.000000,0.000000,0.000000}%
\pgfsetstrokecolor{currentstroke}%
\pgfsetdash{}{0pt}%
\pgfpathmoveto{\pgfqpoint{5.324931in}{1.867898in}}%
\pgfpathcurveto{\pgfqpoint{5.333168in}{1.867898in}}{\pgfqpoint{5.341068in}{1.871171in}}{\pgfqpoint{5.346892in}{1.876995in}}%
\pgfpathcurveto{\pgfqpoint{5.352715in}{1.882819in}}{\pgfqpoint{5.355988in}{1.890719in}}{\pgfqpoint{5.355988in}{1.898955in}}%
\pgfpathcurveto{\pgfqpoint{5.355988in}{1.907191in}}{\pgfqpoint{5.352715in}{1.915091in}}{\pgfqpoint{5.346892in}{1.920915in}}%
\pgfpathcurveto{\pgfqpoint{5.341068in}{1.926739in}}{\pgfqpoint{5.333168in}{1.930011in}}{\pgfqpoint{5.324931in}{1.930011in}}%
\pgfpathcurveto{\pgfqpoint{5.316695in}{1.930011in}}{\pgfqpoint{5.308795in}{1.926739in}}{\pgfqpoint{5.302971in}{1.920915in}}%
\pgfpathcurveto{\pgfqpoint{5.297147in}{1.915091in}}{\pgfqpoint{5.293875in}{1.907191in}}{\pgfqpoint{5.293875in}{1.898955in}}%
\pgfpathcurveto{\pgfqpoint{5.293875in}{1.890719in}}{\pgfqpoint{5.297147in}{1.882819in}}{\pgfqpoint{5.302971in}{1.876995in}}%
\pgfpathcurveto{\pgfqpoint{5.308795in}{1.871171in}}{\pgfqpoint{5.316695in}{1.867898in}}{\pgfqpoint{5.324931in}{1.867898in}}%
\pgfpathclose%
\pgfusepath{stroke,fill}%
\end{pgfscope}%
\begin{pgfscope}%
\pgfpathrectangle{\pgfqpoint{0.894063in}{0.630000in}}{\pgfqpoint{6.713438in}{2.060556in}} %
\pgfusepath{clip}%
\pgfsetbuttcap%
\pgfsetroundjoin%
\definecolor{currentfill}{rgb}{0.000000,0.000000,1.000000}%
\pgfsetfillcolor{currentfill}%
\pgfsetlinewidth{1.003750pt}%
\definecolor{currentstroke}{rgb}{0.000000,0.000000,0.000000}%
\pgfsetstrokecolor{currentstroke}%
\pgfsetdash{}{0pt}%
\pgfpathmoveto{\pgfqpoint{7.338963in}{2.221872in}}%
\pgfpathcurveto{\pgfqpoint{7.347199in}{2.221872in}}{\pgfqpoint{7.355099in}{2.225145in}}{\pgfqpoint{7.360923in}{2.230969in}}%
\pgfpathcurveto{\pgfqpoint{7.366747in}{2.236793in}}{\pgfqpoint{7.370019in}{2.244693in}}{\pgfqpoint{7.370019in}{2.252929in}}%
\pgfpathcurveto{\pgfqpoint{7.370019in}{2.261165in}}{\pgfqpoint{7.366747in}{2.269065in}}{\pgfqpoint{7.360923in}{2.274889in}}%
\pgfpathcurveto{\pgfqpoint{7.355099in}{2.280713in}}{\pgfqpoint{7.347199in}{2.283985in}}{\pgfqpoint{7.338963in}{2.283985in}}%
\pgfpathcurveto{\pgfqpoint{7.330726in}{2.283985in}}{\pgfqpoint{7.322826in}{2.280713in}}{\pgfqpoint{7.317002in}{2.274889in}}%
\pgfpathcurveto{\pgfqpoint{7.311178in}{2.269065in}}{\pgfqpoint{7.307906in}{2.261165in}}{\pgfqpoint{7.307906in}{2.252929in}}%
\pgfpathcurveto{\pgfqpoint{7.307906in}{2.244693in}}{\pgfqpoint{7.311178in}{2.236793in}}{\pgfqpoint{7.317002in}{2.230969in}}%
\pgfpathcurveto{\pgfqpoint{7.322826in}{2.225145in}}{\pgfqpoint{7.330726in}{2.221872in}}{\pgfqpoint{7.338963in}{2.221872in}}%
\pgfpathclose%
\pgfusepath{stroke,fill}%
\end{pgfscope}%
\begin{pgfscope}%
\pgfpathrectangle{\pgfqpoint{0.894063in}{0.630000in}}{\pgfqpoint{6.713438in}{2.060556in}} %
\pgfusepath{clip}%
\pgfsetbuttcap%
\pgfsetroundjoin%
\definecolor{currentfill}{rgb}{0.000000,0.000000,1.000000}%
\pgfsetfillcolor{currentfill}%
\pgfsetlinewidth{1.003750pt}%
\definecolor{currentstroke}{rgb}{0.000000,0.000000,0.000000}%
\pgfsetstrokecolor{currentstroke}%
\pgfsetdash{}{0pt}%
\pgfpathmoveto{\pgfqpoint{7.204694in}{2.203928in}}%
\pgfpathcurveto{\pgfqpoint{7.212930in}{2.203928in}}{\pgfqpoint{7.220830in}{2.207200in}}{\pgfqpoint{7.226654in}{2.213024in}}%
\pgfpathcurveto{\pgfqpoint{7.232478in}{2.218848in}}{\pgfqpoint{7.235750in}{2.226748in}}{\pgfqpoint{7.235750in}{2.234984in}}%
\pgfpathcurveto{\pgfqpoint{7.235750in}{2.243221in}}{\pgfqpoint{7.232478in}{2.251121in}}{\pgfqpoint{7.226654in}{2.256945in}}%
\pgfpathcurveto{\pgfqpoint{7.220830in}{2.262769in}}{\pgfqpoint{7.212930in}{2.266041in}}{\pgfqpoint{7.204694in}{2.266041in}}%
\pgfpathcurveto{\pgfqpoint{7.196457in}{2.266041in}}{\pgfqpoint{7.188557in}{2.262769in}}{\pgfqpoint{7.182733in}{2.256945in}}%
\pgfpathcurveto{\pgfqpoint{7.176910in}{2.251121in}}{\pgfqpoint{7.173637in}{2.243221in}}{\pgfqpoint{7.173637in}{2.234984in}}%
\pgfpathcurveto{\pgfqpoint{7.173637in}{2.226748in}}{\pgfqpoint{7.176910in}{2.218848in}}{\pgfqpoint{7.182733in}{2.213024in}}%
\pgfpathcurveto{\pgfqpoint{7.188557in}{2.207200in}}{\pgfqpoint{7.196457in}{2.203928in}}{\pgfqpoint{7.204694in}{2.203928in}}%
\pgfpathclose%
\pgfusepath{stroke,fill}%
\end{pgfscope}%
\begin{pgfscope}%
\pgfpathrectangle{\pgfqpoint{0.894063in}{0.630000in}}{\pgfqpoint{6.713438in}{2.060556in}} %
\pgfusepath{clip}%
\pgfsetbuttcap%
\pgfsetroundjoin%
\definecolor{currentfill}{rgb}{0.000000,0.000000,1.000000}%
\pgfsetfillcolor{currentfill}%
\pgfsetlinewidth{1.003750pt}%
\definecolor{currentstroke}{rgb}{0.000000,0.000000,0.000000}%
\pgfsetstrokecolor{currentstroke}%
\pgfsetdash{}{0pt}%
\pgfpathmoveto{\pgfqpoint{6.264813in}{2.022152in}}%
\pgfpathcurveto{\pgfqpoint{6.273049in}{2.022152in}}{\pgfqpoint{6.280949in}{2.025424in}}{\pgfqpoint{6.286773in}{2.031248in}}%
\pgfpathcurveto{\pgfqpoint{6.292597in}{2.037072in}}{\pgfqpoint{6.295869in}{2.044972in}}{\pgfqpoint{6.295869in}{2.053208in}}%
\pgfpathcurveto{\pgfqpoint{6.295869in}{2.061444in}}{\pgfqpoint{6.292597in}{2.069344in}}{\pgfqpoint{6.286773in}{2.075168in}}%
\pgfpathcurveto{\pgfqpoint{6.280949in}{2.080992in}}{\pgfqpoint{6.273049in}{2.084265in}}{\pgfqpoint{6.264813in}{2.084265in}}%
\pgfpathcurveto{\pgfqpoint{6.256576in}{2.084265in}}{\pgfqpoint{6.248676in}{2.080992in}}{\pgfqpoint{6.242852in}{2.075168in}}%
\pgfpathcurveto{\pgfqpoint{6.237028in}{2.069344in}}{\pgfqpoint{6.233756in}{2.061444in}}{\pgfqpoint{6.233756in}{2.053208in}}%
\pgfpathcurveto{\pgfqpoint{6.233756in}{2.044972in}}{\pgfqpoint{6.237028in}{2.037072in}}{\pgfqpoint{6.242852in}{2.031248in}}%
\pgfpathcurveto{\pgfqpoint{6.248676in}{2.025424in}}{\pgfqpoint{6.256576in}{2.022152in}}{\pgfqpoint{6.264813in}{2.022152in}}%
\pgfpathclose%
\pgfusepath{stroke,fill}%
\end{pgfscope}%
\begin{pgfscope}%
\pgfpathrectangle{\pgfqpoint{0.894063in}{0.630000in}}{\pgfqpoint{6.713438in}{2.060556in}} %
\pgfusepath{clip}%
\pgfsetbuttcap%
\pgfsetroundjoin%
\definecolor{currentfill}{rgb}{0.000000,0.000000,1.000000}%
\pgfsetfillcolor{currentfill}%
\pgfsetlinewidth{1.003750pt}%
\definecolor{currentstroke}{rgb}{0.000000,0.000000,0.000000}%
\pgfsetstrokecolor{currentstroke}%
\pgfsetdash{}{0pt}%
\pgfpathmoveto{\pgfqpoint{7.473231in}{2.246623in}}%
\pgfpathcurveto{\pgfqpoint{7.481468in}{2.246623in}}{\pgfqpoint{7.489368in}{2.249895in}}{\pgfqpoint{7.495192in}{2.255719in}}%
\pgfpathcurveto{\pgfqpoint{7.501015in}{2.261543in}}{\pgfqpoint{7.504288in}{2.269443in}}{\pgfqpoint{7.504288in}{2.277679in}}%
\pgfpathcurveto{\pgfqpoint{7.504288in}{2.285915in}}{\pgfqpoint{7.501015in}{2.293815in}}{\pgfqpoint{7.495192in}{2.299639in}}%
\pgfpathcurveto{\pgfqpoint{7.489368in}{2.305463in}}{\pgfqpoint{7.481468in}{2.308736in}}{\pgfqpoint{7.473231in}{2.308736in}}%
\pgfpathcurveto{\pgfqpoint{7.464995in}{2.308736in}}{\pgfqpoint{7.457095in}{2.305463in}}{\pgfqpoint{7.451271in}{2.299639in}}%
\pgfpathcurveto{\pgfqpoint{7.445447in}{2.293815in}}{\pgfqpoint{7.442175in}{2.285915in}}{\pgfqpoint{7.442175in}{2.277679in}}%
\pgfpathcurveto{\pgfqpoint{7.442175in}{2.269443in}}{\pgfqpoint{7.445447in}{2.261543in}}{\pgfqpoint{7.451271in}{2.255719in}}%
\pgfpathcurveto{\pgfqpoint{7.457095in}{2.249895in}}{\pgfqpoint{7.464995in}{2.246623in}}{\pgfqpoint{7.473231in}{2.246623in}}%
\pgfpathclose%
\pgfusepath{stroke,fill}%
\end{pgfscope}%
\begin{pgfscope}%
\pgfpathrectangle{\pgfqpoint{0.894063in}{0.630000in}}{\pgfqpoint{6.713438in}{2.060556in}} %
\pgfusepath{clip}%
\pgfsetbuttcap%
\pgfsetroundjoin%
\definecolor{currentfill}{rgb}{0.000000,0.000000,1.000000}%
\pgfsetfillcolor{currentfill}%
\pgfsetlinewidth{1.003750pt}%
\definecolor{currentstroke}{rgb}{0.000000,0.000000,0.000000}%
\pgfsetstrokecolor{currentstroke}%
\pgfsetdash{}{0pt}%
\pgfpathmoveto{\pgfqpoint{5.056394in}{1.820935in}}%
\pgfpathcurveto{\pgfqpoint{5.064630in}{1.820935in}}{\pgfqpoint{5.072530in}{1.824208in}}{\pgfqpoint{5.078354in}{1.830032in}}%
\pgfpathcurveto{\pgfqpoint{5.084178in}{1.835856in}}{\pgfqpoint{5.087450in}{1.843756in}}{\pgfqpoint{5.087450in}{1.851992in}}%
\pgfpathcurveto{\pgfqpoint{5.087450in}{1.860228in}}{\pgfqpoint{5.084178in}{1.868128in}}{\pgfqpoint{5.078354in}{1.873952in}}%
\pgfpathcurveto{\pgfqpoint{5.072530in}{1.879776in}}{\pgfqpoint{5.064630in}{1.883048in}}{\pgfqpoint{5.056394in}{1.883048in}}%
\pgfpathcurveto{\pgfqpoint{5.048157in}{1.883048in}}{\pgfqpoint{5.040257in}{1.879776in}}{\pgfqpoint{5.034433in}{1.873952in}}%
\pgfpathcurveto{\pgfqpoint{5.028610in}{1.868128in}}{\pgfqpoint{5.025337in}{1.860228in}}{\pgfqpoint{5.025337in}{1.851992in}}%
\pgfpathcurveto{\pgfqpoint{5.025337in}{1.843756in}}{\pgfqpoint{5.028610in}{1.835856in}}{\pgfqpoint{5.034433in}{1.830032in}}%
\pgfpathcurveto{\pgfqpoint{5.040257in}{1.824208in}}{\pgfqpoint{5.048157in}{1.820935in}}{\pgfqpoint{5.056394in}{1.820935in}}%
\pgfpathclose%
\pgfusepath{stroke,fill}%
\end{pgfscope}%
\begin{pgfscope}%
\pgfpathrectangle{\pgfqpoint{0.894063in}{0.630000in}}{\pgfqpoint{6.713438in}{2.060556in}} %
\pgfusepath{clip}%
\pgfsetbuttcap%
\pgfsetroundjoin%
\definecolor{currentfill}{rgb}{0.000000,0.000000,1.000000}%
\pgfsetfillcolor{currentfill}%
\pgfsetlinewidth{1.003750pt}%
\definecolor{currentstroke}{rgb}{0.000000,0.000000,0.000000}%
\pgfsetstrokecolor{currentstroke}%
\pgfsetdash{}{0pt}%
\pgfpathmoveto{\pgfqpoint{2.908094in}{1.443960in}}%
\pgfpathcurveto{\pgfqpoint{2.916330in}{1.443960in}}{\pgfqpoint{2.924230in}{1.447232in}}{\pgfqpoint{2.930054in}{1.453056in}}%
\pgfpathcurveto{\pgfqpoint{2.935878in}{1.458880in}}{\pgfqpoint{2.939150in}{1.466780in}}{\pgfqpoint{2.939150in}{1.475016in}}%
\pgfpathcurveto{\pgfqpoint{2.939150in}{1.483252in}}{\pgfqpoint{2.935878in}{1.491153in}}{\pgfqpoint{2.930054in}{1.496976in}}%
\pgfpathcurveto{\pgfqpoint{2.924230in}{1.502800in}}{\pgfqpoint{2.916330in}{1.506073in}}{\pgfqpoint{2.908094in}{1.506073in}}%
\pgfpathcurveto{\pgfqpoint{2.899857in}{1.506073in}}{\pgfqpoint{2.891957in}{1.502800in}}{\pgfqpoint{2.886133in}{1.496976in}}%
\pgfpathcurveto{\pgfqpoint{2.880310in}{1.491153in}}{\pgfqpoint{2.877037in}{1.483252in}}{\pgfqpoint{2.877037in}{1.475016in}}%
\pgfpathcurveto{\pgfqpoint{2.877037in}{1.466780in}}{\pgfqpoint{2.880310in}{1.458880in}}{\pgfqpoint{2.886133in}{1.453056in}}%
\pgfpathcurveto{\pgfqpoint{2.891957in}{1.447232in}}{\pgfqpoint{2.899857in}{1.443960in}}{\pgfqpoint{2.908094in}{1.443960in}}%
\pgfpathclose%
\pgfusepath{stroke,fill}%
\end{pgfscope}%
\begin{pgfscope}%
\pgfpathrectangle{\pgfqpoint{0.894063in}{0.630000in}}{\pgfqpoint{6.713438in}{2.060556in}} %
\pgfusepath{clip}%
\pgfsetbuttcap%
\pgfsetroundjoin%
\definecolor{currentfill}{rgb}{0.000000,0.000000,1.000000}%
\pgfsetfillcolor{currentfill}%
\pgfsetlinewidth{1.003750pt}%
\definecolor{currentstroke}{rgb}{0.000000,0.000000,0.000000}%
\pgfsetstrokecolor{currentstroke}%
\pgfsetdash{}{0pt}%
\pgfpathmoveto{\pgfqpoint{3.445169in}{1.540152in}}%
\pgfpathcurveto{\pgfqpoint{3.453405in}{1.540152in}}{\pgfqpoint{3.461305in}{1.543425in}}{\pgfqpoint{3.467129in}{1.549249in}}%
\pgfpathcurveto{\pgfqpoint{3.472953in}{1.555072in}}{\pgfqpoint{3.476225in}{1.562973in}}{\pgfqpoint{3.476225in}{1.571209in}}%
\pgfpathcurveto{\pgfqpoint{3.476225in}{1.579445in}}{\pgfqpoint{3.472953in}{1.587345in}}{\pgfqpoint{3.467129in}{1.593169in}}%
\pgfpathcurveto{\pgfqpoint{3.461305in}{1.598993in}}{\pgfqpoint{3.453405in}{1.602265in}}{\pgfqpoint{3.445169in}{1.602265in}}%
\pgfpathcurveto{\pgfqpoint{3.436932in}{1.602265in}}{\pgfqpoint{3.429032in}{1.598993in}}{\pgfqpoint{3.423208in}{1.593169in}}%
\pgfpathcurveto{\pgfqpoint{3.417385in}{1.587345in}}{\pgfqpoint{3.414112in}{1.579445in}}{\pgfqpoint{3.414112in}{1.571209in}}%
\pgfpathcurveto{\pgfqpoint{3.414112in}{1.562973in}}{\pgfqpoint{3.417385in}{1.555072in}}{\pgfqpoint{3.423208in}{1.549249in}}%
\pgfpathcurveto{\pgfqpoint{3.429032in}{1.543425in}}{\pgfqpoint{3.436932in}{1.540152in}}{\pgfqpoint{3.445169in}{1.540152in}}%
\pgfpathclose%
\pgfusepath{stroke,fill}%
\end{pgfscope}%
\begin{pgfscope}%
\pgfpathrectangle{\pgfqpoint{0.894063in}{0.630000in}}{\pgfqpoint{6.713438in}{2.060556in}} %
\pgfusepath{clip}%
\pgfsetbuttcap%
\pgfsetroundjoin%
\definecolor{currentfill}{rgb}{0.000000,0.000000,1.000000}%
\pgfsetfillcolor{currentfill}%
\pgfsetlinewidth{1.003750pt}%
\definecolor{currentstroke}{rgb}{0.000000,0.000000,0.000000}%
\pgfsetstrokecolor{currentstroke}%
\pgfsetdash{}{0pt}%
\pgfpathmoveto{\pgfqpoint{4.116513in}{1.652511in}}%
\pgfpathcurveto{\pgfqpoint{4.124749in}{1.652511in}}{\pgfqpoint{4.132649in}{1.655784in}}{\pgfqpoint{4.138473in}{1.661608in}}%
\pgfpathcurveto{\pgfqpoint{4.144297in}{1.667432in}}{\pgfqpoint{4.147569in}{1.675332in}}{\pgfqpoint{4.147569in}{1.683568in}}%
\pgfpathcurveto{\pgfqpoint{4.147569in}{1.691804in}}{\pgfqpoint{4.144297in}{1.699704in}}{\pgfqpoint{4.138473in}{1.705528in}}%
\pgfpathcurveto{\pgfqpoint{4.132649in}{1.711352in}}{\pgfqpoint{4.124749in}{1.714624in}}{\pgfqpoint{4.116513in}{1.714624in}}%
\pgfpathcurveto{\pgfqpoint{4.108276in}{1.714624in}}{\pgfqpoint{4.100376in}{1.711352in}}{\pgfqpoint{4.094552in}{1.705528in}}%
\pgfpathcurveto{\pgfqpoint{4.088728in}{1.699704in}}{\pgfqpoint{4.085456in}{1.691804in}}{\pgfqpoint{4.085456in}{1.683568in}}%
\pgfpathcurveto{\pgfqpoint{4.085456in}{1.675332in}}{\pgfqpoint{4.088728in}{1.667432in}}{\pgfqpoint{4.094552in}{1.661608in}}%
\pgfpathcurveto{\pgfqpoint{4.100376in}{1.655784in}}{\pgfqpoint{4.108276in}{1.652511in}}{\pgfqpoint{4.116513in}{1.652511in}}%
\pgfpathclose%
\pgfusepath{stroke,fill}%
\end{pgfscope}%
\begin{pgfscope}%
\pgfpathrectangle{\pgfqpoint{0.894063in}{0.630000in}}{\pgfqpoint{6.713438in}{2.060556in}} %
\pgfusepath{clip}%
\pgfsetbuttcap%
\pgfsetroundjoin%
\definecolor{currentfill}{rgb}{0.000000,0.000000,1.000000}%
\pgfsetfillcolor{currentfill}%
\pgfsetlinewidth{1.003750pt}%
\definecolor{currentstroke}{rgb}{0.000000,0.000000,0.000000}%
\pgfsetstrokecolor{currentstroke}%
\pgfsetdash{}{0pt}%
\pgfpathmoveto{\pgfqpoint{1.431138in}{1.251139in}}%
\pgfpathcurveto{\pgfqpoint{1.439374in}{1.251139in}}{\pgfqpoint{1.447274in}{1.254411in}}{\pgfqpoint{1.453098in}{1.260235in}}%
\pgfpathcurveto{\pgfqpoint{1.458922in}{1.266059in}}{\pgfqpoint{1.462194in}{1.273959in}}{\pgfqpoint{1.462194in}{1.282195in}}%
\pgfpathcurveto{\pgfqpoint{1.462194in}{1.290432in}}{\pgfqpoint{1.458922in}{1.298332in}}{\pgfqpoint{1.453098in}{1.304156in}}%
\pgfpathcurveto{\pgfqpoint{1.447274in}{1.309979in}}{\pgfqpoint{1.439374in}{1.313252in}}{\pgfqpoint{1.431138in}{1.313252in}}%
\pgfpathcurveto{\pgfqpoint{1.422901in}{1.313252in}}{\pgfqpoint{1.415001in}{1.309979in}}{\pgfqpoint{1.409177in}{1.304156in}}%
\pgfpathcurveto{\pgfqpoint{1.403353in}{1.298332in}}{\pgfqpoint{1.400081in}{1.290432in}}{\pgfqpoint{1.400081in}{1.282195in}}%
\pgfpathcurveto{\pgfqpoint{1.400081in}{1.273959in}}{\pgfqpoint{1.403353in}{1.266059in}}{\pgfqpoint{1.409177in}{1.260235in}}%
\pgfpathcurveto{\pgfqpoint{1.415001in}{1.254411in}}{\pgfqpoint{1.422901in}{1.251139in}}{\pgfqpoint{1.431138in}{1.251139in}}%
\pgfpathclose%
\pgfusepath{stroke,fill}%
\end{pgfscope}%
\begin{pgfscope}%
\pgfpathrectangle{\pgfqpoint{0.894063in}{0.630000in}}{\pgfqpoint{6.713438in}{2.060556in}} %
\pgfusepath{clip}%
\pgfsetbuttcap%
\pgfsetroundjoin%
\definecolor{currentfill}{rgb}{0.000000,0.000000,1.000000}%
\pgfsetfillcolor{currentfill}%
\pgfsetlinewidth{1.003750pt}%
\definecolor{currentstroke}{rgb}{0.000000,0.000000,0.000000}%
\pgfsetstrokecolor{currentstroke}%
\pgfsetdash{}{0pt}%
\pgfpathmoveto{\pgfqpoint{2.773825in}{1.423189in}}%
\pgfpathcurveto{\pgfqpoint{2.782061in}{1.423189in}}{\pgfqpoint{2.789961in}{1.426462in}}{\pgfqpoint{2.795785in}{1.432286in}}%
\pgfpathcurveto{\pgfqpoint{2.801609in}{1.438109in}}{\pgfqpoint{2.804881in}{1.446009in}}{\pgfqpoint{2.804881in}{1.454246in}}%
\pgfpathcurveto{\pgfqpoint{2.804881in}{1.462482in}}{\pgfqpoint{2.801609in}{1.470382in}}{\pgfqpoint{2.795785in}{1.476206in}}%
\pgfpathcurveto{\pgfqpoint{2.789961in}{1.482030in}}{\pgfqpoint{2.782061in}{1.485302in}}{\pgfqpoint{2.773825in}{1.485302in}}%
\pgfpathcurveto{\pgfqpoint{2.765589in}{1.485302in}}{\pgfqpoint{2.757689in}{1.482030in}}{\pgfqpoint{2.751865in}{1.476206in}}%
\pgfpathcurveto{\pgfqpoint{2.746041in}{1.470382in}}{\pgfqpoint{2.742769in}{1.462482in}}{\pgfqpoint{2.742769in}{1.454246in}}%
\pgfpathcurveto{\pgfqpoint{2.742769in}{1.446009in}}{\pgfqpoint{2.746041in}{1.438109in}}{\pgfqpoint{2.751865in}{1.432286in}}%
\pgfpathcurveto{\pgfqpoint{2.757689in}{1.426462in}}{\pgfqpoint{2.765589in}{1.423189in}}{\pgfqpoint{2.773825in}{1.423189in}}%
\pgfpathclose%
\pgfusepath{stroke,fill}%
\end{pgfscope}%
\begin{pgfscope}%
\pgfpathrectangle{\pgfqpoint{0.894063in}{0.630000in}}{\pgfqpoint{6.713438in}{2.060556in}} %
\pgfusepath{clip}%
\pgfsetbuttcap%
\pgfsetroundjoin%
\definecolor{currentfill}{rgb}{0.000000,0.000000,1.000000}%
\pgfsetfillcolor{currentfill}%
\pgfsetlinewidth{1.003750pt}%
\definecolor{currentstroke}{rgb}{0.000000,0.000000,0.000000}%
\pgfsetstrokecolor{currentstroke}%
\pgfsetdash{}{0pt}%
\pgfpathmoveto{\pgfqpoint{1.565406in}{1.260841in}}%
\pgfpathcurveto{\pgfqpoint{1.573643in}{1.260841in}}{\pgfqpoint{1.581543in}{1.264113in}}{\pgfqpoint{1.587367in}{1.269937in}}%
\pgfpathcurveto{\pgfqpoint{1.593190in}{1.275761in}}{\pgfqpoint{1.596463in}{1.283661in}}{\pgfqpoint{1.596463in}{1.291898in}}%
\pgfpathcurveto{\pgfqpoint{1.596463in}{1.300134in}}{\pgfqpoint{1.593190in}{1.308034in}}{\pgfqpoint{1.587367in}{1.313858in}}%
\pgfpathcurveto{\pgfqpoint{1.581543in}{1.319682in}}{\pgfqpoint{1.573643in}{1.322954in}}{\pgfqpoint{1.565406in}{1.322954in}}%
\pgfpathcurveto{\pgfqpoint{1.557170in}{1.322954in}}{\pgfqpoint{1.549270in}{1.319682in}}{\pgfqpoint{1.543446in}{1.313858in}}%
\pgfpathcurveto{\pgfqpoint{1.537622in}{1.308034in}}{\pgfqpoint{1.534350in}{1.300134in}}{\pgfqpoint{1.534350in}{1.291898in}}%
\pgfpathcurveto{\pgfqpoint{1.534350in}{1.283661in}}{\pgfqpoint{1.537622in}{1.275761in}}{\pgfqpoint{1.543446in}{1.269937in}}%
\pgfpathcurveto{\pgfqpoint{1.549270in}{1.264113in}}{\pgfqpoint{1.557170in}{1.260841in}}{\pgfqpoint{1.565406in}{1.260841in}}%
\pgfpathclose%
\pgfusepath{stroke,fill}%
\end{pgfscope}%
\begin{pgfscope}%
\pgfpathrectangle{\pgfqpoint{0.894063in}{0.630000in}}{\pgfqpoint{6.713438in}{2.060556in}} %
\pgfusepath{clip}%
\pgfsetbuttcap%
\pgfsetroundjoin%
\definecolor{currentfill}{rgb}{0.000000,0.000000,1.000000}%
\pgfsetfillcolor{currentfill}%
\pgfsetlinewidth{1.003750pt}%
\definecolor{currentstroke}{rgb}{0.000000,0.000000,0.000000}%
\pgfsetstrokecolor{currentstroke}%
\pgfsetdash{}{0pt}%
\pgfpathmoveto{\pgfqpoint{4.250781in}{1.670532in}}%
\pgfpathcurveto{\pgfqpoint{4.259018in}{1.670532in}}{\pgfqpoint{4.266918in}{1.673805in}}{\pgfqpoint{4.272742in}{1.679629in}}%
\pgfpathcurveto{\pgfqpoint{4.278565in}{1.685453in}}{\pgfqpoint{4.281838in}{1.693353in}}{\pgfqpoint{4.281838in}{1.701589in}}%
\pgfpathcurveto{\pgfqpoint{4.281838in}{1.709825in}}{\pgfqpoint{4.278565in}{1.717725in}}{\pgfqpoint{4.272742in}{1.723549in}}%
\pgfpathcurveto{\pgfqpoint{4.266918in}{1.729373in}}{\pgfqpoint{4.259018in}{1.732645in}}{\pgfqpoint{4.250781in}{1.732645in}}%
\pgfpathcurveto{\pgfqpoint{4.242545in}{1.732645in}}{\pgfqpoint{4.234645in}{1.729373in}}{\pgfqpoint{4.228821in}{1.723549in}}%
\pgfpathcurveto{\pgfqpoint{4.222997in}{1.717725in}}{\pgfqpoint{4.219725in}{1.709825in}}{\pgfqpoint{4.219725in}{1.701589in}}%
\pgfpathcurveto{\pgfqpoint{4.219725in}{1.693353in}}{\pgfqpoint{4.222997in}{1.685453in}}{\pgfqpoint{4.228821in}{1.679629in}}%
\pgfpathcurveto{\pgfqpoint{4.234645in}{1.673805in}}{\pgfqpoint{4.242545in}{1.670532in}}{\pgfqpoint{4.250781in}{1.670532in}}%
\pgfpathclose%
\pgfusepath{stroke,fill}%
\end{pgfscope}%
\begin{pgfscope}%
\pgfpathrectangle{\pgfqpoint{0.894063in}{0.630000in}}{\pgfqpoint{6.713438in}{2.060556in}} %
\pgfusepath{clip}%
\pgfsetbuttcap%
\pgfsetroundjoin%
\definecolor{currentfill}{rgb}{0.000000,0.000000,1.000000}%
\pgfsetfillcolor{currentfill}%
\pgfsetlinewidth{1.003750pt}%
\definecolor{currentstroke}{rgb}{0.000000,0.000000,0.000000}%
\pgfsetstrokecolor{currentstroke}%
\pgfsetdash{}{0pt}%
\pgfpathmoveto{\pgfqpoint{3.847975in}{1.607939in}}%
\pgfpathcurveto{\pgfqpoint{3.856211in}{1.607939in}}{\pgfqpoint{3.864111in}{1.611211in}}{\pgfqpoint{3.869935in}{1.617035in}}%
\pgfpathcurveto{\pgfqpoint{3.875759in}{1.622859in}}{\pgfqpoint{3.879031in}{1.630759in}}{\pgfqpoint{3.879031in}{1.638995in}}%
\pgfpathcurveto{\pgfqpoint{3.879031in}{1.647231in}}{\pgfqpoint{3.875759in}{1.655132in}}{\pgfqpoint{3.869935in}{1.660955in}}%
\pgfpathcurveto{\pgfqpoint{3.864111in}{1.666779in}}{\pgfqpoint{3.856211in}{1.670052in}}{\pgfqpoint{3.847975in}{1.670052in}}%
\pgfpathcurveto{\pgfqpoint{3.839739in}{1.670052in}}{\pgfqpoint{3.831839in}{1.666779in}}{\pgfqpoint{3.826015in}{1.660955in}}%
\pgfpathcurveto{\pgfqpoint{3.820191in}{1.655132in}}{\pgfqpoint{3.816919in}{1.647231in}}{\pgfqpoint{3.816919in}{1.638995in}}%
\pgfpathcurveto{\pgfqpoint{3.816919in}{1.630759in}}{\pgfqpoint{3.820191in}{1.622859in}}{\pgfqpoint{3.826015in}{1.617035in}}%
\pgfpathcurveto{\pgfqpoint{3.831839in}{1.611211in}}{\pgfqpoint{3.839739in}{1.607939in}}{\pgfqpoint{3.847975in}{1.607939in}}%
\pgfpathclose%
\pgfusepath{stroke,fill}%
\end{pgfscope}%
\begin{pgfscope}%
\pgfpathrectangle{\pgfqpoint{0.894063in}{0.630000in}}{\pgfqpoint{6.713438in}{2.060556in}} %
\pgfusepath{clip}%
\pgfsetbuttcap%
\pgfsetroundjoin%
\definecolor{currentfill}{rgb}{0.000000,0.000000,1.000000}%
\pgfsetfillcolor{currentfill}%
\pgfsetlinewidth{1.003750pt}%
\definecolor{currentstroke}{rgb}{0.000000,0.000000,0.000000}%
\pgfsetstrokecolor{currentstroke}%
\pgfsetdash{}{0pt}%
\pgfpathmoveto{\pgfqpoint{7.607500in}{2.273027in}}%
\pgfpathcurveto{\pgfqpoint{7.615736in}{2.273027in}}{\pgfqpoint{7.623636in}{2.276299in}}{\pgfqpoint{7.629460in}{2.282123in}}%
\pgfpathcurveto{\pgfqpoint{7.635284in}{2.287947in}}{\pgfqpoint{7.638556in}{2.295847in}}{\pgfqpoint{7.638556in}{2.304084in}}%
\pgfpathcurveto{\pgfqpoint{7.638556in}{2.312320in}}{\pgfqpoint{7.635284in}{2.320220in}}{\pgfqpoint{7.629460in}{2.326044in}}%
\pgfpathcurveto{\pgfqpoint{7.623636in}{2.331868in}}{\pgfqpoint{7.615736in}{2.335140in}}{\pgfqpoint{7.607500in}{2.335140in}}%
\pgfpathcurveto{\pgfqpoint{7.599264in}{2.335140in}}{\pgfqpoint{7.591364in}{2.331868in}}{\pgfqpoint{7.585540in}{2.326044in}}%
\pgfpathcurveto{\pgfqpoint{7.579716in}{2.320220in}}{\pgfqpoint{7.576444in}{2.312320in}}{\pgfqpoint{7.576444in}{2.304084in}}%
\pgfpathcurveto{\pgfqpoint{7.576444in}{2.295847in}}{\pgfqpoint{7.579716in}{2.287947in}}{\pgfqpoint{7.585540in}{2.282123in}}%
\pgfpathcurveto{\pgfqpoint{7.591364in}{2.276299in}}{\pgfqpoint{7.599264in}{2.273027in}}{\pgfqpoint{7.607500in}{2.273027in}}%
\pgfpathclose%
\pgfusepath{stroke,fill}%
\end{pgfscope}%
\begin{pgfscope}%
\pgfpathrectangle{\pgfqpoint{0.894063in}{0.630000in}}{\pgfqpoint{6.713438in}{2.060556in}} %
\pgfusepath{clip}%
\pgfsetbuttcap%
\pgfsetroundjoin%
\definecolor{currentfill}{rgb}{0.000000,0.000000,1.000000}%
\pgfsetfillcolor{currentfill}%
\pgfsetlinewidth{1.003750pt}%
\definecolor{currentstroke}{rgb}{0.000000,0.000000,0.000000}%
\pgfsetstrokecolor{currentstroke}%
\pgfsetdash{}{0pt}%
\pgfpathmoveto{\pgfqpoint{4.385050in}{1.694347in}}%
\pgfpathcurveto{\pgfqpoint{4.393286in}{1.694347in}}{\pgfqpoint{4.401186in}{1.697619in}}{\pgfqpoint{4.407010in}{1.703443in}}%
\pgfpathcurveto{\pgfqpoint{4.412834in}{1.709267in}}{\pgfqpoint{4.416106in}{1.717167in}}{\pgfqpoint{4.416106in}{1.725403in}}%
\pgfpathcurveto{\pgfqpoint{4.416106in}{1.733639in}}{\pgfqpoint{4.412834in}{1.741539in}}{\pgfqpoint{4.407010in}{1.747363in}}%
\pgfpathcurveto{\pgfqpoint{4.401186in}{1.753187in}}{\pgfqpoint{4.393286in}{1.756460in}}{\pgfqpoint{4.385050in}{1.756460in}}%
\pgfpathcurveto{\pgfqpoint{4.376814in}{1.756460in}}{\pgfqpoint{4.368914in}{1.753187in}}{\pgfqpoint{4.363090in}{1.747363in}}%
\pgfpathcurveto{\pgfqpoint{4.357266in}{1.741539in}}{\pgfqpoint{4.353994in}{1.733639in}}{\pgfqpoint{4.353994in}{1.725403in}}%
\pgfpathcurveto{\pgfqpoint{4.353994in}{1.717167in}}{\pgfqpoint{4.357266in}{1.709267in}}{\pgfqpoint{4.363090in}{1.703443in}}%
\pgfpathcurveto{\pgfqpoint{4.368914in}{1.697619in}}{\pgfqpoint{4.376814in}{1.694347in}}{\pgfqpoint{4.385050in}{1.694347in}}%
\pgfpathclose%
\pgfusepath{stroke,fill}%
\end{pgfscope}%
\begin{pgfscope}%
\pgfpathrectangle{\pgfqpoint{0.894063in}{0.630000in}}{\pgfqpoint{6.713438in}{2.060556in}} %
\pgfusepath{clip}%
\pgfsetbuttcap%
\pgfsetroundjoin%
\definecolor{currentfill}{rgb}{0.000000,0.000000,1.000000}%
\pgfsetfillcolor{currentfill}%
\pgfsetlinewidth{1.003750pt}%
\definecolor{currentstroke}{rgb}{0.000000,0.000000,0.000000}%
\pgfsetstrokecolor{currentstroke}%
\pgfsetdash{}{0pt}%
\pgfpathmoveto{\pgfqpoint{6.533350in}{2.079812in}}%
\pgfpathcurveto{\pgfqpoint{6.541586in}{2.079812in}}{\pgfqpoint{6.549486in}{2.083084in}}{\pgfqpoint{6.555310in}{2.088908in}}%
\pgfpathcurveto{\pgfqpoint{6.561134in}{2.094732in}}{\pgfqpoint{6.564406in}{2.102632in}}{\pgfqpoint{6.564406in}{2.110868in}}%
\pgfpathcurveto{\pgfqpoint{6.564406in}{2.119105in}}{\pgfqpoint{6.561134in}{2.127005in}}{\pgfqpoint{6.555310in}{2.132829in}}%
\pgfpathcurveto{\pgfqpoint{6.549486in}{2.138652in}}{\pgfqpoint{6.541586in}{2.141925in}}{\pgfqpoint{6.533350in}{2.141925in}}%
\pgfpathcurveto{\pgfqpoint{6.525114in}{2.141925in}}{\pgfqpoint{6.517214in}{2.138652in}}{\pgfqpoint{6.511390in}{2.132829in}}%
\pgfpathcurveto{\pgfqpoint{6.505566in}{2.127005in}}{\pgfqpoint{6.502294in}{2.119105in}}{\pgfqpoint{6.502294in}{2.110868in}}%
\pgfpathcurveto{\pgfqpoint{6.502294in}{2.102632in}}{\pgfqpoint{6.505566in}{2.094732in}}{\pgfqpoint{6.511390in}{2.088908in}}%
\pgfpathcurveto{\pgfqpoint{6.517214in}{2.083084in}}{\pgfqpoint{6.525114in}{2.079812in}}{\pgfqpoint{6.533350in}{2.079812in}}%
\pgfpathclose%
\pgfusepath{stroke,fill}%
\end{pgfscope}%
\begin{pgfscope}%
\pgfpathrectangle{\pgfqpoint{0.894063in}{0.630000in}}{\pgfqpoint{6.713438in}{2.060556in}} %
\pgfusepath{clip}%
\pgfsetbuttcap%
\pgfsetroundjoin%
\definecolor{currentfill}{rgb}{0.000000,0.000000,1.000000}%
\pgfsetfillcolor{currentfill}%
\pgfsetlinewidth{1.003750pt}%
\definecolor{currentstroke}{rgb}{0.000000,0.000000,0.000000}%
\pgfsetstrokecolor{currentstroke}%
\pgfsetdash{}{0pt}%
\pgfpathmoveto{\pgfqpoint{1.296869in}{0.719215in}}%
\pgfpathcurveto{\pgfqpoint{1.305105in}{0.719215in}}{\pgfqpoint{1.313005in}{0.722487in}}{\pgfqpoint{1.318829in}{0.728311in}}%
\pgfpathcurveto{\pgfqpoint{1.324653in}{0.734135in}}{\pgfqpoint{1.327925in}{0.742035in}}{\pgfqpoint{1.327925in}{0.750272in}}%
\pgfpathcurveto{\pgfqpoint{1.327925in}{0.758508in}}{\pgfqpoint{1.324653in}{0.766408in}}{\pgfqpoint{1.318829in}{0.772232in}}%
\pgfpathcurveto{\pgfqpoint{1.313005in}{0.778056in}}{\pgfqpoint{1.305105in}{0.781328in}}{\pgfqpoint{1.296869in}{0.781328in}}%
\pgfpathcurveto{\pgfqpoint{1.288632in}{0.781328in}}{\pgfqpoint{1.280732in}{0.778056in}}{\pgfqpoint{1.274908in}{0.772232in}}%
\pgfpathcurveto{\pgfqpoint{1.269085in}{0.766408in}}{\pgfqpoint{1.265812in}{0.758508in}}{\pgfqpoint{1.265812in}{0.750272in}}%
\pgfpathcurveto{\pgfqpoint{1.265812in}{0.742035in}}{\pgfqpoint{1.269085in}{0.734135in}}{\pgfqpoint{1.274908in}{0.728311in}}%
\pgfpathcurveto{\pgfqpoint{1.280732in}{0.722487in}}{\pgfqpoint{1.288632in}{0.719215in}}{\pgfqpoint{1.296869in}{0.719215in}}%
\pgfpathclose%
\pgfusepath{stroke,fill}%
\end{pgfscope}%
\begin{pgfscope}%
\pgfpathrectangle{\pgfqpoint{0.894063in}{0.630000in}}{\pgfqpoint{6.713438in}{2.060556in}} %
\pgfusepath{clip}%
\pgfsetbuttcap%
\pgfsetroundjoin%
\definecolor{currentfill}{rgb}{0.000000,0.000000,1.000000}%
\pgfsetfillcolor{currentfill}%
\pgfsetlinewidth{1.003750pt}%
\definecolor{currentstroke}{rgb}{0.000000,0.000000,0.000000}%
\pgfsetstrokecolor{currentstroke}%
\pgfsetdash{}{0pt}%
\pgfpathmoveto{\pgfqpoint{4.519319in}{1.719974in}}%
\pgfpathcurveto{\pgfqpoint{4.527555in}{1.719974in}}{\pgfqpoint{4.535455in}{1.723246in}}{\pgfqpoint{4.541279in}{1.729070in}}%
\pgfpathcurveto{\pgfqpoint{4.547103in}{1.734894in}}{\pgfqpoint{4.550375in}{1.742794in}}{\pgfqpoint{4.550375in}{1.751031in}}%
\pgfpathcurveto{\pgfqpoint{4.550375in}{1.759267in}}{\pgfqpoint{4.547103in}{1.767167in}}{\pgfqpoint{4.541279in}{1.772991in}}%
\pgfpathcurveto{\pgfqpoint{4.535455in}{1.778815in}}{\pgfqpoint{4.527555in}{1.782087in}}{\pgfqpoint{4.519319in}{1.782087in}}%
\pgfpathcurveto{\pgfqpoint{4.511082in}{1.782087in}}{\pgfqpoint{4.503182in}{1.778815in}}{\pgfqpoint{4.497358in}{1.772991in}}%
\pgfpathcurveto{\pgfqpoint{4.491535in}{1.767167in}}{\pgfqpoint{4.488262in}{1.759267in}}{\pgfqpoint{4.488262in}{1.751031in}}%
\pgfpathcurveto{\pgfqpoint{4.488262in}{1.742794in}}{\pgfqpoint{4.491535in}{1.734894in}}{\pgfqpoint{4.497358in}{1.729070in}}%
\pgfpathcurveto{\pgfqpoint{4.503182in}{1.723246in}}{\pgfqpoint{4.511082in}{1.719974in}}{\pgfqpoint{4.519319in}{1.719974in}}%
\pgfpathclose%
\pgfusepath{stroke,fill}%
\end{pgfscope}%
\begin{pgfscope}%
\pgfpathrectangle{\pgfqpoint{0.894063in}{0.630000in}}{\pgfqpoint{6.713438in}{2.060556in}} %
\pgfusepath{clip}%
\pgfsetbuttcap%
\pgfsetroundjoin%
\definecolor{currentfill}{rgb}{0.000000,0.000000,1.000000}%
\pgfsetfillcolor{currentfill}%
\pgfsetlinewidth{1.003750pt}%
\definecolor{currentstroke}{rgb}{0.000000,0.000000,0.000000}%
\pgfsetstrokecolor{currentstroke}%
\pgfsetdash{}{0pt}%
\pgfpathmoveto{\pgfqpoint{2.505288in}{1.376474in}}%
\pgfpathcurveto{\pgfqpoint{2.513524in}{1.376474in}}{\pgfqpoint{2.521424in}{1.379746in}}{\pgfqpoint{2.527248in}{1.385570in}}%
\pgfpathcurveto{\pgfqpoint{2.533072in}{1.391394in}}{\pgfqpoint{2.536344in}{1.399294in}}{\pgfqpoint{2.536344in}{1.407530in}}%
\pgfpathcurveto{\pgfqpoint{2.536344in}{1.415766in}}{\pgfqpoint{2.533072in}{1.423666in}}{\pgfqpoint{2.527248in}{1.429490in}}%
\pgfpathcurveto{\pgfqpoint{2.521424in}{1.435314in}}{\pgfqpoint{2.513524in}{1.438587in}}{\pgfqpoint{2.505288in}{1.438587in}}%
\pgfpathcurveto{\pgfqpoint{2.497051in}{1.438587in}}{\pgfqpoint{2.489151in}{1.435314in}}{\pgfqpoint{2.483327in}{1.429490in}}%
\pgfpathcurveto{\pgfqpoint{2.477503in}{1.423666in}}{\pgfqpoint{2.474231in}{1.415766in}}{\pgfqpoint{2.474231in}{1.407530in}}%
\pgfpathcurveto{\pgfqpoint{2.474231in}{1.399294in}}{\pgfqpoint{2.477503in}{1.391394in}}{\pgfqpoint{2.483327in}{1.385570in}}%
\pgfpathcurveto{\pgfqpoint{2.489151in}{1.379746in}}{\pgfqpoint{2.497051in}{1.376474in}}{\pgfqpoint{2.505288in}{1.376474in}}%
\pgfpathclose%
\pgfusepath{stroke,fill}%
\end{pgfscope}%
\begin{pgfscope}%
\pgfpathrectangle{\pgfqpoint{0.894063in}{0.630000in}}{\pgfqpoint{6.713438in}{2.060556in}} %
\pgfusepath{clip}%
\pgfsetbuttcap%
\pgfsetroundjoin%
\definecolor{currentfill}{rgb}{0.000000,0.000000,1.000000}%
\pgfsetfillcolor{currentfill}%
\pgfsetlinewidth{1.003750pt}%
\definecolor{currentstroke}{rgb}{0.000000,0.000000,0.000000}%
\pgfsetstrokecolor{currentstroke}%
\pgfsetdash{}{0pt}%
\pgfpathmoveto{\pgfqpoint{5.459200in}{1.897929in}}%
\pgfpathcurveto{\pgfqpoint{5.467436in}{1.897929in}}{\pgfqpoint{5.475336in}{1.901202in}}{\pgfqpoint{5.481160in}{1.907026in}}%
\pgfpathcurveto{\pgfqpoint{5.486984in}{1.912850in}}{\pgfqpoint{5.490256in}{1.920750in}}{\pgfqpoint{5.490256in}{1.928986in}}%
\pgfpathcurveto{\pgfqpoint{5.490256in}{1.937222in}}{\pgfqpoint{5.486984in}{1.945122in}}{\pgfqpoint{5.481160in}{1.950946in}}%
\pgfpathcurveto{\pgfqpoint{5.475336in}{1.956770in}}{\pgfqpoint{5.467436in}{1.960042in}}{\pgfqpoint{5.459200in}{1.960042in}}%
\pgfpathcurveto{\pgfqpoint{5.450964in}{1.960042in}}{\pgfqpoint{5.443064in}{1.956770in}}{\pgfqpoint{5.437240in}{1.950946in}}%
\pgfpathcurveto{\pgfqpoint{5.431416in}{1.945122in}}{\pgfqpoint{5.428144in}{1.937222in}}{\pgfqpoint{5.428144in}{1.928986in}}%
\pgfpathcurveto{\pgfqpoint{5.428144in}{1.920750in}}{\pgfqpoint{5.431416in}{1.912850in}}{\pgfqpoint{5.437240in}{1.907026in}}%
\pgfpathcurveto{\pgfqpoint{5.443064in}{1.901202in}}{\pgfqpoint{5.450964in}{1.897929in}}{\pgfqpoint{5.459200in}{1.897929in}}%
\pgfpathclose%
\pgfusepath{stroke,fill}%
\end{pgfscope}%
\begin{pgfscope}%
\pgfpathrectangle{\pgfqpoint{0.894063in}{0.630000in}}{\pgfqpoint{6.713438in}{2.060556in}} %
\pgfusepath{clip}%
\pgfsetbuttcap%
\pgfsetroundjoin%
\definecolor{currentfill}{rgb}{0.000000,0.000000,1.000000}%
\pgfsetfillcolor{currentfill}%
\pgfsetlinewidth{1.003750pt}%
\definecolor{currentstroke}{rgb}{0.000000,0.000000,0.000000}%
\pgfsetstrokecolor{currentstroke}%
\pgfsetdash{}{0pt}%
\pgfpathmoveto{\pgfqpoint{6.936156in}{2.148275in}}%
\pgfpathcurveto{\pgfqpoint{6.944393in}{2.148275in}}{\pgfqpoint{6.952293in}{2.151548in}}{\pgfqpoint{6.958117in}{2.157371in}}%
\pgfpathcurveto{\pgfqpoint{6.963940in}{2.163195in}}{\pgfqpoint{6.967213in}{2.171095in}}{\pgfqpoint{6.967213in}{2.179332in}}%
\pgfpathcurveto{\pgfqpoint{6.967213in}{2.187568in}}{\pgfqpoint{6.963940in}{2.195468in}}{\pgfqpoint{6.958117in}{2.201292in}}%
\pgfpathcurveto{\pgfqpoint{6.952293in}{2.207116in}}{\pgfqpoint{6.944393in}{2.210388in}}{\pgfqpoint{6.936156in}{2.210388in}}%
\pgfpathcurveto{\pgfqpoint{6.927920in}{2.210388in}}{\pgfqpoint{6.920020in}{2.207116in}}{\pgfqpoint{6.914196in}{2.201292in}}%
\pgfpathcurveto{\pgfqpoint{6.908372in}{2.195468in}}{\pgfqpoint{6.905100in}{2.187568in}}{\pgfqpoint{6.905100in}{2.179332in}}%
\pgfpathcurveto{\pgfqpoint{6.905100in}{2.171095in}}{\pgfqpoint{6.908372in}{2.163195in}}{\pgfqpoint{6.914196in}{2.157371in}}%
\pgfpathcurveto{\pgfqpoint{6.920020in}{2.151548in}}{\pgfqpoint{6.927920in}{2.148275in}}{\pgfqpoint{6.936156in}{2.148275in}}%
\pgfpathclose%
\pgfusepath{stroke,fill}%
\end{pgfscope}%
\begin{pgfscope}%
\pgfpathrectangle{\pgfqpoint{0.894063in}{0.630000in}}{\pgfqpoint{6.713438in}{2.060556in}} %
\pgfusepath{clip}%
\pgfsetbuttcap%
\pgfsetroundjoin%
\definecolor{currentfill}{rgb}{0.000000,0.000000,1.000000}%
\pgfsetfillcolor{currentfill}%
\pgfsetlinewidth{1.003750pt}%
\definecolor{currentstroke}{rgb}{0.000000,0.000000,0.000000}%
\pgfsetstrokecolor{currentstroke}%
\pgfsetdash{}{0pt}%
\pgfpathmoveto{\pgfqpoint{5.862006in}{1.959640in}}%
\pgfpathcurveto{\pgfqpoint{5.870243in}{1.959640in}}{\pgfqpoint{5.878143in}{1.962913in}}{\pgfqpoint{5.883967in}{1.968736in}}%
\pgfpathcurveto{\pgfqpoint{5.889790in}{1.974560in}}{\pgfqpoint{5.893063in}{1.982460in}}{\pgfqpoint{5.893063in}{1.990697in}}%
\pgfpathcurveto{\pgfqpoint{5.893063in}{1.998933in}}{\pgfqpoint{5.889790in}{2.006833in}}{\pgfqpoint{5.883967in}{2.012657in}}%
\pgfpathcurveto{\pgfqpoint{5.878143in}{2.018481in}}{\pgfqpoint{5.870243in}{2.021753in}}{\pgfqpoint{5.862006in}{2.021753in}}%
\pgfpathcurveto{\pgfqpoint{5.853770in}{2.021753in}}{\pgfqpoint{5.845870in}{2.018481in}}{\pgfqpoint{5.840046in}{2.012657in}}%
\pgfpathcurveto{\pgfqpoint{5.834222in}{2.006833in}}{\pgfqpoint{5.830950in}{1.998933in}}{\pgfqpoint{5.830950in}{1.990697in}}%
\pgfpathcurveto{\pgfqpoint{5.830950in}{1.982460in}}{\pgfqpoint{5.834222in}{1.974560in}}{\pgfqpoint{5.840046in}{1.968736in}}%
\pgfpathcurveto{\pgfqpoint{5.845870in}{1.962913in}}{\pgfqpoint{5.853770in}{1.959640in}}{\pgfqpoint{5.862006in}{1.959640in}}%
\pgfpathclose%
\pgfusepath{stroke,fill}%
\end{pgfscope}%
\begin{pgfscope}%
\pgfpathrectangle{\pgfqpoint{0.894063in}{0.630000in}}{\pgfqpoint{6.713438in}{2.060556in}} %
\pgfusepath{clip}%
\pgfsetbuttcap%
\pgfsetroundjoin%
\definecolor{currentfill}{rgb}{0.000000,0.000000,1.000000}%
\pgfsetfillcolor{currentfill}%
\pgfsetlinewidth{1.003750pt}%
\definecolor{currentstroke}{rgb}{0.000000,0.000000,0.000000}%
\pgfsetstrokecolor{currentstroke}%
\pgfsetdash{}{0pt}%
\pgfpathmoveto{\pgfqpoint{7.070425in}{2.170135in}}%
\pgfpathcurveto{\pgfqpoint{7.078661in}{2.170135in}}{\pgfqpoint{7.086561in}{2.173407in}}{\pgfqpoint{7.092385in}{2.179231in}}%
\pgfpathcurveto{\pgfqpoint{7.098209in}{2.185055in}}{\pgfqpoint{7.101481in}{2.192955in}}{\pgfqpoint{7.101481in}{2.201191in}}%
\pgfpathcurveto{\pgfqpoint{7.101481in}{2.209428in}}{\pgfqpoint{7.098209in}{2.217328in}}{\pgfqpoint{7.092385in}{2.223152in}}%
\pgfpathcurveto{\pgfqpoint{7.086561in}{2.228975in}}{\pgfqpoint{7.078661in}{2.232248in}}{\pgfqpoint{7.070425in}{2.232248in}}%
\pgfpathcurveto{\pgfqpoint{7.062189in}{2.232248in}}{\pgfqpoint{7.054289in}{2.228975in}}{\pgfqpoint{7.048465in}{2.223152in}}%
\pgfpathcurveto{\pgfqpoint{7.042641in}{2.217328in}}{\pgfqpoint{7.039369in}{2.209428in}}{\pgfqpoint{7.039369in}{2.201191in}}%
\pgfpathcurveto{\pgfqpoint{7.039369in}{2.192955in}}{\pgfqpoint{7.042641in}{2.185055in}}{\pgfqpoint{7.048465in}{2.179231in}}%
\pgfpathcurveto{\pgfqpoint{7.054289in}{2.173407in}}{\pgfqpoint{7.062189in}{2.170135in}}{\pgfqpoint{7.070425in}{2.170135in}}%
\pgfpathclose%
\pgfusepath{stroke,fill}%
\end{pgfscope}%
\begin{pgfscope}%
\pgfpathrectangle{\pgfqpoint{0.894063in}{0.630000in}}{\pgfqpoint{6.713438in}{2.060556in}} %
\pgfusepath{clip}%
\pgfsetbuttcap%
\pgfsetroundjoin%
\definecolor{currentfill}{rgb}{0.000000,0.000000,1.000000}%
\pgfsetfillcolor{currentfill}%
\pgfsetlinewidth{1.003750pt}%
\definecolor{currentstroke}{rgb}{0.000000,0.000000,0.000000}%
\pgfsetstrokecolor{currentstroke}%
\pgfsetdash{}{0pt}%
\pgfpathmoveto{\pgfqpoint{3.176631in}{1.498093in}}%
\pgfpathcurveto{\pgfqpoint{3.184868in}{1.498093in}}{\pgfqpoint{3.192768in}{1.501366in}}{\pgfqpoint{3.198592in}{1.507190in}}%
\pgfpathcurveto{\pgfqpoint{3.204415in}{1.513014in}}{\pgfqpoint{3.207688in}{1.520914in}}{\pgfqpoint{3.207688in}{1.529150in}}%
\pgfpathcurveto{\pgfqpoint{3.207688in}{1.537386in}}{\pgfqpoint{3.204415in}{1.545286in}}{\pgfqpoint{3.198592in}{1.551110in}}%
\pgfpathcurveto{\pgfqpoint{3.192768in}{1.556934in}}{\pgfqpoint{3.184868in}{1.560206in}}{\pgfqpoint{3.176631in}{1.560206in}}%
\pgfpathcurveto{\pgfqpoint{3.168395in}{1.560206in}}{\pgfqpoint{3.160495in}{1.556934in}}{\pgfqpoint{3.154671in}{1.551110in}}%
\pgfpathcurveto{\pgfqpoint{3.148847in}{1.545286in}}{\pgfqpoint{3.145575in}{1.537386in}}{\pgfqpoint{3.145575in}{1.529150in}}%
\pgfpathcurveto{\pgfqpoint{3.145575in}{1.520914in}}{\pgfqpoint{3.148847in}{1.513014in}}{\pgfqpoint{3.154671in}{1.507190in}}%
\pgfpathcurveto{\pgfqpoint{3.160495in}{1.501366in}}{\pgfqpoint{3.168395in}{1.498093in}}{\pgfqpoint{3.176631in}{1.498093in}}%
\pgfpathclose%
\pgfusepath{stroke,fill}%
\end{pgfscope}%
\begin{pgfscope}%
\pgfpathrectangle{\pgfqpoint{0.894063in}{0.630000in}}{\pgfqpoint{6.713438in}{2.060556in}} %
\pgfusepath{clip}%
\pgfsetbuttcap%
\pgfsetroundjoin%
\definecolor{currentfill}{rgb}{0.000000,0.000000,1.000000}%
\pgfsetfillcolor{currentfill}%
\pgfsetlinewidth{1.003750pt}%
\definecolor{currentstroke}{rgb}{0.000000,0.000000,0.000000}%
\pgfsetstrokecolor{currentstroke}%
\pgfsetdash{}{0pt}%
\pgfpathmoveto{\pgfqpoint{2.102481in}{1.299338in}}%
\pgfpathcurveto{\pgfqpoint{2.110718in}{1.299338in}}{\pgfqpoint{2.118618in}{1.302610in}}{\pgfqpoint{2.124442in}{1.308434in}}%
\pgfpathcurveto{\pgfqpoint{2.130265in}{1.314258in}}{\pgfqpoint{2.133538in}{1.322158in}}{\pgfqpoint{2.133538in}{1.330395in}}%
\pgfpathcurveto{\pgfqpoint{2.133538in}{1.338631in}}{\pgfqpoint{2.130265in}{1.346531in}}{\pgfqpoint{2.124442in}{1.352355in}}%
\pgfpathcurveto{\pgfqpoint{2.118618in}{1.358179in}}{\pgfqpoint{2.110718in}{1.361451in}}{\pgfqpoint{2.102481in}{1.361451in}}%
\pgfpathcurveto{\pgfqpoint{2.094245in}{1.361451in}}{\pgfqpoint{2.086345in}{1.358179in}}{\pgfqpoint{2.080521in}{1.352355in}}%
\pgfpathcurveto{\pgfqpoint{2.074697in}{1.346531in}}{\pgfqpoint{2.071425in}{1.338631in}}{\pgfqpoint{2.071425in}{1.330395in}}%
\pgfpathcurveto{\pgfqpoint{2.071425in}{1.322158in}}{\pgfqpoint{2.074697in}{1.314258in}}{\pgfqpoint{2.080521in}{1.308434in}}%
\pgfpathcurveto{\pgfqpoint{2.086345in}{1.302610in}}{\pgfqpoint{2.094245in}{1.299338in}}{\pgfqpoint{2.102481in}{1.299338in}}%
\pgfpathclose%
\pgfusepath{stroke,fill}%
\end{pgfscope}%
\begin{pgfscope}%
\pgfpathrectangle{\pgfqpoint{0.894063in}{0.630000in}}{\pgfqpoint{6.713438in}{2.060556in}} %
\pgfusepath{clip}%
\pgfsetbuttcap%
\pgfsetroundjoin%
\definecolor{currentfill}{rgb}{0.000000,0.000000,1.000000}%
\pgfsetfillcolor{currentfill}%
\pgfsetlinewidth{1.003750pt}%
\definecolor{currentstroke}{rgb}{0.000000,0.000000,0.000000}%
\pgfsetstrokecolor{currentstroke}%
\pgfsetdash{}{0pt}%
\pgfpathmoveto{\pgfqpoint{1.968213in}{1.293421in}}%
\pgfpathcurveto{\pgfqpoint{1.976449in}{1.293421in}}{\pgfqpoint{1.984349in}{1.296694in}}{\pgfqpoint{1.990173in}{1.302518in}}%
\pgfpathcurveto{\pgfqpoint{1.995997in}{1.308342in}}{\pgfqpoint{1.999269in}{1.316242in}}{\pgfqpoint{1.999269in}{1.324478in}}%
\pgfpathcurveto{\pgfqpoint{1.999269in}{1.332714in}}{\pgfqpoint{1.995997in}{1.340614in}}{\pgfqpoint{1.990173in}{1.346438in}}%
\pgfpathcurveto{\pgfqpoint{1.984349in}{1.352262in}}{\pgfqpoint{1.976449in}{1.355534in}}{\pgfqpoint{1.968213in}{1.355534in}}%
\pgfpathcurveto{\pgfqpoint{1.959976in}{1.355534in}}{\pgfqpoint{1.952076in}{1.352262in}}{\pgfqpoint{1.946252in}{1.346438in}}%
\pgfpathcurveto{\pgfqpoint{1.940428in}{1.340614in}}{\pgfqpoint{1.937156in}{1.332714in}}{\pgfqpoint{1.937156in}{1.324478in}}%
\pgfpathcurveto{\pgfqpoint{1.937156in}{1.316242in}}{\pgfqpoint{1.940428in}{1.308342in}}{\pgfqpoint{1.946252in}{1.302518in}}%
\pgfpathcurveto{\pgfqpoint{1.952076in}{1.296694in}}{\pgfqpoint{1.959976in}{1.293421in}}{\pgfqpoint{1.968213in}{1.293421in}}%
\pgfpathclose%
\pgfusepath{stroke,fill}%
\end{pgfscope}%
\begin{pgfscope}%
\pgfpathrectangle{\pgfqpoint{0.894063in}{0.630000in}}{\pgfqpoint{6.713438in}{2.060556in}} %
\pgfusepath{clip}%
\pgfsetbuttcap%
\pgfsetroundjoin%
\definecolor{currentfill}{rgb}{0.000000,0.000000,1.000000}%
\pgfsetfillcolor{currentfill}%
\pgfsetlinewidth{1.003750pt}%
\definecolor{currentstroke}{rgb}{0.000000,0.000000,0.000000}%
\pgfsetstrokecolor{currentstroke}%
\pgfsetdash{}{0pt}%
\pgfpathmoveto{\pgfqpoint{3.310900in}{1.523556in}}%
\pgfpathcurveto{\pgfqpoint{3.319136in}{1.523556in}}{\pgfqpoint{3.327036in}{1.526828in}}{\pgfqpoint{3.332860in}{1.532652in}}%
\pgfpathcurveto{\pgfqpoint{3.338684in}{1.538476in}}{\pgfqpoint{3.341956in}{1.546376in}}{\pgfqpoint{3.341956in}{1.554612in}}%
\pgfpathcurveto{\pgfqpoint{3.341956in}{1.562849in}}{\pgfqpoint{3.338684in}{1.570749in}}{\pgfqpoint{3.332860in}{1.576573in}}%
\pgfpathcurveto{\pgfqpoint{3.327036in}{1.582397in}}{\pgfqpoint{3.319136in}{1.585669in}}{\pgfqpoint{3.310900in}{1.585669in}}%
\pgfpathcurveto{\pgfqpoint{3.302664in}{1.585669in}}{\pgfqpoint{3.294764in}{1.582397in}}{\pgfqpoint{3.288940in}{1.576573in}}%
\pgfpathcurveto{\pgfqpoint{3.283116in}{1.570749in}}{\pgfqpoint{3.279844in}{1.562849in}}{\pgfqpoint{3.279844in}{1.554612in}}%
\pgfpathcurveto{\pgfqpoint{3.279844in}{1.546376in}}{\pgfqpoint{3.283116in}{1.538476in}}{\pgfqpoint{3.288940in}{1.532652in}}%
\pgfpathcurveto{\pgfqpoint{3.294764in}{1.526828in}}{\pgfqpoint{3.302664in}{1.523556in}}{\pgfqpoint{3.310900in}{1.523556in}}%
\pgfpathclose%
\pgfusepath{stroke,fill}%
\end{pgfscope}%
\begin{pgfscope}%
\pgfpathrectangle{\pgfqpoint{0.894063in}{0.630000in}}{\pgfqpoint{6.713438in}{2.060556in}} %
\pgfusepath{clip}%
\pgfsetbuttcap%
\pgfsetroundjoin%
\definecolor{currentfill}{rgb}{0.000000,0.000000,1.000000}%
\pgfsetfillcolor{currentfill}%
\pgfsetlinewidth{1.003750pt}%
\definecolor{currentstroke}{rgb}{0.000000,0.000000,0.000000}%
\pgfsetstrokecolor{currentstroke}%
\pgfsetdash{}{0pt}%
\pgfpathmoveto{\pgfqpoint{5.593469in}{1.920443in}}%
\pgfpathcurveto{\pgfqpoint{5.601705in}{1.920443in}}{\pgfqpoint{5.609605in}{1.923715in}}{\pgfqpoint{5.615429in}{1.929539in}}%
\pgfpathcurveto{\pgfqpoint{5.621253in}{1.935363in}}{\pgfqpoint{5.624525in}{1.943263in}}{\pgfqpoint{5.624525in}{1.951499in}}%
\pgfpathcurveto{\pgfqpoint{5.624525in}{1.959735in}}{\pgfqpoint{5.621253in}{1.967635in}}{\pgfqpoint{5.615429in}{1.973459in}}%
\pgfpathcurveto{\pgfqpoint{5.609605in}{1.979283in}}{\pgfqpoint{5.601705in}{1.982556in}}{\pgfqpoint{5.593469in}{1.982556in}}%
\pgfpathcurveto{\pgfqpoint{5.585232in}{1.982556in}}{\pgfqpoint{5.577332in}{1.979283in}}{\pgfqpoint{5.571508in}{1.973459in}}%
\pgfpathcurveto{\pgfqpoint{5.565685in}{1.967635in}}{\pgfqpoint{5.562412in}{1.959735in}}{\pgfqpoint{5.562412in}{1.951499in}}%
\pgfpathcurveto{\pgfqpoint{5.562412in}{1.943263in}}{\pgfqpoint{5.565685in}{1.935363in}}{\pgfqpoint{5.571508in}{1.929539in}}%
\pgfpathcurveto{\pgfqpoint{5.577332in}{1.923715in}}{\pgfqpoint{5.585232in}{1.920443in}}{\pgfqpoint{5.593469in}{1.920443in}}%
\pgfpathclose%
\pgfusepath{stroke,fill}%
\end{pgfscope}%
\begin{pgfscope}%
\pgfpathrectangle{\pgfqpoint{0.894063in}{0.630000in}}{\pgfqpoint{6.713438in}{2.060556in}} %
\pgfusepath{clip}%
\pgfsetbuttcap%
\pgfsetroundjoin%
\definecolor{currentfill}{rgb}{0.000000,0.000000,1.000000}%
\pgfsetfillcolor{currentfill}%
\pgfsetlinewidth{1.003750pt}%
\definecolor{currentstroke}{rgb}{0.000000,0.000000,0.000000}%
\pgfsetstrokecolor{currentstroke}%
\pgfsetdash{}{0pt}%
\pgfpathmoveto{\pgfqpoint{3.042363in}{1.472902in}}%
\pgfpathcurveto{\pgfqpoint{3.050599in}{1.472902in}}{\pgfqpoint{3.058499in}{1.476174in}}{\pgfqpoint{3.064323in}{1.481998in}}%
\pgfpathcurveto{\pgfqpoint{3.070147in}{1.487822in}}{\pgfqpoint{3.073419in}{1.495722in}}{\pgfqpoint{3.073419in}{1.503958in}}%
\pgfpathcurveto{\pgfqpoint{3.073419in}{1.512194in}}{\pgfqpoint{3.070147in}{1.520094in}}{\pgfqpoint{3.064323in}{1.525918in}}%
\pgfpathcurveto{\pgfqpoint{3.058499in}{1.531742in}}{\pgfqpoint{3.050599in}{1.535015in}}{\pgfqpoint{3.042363in}{1.535015in}}%
\pgfpathcurveto{\pgfqpoint{3.034126in}{1.535015in}}{\pgfqpoint{3.026226in}{1.531742in}}{\pgfqpoint{3.020402in}{1.525918in}}%
\pgfpathcurveto{\pgfqpoint{3.014578in}{1.520094in}}{\pgfqpoint{3.011306in}{1.512194in}}{\pgfqpoint{3.011306in}{1.503958in}}%
\pgfpathcurveto{\pgfqpoint{3.011306in}{1.495722in}}{\pgfqpoint{3.014578in}{1.487822in}}{\pgfqpoint{3.020402in}{1.481998in}}%
\pgfpathcurveto{\pgfqpoint{3.026226in}{1.476174in}}{\pgfqpoint{3.034126in}{1.472902in}}{\pgfqpoint{3.042363in}{1.472902in}}%
\pgfpathclose%
\pgfusepath{stroke,fill}%
\end{pgfscope}%
\begin{pgfscope}%
\pgfpathrectangle{\pgfqpoint{0.894063in}{0.630000in}}{\pgfqpoint{6.713438in}{2.060556in}} %
\pgfusepath{clip}%
\pgfsetbuttcap%
\pgfsetroundjoin%
\definecolor{currentfill}{rgb}{0.000000,0.000000,1.000000}%
\pgfsetfillcolor{currentfill}%
\pgfsetlinewidth{1.003750pt}%
\definecolor{currentstroke}{rgb}{0.000000,0.000000,0.000000}%
\pgfsetstrokecolor{currentstroke}%
\pgfsetdash{}{0pt}%
\pgfpathmoveto{\pgfqpoint{5.190663in}{1.847728in}}%
\pgfpathcurveto{\pgfqpoint{5.198899in}{1.847728in}}{\pgfqpoint{5.206799in}{1.851001in}}{\pgfqpoint{5.212623in}{1.856825in}}%
\pgfpathcurveto{\pgfqpoint{5.218447in}{1.862649in}}{\pgfqpoint{5.221719in}{1.870549in}}{\pgfqpoint{5.221719in}{1.878785in}}%
\pgfpathcurveto{\pgfqpoint{5.221719in}{1.887021in}}{\pgfqpoint{5.218447in}{1.894921in}}{\pgfqpoint{5.212623in}{1.900745in}}%
\pgfpathcurveto{\pgfqpoint{5.206799in}{1.906569in}}{\pgfqpoint{5.198899in}{1.909841in}}{\pgfqpoint{5.190663in}{1.909841in}}%
\pgfpathcurveto{\pgfqpoint{5.182426in}{1.909841in}}{\pgfqpoint{5.174526in}{1.906569in}}{\pgfqpoint{5.168702in}{1.900745in}}%
\pgfpathcurveto{\pgfqpoint{5.162878in}{1.894921in}}{\pgfqpoint{5.159606in}{1.887021in}}{\pgfqpoint{5.159606in}{1.878785in}}%
\pgfpathcurveto{\pgfqpoint{5.159606in}{1.870549in}}{\pgfqpoint{5.162878in}{1.862649in}}{\pgfqpoint{5.168702in}{1.856825in}}%
\pgfpathcurveto{\pgfqpoint{5.174526in}{1.851001in}}{\pgfqpoint{5.182426in}{1.847728in}}{\pgfqpoint{5.190663in}{1.847728in}}%
\pgfpathclose%
\pgfusepath{stroke,fill}%
\end{pgfscope}%
\begin{pgfscope}%
\pgfpathrectangle{\pgfqpoint{0.894063in}{0.630000in}}{\pgfqpoint{6.713438in}{2.060556in}} %
\pgfusepath{clip}%
\pgfsetbuttcap%
\pgfsetroundjoin%
\definecolor{currentfill}{rgb}{0.000000,0.000000,1.000000}%
\pgfsetfillcolor{currentfill}%
\pgfsetlinewidth{1.003750pt}%
\definecolor{currentstroke}{rgb}{0.000000,0.000000,0.000000}%
\pgfsetstrokecolor{currentstroke}%
\pgfsetdash{}{0pt}%
\pgfpathmoveto{\pgfqpoint{6.801888in}{2.128729in}}%
\pgfpathcurveto{\pgfqpoint{6.810124in}{2.128729in}}{\pgfqpoint{6.818024in}{2.132002in}}{\pgfqpoint{6.823848in}{2.137826in}}%
\pgfpathcurveto{\pgfqpoint{6.829672in}{2.143650in}}{\pgfqpoint{6.832944in}{2.151550in}}{\pgfqpoint{6.832944in}{2.159786in}}%
\pgfpathcurveto{\pgfqpoint{6.832944in}{2.168022in}}{\pgfqpoint{6.829672in}{2.175922in}}{\pgfqpoint{6.823848in}{2.181746in}}%
\pgfpathcurveto{\pgfqpoint{6.818024in}{2.187570in}}{\pgfqpoint{6.810124in}{2.190842in}}{\pgfqpoint{6.801888in}{2.190842in}}%
\pgfpathcurveto{\pgfqpoint{6.793651in}{2.190842in}}{\pgfqpoint{6.785751in}{2.187570in}}{\pgfqpoint{6.779927in}{2.181746in}}%
\pgfpathcurveto{\pgfqpoint{6.774103in}{2.175922in}}{\pgfqpoint{6.770831in}{2.168022in}}{\pgfqpoint{6.770831in}{2.159786in}}%
\pgfpathcurveto{\pgfqpoint{6.770831in}{2.151550in}}{\pgfqpoint{6.774103in}{2.143650in}}{\pgfqpoint{6.779927in}{2.137826in}}%
\pgfpathcurveto{\pgfqpoint{6.785751in}{2.132002in}}{\pgfqpoint{6.793651in}{2.128729in}}{\pgfqpoint{6.801888in}{2.128729in}}%
\pgfpathclose%
\pgfusepath{stroke,fill}%
\end{pgfscope}%
\begin{pgfscope}%
\pgfpathrectangle{\pgfqpoint{0.894063in}{0.630000in}}{\pgfqpoint{6.713438in}{2.060556in}} %
\pgfusepath{clip}%
\pgfsetbuttcap%
\pgfsetroundjoin%
\definecolor{currentfill}{rgb}{0.000000,0.000000,1.000000}%
\pgfsetfillcolor{currentfill}%
\pgfsetlinewidth{1.003750pt}%
\definecolor{currentstroke}{rgb}{0.000000,0.000000,0.000000}%
\pgfsetstrokecolor{currentstroke}%
\pgfsetdash{}{0pt}%
\pgfpathmoveto{\pgfqpoint{3.579438in}{1.567987in}}%
\pgfpathcurveto{\pgfqpoint{3.587674in}{1.567987in}}{\pgfqpoint{3.595574in}{1.571260in}}{\pgfqpoint{3.601398in}{1.577084in}}%
\pgfpathcurveto{\pgfqpoint{3.607222in}{1.582908in}}{\pgfqpoint{3.610494in}{1.590808in}}{\pgfqpoint{3.610494in}{1.599044in}}%
\pgfpathcurveto{\pgfqpoint{3.610494in}{1.607280in}}{\pgfqpoint{3.607222in}{1.615180in}}{\pgfqpoint{3.601398in}{1.621004in}}%
\pgfpathcurveto{\pgfqpoint{3.595574in}{1.626828in}}{\pgfqpoint{3.587674in}{1.630100in}}{\pgfqpoint{3.579438in}{1.630100in}}%
\pgfpathcurveto{\pgfqpoint{3.571201in}{1.630100in}}{\pgfqpoint{3.563301in}{1.626828in}}{\pgfqpoint{3.557477in}{1.621004in}}%
\pgfpathcurveto{\pgfqpoint{3.551653in}{1.615180in}}{\pgfqpoint{3.548381in}{1.607280in}}{\pgfqpoint{3.548381in}{1.599044in}}%
\pgfpathcurveto{\pgfqpoint{3.548381in}{1.590808in}}{\pgfqpoint{3.551653in}{1.582908in}}{\pgfqpoint{3.557477in}{1.577084in}}%
\pgfpathcurveto{\pgfqpoint{3.563301in}{1.571260in}}{\pgfqpoint{3.571201in}{1.567987in}}{\pgfqpoint{3.579438in}{1.567987in}}%
\pgfpathclose%
\pgfusepath{stroke,fill}%
\end{pgfscope}%
\begin{pgfscope}%
\pgfpathrectangle{\pgfqpoint{0.894063in}{0.630000in}}{\pgfqpoint{6.713438in}{2.060556in}} %
\pgfusepath{clip}%
\pgfsetbuttcap%
\pgfsetroundjoin%
\definecolor{currentfill}{rgb}{0.000000,0.000000,1.000000}%
\pgfsetfillcolor{currentfill}%
\pgfsetlinewidth{1.003750pt}%
\definecolor{currentstroke}{rgb}{0.000000,0.000000,0.000000}%
\pgfsetstrokecolor{currentstroke}%
\pgfsetdash{}{0pt}%
\pgfpathmoveto{\pgfqpoint{2.371019in}{1.349251in}}%
\pgfpathcurveto{\pgfqpoint{2.379255in}{1.349251in}}{\pgfqpoint{2.387155in}{1.352523in}}{\pgfqpoint{2.392979in}{1.358347in}}%
\pgfpathcurveto{\pgfqpoint{2.398803in}{1.364171in}}{\pgfqpoint{2.402075in}{1.372071in}}{\pgfqpoint{2.402075in}{1.380307in}}%
\pgfpathcurveto{\pgfqpoint{2.402075in}{1.388543in}}{\pgfqpoint{2.398803in}{1.396443in}}{\pgfqpoint{2.392979in}{1.402267in}}%
\pgfpathcurveto{\pgfqpoint{2.387155in}{1.408091in}}{\pgfqpoint{2.379255in}{1.411364in}}{\pgfqpoint{2.371019in}{1.411364in}}%
\pgfpathcurveto{\pgfqpoint{2.362782in}{1.411364in}}{\pgfqpoint{2.354882in}{1.408091in}}{\pgfqpoint{2.349058in}{1.402267in}}%
\pgfpathcurveto{\pgfqpoint{2.343235in}{1.396443in}}{\pgfqpoint{2.339962in}{1.388543in}}{\pgfqpoint{2.339962in}{1.380307in}}%
\pgfpathcurveto{\pgfqpoint{2.339962in}{1.372071in}}{\pgfqpoint{2.343235in}{1.364171in}}{\pgfqpoint{2.349058in}{1.358347in}}%
\pgfpathcurveto{\pgfqpoint{2.354882in}{1.352523in}}{\pgfqpoint{2.362782in}{1.349251in}}{\pgfqpoint{2.371019in}{1.349251in}}%
\pgfpathclose%
\pgfusepath{stroke,fill}%
\end{pgfscope}%
\begin{pgfscope}%
\pgfpathrectangle{\pgfqpoint{0.894063in}{0.630000in}}{\pgfqpoint{6.713438in}{2.060556in}} %
\pgfusepath{clip}%
\pgfsetbuttcap%
\pgfsetroundjoin%
\definecolor{currentfill}{rgb}{0.000000,0.000000,1.000000}%
\pgfsetfillcolor{currentfill}%
\pgfsetlinewidth{1.003750pt}%
\definecolor{currentstroke}{rgb}{0.000000,0.000000,0.000000}%
\pgfsetstrokecolor{currentstroke}%
\pgfsetdash{}{0pt}%
\pgfpathmoveto{\pgfqpoint{3.982244in}{1.630040in}}%
\pgfpathcurveto{\pgfqpoint{3.990480in}{1.630040in}}{\pgfqpoint{3.998380in}{1.633312in}}{\pgfqpoint{4.004204in}{1.639136in}}%
\pgfpathcurveto{\pgfqpoint{4.010028in}{1.644960in}}{\pgfqpoint{4.013300in}{1.652860in}}{\pgfqpoint{4.013300in}{1.661096in}}%
\pgfpathcurveto{\pgfqpoint{4.013300in}{1.669332in}}{\pgfqpoint{4.010028in}{1.677232in}}{\pgfqpoint{4.004204in}{1.683056in}}%
\pgfpathcurveto{\pgfqpoint{3.998380in}{1.688880in}}{\pgfqpoint{3.990480in}{1.692153in}}{\pgfqpoint{3.982244in}{1.692153in}}%
\pgfpathcurveto{\pgfqpoint{3.974007in}{1.692153in}}{\pgfqpoint{3.966107in}{1.688880in}}{\pgfqpoint{3.960283in}{1.683056in}}%
\pgfpathcurveto{\pgfqpoint{3.954460in}{1.677232in}}{\pgfqpoint{3.951187in}{1.669332in}}{\pgfqpoint{3.951187in}{1.661096in}}%
\pgfpathcurveto{\pgfqpoint{3.951187in}{1.652860in}}{\pgfqpoint{3.954460in}{1.644960in}}{\pgfqpoint{3.960283in}{1.639136in}}%
\pgfpathcurveto{\pgfqpoint{3.966107in}{1.633312in}}{\pgfqpoint{3.974007in}{1.630040in}}{\pgfqpoint{3.982244in}{1.630040in}}%
\pgfpathclose%
\pgfusepath{stroke,fill}%
\end{pgfscope}%
\begin{pgfscope}%
\pgfpathrectangle{\pgfqpoint{0.894063in}{0.630000in}}{\pgfqpoint{6.713438in}{2.060556in}} %
\pgfusepath{clip}%
\pgfsetbuttcap%
\pgfsetroundjoin%
\definecolor{currentfill}{rgb}{0.000000,0.000000,1.000000}%
\pgfsetfillcolor{currentfill}%
\pgfsetlinewidth{1.003750pt}%
\definecolor{currentstroke}{rgb}{0.000000,0.000000,0.000000}%
\pgfsetstrokecolor{currentstroke}%
\pgfsetdash{}{0pt}%
\pgfpathmoveto{\pgfqpoint{4.653588in}{1.743494in}}%
\pgfpathcurveto{\pgfqpoint{4.661824in}{1.743494in}}{\pgfqpoint{4.669724in}{1.746766in}}{\pgfqpoint{4.675548in}{1.752590in}}%
\pgfpathcurveto{\pgfqpoint{4.681372in}{1.758414in}}{\pgfqpoint{4.684644in}{1.766314in}}{\pgfqpoint{4.684644in}{1.774550in}}%
\pgfpathcurveto{\pgfqpoint{4.684644in}{1.782787in}}{\pgfqpoint{4.681372in}{1.790687in}}{\pgfqpoint{4.675548in}{1.796511in}}%
\pgfpathcurveto{\pgfqpoint{4.669724in}{1.802334in}}{\pgfqpoint{4.661824in}{1.805607in}}{\pgfqpoint{4.653588in}{1.805607in}}%
\pgfpathcurveto{\pgfqpoint{4.645351in}{1.805607in}}{\pgfqpoint{4.637451in}{1.802334in}}{\pgfqpoint{4.631627in}{1.796511in}}%
\pgfpathcurveto{\pgfqpoint{4.625803in}{1.790687in}}{\pgfqpoint{4.622531in}{1.782787in}}{\pgfqpoint{4.622531in}{1.774550in}}%
\pgfpathcurveto{\pgfqpoint{4.622531in}{1.766314in}}{\pgfqpoint{4.625803in}{1.758414in}}{\pgfqpoint{4.631627in}{1.752590in}}%
\pgfpathcurveto{\pgfqpoint{4.637451in}{1.746766in}}{\pgfqpoint{4.645351in}{1.743494in}}{\pgfqpoint{4.653588in}{1.743494in}}%
\pgfpathclose%
\pgfusepath{stroke,fill}%
\end{pgfscope}%
\begin{pgfscope}%
\pgfpathrectangle{\pgfqpoint{0.894063in}{0.630000in}}{\pgfqpoint{6.713438in}{2.060556in}} %
\pgfusepath{clip}%
\pgfsetbuttcap%
\pgfsetroundjoin%
\definecolor{currentfill}{rgb}{0.000000,0.000000,1.000000}%
\pgfsetfillcolor{currentfill}%
\pgfsetlinewidth{1.003750pt}%
\definecolor{currentstroke}{rgb}{0.000000,0.000000,0.000000}%
\pgfsetstrokecolor{currentstroke}%
\pgfsetdash{}{0pt}%
\pgfpathmoveto{\pgfqpoint{3.713706in}{1.589770in}}%
\pgfpathcurveto{\pgfqpoint{3.721943in}{1.589770in}}{\pgfqpoint{3.729843in}{1.593043in}}{\pgfqpoint{3.735667in}{1.598867in}}%
\pgfpathcurveto{\pgfqpoint{3.741490in}{1.604691in}}{\pgfqpoint{3.744763in}{1.612591in}}{\pgfqpoint{3.744763in}{1.620827in}}%
\pgfpathcurveto{\pgfqpoint{3.744763in}{1.629063in}}{\pgfqpoint{3.741490in}{1.636963in}}{\pgfqpoint{3.735667in}{1.642787in}}%
\pgfpathcurveto{\pgfqpoint{3.729843in}{1.648611in}}{\pgfqpoint{3.721943in}{1.651883in}}{\pgfqpoint{3.713706in}{1.651883in}}%
\pgfpathcurveto{\pgfqpoint{3.705470in}{1.651883in}}{\pgfqpoint{3.697570in}{1.648611in}}{\pgfqpoint{3.691746in}{1.642787in}}%
\pgfpathcurveto{\pgfqpoint{3.685922in}{1.636963in}}{\pgfqpoint{3.682650in}{1.629063in}}{\pgfqpoint{3.682650in}{1.620827in}}%
\pgfpathcurveto{\pgfqpoint{3.682650in}{1.612591in}}{\pgfqpoint{3.685922in}{1.604691in}}{\pgfqpoint{3.691746in}{1.598867in}}%
\pgfpathcurveto{\pgfqpoint{3.697570in}{1.593043in}}{\pgfqpoint{3.705470in}{1.589770in}}{\pgfqpoint{3.713706in}{1.589770in}}%
\pgfpathclose%
\pgfusepath{stroke,fill}%
\end{pgfscope}%
\begin{pgfscope}%
\pgfpathrectangle{\pgfqpoint{0.894063in}{0.630000in}}{\pgfqpoint{6.713438in}{2.060556in}} %
\pgfusepath{clip}%
\pgfsetbuttcap%
\pgfsetroundjoin%
\definecolor{currentfill}{rgb}{0.000000,0.000000,1.000000}%
\pgfsetfillcolor{currentfill}%
\pgfsetlinewidth{1.003750pt}%
\definecolor{currentstroke}{rgb}{0.000000,0.000000,0.000000}%
\pgfsetstrokecolor{currentstroke}%
\pgfsetdash{}{0pt}%
\pgfpathmoveto{\pgfqpoint{2.236750in}{1.320568in}}%
\pgfpathcurveto{\pgfqpoint{2.244986in}{1.320568in}}{\pgfqpoint{2.252886in}{1.323840in}}{\pgfqpoint{2.258710in}{1.329664in}}%
\pgfpathcurveto{\pgfqpoint{2.264534in}{1.335488in}}{\pgfqpoint{2.267806in}{1.343388in}}{\pgfqpoint{2.267806in}{1.351624in}}%
\pgfpathcurveto{\pgfqpoint{2.267806in}{1.359860in}}{\pgfqpoint{2.264534in}{1.367761in}}{\pgfqpoint{2.258710in}{1.373584in}}%
\pgfpathcurveto{\pgfqpoint{2.252886in}{1.379408in}}{\pgfqpoint{2.244986in}{1.382681in}}{\pgfqpoint{2.236750in}{1.382681in}}%
\pgfpathcurveto{\pgfqpoint{2.228514in}{1.382681in}}{\pgfqpoint{2.220614in}{1.379408in}}{\pgfqpoint{2.214790in}{1.373584in}}%
\pgfpathcurveto{\pgfqpoint{2.208966in}{1.367761in}}{\pgfqpoint{2.205694in}{1.359860in}}{\pgfqpoint{2.205694in}{1.351624in}}%
\pgfpathcurveto{\pgfqpoint{2.205694in}{1.343388in}}{\pgfqpoint{2.208966in}{1.335488in}}{\pgfqpoint{2.214790in}{1.329664in}}%
\pgfpathcurveto{\pgfqpoint{2.220614in}{1.323840in}}{\pgfqpoint{2.228514in}{1.320568in}}{\pgfqpoint{2.236750in}{1.320568in}}%
\pgfpathclose%
\pgfusepath{stroke,fill}%
\end{pgfscope}%
\begin{pgfscope}%
\pgfsetrectcap%
\pgfsetmiterjoin%
\pgfsetlinewidth{1.003750pt}%
\definecolor{currentstroke}{rgb}{0.000000,0.000000,0.000000}%
\pgfsetstrokecolor{currentstroke}%
\pgfsetdash{}{0pt}%
\pgfpathmoveto{\pgfqpoint{0.894063in}{2.690556in}}%
\pgfpathlineto{\pgfqpoint{7.607500in}{2.690556in}}%
\pgfusepath{stroke}%
\end{pgfscope}%
\begin{pgfscope}%
\pgfsetrectcap%
\pgfsetmiterjoin%
\pgfsetlinewidth{1.003750pt}%
\definecolor{currentstroke}{rgb}{0.000000,0.000000,0.000000}%
\pgfsetstrokecolor{currentstroke}%
\pgfsetdash{}{0pt}%
\pgfpathmoveto{\pgfqpoint{7.607500in}{0.630000in}}%
\pgfpathlineto{\pgfqpoint{7.607500in}{2.690556in}}%
\pgfusepath{stroke}%
\end{pgfscope}%
\begin{pgfscope}%
\pgfsetrectcap%
\pgfsetmiterjoin%
\pgfsetlinewidth{1.003750pt}%
\definecolor{currentstroke}{rgb}{0.000000,0.000000,0.000000}%
\pgfsetstrokecolor{currentstroke}%
\pgfsetdash{}{0pt}%
\pgfpathmoveto{\pgfqpoint{0.894063in}{0.630000in}}%
\pgfpathlineto{\pgfqpoint{7.607500in}{0.630000in}}%
\pgfusepath{stroke}%
\end{pgfscope}%
\begin{pgfscope}%
\pgfsetrectcap%
\pgfsetmiterjoin%
\pgfsetlinewidth{1.003750pt}%
\definecolor{currentstroke}{rgb}{0.000000,0.000000,0.000000}%
\pgfsetstrokecolor{currentstroke}%
\pgfsetdash{}{0pt}%
\pgfpathmoveto{\pgfqpoint{0.894063in}{0.630000in}}%
\pgfpathlineto{\pgfqpoint{0.894063in}{2.690556in}}%
\pgfusepath{stroke}%
\end{pgfscope}%
\begin{pgfscope}%
\pgfsetbuttcap%
\pgfsetroundjoin%
\definecolor{currentfill}{rgb}{0.000000,0.000000,0.000000}%
\pgfsetfillcolor{currentfill}%
\pgfsetlinewidth{0.501875pt}%
\definecolor{currentstroke}{rgb}{0.000000,0.000000,0.000000}%
\pgfsetstrokecolor{currentstroke}%
\pgfsetdash{}{0pt}%
\pgfsys@defobject{currentmarker}{\pgfqpoint{0.000000in}{0.000000in}}{\pgfqpoint{0.000000in}{0.055556in}}{%
\pgfpathmoveto{\pgfqpoint{0.000000in}{0.000000in}}%
\pgfpathlineto{\pgfqpoint{0.000000in}{0.055556in}}%
\pgfusepath{stroke,fill}%
}%
\begin{pgfscope}%
\pgfsys@transformshift{0.894063in}{0.630000in}%
\pgfsys@useobject{currentmarker}{}%
\end{pgfscope}%
\end{pgfscope}%
\begin{pgfscope}%
\pgfsetbuttcap%
\pgfsetroundjoin%
\definecolor{currentfill}{rgb}{0.000000,0.000000,0.000000}%
\pgfsetfillcolor{currentfill}%
\pgfsetlinewidth{0.501875pt}%
\definecolor{currentstroke}{rgb}{0.000000,0.000000,0.000000}%
\pgfsetstrokecolor{currentstroke}%
\pgfsetdash{}{0pt}%
\pgfsys@defobject{currentmarker}{\pgfqpoint{0.000000in}{-0.055556in}}{\pgfqpoint{0.000000in}{0.000000in}}{%
\pgfpathmoveto{\pgfqpoint{0.000000in}{0.000000in}}%
\pgfpathlineto{\pgfqpoint{0.000000in}{-0.055556in}}%
\pgfusepath{stroke,fill}%
}%
\begin{pgfscope}%
\pgfsys@transformshift{0.894063in}{2.690556in}%
\pgfsys@useobject{currentmarker}{}%
\end{pgfscope}%
\end{pgfscope}%
\begin{pgfscope}%
\pgftext[x=0.894063in,y=0.574444in,,top]{\sffamily\fontsize{12.000000}{14.400000}\selectfont 0}%
\end{pgfscope}%
\begin{pgfscope}%
\pgfsetbuttcap%
\pgfsetroundjoin%
\definecolor{currentfill}{rgb}{0.000000,0.000000,0.000000}%
\pgfsetfillcolor{currentfill}%
\pgfsetlinewidth{0.501875pt}%
\definecolor{currentstroke}{rgb}{0.000000,0.000000,0.000000}%
\pgfsetstrokecolor{currentstroke}%
\pgfsetdash{}{0pt}%
\pgfsys@defobject{currentmarker}{\pgfqpoint{0.000000in}{0.000000in}}{\pgfqpoint{0.000000in}{0.055556in}}{%
\pgfpathmoveto{\pgfqpoint{0.000000in}{0.000000in}}%
\pgfpathlineto{\pgfqpoint{0.000000in}{0.055556in}}%
\pgfusepath{stroke,fill}%
}%
\begin{pgfscope}%
\pgfsys@transformshift{2.236750in}{0.630000in}%
\pgfsys@useobject{currentmarker}{}%
\end{pgfscope}%
\end{pgfscope}%
\begin{pgfscope}%
\pgfsetbuttcap%
\pgfsetroundjoin%
\definecolor{currentfill}{rgb}{0.000000,0.000000,0.000000}%
\pgfsetfillcolor{currentfill}%
\pgfsetlinewidth{0.501875pt}%
\definecolor{currentstroke}{rgb}{0.000000,0.000000,0.000000}%
\pgfsetstrokecolor{currentstroke}%
\pgfsetdash{}{0pt}%
\pgfsys@defobject{currentmarker}{\pgfqpoint{0.000000in}{-0.055556in}}{\pgfqpoint{0.000000in}{0.000000in}}{%
\pgfpathmoveto{\pgfqpoint{0.000000in}{0.000000in}}%
\pgfpathlineto{\pgfqpoint{0.000000in}{-0.055556in}}%
\pgfusepath{stroke,fill}%
}%
\begin{pgfscope}%
\pgfsys@transformshift{2.236750in}{2.690556in}%
\pgfsys@useobject{currentmarker}{}%
\end{pgfscope}%
\end{pgfscope}%
\begin{pgfscope}%
\pgftext[x=2.236750in,y=0.574444in,,top]{\sffamily\fontsize{12.000000}{14.400000}\selectfont 1000}%
\end{pgfscope}%
\begin{pgfscope}%
\pgfsetbuttcap%
\pgfsetroundjoin%
\definecolor{currentfill}{rgb}{0.000000,0.000000,0.000000}%
\pgfsetfillcolor{currentfill}%
\pgfsetlinewidth{0.501875pt}%
\definecolor{currentstroke}{rgb}{0.000000,0.000000,0.000000}%
\pgfsetstrokecolor{currentstroke}%
\pgfsetdash{}{0pt}%
\pgfsys@defobject{currentmarker}{\pgfqpoint{0.000000in}{0.000000in}}{\pgfqpoint{0.000000in}{0.055556in}}{%
\pgfpathmoveto{\pgfqpoint{0.000000in}{0.000000in}}%
\pgfpathlineto{\pgfqpoint{0.000000in}{0.055556in}}%
\pgfusepath{stroke,fill}%
}%
\begin{pgfscope}%
\pgfsys@transformshift{3.579438in}{0.630000in}%
\pgfsys@useobject{currentmarker}{}%
\end{pgfscope}%
\end{pgfscope}%
\begin{pgfscope}%
\pgfsetbuttcap%
\pgfsetroundjoin%
\definecolor{currentfill}{rgb}{0.000000,0.000000,0.000000}%
\pgfsetfillcolor{currentfill}%
\pgfsetlinewidth{0.501875pt}%
\definecolor{currentstroke}{rgb}{0.000000,0.000000,0.000000}%
\pgfsetstrokecolor{currentstroke}%
\pgfsetdash{}{0pt}%
\pgfsys@defobject{currentmarker}{\pgfqpoint{0.000000in}{-0.055556in}}{\pgfqpoint{0.000000in}{0.000000in}}{%
\pgfpathmoveto{\pgfqpoint{0.000000in}{0.000000in}}%
\pgfpathlineto{\pgfqpoint{0.000000in}{-0.055556in}}%
\pgfusepath{stroke,fill}%
}%
\begin{pgfscope}%
\pgfsys@transformshift{3.579438in}{2.690556in}%
\pgfsys@useobject{currentmarker}{}%
\end{pgfscope}%
\end{pgfscope}%
\begin{pgfscope}%
\pgftext[x=3.579438in,y=0.574444in,,top]{\sffamily\fontsize{12.000000}{14.400000}\selectfont 2000}%
\end{pgfscope}%
\begin{pgfscope}%
\pgfsetbuttcap%
\pgfsetroundjoin%
\definecolor{currentfill}{rgb}{0.000000,0.000000,0.000000}%
\pgfsetfillcolor{currentfill}%
\pgfsetlinewidth{0.501875pt}%
\definecolor{currentstroke}{rgb}{0.000000,0.000000,0.000000}%
\pgfsetstrokecolor{currentstroke}%
\pgfsetdash{}{0pt}%
\pgfsys@defobject{currentmarker}{\pgfqpoint{0.000000in}{0.000000in}}{\pgfqpoint{0.000000in}{0.055556in}}{%
\pgfpathmoveto{\pgfqpoint{0.000000in}{0.000000in}}%
\pgfpathlineto{\pgfqpoint{0.000000in}{0.055556in}}%
\pgfusepath{stroke,fill}%
}%
\begin{pgfscope}%
\pgfsys@transformshift{4.922125in}{0.630000in}%
\pgfsys@useobject{currentmarker}{}%
\end{pgfscope}%
\end{pgfscope}%
\begin{pgfscope}%
\pgfsetbuttcap%
\pgfsetroundjoin%
\definecolor{currentfill}{rgb}{0.000000,0.000000,0.000000}%
\pgfsetfillcolor{currentfill}%
\pgfsetlinewidth{0.501875pt}%
\definecolor{currentstroke}{rgb}{0.000000,0.000000,0.000000}%
\pgfsetstrokecolor{currentstroke}%
\pgfsetdash{}{0pt}%
\pgfsys@defobject{currentmarker}{\pgfqpoint{0.000000in}{-0.055556in}}{\pgfqpoint{0.000000in}{0.000000in}}{%
\pgfpathmoveto{\pgfqpoint{0.000000in}{0.000000in}}%
\pgfpathlineto{\pgfqpoint{0.000000in}{-0.055556in}}%
\pgfusepath{stroke,fill}%
}%
\begin{pgfscope}%
\pgfsys@transformshift{4.922125in}{2.690556in}%
\pgfsys@useobject{currentmarker}{}%
\end{pgfscope}%
\end{pgfscope}%
\begin{pgfscope}%
\pgftext[x=4.922125in,y=0.574444in,,top]{\sffamily\fontsize{12.000000}{14.400000}\selectfont 3000}%
\end{pgfscope}%
\begin{pgfscope}%
\pgfsetbuttcap%
\pgfsetroundjoin%
\definecolor{currentfill}{rgb}{0.000000,0.000000,0.000000}%
\pgfsetfillcolor{currentfill}%
\pgfsetlinewidth{0.501875pt}%
\definecolor{currentstroke}{rgb}{0.000000,0.000000,0.000000}%
\pgfsetstrokecolor{currentstroke}%
\pgfsetdash{}{0pt}%
\pgfsys@defobject{currentmarker}{\pgfqpoint{0.000000in}{0.000000in}}{\pgfqpoint{0.000000in}{0.055556in}}{%
\pgfpathmoveto{\pgfqpoint{0.000000in}{0.000000in}}%
\pgfpathlineto{\pgfqpoint{0.000000in}{0.055556in}}%
\pgfusepath{stroke,fill}%
}%
\begin{pgfscope}%
\pgfsys@transformshift{6.264813in}{0.630000in}%
\pgfsys@useobject{currentmarker}{}%
\end{pgfscope}%
\end{pgfscope}%
\begin{pgfscope}%
\pgfsetbuttcap%
\pgfsetroundjoin%
\definecolor{currentfill}{rgb}{0.000000,0.000000,0.000000}%
\pgfsetfillcolor{currentfill}%
\pgfsetlinewidth{0.501875pt}%
\definecolor{currentstroke}{rgb}{0.000000,0.000000,0.000000}%
\pgfsetstrokecolor{currentstroke}%
\pgfsetdash{}{0pt}%
\pgfsys@defobject{currentmarker}{\pgfqpoint{0.000000in}{-0.055556in}}{\pgfqpoint{0.000000in}{0.000000in}}{%
\pgfpathmoveto{\pgfqpoint{0.000000in}{0.000000in}}%
\pgfpathlineto{\pgfqpoint{0.000000in}{-0.055556in}}%
\pgfusepath{stroke,fill}%
}%
\begin{pgfscope}%
\pgfsys@transformshift{6.264813in}{2.690556in}%
\pgfsys@useobject{currentmarker}{}%
\end{pgfscope}%
\end{pgfscope}%
\begin{pgfscope}%
\pgftext[x=6.264813in,y=0.574444in,,top]{\sffamily\fontsize{12.000000}{14.400000}\selectfont 4000}%
\end{pgfscope}%
\begin{pgfscope}%
\pgfsetbuttcap%
\pgfsetroundjoin%
\definecolor{currentfill}{rgb}{0.000000,0.000000,0.000000}%
\pgfsetfillcolor{currentfill}%
\pgfsetlinewidth{0.501875pt}%
\definecolor{currentstroke}{rgb}{0.000000,0.000000,0.000000}%
\pgfsetstrokecolor{currentstroke}%
\pgfsetdash{}{0pt}%
\pgfsys@defobject{currentmarker}{\pgfqpoint{0.000000in}{0.000000in}}{\pgfqpoint{0.000000in}{0.055556in}}{%
\pgfpathmoveto{\pgfqpoint{0.000000in}{0.000000in}}%
\pgfpathlineto{\pgfqpoint{0.000000in}{0.055556in}}%
\pgfusepath{stroke,fill}%
}%
\begin{pgfscope}%
\pgfsys@transformshift{7.607500in}{0.630000in}%
\pgfsys@useobject{currentmarker}{}%
\end{pgfscope}%
\end{pgfscope}%
\begin{pgfscope}%
\pgfsetbuttcap%
\pgfsetroundjoin%
\definecolor{currentfill}{rgb}{0.000000,0.000000,0.000000}%
\pgfsetfillcolor{currentfill}%
\pgfsetlinewidth{0.501875pt}%
\definecolor{currentstroke}{rgb}{0.000000,0.000000,0.000000}%
\pgfsetstrokecolor{currentstroke}%
\pgfsetdash{}{0pt}%
\pgfsys@defobject{currentmarker}{\pgfqpoint{0.000000in}{-0.055556in}}{\pgfqpoint{0.000000in}{0.000000in}}{%
\pgfpathmoveto{\pgfqpoint{0.000000in}{0.000000in}}%
\pgfpathlineto{\pgfqpoint{0.000000in}{-0.055556in}}%
\pgfusepath{stroke,fill}%
}%
\begin{pgfscope}%
\pgfsys@transformshift{7.607500in}{2.690556in}%
\pgfsys@useobject{currentmarker}{}%
\end{pgfscope}%
\end{pgfscope}%
\begin{pgfscope}%
\pgftext[x=7.607500in,y=0.574444in,,top]{\sffamily\fontsize{12.000000}{14.400000}\selectfont 5000}%
\end{pgfscope}%
\begin{pgfscope}%
\pgftext[x=4.250781in,y=0.343705in,,top]{\sffamily\fontsize{12.000000}{14.400000}\selectfont Requests Served}%
\end{pgfscope}%
\begin{pgfscope}%
\pgfsetbuttcap%
\pgfsetroundjoin%
\definecolor{currentfill}{rgb}{0.000000,0.000000,0.000000}%
\pgfsetfillcolor{currentfill}%
\pgfsetlinewidth{0.501875pt}%
\definecolor{currentstroke}{rgb}{0.000000,0.000000,0.000000}%
\pgfsetstrokecolor{currentstroke}%
\pgfsetdash{}{0pt}%
\pgfsys@defobject{currentmarker}{\pgfqpoint{0.000000in}{0.000000in}}{\pgfqpoint{0.055556in}{0.000000in}}{%
\pgfpathmoveto{\pgfqpoint{0.000000in}{0.000000in}}%
\pgfpathlineto{\pgfqpoint{0.055556in}{0.000000in}}%
\pgfusepath{stroke,fill}%
}%
\begin{pgfscope}%
\pgfsys@transformshift{0.894063in}{0.630000in}%
\pgfsys@useobject{currentmarker}{}%
\end{pgfscope}%
\end{pgfscope}%
\begin{pgfscope}%
\pgfsetbuttcap%
\pgfsetroundjoin%
\definecolor{currentfill}{rgb}{0.000000,0.000000,0.000000}%
\pgfsetfillcolor{currentfill}%
\pgfsetlinewidth{0.501875pt}%
\definecolor{currentstroke}{rgb}{0.000000,0.000000,0.000000}%
\pgfsetstrokecolor{currentstroke}%
\pgfsetdash{}{0pt}%
\pgfsys@defobject{currentmarker}{\pgfqpoint{-0.055556in}{0.000000in}}{\pgfqpoint{0.000000in}{0.000000in}}{%
\pgfpathmoveto{\pgfqpoint{0.000000in}{0.000000in}}%
\pgfpathlineto{\pgfqpoint{-0.055556in}{0.000000in}}%
\pgfusepath{stroke,fill}%
}%
\begin{pgfscope}%
\pgfsys@transformshift{7.607500in}{0.630000in}%
\pgfsys@useobject{currentmarker}{}%
\end{pgfscope}%
\end{pgfscope}%
\begin{pgfscope}%
\pgftext[x=0.838507in,y=0.630000in,right,]{\sffamily\fontsize{12.000000}{14.400000}\selectfont 0}%
\end{pgfscope}%
\begin{pgfscope}%
\pgfsetbuttcap%
\pgfsetroundjoin%
\definecolor{currentfill}{rgb}{0.000000,0.000000,0.000000}%
\pgfsetfillcolor{currentfill}%
\pgfsetlinewidth{0.501875pt}%
\definecolor{currentstroke}{rgb}{0.000000,0.000000,0.000000}%
\pgfsetstrokecolor{currentstroke}%
\pgfsetdash{}{0pt}%
\pgfsys@defobject{currentmarker}{\pgfqpoint{0.000000in}{0.000000in}}{\pgfqpoint{0.055556in}{0.000000in}}{%
\pgfpathmoveto{\pgfqpoint{0.000000in}{0.000000in}}%
\pgfpathlineto{\pgfqpoint{0.055556in}{0.000000in}}%
\pgfusepath{stroke,fill}%
}%
\begin{pgfscope}%
\pgfsys@transformshift{0.894063in}{0.924365in}%
\pgfsys@useobject{currentmarker}{}%
\end{pgfscope}%
\end{pgfscope}%
\begin{pgfscope}%
\pgfsetbuttcap%
\pgfsetroundjoin%
\definecolor{currentfill}{rgb}{0.000000,0.000000,0.000000}%
\pgfsetfillcolor{currentfill}%
\pgfsetlinewidth{0.501875pt}%
\definecolor{currentstroke}{rgb}{0.000000,0.000000,0.000000}%
\pgfsetstrokecolor{currentstroke}%
\pgfsetdash{}{0pt}%
\pgfsys@defobject{currentmarker}{\pgfqpoint{-0.055556in}{0.000000in}}{\pgfqpoint{0.000000in}{0.000000in}}{%
\pgfpathmoveto{\pgfqpoint{0.000000in}{0.000000in}}%
\pgfpathlineto{\pgfqpoint{-0.055556in}{0.000000in}}%
\pgfusepath{stroke,fill}%
}%
\begin{pgfscope}%
\pgfsys@transformshift{7.607500in}{0.924365in}%
\pgfsys@useobject{currentmarker}{}%
\end{pgfscope}%
\end{pgfscope}%
\begin{pgfscope}%
\pgftext[x=0.838507in,y=0.924365in,right,]{\sffamily\fontsize{12.000000}{14.400000}\selectfont 200}%
\end{pgfscope}%
\begin{pgfscope}%
\pgfsetbuttcap%
\pgfsetroundjoin%
\definecolor{currentfill}{rgb}{0.000000,0.000000,0.000000}%
\pgfsetfillcolor{currentfill}%
\pgfsetlinewidth{0.501875pt}%
\definecolor{currentstroke}{rgb}{0.000000,0.000000,0.000000}%
\pgfsetstrokecolor{currentstroke}%
\pgfsetdash{}{0pt}%
\pgfsys@defobject{currentmarker}{\pgfqpoint{0.000000in}{0.000000in}}{\pgfqpoint{0.055556in}{0.000000in}}{%
\pgfpathmoveto{\pgfqpoint{0.000000in}{0.000000in}}%
\pgfpathlineto{\pgfqpoint{0.055556in}{0.000000in}}%
\pgfusepath{stroke,fill}%
}%
\begin{pgfscope}%
\pgfsys@transformshift{0.894063in}{1.218730in}%
\pgfsys@useobject{currentmarker}{}%
\end{pgfscope}%
\end{pgfscope}%
\begin{pgfscope}%
\pgfsetbuttcap%
\pgfsetroundjoin%
\definecolor{currentfill}{rgb}{0.000000,0.000000,0.000000}%
\pgfsetfillcolor{currentfill}%
\pgfsetlinewidth{0.501875pt}%
\definecolor{currentstroke}{rgb}{0.000000,0.000000,0.000000}%
\pgfsetstrokecolor{currentstroke}%
\pgfsetdash{}{0pt}%
\pgfsys@defobject{currentmarker}{\pgfqpoint{-0.055556in}{0.000000in}}{\pgfqpoint{0.000000in}{0.000000in}}{%
\pgfpathmoveto{\pgfqpoint{0.000000in}{0.000000in}}%
\pgfpathlineto{\pgfqpoint{-0.055556in}{0.000000in}}%
\pgfusepath{stroke,fill}%
}%
\begin{pgfscope}%
\pgfsys@transformshift{7.607500in}{1.218730in}%
\pgfsys@useobject{currentmarker}{}%
\end{pgfscope}%
\end{pgfscope}%
\begin{pgfscope}%
\pgftext[x=0.838507in,y=1.218730in,right,]{\sffamily\fontsize{12.000000}{14.400000}\selectfont 400}%
\end{pgfscope}%
\begin{pgfscope}%
\pgfsetbuttcap%
\pgfsetroundjoin%
\definecolor{currentfill}{rgb}{0.000000,0.000000,0.000000}%
\pgfsetfillcolor{currentfill}%
\pgfsetlinewidth{0.501875pt}%
\definecolor{currentstroke}{rgb}{0.000000,0.000000,0.000000}%
\pgfsetstrokecolor{currentstroke}%
\pgfsetdash{}{0pt}%
\pgfsys@defobject{currentmarker}{\pgfqpoint{0.000000in}{0.000000in}}{\pgfqpoint{0.055556in}{0.000000in}}{%
\pgfpathmoveto{\pgfqpoint{0.000000in}{0.000000in}}%
\pgfpathlineto{\pgfqpoint{0.055556in}{0.000000in}}%
\pgfusepath{stroke,fill}%
}%
\begin{pgfscope}%
\pgfsys@transformshift{0.894063in}{1.513095in}%
\pgfsys@useobject{currentmarker}{}%
\end{pgfscope}%
\end{pgfscope}%
\begin{pgfscope}%
\pgfsetbuttcap%
\pgfsetroundjoin%
\definecolor{currentfill}{rgb}{0.000000,0.000000,0.000000}%
\pgfsetfillcolor{currentfill}%
\pgfsetlinewidth{0.501875pt}%
\definecolor{currentstroke}{rgb}{0.000000,0.000000,0.000000}%
\pgfsetstrokecolor{currentstroke}%
\pgfsetdash{}{0pt}%
\pgfsys@defobject{currentmarker}{\pgfqpoint{-0.055556in}{0.000000in}}{\pgfqpoint{0.000000in}{0.000000in}}{%
\pgfpathmoveto{\pgfqpoint{0.000000in}{0.000000in}}%
\pgfpathlineto{\pgfqpoint{-0.055556in}{0.000000in}}%
\pgfusepath{stroke,fill}%
}%
\begin{pgfscope}%
\pgfsys@transformshift{7.607500in}{1.513095in}%
\pgfsys@useobject{currentmarker}{}%
\end{pgfscope}%
\end{pgfscope}%
\begin{pgfscope}%
\pgftext[x=0.838507in,y=1.513095in,right,]{\sffamily\fontsize{12.000000}{14.400000}\selectfont 600}%
\end{pgfscope}%
\begin{pgfscope}%
\pgfsetbuttcap%
\pgfsetroundjoin%
\definecolor{currentfill}{rgb}{0.000000,0.000000,0.000000}%
\pgfsetfillcolor{currentfill}%
\pgfsetlinewidth{0.501875pt}%
\definecolor{currentstroke}{rgb}{0.000000,0.000000,0.000000}%
\pgfsetstrokecolor{currentstroke}%
\pgfsetdash{}{0pt}%
\pgfsys@defobject{currentmarker}{\pgfqpoint{0.000000in}{0.000000in}}{\pgfqpoint{0.055556in}{0.000000in}}{%
\pgfpathmoveto{\pgfqpoint{0.000000in}{0.000000in}}%
\pgfpathlineto{\pgfqpoint{0.055556in}{0.000000in}}%
\pgfusepath{stroke,fill}%
}%
\begin{pgfscope}%
\pgfsys@transformshift{0.894063in}{1.807460in}%
\pgfsys@useobject{currentmarker}{}%
\end{pgfscope}%
\end{pgfscope}%
\begin{pgfscope}%
\pgfsetbuttcap%
\pgfsetroundjoin%
\definecolor{currentfill}{rgb}{0.000000,0.000000,0.000000}%
\pgfsetfillcolor{currentfill}%
\pgfsetlinewidth{0.501875pt}%
\definecolor{currentstroke}{rgb}{0.000000,0.000000,0.000000}%
\pgfsetstrokecolor{currentstroke}%
\pgfsetdash{}{0pt}%
\pgfsys@defobject{currentmarker}{\pgfqpoint{-0.055556in}{0.000000in}}{\pgfqpoint{0.000000in}{0.000000in}}{%
\pgfpathmoveto{\pgfqpoint{0.000000in}{0.000000in}}%
\pgfpathlineto{\pgfqpoint{-0.055556in}{0.000000in}}%
\pgfusepath{stroke,fill}%
}%
\begin{pgfscope}%
\pgfsys@transformshift{7.607500in}{1.807460in}%
\pgfsys@useobject{currentmarker}{}%
\end{pgfscope}%
\end{pgfscope}%
\begin{pgfscope}%
\pgftext[x=0.838507in,y=1.807460in,right,]{\sffamily\fontsize{12.000000}{14.400000}\selectfont 800}%
\end{pgfscope}%
\begin{pgfscope}%
\pgfsetbuttcap%
\pgfsetroundjoin%
\definecolor{currentfill}{rgb}{0.000000,0.000000,0.000000}%
\pgfsetfillcolor{currentfill}%
\pgfsetlinewidth{0.501875pt}%
\definecolor{currentstroke}{rgb}{0.000000,0.000000,0.000000}%
\pgfsetstrokecolor{currentstroke}%
\pgfsetdash{}{0pt}%
\pgfsys@defobject{currentmarker}{\pgfqpoint{0.000000in}{0.000000in}}{\pgfqpoint{0.055556in}{0.000000in}}{%
\pgfpathmoveto{\pgfqpoint{0.000000in}{0.000000in}}%
\pgfpathlineto{\pgfqpoint{0.055556in}{0.000000in}}%
\pgfusepath{stroke,fill}%
}%
\begin{pgfscope}%
\pgfsys@transformshift{0.894063in}{2.101825in}%
\pgfsys@useobject{currentmarker}{}%
\end{pgfscope}%
\end{pgfscope}%
\begin{pgfscope}%
\pgfsetbuttcap%
\pgfsetroundjoin%
\definecolor{currentfill}{rgb}{0.000000,0.000000,0.000000}%
\pgfsetfillcolor{currentfill}%
\pgfsetlinewidth{0.501875pt}%
\definecolor{currentstroke}{rgb}{0.000000,0.000000,0.000000}%
\pgfsetstrokecolor{currentstroke}%
\pgfsetdash{}{0pt}%
\pgfsys@defobject{currentmarker}{\pgfqpoint{-0.055556in}{0.000000in}}{\pgfqpoint{0.000000in}{0.000000in}}{%
\pgfpathmoveto{\pgfqpoint{0.000000in}{0.000000in}}%
\pgfpathlineto{\pgfqpoint{-0.055556in}{0.000000in}}%
\pgfusepath{stroke,fill}%
}%
\begin{pgfscope}%
\pgfsys@transformshift{7.607500in}{2.101825in}%
\pgfsys@useobject{currentmarker}{}%
\end{pgfscope}%
\end{pgfscope}%
\begin{pgfscope}%
\pgftext[x=0.838507in,y=2.101825in,right,]{\sffamily\fontsize{12.000000}{14.400000}\selectfont 1000}%
\end{pgfscope}%
\begin{pgfscope}%
\pgfsetbuttcap%
\pgfsetroundjoin%
\definecolor{currentfill}{rgb}{0.000000,0.000000,0.000000}%
\pgfsetfillcolor{currentfill}%
\pgfsetlinewidth{0.501875pt}%
\definecolor{currentstroke}{rgb}{0.000000,0.000000,0.000000}%
\pgfsetstrokecolor{currentstroke}%
\pgfsetdash{}{0pt}%
\pgfsys@defobject{currentmarker}{\pgfqpoint{0.000000in}{0.000000in}}{\pgfqpoint{0.055556in}{0.000000in}}{%
\pgfpathmoveto{\pgfqpoint{0.000000in}{0.000000in}}%
\pgfpathlineto{\pgfqpoint{0.055556in}{0.000000in}}%
\pgfusepath{stroke,fill}%
}%
\begin{pgfscope}%
\pgfsys@transformshift{0.894063in}{2.396190in}%
\pgfsys@useobject{currentmarker}{}%
\end{pgfscope}%
\end{pgfscope}%
\begin{pgfscope}%
\pgfsetbuttcap%
\pgfsetroundjoin%
\definecolor{currentfill}{rgb}{0.000000,0.000000,0.000000}%
\pgfsetfillcolor{currentfill}%
\pgfsetlinewidth{0.501875pt}%
\definecolor{currentstroke}{rgb}{0.000000,0.000000,0.000000}%
\pgfsetstrokecolor{currentstroke}%
\pgfsetdash{}{0pt}%
\pgfsys@defobject{currentmarker}{\pgfqpoint{-0.055556in}{0.000000in}}{\pgfqpoint{0.000000in}{0.000000in}}{%
\pgfpathmoveto{\pgfqpoint{0.000000in}{0.000000in}}%
\pgfpathlineto{\pgfqpoint{-0.055556in}{0.000000in}}%
\pgfusepath{stroke,fill}%
}%
\begin{pgfscope}%
\pgfsys@transformshift{7.607500in}{2.396190in}%
\pgfsys@useobject{currentmarker}{}%
\end{pgfscope}%
\end{pgfscope}%
\begin{pgfscope}%
\pgftext[x=0.838507in,y=2.396190in,right,]{\sffamily\fontsize{12.000000}{14.400000}\selectfont 1200}%
\end{pgfscope}%
\begin{pgfscope}%
\pgfsetbuttcap%
\pgfsetroundjoin%
\definecolor{currentfill}{rgb}{0.000000,0.000000,0.000000}%
\pgfsetfillcolor{currentfill}%
\pgfsetlinewidth{0.501875pt}%
\definecolor{currentstroke}{rgb}{0.000000,0.000000,0.000000}%
\pgfsetstrokecolor{currentstroke}%
\pgfsetdash{}{0pt}%
\pgfsys@defobject{currentmarker}{\pgfqpoint{0.000000in}{0.000000in}}{\pgfqpoint{0.055556in}{0.000000in}}{%
\pgfpathmoveto{\pgfqpoint{0.000000in}{0.000000in}}%
\pgfpathlineto{\pgfqpoint{0.055556in}{0.000000in}}%
\pgfusepath{stroke,fill}%
}%
\begin{pgfscope}%
\pgfsys@transformshift{0.894063in}{2.690556in}%
\pgfsys@useobject{currentmarker}{}%
\end{pgfscope}%
\end{pgfscope}%
\begin{pgfscope}%
\pgfsetbuttcap%
\pgfsetroundjoin%
\definecolor{currentfill}{rgb}{0.000000,0.000000,0.000000}%
\pgfsetfillcolor{currentfill}%
\pgfsetlinewidth{0.501875pt}%
\definecolor{currentstroke}{rgb}{0.000000,0.000000,0.000000}%
\pgfsetstrokecolor{currentstroke}%
\pgfsetdash{}{0pt}%
\pgfsys@defobject{currentmarker}{\pgfqpoint{-0.055556in}{0.000000in}}{\pgfqpoint{0.000000in}{0.000000in}}{%
\pgfpathmoveto{\pgfqpoint{0.000000in}{0.000000in}}%
\pgfpathlineto{\pgfqpoint{-0.055556in}{0.000000in}}%
\pgfusepath{stroke,fill}%
}%
\begin{pgfscope}%
\pgfsys@transformshift{7.607500in}{2.690556in}%
\pgfsys@useobject{currentmarker}{}%
\end{pgfscope}%
\end{pgfscope}%
\begin{pgfscope}%
\pgftext[x=0.838507in,y=2.690556in,right,]{\sffamily\fontsize{12.000000}{14.400000}\selectfont 1400}%
\end{pgfscope}%
\begin{pgfscope}%
\pgftext[x=0.344909in,y=1.660278in,,bottom,rotate=90.000000]{\sffamily\fontsize{12.000000}{14.400000}\selectfont Memory (Mb)}%
\end{pgfscope}%
\begin{pgfscope}%
\pgftext[x=4.250781in,y=2.760000in,,base]{\sffamily\fontsize{14.400000}{17.280000}\selectfont Nonshared memory}%
\end{pgfscope}%
\end{pgfpicture}%
\makeatother%
\endgroup%
}
    \caption{Memory usage of standard Garbage Collector running 4 workers after $X$ requests.  Each dot represents a worker's memory consumption after $X$ requests.}\label{fig:standard_mem}
\end{figure}

\subsection{Latency}

The table below shows the time per request for our deployment with and without the optimization.  Figure \ref{fig:latency} shows the same information in a graphical format.  We can see that for a small amount of traffic, the standard minimark GC responds to more requests per second than our patched version.  However, once the number of requests per second exceeded around 13k, the total memory of the system exceeded the 8GB allotted and the deployment started paging aggressively.  On the other hand the patched GC continues to settles down for larger numbers of requests.  The upward trend in response rate is a result of PyPy's tracing Just-in-Time compilation - over the course of a program's lifetime PyPy will compile commonly executed codepaths.  When the application is freshly started, PyPy interprets the source code and spends time compiling codepaths.  After the application has been running for a while, PyPy spends less time interpreting and compiling and more time executing native code \cite{pypy-doc}

\begin{center}
\begin{tabular}{| l || l | l |}
    \hline
     & \textbf{Standard} & \textbf{Patched} \\
    \hline
    5k Requests & 190.8 conn/sec & 67.9 conn/sec \\
    10k Requests & 269.2 conn/sec & 148.7 conn/sec \\
    15k Requests & 97.7 conn/sec & 188.8 conn/sec \\
    20k Requests & 103.2 conn/sec & 190.9 conn/sec \\
    \hline
\end{tabular}
\end{center}

From this data we can see that there is time overhead in our optimization.  If we had allocated more than 8GB of memory to the test, the Standard GC process would likely not have paged and would have responded to more requests per second than the patched version.  However the patched version used less memory overall and thus in our setup did not spend any time paging.  

\begin{figure}
    \resizebox{0.5\textwidth}{!}{%% Creator: Matplotlib, PGF backend
%%
%% To include the figure in your LaTeX document, write
%%   \input{<filename>.pgf}
%%
%% Make sure the required packages are loaded in your preamble
%%   \usepackage{pgf}
%%
%% Figures using additional raster images can only be included by \input if
%% they are in the same directory as the main LaTeX file. For loading figures
%% from other directories you can use the `import` package
%%   \usepackage{import}
%% and then include the figures with
%%   \import{<path to file>}{<filename>.pgf}
%%
%% Matplotlib used the following preamble
%%   \usepackage{fontspec}
%%   \setmainfont{Bitstream Vera Serif}
%%   \setsansfont{Bitstream Vera Sans}
%%   \setmonofont{Bitstream Vera Sans Mono}
%%
\begingroup%
\makeatletter%
\begin{pgfpicture}%
\pgfpathrectangle{\pgfpointorigin}{\pgfqpoint{8.000000in}{6.000000in}}%
\pgfusepath{use as bounding box, clip}%
\begin{pgfscope}%
\pgfsetbuttcap%
\pgfsetmiterjoin%
\definecolor{currentfill}{rgb}{1.000000,1.000000,1.000000}%
\pgfsetfillcolor{currentfill}%
\pgfsetlinewidth{0.000000pt}%
\definecolor{currentstroke}{rgb}{1.000000,1.000000,1.000000}%
\pgfsetstrokecolor{currentstroke}%
\pgfsetdash{}{0pt}%
\pgfpathmoveto{\pgfqpoint{0.000000in}{0.000000in}}%
\pgfpathlineto{\pgfqpoint{8.000000in}{0.000000in}}%
\pgfpathlineto{\pgfqpoint{8.000000in}{6.000000in}}%
\pgfpathlineto{\pgfqpoint{0.000000in}{6.000000in}}%
\pgfpathclose%
\pgfusepath{fill}%
\end{pgfscope}%
\begin{pgfscope}%
\pgfsetbuttcap%
\pgfsetmiterjoin%
\definecolor{currentfill}{rgb}{1.000000,1.000000,1.000000}%
\pgfsetfillcolor{currentfill}%
\pgfsetlinewidth{0.000000pt}%
\definecolor{currentstroke}{rgb}{0.000000,0.000000,0.000000}%
\pgfsetstrokecolor{currentstroke}%
\pgfsetstrokeopacity{0.000000}%
\pgfsetdash{}{0pt}%
\pgfpathmoveto{\pgfqpoint{0.787813in}{0.630000in}}%
\pgfpathlineto{\pgfqpoint{7.820000in}{0.630000in}}%
\pgfpathlineto{\pgfqpoint{7.820000in}{5.600556in}}%
\pgfpathlineto{\pgfqpoint{0.787813in}{5.600556in}}%
\pgfpathclose%
\pgfusepath{fill}%
\end{pgfscope}%
\begin{pgfscope}%
\pgfpathrectangle{\pgfqpoint{0.787813in}{0.630000in}}{\pgfqpoint{7.032188in}{4.970556in}} %
\pgfusepath{clip}%
\pgfsetbuttcap%
\pgfsetmiterjoin%
\definecolor{currentfill}{rgb}{1.000000,0.000000,0.000000}%
\pgfsetfillcolor{currentfill}%
\pgfsetlinewidth{1.003750pt}%
\definecolor{currentstroke}{rgb}{0.000000,0.000000,0.000000}%
\pgfsetstrokecolor{currentstroke}%
\pgfsetdash{}{0pt}%
\pgfpathmoveto{\pgfqpoint{1.051520in}{0.630000in}}%
\pgfpathlineto{\pgfqpoint{1.666836in}{0.630000in}}%
\pgfpathlineto{\pgfqpoint{1.666836in}{3.791273in}}%
\pgfpathlineto{\pgfqpoint{1.051520in}{3.791273in}}%
\pgfpathclose%
\pgfusepath{stroke,fill}%
\end{pgfscope}%
\begin{pgfscope}%
\pgfpathrectangle{\pgfqpoint{0.787813in}{0.630000in}}{\pgfqpoint{7.032188in}{4.970556in}} %
\pgfusepath{clip}%
\pgfsetbuttcap%
\pgfsetmiterjoin%
\definecolor{currentfill}{rgb}{1.000000,0.000000,0.000000}%
\pgfsetfillcolor{currentfill}%
\pgfsetlinewidth{1.003750pt}%
\definecolor{currentstroke}{rgb}{0.000000,0.000000,0.000000}%
\pgfsetstrokecolor{currentstroke}%
\pgfsetdash{}{0pt}%
\pgfpathmoveto{\pgfqpoint{2.809566in}{0.630000in}}%
\pgfpathlineto{\pgfqpoint{3.424883in}{0.630000in}}%
\pgfpathlineto{\pgfqpoint{3.424883in}{5.090245in}}%
\pgfpathlineto{\pgfqpoint{2.809566in}{5.090245in}}%
\pgfpathclose%
\pgfusepath{stroke,fill}%
\end{pgfscope}%
\begin{pgfscope}%
\pgfpathrectangle{\pgfqpoint{0.787813in}{0.630000in}}{\pgfqpoint{7.032188in}{4.970556in}} %
\pgfusepath{clip}%
\pgfsetbuttcap%
\pgfsetmiterjoin%
\definecolor{currentfill}{rgb}{1.000000,0.000000,0.000000}%
\pgfsetfillcolor{currentfill}%
\pgfsetlinewidth{1.003750pt}%
\definecolor{currentstroke}{rgb}{0.000000,0.000000,0.000000}%
\pgfsetstrokecolor{currentstroke}%
\pgfsetdash{}{0pt}%
\pgfpathmoveto{\pgfqpoint{4.567613in}{0.630000in}}%
\pgfpathlineto{\pgfqpoint{5.182930in}{0.630000in}}%
\pgfpathlineto{\pgfqpoint{5.182930in}{2.248744in}}%
\pgfpathlineto{\pgfqpoint{4.567613in}{2.248744in}}%
\pgfpathclose%
\pgfusepath{stroke,fill}%
\end{pgfscope}%
\begin{pgfscope}%
\pgfpathrectangle{\pgfqpoint{0.787813in}{0.630000in}}{\pgfqpoint{7.032188in}{4.970556in}} %
\pgfusepath{clip}%
\pgfsetbuttcap%
\pgfsetmiterjoin%
\definecolor{currentfill}{rgb}{1.000000,0.000000,0.000000}%
\pgfsetfillcolor{currentfill}%
\pgfsetlinewidth{1.003750pt}%
\definecolor{currentstroke}{rgb}{0.000000,0.000000,0.000000}%
\pgfsetstrokecolor{currentstroke}%
\pgfsetdash{}{0pt}%
\pgfpathmoveto{\pgfqpoint{6.325660in}{0.630000in}}%
\pgfpathlineto{\pgfqpoint{6.940977in}{0.630000in}}%
\pgfpathlineto{\pgfqpoint{6.940977in}{2.339871in}}%
\pgfpathlineto{\pgfqpoint{6.325660in}{2.339871in}}%
\pgfpathclose%
\pgfusepath{stroke,fill}%
\end{pgfscope}%
\begin{pgfscope}%
\pgfpathrectangle{\pgfqpoint{0.787813in}{0.630000in}}{\pgfqpoint{7.032188in}{4.970556in}} %
\pgfusepath{clip}%
\pgfsetbuttcap%
\pgfsetmiterjoin%
\definecolor{currentfill}{rgb}{0.000000,0.000000,1.000000}%
\pgfsetfillcolor{currentfill}%
\pgfsetlinewidth{1.003750pt}%
\definecolor{currentstroke}{rgb}{0.000000,0.000000,0.000000}%
\pgfsetstrokecolor{currentstroke}%
\pgfsetdash{}{0pt}%
\pgfpathmoveto{\pgfqpoint{1.666836in}{0.630000in}}%
\pgfpathlineto{\pgfqpoint{2.282152in}{0.630000in}}%
\pgfpathlineto{\pgfqpoint{2.282152in}{1.755002in}}%
\pgfpathlineto{\pgfqpoint{1.666836in}{1.755002in}}%
\pgfpathclose%
\pgfusepath{stroke,fill}%
\end{pgfscope}%
\begin{pgfscope}%
\pgfpathrectangle{\pgfqpoint{0.787813in}{0.630000in}}{\pgfqpoint{7.032188in}{4.970556in}} %
\pgfusepath{clip}%
\pgfsetbuttcap%
\pgfsetmiterjoin%
\definecolor{currentfill}{rgb}{0.000000,0.000000,1.000000}%
\pgfsetfillcolor{currentfill}%
\pgfsetlinewidth{1.003750pt}%
\definecolor{currentstroke}{rgb}{0.000000,0.000000,0.000000}%
\pgfsetstrokecolor{currentstroke}%
\pgfsetdash{}{0pt}%
\pgfpathmoveto{\pgfqpoint{3.424883in}{0.630000in}}%
\pgfpathlineto{\pgfqpoint{4.040199in}{0.630000in}}%
\pgfpathlineto{\pgfqpoint{4.040199in}{3.093739in}}%
\pgfpathlineto{\pgfqpoint{3.424883in}{3.093739in}}%
\pgfpathclose%
\pgfusepath{stroke,fill}%
\end{pgfscope}%
\begin{pgfscope}%
\pgfpathrectangle{\pgfqpoint{0.787813in}{0.630000in}}{\pgfqpoint{7.032188in}{4.970556in}} %
\pgfusepath{clip}%
\pgfsetbuttcap%
\pgfsetmiterjoin%
\definecolor{currentfill}{rgb}{0.000000,0.000000,1.000000}%
\pgfsetfillcolor{currentfill}%
\pgfsetlinewidth{1.003750pt}%
\definecolor{currentstroke}{rgb}{0.000000,0.000000,0.000000}%
\pgfsetstrokecolor{currentstroke}%
\pgfsetdash{}{0pt}%
\pgfpathmoveto{\pgfqpoint{5.182930in}{0.630000in}}%
\pgfpathlineto{\pgfqpoint{5.798246in}{0.630000in}}%
\pgfpathlineto{\pgfqpoint{5.798246in}{3.758136in}}%
\pgfpathlineto{\pgfqpoint{5.182930in}{3.758136in}}%
\pgfpathclose%
\pgfusepath{stroke,fill}%
\end{pgfscope}%
\begin{pgfscope}%
\pgfpathrectangle{\pgfqpoint{0.787813in}{0.630000in}}{\pgfqpoint{7.032188in}{4.970556in}} %
\pgfusepath{clip}%
\pgfsetbuttcap%
\pgfsetmiterjoin%
\definecolor{currentfill}{rgb}{0.000000,0.000000,1.000000}%
\pgfsetfillcolor{currentfill}%
\pgfsetlinewidth{1.003750pt}%
\definecolor{currentstroke}{rgb}{0.000000,0.000000,0.000000}%
\pgfsetstrokecolor{currentstroke}%
\pgfsetdash{}{0pt}%
\pgfpathmoveto{\pgfqpoint{6.940977in}{0.630000in}}%
\pgfpathlineto{\pgfqpoint{7.556293in}{0.630000in}}%
\pgfpathlineto{\pgfqpoint{7.556293in}{3.792930in}}%
\pgfpathlineto{\pgfqpoint{6.940977in}{3.792930in}}%
\pgfpathclose%
\pgfusepath{stroke,fill}%
\end{pgfscope}%
\begin{pgfscope}%
\pgfsetrectcap%
\pgfsetmiterjoin%
\pgfsetlinewidth{1.003750pt}%
\definecolor{currentstroke}{rgb}{0.000000,0.000000,0.000000}%
\pgfsetstrokecolor{currentstroke}%
\pgfsetdash{}{0pt}%
\pgfpathmoveto{\pgfqpoint{0.787813in}{5.600556in}}%
\pgfpathlineto{\pgfqpoint{7.820000in}{5.600556in}}%
\pgfusepath{stroke}%
\end{pgfscope}%
\begin{pgfscope}%
\pgfsetrectcap%
\pgfsetmiterjoin%
\pgfsetlinewidth{1.003750pt}%
\definecolor{currentstroke}{rgb}{0.000000,0.000000,0.000000}%
\pgfsetstrokecolor{currentstroke}%
\pgfsetdash{}{0pt}%
\pgfpathmoveto{\pgfqpoint{7.820000in}{0.630000in}}%
\pgfpathlineto{\pgfqpoint{7.820000in}{5.600556in}}%
\pgfusepath{stroke}%
\end{pgfscope}%
\begin{pgfscope}%
\pgfsetrectcap%
\pgfsetmiterjoin%
\pgfsetlinewidth{1.003750pt}%
\definecolor{currentstroke}{rgb}{0.000000,0.000000,0.000000}%
\pgfsetstrokecolor{currentstroke}%
\pgfsetdash{}{0pt}%
\pgfpathmoveto{\pgfqpoint{0.787813in}{0.630000in}}%
\pgfpathlineto{\pgfqpoint{7.820000in}{0.630000in}}%
\pgfusepath{stroke}%
\end{pgfscope}%
\begin{pgfscope}%
\pgfsetrectcap%
\pgfsetmiterjoin%
\pgfsetlinewidth{1.003750pt}%
\definecolor{currentstroke}{rgb}{0.000000,0.000000,0.000000}%
\pgfsetstrokecolor{currentstroke}%
\pgfsetdash{}{0pt}%
\pgfpathmoveto{\pgfqpoint{0.787813in}{0.630000in}}%
\pgfpathlineto{\pgfqpoint{0.787813in}{5.600556in}}%
\pgfusepath{stroke}%
\end{pgfscope}%
\begin{pgfscope}%
\pgfsetbuttcap%
\pgfsetroundjoin%
\definecolor{currentfill}{rgb}{0.000000,0.000000,0.000000}%
\pgfsetfillcolor{currentfill}%
\pgfsetlinewidth{0.501875pt}%
\definecolor{currentstroke}{rgb}{0.000000,0.000000,0.000000}%
\pgfsetstrokecolor{currentstroke}%
\pgfsetdash{}{0pt}%
\pgfsys@defobject{currentmarker}{\pgfqpoint{0.000000in}{0.000000in}}{\pgfqpoint{0.000000in}{0.055556in}}{%
\pgfpathmoveto{\pgfqpoint{0.000000in}{0.000000in}}%
\pgfpathlineto{\pgfqpoint{0.000000in}{0.055556in}}%
\pgfusepath{stroke,fill}%
}%
\begin{pgfscope}%
\pgfsys@transformshift{1.666836in}{0.630000in}%
\pgfsys@useobject{currentmarker}{}%
\end{pgfscope}%
\end{pgfscope}%
\begin{pgfscope}%
\pgfsetbuttcap%
\pgfsetroundjoin%
\definecolor{currentfill}{rgb}{0.000000,0.000000,0.000000}%
\pgfsetfillcolor{currentfill}%
\pgfsetlinewidth{0.501875pt}%
\definecolor{currentstroke}{rgb}{0.000000,0.000000,0.000000}%
\pgfsetstrokecolor{currentstroke}%
\pgfsetdash{}{0pt}%
\pgfsys@defobject{currentmarker}{\pgfqpoint{0.000000in}{-0.055556in}}{\pgfqpoint{0.000000in}{0.000000in}}{%
\pgfpathmoveto{\pgfqpoint{0.000000in}{0.000000in}}%
\pgfpathlineto{\pgfqpoint{0.000000in}{-0.055556in}}%
\pgfusepath{stroke,fill}%
}%
\begin{pgfscope}%
\pgfsys@transformshift{1.666836in}{5.600556in}%
\pgfsys@useobject{currentmarker}{}%
\end{pgfscope}%
\end{pgfscope}%
\begin{pgfscope}%
\pgftext[x=1.666836in,y=0.574444in,,top]{\sffamily\fontsize{12.000000}{14.400000}\selectfont 5k}%
\end{pgfscope}%
\begin{pgfscope}%
\pgfsetbuttcap%
\pgfsetroundjoin%
\definecolor{currentfill}{rgb}{0.000000,0.000000,0.000000}%
\pgfsetfillcolor{currentfill}%
\pgfsetlinewidth{0.501875pt}%
\definecolor{currentstroke}{rgb}{0.000000,0.000000,0.000000}%
\pgfsetstrokecolor{currentstroke}%
\pgfsetdash{}{0pt}%
\pgfsys@defobject{currentmarker}{\pgfqpoint{0.000000in}{0.000000in}}{\pgfqpoint{0.000000in}{0.055556in}}{%
\pgfpathmoveto{\pgfqpoint{0.000000in}{0.000000in}}%
\pgfpathlineto{\pgfqpoint{0.000000in}{0.055556in}}%
\pgfusepath{stroke,fill}%
}%
\begin{pgfscope}%
\pgfsys@transformshift{3.424883in}{0.630000in}%
\pgfsys@useobject{currentmarker}{}%
\end{pgfscope}%
\end{pgfscope}%
\begin{pgfscope}%
\pgfsetbuttcap%
\pgfsetroundjoin%
\definecolor{currentfill}{rgb}{0.000000,0.000000,0.000000}%
\pgfsetfillcolor{currentfill}%
\pgfsetlinewidth{0.501875pt}%
\definecolor{currentstroke}{rgb}{0.000000,0.000000,0.000000}%
\pgfsetstrokecolor{currentstroke}%
\pgfsetdash{}{0pt}%
\pgfsys@defobject{currentmarker}{\pgfqpoint{0.000000in}{-0.055556in}}{\pgfqpoint{0.000000in}{0.000000in}}{%
\pgfpathmoveto{\pgfqpoint{0.000000in}{0.000000in}}%
\pgfpathlineto{\pgfqpoint{0.000000in}{-0.055556in}}%
\pgfusepath{stroke,fill}%
}%
\begin{pgfscope}%
\pgfsys@transformshift{3.424883in}{5.600556in}%
\pgfsys@useobject{currentmarker}{}%
\end{pgfscope}%
\end{pgfscope}%
\begin{pgfscope}%
\pgftext[x=3.424883in,y=0.574444in,,top]{\sffamily\fontsize{12.000000}{14.400000}\selectfont 10k}%
\end{pgfscope}%
\begin{pgfscope}%
\pgfsetbuttcap%
\pgfsetroundjoin%
\definecolor{currentfill}{rgb}{0.000000,0.000000,0.000000}%
\pgfsetfillcolor{currentfill}%
\pgfsetlinewidth{0.501875pt}%
\definecolor{currentstroke}{rgb}{0.000000,0.000000,0.000000}%
\pgfsetstrokecolor{currentstroke}%
\pgfsetdash{}{0pt}%
\pgfsys@defobject{currentmarker}{\pgfqpoint{0.000000in}{0.000000in}}{\pgfqpoint{0.000000in}{0.055556in}}{%
\pgfpathmoveto{\pgfqpoint{0.000000in}{0.000000in}}%
\pgfpathlineto{\pgfqpoint{0.000000in}{0.055556in}}%
\pgfusepath{stroke,fill}%
}%
\begin{pgfscope}%
\pgfsys@transformshift{5.182930in}{0.630000in}%
\pgfsys@useobject{currentmarker}{}%
\end{pgfscope}%
\end{pgfscope}%
\begin{pgfscope}%
\pgfsetbuttcap%
\pgfsetroundjoin%
\definecolor{currentfill}{rgb}{0.000000,0.000000,0.000000}%
\pgfsetfillcolor{currentfill}%
\pgfsetlinewidth{0.501875pt}%
\definecolor{currentstroke}{rgb}{0.000000,0.000000,0.000000}%
\pgfsetstrokecolor{currentstroke}%
\pgfsetdash{}{0pt}%
\pgfsys@defobject{currentmarker}{\pgfqpoint{0.000000in}{-0.055556in}}{\pgfqpoint{0.000000in}{0.000000in}}{%
\pgfpathmoveto{\pgfqpoint{0.000000in}{0.000000in}}%
\pgfpathlineto{\pgfqpoint{0.000000in}{-0.055556in}}%
\pgfusepath{stroke,fill}%
}%
\begin{pgfscope}%
\pgfsys@transformshift{5.182930in}{5.600556in}%
\pgfsys@useobject{currentmarker}{}%
\end{pgfscope}%
\end{pgfscope}%
\begin{pgfscope}%
\pgftext[x=5.182930in,y=0.574444in,,top]{\sffamily\fontsize{12.000000}{14.400000}\selectfont 15k}%
\end{pgfscope}%
\begin{pgfscope}%
\pgfsetbuttcap%
\pgfsetroundjoin%
\definecolor{currentfill}{rgb}{0.000000,0.000000,0.000000}%
\pgfsetfillcolor{currentfill}%
\pgfsetlinewidth{0.501875pt}%
\definecolor{currentstroke}{rgb}{0.000000,0.000000,0.000000}%
\pgfsetstrokecolor{currentstroke}%
\pgfsetdash{}{0pt}%
\pgfsys@defobject{currentmarker}{\pgfqpoint{0.000000in}{0.000000in}}{\pgfqpoint{0.000000in}{0.055556in}}{%
\pgfpathmoveto{\pgfqpoint{0.000000in}{0.000000in}}%
\pgfpathlineto{\pgfqpoint{0.000000in}{0.055556in}}%
\pgfusepath{stroke,fill}%
}%
\begin{pgfscope}%
\pgfsys@transformshift{6.940977in}{0.630000in}%
\pgfsys@useobject{currentmarker}{}%
\end{pgfscope}%
\end{pgfscope}%
\begin{pgfscope}%
\pgfsetbuttcap%
\pgfsetroundjoin%
\definecolor{currentfill}{rgb}{0.000000,0.000000,0.000000}%
\pgfsetfillcolor{currentfill}%
\pgfsetlinewidth{0.501875pt}%
\definecolor{currentstroke}{rgb}{0.000000,0.000000,0.000000}%
\pgfsetstrokecolor{currentstroke}%
\pgfsetdash{}{0pt}%
\pgfsys@defobject{currentmarker}{\pgfqpoint{0.000000in}{-0.055556in}}{\pgfqpoint{0.000000in}{0.000000in}}{%
\pgfpathmoveto{\pgfqpoint{0.000000in}{0.000000in}}%
\pgfpathlineto{\pgfqpoint{0.000000in}{-0.055556in}}%
\pgfusepath{stroke,fill}%
}%
\begin{pgfscope}%
\pgfsys@transformshift{6.940977in}{5.600556in}%
\pgfsys@useobject{currentmarker}{}%
\end{pgfscope}%
\end{pgfscope}%
\begin{pgfscope}%
\pgftext[x=6.940977in,y=0.574444in,,top]{\sffamily\fontsize{12.000000}{14.400000}\selectfont 20k}%
\end{pgfscope}%
\begin{pgfscope}%
\pgftext[x=4.303906in,y=0.343705in,,top]{\sffamily\fontsize{12.000000}{14.400000}\selectfont Number of Requests}%
\end{pgfscope}%
\begin{pgfscope}%
\pgfsetbuttcap%
\pgfsetroundjoin%
\definecolor{currentfill}{rgb}{0.000000,0.000000,0.000000}%
\pgfsetfillcolor{currentfill}%
\pgfsetlinewidth{0.501875pt}%
\definecolor{currentstroke}{rgb}{0.000000,0.000000,0.000000}%
\pgfsetstrokecolor{currentstroke}%
\pgfsetdash{}{0pt}%
\pgfsys@defobject{currentmarker}{\pgfqpoint{0.000000in}{0.000000in}}{\pgfqpoint{0.055556in}{0.000000in}}{%
\pgfpathmoveto{\pgfqpoint{0.000000in}{0.000000in}}%
\pgfpathlineto{\pgfqpoint{0.055556in}{0.000000in}}%
\pgfusepath{stroke,fill}%
}%
\begin{pgfscope}%
\pgfsys@transformshift{0.787813in}{0.630000in}%
\pgfsys@useobject{currentmarker}{}%
\end{pgfscope}%
\end{pgfscope}%
\begin{pgfscope}%
\pgfsetbuttcap%
\pgfsetroundjoin%
\definecolor{currentfill}{rgb}{0.000000,0.000000,0.000000}%
\pgfsetfillcolor{currentfill}%
\pgfsetlinewidth{0.501875pt}%
\definecolor{currentstroke}{rgb}{0.000000,0.000000,0.000000}%
\pgfsetstrokecolor{currentstroke}%
\pgfsetdash{}{0pt}%
\pgfsys@defobject{currentmarker}{\pgfqpoint{-0.055556in}{0.000000in}}{\pgfqpoint{0.000000in}{0.000000in}}{%
\pgfpathmoveto{\pgfqpoint{0.000000in}{0.000000in}}%
\pgfpathlineto{\pgfqpoint{-0.055556in}{0.000000in}}%
\pgfusepath{stroke,fill}%
}%
\begin{pgfscope}%
\pgfsys@transformshift{7.820000in}{0.630000in}%
\pgfsys@useobject{currentmarker}{}%
\end{pgfscope}%
\end{pgfscope}%
\begin{pgfscope}%
\pgftext[x=0.732257in,y=0.630000in,right,]{\sffamily\fontsize{12.000000}{14.400000}\selectfont 0}%
\end{pgfscope}%
\begin{pgfscope}%
\pgfsetbuttcap%
\pgfsetroundjoin%
\definecolor{currentfill}{rgb}{0.000000,0.000000,0.000000}%
\pgfsetfillcolor{currentfill}%
\pgfsetlinewidth{0.501875pt}%
\definecolor{currentstroke}{rgb}{0.000000,0.000000,0.000000}%
\pgfsetstrokecolor{currentstroke}%
\pgfsetdash{}{0pt}%
\pgfsys@defobject{currentmarker}{\pgfqpoint{0.000000in}{0.000000in}}{\pgfqpoint{0.055556in}{0.000000in}}{%
\pgfpathmoveto{\pgfqpoint{0.000000in}{0.000000in}}%
\pgfpathlineto{\pgfqpoint{0.055556in}{0.000000in}}%
\pgfusepath{stroke,fill}%
}%
\begin{pgfscope}%
\pgfsys@transformshift{0.787813in}{1.458426in}%
\pgfsys@useobject{currentmarker}{}%
\end{pgfscope}%
\end{pgfscope}%
\begin{pgfscope}%
\pgfsetbuttcap%
\pgfsetroundjoin%
\definecolor{currentfill}{rgb}{0.000000,0.000000,0.000000}%
\pgfsetfillcolor{currentfill}%
\pgfsetlinewidth{0.501875pt}%
\definecolor{currentstroke}{rgb}{0.000000,0.000000,0.000000}%
\pgfsetstrokecolor{currentstroke}%
\pgfsetdash{}{0pt}%
\pgfsys@defobject{currentmarker}{\pgfqpoint{-0.055556in}{0.000000in}}{\pgfqpoint{0.000000in}{0.000000in}}{%
\pgfpathmoveto{\pgfqpoint{0.000000in}{0.000000in}}%
\pgfpathlineto{\pgfqpoint{-0.055556in}{0.000000in}}%
\pgfusepath{stroke,fill}%
}%
\begin{pgfscope}%
\pgfsys@transformshift{7.820000in}{1.458426in}%
\pgfsys@useobject{currentmarker}{}%
\end{pgfscope}%
\end{pgfscope}%
\begin{pgfscope}%
\pgftext[x=0.732257in,y=1.458426in,right,]{\sffamily\fontsize{12.000000}{14.400000}\selectfont 50}%
\end{pgfscope}%
\begin{pgfscope}%
\pgfsetbuttcap%
\pgfsetroundjoin%
\definecolor{currentfill}{rgb}{0.000000,0.000000,0.000000}%
\pgfsetfillcolor{currentfill}%
\pgfsetlinewidth{0.501875pt}%
\definecolor{currentstroke}{rgb}{0.000000,0.000000,0.000000}%
\pgfsetstrokecolor{currentstroke}%
\pgfsetdash{}{0pt}%
\pgfsys@defobject{currentmarker}{\pgfqpoint{0.000000in}{0.000000in}}{\pgfqpoint{0.055556in}{0.000000in}}{%
\pgfpathmoveto{\pgfqpoint{0.000000in}{0.000000in}}%
\pgfpathlineto{\pgfqpoint{0.055556in}{0.000000in}}%
\pgfusepath{stroke,fill}%
}%
\begin{pgfscope}%
\pgfsys@transformshift{0.787813in}{2.286852in}%
\pgfsys@useobject{currentmarker}{}%
\end{pgfscope}%
\end{pgfscope}%
\begin{pgfscope}%
\pgfsetbuttcap%
\pgfsetroundjoin%
\definecolor{currentfill}{rgb}{0.000000,0.000000,0.000000}%
\pgfsetfillcolor{currentfill}%
\pgfsetlinewidth{0.501875pt}%
\definecolor{currentstroke}{rgb}{0.000000,0.000000,0.000000}%
\pgfsetstrokecolor{currentstroke}%
\pgfsetdash{}{0pt}%
\pgfsys@defobject{currentmarker}{\pgfqpoint{-0.055556in}{0.000000in}}{\pgfqpoint{0.000000in}{0.000000in}}{%
\pgfpathmoveto{\pgfqpoint{0.000000in}{0.000000in}}%
\pgfpathlineto{\pgfqpoint{-0.055556in}{0.000000in}}%
\pgfusepath{stroke,fill}%
}%
\begin{pgfscope}%
\pgfsys@transformshift{7.820000in}{2.286852in}%
\pgfsys@useobject{currentmarker}{}%
\end{pgfscope}%
\end{pgfscope}%
\begin{pgfscope}%
\pgftext[x=0.732257in,y=2.286852in,right,]{\sffamily\fontsize{12.000000}{14.400000}\selectfont 100}%
\end{pgfscope}%
\begin{pgfscope}%
\pgfsetbuttcap%
\pgfsetroundjoin%
\definecolor{currentfill}{rgb}{0.000000,0.000000,0.000000}%
\pgfsetfillcolor{currentfill}%
\pgfsetlinewidth{0.501875pt}%
\definecolor{currentstroke}{rgb}{0.000000,0.000000,0.000000}%
\pgfsetstrokecolor{currentstroke}%
\pgfsetdash{}{0pt}%
\pgfsys@defobject{currentmarker}{\pgfqpoint{0.000000in}{0.000000in}}{\pgfqpoint{0.055556in}{0.000000in}}{%
\pgfpathmoveto{\pgfqpoint{0.000000in}{0.000000in}}%
\pgfpathlineto{\pgfqpoint{0.055556in}{0.000000in}}%
\pgfusepath{stroke,fill}%
}%
\begin{pgfscope}%
\pgfsys@transformshift{0.787813in}{3.115278in}%
\pgfsys@useobject{currentmarker}{}%
\end{pgfscope}%
\end{pgfscope}%
\begin{pgfscope}%
\pgfsetbuttcap%
\pgfsetroundjoin%
\definecolor{currentfill}{rgb}{0.000000,0.000000,0.000000}%
\pgfsetfillcolor{currentfill}%
\pgfsetlinewidth{0.501875pt}%
\definecolor{currentstroke}{rgb}{0.000000,0.000000,0.000000}%
\pgfsetstrokecolor{currentstroke}%
\pgfsetdash{}{0pt}%
\pgfsys@defobject{currentmarker}{\pgfqpoint{-0.055556in}{0.000000in}}{\pgfqpoint{0.000000in}{0.000000in}}{%
\pgfpathmoveto{\pgfqpoint{0.000000in}{0.000000in}}%
\pgfpathlineto{\pgfqpoint{-0.055556in}{0.000000in}}%
\pgfusepath{stroke,fill}%
}%
\begin{pgfscope}%
\pgfsys@transformshift{7.820000in}{3.115278in}%
\pgfsys@useobject{currentmarker}{}%
\end{pgfscope}%
\end{pgfscope}%
\begin{pgfscope}%
\pgftext[x=0.732257in,y=3.115278in,right,]{\sffamily\fontsize{12.000000}{14.400000}\selectfont 150}%
\end{pgfscope}%
\begin{pgfscope}%
\pgfsetbuttcap%
\pgfsetroundjoin%
\definecolor{currentfill}{rgb}{0.000000,0.000000,0.000000}%
\pgfsetfillcolor{currentfill}%
\pgfsetlinewidth{0.501875pt}%
\definecolor{currentstroke}{rgb}{0.000000,0.000000,0.000000}%
\pgfsetstrokecolor{currentstroke}%
\pgfsetdash{}{0pt}%
\pgfsys@defobject{currentmarker}{\pgfqpoint{0.000000in}{0.000000in}}{\pgfqpoint{0.055556in}{0.000000in}}{%
\pgfpathmoveto{\pgfqpoint{0.000000in}{0.000000in}}%
\pgfpathlineto{\pgfqpoint{0.055556in}{0.000000in}}%
\pgfusepath{stroke,fill}%
}%
\begin{pgfscope}%
\pgfsys@transformshift{0.787813in}{3.943704in}%
\pgfsys@useobject{currentmarker}{}%
\end{pgfscope}%
\end{pgfscope}%
\begin{pgfscope}%
\pgfsetbuttcap%
\pgfsetroundjoin%
\definecolor{currentfill}{rgb}{0.000000,0.000000,0.000000}%
\pgfsetfillcolor{currentfill}%
\pgfsetlinewidth{0.501875pt}%
\definecolor{currentstroke}{rgb}{0.000000,0.000000,0.000000}%
\pgfsetstrokecolor{currentstroke}%
\pgfsetdash{}{0pt}%
\pgfsys@defobject{currentmarker}{\pgfqpoint{-0.055556in}{0.000000in}}{\pgfqpoint{0.000000in}{0.000000in}}{%
\pgfpathmoveto{\pgfqpoint{0.000000in}{0.000000in}}%
\pgfpathlineto{\pgfqpoint{-0.055556in}{0.000000in}}%
\pgfusepath{stroke,fill}%
}%
\begin{pgfscope}%
\pgfsys@transformshift{7.820000in}{3.943704in}%
\pgfsys@useobject{currentmarker}{}%
\end{pgfscope}%
\end{pgfscope}%
\begin{pgfscope}%
\pgftext[x=0.732257in,y=3.943704in,right,]{\sffamily\fontsize{12.000000}{14.400000}\selectfont 200}%
\end{pgfscope}%
\begin{pgfscope}%
\pgfsetbuttcap%
\pgfsetroundjoin%
\definecolor{currentfill}{rgb}{0.000000,0.000000,0.000000}%
\pgfsetfillcolor{currentfill}%
\pgfsetlinewidth{0.501875pt}%
\definecolor{currentstroke}{rgb}{0.000000,0.000000,0.000000}%
\pgfsetstrokecolor{currentstroke}%
\pgfsetdash{}{0pt}%
\pgfsys@defobject{currentmarker}{\pgfqpoint{0.000000in}{0.000000in}}{\pgfqpoint{0.055556in}{0.000000in}}{%
\pgfpathmoveto{\pgfqpoint{0.000000in}{0.000000in}}%
\pgfpathlineto{\pgfqpoint{0.055556in}{0.000000in}}%
\pgfusepath{stroke,fill}%
}%
\begin{pgfscope}%
\pgfsys@transformshift{0.787813in}{4.772130in}%
\pgfsys@useobject{currentmarker}{}%
\end{pgfscope}%
\end{pgfscope}%
\begin{pgfscope}%
\pgfsetbuttcap%
\pgfsetroundjoin%
\definecolor{currentfill}{rgb}{0.000000,0.000000,0.000000}%
\pgfsetfillcolor{currentfill}%
\pgfsetlinewidth{0.501875pt}%
\definecolor{currentstroke}{rgb}{0.000000,0.000000,0.000000}%
\pgfsetstrokecolor{currentstroke}%
\pgfsetdash{}{0pt}%
\pgfsys@defobject{currentmarker}{\pgfqpoint{-0.055556in}{0.000000in}}{\pgfqpoint{0.000000in}{0.000000in}}{%
\pgfpathmoveto{\pgfqpoint{0.000000in}{0.000000in}}%
\pgfpathlineto{\pgfqpoint{-0.055556in}{0.000000in}}%
\pgfusepath{stroke,fill}%
}%
\begin{pgfscope}%
\pgfsys@transformshift{7.820000in}{4.772130in}%
\pgfsys@useobject{currentmarker}{}%
\end{pgfscope}%
\end{pgfscope}%
\begin{pgfscope}%
\pgftext[x=0.732257in,y=4.772130in,right,]{\sffamily\fontsize{12.000000}{14.400000}\selectfont 250}%
\end{pgfscope}%
\begin{pgfscope}%
\pgfsetbuttcap%
\pgfsetroundjoin%
\definecolor{currentfill}{rgb}{0.000000,0.000000,0.000000}%
\pgfsetfillcolor{currentfill}%
\pgfsetlinewidth{0.501875pt}%
\definecolor{currentstroke}{rgb}{0.000000,0.000000,0.000000}%
\pgfsetstrokecolor{currentstroke}%
\pgfsetdash{}{0pt}%
\pgfsys@defobject{currentmarker}{\pgfqpoint{0.000000in}{0.000000in}}{\pgfqpoint{0.055556in}{0.000000in}}{%
\pgfpathmoveto{\pgfqpoint{0.000000in}{0.000000in}}%
\pgfpathlineto{\pgfqpoint{0.055556in}{0.000000in}}%
\pgfusepath{stroke,fill}%
}%
\begin{pgfscope}%
\pgfsys@transformshift{0.787813in}{5.600556in}%
\pgfsys@useobject{currentmarker}{}%
\end{pgfscope}%
\end{pgfscope}%
\begin{pgfscope}%
\pgfsetbuttcap%
\pgfsetroundjoin%
\definecolor{currentfill}{rgb}{0.000000,0.000000,0.000000}%
\pgfsetfillcolor{currentfill}%
\pgfsetlinewidth{0.501875pt}%
\definecolor{currentstroke}{rgb}{0.000000,0.000000,0.000000}%
\pgfsetstrokecolor{currentstroke}%
\pgfsetdash{}{0pt}%
\pgfsys@defobject{currentmarker}{\pgfqpoint{-0.055556in}{0.000000in}}{\pgfqpoint{0.000000in}{0.000000in}}{%
\pgfpathmoveto{\pgfqpoint{0.000000in}{0.000000in}}%
\pgfpathlineto{\pgfqpoint{-0.055556in}{0.000000in}}%
\pgfusepath{stroke,fill}%
}%
\begin{pgfscope}%
\pgfsys@transformshift{7.820000in}{5.600556in}%
\pgfsys@useobject{currentmarker}{}%
\end{pgfscope}%
\end{pgfscope}%
\begin{pgfscope}%
\pgftext[x=0.732257in,y=5.600556in,right,]{\sffamily\fontsize{12.000000}{14.400000}\selectfont 300}%
\end{pgfscope}%
\begin{pgfscope}%
\pgftext[x=0.344697in,y=3.115278in,,bottom,rotate=90.000000]{\sffamily\fontsize{12.000000}{14.400000}\selectfont Requests/sec (more is better)}%
\end{pgfscope}%
\begin{pgfscope}%
\pgftext[x=4.303906in,y=5.670000in,,base]{\sffamily\fontsize{14.400000}{17.280000}\selectfont Latency by number of requests}%
\end{pgfscope}%
\begin{pgfscope}%
\pgfsetbuttcap%
\pgfsetmiterjoin%
\definecolor{currentfill}{rgb}{1.000000,1.000000,1.000000}%
\pgfsetfillcolor{currentfill}%
\pgfsetlinewidth{1.003750pt}%
\definecolor{currentstroke}{rgb}{0.000000,0.000000,0.000000}%
\pgfsetstrokecolor{currentstroke}%
\pgfsetdash{}{0pt}%
\pgfpathmoveto{\pgfqpoint{5.728418in}{4.853446in}}%
\pgfpathlineto{\pgfqpoint{7.720000in}{4.853446in}}%
\pgfpathlineto{\pgfqpoint{7.720000in}{5.500556in}}%
\pgfpathlineto{\pgfqpoint{5.728418in}{5.500556in}}%
\pgfpathclose%
\pgfusepath{stroke,fill}%
\end{pgfscope}%
\begin{pgfscope}%
\pgfsetbuttcap%
\pgfsetmiterjoin%
\definecolor{currentfill}{rgb}{1.000000,0.000000,0.000000}%
\pgfsetfillcolor{currentfill}%
\pgfsetlinewidth{1.003750pt}%
\definecolor{currentstroke}{rgb}{0.000000,0.000000,0.000000}%
\pgfsetstrokecolor{currentstroke}%
\pgfsetdash{}{0pt}%
\pgfpathmoveto{\pgfqpoint{5.808418in}{5.268602in}}%
\pgfpathlineto{\pgfqpoint{6.208418in}{5.268602in}}%
\pgfpathlineto{\pgfqpoint{6.208418in}{5.408602in}}%
\pgfpathlineto{\pgfqpoint{5.808418in}{5.408602in}}%
\pgfpathclose%
\pgfusepath{stroke,fill}%
\end{pgfscope}%
\begin{pgfscope}%
\pgftext[x=6.368418in,y=5.268602in,left,base]{\sffamily\fontsize{14.400000}{17.280000}\selectfont Standard GC}%
\end{pgfscope}%
\begin{pgfscope}%
\pgfsetbuttcap%
\pgfsetmiterjoin%
\definecolor{currentfill}{rgb}{0.000000,0.000000,1.000000}%
\pgfsetfillcolor{currentfill}%
\pgfsetlinewidth{1.003750pt}%
\definecolor{currentstroke}{rgb}{0.000000,0.000000,0.000000}%
\pgfsetstrokecolor{currentstroke}%
\pgfsetdash{}{0pt}%
\pgfpathmoveto{\pgfqpoint{5.808418in}{4.975048in}}%
\pgfpathlineto{\pgfqpoint{6.208418in}{4.975048in}}%
\pgfpathlineto{\pgfqpoint{6.208418in}{5.115048in}}%
\pgfpathlineto{\pgfqpoint{5.808418in}{5.115048in}}%
\pgfpathclose%
\pgfusepath{stroke,fill}%
\end{pgfscope}%
\begin{pgfscope}%
\pgftext[x=6.368418in,y=4.975048in,left,base]{\sffamily\fontsize{14.400000}{17.280000}\selectfont Patched GC}%
\end{pgfscope}%
\end{pgfpicture}%
\makeatother%
\endgroup%
}
    \caption{Request Rate for different Request Quantities.  Note Standard GC started paging after 10k requests}\label{fig:latency}
\end{figure}

\section{Future Work}\label{sec:futurework}

While the optimization explored in this paper account for noticeably memory savings, there is much more that can be done in the future.

In a CGGC like PyPy, when objects are copied from the nursery into the mature generation, all the mature objects which pointed to the new object in the nursery must be updated to refer to that objects new post-copy location.  This results in writes to mature objects that can be avoided.  If instead of updating the pointers to young objects in place in the old objects, there were a reference table so that all references from old to young objects were indirect, then when young objects were copied only the reference table would need to be updated.  This lets the mature objects remain untouched even when the object which they are referencing moves.  

Additionally, the benchmark used in this paper was concocted specifically for testing purposes.  The results from using this benchmark may or may not accurately reflect the reality of real-world Django deployments.  Future work would likely take the findings in this paper and apply them to actual existing production websites.  Finding an open source website that runs on Django that is relatively easy to deploy would be paramount in bringing this research to real-world relevance.  

Lastly it would be interesting to see how the preservation of shared memory affects the cache miss rate of the worker processes.  In theory, if the worker processes are sharing physical memory then a single line-item in the cache can be shared amongst worker processes.  Whether this happens in would be worth looking into.  

\section{Related work}\label{sec:relatedwork}

This paper is very much inspired by recent updates in CPython's and Ruby MRI's Garbage Collectors which were updated in 2017 and 2014 respectively.  

\subsection{Copy-On-Write optimization in Ruby MRI's MSGC}

Ruby 2.0 contains a similar update to the GC that improved CoW performance.  Ruby's GC is a classic CGGC with a young generation and an old generation that is occasionally Mark-and-Swept.  In the old algorithm, every object contained a ``visited'' bit that would note whether the object had visited by the tracer.  In the new GC algorithm, these ``visited'' bits are stored outside the objects being traced.  In this way, the objects themselves don't need to be updated during a GC event \cite{incremental_ruby}.  

The optimization described in this paper is very similar to this, however we are specifically testing this optimization in the context of Python for a webserver workload.  

\subsection{Copy-On-Write Friendly Garbage Collection in CPython at Instagram}

A 2017 article from Instagram's Engineering series describes changes to CPython's Garbage Collector made at Instagram aimed at improving memory performance in their web-tier servers.  The experimentation section of this paper is heavily inspired by the test setup used at Instagram.

In their article, the Instagram engineering team explored multiple ways of reducing memory usage.  The first modification they tried was taking the reference counts stored in every Python object and moving them to a separate page.  The intent was that when reference counts get updated, only the reference count page needed to be changed and the page storing the objects could remain shared.  In practice this modification did not reduce memory usage \cite{dismissing_garbage}.  

The modification Instagram ended up implementing was to have the programmer manually mark certain objects as ``GC-Frozen''.  Objects that are GC-Frozen are not touched by the Garbage Collector and so the Garbage Collector state embedded within them is not touched.  Because Instagram's specific use case involves a lot of static objects that are shared between processes, this worked effectively \cite{fixing_garbage}.  

\section{Conclusion}

The results in this paper clearly show that by moving the Garbage Collector state out of the objects and into isolated Garbage Collector memory, we can prevent unnecessary memory writes and we can preserve shared memory between forked processes.  While the changes described in this paper have some runtime overhead in terms of mapping objects in memory to their associated Garbage Collector state, we found that for our deployment the savings in memory far outweighed the runtime overhead. 

\newpage
\onecolumn
\nocite{*}
\bibliographystyle{IEEEannot}
\bibliography{bibliography.bib}

\end{sloppypar}
\end{document}
